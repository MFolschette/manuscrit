\documentclass[12pt,french,francais,nofancyChapter,nofancyPart]{these-LUNAM}
%, fancyChapter, fancyPart



% Encodage d'entrée
\usepackage[utf8]{inputenc}
\usepackage[T1]{fontenc}
\usepackage{textcomp}  % Meilleur rendu de certains caractères
\usepackage{amsfonts}
\usepackage{amssymb}
\usepackage{stmaryrd}
\usepackage{textgreek}



% Mise en page
\geometry{inner=2.5cm, outer=2.5cm, top=2.5cm, bottom=2cm} % Format A4
%\geometry{inner=1.5cm, outer=1.5cm, top=2cm, bottom=2cm}

% En-têtes
% \renewcommand{\chaptermark}[1]{\markboth{Chapitre\ \thechapter.\ #1}{}}
% \renewcommand{\sectionmark}[1]{\markright{\thesection.\ #1}{}}
%\pagestyle{myheadings}

%\renewcommand{\chaptermark}[1]{\markboth{Chapitre\ \thechapter.\ #1}{}}
%\renewcommand{\sectionmark}[1]{\markright{\thesection.\ #1}{}}

\usepackage{fancyhdr}
\pagestyle{fancy}
\renewcommand{\chaptermark}[1]{\markboth{Chapitre\ \thechapter\ ---\ #1}{}}
\renewcommand{\sectionmark}[1]{\markright{\thesection\ ---\ #1}{}}
\renewcommand{\headrulewidth}{0.4pt}
\fancyhead{}
\fancyfoot{}
\fancyhead[LE,RO]{\thepage}
\fancyhead[LO]{\leftmark}
\fancyhead[RE]{\rightmark}

% Style des listes
\let\oldenumerate\enumerate
\renewcommand{\enumerate}{
  \oldenumerate
  \setlength{\itemsep}{1pt}
  \setlength{\parskip}{0pt}
  \setlength{\parsep}{0pt}
}

\let\olditemize\itemize
\renewcommand{\itemize}{
  \olditemize
  \setlength{\itemsep}{1pt}
  \setlength{\parskip}{0pt}
  \setlength{\parsep}{0pt}
}

\renewcommand{\labelitemi}{--}
\renewcommand{\labelitemii}{-}

% Sauts de paragraphes
%\setlength{\parskip}{4mm plus3mm minus3mm}
%\usepackage{parskip}
%\setlength{\parindent}{15pt}

% Chapeau
\definecolor{couleurchapeau}{rgb}{0.9,0.9,0.9}
\definecolor{couleurframe}{rgb}{0.7,0.7,0.7}
\newcommand{\chapeau}[1]{\medskip\noindent\begin{center}\colorbox{couleurchapeau}{%
  \begin{minipage}{.95\textwidth}%
  \vspace*{2pt}\hrule\vspace*{1em}%
  \hfill\begin{minipage}{.95\textwidth}#1\end{minipage}\hfill%
  \vspace*{1em}\hrule\vspace*{2pt}%
  \end{minipage}}\end{center}\vspace*{3em}}
\newcommand{\chapeaupublis}[2]{\medskip\noindent\begin{center}\colorbox{couleurchapeau}{%
  \begin{minipage}{.95\textwidth}%
  \vspace*{2pt}\hrule\vspace*{1em}%
  \hfill\begin{minipage}{.95\textwidth}#1\end{minipage}\hfill%
  \vspace*{.7em}\textcolor{gray}{\hrule}\vspace*{.7em}%
  \hfill\begin{minipage}{.95\textwidth}#2\end{minipage}\hfill%
  \vspace*{1em}\hrule\vspace*{2pt}%
  \end{minipage}}\end{center}\vspace*{3em}}

% Théorèmes et définitions grisées
\newsavebox{\mybox}
% \newenvironment{graybox}%
%   {\begin{lrbox}{\mybox}\begin{minipage}{0.98\linewidth}}%
%  {\end{minipage}\end{lrbox}\noindent\framebox{\colorbox{couleurchapeau}{\usebox{\mybox}}}}
\newenvironment{grayframedbox}%
  {\begin{lrbox}{\mybox}\begin{minipage}{0.985\linewidth}}%
  {\end{minipage}\end{lrbox}%
  \noindent\textcolor{couleurframe}{\framebox{\usebox{\mybox}}}}
\newenvironment{framedgraybox}%
  {\begin{lrbox}{\mybox}\begin{minipage}{0.97\linewidth}}%
  {\end{minipage}\end{lrbox}%
  \noindent\textcolor{couleurframe}{\framebox{\colorbox{couleurchapeau}{\usebox{\mybox}}}}}
\newenvironment{graybox}%
  {\begin{lrbox}{\mybox}\begin{minipage}{0.985\linewidth}}%
  {\end{minipage}\end{lrbox}%
  \noindent\colorbox{couleurchapeau}{\usebox{\mybox}}}
%  \noindent\textcolor{white}{\framebox{\colorbox{couleurchapeau}{\usebox{\mybox}}}}}
\makeatletter
\newcommand{\newgraytheorem}[3]{%
  \newenvironment{#1}{\vskip 5\p@\begin{#3}\begin{#2}}{%
  \end{#2}\end{#3}\vskip 5\p@}}
\makeatother

% Police
%\usepackage{times} % times, palatino, charter, lmodern, etc.
\usepackage{cmbright}  % Police sans empattements
%\onehalfspacing    % augmenter l'interligne

% Références internes et externes
\usepackage{varioref}   % Références enrichies
\usepackage[french]{refstyle}
\newcommand{\vdefref}[2][]{\defref[vref,#1]{#2}}
\newcommand{\vthmref}[2][]{\thmref[vref,#1]{#2}}
\newcommand{\vlemref}[2][]{\lemref[vref,#1]{#2}}
\newcommand{\vsecref}[2][]{\secref[vref,#1]{#2}}
\newcommand{\vfigref}[2][]{\figref[vref,#1]{#2}}
\newcommand{\vcrref}[2][]{\crref[vref,#1]{#2}}
\newcommand{\vpropref}[2][]{\propref[vref,#1]{#2}}
\newcommand{\vdefpageref}[2][]{\defpageref[vref,#1]{#2}}
\newcommand{\vexpageref}[2][]{\expageref[vref,#1]{#2}}

\usepackage[agsm,dcucite]{harvard}  % Citations style Harvard
\renewcommand{\harvardand}[1]{\& }
\renewcommand{\harvardurl}[1]{\url{#1}}
\newcommand{\citefullname}[2]{#2 \citeasnoun{#1}}

\usepackage{url}     % Mise en forme + liens pour URLs
\usepackage{hyperref}   % Liens sur les références
\hypersetup{colorlinks=true, linkcolor=black, urlcolor=black}

% Divers
%\usepackage{color}     % Couleurs du texte
\usepackage{array}      % Tableaux évolués
\usepackage{comment}    % Blocs de commentaires
\usepackage{enumerate}  % Personnalisation de la numérotation des listes
\usepackage{ziffer}     % Typographie française pour les nombres
\ZifferAus

% Espace avant le ; en mode maths
\makeatletter
% commande adaptée à partir de celle donnée pour la factorielle dans
% \url{http://www.math.bme.hu/latex/magyar_pre_tug2004.pdf#page=28}
\def\fixmathspacing#1#2#3{%
\def#1{#3}%
\expandafter\addto\csname \expandafter\ifx
\csname mathoptions@on\endcsname\relax% detect nath.sty
check@mathfonts\else mathoptions@on\fi\endcsname{%
\catcode`#212 \mathcode`#2"8000
\begingroup\lccode`~`#2\lowercase{\endgroup\def~}{#1}}
}
\makeatother
\fixmathspacing{\fixedsemicolon}{;}{\mathclose{}\mathpunct{}\mathpunct{\mathchar"603B}}



% Types de théorèmes
\usepackage{amsthm}

\newtheorem{property}{Propriété}[chapter]
\newtheorem{gtheorem}{Théorème}[chapter]
\newgraytheorem{theorem}{gtheorem}{framedgraybox}
% \newenvironment{theorem}{\vskip 5\p@\begin{graybox}\begin{gtheorem}}{%
%   \end{gtheorem}\end{graybox}\vskip 5\p@}
\newtheorem{lemma}{Lemme}[chapter]
\newtheorem{gproposition}{Proposition}[chapter]
\newgraytheorem{proposition}{gproposition}{grayframedbox}

\theoremstyle{definition}
\newtheorem{gdefinition}{Définition}[chapter]
\newgraytheorem{definition}{gdefinition}{graybox}
% \newenvironment{definition}{\vskip 5\p@\begin{framedgraybox}\begin{gdefinition}}{%
%   \end{gdefinition}\end{graybox}\vskip 5\p@}
\newtheorem{critere}{Critère}[chapter]

\theoremstyle{remark}
\newtheorem*{gexample}{Exemple}
%\newenvironment{example}{\myskip\noindent\textit{\textcolor{darkgray}{Exemple.}}}{}
\newenvironment{example}{\myskip\noindent\colorbox{couleurchapeau}{%
  \rule{0pt}{.8em}\textit{Exemple.}}}{}
\newtheorem*{remark}{Remarque}



% Macros et styles
% Macros pour les notations mathématiques

% Macros définitions
%\def\DEF{\stackrel{\Delta}=}
%\def\EQDEF{\stackrel{\Delta}\Leftrightarrow}
\def\DEF{\stackrel{def}=}
\def\EQDEF{\stackrel{def}\Leftrightarrow}

% Macros générales
\newcommand{\sR}{\mathbb{R}}
\newcommand{\sN}{\mathbb{N}}
\newcommand{\sNN}{\mathbb{N}^\bullet}
%\newcommand{\segm}[2]{\llbracket #1; #2 \rrbracket}
%\newcommand{\couplegenerique}[5]{\mathopen{#1}#2\mathclose{}\mathpunct{}#3#4\mathclose{#5}}
\newcommand{\couplegenerique}[5]{\mathopen{#1}#2\mathclose{}#3#4\mathclose{#5}}
\newcommand{\segm}[2]{\couplegenerique{\llbracket}{#1}{;}{#2}{\rrbracket}}
\newcommand{\interv}[2]{\couplegenerique{[}{#1}{;}{#2}{]}}
\newcommand{\couple}[2]{\couplegenerique{(}{#1}{;}{#2}{)}}

\def\indexes#1{\mathbb{I}^{#1}}
\def\f#1{\mathsf{#1}}

% Notations générales pour PH
\newcommand{\PH}{\mathcal{PH}}
%\newcommand{\PHs}{\mathcal{S}}
\newcommand{\PHs}{\Sigma}
%\newcommand{\PHp}{\mathcal{P}}
\newcommand{\PHp}{\textcolor{red}{\mathcal{P}}}
%\newcommand{\PHproc}{\mathcal{P}}
\newcommand{\PHproc}{\mathbf{Proc}}
\newcommand{\Proc}{\PHproc}
\newcommand{\PHh}{\mathcal{H}}
\newcommand{\PHa}{\PHh}
%\newcommand{\PHa}{\mathcal{A}}
\newcommand{\PHl}{\mathcal{L}}

\newcommand{\PHhitter}{\mathsf{frappeur}}
\newcommand{\PHtarget}{\mathsf{cible}}
\newcommand{\PHbounce}{\mathsf{bond}}
\newcommand{\PHinvariant}{\mathsf{invariant}}
%\newcommand{\PHsort}{\Sigma}
\newcommand{\PHsort}{\mathsf{sorte}}
\newcommand{\sort}[1]{\PHsort(#1)}
\newcommand{\sorte}{\sort}
\newcommand{\PHsorts}{\mathsf{sortes}}
\newcommand{\sortes}[1]{\PHsorts(#1)}
\newcommand{\sorts}{\sortes}

\newcommand{\hitter}[1]{\PHhitter(#1)}
\newcommand{\target}[1]{\PHtarget(#1)}
\newcommand{\bounce}[1]{\PHbounce(#1)}
\newcommand{\frappeur}[1]{\PHhitter(#1)}
\newcommand{\cible}[1]{\PHtarget(#1)}
\newcommand{\bond}[1]{\PHbounce(#1)}
\newcommand{\invariant}[1]{\PHinvariant(#1)}

% Priorités
\newcommand{\prio}{\mathsf{prio}}

% Arcs neutralisants
\newcommand{\PHn}{\mathcal{N}}
\newcommand{\PHansymbol}{\rtimes}
%\newcommand{\PHansymbol}{\multimap}
%\newcommand{\PHansymbol}{-\!\!\!\bullet\ }
\newcommand{\PHan}[2]{#1\PHansymbol#2}
\newcommand{\PHbloquant}{\mathsf{bloquante}}
\newcommand{\PHbloque}{\mathsf{bloqu\acute{e}e}}

%\newcommand{\PHfrappeur}{\mathsf{frappeur}}
%\newcommand{\PHcible}{\mathsf{cible}}
%\newcommand{\PHbond}{\mathsf{bond}}
%\newcommand{\PHsorte}{\mathsf{sorte}}
%\newcommand{\PHbloquant}{\mathsf{bloquante}}
%\newcommand{\PHbloque}{\mathsf{bloquee}}

%\newcommand{\PHfrappeR}{\textcolor{red}{\rightarrow}}
%\newcommand{\PHmonte}{\textcolor{red}{\Rsh}}

\newcommand{\PHhitA}{\rightarrow}
\newcommand{\PHhitB}{\Rsh}
%\newcommand{\PHfrappe}[3]{\mbox{$#1\PHhitA#2\PHhitB#3$}}
%\newcommand{\PHfrappebond}[2]{\mbox{$#1\PHhitB#2$}}
\newcommand{\PHhit}[3]{#1\PHhitA#2\PHhitB#3}
\newcommand{\PHfrappe}{\PHhit}
\newcommand{\PHhbounce}[2]{#1\PHhitB#2}
\newcommand{\PHobj}[2]{\mbox{$#1\PHhitB^*\!#2$}}
\newcommand{\PHconcat}{::}
\newcommand{\cons}{::}
%\newcommand{\PHneutralise}{\rtimes}

\newcommand{\PHget}[2]{{#1[#2]}}
\newcommand{\get}{\PHget}
%\def\PHget#1#2{{#1[#2]}}
%\newcommand{\PHchange}[2]{#1\langle #2 \rangle}
%\newcommand{\PHchange}[2]{(#1 \Lleftarrow #2)}
%\newcommand{\PHarcn}[2]{\mbox{$#1\PHneutralise#2$}}
\newcommand{\PHplay}{\cdot}

\newcommand{\PHstate}[1]{\mbox{$\langle #1 \rangle$}}
\newcommand{\etat}[1]{\PHstate{#1}}

\newcommand{\premsymbol}{\mathsf{premier}}
\newcommand{\dersymbol}{\mathsf{dernier}}
\newcommand{\suppsymbol}{\mathsf{support}}
\newcommand{\finsymbol}{\mathsf{fin}}
\newcommand{\prem}[2]{\premsymbol_{#1}(#2)}
\newcommand{\der}[2]{\dersymbol_{#1}(#2)}
\newcommand{\supp}[1]{\suppsymbol(#1)}
\newcommand{\fin}[1]{\finsymbol(#1)}

\def\DNtrans{\rightarrow_{ADN}}
\def\DNdef{(\mathbb F, \langle f^1, \dots, f^n\rangle)}
\def\DNdep{\f{dep}}
%\def\PHPtrans{\rightarrow_{PH}}
\newcommand{\PHtrans}[1][\PH]{\trans{#1}}
\def\get#1#2{#1[{#2}]}
\def\encodeF#1{\mathbf{#1}}
\def\toPH{\encodeF{PH}}
\def\card#1{|#1|}
\def\decode#1{\llbracket#1\rrbracket}
\def\encode#1{\llparenthesis#1\rrparenthesis}
\def\Hits{\PHa}
\def\hit{\PHhit}
\def\play{\cdot}
\newcommand{\recouvre}{\Cap}
%\newcommand{\recouvre}{/}

% General macros
\newcommand{\angles}[1]{{\langle #1 \rangle}}
\newcommand{\trans}[1]{\rightarrow_{#1}}
\newcommand{\mtrans}[1]{\rightsquigarrow_{#1}}
%\newcommand{\bigtimes}[1]{\underset{#1}{\times}}
%\newcommand{\bigtimes}[1]{\times_{#1}}
\newcommand{\bigtimes}[1]{\bigotimes_{#1}}
\newcommand{\emptyseq}{\epsilon}

% Sub-sets
\newcommand{\PHsublize}[1]{#1^\lozenge}
\newcommand{\PHsubl}[1][\PHl]{\PHsublize{#1}}
%\newcommand{\PHsublsetize}[1]{#1^{\blacklozenge}}
%\newcommand{\PHsublset}[1][\PHl]{\PHsublsetize{#1}}
\newcommand{\PHsublsetize}[1]{#1^{\lozenge}}
\newcommand{\PHsublset}[1][\Proc]{\PHsublsetize{#1}}

% Actions plurielles
\newcommand{\PHhitmultsymbol}{\rightarrowtail}
\newcommand{\PHhitmult}[2]{\mbox{$#1 \PHhitmultsymbol #2$}}
\newcommand{\PHfrappemult}{\PHhitmult}
\newcommand{\PHfrappemults}[2]{\PHhitmult{\{#1\}}{\{#2\}}}

% Traduction en réseau d'automates
\newcommand{\phmtoansymbol}{\f{toRAS}}
\newcommand{\phmtoan}[1][\PH]{\phmtoansymbol(#1)}
\newcommand{\antophmsymbol}{\f{toPH}}
\newcommand{\antophm}[1][\AN]{\antophmsymbol(#1)}
\newcommand{\AN}{\f{RAS}}
\newcommand{\ANs}{\PHs}
\newcommand{\ANl}{\PHl}
\newcommand{\ANi}{I}
\newcommand{\ANt}{T}
\newcommand{\precond}[1]{{}^\bullet #1}
\newcommand{\postcond}[1]{#1 {}^\bullet}
\newcommand{\invcond}[1]{\precond{\postcond{#1}}}
\newcommand{\ANaction}[3]{#1 \xrightarrow{#2} #3}
\newcommand{\ANtrans}{\trans{\AN}}
\newcommand{\ANProc}{\mathbf{\acute{E}L}}

% Traduction en PHp
%\newcommand{\phmtophpsymbol}{\f{PHp}}
%\newcommand{\phmtophp}[1][\PH]{\phmtophpsymbol(#1)}
\newcommand{\phmtophp}[1][\PH]{\llparenthesis #1 \rrparenthesis}
\newcommand{\tophp}[1]{\llparenthesis #1 \rrparenthesis}
\newcommand{\tophm}[1]{(\!(#1)\!)}
\newcommand{\PHssr}[1][\PHs']{#1_{\srsymbol}}
\newcommand{\PHsscf}[1][\PHs']{#1_{\scfsymbol}}
\newcommand{\srsymbol}{r}
\newcommand{\sr}[1]{\srsymbol^{#1}}
\newcommand{\scfsymbol}{f}
\newcommand{\scf}[1]{\scfsymbol^{#1}}

% Propriétés de jouabilité
\newcommand{\F}{F}
\newcommand{\Fopsymbol}{\F}
\newcommand{\Fop}[1]{\Fopsymbol(#1)}
\newcommand{\Fconj}[1]{\mathsf{conj}(#1)}

% Opérateurs de jouabilité
\newcommand{\Feval}[2]{\left[#1\right](#2)}

\newcommand{\Fopph}[1]{\Fopsymbol(#1)}
\newcommand{\Fopphansubsymbol}{{an}}
\newcommand{\Fopphan}[1]{\Fopsymbol_\Fopphansubsymbol(#1)}
\newcommand{\Fopphpsubsymbol}{{p}}
\newcommand{\Fopphp}[1]{\Fopsymbol_\Fopphpsubsymbol(#1)}
\newcommand{\Fopphmsubsymbol}{{plur}}
\newcommand{\Fopphm}[1]{\Fopsymbol_\Fopphmsubsymbol(#1)}

% Merge / Fusion
\newcommand{\PHmergesymbol}{\f{fusion}}
\newcommand{\PHmerge}{\PHmergesymbol}
\newcommand{\PHmergean}{\PHmergesymbol_{an}}

% Flattening / Aplatissement
\newcommand{\ov}{\overline}
\newcommand{\os}{\ov{s}}
\newcommand{\oPH}{\ov{\PH}}

\newcommand{\PHdep}[2]{\f{dep}^{#1}(#2)}
\newcommand{\PHflat}{\f{flat}}
\newcommand{\flats}[1]{\lceil #1 \rceil}
\newcommand{\unflats}[1]{\lfloor #1 \rfloor}

\newcommand{\Fopaplatsubsymbol}{{plat}}
\newcommand{\Fopaplat}[1]{\Fopsymbol_\Fopaplatsubsymbol(#1)}

\newcommand{\mysigma}{\sigma}
\newcommand{\mypi}{\pi}
\newcommand{\n}[1][h]{n^{#1}}
\newcommand{\m}[1][h,i]{m^{#1}}

\def\toset#1{\widetilde{#1}}

\newcommand{\virtualhitters}{\f{FrappeursVirtuels}}

% PHcanonique vers PHm
\newcommand{\PHmult}{\f{plur}}

% PH canonique
\newcommand{\components}{\mathrm{\Gamma}}
\newcommand{\cs}{\Delta}
\def\conn{\f{conn}^1}
\def\compin{\f{comp}^1}
\newcommand{\update}{\mathsf{update}}
% \def\restriction#1#2{\mathchoice
%               {\setbox1\hbox{${\displaystyle #1}_{\scriptstyle #2}$}
%               \restrictionaux{#1}{#2}}
%               {\setbox1\hbox{${\textstyle #1}_{\scriptstyle #2}$}
%               \restrictionaux{#1}{#2}}
%               {\setbox1\hbox{${\scriptstyle #1}_{\scriptscriptstyle #2}$}
%               \restrictionaux{#1}{#2}}
%               {\setbox1\hbox{${\scriptscriptstyle #1}_{\scriptscriptstyle #2}$}
%               \restrictionaux{#1}{#2}}}
% \def\restrictionaux#1#2{{#1\,\smash{\vrule height .8\ht1 depth .85\dp1}}_{\,#2}} 

%\newcommand{\reductionsymbol}[1]{#1_{\restriction_{1}}}
%\newcommand{\reductionsymbol}[1]{#1_{|_{1}}}
\newcommand{\reductionsymbol}[1]{#1^1}
\newcommand{\reduction}[1][\PH]{\reductionsymbol{#1}}
\newcommand{\reductionsce}[1][\Sce]{\reductionsymbol{#1}}
\newcommand{\tempop}{\mathsf{P}}

% Modèle de Thomas
\newcommand{\RRB}{\mathsf{T}}
\newcommand{\GI}{\mathcal{G}}
\newcommand{\RRBreg}[1]{\Gamma^{-1}(#1)}
\newcommand{\RRBressymbol}{\mathsf{Res}}
\newcommand{\RRBres}[2]{\RRBressymbol_{#1}(#2)}
\newcommand{\RRBallres}[1]{\textcolor{red}{\mathsf{Res}_{#1}}}
\newcommand{\RRBget}[2]{\PHget{#1}{#2}}
\newcommand{\RRBetat}[1]{\etat{#1}}
\newcommand{\RRBstate}[1]{\RRBetat{#1}}
\newcommand{\RRBstates}{\mathcal{S}}
\newcommand{\RRBedgef}[4]{#1 \xrightarrow{#2 #3} #4}
\newcommand{\arcf}{\RRBedgef}
\newcommand{\arc}[4]{\RRBedgef{#1}{#2,}{#3}{#4}}
%\newcommand{\RRBtrans}[2]{#1 \trans{\RRB} #2}
\newcommand{\RRBtrans}[2]{#1 \trans{} #2}
\newcommand{\RDA}{\f{RDA}}
\newcommand{\RRBtransrda}[2]{#1 \trans{\RDA} #2}
%\newcommand{\uns}{\pm}
\newcommand{\uns}{\circ}

\newcommand{\levelssymbol}{\mathsf{niveaux}}
\newcommand{\levels}[2]{\levelssymbol(#1\rightarrow #2)}
\newcommand{\ulevels}[2]{\overline{\levelssymbol}(#1\rightarrow #2)}

\newcommand{\leqsegm}{\leq_{\llbracket\rrbracket}}
\newcommand{\ltsegm}{<_{\llbracket\rrbracket}}

% Traduction vers le modèle de Thomas
\newcommand{\irB}{B}
\newcommand{\irF}{F}
\newcommand{\pred}{\f{pred}}
\newcommand{\reg}{\f{reg}}
\newcommand{\allFocals}[3]{\mathsf{allFocals}^{#1}_{#2}(#3)}
\newcommand{\focals}{\f{foc}}

% Réseaux de Petri
\newcommand{\PT}{\f{RdP}}
\newcommand{\PTp}{P}
\newcommand{\PTt}{T}
\newcommand{\PTPre}{Pre}
\newcommand{\PTPost}{Post}
\newcommand{\PTLect}{Lect}
\newcommand{\PTInh}{Inh}
\newcommand{\PTtrans}{\trans{\PT}}

\newcommand{\phmtoptsymbol}{\f{toRdP}}
\newcommand{\phmtopt}[1][\PH]{\phmtoptsymbol(#1)}

% Biocham
%\newcommand{\BCc}{C}
\newcommand{\BCe}{E}
\newcommand{\BCc}[1][\BCe]{C^{#1}}

\newcommand{\bctophmsymbol}{\f{toPH}}
\newcommand{\bctophm}[1][\BCe]{\bctophmsymbol(#1)}
\newcommand{\BCtrans}[1][\BCe]{\trans{#1}}

\usepackage{ifthen}
\usepackage{tikz}
\usetikzlibrary{arrows,shapes}

\definecolor{lightgray}{rgb}{0.8,0.8,0.8}
\definecolor{lightgrey}{rgb}{0.8,0.8,0.8}

\tikzstyle{boxed ph}=[]
\tikzstyle{sort}=[fill=lightgray,rounded corners]
\tikzstyle{process}=[circle,draw,minimum size=15pt,fill=white,
font=\footnotesize,inner sep=1pt]
\tikzstyle{black process}=[process, fill=black,text=white, font=\bfseries]
\tikzstyle{gray process}=[process, draw=black, fill=lightgray]
\tikzstyle{current process}=[process, draw=black, fill=lightgray]
\tikzstyle{process box}=[white,draw=black,rounded corners]
\tikzstyle{tick label}=[font=\footnotesize]
\tikzstyle{tick}=[black,-]%,densely dotted]
\tikzstyle{hit}=[->,>=stealth']
\tikzstyle{selfhit}=[min distance=30pt,curve to]
\tikzstyle{bounce}=[densely dotted,>=stealth',->]
\tikzstyle{hl}=[font=\bfseries,very thick]
\tikzstyle{hl2}=[hl]
\tikzstyle{nohl}=[font=\normalfont,thin]

\tikzstyle{prio}=[draw,very thick,-stealth]
\tikzstyle{coopupdate}=[dashed]
\tikzstyle{superprio}=[draw,very thick,double,-stealth]

\newcommand{\currentScope}{}
\newcommand{\currentSort}{}
\newcommand{\currentSortLabel}{}
\newcommand{\currentAlign}{}
\newcommand{\currentSize}{}

\newcounter{la}
\newcommand{\TSetSortLabel}[2]{
  \expandafter\repcommand\expandafter{\csname TUserSort@#1\endcsname}{#2}
}
\newcommand{\TSort}[4]{
  \renewcommand{\currentScope}{#1}
  \renewcommand{\currentSort}{#2}
  \renewcommand{\currentSize}{#3}
  \renewcommand{\currentAlign}{#4}
  \ifcsname TUserSort@\currentSort\endcsname
    \renewcommand{\currentSortLabel}{\csname TUserSort@\currentSort\endcsname}
  \else
    \renewcommand{\currentSortLabel}{\currentSort}
  \fi
  \begin{scope}[shift={\currentScope}]
  \ifthenelse{\equal{\currentAlign}{l}}{
    \filldraw[process box] (-0.5,-0.5) rectangle (0.5,\currentSize-0.5);
    \node[sort] at (-0.2,\currentSize-0.4) {\currentSortLabel};
   }{\ifthenelse{\equal{\currentAlign}{r}}{
     \filldraw[process box] (-0.5,-0.5) rectangle (0.5,\currentSize-0.5);
     \node[sort] at (0.2,\currentSize-0.4) {\currentSortLabel};
   }{
    \filldraw[process box] (-0.5,-0.5) rectangle (\currentSize-0.5,0.5);
    \ifthenelse{\equal{\currentAlign}{t}}{
      \node[sort,anchor=east] at (-0.3,0.2) {\currentSortLabel};
    }{
      \node[sort] at (-0.6,-0.2) {\currentSortLabel};
    }
   }}
  \setcounter{la}{\currentSize}
  \addtocounter{la}{-1}
  \foreach \i in {0,...,\value{la}} {
    \TProc{\i}
  }
  \end{scope}
}

\newcommand{\TTickProc}[2]{ % pos, label
  \ifthenelse{\equal{\currentAlign}{l}}{
    \draw[tick] (-0.6,#1) -- (-0.4,#1);
    \node[tick label, anchor=east] at (-0.55,#1) {#2};
   }{\ifthenelse{\equal{\currentAlign}{r}}{
    \draw[tick] (0.6,#1) -- (0.4,#1);
    \node[tick label, anchor=west] at (0.55,#1) {#2};
   }{
    \ifthenelse{\equal{\currentAlign}{t}}{
      \draw[tick] (#1,0.6) -- (#1,0.4);
      \node[tick label, anchor=south] at (#1,0.55) {#2};
    }{
      \draw[tick] (#1,-0.6) -- (#1,-0.4);
      \node[tick label, anchor=north] at (#1,-0.55) {#2};
    }
   }}
}
\newcommand{\TSetTick}[3]{
  \expandafter\repcommand\expandafter{\csname TUserTick@#1_#2\endcsname}{#3}
}

\newcommand{\myProc}[3]{
  \ifcsname TUserTick@\currentSort_#1\endcsname
    \TTickProc{#1}{\csname TUserTick@\currentSort_#1\endcsname}
  \else
    \TTickProc{#1}{#1}
  \fi
  \ifthenelse{\equal{\currentAlign}{l}\or\equal{\currentAlign}{r}}{
    \node[#2] (\currentSort_#1) at (0,#1) {#3};
  }{
    \node[#2] (\currentSort_#1) at (#1,0) {#3};
  }
}
\newcommand{\TSetProcStyle}[2]{
  \expandafter\repcommand\expandafter{\csname TUserProcStyle@#1\endcsname}{#2}
}
\newcommand{\TProc}[1]{
  \ifcsname TUserProcStyle@\currentSort_#1\endcsname
    \myProc{#1}{\csname TUserProcStyle@\currentSort_#1\endcsname}{}
  \else
    \myProc{#1}{process}{}
  \fi
}

\newcommand{\repcommand}[2]{
  \providecommand{#1}{#2}
  \renewcommand{#1}{#2}
}
\newcommand{\THit}[5]{
  \path[hit] (#1) edge[#2] (#3#4);
  \expandafter\repcommand\expandafter{\csname TBounce@#3@#5\endcsname}{#4}
}
\newcommand{\TBounce}[4]{
  (#1\csname TBounce@#1@#3\endcsname) edge[#2] (#3#4)
}

\newcommand{\TState}[1]{
  \foreach \proc in {#1} {
    \node[current process] (\proc) at (\proc.center) {};
  }
}

% procedure, abstractions and dependencies
\newcommand{\abstr}[1]{#1^\wedge}%\text{\textasciicircum}}
\def\BS{\mathbf{BS}}
\def\aBS{\abstr{\BS}}
\def\abeta{\abstr{\beta}}
\def\aZ{\abstr{\zeta}}
\def\aY{\abstr{\xi}}

\def\beforeproc{\vartriangleleft}

\def\powerset{\wp}

\def\Sce{\mathbf{Sce}}
\def\OS{\mathbf{OS}}
\def\Obj{\mathbf{Obj}}
%\def\Proc{\mathbf{Proc}}
%\def\Sol{\mathbf{Sol}}
\newcommand{\Sol}{\mathbf{Sol}}

\usepackage{galois}
\newcommand{\theOSabstr}{toOS}
\newcommand{\OSabstr}[1]{\theOSabstr(#1)}
\newcommand{\theOSconcr}{toSce}
\newcommand{\OSconcr}[1]{\theOSconcr(#1)}

% \def\gO{\mathbb{O}}
% \def\gS{\mathbb{S}}
\def\aS{\mathcal{A}}
\def\Req{\mathrm{Req}}
%\def\Sol{\mathrm{Sol}}
\def\Cont{\mathrm{Cont}}
\def\cBS{\BS_\ctx}
\def\caBS{\aBS_\ctx}
\def\caS{\aS_\ctx}
\def\cSol{\Sol_\ctx}
\def\cReq{\Req_\ctx}
\def\cCont{\Cont_\ctx}

\def\any{\star}

% \def\gProc{\mathrm{maxPROC}}
\def\mCtx{\mathrm{maxCtx}}

%\def\procs{\f{procs}}
\def\objs{\f{objs}}
\def\sat#1{\lceil #1\rceil}

\def\gCont{\f{maxCont}}
\def\lCont{\f{minCont}}
\def\lProc{\f{minProc}}
\def\gProc{\f{maxProc}}

\def\join{\oplus}
\def\concat{\!::\!}
\def\emptyseq{\varepsilon}
\def\ltw{\preccurlyeq_{\OS}}
\def\indexes#1{\mathbb{I}^{#1}}
%\def\indexes#1{\{1..|#1|\}}
\def\supp{\f{support}}
\def\w{\omega}
\def\W{\Omega}
\def\ctx{\varsigma}
\def\Ctx{\mathbf{Ctx}}
\def\mconcr{\gamma}
\def\concr{\mconcr_\ctx}
\def\obj#1#2{{#1\!\Rsh^*\!\!#2}}
\def\objp#1#2#3{\obj{{#1}_{#2}}{{#1}_{#3}}}
\def\A{\mathcal{A}}
\def\cwA{\A_\ctx^\w}
\def\cwReq{\Req_\ctx^\w}
\def\cwSol{\Sol_\ctx^\w}
\def\cwCont{\Cont_\ctx^\w}
\def\gCtx{\f{maxCtx}}
\def\endCtx{\f{endCtx}}
\def\ceil{\f{end}}

%\def\lfp{\mathrm{lfp}\;}
%\def\mlfp#1{\mathrm{lfp}\{#1\}\;}
\newcommand{\lfp}[3]{\mathbf{lfp}\{#1\}\left(#2\mapsto#3\right)}
\def\maxobjs{{\f{maxobjs}}}
\def\maxprocs{{\f{maxprocs}_\ctx}}
\def\objends{{\f{ends}}}

\def\ra{\rho}
\def\rb{\rho^\wedge}
\def\rc{\widetilde{\rho}}
\def\interleave{\f{interleave}}

\def\join{\concat}

\tikzstyle{aS}=[every edge/.style={draw,->,>=stealth}]
\tikzstyle{Asol}=[draw,circle,minimum size=5pt,inner sep=0,node distance=1cm]
\tikzstyle{Aproc}=[draw,node distance=1.2cm]
\tikzstyle{Aobj}=[node distance=1.5cm]
\tikzstyle{Anos}=[font=\Large]

%\tikzstyle{AprocPrio}=[Aproc,double]
\tikzstyle{AsolPrio}=[Asol,double]
\tikzstyle{AprocPrio}=[Aproc,double]
\tikzstyle{aSPrio}=[aS,double]



\def\procs{\mathsf{procs}}
%\def\allprocs{\mathsf{allProcs}}
\def\allprocs{\procs}
%\def\pfp{\mathsf{pfp}}
\def\pfp{\mathsf{focals}^1}
\def\pfpprocs{\mathsf{pfpProcs}}
\def\bounceprocs{\mathsf{bounceProcs}}
\def\newprocs{\mathsf{newProcs}}

\def\aB{\mathcal{B}}
\def\sat#1{\lceil #1\rceil}
\def\cwB{\sat{\aB_\ctx^\w}}
\def\mycwB#1#2{\sat{\aB_{#1}^{#2}}}
\def\Bsol{\sat{\Sol^\w_\ctx}}
\def\Breq{\sat{\Req^\w_\ctx}}
\def\Bcont{\sat{\Cont^\w_\ctx}}

\def\myB{\aB^\w_\ctx}
\def\mysol{\overline{\Sol^\w_\ctx}}
\def\myreq{\overline{\Req^\w_\ctx}}
\def\mycont{\overline{\Cont^\w_\ctx}}

\begin{comment}
\def\PrioCont{\textcolor{red}{\mathrm{PrioCont}}}
\def\mypriocont{\overline{\PrioCont^\w_\ctx}}
\def\cwPrioCont{\PrioCont_\ctx^\w}
\def\Bpriocont{\sat{\PrioCont^\w_\ctx}}
\def\Sat{\PrioCont}
\def\mysat{\overline{\Sat^\w_\ctx}}
\def\cwSat{\Sat_\ctx^\w}
\def\Bsat{\sat{\Sat^\w_\ctx}}

\def\ReqSolPrio{\textcolor{blue}{\mathrm{ReqSolPrio}}}
\def\RSP{\ReqSolPrio}
\def\myrsp{\overline{\RSP^\w_\ctx}}
\def\cwRSP{\RSP_\ctx^\w}
\def\Brsp{\sat{\RSP^\w_\ctx}}
\end{comment}

\newcommand{\csState}{\mathsf{procState}}

\newcommand{\V}{V}
\newcommand{\E}{E}
\newcommand{\cwV}{\V_\ctx^\w}
\newcommand{\cwE}{\E_\ctx^\w}
%\newcommand{\VProc}{\textcolor{red}{\V_\PHproc}}
%\newcommand{\VObj}{\textcolor{red}{\V_\Obj}}
%\newcommand{\VSol}{\V_{Sol}}
%\newcommand{\VSol}{\textcolor{red}{\V_{\Sol}}}
\newcommand{\VProc}{\V \cap \PHproc}
\newcommand{\VObj}{\V \cap \Obj}
\newcommand{\VSol}{\V \cap \Sol}
\newcommand{\cwVProc}{\cwV \cap \PHproc}
\newcommand{\cwVObj}{\cwV \cap \Obj}
\newcommand{\cwVSol}{\cwV \cap \Sol}

\def\Bv{\sat{\cwV}}
\def\Be{\sat{\cwE}}
\def\BvProc{\textcolor{red}{\sat{\cwV}^\PHproc}}
\def\BvObj{\textcolor{red}{\sat{\cwV}^\Obj}}
%\def\BvSol{\sat{\cwV}^{Sol}}
\def\BvSol{\textcolor{red}{\sat{\cwV}^{\Sol}}}

\newcommand{\Bee}[2]{\Be^{#1}_{#2}}

%\def\mlfp#1{\f{pppf}\{#1\}}

\def\PHobjp#1#2#3{\PHobj{{#1}_{#2}}{{#1}_{#3}}}
\def\Obj{\mathbf{Obj}}
\def\powerset{\wp}
\def\gCont{\f{maxCont}}

\def\muconcr{\ell}
\def\uconcr{\muconcr_\ctx}

\begin{comment}
%\newcommand{\abstr}[1]{#1^\wedge}%\text{\textasciicircum}}
%\def\priomax{\mathsf{prio}_{max}}
\def\procs{\mathsf{procs}}
\def\allprocs{\mathsf{allProcs}}
\def\pfp{\mathsf{pfp}}
\def\pfpprocs{\mathsf{pfpProcs}}
%
\def\ctx{\varsigma}
\def\w{\omega}
%\def\aBS{\abstr{\BS}}
%
\def\Req{\mathrm{Req}}
\def\Sol{\mathrm{Sol}}
\def\Cont{\mathrm{Cont}}
\def\A{\mathcal{A}}
\def\cwA{\A_\ctx^\w}
\def\cwReq{\Req_\ctx^\w}
\def\cwSol{\Sol_\ctx^\w}
\def\cwCont{\Cont_\ctx^\w}
%
%
%
\end{comment}

\def\lastprocs{\f{lastprocs}}


% Styles TikZ et couleurs personnalisées

\usepackage{tikz}

\newdimen\pgfex
\newdimen\pgfem
\usetikzlibrary{arrows,shapes,shadows,scopes}
\usetikzlibrary{positioning}
\usetikzlibrary{matrix}
\usetikzlibrary{decorations.text}
\usetikzlibrary{decorations.pathmorphing}
\usetikzlibrary{arrows,shapes}

\definecolor{lightgray}{rgb}{0.8,0.8,0.8}
\definecolor{lightgrey}{rgb}{0.8,0.8,0.8}

\definecolor{lightred}{rgb}{1,0.8,0.8}
\definecolor{lightgreen}{rgb}{0.7,1,0.7}
\definecolor{darkgreen}{rgb}{0,0.5,0}
\definecolor{darkblue}{rgb}{0,0,0.5}
\definecolor{darkyellow}{rgb}{0.5,0.5,0}
\definecolor{lightyellow}{rgb}{1,1,0.6}
\definecolor{darkcyan}{rgb}{0,0.6,0.6}
\definecolor{lightcyan}{rgb}{0.6,1,1}
\definecolor{darkorange}{rgb}{0.8,0.2,0}
\definecolor{notsodarkred}{rgb}{0.8,0,0}

\definecolor{notsodarkgreen}{rgb}{0,0.7,0}

%\definecolor{coloract}{rgb}{0,1,0}
%\definecolor{colorinh}{rgb}{1,0,0}
\colorlet{coloract}{darkgreen}
\colorlet{colorinh}{red}
\colorlet{coloractgray}{lightgreen}
\colorlet{colorinhgray}{lightred}
\colorlet{colorinf}{darkgray}
\colorlet{coloractgray}{lightgreen}
\colorlet{colorinhgray}{lightred}

\colorlet{colorgray}{lightgray}
\colorlet{colorhl}{blue}


\tikzstyle{boxed ph}=[]
\tikzstyle{sort}=[fill=lightgray, rounded corners, draw=black]
\tikzstyle{process}=[circle,draw,minimum size=15pt,fill=white,font=\footnotesize,inner sep=1pt]
%\tikzstyle{black process}=[process, draw=blue, fill=red,text=black,font=\bfseries]
\tikzstyle{gray process}=[process, draw=black, fill=lightgray]
\tikzstyle{highlighted process}=[current process, fill=gray]
\tikzstyle{process box}=[fill=none,draw=black,rounded corners]
%\tikzstyle{current process}=[process, draw=black, fill=lightgray]
\tikzstyle{current process}=[process,fill=lightgray]
\tikzstyle{hl process}=[process,fill=blue!30]
\tikzstyle{tick label}=[font=\footnotesize]
\tikzstyle{tick}=[densely dotted] %-
\tikzstyle{hit}=[->,>=angle 45]
\tikzstyle{selfhit}=[min distance=50pt,curve to]
\tikzstyle{bounce}=[densely dotted,>=stealth',->]
\tikzstyle{ulhit}=[draw=lightgray,fill=lightgray]
\tikzstyle{pulhit}=[fill=lightgray]
\tikzstyle{bulhit}=[draw=lightgray]
\tikzstyle{hl}=[very thick,colorhl]
\tikzstyle{hlb}=[very thick]
\tikzstyle{hlhit}=[hl]
%\tikzstyle{hl2}=[hl]
%\tikzstyle{nohl}=[font=\normalfont,thin]

\tikzstyle{update}=[draw,->,dashed,shorten >=.7cm,shorten <=.7cm]

\tikzstyle{unprio}=[draw,thin]%[double]
%\tikzstyle{prio}=[draw,thick,-stealth]%[double]
\tikzstyle{prio}=[draw,-stealth,double]

\tikzstyle{hitless graph}=[every edge/.style={draw=red,-}]

\tikzstyle{aS}=[every edge/.style={draw,->,>=stealth}]
\tikzstyle{Asol}=[draw,circle,minimum size=5pt,inner sep=0,node distance=1cm]
\tikzstyle{Aproc}=[draw,node distance=1.2cm]
\tikzstyle{Aobj}=[node distance=1.5cm]
\tikzstyle{Anos}=[font=\Large]

\tikzstyle{AsolPrio}=[Asol,double]
\tikzstyle{AprocPrio}=[Aproc,double]
\tikzstyle{aSPrio}=[aS,double]

\colorlet{colorhlwarn}{notsodarkred}
\colorlet{colorhlwarnbg}{lightred}
\tikzstyle{Ahl}=[very thick,fill=colorhlwarnbg,draw=colorhlwarn,text=colorhlwarn]
\tikzstyle{Ahledge}=[very thick,double=colorhlwarnbg,draw=colorhlwarn,color=colorhlwarn]





%\definecolor{darkred}{rgb}{0.5,0,0}



\tikzstyle{grn}=[every node/.style={circle,draw=black,outer sep=2pt,minimum
                size=15pt,text=black}, node distance=1.5cm, ->]
\tikzstyle{inh}=[>=|,-|,draw=colorinh,thick, text=black,label]
\tikzstyle{act}=[->,>=triangle 60,draw=coloract,thick,color=coloract]
\tikzstyle{inhgray}=[>=|,-|,draw=colorinhgray,thick, text=black,label]
\tikzstyle{actgray}=[->,>=triangle 60,draw=coloractgray,thick,color=coloractgray]
\tikzstyle{inf}=[->,draw=colorinf,thick,color=colorinf]
%\tikzstyle{elabel}=[fill=none, above=-1pt, sloped,text=black, minimum size=10pt, outer sep=0, font=\scriptsize,draw=none]
\tikzstyle{elabel}=[fill=none,text=black, above=-2pt,%sloped,
minimum size=10pt, outer sep=0, font=\scriptsize, draw=none]
%\tikzstyle{elabel}=[]


\tikzstyle{plot}=[every path/.style={-}]
\tikzstyle{axe}=[black,->,>=stealth']
\tikzstyle{ticks}=[font=\scriptsize,every node/.style={black}]
\tikzstyle{mean}=[thick]
\tikzstyle{interval}=[line width=5pt,red,draw opacity=0.7]
%\definecolor{lightred}{rgb}{1,0.3,0.3}

%\tikzstyle{hl}=[yellow]
%\tikzstyle{hl2}=[orange]

%\tikzstyle{every matrix}=[ampersand replacement=\&]
%\tikzstyle{shorthandoff}=[]
%\tikzstyle{shorthandon}=[]
\tikzstyle{objective}=[process,very thick,fill=yellow!50]

\def\exmetazoan{
  \exmetazoansimplesorts
  \exmetazoancoopsort
  \exmetazoansimpleactions
  \exmetazoancoopactions
}

\def\exmetazoansimplesorts{
  \TSort{(0,4)}{c}{2}{l}
  \TSort{(1,0)}{f}{2}{l}
  \TSort{(7,4)}{a}{2}{r}
}

\def\exmetazoancoopsort{
  \TSetTick{fc}{0}{00}
  \TSetTick{fc}{1}{01}
  \TSetTick{fc}{2}{10}
  \TSetTick{fc}{3}{11}
  \TSort{(4,1)}{fc}{4}{r}
}

\def\exmetazoansimpleactions{
  \TAction{c_1}{a_1.west}{a_0.north west}{}{right}
  \TAction{f_1}{c_0.west}{c_1.south west}{bend left=30, in=90}{left}
  \TAction{c_1}{c_1.west}{c_0.north west}{selfhit}{right}
  \TAction{f_1.north east}{f_1.south east}{f_0.north east}%
    {selfhit, min distance=30, bend left, out=150, in=90}{left}
  \TAction{f_0.east}{c_1.south east}{c_0.north east}{bend right=60, in=-140}{left}
}

\def\exmetazoancoopactions{
  \TAction{fc_2}{a_0.west}{a_1.south west}{}{left}
  \path (1.8, 0.5) edge[coopupdate] (3.2, 2);
  \path (0.8, 4.5) edge[coopupdate] (3.2, 3);
}



% Raccourcis typographiques
\newcommand{\nbd}{\nobreakdash-}
%\newcommand{\cad}{c.\nbd à\nbd d.\ }
\newcommand{\cad}{c'est-à-dire\ }
\newcommand{\resp}{resp.\ }
\newcommand{\cf}{cf.~}
%\newcommand{\ie}{\cad}

% Styles supplémentaires
\newcommand{\bemph}[1]{\textbf{#1}}
\newcommand{\myskip}{\medskip}
\newcommand{\clspace}{\hspace*{2em}}
\newcommand{\cl}[1]{\\\texttt{\clspace#1}\\}
\newcommand{\tool}[1]{{\rmfamily #1}}

% Césures
\hyphenation{jou-a-bi-li-té}
\hyphenation{jou-a-bi-li-tés}
\hyphenation{at-tei-gna-bi-li-té}
\hyphenation{at-tei-gna-bi-li-tés}
\hyphenation{ap-pro-xi-ma-tion}
\hyphenation{ap-pro-xi-ma-tions}
\hyphenation{sur-ap-pro-xi-ma-tion}

% Raccourcis propres à ce document
\newcommand{\allcr}{\crref{tf,part}}
\newcommand{\enumparamcr}{\crrangeref{params-valide}{param-enum-monotonicity}}
\newcommand{\Pint}{\tool{Pint}}

%\newcommand{\stochainf}{{\small $\infty$}}
%\newcommand{\stochaa}{{\footnotesize $\spadesuit$}}
%\newcommand{\stochab}{{\footnotesize $\clubsuit$}}
%\newcommand{\stochac}{{\footnotesize $\blacktriangledown$}}
\newcommand{\stochainf}{$\infty$}
\newcommand{\stochaa}{$\mathbf{a}$}
\newcommand{\stochab}{$\mathbf{b}$}
\newcommand{\stochac}{$\mathbf{c}$}



% Paquets temporaires
\usepackage{lipsum}
%\usepackage[notref,notcite]{showkeys}

% Commandes À FAIRE
\newcommand{\todo}[1]{\textcolor{red}{\textbf{[[#1]]}}}
\newcommand{\todoplustard}[1]{\textcolor{blue}{\textbf{[[#1]]}}}
\newcommand{\TODO}{\todo{TODO}}
\newcommand{\stodo}[1]{\textcolor{red}{[#1]}}
\newcommand{\tocite}[1][citation]{\stodo{#1}}
\newcommand{\toref}{\stodo{ref}}
\newcommand{\storef}{\toref}
\newcommand{\pref}[1]{\textcolor{blue}{\ref{#1}~[\texttt{#1}]}}

% \newcommand{\todo}[1]{}
% \newcommand{\todoplustard}[1]{}
% \newcommand{\tocite}[1][]{}
%\newcommand{\todoexplicite}{\textcolor{blue}{\textbf{[Cette section est en cours de rédaction]}}}



\titre{Modélisation algébrique de la dynamique multi-échelles
  des réseaux de régulation biologique}
\soustitre{Méthodes de modélisation et d'analyse pour l'enrichissement des Frappes de Processus}

\title{Algebraic Modeling of the Dynamics of Multi-scale
  Biological Regulatory Networks}
\subtitle{Modeling and Analysis Methods for the Enrichment of Process Hitting}

\author{M.}{Maxime}{Folschette}
\discipline{Informatique et applications}
\sectionCNU{27}
\specialty{Bioinformatique}

\institution{ECN}
\doctoralschool{Sciences et technologies de l'information, et mathématiques}
\laboratory{Institut de Recherche en Communications et Cybernétique de Nantes (IRCCyN)\\
  UMR CNRS 6597}
%\thesisnumber{}
\date{8 octobre 2014 (prévision)}



% Étoiler les \reviewer, \supervisor et \cosupervisor absents
\reviewer{M.}{Jean-Paul}{Comet}{Professeur des universités}%
  {Université de Nice -- Sophia Antipolis}

\reviewer{Mme}{Anne}{siegel}{Directrice de recherche CNRS}%
  {IRISA (CNRS \& Université Rennes 1), Inria Rennes}

\examiner{Mme}{Mireille}{Régnier}{Directrice de recherche INRIA}%
  {École Polytechnique \& Université Paris-Sud 11}

\examiner{M.}{Denis}{Thieffry}{Professeur des universités}%
  {École normale supérieure}

%\president{Mme}{Prénom}{Nom}{Titre}{Établissement}
%\guest{Mme}{Prénom}{Nom}{Titre}{Établissement}

\supervisor{M.}{Olivier}{Roux}{Professeur des universités}%
  {École centrale de Nantes}

\cosupervisor{M.}{Morgan}{Magnin}{Maître de conférences}%
  {École centrale de Nantes}



\begin{document}

\begin{resume}
  %\footnotesize
  \scriptsize
  La représentation et l'analyse des grands réseaux de régulation biologique
  sont les deux défis majeurs dans la compréhension des mécanismes du vivant.
  Le travail exposé dans cette thèse se concentre sur les modèles discrets,
  souvent représentés sous la forme de graphes et d'ensembles de paramètres.
  Il s'inspire notamment d'un formalisme préalablement développé,
  appelé Frappes de Processus,
  qui repose sur une représentation atomique d'un ensemble de composants et de leur dynamique.
  Nous proposons dans cette thèse plusieurs représentations alternatives à
  ce formalisme, qui possèdent une plus grande expressivité.
  Ces représentations sont adaptées à l'intégration
  de contraintes discrètes dans les modèles
  provenant de durées relatives ou de relations de synchronisme entre certaines réactions.
  Nous proposons par ailleurs
  une méthode d'analyse de la dynamique par interprétation abstraite
  qui permet de répondre à des questions d'atteignabilité.
  Cette méthode est spécifiquement adaptée à l'étude des modèles de grande taille,
  pouvant contenir plusieurs centaines de composants, et potentiellement davantage.
  Elle repose en effet sur une approximation de la dynamique qui évite ainsi
  l'explosion combinatoire inhérente à ce type d'analyse,
  permettant de répondre en quelques dixièmes de secondes au prix d'être parfois non conclusive.
  Enfin, nous traçons des liens formels entre
  les différents formalismes développés dans cette
  thèse, ainsi qu'avec plusieurs autres modélisation discrètes répandues.
  Nous permettons ainsi à un modèle de jouir des capacités de représentation et d'analyse
  de plusieurs formalismes à la fois.
\end{resume}

\begin{motscles}
  réseaux de régulation biologique,
  frappes de processus,
  modèle de Thomas,
  formalisme asynchrone,
  analyse statique,
  atteignabilité
\end{motscles}

\begin{abstract}
  %\footnotesize
  \scriptsize
  Representing and analyzing large biological regulatory networks
  are the two main challenges of the understanding of the living machinery.
  The work we expose here focuses on discrete modeling,
  usually composed of graphs and sets of parameters,
  and especially on a previously developed framework called Process Hitting,
  which allows to give an atomistic representation of some components
  and their combined dynamics.
  In this thesis, we propose several new frameworks that consist of alternatives
  to the Process Hitting.
  Their higher expressivity permits to integrate discrete constraints into models
  based on the knowledge of reaction durations or synchronicity relationships.
  We also propose a new method to analyze the dynamics of such models by abstract interpretation
  which allows to answer reachability questions
  and is well-suited to large-scale models
  (made of hundreds of components, and potentially more).
  This method relies on an approximation of the dynamics that avoids
  the combinatorial explosion usually inherent to such analyses,
  thus answering in in tenths of a second
  at the price of being sometimes inconclusive.
  At last, we discuss the formal bonds between the different formalisms developed in this thesis
  and the link with some other widespread discrete modelings.
  We propose several translations from and to these other modelings,
  in order to benefit from the high power of modeling and analysis
  of these different frameworks.
\end{abstract}

\begin{keywords}
  biological regulatory networks,
  process hitting,
  Thomas modeling
  asynchronous framework,
  static analysis,
  reachability
\end{keywords}



\maketitle



% Collaborations
% Collaborations

\section{Collaborations}
\seclabel{intro-collaborations}

Une partie du travail de cette thèse a été réalisé dans le cadre d'un stage doctoral
dans l'équipe de recherche de Katsumi Inoue,
au National Institute of Informatics (Tokyo, Japon).
Le sujet de ce stage était :
«~Raisonnement automatique et recherche d'hypothèses pour la biologie des systèmes.~»
Ont ainsi participé financièrement à ce travail le National Institute of Informatics,
\textit{via} l'Inoue Laboratory,
ainsi que la Fondation Centrale Initiatives.

Ce travail s'inscrit par ailleurs dans le projet de recherche ANR blanc BioTempo%
\footnote{Le site du projet est disponible à \url{http://biotempo.genouest.org/}}
dont l'intitulé était : «~Représentations à l'aide de langage, de temps et de modèles hybrides
pour l'analyse de modèles incomplets en biologie moléculaire~»,
et qui s'est étendu de mars 2011 à août 2014.
Le présent travail répond notamment à certains objectifs de la tâche 3 du projet,
qui concerne l'introduction de synchronisations et de données chronométriques
dans les modèles chronologiques.




\setcounter{tocdepth}{2}
\setcounter{secnumdepth}{3}
\tableofcontents



% Introduction
% Introduction

\chapter{Introduction}

\section{Contexte \& Motivations}

Le principal défi posé par l'étude des systèmes dynamiques réels,
qu'ils soient biologiques ou non,
repose dans la modélisation qui en est faite.
Un modèle permet d'abstraire les comportements du système
pour s'intéresser uniquement à ceux qui présentent un intérêt à l'étude,
tout en permettant leur analyse à l'aide d'outils préalablement développés.

Un modèle doit donc dans l'idéal :
\begin{itemize}
  \item être cohérent avec la réalité du système qu'il représente,
  \item reproduire les comportements présentant un intérêt
    et abstraire ceux qui surchargent inutilement le modèle,
%   \item abstraire les comportements qui rendent la représentation et l'analyse inutilement
%     complexe,
  \item faciliter la lecture pour le modélisateur,
  \item permettre l'analyse par des outils appropriés,
  \item permettre la traduction depuis ou vers d'autres formalismes
    et favoriser l'ouverture à d'autres méthodes d'analyse.
\end{itemize}

Nous nous intéressons dans cette thèse aux propriétés dynamiques d'un modèle.
Elles se distinguent des propriétés statiques qui permettent de caractériser le modèle
en fonction de sa taille,
des liens entre les éléments qui le composent,
ou de toute autre propriété portant sur sa structure ou ses attributs,
bien que celles-ci apportent parfois aussi des résultats très généraux concernant la dynamique.
À l'inverse, les propriétés dynamiques portent sur l'évolution et les comportements
possibles d'un modèle, par exemple :
\begin{itemize}
  \item Étant donné un certain état des entrées du système, le modèle est-il capable
    d'en reproduire ou d'en prédire les sorties ?
  \item Y retrouve-t-on des comportements qui s'apparentent à des oscillations
    ou des états stables ?
  \item Les comportements recherchés sont-ils accessibles depuis tous les états ?
    Si non, depuis lesquels ?
  \item Peut-on modifier le modèle pour voir apparaître un comportement donné, et comment ?
\end{itemize}
Répondre à ces questions nécessite une analyse détaillée de la dynamique.

Ainsi, l'utilisation d'un modèle pose un double défi :
sa \bemph{conception} et son \bemph{analyse}.

Il est nécessaire de proposer des outils permettant une modélisation cohérente et juste.
C'est pourquoi nous proposons dans cette thèse plusieurs formalismes nouveaux permettant
des représentations efficaces et complémentaires des systèmes dynamiques étudiés.
L'une des pistes d'enrichissement consiste
en l'introduction de contraintes dynamiques afin de filtrer les comportements non désirés,
par exemple sous la forme 
de relations de prévalence entre les différentes évolutions possibles d'un modèle.
Cela ajoute de plus la possibilité de contraindre les comportements du modèle
en fonction de paramètres temporels issus
par exemple de données relatives aux durées de réaction ou de sensibilisation,
à des mesures de retards entre phénomènes biologiques, etc.
Enfin, l'ajout d'outils de synchronisation à des formalismes purement asynchrones
peut s'avérer nécessaire pour représenter certains comportements simultanés.

% Des outils peuvent être proposés pour affiner la dynamique complète du modèle,
% afin d'éliminer des comportements non désirés.
% L'objectif est donc de pouvoir exprimer des contraintes supplémentaires
% qui ne l'étaient pas auparavant.
% Ainsi, il est possible par exemple
% d'élaborer un système décrivant une relation de prévalence entre les
% différentes évolutions d'un modèle.
% De cette manière, on peut parvenir à intégrer des notions temporelles dans les modèles,
% en se basant sur des données chronométriques.
% Il est aussi possible de pallier certains manques des modèles,
% par exemple en ajoutant des outils de synchronisme à des formalismes purement asynchrones.

Un autre problème récurrent est de parvenir à faire le lien entre plusieurs formalismes.
En effet, des formalismes différents peuvent permettre des approches variées,
et l'application d'outils adaptés à des problèmes précis.
Il s'avère donc nécessaire de pouvoir faire le lien entre différentes versions
d'un formalisme, ou encore de créer des ponts vers d'autres formalismes répandus
dans le cadre des systèmes dynamiques.
Aussi, cette thèse offre une part importante à l'\bemph{étude des relations entre différentes
représentations} complémentaires des modèles dynamiques étudiés,
ainsi qu'avec les modèles classiques pour ce type de représentations.
L'intérêt d'une telle comparaison est de pouvoir mettre en valeur et d'exploiter
les avantages de chacune de ces approches.

Cependant, vérifier de telles propriétés dynamiques présente un coût
en termes de temps d'exécution et de taille mémoire,
lié à la taille du modèle considéré.
Et si, dans les meilleurs cas, ce coût varie de façon polynomiale
en fonction de la taille du modèle,
la plupart des méthodes formelles permettant de vérifier des propriétés dynamiques
font malheureusement face à une explosion combinatoire
qui empêche l'analyse de modèles de grande taille.
C'est pourquoi des méthodes alternatives peuvent être explorées,
comme la vérification par interprétation abstraite,
qui consiste à approcher de façon plus ou moins fine
la dynamique pour diminuer la complexité et simplifier les calculs.
Comme nous le verrons dans la suite de ce manuscrit,
il s'agit d'une des facettes de notre travail,
et nous soutenons que celle-ci constitue une contribution importante
pour l'étude des systèmes considérés.

% ce qui permet d'analyser des modèles de très grande taille avec des ressources raisonnables.
% Cependant, il arrive que ces analyses soient de complexité exponentielle, voire plus,
% en fonction de la taille de modèle.
% Dans ce cas, des modèles de petite taille peuvent être analysés à condition d'avoir
% les capacités de calcul et de stockage nécessaires,
% mais la complexité croît trop vite avec la taille du modèle pour que
% des réseaux de grande taille puissent être abordés par ces méthodes.

%\todo{+ rapide retour sur les contributions}

La \secref{intro-rrb} propose une rapide vue d'ensemble des types de modèles permettant
la représentation et l'étude des systèmes d'interactions,
afin de mieux situer le travail proposé dans cette thèse,
dont les principaux résultats sont ensuite présentés à la \secref{intro-contrib}.
La \secref{intro-plan} présente la façon dont ce manuscrit est organisé,
et la \secref{notations} introduit les différentes notations qui y sont utilisées.



\section{Les réseaux de régulation biologique}
\seclabel{intro-rrb}

%\stodo{[Thomas'73], [Paulevé'11 et '12]}

\tocite[Citer Fages \& Soliman, 2008a ?]

\todoplustard{Ajouter des figures ?
\begin{itemize}
  \item Graphique figurant l'abstraction continu $\rightarrow$ discret
  \item Exemple de graphe des interactions
  \item Exemple de dynamique
  \item ... ?
\end{itemize}
OU :\\
Présenter comment le travail s'insère dans le domaine par rapport à d'autres formalismes
concernant :
\begin{itemize}
  \item les aspects temporels,
  \item l'abstraction,
  \item les grands systèmes,
  \item ...
\end{itemize}
}

L'étude de la machine cellulaire nécessite de s'intéresser aux éléments interagissant
qui la composent :
%des interactions entre les différents composants de la machine cellulaire
gènes, protéines, ARN messager, métabolites, etc.
% 
% Plusieurs approches permettent de représenter et d'analyser le fonctionnement de la machine
% cellulaire.
% L'une d'entre elles consiste à étudier les interactions qui y ont lieu
% entre les différents éléments qui s'y trouvent :
% gènes, protéines, ARN messager, métabolites, etc.
À ce niveau, il est déjà possible d'abstraire dans une certaine mesure
une partie des composants.
%les différents composants peuvent être plus ou moins abstraits.
C'est ainsi qu'un gène, la protéine qu'il code et l'ARN messager correspondant sont
souvent tous trois modélisés par un unique élément,
car la concentration de la protéine et de l'ARN
dépendent directement du niveau d'activation du gène.
Si cette simplification paraît poussée face à la réalité biologique qu'elle représente,
elle permet néanmoins d'abstraire de façon cohérente le mécanisme de création d'une protéine
pour s'intéresser aux relations
entre la présence d'une protéine et son influence sur la production d'un autre.
Comme ce cas de figure est particulièrement répandu dans l'étude des réseaux de régulation,
%Dans la suite nous nous intéresserons particulièrement à ce cas de figure,
nous assimilerons souvent dans la suite une protéine avec son gène codant.
%bien que d'autres peuvent être représentés de la même manière.

Cette simplification permet de mettre en évidence les phénomènes d'interaction entre
les différents éléments entrant en jeu.
En effet, la présence d'un composant (protéine, catalyseur…)
en quantité suffisante peut déclencher
l'activation (une hausse de l'activité) ou l'inhibition (une baisse de l'activité)
d'un ou plusieurs autres éléments, y compris l'élément déclencheur lui-même.
C'est par ce mécanisme que se crée une cascade de réactions entre gènes :
%C'est par ce mécanisme qu'un gène, selon son degré d'activité,
en effet, l'un d'eux, selon son degré d'activité,
permettra la production d'une concentration plus ou moins importante de la protéine qu'il code ;
celle-ci aura alors éventuellement un rôle activateur ou inhibiteur sur un certain
nombre d'autres gènes, et ainsi de suite.
L'ensemble de ces régulations peut être représenté par un graphe des interactions,
où les composants sont modélisés par des nœuds et leurs interactions mutuelles par des arcs.

% À ce niveau, les différents composants peuvent être plus ou moins abstraits.
% Par exemple, un gène, la protéine qu'il code et l'ARN messager correspondant sont
% souvent modélisés par un unique nœud,
% car la concentration de la protéine dépend directement du niveau d'activation du gène.
% Dans la suite nous nous intéresserons particulièrement à ce cas de figure,
% bien que d'autres peuvent être représentés de la même manière.

Afin de représenter la dynamique du modèle,
une valeur abstraite est associée à chaque élément afin de modéliser
son état courant
(niveau d'activité pour un gène,
concentration dans le milieu pour une protéine, etc.).
% À chacun des éléments de modèle est alors associée une valeur modélisant
% son niveau d'activité ou
% sa concentration dans le milieu (pour une protéine).
Dans le cadre de formalismes utilisant des équations différentielles
il s'agit d'une valeur continue,
ce qui permet par exemple de lier entre elles
les concentrations des différentes protéines et leurs dérivées par rapport au temps,
qui représentent alors leurs vitesses d'évolution \cite{tyson1978dynamics}.
Cependant, le manque de données expérimentales précises et fiables peut limiter de telles
approches.
De plus, la résolution analytique ou numérique des équations différentielles est parfois
très complexe, quand elle n'est pas impossible.

Une autre approche consiste donc à symboliser l'activité d'un élément par
une \bemph{valeur discrète} au sein d'un ensemble fini.
Elle se justifie par le fait que la courbe d'évolution de la concentration
d'une protéine forme généralement une sigmoïde lorsque sa production augmente ou diminue.
Cela avait été théorisé notamment par \citefullname{kauffman69}{Stuart A.}
puis par \citefullname{Thomas73}{René}
dans le cadre de formalismes booléens,
c'est-à-dire restreints à deux niveaux discrets par composant, généralement notés
«~0~» et «~1~».
Cette approche a naturellement été étendue par la suite à des formalismes multivalués,
où chaque élément peut posséder plus de deux niveaux d'expression,
généralement représentés par des entiers consécutifs.
Cette approche a l'avantage d'abstraire les valeurs des seuils de concentration,
qui sont généralement mal connues
mais qui permettent de représenter le niveau de concentration
à partir duquel la protéine va influencer un autre composant.
% quel niveau de concentration chaque composant
% possède une influence sur un autre.
Ainsi, à chaque niveau d'expression d'un composant est associé un ensemble de régulations
sur d'autres composants.

Le dernier élément permettant de caractériser la dynamique est l'ajout d'une
\bemph{dimension temporelle}.
Dans le cadre des réseaux de régulation discrets, ce temps prend la forme d'une série
infinie de pas de temps discrets permettant de représenter les évolutions
successives du modèle au cours du temps, sans indication des durées réelles
entre ces pas de temps.
La question du synchronisme des formalismes développés se pose alors :
en effet, s'il existe des systèmes purement synchrones, où entre chaque pas de temps,
toutes les réactions sensibilisées ont lieu,
l'hypothèse de \citefullname{Thomas73}{René}
est au contraire d'abstraire les systèmes biologiques à l'aide
de réseaux de régulation purement \bemph{asynchrones}.
En effet, en l'absence de données temporelles sur le système étudié
(vitesse des réactions, durée des dégradations...)
il n'est pas possible d'assurer que deux protéines dont la production est activée au même moment
seront produites en mêmes concentrations simultanément,
ou que leurs concentrations passeront en même temps
un seuil d'expression (représenté par un niveau discret).

C'est à partir de ces hypothèses
ayant pour but la représentation de réseaux de régulation biologique discrets et asynchrones
%permettant de représenter une abstraction des composants
%entrant en jeu à l'aide de niveau discrets
qu'a été formalisée
la version actuelle du modèle de Thomas \cite{Richard06},
et c'est en s'inspirant de celle-ci qu'ont été conçues par la suite
les Frappes de Processus \cite{PMR10-TCSB}.
%qui permettent toutes deux
Ces deux formalismes permettent en effet
de représenter les interactions entre différents composants
sous la forme de l'évolution séquentielle de niveaux d'expression discrets.
Cependant, ils se distinguent principalement au niveau de la représentation
des interactions entre composants.
En effet, le modèle de Thomas considère les interactions entre composants du point
de vue d'«~influences~» globales, c'est-à-dire d'interactions qui vont globalement avoir
le rôle d'activer ou inhiber les composants entre eux,
tandis que les Frappes de Processus font usage d'«~actions~»,
qui décrivent le saut d'un état local d'un composant à un autre.
De plus, une restriction particulière porte sur ces actions,
qui ne permettent la modification du niveau local d'un composant
que par au plus un autre composant.
Cette différence de point de vue en implique une autre en ce qui concerne la représentation
des coopérations entre composants :
là où le modèle de Thomas fait usage de paramètres décrivant
les états focaux d'un élément en fonction de l'activité de l'ensemble de ses régulateurs
\cite{Snoussi89},
les Frappes de Processus introduisent un composant supplémentaire propre à la modélisation
et qui joue le rôle de porte logique.

Les Frappes de Processus ont fait l'objet de travaux précédents.
Ceux-ci ont notamment permis de montrer qu'elles permettent de représenter une «~superposition~»
de modèles de Thomas, afin de représenter des ensembles de paramètres partiellement connus.
De plus, un travail approfondi a permis d'élaborer de puissantes analyses statiques
portant sur la recherche d'états stables,
mais aussi sur des questions dynamiques comme l'atteignabilité d'un état local \cite{PMR12-MSCS}
qui peut être traitée de façon très efficace, et donc \bemph{appliquée à de très grands modèles}.

\todo{Mentionner et se positionner par rapport au travail de Vincent Danos ($\kappa$ calcul).}

\todo{Se positionner par rapport aux réseaux de Petri (cf. Heiner, Gilbert, Chaouiya).}



\section{Contributions}
\seclabel{intro-contrib}

Les apports de cette thèse se déclinent en trois points principaux :
\begin{itemize}
  \item l'enrichissement des Frappes de Processus par l'introduction de notions
    de prévalence et de synchronisme entre les différentes évolutions possibles ;
  \item le développement de méthodes d'analyse de la dynamique efficaces
    et adaptées aux enrichissements précédemment mentionnés,
    et leur application concluante à des réseaux de régulation biologique de grande taille ;
  \item la description des liens formels entre les différentes sémantiques de Frappes de Processus
    proposées, ainsi qu'avec d'autres formalismes courants pour la représentation
    des réseaux de régulation biologique.
\end{itemize}
L'objectif principal de cette thèse est donc de répondre à la double problématique
de la conception et de l'analyse d'un modèle,
en nous concentrant particulièrement sur l'utilisation et l'amélioration
des Frappes de Processus des outils associés.
Les trois aspects précédents sont détaillés dans la suite
afin d'insister sur leur rôle dans cette démarche.

\subsubsection*{Enrichissement des Frappes de Processus}

Cette thèse se concentre sur plusieurs \bemph{alternatives d'extension du formalisme des
Frappes de Processus dans le but d'enrichir celles-ci}.
L'un des objectifs de cet enrichissement est \emph{la prise en compte de données temporelles},
qui étaient généralement abstraites auparavant,
comme des durées relatives de réactions biochimiques
ou des notions de retard entre les évolutions de certains éléments.
afin d'étudier leur influence sur la dynamique.
Cependant, plutôt que d'intégrer des données continues dans le modèle,
nous proposons de traduire ces données par des formes nouvelles de dynamique.
Pour cela, nous proposons trois approches qui gravitent autour des notions
de \bemph{préemption} et de \bemph{synchronisme} entre les différentes
évolutions possibles du modèle :
\begin{itemize}
  \item les \bemph{classes de priorités} permettent de définir des règles globales de prévalence
    entre des ensembles d'actions, afin d'affiner la dynamique et d'obtenir une expressivité
    équivalente à celle des réseaux booléens,
  \item les \bemph{arcs neutralisants} proposent de raffiner la notion précédente,
    en définissant les prévalences de façon plus atomique entre les actions individuelles,
  \item les \bemph{actions plurielles} permettent enfin d'introduire des comportements
    synchrones entre les actions, dans le but de modéliser des phénomènes simultanés
    comme la création des produits d'une réaction.
\end{itemize}
Nous montrons par ailleurs comment les différentes données temporelles peuvent être
intégrées à l'aide de ces nouveaux outils.

\subsubsection*{Outils efficaces d'analyse de la dynamique}

Nous développons par ailleurs des méthodes permettant d'analyser la dynamique
des modélisations proposées ci-dessus.
Ces méthodes sont basées sur de l'analyse statique par interprétation abstraite
précédemment proposée par \citeasnoun{PMR12-MSCS} :
%c'est-à-dire permettant d'approcher la dynamique afin de réduire la complexité de l'analyse.
leur fonctionnement repose sur l'abstraction de la dynamique globale du système
au profit des dynamiques locales de chaque composant.
Nous enrichissons cette approche afin d'adapter les méthodes développées
aux nouvelles formes de dynamique que nous proposons.
Elles ont l'avantage de posséder \bemph{une complexité polynomiale en la taille du modèle},
ce qui permet de \bemph{traiter efficacement de très grands modèles,
de l'ordre de centaines de composants, en quelques dixièmes de seconde}.
Néanmoins, étant donné qu'elles sont basées sur une approximation de la dynamique,
il est possible qu'elles terminent sans pouvoir conclure,
bien que cela ne soit jamais arrivé sur les exemples testés.
%notamment ceux présentés dans cette thèse.
Nous appliquons notamment ces méthodes à un réseau booléen de quatre-vingt-quatorze (94) composants
préalablement traduit en Frappes de Processus,
et nous montrons que cela permet de conclure
en moins d'une seconde sur plusieurs questions dynamiques.

\subsubsection*{Équivalences entre les Frappes de Processus}

L'analyse statique mentionnée précédemment nécessite l'utilisation
de \bemph{Frappes de Processus canoniques},
une classe particulière
consistant en une restriction de l'un des formalismes mentionnés précédemment.
Celle classe présente d'autres intérêts :
nous montrons notamment qu'elle est
\bemph{aussi expressive que tous les formalismes
de Frappes de Processus} développés dans cette thèse,
et nous donnons les traductions correspondantes.
Cela assure notamment qu'il est toujours possible de naviguer entre les formalismes,
afin de profiter de leurs avantages respectifs,
et notamment de l'utilisation des analyses statiques mentionnées ci-dessus
à tous les autres types de Frappes de Processus, moyennant une traduction du modèle.

\subsubsection*{Équivalences avec d'autres formalismes}

Enfin, nous nous intéressons aux \bemph{liens formels entre les différentes
sémantiques de Frappes de Processus
et les autres modélisations classiques pour la représentation des réseaux de régulation
biologique}.
Nous nous intéressons notamment au \bemph{modèle de Thomas} et aux \bemph{réseaux booléens},
deux formalismes proches
développés spécifiquement pour la représentation de réseaux de régulations,
et nous montrons qu'il est possible de représenter ces deux formalismes de façon exacte
grâce à l'ajout de priorités dans les Frappes de Processus.
Nous abordons aussi la question de la traduction vers les \bemph{réseaux de Petri},
un formalisme plus généraliste mais objet d'un intérêt important du fait des
capacités d'expressivité et d'analyse qu'il offre,
et les intérêts qu'il présente aussi pour la modélisation des réseaux biologiques
\cite{petri1962kommunikation,Chaouiya07petrinet}.
\tocite[Heiner, Gilbert, Chaouiya]
Ces liens permettent d'une part de comprendre la position des Frappes de Processus au sein
de l'ensemble des modélisations discrètes asynchrones,
et offrent d'autre part des outils de traduction depuis et vers ces autres formalismes,
donnant la possibilité d'utiliser les outils d'analyse spécifiquement développés pour ceux-ci.



\section{Organisation du manuscrit}
\seclabel{intro-plan}

Le présent manuscrit est organisé de la manière suivante.

Le \chapref{etatdelart} offre un état des lieux en matière de modélisation et d'analyse
des réseaux de régulation biologique discrets et asynchrones.
Nous nous intéresserons notamment aux réseaux booléens et au modèle de Thomas,
deux types de réseaux très répandus pour des raisons historiques
comme nous l'avons déjà mentionné.
Nous définirons aussi les Frappes de Processus standards,
c'est-à-dire telles que définies par le travail de \citefullname{Pauleve11}{Loïc},
car elles constituent le point de départ de ce travail.

Le \chapref{semantiques} propose plusieurs ajouts aux Frappes de Processus afin de l'enrichir
et d'en augmenter l'expressivité.
Les trois formalismes abordés reposent sur les notions de classes de priorités,
d'arcs neutralisants et d'actions plurielles,
afin de filtrer les comportements désirés du modèle
ou d'ajouter des comportements nécessaires à la modélisation de certains phénomènes.
Nous discutons aussi dans ce chapitre des applications possibles de ces formalismes,
et traitons en partie la question des équivalences et des traductions entre ces formalismes.

Le \chapref{as} se concentre sur une classe particulière de Frappes de Processus avec classes
de priorités, dite «~canonique~».
Cette classe sert de base au développement de méthodes d'analyse statique permettant
de vérifier des propriétés d'atteignabilité locales dans un modèle.
Ces méthodes utilisent un mécanisme d'abstraction pour éviter l'explosion combinatoire
inhérente à toute étude de la dynamique d'un modèle discret, et ainsi rester efficaces.
Nous montrons aussi, traduction à l'appui,
que les Frappes de Processus canoniques sont en réalité équivalentes
aux trois extensions des Frappes de Processus proposées,
ce qui permet d'étendre la portée des résultats précédents à ces formalismes.

Le \chapref{expressivite} dépasse le cadre des Frappes de Processus et s'intéresse aux liens
formels entre les extensions proposées plus haut et plusieurs formalismes classiques discrets
permettant la modélisation des réseaux de régulation biologique.
Nous y prouvons que les extensions des Frappes de Processus permettent d'obtenir la même
expressivité que le modèle de Thomas et les réseaux booléens,
ce qui étend encore la portée de nos résultats.
Nous montrons aussi que les Frappes de Processus sont aussi expressives que le
formalisme classique des automates synchronisés.
Enfin, nous proposons une traduction vers les réseaux de Petri
avec arcs de lecture et arcs inhibiteurs,
et, à l'inverse, une traduction depuis le systèmes d'équations biochimiques
de la sémantique booléenne de Biocham.

Le \chapref{applications} porte sur l'application des modèles et méthodes proposées
dans cette thèse à des réseaux de régulation biologique de grande taille.
Nous y présentons plusieurs systèmes biologiques classiquement étudiés,
la façon dont ils sont représentés en Frappes de Processus,
et les résultats de nos méthodes de traduction et d'analyse.

Enfin, le \chapref{conclusion} nous permet de conclure ce manuscrit
et d'en discuter les résultats afin d'ouvrir sur des perspectives de travaux futurs.

\section{Notations}

On notre $\sN$ l'ensemble des entiers naturels.
Si $i, j \in \sN$, $i < j$, on note $\segm{i}{j} = \{ i, i+1, \ldots, j-1, j \}$
l'ensemble des entiers naturels entre $i$ et $j$ compris.
On note $\emptyseq$ la séquence vide.

L'ensemble des parties d'un ensemble $A$ est noté : $\powerset(A)$.
Si $A$ et $B$ sont deux ensembles, on note
$A \cup B$ leur union, $A \cap B$ leur intersection et $A \times B$ leur produit cartésien.
Si $n \in \sN$, et $(A_i)_{i \in \segm{1}{n}}$ est une séquence d'ensembles, on note
$\bigcup_{i \in \segm{1}{n}} A_i = A_1 \cup A_2 \cup \ldots \cup A_n$ leur union,
$\bigcap_{i \in \segm{1}{n}} A_i = A_1 \cap A_2 \cap \ldots \cap A_n$ leur intersection,
$\bigtimes{i \in \segm{1}{n}} A_i = A_1 \times A_2 \times \ldots \times A_n$ leur produit cartésien.
De plus, par convention :
$\bigcup_{\emptyset} = \bigcap_{\emptyset} = \bigtimes{\emptyset} = \emptyset$.

\todo{indices}



% État de l'art de la modélisation}
% État de l'art de la modélisation

\chapter{État de l'art de la modélisation}
\chaplabel{etatdelart}

\chapeau{%
  Nous proposons dans ce chapitre un état de l'art des formalisations discrètes
  et asynchrones des réseaux de régulation biologique utilisées dans cette thèse.
  Nous y rappelons le principe et la définition du \emph{modèle de Thomas}
  et nous revenons brièvement sur les méthodes d'analyse qui existent sur ce modèle.
  Nous rappelons aussi le formalisme des \emph{Frappes de Processus standards},
  qui seront la base des travaux présentés dans cette thèse,
  et nous faisons une rapide revue des principaux travaux qui ont concerné ce formalisme.
}



Les outils de modélisation des réseaux de régulation biologique se déclinent en de nombreuses
formes permettant de représenter différents comportement des systèmes étudiés.
Au cours de cette thèse, nous nous intéressons tout particulièrement aux modèles discrets
et asynchrones.

Les modèles discrets permettent de représenter des systèmes plus complexes,
comme des systèmes d'équations différentielles,
en abstrayant une partie de la dynamique.
Cette abstraction permet de simplifier le modèle pour en faciliter l'analyse,
à condition de rester cohérente avec la représentation initiale.
Le caractère asynchrone des formalismes vient quant à lui de la constatation suivante :
il est biologiquement très improbable que plusieurs entités d'un système
évoluent en parfaite simultanéité.
Le parallèle avec les systèmes d'équations différentielles est le suivant :
il est rare d'observer plusieurs composants passer un seuil simultanément
au cours d'une évolution continue de leur état.
Ces hypothèses ont notamment été théorisées par \citefullname{Thomas73}{René},
qui en a dérivé le modèle qui porte son nom,
plus tard enrichi par plusieurs travaux successifs
comme l'ajout de paramètres discrets par \citefullname{Snoussi89}{El Houssine}
pour représenter les «~états focaux~» des différents
composants en fonction de l'état du modèle.

Dans ce chapitre, nous rappellerons tout d'abord la définition du modèle de Thomas
à la \secref{thomas}.
Nous dresserons par ailleurs un rapide état de l'art des méthodes d'analyse
de la dynamique de ce modèle qui, bien que facilitées par
l'abstraction discrète et asynchrone qui en est inhérente,
restent l'objet d'une explosion combinatoire non négligeable.
Plusieurs travaux apportent des résultats qui s'appuient uniquement sur les données
du modèles et non sur le calcul de sa dynamique,
ce qui évite de calculer celle-ci explicitement.
Cependant, ces résultats sont très généraux
car ils se focalisent le plus souvent sur la présence ou l'absence de comportements
dans le modèle (comme des oscillations ou des points fixes).
Il est donc rare de pouvoir totalement faire l'impasse
sur une analyse plus détaillée de la dynamique,
qui nécessite cependant de calculer l'espace des états du modèle,
ou une version compressée de celui-ci.

Nous définirons ensuite le formalisme des Frappes de Processus
à la \secref{ph}, tel qu'il a été proposé par \citeasnoun{PMR10-TCSB}.
Il a été conçu comme une alternative complémentaire au modèle de Thomas
avec lequel il partage certaines hypothèses, à savoir la discrétisation des états
et l'asynchronisme de la dynamique,
%basé sur certaines hypothèse les mêmes hypothèses d'états discrets et d'asynchronisme,
bien qu'il soit plus atomique par ailleurs dans sa représentation.
Il permet notamment de modéliser tout modèle de Thomas,
avec cependant une légère sur-approximation de la dynamique.
Bien que cette sur-approximation permette de représenter une classe de modèles de façon
abstraite, elle est cependant indésirable lorsque l'on souhaite modéliser
fidèlement certains comportements.
Des outils ont de surcroît été développés afin de calculer les points fixes d'un modèle de
Frappes de Processus ou encore d'y intégrer des paramètres continus
sous forme de probabilités.
Cependant, l'atout principal des Frappes de Processus réside dans les puissantes
méthodes d'analyse statique par interprétation abstraite qui y sont associées,
permettant ainsi d'effectuer des calculs d'atteignabilité locale
avec une complexité polynomiale dans la taille du modèle considéré.
Cela permet notamment d'étudier la dynamique de grands modèles
---~jusqu'à plusieurs centaines de composants, voire au-delà.

\todoplustard{Processus de modélisation des biologistes :
\begin{itemize}
  \item types de données en entrée,
  \item type de questions pour lesquelles ils attendent des résultats,
  \item repositionnement par rapport à cela.
\end{itemize}}



% Le modèle de Thomas
% Le modèle de Thomas

\section{Le Modèle de Thomas}
\seclabel{thomas}

Nous présentons dans cette section le \emph{modèle de Thomas} (\secref{thomas-def}),
historiquement très utilisé dans la représentation des réseaux de régulation biologique.
Il s'agit d'un modèle asynchrone basé sur un \emph{graphe des interactions} et
une carte de \emph{paramètres} discrets.
Les \emph{réseaux discrets asynchrones} sont aussi définis (\secref{rda-def})
du fait de leurs similitudes avec le modèle de Thomas.
Nous proposons enfin un rapide tour d'horizon des différentes méthodes utilisées
pour analyser la dynamique de ces modèles (\secref{thomas-analyse}).

\subsection{Définition du modèle de Thomas}
\seclabel{thomas-def}

Le modèle dit \emph{de Thomas} a été théorisé et proposé pour la première
fois par \citefullname{Thomas73}{René} dans le but d'abstraire les systèmes
d'équations différentielles permettant la représentation de réseaux de régulation biologique.
Son intuition était de proposer une simplification discrète cohérente de ces systèmes
afin de permettre des analyses plus efficaces.
Nous proposons ici une définition inspirée de divers travaux plus récents,
comportant notamment des paramètres discrets \cite{Snoussi89}
%des composants multivalués \cite{Richard06,BernotSemBRN}
et des arcs non-signés \cite{FPIMR12-CMSB}.

\myskip

% Paramètres de Snoussi : \cite{Snoussi89}
% SMBionet : \cite{Richard06}
% Hypothèses d'activité et monotonicité : \cite{BernotSemBRN}

Un modèle de Thomas représente un ensemble fini de composant se régulant entre eux.
Ces composants sont décrits par un niveau d'expression discret qui
caractérise leur état (taux d'activité pour un gène, concentration pour une protéine, etc.).
Afin de représenter la structure d'un tel système,
on utilise généralement un \emph{graphe des interactions} (\defref{thomas-gi})
dont les nœuds représentent un ensemble de composants
et les arcs (orientés) leurs influences mutuelles.
Les nœuds sont étiquetés par un nom (celui du composant : $a$, $b$, $c$, etc.)
et un plafond (son niveau d'expression maximum : $l_a$, $l_b$, $l_c$, etc.).
Les arcs sont de la forme : $\arc{a}{s}{t}{b}$,
c'est-à-dire étiquetés par un signe $s$ qui représente le type de régulation
($+$ pour une activation, $-$ pour une inhibition
et $\uns$ pour une régulation plus complexe)
et un entier $t$ qui représente le seuil de déclenchement de la réaction
(c'est-à-dire le niveau d'expression du composant régulateur à partir duquel celui-ci
a effectivement une influence sur le composant régulé).
Bien que plusieurs variantes aient été proposées,
nous nous intéressons ici à une définition où
chaque arc est unique, c'est-à-dire qu'il ne peut pas exister deux arcs
$\arc{a}{s}{t}{b}$ et $\arc{a}{s'}{t'}{b}$ étiquetés différemment dans le graphe.
L'utilité des arcs non-signés ($\uns$) est discutée à la fin de cette section.

\begin{definition}[Graphe des interactions]
\deflabel{thomas-gi}
  Un \emph{graphe des interactions} est un couple $\GI = (\components; E)$ où
  $\components$ est l'ensemble fini des \emph{composants},
  étiquetés par un nom et un \emph{plafond},
  et $E$ est l'ensemble fini des \emph{régulations} entre deux nœuds,
  étiquetées par un \emph{signe} et un \emph{seuil} :
    \[E \DEF \{ \arc{a}{s}{t}{b}, \ldots \mid
      a, b \in \components \wedge s \in \{ +, -, \uns \} \wedge t \in \segm{1}{l_a}\}\]
  tel que chaque régulation de $a$ vers $b$ soit unique :
    \[\forall \arc{a}{s}{t}{b} \in E,
      \forall \arc{a}{s'}{t'}{b} \in E, s = s' \wedge t = t' \enspace.\]
\end{definition}
%
Étant donnée cette définition, on note
$E_s \DEF \{ \arc{a}{s}{t}{b} \in E \}$ pour $s \in \{ +, -, \uns \}$.
De plus, pour tout composant $b \in \components$, on note $\RRBreg{b}$ l'ensemble de ses
\emph{régulateurs} :
    \[\RRBreg{b} \DEF \{ a \in \components \mid \exists \arc{a}{s}{t}{b} \in E \}\]
\label{regulateurs}

\begin{example}
  La \figref{thomas}(gauche) représente un graphe des interactions $(\components; E)$ où
  $\components = \{a, b, c\}$, avec $l_a = 2$ et $l_b = l_c = 1$, et :
  \begin{align*}
    E_+ &= \{\arc{b}{+}{1}{a}, \arc{c}{+}{1}{a}\} &
    E_\uns &= \emptyset \\
    E_- &= \{\arc{a}{-}{2}{b}\}
  \end{align*}
  Ainsi :
  \begin{align*}
    \RRBreg{a} &= \{ b, c \} &
    \RRBreg{b} &= \{ a \} \\
    \RRBreg{c} &= \emptyset
  \end{align*}
  Pour des raisons d'illustration, le composant $a$ ne comporte aucun arc sortant avec le seuil
  $1$, mais possède un arc sortant étiqueté avec le seuil $2$.
  Ce type de configuration ne se rencontre habituellement pas dans les modèles de Thomas
  pour des raisons discutées \vpageref{plafond}.
  
  \begin{figure}[ht]
    \begin{minipage}{0.49\textwidth}
    \centering
    \scalebox{1.2}{
    \begin{tikzpicture}[grn]
      \path[use as bounding box] (-0.3,-0.75) rectangle (4,.75);
      \node[inner sep=0] (a) at (2,0) {a};
      \node[inner sep=0] (b) at (0,0) {b};
      \node[inner sep=0] (c) at (3.8,0) {c};
      
      \node[elabel, below=-.8em of a] {$0..2$};
      \node[elabel, below=-.8em of b] {$0..1$};
      \node[elabel, below=-.8em of c] {$0..1$};
      
      \path[->]
        (b) edge[bend right] node[elabel, below=-5pt] {$+1$} (a)
        (c) edge node[elabel, above=-5pt] {$+1$} (a)
        (a) edge[bend right] node[elabel, above=-5pt] {$-2$} (b);
    \end{tikzpicture}
    }
    \end{minipage}
    \begin{minipage}{0.49\textwidth}
    \centering
    \begin{align*}
      K_{a,\emptyset} &= \segm{0}{0} & K_{b,\emptyset} &= \segm{0}{1} \\
      K_{a,\{b\}} &= \segm{1}{1} & K_{b,\{a\}} &= \segm{0}{0} \\
      K_{a,\{c\}} &= \segm{1}{1} &&\\
      K_{a,\{b,c\}} &= \segm{2}{2} & K_{c,\emptyset} &= \segm{0}{1}
    \end{align*}
    \end{minipage}
    \caption{\figlabel{thomas}%
      (gauche)
        Un exemple de graphe des interactions.
        Les composants sont représentés par les nœuds, comportant un nom et un
        ensemble de niveaux d'expression,
        tandis que les régulations sont représentées par des arcs
        étiquetés par un signe et un seuil.
        Par exemple, l'arc de $b$ vers $a$ étiqueté $+1$ représente la régulation $\arc{b}{+}{1}{a}$.
        En d'autres termes, $b$ se comporte comme un activateur de $a$ si son niveau d'expression
        est égal à $1$, et se comporte comme un inhibiteur sinon (c'est-à-dire si son niveau
        d'expression est égal à $0$).
      (droite)
        Un exemple de paramétrisation du graphe des interactions de gauche.
    }
  \end{figure}
\end{example}

Pour tout composant $a$ régulant $b$, c'est-à-dire si $\arc{a}{s}{t}{b} \in E$,
on note $\levels{a}{b}$ (\resp $\ulevels{a}{b}$) l'ensemble des niveaux d'expression
de $a$ qui sont au-dessus (\resp en-dessous) du seuil $t$ (\defref{levels}).
Au niveau de la dynamique,
pour tout niveau d'expression de $a$ appartenant à $\levels{a}{b}$, $a$ est censé avoir
une influence correspondant au signe $s$ sur $b$,
c'est-à-dire être activateur si $s = +$, inhibiteur si $s = -$,
ou avoir une influence indéterminée ou multiple si $s = \uns$ ;
en revanche, pour tout niveau d'expression de $a$ appartenant à $\ulevels{a}{b}$,
l'influence opposée devrait être observée.
Cette hypothèse permet de modéliser la dégradation de $b$ en l'absence de l'activation de $a$
si $s = +$, ou l'activation de $b$ en l'absence de l'inhibition de $a$ si $s = -$.

\begin{definition}[Niveaux effectifs ($\levelssymbol$)]
\deflabel{levels}
  Soit $\GI = (\components; E)$ un graphe des interactions.
  Si $\arc{a}{s}{t}{b} \in E$, on définit :
    \[\levels{a}{b} \DEF \segm{t}{l_a} \quad \text{et} \quad
      \ulevels{a}{b} \DEF \segm{0}{t-1}\]
\end{definition}

\begin{example}
  Sur le graphe des interactions de la \figref{thomas}(gauche)
  on a notamment :
  \begin{align*}
    \levels{a}{b} &= \segm{2}{2} & \ulevels{a}{b} &= \segm{0}{1}
  \end{align*}
\end{example}

\myskip

Un \emph{état} d'un graphe des interactions $(\components; E)$ est un élément de l'ensemble
$\RRBstates \DEF \prod_{a \in \components} \segm{0}{l_a}$.
$\RRBget{s}{a}$ se rapporte au niveau d'expression du composant $a$ dans $s$.
Pour tout état, l'ensemble des \emph{ressources} d'un composant donné est
l'ensemble de ses régulateurs dont le niveau d'expression est supérieur ou égal au seuil
de la régulation (\defref{ressources}).
En d'autres termes, pour chaque état $s$, tout régulateur $b$ d'un composant $a$
est une ressource si et seulement si $\RRBget{s}{b} \in \levels{b}{a}$
(et, inversement, n'en est pas une si $\RRBget{s}{b} \in \ulevels{b}{a}$).
Cependant, la dynamique manque de précision dans toute situation où un
composant est à la fois activé et inhibé par deux composants différents.
C'est pour supprimer ce flou que \citeasnoun{Snoussi89} propose l'ajout d'une
\emph{paramétrisation},
c'est-à-dire d'un ensemble de \emph{paramètres} discrets
qui tiennent lieu de points focaux :
à chaque configuration de ressources d'un composant est associé un paramètre
qui détermine le niveau vers lequel le composant va évoluer.
Nous proposons à la \defref{thomas-param} une extension de cette notion de paramètre
à un intervalle d'entiers afin de gagner en expressivité
\cite{FPIMR12-CMSB}.
L'intérêt des paramètres sous forme d'intervalles est discuté à la fin de cette section.

\begin{definition}[Ressources ($\RRBressymbol$)]
\deflabel{ressources}
  Soit $\GI = (\components; E)$ un graphe des interactions.
  Pour tout composant $a \in \components$ et tout état $s \in \RRBstates$,
  on appelle \emph{ressources de $a$ dans $s$} et on note $\RRBres{a}{s}$
  l'ensemble des régulateurs de $a$ dont le niveau dans $s$ est supérieur au seuil
  de la régulation qui les relie à $a$ :
    \[\RRBres{a}{s} \DEF \{ b \in \RRBreg{a} \mid \RRBget{s}{b} \in \levels{b}{a} \}\]
\end{definition}

\begin{definition}[Paramètre $K_{a, \omega}$ et paramétrisation $K$]
\deflabel{thomas-param}
  Soit $\GI = (\components; E)$ un graphe des interactions.
  Pour un composant $a \in \components$ donné
  et $\omega \subset \RRBreg{a}$ un sous-ensemble de ses régulateurs,
  le \emph{paramètre} $K_{a,\omega} = \segm{i}{j}$ est un intervalle non-vide tel que
  $0 \leq i \leq j \leq l_a$.
  La carte complète $K$ des paramètres sur un graphe des interactions $\GI$
  est appelée \emph{paramétrisation de $\GI$}.
\end{definition}

Un graphe des interactions et une paramétrisation permettent de représenter
un réseau de régulation biologique complet grâce à sa structure et son évolution.
En effet, un paramètre $K_{a,\omega}$ représente un ensemble de valeurs vers lesquelles
le composant $a$ évolue dans tout état où l'ensemble de ses ressources est égal à $\omega$.
Plus précisément, $a$ va évoluer vers la valeur de $K_{a,\omega}$ qui est la plus proche de
son niveau d'expression courant.
On appellera dans la suite \emph{modèle de Thomas} tout couple $(\GI; K)$
formé d'un graphe des interactions et d'une paramétrisation.

Pour finir, René Thomas a formulé deux hypothèses concernant la dynamique de ses modèles.
Tout d'abord, elle doit être asynchrone, c'est-à-dire qu'un seul composant peut évoluer
entre chaque état.
Cette hypothèse rend compte du fait qu'il est infiniment peu probable que deux composants passent
en même temps un seul d'expression discret.
Il a de plus proposé de la rendre unitaire ;
autrement dit, chaque composant ne peut évoluer que d'un niveau d'expression discret à la fois.
Ces deux hypothèses permettent de conserver un certain nombre de propriétés propres aux
systèmes d'équations différentielles dans lesquels l'évolution de chaque composant est continue.
Nous définissons donc la dynamique d'un modèle de Thomas avec paramètres discrets comme suit :
il existe une transition d'un état $s$ vers un autre état $s'$ si et seulement si
il existe un unique un composant $a$ qui évolue entre ces deux états,
d'exactement un niveau d'expression et vers le paramètre $K_{a,\RRBres{a}{s}}$
(\defref{thomas-dynamique}).
Il est à noter cependant que $a$ ne peut pas évoluer si son niveau d'expression dans l'état $s$
appartient déjà à l'intervalle du paramètre $K_{a,\RRBres{a}{s}}$.

\begin{definition}[Dynamique unitaire d'un modèle de Thomas ($\RRBtrans{}{}$)]
\deflabel{thomas-dynamique}
  Pour tout modèle de Thomas $\RRB = (\GI; K)$,
  La dynamique de $\RRB$ est donnée par la relation de transition
  $\RRBtrans{}{}\ \in \RRBstates \times \RRBstates$ définie par :
  \begin{align*}
    \forall s, s' \in \RRBstates, \RRBtrans{s}{s'}
      &\Longleftrightarrow \exists a \in \components,
    \RRBget{s}{a} \notin K_{a, \RRBres{a}{s}} \wedge
      \RRBget{s'}{a} = \RRBget{s}{a} + \delta^a(s) \\
      &\qquad\quad \wedge \forall b \in \components, b \neq a \Rightarrow \RRBget{s}{b} = \RRBget{s'}{b}
  \end{align*}
  avec : $\delta^a(s) = 
    \begin{cases}
      +1 & \text{si } \RRBget{s}{a} < K_{a, \RRBres{a}{s}} \\
      -1 & \text{si } \RRBget{s}{a} > K_{a, \RRBres{a}{s}} \\
    \end{cases}$
\end{definition}
Les symboles «~$<$~» et «~$>$~» de cette définition permettant de comparer un entier à
un intervalle sont définis à la \vsecref{notations}.



\begin{example}
  La \figref{thomas}(droite) donne un exemple de paramétrisation
  du graphe des interactions de la \figref{thomas}(gauche),
  ce qui en fait un modèle de Thomas complet,
  dont la \figref{thomas-dynamique} donne l'espace des états.
%   Les états y sont représentés par des triplets $\RRBetat{a_i, b_j, c_k}$
%   où $i$, $j$ et $k$ représentent respectivement le niveau d'expression de $a$, $b$ et $c$.
%   Dans ce modèle de Thomas, les transitions suivantes sont possibles d'après la
%   \defref{thomas-dynamique} :
%     \[\RRBetat{a_0, b_1, c_1} \rightarrow \RRBetat{a_1, b_1, c_1} \rightarrow
%       \RRBetat{a_2, b_1, c_1} \rightarrow
%       \RRBetat{a_2, b_0, c_1} \rightarrow \RRBetat{a_1, b_0, c_1}\]
%   où $a_i$ représente le composant $a$ au niveau d'expression discret $i$.
  On note notamment la présence de trois état stables pour ce modèle,
  c'est-à-dire trois états depuis lesquels plus aucune évolution n'est possible :
  $\RRBetat{a_0, b_0, c_0}$, $\RRBetat{a_1, b_1, c_0}$ et $\RRBetat{a_1, b_0, c_1}$,
  où $x_i$ représente le niveau d'expression $i$ pour le composant $x$.
%   Cette séquence d'états termine dans un état stable : plus aucune évolution n'est possible
%   depuis l'état $\RRBetat{a_1, b_0, c_1}$.
  
  On peut observer l'aspect unitaire de la dynamique d'un modèle de Thomas sur ce graphe.
  En effet, malgré le paramètre $K_{a,\{b,c\}} = \segm{2}{2}$, le composant $a$ ne peut
  pas directement passer de l'état $a_0$ à l'état $a_2$ en «~sautant~» l'état $a_1$.
  C'est pourquoi on observe les transitions
  $\RRBtrans{\RRBetat{a_0,b_1,c_1}}{\RRBetat{a_1,b_1,c_1}}$ et
  $\RRBtrans{\RRBetat{a_1,b_1,c_1}}{\RRBetat{a_2,b_1,c_1}}$.
  
  Nous notons enfin que les paramètres sous forme d'intervalles permettent notamment
  de rendre un composant immobile.
  C'est par exemple le cas du paramètre $K_{b,\emptyset} = \segm{0}{1}$,
  qui est pris en compte lorsque $a$ n'est pas au niveau $2$.
  Dans une sémantique ne permettant que des paramètres unitaires, une auto-régulation de $b$
  serait nécessaire.
  
  \begin{figure}[ht]
    \begin{center}
    \begin{tikzpicture}
      \matrix [column sep=0.7cm, row sep=1cm]{
        \node (s200) {$\RRBetat{a_2,b_0,c_0}$}; &
        \node (s210) {$\RRBetat{a_2,b_1,c_0}$}; &
        \node (s201) {$\RRBetat{a_2,b_0,c_1}$}; &
        \node (s211) {$\RRBetat{a_2,b_1,c_1}$}; \\
        \node (s100) {$\RRBetat{a_1,b_0,c_0}$}; &
        \node (s110) {$\RRBetat{a_1,b_1,c_0}$}; &
        \node (s101) {$\RRBetat{a_1,b_0,c_1}$}; &
        \node (s111) {$\RRBetat{a_1,b_1,c_1}$}; \\
        \node (s000) {$\RRBetat{a_0,b_0,c_0}$}; &
        \node (s010) {$\RRBetat{a_0,b_1,c_0}$}; &
        \node (s001) {$\RRBetat{a_0,b_0,c_1}$}; &
        \node (s011) {$\RRBetat{a_0,b_1,c_1}$}; \\
      };
      \path[->]
        (s010) edge (s110)
        (s100) edge (s000)
        (s200) edge (s100)
        (s210) edge (s110) edge (s200)
        (s001) edge (s101)
        (s011) edge (s111)
        (s111) edge (s211)
        (s201) edge (s101)
        (s211) edge (s201)
      ;
    \end{tikzpicture}
    \end{center}
    \caption{\figlabel{thomas-dynamique}%
      Représentation de la dynamique du modèle de Thomas donné à la \figref{thomas}.
      Chaque état est représenté par un triplet $\RRBetat{a_i, b_j, c_k}$
      où $i$, $j$ et $k$ représentent respectivement le niveau d'expression de $a$, $b$ et $c$,
      et une transition entre deux états est représentée par une flèche.
    }
  \end{figure}
\end{example}



\subsubsection*{Graphe des interactions minimal}
Il est possible d'inférer le graphe des interactions \emph{minimal} d'un modèle de Thomas,
en ne conservant que les arcs fonctionnels pour la dynamique.
Définissons $f^a(s) = \RRBget{s}{a}$ si $s[a]\in K_{a, \RRBres{a}{s}}$,
et $f^a(s) = \RRBget{s}{a} + \delta^a(s)$ sinon.
Une régulation positive (\resp négative) de $b$ sur $a$ n'est inférée que s'il existe
un état $s$ tel que l'augmentation du niveau de $b$ aurait pour conséquence
une augmentation (\resp diminution) de la valeur de $f^a$ ;
ou, en d'autres termes, s'il existe un état $s$ tel que :
$f^a(s \recouvre b_i) < \text{(\resp $>$) } f^a(s \recouvre b_{i+1})$, 
où $i < l_b$ et $s \recouvre b_i$ est l'état identique à $s$
sauf pour la composante de $b$ qui a été remplacée par $b_i$.
Un tel graphe des interactions peut être utilisé pour inférer des propriétés globales sur
la dynamique \citeaffixed{Richard10,PR11-SASB}{cf. par exemple}.

\subsubsection*{Discussion sur la valeur du plafond}
\label{plafond}
Le plafond $l_a$ d'un composant $a$, qui est le niveau d'expression maximum de $a$,
est généralement choisi comme égal au nombre $\card{\components^+(a)}$
de régulations sortant du nœud $a$ dans le graphe des interactions,
c'est-à-dire le nombre de composants qu'il régule.
En effet, chaque niveau discret dans $\segm{0}{l_a}$ représente un ensemble arbitraire de valeurs
pour une donnée réelle et généralement continue de $a$ (niveau d'activité, concentration, etc.),
et pour laquelle $a$ possède une certaine influence sur plusieurs autres composants qu'il régule.
Ainsi, autoriser des valeurs de $l_a$ plus grandes que $\card{\components^+(a)}$
n'augmente pas l'expressivité du modèle
car plusieurs niveaux d'expression de $a$ auront alors le même rôle au niveau des régulations.
À l'inverse, réduire ce plafond à des valeurs plus petites que $\card{\components^+(a)}$
sous-entendrait que plusieurs seuils de régulations sortant de $a$ sont identiques,
ce qui est biologiquement peu plausible.
Cependant, ces considérations peuvent être ignorées pour des raisons diverses de modélisation,
et nous avons choisi de ne pas contraindre le plafond d'un composant dans ce document
pour des raisons de simplicité.
Ainsi, dans l'exemple de modèle de Thomas de la \figref{thomas},
il serait suffisant d'avoir $l_a = 1$ (et $\arc{a}{-}{1}{b} \in E$)
étant donné que $a$ ne possède qu'une régulation sortante.
La valeur $l_a = 2$ permet cependant de rendre l'exemple plus intéressant
en montrant notamment l'aspect unitaire de la dynamique du modèle de Thomas
(et $\arc{a}{-}{2}{b} \in E$ peut être choisi arbitrairement).

\subsubsection*{Discussion sur les signes}
L'ajout d'arcs non-signés ($\uns$) permet de modéliser des régulations dont la tendance générale
est plus complexe qu'une simple activation ou inhibition.
Il est à noter qu'utiliser $\uns$ comme signe pour étiqueter les régulations en plus
des deux signes habituels $+$ et $-$ n'augmente pas l'expressivité du formalisme.
En effet, les signes n'ont pas d'impact sur la paramétrisation (\defref{thomas-param})
ou sur la dynamique (\defref{thomas-dynamique}).
Ils ont cependant un but informatif, car ils résument les différentes régulations du modèle.
Ils seront d'ailleurs utilisés au \chapref{expressivite}
comme base pour énumérer des paramétrisations incomplètes.
Les arcs non-signés seront de plus exploités pour modéliser un graphe des interactions
avec une connaissance partielle de certaines régulations.
Cependant, comme ils ne sont pas courants dans la littérature,
l'état de l'art de la section suivante portera principalement
sur les modèles de Thomas sans arcs non-signés ($E_\uns = \emptyset$).

%\todo{Hypothèses monotonie / etc.}

\subsubsection*{Discussion sur les intervalles de paramètres}
Tandis que la littérature utilise principalement des paramètres entiers
(c'est-à-dire comportant une unique valeur, par exemple : $K_{a,\omega} = i$)
nous proposons ici d'utiliser des intervalles comme valeurs des paramètres
(par exemple : $K_{a,\omega} = \segm{i}{j}$).
Ces intervalles permettent notamment de faciliter la représentation,
bien qu'il soit intéressant de noter qu'ils augmentent l'expressivité du modèle de Thomas.
En effet, considérons un cas fictif où $K_{a,\omega} = \segm{i}{i+2}$
(c'est-à-dire un intervalle contenant trois valeurs) et $s$ un état tel que
$\RRBres{a}{s} = \omega$.
D'après la \defref{thomas-dynamique}, si $\RRBget{s}{a} \in K_{a,\omega}$, alors $a$
ne peut pas évoluer dans $s$ ; ainsi, $a$ possède trois états locaux stables :
$a_i$, $a_{i+1}$ et $a_{i+2}$.
Ce comportement ne peut pas être retrouvé à l'aide de paramètres entiers,
car il n'est pas possible de distinguer les trois configurations de ressources
correspondant aux trois états locaux.
Au plus deux états stables locaux peuvent être créés à l'aide d'une auto-régulation positive
sur $a$ pour distinguer deux configurations différentes
($\RRBget{s}{a} < t$ et $\RRBget{s}{a} \geq t$)
ce qui n'est cependant pas suffisant pour obtenir trois états stables.



\subsection{Définition des réseaux discrets asynchrones}
\seclabel{rda-def}

Certaines contraintes propres aux modèles de Thomas peuvent être relâchées pour permettre
des comportements supplémentaires.
Ainsi, il est courant de représenter des réseaux de régulation biologique
sous la forme de \emph{réseaux discrets asynchrones}.
Ces réseaux sont aussi fondés sur un graphe des interactions,
mais il utilisent des fonctions d'évolution (\defref{rda-def}) pour plus de permissivité,
en lieu et place de paramètres discrets tels que précédemment
formalisés à la \vdefref{thomas-param}.
Par ailleurs, l'hypothèse d'asynchronisme est conservée car un seul composant peut évoluer
depuis chaque état,
mais leur dynamique n'est pas unitaire dans le cas général
car ce composant peut évoluer d'un nombre arbitraire de niveaux d'expression
(\defref{rda-dynamique}).

\begin{definition}[Réseau discret asynchrone ($\RDA$)]
\deflabel{rda-def}
  Si $\GI = (\components; E)$ est un graphe des interactions,
  un \emph{réseau discret asynchrone} est un couple $\RDA = (\GI; F)$
  avec $F = (f_x)_{x \in \components}$, tels que
  $\forall x \in \components, f_x : \RRBreg{x} \rightarrow \segm{0}{l_x}$.
\end{definition}

\begin{definition}[Dynamique d'un réseau discret asynchrone ($\RRBtransrda{}{}$)]
\deflabel{rda-dynamique}
  Pour tout réseau discret asynchrone $\RDA = (\GI; F)$,
  La dynamique non-unitaire de $\RDA$ est donnée par la relation de transition
  $\RRBtransrda{}{}\ \in \RRBstates \times \RRBstates$ définie par :
  \begin{align*}
    \forall s, s' \in \RRBstates, \RRBtransrda{s}{s'}
      &\Longleftrightarrow \exists a \in \components,
    \RRBget{s}{a} \neq f_a(s) \wedge
      \RRBget{s'}{a} = f_a(s) \\
      &\qquad\quad \wedge \forall b \in \components, b \neq a \Rightarrow \RRBget{s}{b} = \RRBget{s'}{b}
  \end{align*}
\end{definition}

Enfin, nous désignons par \emph{réseau booléen asynchrone}
tout réseau discret asynchrone dont les composants ne possèdent que deux niveaux discrets,
c'est-à-dire tel que : $\forall x \in \components, l_x = 1$.



\subsection{Analyses formelles du modèle de Thomas}
\seclabel{thomas-analyse}

Nous proposons dans cette section un tour d'horizon rapide des différentes méthodes
d'analyse de la dynamique des modèles de Thomas.
Nous nous concentrons principalement sur les méthodes d'analyse statique,
c'est-à-dire permettant d'obtenir des résultats sur la dynamique d'un modèle
sans nécessiter d'en calculer la dynamique de façon complète et explicite.
Ces approches ont l'avantage de jouir d'une faible complexité, même si les résultats formels
qu'elles fournissent sont généralement moins précis ou approximés.
L'analyse du graphe des interactions seul peut ainsi permettre d'obtenir certains
résultats sur sa dynamique,
et la prise en compte de la paramétrisation permet d'affiner cette approche.
Cependant, ces résultats sont insuffisants dans beaucoup de cas,
nécessitant des approches plus exhaustives.
L'analyse complète de la dynamique d'un modèle de Thomas
est généralement invoquée en dernier recours, car elle peut être coûteuse en temps
et en espace mémoire du fait de l'explosion combinatoire inhérente au calcul
de l'espace des états.

\subsubsection*{Analyse du graphe des interactions}
Plusieurs travaux se sont intéressés aux conclusions qui pouvaient être
tirées du seul graphe des interactions.
La recherche d'\emph{états stables}, aussi appelés \emph{points fixes},
présente un intérêt particulier dans l'étude des réseaux de régulation biologique
et a été de fait très étudiée.
Les conjectures de \citefullname{Thomas81}{René}
tracent notamment un lien entre la présence de circuits dans les régulations et celle
d'oscillations ou d'états stables.
Elles se formulent comme suit :
\begin{itemize}
  \item un circuit positif (c.-à-d.~comportant un nombre pair de régulations négatives)
    est une condition nécessaire à la présence de plusieurs points fixes,
  \item un circuit négatif (c.-à-d.~comportant un nombre impair de régulations négatives)
    est une condition nécessaire à la présence d'oscillations soutenues dans la dynamique,
    et donc à la présence d'un attracteur constitué d'au moins deux états.
\end{itemize}
Ces conjectures ont par la suite été démontrées
dans le cas booléen, c'est-à-dire lorsque $\forall x \in \components, l_x = 1$
\cite{RRT08,Richard06thesis}
ainsi que dans le cas multivalué \cite{RiCo07,Richard10}.

Plusieurs résultats viennent de surcroît enrichir ce résultat concernant
l'existence de points fixes.
Des travaux ont par exemple
permis d'obtenir une valeur maximale du nombre de points fixes possibles
dans un modèle de Thomas tant booléen \cite{aracena2008maximum} que multivalué \cite{Richard09}.
Cette valeur dépend du nombre de composants à supprimer pour éliminer
tout circuit positif dans le modèle.
\citeasnoun{PR10-CRAS} ont quant à eux proposé une valeur minimale du nombre de
\emph{points fixes topologiques} d'un modèle booléen.
Ces points fixes ont la particularité de ne pas dépendre de la paramétrisation,
ce qui permet de généraliser un tel résultat à un ensemble de modèles de Thomas.

\subsubsection*{Analyse de la paramétrisation}
Les conditions données précédemment sont \emph{nécessaires} mais non \emph{suffisantes}
pour observer l'apparition des comportements décrits
(oscillations soutenues et multistationnarité).
Plusieurs travaux permettent d'analyser la fonctionnalité des circuits,
c'est-à-dire leur capacité à effectivement «~générer~» ces comportements,
et donc de rendre ces conditions susnommées nécessaires.
Partant d'un modèle de Thomas donné,
il est possible de représenter sa paramétrisation
à l'aide de diagrammes de décision binaires ou multivalués \cite{Bryant86,Srinivasan90}
afin d'en déduire
l'ensemble des états stables mais aussi d'analyser la fonctionnalité des circuits
en effectuant des produits de diagrammes de décision \cite{Naldi07}.
Les différentes façons dont les oscillations soutenues et la multistationnarité
sont «~générées~»
ont aussi été étudiées, ce qui a permis de préciser ces conditions \cite{Comet13}.

% Compression des paramètres : \cite{Naldi07}
%   qui utilisent des BDD : \cite{Bryant86,Srinivasan90}
% Permet certaines analyses efficaces : points fixes et fonctionnalité des circuits

% \subsubsection*{Caractérisation et recherche des points fixes}
% 
% \todo{dispatcher}
% 
% La recherche d'\emph{états stables}, aussi appelés \emph{points fixes},
% présente un intérêt particulier dans l'étude des réseaux de régulation biologique.
% Les conjectures de \citefullname{Thomas81}{René},
% qui ont par la suite été démontrées \cite{RiCo07}
% portent notamment sur ces questions
% en formulant une condition nécessaire à l'existence de plusieurs points fixes.
% Cette condition n'est cependant pas suffisante
% et ne renseigne pas sur le nombre de points fixes.
% %Dans le cadre d'un modèle booléen (tel que $\forall x \in \components, l_x = 1$)
% L'analyse des interactions seules permet de donner une limite supérieure au nombre
% de points fixes \cite{aracena2008maximum,Richard09}
% ainsi que du nombre de points fixes topologiques,
% (c'est-à-dire ne dépendant pas des paramètres) dans le cas booléen
% \cite{PR10-CRAS},
% sans pour autant proposer une énumération exhaustive.
% \citeasnoun{Naldi07} ont proposé une méthode efficace
% basée sur l'utilisation de diagrammes de décision binaires
% pour l'énumération des points fixes.

\subsubsection*{Analyse de la dynamique}
Afin d'obtenir des résultats plus précis sur le comportement d'un modèle,
comme l'activation d'un composant ou la présence d'oscillations sur une partie
donnée du modèle,
ce que les méthodes décrites précédemment ne permettent pas d'obtenir,
il peut devenir nécessaire de faire appel à des techniques
de \textit{model checking} plus poussées
\citeaffixed{Richard06,gallet14adapting}{voir p.~ex.}.
Ces méthodes permettent généralement une analyse exhaustive de la dynamique,
tout en devant faire face à l'explosion combinatoire typique de ce type d'analyse,
qui limite grandement la taille des modèles étudiés.
De telles analyses ont été théorisées par \citeasnoun{bernot04}
et se basent sur l'expression de propriétés exprimées en logiques temporelles
CTL \cite{Clarke82} ou LTL \cite{Pnueli77}.
Elles possèdent l'avantage d'être très complètes
car la dynamique est entièrement explorée et l'expressivité des logiques temporelles permet
d'exprimer beaucoup de propriétés intéressantes,
mais au prix d'une complexité très importante.
Afin de réduire cette complexité,
\citeasnoun{Naldi09} proposent d'effectuer de la réduction de modèles
en éliminant certaine nœuds qui ont un rôle secondaire.
Le modèle obtenu possède une taille inférieure et certaines bonnes propriétés sont conservées,
comme l'ensemble des points fixes et des attracteurs,
au prix d'une dynamique parfois sous-approximée par rapport au modèle d'origine.


% Les Frappes de Processus
\section{Les Frappes de Processus standards}

Cette section définit les Frappes de Processus standards
telles qu'elles ont été formalisées par \citeasnoun{PMR10-TCSB}.
Les Frappes de Processus offrent une représentation discrète, asynchrone et indéterministe
avec une définition atomique des interactions entre les différents composants du modèle.
Elles sont particulièrement adaptées aux représentations des réseaux de régulation biologiques
bien qu'elles soient assez générales pour potentiellement permettre des applications plus
larges en informatique ou dans d'autres domaines.

Nous donnerons tout d'abord à la \secref{ph} une définition des Frappes de Processus standards
%accompagnée d'une discussion,
ainsi qu'un certain nombre de définitions formelles supplémentaires
nécessaires pour la suite de ce manuscrit.
Nous détaillons ensuite à la \secref{sc} le mécanisme particulier des sortes coopératives
tel qu'il avait été formalisé dans les travaux de Loïc Paulevé,
et nous montrerons qu'il permet de représenter l'action conjointe de plusieurs composants,
mais avec certains défauts difficiles à corriger dans cette version du formalisme.
Enfin, nous aborderons brièvement les travaux précédent
concernant l'analyse statique (\secref{ph-as-pf}) qui permet d'obtenir des résultats
sur l'ensemble des états stables ou l'atteignabilité d'un niveau local au sein d'un modèle,
et concernant l'ajout de paramètres stochastiques (\secref{ph-stocha})
dans le but d'introduire une composante temporelle continue dans les Frappes de Processus.



\subsection{Définition}
\seclabel{ph}

Les \emph{Frappes de Processus standards} telles que données à la \defref{ph},
aussi appelées \textit{Process Hitting},
ou plus simplement \emph{Frappes de Processus} dans ce chapitre,
permettent une modélisation atomique et asynchrone des interactions entre composants.
Un modèle de Frappes de Processus standards comporte un nombre fini de \emph{sortes}
généralement notées $a$, $b$, $c$...
Celles-ci permettent de représenter les différentes entités du modèle,
qu'il s'agisse de composant ayant une réalité biologique (gène, protéine...)
ou d'entités nécessaire à la modélisation (comme les sortes coopératives
qui seront décrites à la \secref{sc}).
Chaque sortes contient plusieurs \emph{processus},
qui représentent les différentes
niveaux d'expression discrets accessibles par la sorte,
et qui sont notés $a_i$
où $a$ est le nom de la sorte et $i$ l'indice du processus dans cette sorte.
Un processus n'appartient qu'à une unique sorte.
Chaque processus est dit \emph{actif} s'il représente le niveau d'expression
dans lequel doit se trouver sa sorte à un certain moment.
Un \emph{état} du modèle est donc décrit par l'ensemble des processus actifs à un instant donné,
avec exactement un processus actif par sorte
---~afin de ne pas sur-représenter ou sous-représenter le niveau d'expression courant d'une sorte.

La dynamique est introduite dans les Frappes de Processus par des \emph{actions}
qui permettent de modifier le processus actif d'une sorte,
à la condition éventuelle qu'un processus donné d'une autre sorte soit présent.
Une action consiste donc en un triplet de processus $\PHfrappe{a_i}{b_j}{b_k}$
qui se lit : «~$a_i$ frappe $b_j$ pour le faire bondir en $b_k$~»,
et qui signifie que si, dans un état donné, les processus $a_i$ et $b_j$ sont
tous les deux présents, alors il est possible d'activer $b_k$ (et de désactiver $b_j$)
dans l'état suivant.
Autrement dit, le processus actif de la sorte $b$ peut \emph{bondir}
de $b_j$ à $b_k$ à condition que $a_i$ soit actif ;
lorsque cela arrive, on dit qu'on a \emph{joué} l'action $\PHfrappe{a_i}{b_j}{b_k}$.
Par convention, on contraint de plus que $b_j \neq b_k$ pour assurer que le jeu d'une action
provoque bien un changement de processus actif.
Il est aussi possible de définir une auto-action, où $a_i = b_j$ (et nécessairement $a = b$),
qui permet de représenter le cas particulier où le processus $b_j$ peut bondir en $b_k$
sans autre condition.

Les Frappes de Processus sont conçues comme un formalisme
à temps discret asynchrone, ce qui signifie que
l'évolution d'un tel modèle est modélisée par une succession de pas de temps discrets
qui représentent la succession des états du modèle,
et exactement une action est jouée entre deux états successifs.
Cela implique qu'un seul processus actif à la fois peut bondir entre deux
pas de temps successifs, et donc qu'une seule sorte peut évoluer entre deux états.
De plus, cela rend la dynamique Frappes de Processus indéterministe,
car à tout état du modèle peuvent correspondre plusieurs états successeurs
dans le cas où plusieurs actions peuvent y être jouées.
Enfin, nous notons que si aucune action n'est jouable dans un état, alors celui-ci
ne possède pas de successeurs et le modèle ne peut plus évoluer.

\begin{definition}[Frappes de Processus standards]
\deflabel{ph}
  Les \emph{Frappes de Processus standards} sont définies
  par un triplet $\PH = (\PHs; \PHl; \PHh)$, où :
  \begin{itemize}
    \item $\PHs \DEF \{a, b, \dots\}$ est l'ensemble fini et dénombrable des \emph{sortes} ;
    \item $\PHl \DEF \bigtimes{a \in \PHs} \PHl_a$ est l'ensemble fini des \emph{états},
      où $\PHl_a = \{a_0, \ldots, a_{l_a}\}$ est l'ensemble fini et dénombrable
      des \emph{processus} de la sorte $a \in \PHs$ et $l_a \in \sN^*$.
      Chaque processus appartient à une unique sorte :
      $\forall (a_i; b_j) \in \PHl_a \times \PHl_b, a \neq b \Rightarrow a_i \neq b_j$ ;
    \item $\PHh \DEF \{\PHfrappe{a_i}{b_j}{b_k} \mid (a; b) \in \PHs \times \PHs \wedge
      (a_i; b_j; b_k) \in \PHl_a \times \PHl_b \times \PHl_b \wedge
      b_j \neq b_k \wedge a = b \Rightarrow a_i = b_j \}$ est l'ensemble fini des \emph{actions}.
  \end{itemize}
\end{definition}
%
\noindent
On note $\PHproc \DEF \bigcup_{a \in \PHs} \PHl_a$ l'ensemble de tous les processus.
La sorte d'un processus $a_i$ est donnée par $\PHsort(a_i) = a$ ;
on définit aussi l'ensemble des sortes d'un ensemble de processus par :
$\forall A \subset \Proc, \sortes{A} = \{ \sorte{p} \mid p \in A \}$.
Étant donné un état $s \in \PHl$, le processus de la sorte $a \in \PHs$ présent dans $s$ est donné
par $\PHget{s}{a}$, \cad la coordonnée correspondant à $a$ dans l'état $s$.
Si $a_i \in \PHl_a$, nous définissons la notation : $a_i \in s \EQDEF \PHget{s}{a} = a_i$.
Pour toute action $h = \PHfrappe{a_i}{b_j}{b_k} \in \PHh$,
$a_i$ est appelé le \emph{frappeur}, $b_j$ la \emph{cible} et $b_k$ le \emph{bond} de $h$,
et on note : $\hitter{h} = a_i$, $\target{h} = b_j$ et $\bounce{h} = b_k$.

\begin{example}
\exlabel{metazoan-ph-nocoop}
  La figure \figref{metazoan-ph-nocoop} illustre une représentation possible des
  Frappes de Processus standards.
  Le modèle $\PH = (\PHs, \PHl, \PHh)$ représenté comporte trois sortes :
  $\PHs = \{ a, c, f \}$.
  Chaque sorte comporte exactement deux processus :
  \[\PHl_a = \{ a_0, a_1 \} \enspace;\quad
    \PHl_b = \{ b_0, b_1 \} \enspace;\quad
    \PHl_f = \{ f_0, f_1 \} \enspace.\]
  On peut notamment en déduire le nombre total d'états du système :
  $\card{\PHl} = 2^3 = 8$.
  Enfin, ce modèle comporte 7 actions :
  \begin{align*}
    \PHh = \{\qquad
      \PHfrappe{c_1}{a_1}{a_0}\quad; && \PHfrappe{c_0}{a_0}{a_1}\quad;& \\
      \PHfrappe{c_1}{c_1}{c_0}\quad; && \PHfrappe{f_1}{c_0}{c_1}\quad;& \\
      \PHfrappe{f_1}{a_0}{a_1}\quad; && \PHfrappe{f_0}{c_1}{c_0}\quad;& \\
      \PHfrappe{f_1}{f_1}{f_0}\quad\; &&& 
    \qquad\}
  \end{align*}
  
  Ces Frappes de Processus représentent un modèle simplifié du mécanisme
  de segmentation des métazoaires qui permet par exemple de décrire la production de rayures
  chez les drosophiles.
  Il a été originellement établi \textit{in silico} par \citeasnoun{MSB4100192}
  à l'aide d'un formalisme à base d'équations différentielles,
  et le modèle présenté ici est inspiré du modèle proposé par \citeasnoun{PMR10-TCSB}.
  
  Les trois sortes $a$, $b$ et $c$ de ce modèle représentent différents gènes du système,
  que nous qualifierons d'\emph{actifs} dans le reste de ce document lorsqu'ils seront à l'état 1.
  La production de pigment est déclenchée par le produit du gène $a$,
  et une succession d'activations de celui-ci permet donc de produire des rayures.
  Pour que celles-ci soient régulières, il est donc nécessaire que la durée d'activation
  de $a$ soit constante, et que la durée entre deux activations le soit aussi.
  Ce mécanisme est réglé par le gène $c$ qui inhibe à la fois le gène $a$ à intervalles réguliers,
  et s'inhibe lui-même afin d'avoir le rôle d'une horloge.
  Enfin, le procédé complet est dirigé par le gène $f$ qui, lorsqu'il est actif,
  permet la progression d'un front au niveau duquel les pigments sont déposés.
  Ce gène peut aussi s'auto-inhiber après une certaine période, faisant cesser
  l'oscillation de l'horloge et ainsi la production de rayures.
  
  \begin{figure}[ht]
  \begin{center}
  \scalebox{1.5}{\begin{tikzpicture}
    \TSort{(0,4)}{c}{2}{l}
    \TSort{(1,1)}{f}{2}{l}
    \TSort{(3,4)}{a}{2}{r}
    
    \TAction{c_1}{a_1.west}{a_0.north west}{}{right}
    \TAction{f_1}{c_0.west}{c_1.south west}{bend left=50, in=90}{left}
    \TAction{c_1}{c_1.west}{c_0.north west}{selfhit}{right}
    \TAction{f_1.north east}{f_1.south east}{f_0.north east}%
      {selfhit, min distance=30, bend left, out=150, in=90}{left}
    \TAction{f_0.east}{c_1.south east}{c_0.north east}{bend right=80, in=-120}{left}
    
    % Fausse coopération
    \TAction{f_1.north}{a_0.south west}{a_1.south west}{bend left=20}{left}
    \TAction{c_0}{a_0.west}{a_1.south}{}{left}
    
    \TState{f_1, a_0, c_0}
  \end{tikzpicture}}
  \caption{\figlabel{metazoan-ph-nocoop}%
    Un exemple de Frappes de Processus standards.
    Les sortes sont représentées par des rectangles arrondis 
    contenant des cercles représentant les processus.
    Ainsi, le processus $a_1$ est représenté par le cercle marqué «~1~»
    dans le rectangle étiqueté «~$a$~», etc.
    Chaque action est de plus symbolisée par un couple de flèches,
    l'une en trait plein et l'autre en pointillés ;
    par exemple, l'action $\PHfrappe{c_1}{a_1}{a_0}$
    est représentée par une flèche pleine entre les processus $c_1$ et $a_1$
    suivie d'une flèche en pointillés entre $a_1$ et $a_0$.
    Enfin, les processus grisés représentent un état possible
    pour ces Frappes de Processus : $\etat{a_0, c_0, f_1}$, qui peut aussi faire office
    d'état initial pour ce modèle.
  }
  \end{center}
  \end{figure}
\end{example}

Les séquences d'actions ont un rôle particulier pour les Frappes de Processus.
Elles permettent notamment d'abstraire une dynamique locale en se concentrant
sur la conséquence plutôt que sur la cause,
et seront notamment utiles pour la méthode d'analyse statique développée
au \chapref{phcanonique}.
Pour toute séquence d'actions $A$,
on note $\sortes{A}$ l'ensemble des sortes dont au moins un processus figure dans $A$
en tant que frappeur, cible ou bond d'une action.
De plus, pour toute sorte $a$,
on note $\prem{a}{A}$ le premier processus référencé de $a$,
$\sup{A}$ l'ensemble de tous ces premiers processus,
$\der{a}{A}$ le dernier processus référencé de $a$
et $\fin{A}$ l'ensemble de tous ces derniers processus.
Ces notations sont formellement définies dans la \defref{premder}.

\begin{definition}[$\premsymbol$, $\dersymbol$, $\suppsymbol$ et $\finsymbol$]
\deflabel{premder}
  Pour toute séquence d'actions $A$ et pour toute sorte $a$,
  $\prem{a}{A}$ est le premier processus de $a$ référencé dans $A$,
  en tant que frappeur ou en tant que cible,
  et $\der{a}{A}$ en est le dernier,
  en tant que frappeur ou en tant que bond.
  \begin{align}
  \eqlabel{prem}
    \prem{a}{A} &=
      \begin{cases}
        \varnothing & \text{si } a \notin \sortes{A}, \\
        \hitter{A_m} & \text{si } m = \min\{ n \in \indexes{A} \mid a \in \sortes{A_n} \}
          \wedge \sorte{\hitter{A_m}} = a, \\
        \target{A_m} & \text{sinon si } m = \min\{ n \in \indexes{A} \mid a \in \sortes{A_n} \}
          \wedge \sorte{\target{A_m}} = a \enspace;
      \end{cases} \\
  \eqlabel{der}
    \der{a}{A} &=
      \begin{cases}
        \varnothing & \text{si } a \notin \sorts(A), \\
        \bounce{A_m} & \text{si } m = \max\{ n \in \indexes{A} \mid a \in \sortes{A_n} \}
          \wedge \sorte{\bounce{A_m}} = a, \\
        \hitter{A_m} & \text{sinon si } m = \max\{ n \in \indexes{A} \mid a \in \sortes{A_n} \}
          \wedge \sorte{\hitter{A_m}} = a \enspace.
      \end{cases}
  \end{align}

  Pour toute séquence d'actions $A$,
  $\supp{A}$ et $\fin{A}$ renvoient respectivement à l'ensemble des premiers
  et derniers processus de $A$ pour toutes les sortes qui y figurent.
  \begin{align}
  \eqlabel{supp}
    \supp{A} &= \{ p \in \Proc \mid \sorte{p} \in \sorte{A} \wedge
      p = \prem{\sort{p}}{\delta} \} \enspace, \\
  \eqlabel{fin}
    \fin{A} &= \{ p \in \Proc \mid \sorte{p} \in \sorte{A} \wedge
      p = \der{\sort{p}}{\delta} \} \enspace.
  \end{align}
\end{definition}

La \defref{substate} établit la notion de sous-état sur un ensemble de sortes,
\cad un ensemble de processus qui sont deux à deux de sortes différentes,
ce qui permet de ne considérer qu'une partie d'un état complet.
Nous notons $\PHsubl$ l'ensemble de tous les sous-états et nous constatons qu'un
état est \textit{a fortiori} un sous-état : $\PHl \subset \PHsubl$.
Nous notons de plus $\PHsublset$ l'ensemble des sous-états désordonnés,
c'est-à-dire dont l'ordre entre les sortes a été oublié.
Le recouvrement d'un état $s$ par un processus $a_i$ est formalisé à la \defref{statecap}
par un état identique à $s$, sauf pour le processus de $a$ qui a été remplacé par $a_i$,
ce qui permettra de définir la dynamique des Frappes de Processus %avec $k$ classes de priorités
à la \defref{play}.
La définition de recouvrement est aussi étendue à un sous-état désordonné,
autrement dit, à un ensemble de processus contenant au plus un processus par sorte.

\begin{definition}[Sous-états ($\PHsublize{\PHl}$)]
\deflabel{substate}
  Si $S \subset \PHs$ est un ensemble de sortes, un sous-état sur $S$ est un élément de :
  $\PHsubl[\PHl]_S \DEF \bigtimes{a \in S} \PHl_a$.
  L'ensemble de tous les sous-états est noté :
  $\PHsubl[\PHl] \DEF \bigcup_{S \in\powerset(\PHs)} \PHsubl[\PHl]_S$.
  
  \noindent
  De plus, si $\mysigma \in \PHsubl[\PHl]$ et $s \in \PHl$, on note alors :
  \[\mysigma \subseteq s \EQDEF \forall a_i \in \Proc, a_i \in \mysigma \Rightarrow a_i \in s
    \enspace.\]
  
  \noindent
  Enfin, si $S \subset \PHs$, on note :
  $\PHsublset_S = \{ \toset{ps} \subset \Proc \mid ps \in \PHsubl_S \}$
  et
  $\PHsublset = \{ \toset{ps} \subset \Proc \mid ps \in \PHsubl \}$.
\end{definition}
%
\begin{definition}[Recouvrement ($\recouvre : \PHl \times \PHproc \rightarrow \PHl$)]
\deflabel{statecap}
  Étant donné un état $s \in \PHl$ et un processus $a_i \in \PHproc$,
  $(s \recouvre a_i)$ est l'état défini par :
  $\PHget{(s \recouvre a_i)}{a} = a_i \wedge \forall b \neq a, \PHget{(s \recouvre a_i)}{b} = \PHget{s}{b}$.
  On étend de plus cette définition à un ensemble de processus
  par le recouvrement de l'état par chaque processus,
  à condition que les processus de l'ensemble soient tous de sortes différentes :
  $\forall ps \in \PHsublset, s \recouvre ps = s \underset{a_i \in ps}{\recouvre} a_i$.
\end{definition}

Une propriété de jouabilité telle que décrite à la \defref{ppl}
est équivalente à une formule booléenne dont les atomes sont des processus dans $\Proc$.
Le langage des propriétés de jouabilité permet de décrire la présence d'une configuration
de processus actifs dans un état donné.
Il permet notamment de décrire en termes formels la jouabilité d'une action,
ce qui est immédiatement mis en pratique dans la \defref{fopph}.

\begin{definition}[Propriété de jouabilité ($\F$)]
  \label{def:ppl}
  Une \emph{propriété de jouabilité} est un élément du langage $\F$ défini \todo{inductivement ?} par :
  \begin{itemize}
    \item $\top$ et $\bot$ appartiennent à $\F$ ;
    \item si $a \in \PHs$ et $a_i \in \PHl_a$, alors $a_i$ appartient à $\F$
      et est appelé un \emph{atome} ;
    \item si $P \in \F$ et $Q \in \F$,
      alors $\neg P$, $P \wedge Q$ et $P \vee Q$ appartiennent à $\F$.
  \end{itemize}
%
  Si $P \in \F$ est une propriété de jouabilité et $\mysigma \in \PHsubl$ est un sous-état,
  on note $\Feval{P}{\mysigma}$ l'\emph{évaluation} de $P$ dans $\mysigma$:
  \begin{itemize}
    \item si $P = a_i \in \PHl_a$ est un atome, avec $a \in \PHs$,
      alors $\Feval{a_i}{\mysigma}$ est vraie si et seulement si $a_i \in \mysigma$ ;
    \item si $P$ n'est pas un atome, alors $\Feval{P}{\mysigma}$ est vraie si et seulement si
      on peut l'évaluer récursivement comme vraie en utilisant la sémantique habituelle des
      opérateurs $\neg$, $\wedge$ et $\vee$ et des constantes $\top$ et $\bot$.
  \end{itemize}
%
  Une fonction $\Fopsymbol : \PHh \rightarrow \F$ associant à toute action une propriété de jouabilité
  est appelée un \emph{opérateur de jouabilité}.
\end{definition}

Étant donné que ce langage n'utilise que des opérateur logiques classiques,
les propriétés de la logique booléenne sont applicables aux propriétés de jouabilité,
à savoir celles concernant la distributivité, l'associativité et la commutativité,
ainsi que les lois de De Morgan concernant la négation.

Il en résulte notamment la propriété suivante, permettant d'évaluer la négation d'un atome,
et qui dérive naturellement du fait que si un processus n'est pas actif dans un état donné,
cela signifie alors qu'un autre processus de la même sorte l'est :
\[\forall a \in \PHs, \forall a_i \in \PHl_a, \forall \mysigma \in \PHsubl,
  \Feval{\neg a_i}{\mysigma} \Leftrightarrow
  \Feval{\bigvee_{\substack{a_j \in \PHl_a\\a_j \neq a_i}} a_j}{\mysigma}\]

Enfin, on note dans la suite :
\[\forall A \in \PHsublset, \Fconj{A} \equiv \bigwedge_{p \in A} p \enspace.\]

\todo{Gluer}

\begin{definition}[Opérateur de jouabilité ($\Fopsymbol : \PHh \rightarrow \F$)]
\deflabel{fopph}
  L'opérateur de jouabilité des Frappes de Processus est défini par :
  \[\forall h \in \PHh, \Fop{h} \equiv \hitter{h} \wedge \target{h} \enspace.\]
\end{definition}

\begin{definition}[Dynamique des Frappes de Processus ($\PHtrans$)]
\deflabel{play}
  Une action $h \in \PHh$ est dite \emph{jouable}
  dans l'état $s \in \PHl$ si et seulement si :
  $\Feval{\Fop{h}}{s}$.
%  $\target{h} \in s \wedge \Feval{\Fop{h}}{s}$.
  Dans ce cas, $(s \PHplay h)$ est l'état résultant du jeu de l'action $h$ dans $s$,
  et on le définit par : $(s \PHplay h) = s \recouvre \bounce{h}$.
  De plus, on note alors : $s \PHtrans (s \PHplay h)$.

  Si $s \in \PHl$, un \emph{scénario} $\delta$ dans $s$
  est une séquence d'actions de $\PHh$ qui peuvent être jouées successivement dans $s$.
  L'ensemble de tous les scénarios dans $s$ est noté $\Sce(s)$.
\end{definition}

\begin{example}
  La séquence d'actions suivantes est un scénario dans l'état \TODO des Frappes de Processus
  de la \figref{metazoan-ph-nocoop} :
  \TODO
\end{example}



\subsection{Sortes coopératives}
\seclabel{sc}

\TODO


\todo{Figure d'exemple de sorte coopérative}

\begin{example}
\exlabel{metazoan-ph}
  Le modèle de Frappes de Processus de la \figref{metazoan-ph-nocoop}
  représentant le mécanisme de segmentation métazoaire évoqué à la page
  \expageref{metazoan-ph-nocoop},
  la production de pigment devrait uniquement être possible à la condition suivante :
  «~$f$ est actif et $c$ n'est pas actif~».
  Or dans l'état courant du modèle,
  la désactivation du gène $f$ n'empêche pas la production de pigment,
  car depuis tout état contenant $f_0$, il est toujours possible d'activer $a$
  à l'aide des actions $\PHfrappe{f_0}{c_1}{c_0}$ et $\PHfrappe{c_0}{a_0}{a_1}$.
  
  Afin de pallier partiellement ce défaut, il est possible d'ajouter une sorte coopérative $fc$,
  comme à la \figref{metazoan-ph},
  afin de détecter la présence de $f_1$ et $c_0$.
  Les deux actions $\PHfrappe{c_0}{a_0}{a_1}$ et $\PHfrappe{c_0}{a_0}{a_1}$
  sont alors remplacées par une action $\PHfrappe{fc_{10}}{a_0}{a_1}$
  afin d'avoir une véritable coopération entre ces deux processus pour activer $a$.
  
  \begin{figure}[ht]
  \begin{center}
  \begin{tikzpicture}
    \TSort{(0,4)}{c}{2}{l}
    \TSort{(1,0)}{f}{2}{l}
    \TSort{(7,4)}{a}{2}{r}
    
    \TSetTick{fc}{0}{00}
    \TSetTick{fc}{1}{01}
    \TSetTick{fc}{2}{10}
    \TSetTick{fc}{3}{11}
    \TSort{(4,1)}{fc}{4}{r}
    
    \TAction{fc_2}{a_0.west}{a_1.south west}{}{left}
    \TAction{c_1}{a_1.west}{a_0.north west}{}{right}
    \TAction{f_1}{c_0.west}{c_1.south west}{bend left=30, in=90}{left}
    \TAction{c_1}{c_1.west}{c_0.north west}{selfhit}{right}
    \TAction{f_1.north east}{f_1.south east}{f_0.north east}%
      {selfhit, min distance=30, bend left, out=150, in=90}{left}
    \TAction{f_0.east}{c_1.south east}{c_0.north east}{bend right=60, in=-140}{left}
    
    \path (1.8, 0.5) edge[coopupdate] (3.2, 2);
    \path (0.8, 4.5) edge[coopupdate] (3.2, 3);
    
    \TState{f_1, a_0, c_0, fc_2}
  \end{tikzpicture}
  \caption{\figlabel{metazoan-ph}%
    Amélioration du modèle de Frappes de Processus de la \figref{metazoan-ph-nocoop}
    à l'aide de la sorte coopérative $fc$.
    Les processus de cette sorte représentent les différents sous-états formés par les
    deux sortes $f$ et $c$.
    Ainsi, $fc_{00}$ représente le fait que $f_0$ et $c_0$ sont actifs, etc.
    Les actions permettant la mise à jour de cette sorte coopérative n'ont pas
    été représentée explicitement mais sont symbolisées par les deux flèches
    en zigzag provenant de $f$ et $c$.
  }
  \end{center}
  \end{figure}
\end{example}

\todo{Explications sur l'imperfection des sortes coopératives}

\begin{example}
  Malgré l'ajout d'une sorte coopérative $fc$ dans le modèle
  de la \figref{metazoan-ph-nocoop},
  il faut noter que le comportement désiré n'est pas exactement atteint.
  En effet, l'ajout de cette sorte coopérative devait permettre d'éviter toute activation de $a$
  lorsque $f$ devenait inactif, en permettant par exemple de jouer ce type de scénario
  depuis l'état initial $\etat{a_0, c_0, f_1, fc_{10}}$ :
    \[\PHfrappe{f_1}{f_1}{f_0} \cons \PHfrappe{f_0}{fc_{10}}{fc_{00}}\]
  après lequel il n'est plus possible d'atteindre un état où $a_1$ est actif.
  
  Cependant, il se trouve qu'il existe encore un cas particulier où $a$ peut être activé
  malgré la présence de $f_0$.
  Ce cas particulier relève du comportement mis en valeur \todo{dans les paragraphes ci-dessus},
  où une action a pour frappeur un processus de sorte coopératif qui ne devrait pas être
  actif si celle-ci avait été mise à jour.
  Il s'observe par exemple en jouant le scénario suivant depuis l'état initial
  $\etat{a_0, c_0, f_1, fc_{10}}$ :
    \[\PHfrappe{f_1}{f_1}{f_0} \cons \PHfrappe{fc_{10}}{a_0}{a_1} \enspace,\]
  ce qui est possible parce que la sorte $fc$ n'a pas été mise à jour avant le jeu
  de l'action $\PHfrappe{fc_{10}}{a_0}{a_1}$.
\end{example}

\begin{comment}
\subsection{Modelling cooperation}
\label{ssec:cooperation}

Cooperation between processes to make another process bounce can be expressed in PH by building a \emph{cooperative sort}, as described in \cite{PMR10-TCSB}.
\pref{fig:ph-livelock} shows an example of cooperation between processes $a_1$ and $b_1$ to make $c_0$ bounce to $c_1$:
a cooperative sort $ab$ is defined with 4 processes (one for each sub-state of the presence of processes $a_1$ and $b_1$).
For the sake of clarity, the processes of $ab$ are indexed using the sub-state they represent.
Hence, $ab_{10}$ represents the sub-state $\PHstate{a_1,b_0}$, and so on.
Each process of sort $a$ and $b$ hit $ab$ to make it bounce to the process reflecting the status of the sorts $a$ and $b$
(\eg $\PHfrappe{a_1}{ab_{00}}{ab_{10}}$ and $\PHfrappe{a_1}{ab_{01}}{ab_{11}}$).
Then, to represent the cooperation between $a_1$ and $b_1$, the process $ab_{11}$ hits $c_0$ to make it bounce to $c_1$ instead of independent hits from $a_1$ and $b_1$.

We note that cooperative sorts are standard PH sorts and do not involve any
special treatment regarding the semantics of related actions.
Furthermore, it is possible to “factorise” cooperative sorts in order to decrease the number of processes created within each cooperative sort.
For example, if three processes $x_1$, $y_1$ and $z_1$ cooperate,
it is preferable to create a cooperative sort $xy$ with 4 processes to state the presence of $x_1$ and $y_1$
and a second cooperative sort $xyz$ with 4 processes to state the presence of $xy_{11}$ and $z_1$,
rather than a unique cooperative sort with 8 processes stating the presence of $x_1$, $y_1$ and $z_1$.
This “factorisation” allows to prevent the combinatorial explosion of the number of processes in cooperative sorts,
especially for cooperations between more than three processes.
It may have computational consequences as the static analysis method developed in~\pref{sec:sa} does not suffer from the number of sorts but from the number of processes in each sort.

The construction of cooperation in PH allows to encode any Boolean function between cooperating processes \cite{PMR10-TCSB}.
Due to the introduction of priorities into the PH framework,
it is possible to build cooperations with no temporal shift by defining actions updating the cooperative sorts with the highest class of priority.
This allows to gain the same expressivity in PH than in Boolean networks, as stated in \pref{sec:dn}.
The aim of this paper is to allow the static analysis of the dynamics to be handled on PH models involving such prioritised actions updating cooperative sorts.
\end{comment}



\subsection{Analyse statique}
\seclabel{ph-as-pf}

\todo{Rappels sur les résultats d'analyse statique (cf. MSCS'10)}

\todo{+ Points fixes}



\subsection{Analyse stochastique}
\seclabel{ph-stocha}

\todo{Rappels rapides ?}



% Enrichissement des Frappes de Processus pour l'aide à la modélisation

\chapter{Enrichissement des Frappes de Processus pour l'aide à la modélisation}
\chaplabel{sem}
\chaplabel{semantiques}

\chapeau{%
  La sémantique standard des Frappes de Processus de la \secref{ph} peut s'avérer insuffisante pour
  prendre en compte certaines informations connues sur le système étudié, comme des informations
  en terme de vitesse de réaction.
  De plus, certains comportement non désirés apparaissent dans les Frappes de Processus standards
  dès que l'on cherche à synchroniser plusieurs processus.
  Ce chapitre propose d'étendre sa sémantique afin de pallier ces problèmes
  en enrichissant les modèles sur deux axes :
  \begin{itemize}
    \item la préemption entre actions, qui permet d'empêcher le jeu d'une action sous certaines
      conditions,
    \item la simultanéité d'actions, qui permet de forcer le jeu simultané de plusieurs actions.
  \end{itemize}
  Ces deux axes ont pour but de permettre ou de faciliter l'intégration de telles informations
  au sein des modèles.

  Du premier axe découlent deux extensions aux Frappes de Processus
  prenant la forme d'arcs neutralisants, qui modélisent la préemption d'une action par une autre,
  et de classes de priorités, qui modélisent la préemption d'un ensemble d'actions par un autre.
  Le second axe apporte à une troisième extension, reposant sur la notion d'actions plurielles.
%   Les trois extensions aux Frappes de Processus proposées dans ce chapitre prennent la forme
%   de classes de priorités, d'arcs neutralisants et d'actions plurielles.
  Après les avoir définis définies dans ce chapitre,
  nous montrons que ces trois modélisations sont (faiblement) bisimilaires.
  De plus, il est à noter que l'analyse statique développée au \chapref{phcanonique}
  peut s'appliquer à toutes ces extensions, moyennant une traduction.
%   Les classes de priorités sont d'ailleurs réutilisées
%   pour définir la forme canonique des Frappes de Processus.
}

Nous présentons dans ce chapitre les trois sémantiques de Frappes de Processus développées
pour enrichir l'expressivité de ce formalisme.
Nous les définissons et discutons de leurs avantages en termes de modélisation
pour les réseaux de régulation biologiques ou les réseaux de réactions biochimiques.
De plus, nous traçons des liens formels entre ces différents formalismes
afin de mieux comprendre leur complémentarité
et d'offrir une bonne souplesse de représentation et d'analyse.

\myskip

Nous proposons dans le présent chapitre des outils permettant
d'enrichir un modèle de Frappes de Processus à l'aide de
contraintes dérivées d'informations biologiques
comme les vitesses de réaction,
la connaissance d'inhibitions de réactions en présence de certains composants,
ou celle des réactions précises ayant lieu au sein du système, etc.
La connaissance de telles informations peut permettre de privilégier un chemin
sur un autre
à d'un moment clef de l'évolution du système, empêchant ainsi une certaine évolution du système,
en favorisant l'apparition d'une autre.
Leur intégration permet donc d'affiner le modèle en réduisant les comportements possibles
afin d'obtenir un modèle plus proche du système étudié.
Ces connaissances peuvent être intégrées sous la forme de préemptions (la jouabilité d'une action
peut empêcher la jouabilité d'une autre action) ou de simultanéité
(plusieurs processus peuvent évoluer simultanément).

Pourtant, la sémantique standard des Frappes de Processus
développée par \citeasnoun{PMR10-TCSB}
et rappelée à la \vsecref{ph}
ne permet pas de concilier l'introduction de contraintes dérivées d'informations biologiques
et une bonne capacité d'analyse des modèles ainsi créés.
En effet, s'il est possible d'y intégrer des informations temporelles
sous la forme de paramètres stochastiques, tel que mentionné à la \vsecref{ph-stocha},
en revanche les analyses puissantes de la dynamiques rappelées à la \vsecref{ph-as-pf}
ne sont alors plus valables.
En effet, celles-ci ne prennent pas en compte les fenêtres de tir introduites par les
paramètres stochastiques.
L'analyse des modèles doit alors être effectuée à l'aide d'outils de model checking
probabilistes, qui doivent faire face à l'explosion combinatoire provoquée par l'ajout
de la dimension temporelle continue.
Il n'est alors généralement plus possible d'envisager d'étudier
les modèles de plus de cinq composants avec une précision acceptable
\cite[p.~170]{Pauleve11}.

De plus, au niveau de la modélisation, la représentation des coopérations avec des Frappes de
Processus standards souffre de certaines lacunes.
Ces coopérations sont modélisées à l'aide de sortes coopératives, décrites à la \secref{sc},
%dont la dynamique n'est pas strictement équivalente à celle d'une porte logique autorisant
%le jeu d'une action en présence de certains processus donnés.
et souffrent d'un décalage temporel entre les sortes à représenter et
la mise à jour du processus actif de la sortie coopérative,
qui peut entraîner l'existence de «~faux états~» pour la sorte coopérative.
%Ce décalage temporel implique que ce processus actif représente toujours
%une combinaison d'états passés des sortes en amont.
%S'il s'avère que cette combinaison coïncide dans la majeure partie des cas avec un état passé
%effectif,
%ou avec l'état présent, cela n'est pas toujours le cas, et il est possible qu'une sorte coopérative
%représente une combinaison d'états en pratique inaccessible pour les sortes en amont.
Plusieurs solutions sont proposées dans ce chapitre, qu'il s'agisse de rendre les actions de mise à
jour des sortes coopératives prioritaires ou plus simplement de les remplacer par une forme plus
complexe d'action.

L'une des pistes permettant d'affiner un modèle de Frappes de Processus consiste à y intégrer
des informations de préemption entre les actions afin d'affiner la dynamique.
Une telle approche permet de modéliser des contraintes temporelles,
toute action modélisant une réaction très rapide étant par exemple systématiquement jouée
avant les actions modélisant des réactions plus lentes.
D'autres contraintes peuvent aussi être prises en compte, comme la concurrence entre réactions,
mais aussi la représentation de processus propres à la modélisation et n'ayant pas nécessairement
de sens biologique.

Les formes alternatives de Frappes de Processus présentées dans ce chapitre se concentrent donc
sur les notions de préemption et de simultanéité d'une action par rapport à une autre.
La préemption permet à une action d'avoir priorité sur une autre ou, du point de vue inverse,
permet d'empêcher le jeu d'une action dans une situation où elle pourrait normalement être jouable
selon la dynamique des Frappes de Processus standards.
Une telle préemption peut être opérée de façon généralisée
%par une action sur un ensemble d'autres,
comme c'est le cas pour les Frappes de Processus avec classes de priorités (\secref{php}),
où chaque action peut bloquer l'ensemble des actions de priorité inférieure ;
ou de façon ponctuelle, comme au sein des Frappes de Processus avec arcs neutralisants
(\secref{phan}),
qui permet de définir des relations plus fines de préemption entre actions individuelles.
À l'inverse, la simultanéité entre actions permet de s'assurer qu'un ensemble de frappes est joué
de façon simultanée, ou plus généralement qu'un ensemble de processus bondit en même temps,
comme permettent de le représenter les Frappes de Processus avec actions plurielles (\secref{phm}).
Nous montrons enfin au cours de ce chapitre que ces différentes sémantiques sont deux à deux aussi
expressives.

Les apports de ces formes alternatives de Frappes de Processus permettent de restreindre
la dynamique
d'un modèle par rapport aux Frappes de Processus standards.
Elles se posent en alternatives à l'ajout de paramètres stochastiques dans les Frappes de Processus
(cf. \vsecref{ph-stocha})
qui permettent d'ajouter une dimension probabiliste dans ce formalisme.
Leur principal atout est de renforcer la puissance d'expression des Frappes de Processus,
ce qui a pour conséquence de simplifier l'écriture et la lecture des modèles,
mais aussi d'offrir de nouvelles capacités de modélisation.
%selon les informations que l'on souhaite y intégrer.
Par ailleurs,
ces différents formalismes sont tous compatibles avec les méthodes d'analyse statique
développées au \chapref{phcanonique},
ce qui assure de pouvoir étudier efficacement la dynamique des modèles créés.

Ces différentes notions font naturellement écho aux problématiques plus générales d'enrichissement
dans les modèles discrets.
Ainsi, on peut par exemple rapprocher la notion d'arc neutralisant des Frappes de Processus
à celle propre aux réseaux de Petri,
à la différence que celle-ci permet de préempter des actions en fonction de l'activité
d'une place (correspondant ici à un processus ou à une sorte, selon le point de vue).
De même, la notion de priorités est aussi présente dans certaines sémantiques de réseaux
de Petri \toref.
Enfin, les actions plurielles permet de se rapprocher de la classe des modèles
à dynamique synchrone \todo{Biocham, graphes des états...}.

La définition du formalisme des Frappes de Processus avec classes de priorités
a été publiée dans \cite{FPMR13-CS2Bio}.



% Frappes de Processus avec classes de priorités
\section{Frappes de Processus avec classes de priorités}
\seclabel{php}

Pour tout entier naturel $k$ non nul,
les \emph{Frappes de Processus avec $k$ classes de priorités}
%(aussi appelées «~Frappes de Processus~» dans la suite, lorsque ce n'est pas ambigu),
sont des Frappes de Processus dont l'ensemble des actions est partitionné
en $k$ ensembles, chacun étant associé à une classe de priorité distincte.
Cela signifie qu'une action est jouable dans un état si et seulement si,
en plus de la condition de la présence du frappeur et de la cible,
il n'existe aucune autre action appartenant à une classe de priorité plus grande
qui soit aussi jouable dans cet état.
Il est à noter que les classes de priorités sont étiquetées de façon décroissante par les entiers de
l'ensemble $\segm{1}{k}$ en fonction de l'importance de la priorité ;
autrement dit, la classe de priorités 1 contient les actions les plus prioritaires,
ne pouvant jamais être préemptées,
tandis que la jouabilité d'une action de la classe de priorité $k$ ne peut pas empêcher le jeu
d'une autre action.
Un exemple de ce type de modèle est donné par la \figref{metazoan-php},
où les différentes priorités sont signifiées par des étiquettes numérotées sur les actions.

Cette modélisation permet notamment de distinguer les actions en fonction 
de différents critères comme
leur vitesse d'exécution (les actions les plus rapides étant jouées en priorité),
ou tout autre paramètre permettant de déterminer l'existence de la préemption d'une action
en fonction de la possibilité d'en jouer une autre.
L'application la plus poussée de cette utilisation consisterait à
classer les actions d'un modèle en fonction d'un tel critère,
et à attribuer à chaque classe de priorité une action unique en fonction de ce classement.
De cette manière,
les actions seraient arrangées selon un ordre total défini par leurs priorités.
%afin de rendre compte de la priorité de chaque action en fonction de chaque autre.

De même, ces classes de priorités permettent de prendre en compte des comportements non biologiques
inhérents à la modélisation.
Il est par exemple possible de donner une priorité différente aux
actions qui n'ont pas de sens biologique propre
---~mais dont ce sens émerge uniquement dans leur relation avec d'autres actions.
L'application la plus immédiate de ce cas est celle des actions de mise à jour
des sortes coopératives, comme cela est développé au \chapref{phcanonique},
où une classe de priorités supérieure offre l'avantage de supprimer les effets d'entrelacement,
et ainsi de simuler le comportement d'une véritable porte logique
sans le problème de décalage temporel soulevé à la \secref{sc}.

Cette représentation basée sur des classes de priorités permet de modéliser un système
dont les actions peuvent être distinguées en plusieurs classes en fonction de leur importance,
de leur vitesse d'exécution, ou encore d'autres facteurs leur donnant prévalence sur d'autres.
Chaque action peut donc en préempter un ensemble d'autres en fonction de sa classe de priorité.
Cela permet une représentation compacte des rapports de priorités entre actions
ou, autrement dit, de leur ordonnancement,
qui présente néanmoins quelques lacunes.
Les phénomènes d'accumulation, notamment, n'y sont pas représentés ;
un cycle d'actions prioritaires ne peut jamais être interrompu par une action moins prioritaire,
menant à un cycle infini et pouvant contredire la réalité biologique.
De plus, les classes de priorités définies pour un modèle sont invariables;
certains modèles pourraient cependant nécessiter l'évolution de certaines classes de priorités
en fonction de la présence ou de l'absence d'un composant dans un état donné.
Enfin, elles peuvent ne pas permettre la précision nécessaire à une représentation fidèle de
certains modèles, notamment lorsqu'il est nécessaire de définir des préemptions ponctuelles
comme le permettent les Frappes de Processus avec arcs neutralisants
présentées à la \secref{phan}.



\subsection{Définition}
\seclabel{php-def}

\begin{definition}[Frappes de Processus avec $k$ classes de priorités]
\deflabel{php}
  Si $k \in \sN^*$, les \emph{Frappes de Processus avec $k$ classes de priorités} sont définies
  par un triplet $\PH = (\PHs; \PHl; \PHh^{\langle k \rangle})$,
  où $\PHh^{\langle k \rangle} = (\PHh^{(1)}; \dots; \PHh^{(k)})$ est un $k$-uplet, et :
  \begin{itemize}
    \item $\PHs \DEF \{a, b, \dots\}$ est l'ensemble fini et dénombrable des \emph{sortes} ;
    \item $\PHl \DEF \bigtimes{a \in \PHs} \PHl_a$ est l'ensemble fini des \emph{états},
      où $\PHl_a = \{a_0, \ldots, a_{l_a}\}$ est l'ensemble fini et dénombrable
      des \emph{processus} de la sorte $a \in \PHs$ et $l_a \in \sN^*$.
      Chaque processus appartient à une unique sorte :
      $\forall (a_i; b_j) \in \PHl_a \times \PHl_b, a \neq b \Rightarrow a_i \neq b_j$ ;
    \item pour tout $n \in \llbracket 1; k \rrbracket$,
      $\PHh^{(n)} \DEF \{\PHfrappe{a_i}{b_j}{b_l} \mid (a; b) \in \PHs^2 \wedge
      (a_i; b_j; b_l) \in \PHl_a \times \PHl_b \times \PHl_b \wedge
      b_j \neq b_l \wedge a = b \Rightarrow a_i = b_j \}$ est l'ensemble fini
      des \emph{actions de priorité $n$}.
  \end{itemize}
  On note $\PHh \DEF \bigcup_{n \in \segm{1}{k}} \PHh^{(n)}$ l'ensemble de toutes les actions
  et, pour tout $n \in \sN^*$ et $h \in \PHh^{(n)}$, $\prio(h) \DEF n$.
\end{definition}
%
\noindent
On réutilise de surcroît les notations définies à la \secref{ph} concernant les états et
l'extraction de la sorte d'un processus.

%La sorte d'un processus $a_i$ est donnée par $\PHsort(a_i) = a$.
%Étant donné un état $s \in \PHl$, le processus de la sorte $a \in \PHs$ présent dans $s$ est donné
%par $\PHget{s}{a}$, \cad la coordonnée correspondant à $a$ dans l'état $s$.
%Si $a_i \in \PHl_a$, nous définissons la notation : $a_i \in s \EQDEF \PHget{s}{a} = a_i$.

À l'instar de la \secref{ph}, il faut définir un opérateur de jouabilité
pour déterminer la dynamique des Frappes de Processus avec $k$ classes de priorités.
Cependant, à l'inverse de celui des Frappes de Processus standards (\defref{fopph})
il faut ici prendre en compte la possible présence d'actions jouables appartenant à des
classes de priorités supérieures.
Pour cela, il est suffisant de vérifier que le frappeur et la cible de toute action
de priorité plus importante ne sont pas simultanément présents.
En effet, prenons deux actions $g, h \in \PHh$ avec : $\prio(g) < \prio(h)$, et
un état $s \in \PHl$ tel que $\frappeur{g} \in s \wedge \cible{g} \in s$ ;
Deux cas de figures sont alors possibles :
\begin{itemize}
  \item l'action $g$ est jouable dans $s$ --- autrement dit, aucune autre action de priorité plus
    importante ne la préempte -- et elle préempte $h$ en conséquence,
  \item l'action $g$ n'est pas jouable dans $s$, ce qui signifie qu'elle est préemptée par une
    action de priorité plus importante, qui préempte alors aussi l'action $h$.
\end{itemize}
Dans les deux cas, $h$ n'est pas jouable, ce qui montre qu'il est suffisant de n'observer
que la présence simultanée du frappeur et de la cible de chaque action de priorité supérieure
pour déterminer la jouabilité de $h$.
Nous obtenons alors l'opérateur de jouabilité donné à la \defref{fopphp}.

\begin{definition}[Opérateur de jouabilité ($\Fopsymbol_\Fopphpsubsymbol : \PHh \rightarrow \F$)]
\deflabel{fopphp}
  L'opérateur de jouabilité des Frappes de Processus avec $k$ classes de priorités est défini par :
  \[\forall h \in \PHh, \Fopphp{h} \equiv \hitter{h} \wedge \target{h} \wedge
    \left( \bigwedge_{\substack{g \in \PHh^{(n)}\\n < \prio(h)}}
    \neg \left( \hitter{g} \wedge \target{g} \right) \right)\]
\end{definition}



\begin{example}
\exlabel{metazoan-php}
  Nous illustrons les possibilités offertes par l'introduction des classes de priorités
  par la \figref{metazoan-php} qui représente un modèle de Frappes de Processus
  avec 3 classes de priorités $\PH = (\PHs, \PHl, \PHh^{\angles{3}})$.
  Celui-ci reprend la structure du modèle de Frappes de Processus standards
  de la \figref{metazoan-ph},
  en y ajoutant trois classes de priorités permettant de distinguer trois types d'actions :
  \begin{itemize}
    \item les actions de $\PHh^{(1)}$ permettent d'assigner une priorité maximale
      aux actions mettant à jour la sorte coopérative $fc$,
      et peuvent être considérées comme «~instantanées~» du point de vue du reste du modèle ;
    \item les actions de $\PHh^{(2)}$ assurent que la sorte $a$ est mise à jour immédiatement
      en fonction de l'évolution de $f$ et $c$,
      et peuvent être vues comme «~urgentes~» par rapport aux actions de $\PHh^{(3)}$ ;
    \item enfin, les actions restantes sont par conséquent considérées comme «~lentes~»
      ou «~peu urgentes~» en regard du reste du modèle ;
      il s'agit des actions de $\PHh^{(3)}$, qui représentent des processus biologiques
      plus lents.
  \end{itemize}
  Comme expliqué par la suite au \chapref{phcanonique},
  assigner la priorité maximale aux actions permettant la mise à jour des
  sortes coopératives permet d'éviter les comportements indésirables
  décrits à la page \expageref{metazoan-ph-nocoop}.
  De même, accorder aux actions de $\PHh^{(2)}$ le statut d'«~urgentes~»
  permet de s'assurer qu'elles seront jouées avant les actions de $\PHh^{(3)}$.
  Dans ce modèle, cela se traduit par le fait que l'activation ou la désactivation de $a$
  est forcée lorsque $c$ et $f$ évoluent,
  ce qui restreint la dynamique aux seuls comportements désirés.
  En effet, la seule dynamique possible, en partant de l'état initial
  $\etat{a_0, c_0, f_1, fc_{10}}$,
  consiste en un comportement stationnaire oscillant,
  où $c$ et $a$ sont alternativement activés et désactivés,
  interrompu par la désactivation de $f$ qui entraîne irrémédiablement celle de $c$,
  sans possibilité de le ré-activer par la suite,
  et provoque cette fois un comportement stationnaire constant
  (qui se traduit en Frappes de Processus par un point fixe
  où $a$ reste indéfiniment à sa dernière valeur (actif ou non).
  Le comportement stationnaire est donné par le scénario suivant,
  jouable dans l'état initial $\etat{a_0, c_0, f_1, fc_{10}}$ :
  \begin{align*}
    \PHfrappe{fc_{10}}{a_0}{a_1} &\cons
    \PHfrappe{f_1}{c_0}{c_1} \cons
    \PHfrappe{c_1}{fc_{10}}{fc_{11}} \cons \\
    &\PHfrappe{c_1}{a_1}{a_0} \cons
    \PHfrappe{c_1}{c_1}{c_0} \cons
    \PHfrappe{c_0}{fc_{11}}{fc_{10}}
  \end{align*}
  L'interruption de ce comportement stationnaire se fait grâce à l'auto-action
  $\PHfrappe{f_1}{f_1}{f_0}$, qui est de priorité 3, et donc jouable uniquement
  dans les deux états suivants :
  $\etat{a_1, c_0, f_1, fc_{10}}$ et $\etat{a_0, c_1, f_1, fc_{11}}$.
  Depuis le premier état, la désactivation est opérée par le scénario suivant :
  \[
    \PHfrappe{f_1}{f_1}{f_0} \cons
    \PHfrappe{f_0}{fc_{10}}{fc_{00}}
    \enspace,
  \]
  qui termine dans l'état $\etat{a_1, c_0, f_0, fc_{00}}$ et conserve donc le processus $a_1$,
  tandis que depuis le deuxième état, la désactivation est opérée par le scénario :
  \[
    \PHfrappe{f_1}{f_1}{f_0} \cons
    \PHfrappe{f_0}{fc_{11}}{fc_{01}} \cons
    \PHfrappe{f_0}{c_1}{c_0} \cons
    \PHfrappe{c_0}{fc_{01}}{fc_{00}}
    \enspace,
  \]
  qui termine en $\etat{a_0, c_0, f_0, fc_{00}}$ et conserve cette fois le processus $a_0$.
  
  \begin{figure}[ht]
  \begin{center}
  \begin{tikzpicture}
    \exmetazoan
    
    \node[labelprio1] at (2.55,3.85) {$1$}; % c => fc
    \node[labelprio1] at (2.75,1) {$1$};    % f => fc
    \node[labelprio2] at (5.5,3.85) {$2$};  % fc_10 -> a_0 / 1
    \node[labelprio2] at (3.5,5.3) {$2$};   % c_1 -> a_1 / 0
    \node[labelprio3] at (0,2.5) {$3$};     % f_1 -> c_0 / 1
    \node[labelprio3] at (0.8,5.8) {$3$};   % c_1 -> c_1 / 0
    \node[labelprio3] at (2.15,2.5) {$3$};  % f_0 -> c_1 / 0
    \node[labelprio3] at (1.5,1.8) {$3$};   % f_1 -> f_1 / 0
    
    \TState{f_1, a_0, c_0, fc_2}
  \end{tikzpicture}
  \caption{\figlabel{metazoan-php}%
    Exemple de Frappes de Processus avec 3 classes de priorités.
    Ce modèle est issu de celui de la \figref{metazoan-ph}
    auquel ont été rajoutées des classes de priorités.
    Les étiquettes numérotées (de 1 à 3) placées contre les flèches représentant les actions
    symbolisent leur appartenance à une classe de priorités donnée ;
    ainsi, on a notamment :
    $\PHh^{(2)} = \{ \PHfrappe{fc_{10}}{a_0}{a_1} ; \PHfrappe{c_1}{a_1}{a_0} \}$.
  }
  \end{center}
  \end{figure}
\end{example}



\subsection{Équivalences entre Frappes de Processus avec $k$ classes de priorités}

Nous montrons à la \secref{aplatissement} que les Frappes de Processus avec $k$ classes
de priorités sont aussi expressives que les Frappes de Processus avec $n$ classes de
priorités, pour tout $k, n \in \sNN$.
Nous donnons pour cela un résultat encore plus fort : tout modèle de Frappes de Processus avec $k$
classes de priorités peut être traduit en Frappes de Processus canoniques,
comme défini à la \secref{phcanonique-def},
qui sont des Frappes de Processus avec 2 classes de priorités
avec une forme particulière.
À l'inverse, les Frappes de Processus avec 2 classes de priorités sont
naturellement à fortiori
des Frappes de Processus avec $k$ classes de priorités, pour tout $k \in \sNN$.

Nous notons cependant que ce résultat n'inclut pas les Frappes de Processus
avec 1 classe de priorité (c'est-à-dire les Frappes de Processus standards).
En effet, il est intuité que leur expressivité est strictement moindre
que les Frappes de Processus possédant plusieurs classes de priorités,
mais nous ne démontrons pas ce résultat ici.

\begin{theorem}[Équivalences entres Frappes de Processus avec classes de priorités]
\thmlabel{equivphpkn}
  Pour tout $k, n \in \sNN$, les Frappes de Processus avec $k$ classes de priorités
  sont aussi expressives que les Frappes de Processus avec $n$ classes de priorités.
\end{theorem}

\begin{proof}
  Nous utilisons pour cette démonstration
  l'opérateur d'aplatissement $\PHflat$ donné à la \defref[vref]{fopaplat}
  et les résultats qui le concernent.
  Soient $k, n \in \sNN$,
  et soient $\PH$ des Frappes de Processus avec $k$ classes de priorités.
  D'après le \thmref{bisimulaplatissement},
  l'aplatissement $\PHflat(\PH)$ est faiblement bisimilaire à $\PH$.
  Posons :
  \begin{itemize}
    \item $\PHflat(\PH) = (\PHs; \PHl; \PHh^{\angles{2}})$
      avec $\PHh^{\angles{2}} = (\PHh^{(1)}; \PHh^{(2)})$,
    \item $\PH' = (\PHs; \PHl; \PHh'^{\angles{n}})$
      avec $\PHh'^{\angles{n}} = (\PHh^{(1)}; \PHh^{(2)}; \PHh'^{(3)}; \ldots; \PHh'^{(n)})$, \\
      \hspace*{1em} où $\forall i \in \segm{3}{n}, \PHh'^{(i)} = \emptyset$.
  \end{itemize}
  Autrement dit, $\PH'$ sont les Frappes de Processus avec $n$ classes de priorités
  identiques à $\PHflat(\PH)$, où des classes de priorités vides ont été
  artificiellement ajoutées.
  Cela est possible car $n \geq 2$.
  Ainsi, $\PH'$ possède la même dynamique que $\PHflat(\PH)$,
  et est donc faiblement bisimilaire à $\PH$
  (autrement dit, sa dynamique est équivalente au jeu des actions de priorité 1 près).
  Ainsi, pour tout $k, n \in \sNN$,
  toutes Frappes de Processus avec $k$ classes de priorités peuvent être représentées
  en Frappes de Processus avec $n$ classes de priorités.
\end{proof}



\subsection{Réutilisation des résultats existants}
\seclabel{php-outils}

Nous discutons dans cette section de la transposition des différents outils et
résultats concernant les Frappes de Processus standards
aux Frappes de Processus avec classes de priorités.
Nous aborderons la question de la recherche des points fixes d'un modèle (\secref{php-outils-pf}),
de l'analyse statique des modèles (\secref{php-outils-as})
et des possibles (ré)utilisations des paramètres stochastiques (\secref{php-outils-stocha}).

\subsubsection{Points fixes}
\seclabel{php-outils-pf}

Nous montrons dans cette sous-section que les points fixes des Frappes de Processus
avec classes de priorités peuvent être obtenus de façon similaire à ceux
des Frappes de Processus standards.
Pour cela, nous définissons la notion de \emph{fusion} d'un modèle de Frappes de Processus
%avec classes de priorités,
qui consiste à fusionner les classes de priorités en une seule,
afin de retrouver un modèle de Frappes de Processus standards.

\myskip

Pour tout modèle de Frappes de Processus avec $k$ classes de priorités, pour $k \in \sN^*$,
nous notons dans la suite $\PHmerge(\PH)$
la \emph{fusion} de $\PH$,
c'est-à-dire le même modèle dont les classes de priorités ont été fusionnées
(\defref{fusion}).
En d'autres termes, il s'agit d'un modèle de Frappes de Processus standards dont
l'ensemble des actions est l'union de toutes les classes de priorités de $\PH$.

On peut constater que pour un modèle donné, si l'ensemble de toutes les actions reste le même,
ajouter (ou retirer) des classes de priorités à des Frappes de Processus ne change pas
l'ensemble de ses points fixes.
En effet, le \thmref{php-pf} stipule que
l'ensemble des points fixes des Frappes de Processus avec $k$
classes de priorités $\PH$ est identique à l'ensemble des points fixes de
sa fusion $\PHmerge(\PH)$.
Cela se démontre simplement en constatant qu'il existe une action jouable dans un état donné du
modèle $\PH$ si et seulement si il en existe une dans le même état du modèle fusionné.
En effet, si une action est jouable dans un état donné de $\PHmerge(\PH)$,
alors soit elle est jouable car non préemptée dans le même état de $\PH$,
soit elle ne l'est pas car elle est préemptée par une autre action qui, elle, est jouable.
L'autre sens de la démonstration est immédiat car l'ajout de priorités restreint la dynamique
et n'ajoute aucun comportement supplémentaire possible.
Ce résultat permet d'appliquer aux Frappes de Processus avec $k$ classes de priorités
les méthodes de recherche de points fixes développée pour les Frappes de Processus standards.
L'une d'entre elles repose sur la recherche de $n$\nbd cliques (\vsecref{ph-pf})
et sa résolution jouit aujourd'hui de méthodes de résolutions performantes.
D'autres méthodes de recherche peuvent être envisagées,
par exemple par l'utilisation de programmation logique.
En effet, la formalisation du problème de recherche de points fixes est très simple,
et sa résolution est donc traitée efficacement par des méthodes SAT ou par ASP.
Enfin, ce résultat permet aussi de conclure quant aux ensembles de points fixes de deux Frappes
de Processus avec un nombre de classes de priorités différent, à condition que
leurs modèles fusionnés soient identiques.

\begin{definition}[Fusion ($\PHmerge$)]
\deflabel{fusion}
  Soient $\PH = (\PHs, \PHl, \PHh^{\angles{k}})$ des Frappes de Processus avec $k$
  classes de priorités, où $k \in \sN^*$.
  On note $\PHmerge(\PH) = (\PHs, \PHl, \PHh)$
  les Frappes de Processus standards appelées \emph{fusion} de $\PH$,
  dont l'ensemble des actions est l'union de toutes les classes de priorités de $\PH$.
\end{definition}

\begin{theorem}[Points fixes des Frappes de Processus avec classes de priorités]
\thmlabel{php-pf}
  Soient $\PH = (\PHs, \PHl, \PHh^{\angles{k}})$, où $k \in \sN^*$,
  des Frappes de Processus avec $k$ classes de priorités,
%  et $\PH' = (\PHs, \PHl, \PHh')$ les Frappes de Processus standards avec $\PHh' = \PHh$.
  et $r \in \PHl$ :
  \[\exists s \in \PHl, r \trans{\PH} s \Longleftrightarrow
    \exists s' \in \PHl, r \trans{\PHmerge(\PH)} s'\]
\end{theorem}

\begin{proof}
  On pose : $\PH' = \PHmerge(\PH)$.

  ($\Rightarrow$) Supposons qu'il existe $s \in \PHl$ tel que $r \trans{\PH} s$;
    cela signifie qu'une action $h \in \PHh$ est jouable dans $\PH$.
    Cette action est donc aussi jouable dans $\PH'$ car son frappeur et sa cible sont présents,
    d'où : $r \trans{\PH'} (r \play h)$.
    
  ($\Leftarrow$) Supposons qu'il existe $s' \in \PHl$ tel que $r \trans{\PH'} s'$ ;
    cela signifie qu'une action $g \in \PHh$ est jouable dans $\PH'$.
    Le frappeur et la cible de $g$ sont donc présents dans $r$.
    \begin{itemize}
      \item Si cette action n'est pas préemptée par une autre action dans $\PH$,
        elle est alors jouable et $r \trans{\PH} (r \play g)$ ;
      \item Si cette action est préemptée par une autre action $g'$ dans $\PH$,
        cela signifie que cette action $g'$ est jouable, et $r \trans{\PH} (r \play g')$.
        \qedhere
    \end{itemize}
\end{proof}

\subsubsection{Analyse statique}
\seclabel{php-outils-as}

Afin de permettre une étude efficace des Frappes de Processus standards de grande taille,
une analyse statique par interprétation abstraite avait été développée
par \citeasnoun{PMR12-MSCS}.
Son principe est rappelé à la \vsecref{ph-as}.
L'ajout de classes de priorités au formalisme a pour conséquence d'en restreindre la dynamique,
mais n'ajoute aucun comportement supplémentaire.
Ainsi, pour tout modèle de Frappes de Processus avec $k$ classes de priorités,
il est toujours possible de réutiliser l'analyse statique par sur-approximation
en l'appliquant au modèle $\PHmerge(\PH)$.
Bien que toujours exacte, cette analyse pourra néanmoins être moins conclusive,
n'ayant pas été spécifiquement adaptée aux modèles comportant des classes de priorités.

En revanche, l'analyse statique par sous-approximation n'est plus valable,
car elle ne prend pas en compte les possibles préemptions entre actions qui rendent
impossibles certaines atteignabilités.
C'est pourquoi une nouvelle version de l'analyse statique par sous-approximation sera développée
au \chapref{phcanonique}
sur une classe particulière de Frappes de Processus avec 2 classes de priorités, appelées
Frappes de Processus canoniques.
Cette classe particulière n'autorise les actions avec une forte priorité que pour la mise à jour
des sortes coopératives.
Cependant, nous montrons aussi à la \vsecref{phcanonique-equiv}
que toutes Frappes de Processus avec un nombre quelconque de classes de priorités
peuvent être traduites en Frappes de Processus canoniques équivalentes,
étendant ainsi l'analyse statique mentionnée aux modèles de Frappes de Processus
avec $k$ classes de priorités.
%, et nous donnons une telle traduction à la
%\defref{aplatissement}.

\subsubsection{Paramètres stochastiques}
\seclabel{php-outils-stocha}

Il est théoriquement toujours possible d'utiliser,
dans des Frappes de Processus avec $k$ classes de priorités,
des paramètres stochastiques tels que ceux
développés par \citeasnoun{PMR10-TCSB} et mentionnés à la \vsecref{ph-stocha}.
Il suffirait pour cela d'empêcher la sensibilisation de toute action préemptée par une action
de priorité plus importante.
Cependant, un autre parallèle intéressant peut être tracé entre l'approche
par définition de classes de priorités
et l'approche par introduction de paramètres stochastiques.

\myskip

L'ajout de paramètres stochastiques a pour but d'assigner un intervalle de tir temporel à
chaque action, afin de s'assurer (avec un certain niveau de confiance) que l'action sera
nécessairement tirée dans cet intervalle à partir du moment où elle est devenue jouable.
La simulation stochastique développée par \citeasnoun{PMR10-TCSB}
ne permet actuellement pas de prendre en compte des classes de priorités entre actions.
Il faudrait en effet pour cela raffiner la machine stochastique afin d'y intégrer des contraintes
supplémentaires concernant la jouabilité et la sensibilisation de chaque action.
Cependant, l'aplatissement proposé à la \vdefref{aplatissement}
permet théoriquement d'obtenir un modèle équivalent,
utilisable avec la simulation stochastique.
Nous détaillons ici le principe de ce procédé sans toutefois donner de preuve de sa fidélité.

L'aplatissement mentionné ci-dessus permet en effet d'obtenir un modèle de Frappes de Processus
avec 2 classes de priorités ayant une certaine forme qui permet de distinguer
les actions instantanées (de priorité 1) propres à la modélisation
des actions possédant une durée (de priorité 2) et permettant de représenter des processus
biologiques.
On peut ainsi, dans le modèle obtenu, attribuer à chaque action secondaire (\cad de priorité 2)
des paramètres stochastiques identiques à ceux de l'action originelle dont elle est issue,
et à chaque action primaire (\cad de priorité 1) une absorption de stochasticité infinie.
Le modèle résultant devrait alors posséder une dynamique identique à un potentiel modèle hybride
mêlant classes de priorités et paramètres stochastiques.

\myskip

Plutôt que d'intégrer des données stochastiques connues dans un modèle,
il est aussi possible de s'en servir pour l'obtention d'un modèle discret.
En effet, pour créer un modèle de Frappes de Processus avec classes de priorités,
il est nécessaire de connaître certaines relations entre les phénomènes modélisés
afin de répartir correctement les actions entre les différentes classes de priorités.
Il peut s'agit de données de préemption (un phénomène en bloque un autre),
de durée (un phénomène est beaucoup plus rapide qu'un autre),
de vitesse de déclenchement (un phénomène se déclenche toujours avant un autre), etc.
La piste que nous présentons dans la suite est l'utilisation d'un modèle de Frappes de Processus
standards enrichi à l'aide de paramètres stochastiques.
En effet, de tels paramètres stochastiques définis pour chaque action correspondent
à une fenêtre de tir avec un intervalle de confiance donné (généralement 95~\%),
qu'il est possible d'approximer \cite[p.~72]{Pauleve11},
et donc d'utiliser de façon interchangeable.
Ces paramètres stochastiques permettent donc au modélisateur de rendre une action
d'autant plus «~urgente~» que sa fenêtre de tir est proche de sont instant
de sensibilisation.

Ainsi, sous la condition que l'on peut distinguer les intervalles de tir en différentes classes
entre lesquelles les intervalles ne se recouvrent pas,
il est possible d'approximer la modélisation avec des paramètres stochastiques à
l'aide de classes de priorités.
En associant une classe de priorités à chaque ensemble d'intervalles de tir
---~la priorité la plus haute étant naturellement associée à la classe dont les intervalles
sont les plus proches de zéro,~---
on retrouve alors un modèle dont les caractéristiques dynamiques sont proches
du modèle initial.
En effet, le système de classes de priorités permet d'approcher une dynamique où
chaque intervalle est joué en priorité avant tous les suivants.

\begin{example}
\exlabel{paramsstocha-php}
  Le modèle de Frappes de Processus standards de la \vfigref{metazoan-ph}
  peut être enrichi à l'aide des paramètres proposés \vexpageref{metazoan-stocha-params},
  permettant ainsi
  de contraindre (à l'intervalle de confiance près) le jeu des actions entre elles,
  et favorisant le jeu des actions «~urgentes~».
  Cela permet notamment de rendre «~instantanées~»
  les actions de mise à jour des sortes coopératives,
  mais aussi de rendre «~urgentes~» les actions faisant bondir $a$,
  et de réguler l'évolution de $c$ afin de lui donner un rôle d'horloge.
  
  Si on observe les fenêtres de tir définies pour ce modèle, on constate qu'elles
  ne se recouvrent pas ; autrement dit, on peut clairement distinguer des classes
  d'actions qui seront tirées en priorité par rapport à d'autres classes.
  Nous proposons donc de traduire ces classes de fenêtres de tir en classes de priorités
  qui auront le même rôle : empêcher le jeu d'une action tant que d'autres actions
  plus «~urgentes~» sont jouables.
  Nous décidons de distinguer pour cet exemple trois classes de priorités :
  \begin{itemize}
    \item les actions instantanées car de taux infini (\stochainf) feront partie de la classe de
      plus forte priorité,
    \item les actions qui influencent $a$ (\stochaa) peuvent être considérés comme «~urgentes~»,
      et formeront la classe de priorité intermédiaire,
    \item les autres actions (\stochab{} et \stochac) seront regroupées dans la classe de priorité faible,
      car considérées comme «~peu urgentes~».
  \end{itemize}
  Nous obtenons ainsi le modèle donné à la \figref{metazoan-php},
  dont la dynamique a été précédemment discutée \vexpageref{metazoan-php}.
  Nous constatons que la dynamique du modèle en Frappes de Processus avec classes de priorités
  est «~strict~» : il n'y a pas d'intervalle de confiance et plus aucune notion de probabilités.
\end{example}

\myskip

Cependant, il n'est évidemment pas possible d'atteindre
avec cette méthode le même niveau de précision, en utilisant un nombre
discret de classes de priorités,
qu'avec des intervalles de tir définis sur une ligne temporelle continue.
Il n'est par exemple
pas possible de représenter fidèlement le recouvrement de deux intervalles de tir,
qui aurait pour conséquence de favoriser le tir d'une première action sans totalement
préempter la seconde action,
autrement qu'en mettant ces deux actions au même niveau de priorité.
À l'inverse, deux intervalles de tir qui ne se recouvrent pas devraient être associés à la même
classe de priorités s'ils sont tous deux recouverts par un troisième intervalle,
ce qui aura pour conséquence de mettre les trois actions sur un pied d'égalité, alors que les
paramètres stochastiques représentent une situation différente.

%\begin{example}
  \todoplustard{Exemple de trois intervalles qui se recouvrent et impossibles à traduire fidèlement en PHp ?}
%\end{example}

Enfin, il faut noter que
les phénomènes d'accumulation ou de retard ne sont pas pris en compte dans la modélisation
par classes de priorités.
En effet, si deux actions n'ont pas de sorte en commun, elles devraient dans l'idéal pouvoir
évoluer de façon indépendante, ce qui est notamment permis par la simulation stochastique.
En revanche, dans un formalisme avec classes de priorités, si l'une des actions est plus
prioritaire que l'autre, elle exercera tout de même sa préemption sur l'autre.
L'une des façons de pallier ce défaut est l'utilisation d'arcs neutralisants,
comme développé à la \v secref{phan}.

\begin{example}
\exlabel{paramsstocha-php-accumulation}
  L'exemple de la \vfigref{metazoan-php}
  peut être obtenu à l'aide de paramètres stochastiques,
  comme expliqué \vexpageref{paramsstocha-php}.
  Cependant, cette approche possède certaines limites ;
  il n'est pas possible par exemple de n'autoriser la désactivation de $f$
  qu'après un nombre donné d'oscillations de $a$ ou de $c$.
  Pour représenter cela, un modélisateur pourrait être tenté d'introduire une classe de
  priorité $4$ afin d'y intégrer l'action $\PHfrappe{f_1}{f_1}{f_0}$,
  en effectuant un parallèle avec les paramètres stochastiques proposés
  \vexpageref{metazoan-stocha-params}
  qui permettent de ne tirer cette action qu'après un certain nombre d'oscillations ;
  ou encore en constatant que les paramètres de cette action devraient permettre
  la création d'une telle classe de priorités, car son intervalle de tir n'en recouvre
  aucun autre.
  Cependant, un tel choix de conception aurait uniquement pour effet de rendre ladite action
  injouable, car sans cesse préemptée par des actions de priorité supérieure.
  En effet, une telle modélisation en Frappes de Processus avec $4$ classes de priorités
  fait abstraction du temps continu,
  ce qui signifie qu'il n'y a plus de notion d'«~accumulation~» du temps de sensibilisation,
  sur laquelle reposait le fait de pouvoir jouer l'action en question après une certaine
  durée de sensibilisation.
\end{example}

Nous avons ici montré l'une des limites de la modélisation par
Frappes de Processus avec classes de priorités, qui fait abstraction du temps
continu sur lequel se basent les paramètres stochastiques proposés par \citeasnoun{PMR10-TCSB}.
Le modélisateur doit effectivement être prudent pour ne pas rendre impossibles des
comportements qui sont seulement retardés par des modélisations chronométriques.
Cependant, à condition d'éviter ces écueils, nous avons montré qu'il est possible
de représenter un modèle avec des données temporelles à l'aide de ce formalisme,
et donc d'en étudier efficacement la dynamique à l'aide
de la traduction et des méthodes d'analyse statique proposés au \chapref{as}.


% Frappes de Processus avec arcs neutralisants
\section{Frappes de Processus avec arcs neutralisants}
\seclabel{phan}

Nous introduisons ici la notion d'\emph{arc neutralisant} dans les Frappes de Processus
afin de représenter la préemption d'une action par une seule autre.
Les \emph{Frappes de Processus avec arcs neutralisants} (\defref{php})
permettent notamment une modélisation plus atomique
par rapport aux classes de priorités présentées à la \secref{php}.

Un arc neutralisant est un couple d'actions noté $\PHan{h_1}{h_2}$,
où $h_1$ est appelée \emph{action bloquante},
et peut préempter $h_2$, appelée \emph{action bloquée},
dans certaines situations.
Avec la présence d'arcs neutralisants, une action est dite \emph{activée} dans un état donné si
son frappeur et sa cible y sont présents ;
une action est donc activée pour les Frappes de Processus avec arcs neutralisants
là où elle était immédiatement jouable pour les Frappes de Processus standards (\defref{fopph}).
Une action est \emph{jouable} pour les Frappes de Processus avec arcs neutralisants
si et seulement si elle est activée,
et que pour tout arc neutralisant la bloquant, son action bloquante n'est pas activée.
Une action activée mais qui n'est pas jouable est dite \emph{neutralisée}.

Il est à noter que la neutralisation d'une action par une autre ne dépend donc pas de la jouabilité
de l'action bloquante, mais uniquement de son activation.
Cela permet d'avoir un modèle cohérent, sans quoi certaines situations pourraient ne pas être
définies, notamment dans le cas d'un interblocage.
Ainsi, faire reposer la neutralisation d'une action bloquée sur la jouabilité de l'action bloquante
devient inextricable dans un cas comme le suivant :
$\PHan{h_1}{h_2}$, $\PHan{h_2}{h_3}$ et $\PHan{h_3}{h_1}$,
car si les trois actions $h1$, $h2$ et $h3$ sont actives, leur jouabilité reste indéterminée.
En revanche, si cette neutralisation repose sur l'état activé d'une action,
la situation précédente se résout immédiatement car aucune des trois actions n'est jouable.
On constate par ailleurs qu'une action peut en neutraliser une autre
même si elle-même est neutralisée.
Nous ne nous avancerons cependant pas sur la signification biologique de ce fait.

Enfin, il semble nécessaire de faire un parallèle entre les arcs inhibiteurs développés ici
et les arcs du même nom utilisés dans les réseaux de Petri \toref.
Leur rôle est effectivement proche, leur but étant la préemption d'une action en fonction
d'une condition extérieure à son déclenchement.
Malgré tout, les arcs inhibiteurs des réseaux de Petri se différencient car ils
reposent sur la présence d'un certain nombre de jetons dans une place
---~ce qui se traduirait, en Frappes de Processus,
par la présence d'un certain processus actif d'une sorte donnée~---
là où les arcs inhibiteurs des Frappes de Processus ne permettent l'inhibition
qu'en fonction de l'activité d'une autre action.
Cependant, un tel choix de conception peut aisément se pallier dans un sens comme dans l'autre.
En effet, représenter l'activité d'une action revient à créer deux arcs inhibiteurs en
réseaux de Petri ---~l'un pour le frappeur et l'autre pour la cible.
À l'inverse, il est possible qu'une action $h$ se bloque elle-même, à l'aide d'un 
«~auto-arc neutralisant~» $\PHan{h}{h}$,
permettant ainsi de bloquer une autre action $g$
en fonction de la présence du processus $\frappeur{h}$
à l'aide d'un deuxième arc neutralisant $\PHan{h}{g}$.

\todo{Schéma exemple}

\subsection{Définition}
\seclabel{phan-def}

\begin{definition}[Frappes de Processus avec arcs neutralisants]
\deflabel{phan}
  Les \emph{Frappes de Processus avec arcs neutralisants} sont définies par
  un quadruplet $\PH = (\PHs ; \PHl ; \PHh ; \PHn)$, où :
  \begin{itemize}
    \item $\PHs \DEF \{a, b, \dots\}$ est l'ensemble fini et dénombrable des \emph{sortes} ;
    \item $\PHl \DEF \bigtimes{a \in \PHs} \PHl_a$ est l'ensemble fini des \emph{états},
      où $\PHl_a = \{a_0, \ldots, a_{l_a}\}$ est l'ensemble fini et dénombrable
      des \emph{processus} de la sorte $a \in \PHs$ et $l_a \in \sN^*$.
      Chaque processus appartient à une unique sorte :
      $\forall (a_i; b_j) \in \PHl_a \times \PHl_b, a \neq b \Rightarrow a_i \neq b_j$ ;
    \item $\PHh \DEF \{\PHfrappe{a_i}{b_j}{b_l} \mid (a; b) \in \PHs^2 \wedge
      (a_i; b_j; b_l) \in \PHl_a \times \PHl_b \times \PHl_b \wedge
      b_j \neq b_l \wedge a = b \Rightarrow a_i = b_j \}$ est l'ensemble fini des actions ;
    \item $\PHn = \{\PHan{h_1}{h_2} \mid (h_1 ; h_2) \in \PHh \times \PHh\}$
      est l'ensemble fini des arcs neutralisants.
  \end{itemize}
\end{definition}

Un arc neutralisant $u = \PHan{h_1}{h_2} \in \PHn$ est donc un couple d'actions.
On note $\PHbloquant(u) = h_1$ la première action du couple $u$
et $\PHbloque(u) = h_2$ sa seconde action.
On réutilise par ailleurs les autres notations définies à la \secref{ph}.

L'opérateur de jouabilité des frappes de Processus avec arcs neutralisants (\defref{fopphan})
se concentre sur la présence du frappeur et de la cible de l'action considérée,
et sur celle de toutes ses actions bloquantes.
En ce sens, il est semblable à celui des Frappes de Processus avec $k$ classes de priorités
(\defref{fopphp}).

\begin{definition}[Opérateur de jouabilité ($\Fopsymbol_\Fopphansubsymbol : \PHh \rightarrow \F$)]
\deflabel{fopphan}
  L'opérateur de jouabilité des Frappes de Processus avec arcs neutralisants est défini par :
  \[\forall h \in \PHh, \Fopphan{h} \equiv \hitter{h} \wedge \target{h} \wedge
    \left( \bigwedge_{\substack{g \in \PHh\\\PHan{g}{h} \in \PHn}}
    \neg \left( \hitter{g} \wedge \target{g} \right) \right)\]
\end{definition}



\subsection{Équivalence avec les Frappes de Processus avec $k$ classes de priorités}
\seclabel{phan-equiv-php}

Les Frappes de Processus avec arcs neutralisants ont une expressivité équivalente
aux Frappes de Processus avec $k$ classes de priorité, où $k \in \sN^*$.
En effet, la \thmref{phpbisimphan} propose une traduction des Frappes de Processus
avec $k$ classes de priorités en Frappes de Processus avec arcs neutralisants.
À l'inverse, les Frappes de Processus avec arcs neutralisants peuvent être représentée
en Frappes de Processus canoniques, qui sont une sous-classe des Frappes de Processus avec
2 classes de priorités, comme démontré à la \secref{phan-aplatissement}.

\begin{theorem}[Équivalence avec les classes de priorités]
\thmlabel{phpbisimphan}
  Soient $\PH = (\PHs, \PHl, \PHh^{\angles{k}})$ des Frappes de Processus avec $k$ classes
  de priorités, où $k \in \sN^*$.
  Il existe un modèle $\oPH$ de Frappes de Processus avec arcs neutralisants tel que :
  \[\forall s, s' \in \PHl, s \trans{\PH} s' \Longleftrightarrow s \trans{\oPH} s' \]
\end{theorem}

\begin{proof}
  On pose : $\oPH = (\PHs, \PHl, \PHh, \PHn)$ les Frappes de Processus avec arcs neutralisants
  dont l'ensemble des actions est l'union de toutes les classes de priorités de $\PH$, et :
  $\PHn = \{ \PHan{g}{h} \mid g \in \PHh^{(n)}, h \in \PHh^{(k)}, n < k \}$.
  Soit $h \in \PHh$. et
  D'après la définition de $\oPH$, on constate que :
  $\{ g \in \PHh \mid \PHan{g}{h} \in \PHn \} = \{ g \in \PHh \mid n < \prio(h) \}$.
  Ainsi, $\forall h \in \PHh, \Fopphp{h} = \Fopphan{h}$,
  ce qui implique que les action jouables dans l'état $s$ du le modèle $\PH$
  sont exactement les actions jouables dans l'état $s$ du modèle $\oPH$.
  D'où : $s \trans{\PH} s' \Longleftrightarrow s \trans{\oPH} s'$.
\end{proof}



\subsection{Réutilisation des outils existants}
\seclabel{phan-outils}

\todo{Revoir le titre de cette section}

\todo{Faudrait-il fusionner cette section avec ses semblables des \secref{php} et \secref{phm} ?}

\subsubsection{Points fixes}
\seclabel{phan-outils-pf}

De façon analogue à la \secref{php-outils-pf}, nous définissions ici
la \emph{fusion} d'un modèle de Frappes de Processus avec arcs neutralisants
comme étant le même modèle en Frappes de Processus standards,
dont les arcs neutralisants ont été ignorés (\defref{phan-fusion}).
Cette définition nous permet d'avancer le résultat suivant :
les points fixes de $\PHmergean(\PH)$ sont des points fixes de $\PH$,
d'après la contraposée du \thmref{php-pf}.
Il n'est en effet pas possible d'avancer de résultat plus précis de façon immédiate,
car les arcs neutralisants peuvent former des boucles d'actions qui se neutralisent entre elles,
rajoutant ainsi des points fixes supplémentaires par rapport au modèle fusionné.
Ce cas ne se présente pas pour les Frappes de Processus avec $k$ classes de priorités.

\begin{definition}[Fusion ($\PHmergean$)]
\deflabel{phan-fusion}
  Soient $\PH = (\PHs, \PHl, \PHh, \PHn)$ des Frappes de Processus avec arcs neutralisants.
  On note $\PHmergean(\PH) = (\PHs, \PHl, \PHh)$
  les Frappes de Processus standards appelées \emph{fusion} de $\PH$,
  dont on a retiré les arcs neutralisants.
\end{definition}

\begin{theorem}[Points fixes des Frappes de Processus avec arcs neutralisants]
\thmlabel{phan-pf}
  Soient $\PH = (\PHs, \PHl, \PHh, \PHn)$ des Frappes de Processus avec arcs neutralisants,
  et $r \in \PHl$ :
  \[\exists s \in \PHl, r \trans{\PH} s \Longrightarrow
    \exists s' \in \PHl, r \trans{\PHmergean(\PH)} s'\]
\end{theorem}

\begin{proof}
  Si une action est jouable dans $r$, cela signifie notamment
  que son frappeur et sa cible sont présents dans $r$.
  Cette condition est suffisante pour que cette même action soit jouable dans $\PHmergean(\PH)$.
\end{proof}

\todo{Les cycles d'actions sont-ils suffisants pour trouver tous les points fixes ?}

Pur finir, il est toujours envisageable d'aplatir des Frappes de Processus avec
arcs neutralisants en Frappes de Processus canoniques,
comme décrit à la \secref{phan-aplatissement},
afin d'effectuer une recherche de points fixes à l'aide de la méthode
proposée à la \secref{php-outils-pf}.
Comme le modèle aplati de Frappes de Processus avec arcs neutralisants possède la même dynamique,
il est possible de retrouver les points fixes du modèle original de cette façon,
en supprimant simplement les sortes coopératives des résultats.

\subsubsection{Analyse statique}
\seclabel{phan-outils-as}

De même qu'à la \secref{php-outils-as},
étant donné que les Frappes de Processus avec arcs neutralisants permettent de restreindre
la dynamique par rapport aux Frappes de Processus standards, il est toujours possible
d'utiliser l'analyse statique par sur-approximation,
mais l'analyse statique par sous-approximation n'est plus valable.

Cependant, nous proposons au \chapref{phcanonique}
une traduction des Frappes de Processus avec arcs neutralisants en Frappes de Processus canoniques,
qui sont des Frappes de Processus avec 2 classes de priorités avec des contraintes précises,
ainsi qu'une nouvelle approche d'analyse statique par sous-approximation
qui s'applique à cette classe particulière de modèles.
L'analyse statique par sous-approximation est donc possible sur les Frappes de Processus
avec arcs neutralisants, au prix de cette traduction.

\subsubsection{Paramètres stochastiques}

Enfin, à l'instar de la \secref{php-outils-stocha},
il n'est pas possible d'utiliser directement les outils développés pour la simulation stochastique
car ceux-ci ne prennent pas en compte la présence d'arcs neutralisants.
Malgré cela, une telle utilisation reste théoriquement possible à condition de prendre
en compte les neutralisations entre actions.

Cependant, à nouveau, un parallèle intéressant peut être tracé entre les arcs neutralisants
et l'introduction de paramètres stochastiques dans le modèle.
En effet, étant donné un ensemble de paramètres stochastiques,
si les intervalles de tir de deux actions sont disjoints,
alors il est possible de modéliser cela par un arc neutralisant
dont l'action bloquante est l'action la plus «~rapide~»
(\cad donc l'intervalle est le plus proche de zéro)
et l'action bloquée est la plus «~lente~»
(\cad donc l'intervalle est le plus éloigné de zéro).

Cette représentation est avantageuse par rapport à celle proposée à la
\secref{php-outils-stocha}.
En effet, il n'est pas nécessaire pour créer le modèle
de distinguer des classes globales d'actions selon leurs
intervalles de tir, mais seulement de déterminer si les intervalles de tir de toutes les actions
sont, deux à deux, disjoints (voire assez éloignés selon des critères qui peuvent être
fixés arbitrairement).
Cela permet de plus d'obtenir des relations plus fines entre actions,
là où les Frappes de Processus ne permettent que la distinction de classes de priorités
globales, qui sont parfois trop grossières pour certaines modélisations.
Cependant, le problème de la représentation de l'accumulation persiste :
une partie du modèle évoluant «~plus rapidement~» peut totalement préempter une
action «~plus lente~» si celle-ci est sans cesse neutralisées.
Or en pratique, un tel cas devrait en définitive autoriser l'action «~lente~» à s'exécuter
après un certain temps.
Cela peut être corrigé en supprimant quelques arcs neutralisants bien choisis,
ou encore en ne permettant pas la création d'arcs neutralisants entre les parties indépendantes
du modèle, mais cela nécessite une analyse préalable assez poussée de la dynamique
du système.


% Frappes de Processus avec actions plurielles
% Enrichissement avec simultanéité des actions :
\section{Frappes de Processus avec actions plurielles}
\seclabel{phm}

Il peut être intéressant de vouloir représenter un système au niveau de ses réactions biochimiques,
c'est-à-dire des réactions entre les différents composants présents.
De telles réactions peuvent avoir différentes formes (transformation, complexation, dissociation...),
et il est fréquent qu'elles fassent intervenir plusieurs réactifs et plusieurs produits.
Biocham \todo{ref} propose par exemple de modéliser un tel système de réactions biochimiques à l'aide
d'un ensemble de règles de réaction de la forme :
$X \xrightarrow{Y} Z$,
ou encore :
$X + Y \rightarrow Y + Z$,
où $X$ est un ensemble de réactifs, $Y$ un ensemble de catalyseurs et $Z$ un ensemble de produits.

Les \emph{Frappes de Processus avec actions plurielles} permettent de représenter de telles réactions
mettant en jeu un nombre arbitraire de réactifs, de produits et de catalyseurs.
Ainsi, une réaction de la forme : $X \xrightarrow{Y} Z$
peut être représentée à l'aide de l'action $\PHfrappemult{A}{B}$
où $A$ et $B$ sont deux ensembles des processus,
$A$ regroupant tous les processus représentant les composants nécessaires à initier la réaction,
et $B$ tous les processus qui ont évolué pendant la réaction.
%$A$ regroupant tous les processus permettant de représenter les composants de $X \cup Y$,
%et $B$ ceux de $Z$.
Une telle action peut donc être jouée dans un état contenant tous les processus de $A$
%(c'est-à-dire les réactifs et les catalyseurs)
et fait évoluer celui-ci vers un état contenant tous les processus de $B$, % (les produits),
les autres processus restant inchangés. % (car n'intervenant pas ou en tant que catalyseurs).
Cela implique toutefois que pour tout processus de $B$, il existe un autre processus de la même
sorte dans $A$.
Les Frappes de Processus avec actions plurielles permettent donc de représenter
un nombre arbitraire de bonds simultanés
---~autrement dit, de changements simultanés de processus actifs~---
déclenchés par un nombre arbitraire de requis
---~sous la forme de processus actifs .

Un parallèle peut être tracé d'une part entre $A$ et l'ensemble des réactifs et catalyseurs,
et d'autre part entre $B$ et l'ensemble des produits.
Cependant, la modélisation par Frappes de Processus avec actions plurielles
nécessite aussi de donner explicitement les composants sont absents.
Par exemple, une réaction de complexation du type : $x + y \rightarrow c$ %x\!\!-\!\!y$
sera représentée en Frappes de Processus avec actions plurielles par trois sortes de deux processus
$x$, $y$ et $c$ représentant respectivement les deux réactifs et le complexe produit,
et par l'action $\PHfrappemults{x_1, y_1, c_0}{x_0, y_0, c_1}$.
%si on ne prend pas en compte la disparition des réactifs.
Autrement dit, il est nécessaire de décomposer chaque élément en fonction de sa présence
($x_1$ et $y_1$)
ou de son absence ($c_0$) au début comme à la fin de la réaction,
et pas uniquement d'indiquer les composants présents en tant que réactifs ou produits.

On note ainsi qu'une réaction de la forme $\PHfrappemult{\{ a_0, b_0, c_0 \}}{\{ a_1, b_1 \}}$
ne peut être jouée si l'une des deux sortes entrant en jeu, $a$ et $b$, est déjà au niveau $1$,
même si l'autre est au niveau $0$.
Un tel comportement a du sens lorsque les différents processus d'une sorte
($a_0$ et $a_1$, par exemple)
représentent différents états d'une même molécule :
la réaction ne peut alors pas être jouée pour des raisons de stœchiométrie.
Cependant, si ces différents processus représentent plutôt des niveaux de concentration
($a_1$ représentant par exemple un niveau de concentration de la molécule $a$ plus élevé que $a_0$),
cette restriction n'a plus de sens car une plus forte concentration d'une des entités
ne devrait pas empêcher la réaction d'avoir lieu et de produire la seconde entité.
Cela peut néanmoins être corrigé en ajoutant les actions
$\PHfrappemult{\{ a_1, b_0, c_0 \}}{\{ a_1, b_1 \}}$ et
$\PHfrappemult{\{ a_0, b_1, c_0 \}}{\{ a_1, b_1 \}}$,
ou encore en séparant la production de $a_1$ et de $b_1$ en deux actions (ou ensemble d'actions)
distinctes.

Cette forme des Frappes de Processus peut être aisément représentée à l'aide d'un réseau
d'automates synchronisés, chaque sorte ayant le rôle d'un automate et chaque action celui d'un
ensemble de transitions étiquetées avec le même libellé partant chacune d'un processus dans $A$ et
arrivant dans le processus de la même sorte dans $A \recouvre B$,
comme décrit à la \secref{phm2an}.
On peut aussi la représenter à l'aide de Frappes de Processus avec 4 classes de priorités,
comme détaillé à la \secref{phm2php} ;
cependant, cette représentation a l'inconvénient d'être moins claire car faisant intervenir
un nombre important d'actions et de sortes supplémentaires.

\todo{Schéma exemple}



\subsection{Définition}

La \defref{phm} formalise la notion de Frappes de Processus avec actions plurielles,
en accord avec la discussion informelle ci-dessus :
une action plurielle est constituée de deux ensembles de processus de sortes distinctes,
qui représentent l'ensemble des frappeurs et celui des bonds.
Cela permet de formaliser le déclenchement d'une action par une synchronisation exacte
entre un nombre arbitraire de frappeurs,
et une synchronisation exacte entre plusieurs bonds lorsque l'action est jouée.
% Nous étendons aussi la notion de recouvrement à la \defref{recouvrementps}
% au cas du recouvrement d'un sous-état désordonné par un autre,
% ce qui permet, conjointement à la définition de
Nous définissons de plus l'opérateur de jouabilité
des Frappes de Processus avec actions plurielles
à la \defref{fopphm},
afin de formaliser la dynamique de ce type de modèles
à l'aide de la sémantique donnée à la \defref[vref]{play}.

\begin{definition}[Frappes de Processus avec actions plurielles]
\deflabel{phm}
  Les \emph{Frappes de Processus avec actions plurielles} sont définies
  par un triplet $\PH = (\PHs; \PHl; \PHh)$, où :
  \begin{itemize}
    \item $\PHs \DEF \{a, b, \dots\}$ est l'ensemble fini et dénombrable des \emph{sortes} ;
    \item $\PHl \DEF \bigtimes{a \in \PHs} \PHl_a$ est l'ensemble fini des \emph{états},
      où $\PHl_a = \{a_0, \ldots, a_{l_a}\}$ est l'ensemble fini et dénombrable
      des \emph{processus} de la sorte $a \in \PHs$ et $l_a \in \sN^*$,
      chaque processus appartenant à une unique sorte :
      $\forall (a_i; b_j) \in \PHl_a \times \PHl_b, a \neq b \Rightarrow a_i \neq b_j$ ;
    \item $\PHh \DEF \{\PHfrappemult{A}{B} \mid A, B \in \PHsublset \setminus \emptyset \wedge
      \forall q \in B, \exists p \in A, (p \neq q \wedge \PHsort(p) = \PHsort(q)) \}$
      est l'ensemble fini des \emph{actions}.
  \end{itemize}
\end{definition}
%
\noindent
Pour toute action $h = \PHfrappemult{A}{B} \in \PHh$,
$A$ est appelé le \emph{frappeur} et $B$ le \emph{bond} de $h$,
et on note : $\hitter{h} = A$, $\bounce{h} = B$.
On note de plus :
$\target{h} = \{ p \in A \mid \exists q \in B, \PHsort(p) = \PHsort(q) \}$,
et : $\invariant{h} = \{ p \in A \mid \sort{p} \notin \sortes{B} \}$.

\begin{definition}[Opérateur de jouabilité ($\Fopsymbol_\Fopphmsubsymbol : \PHh \rightarrow \F$)]
\deflabel{fopphm}
  L'opérateur de jouabilité des Frappes de Processus avec actions plurielles est défini par :
  \[\forall h \in \PHh, \Fopphm{h} \equiv \bigwedge_{p \in \hitter{h}} p \enspace.\]
\end{definition}



\subsection{Équivalence avec les réseaux d'automates synchronisés}
\seclabel{phm2an}

\todo{À déplacer dans le chapitre 5 ?}

\todo{Glu}

Nous nous intéressons ici au lien entre les Frappes de Processus avec actions plurielles
et les réseaux d'automates synchronisés.
Nous montrons notamment que ces deux formalismes sont équivalents
et nous exhibons pour cela deux traductions d'un formalisme vers l'autre
(\defref{phm2an,an2phm}\vpageref{def:phm2an}).
%(\defref[s]{phm2an} et \defref*[vref]{an2phm}).
Cette équivalence est intéressante car elle montre clairement le lien entre ce formalisme
de Frappes de Processus et celui plus répandu des réseaux d'automates synchronisés.
De plus, les traductions permettent de naviguer entre les deux représentations
afin d'en utiliser les différents outils.

Nous rappelons tout d'abord la définition d'un réseau d'automates synchronisés (\defref{an})
ainsi que la relation de transition entre deux états d'un tel modèle (\defref{an-sem})
ce qui permet d'en définir la dynamique.

\begin{definition}[Réseau d'automates synchronisés]
\deflabel{an}
  Un \emph{réseau d'automates synchronisés} est un quadruplet $\AN = (\ANs; \ANl; \ANi; \ANt)$
  où :
  \begin{itemize}
    \item $\ANs \DEF \{a, b, \dots\}$ est l'ensemble fini et dénombrable des \emph{automates} ;
    \item $\ANl \DEF \bigtimes{a \in \ANs} \ANl_a$ est l'ensemble fini des \emph{états},
      où $\ANl_a = \{a_0, \ldots, a_{l_a}\}$ est l'ensemble fini et dénombrable
      des \emph{états locaux} de l'automate $a \in \ANs$ et $l_a \in \sN^*$,
      chaque état local appartenant à un unique automate :
      $\forall (a_i; b_j) \in \ANl_a \times \ANl_b, a \neq b \Rightarrow a_i \neq b_j$ ;
    \item $\ANi \DEF \{\ell_1, \dots, \ell_m\}$ est l'ensemble fini des
      \emph{libellés} de transitions ;
    \item $\ANt \DEF \{ \ANaction{a_i}{\ell}{a_j} \mid a \in \ANs \wedge a_i \in \ANl_a \wedge
      \ell \in \ANi \}$ est l'ensemble fini des \emph{transitions} entre états locaux.
  \end{itemize}
  Pour tout libellé $\ell \in \ANi$, on note
  $\precond{\ell} \DEF \{ a_i \mid \ANaction{a_i}{\ell}{a_j} \in \ANt \}$
  et $\postcond{\ell} \DEF \{ a_j \mid \ANaction{a_i}{\ell}{a_j} \in \ANt \}$.
%   et $\invcond{\ell} \DEF \{ a_i \mid \ANaction{a_i}{\ell}{a_i} \in \ANt \}$.
  L'ensemble des états locaux des automates est dénoté par
  $\ANProc \DEF \bigcup_{a \in \ANs} \ANl_a$.
  Enfin, étant donné un état global $s \in \ANl$, $s(a) = a_i \in \ANl_a$
  fait référence à l'état local de l'automate $a \in \ANs$.
\end{definition}

\begin{definition}[Sémantique des réseaux d'automates ($\ANtrans$)]
\deflabel{an-sem}
  Étant donné un réseau d'automates synchronisés $\AN = (\ANs; \ANl; \ANi; \ANt)$,
  un libellé $\ell$ est dit \emph{jouable} dans un état $s \in \ANl$ si et seulement si :
  $\forall a_i \in \precond{\ell}, s(a) = a_i$.
  Dans ce cas, on note $(s \play \ell)$ l'état résultant du jeu de toutes les transitions
  libellées par $\ell$, défini par :
  $s \play \ell = s \recouvre \postcond{\ell}$.
%   $\forall a_j \in \postcond{\ell}, (s \play \ell)(a) = a_j \wedge
%     \forall b \in \ANs, \ANl_b \cap \precond{\ell} = \emptyset \Rightarrow
%     (s \play \ell)(b) = s(b)$.
  De plus, on note alors : $s \ANtrans (s \play \ell)$.
%   Étant donné un réseau d'automates synchronisés $\AN = (\ANs; \ANl; \ANi; \ANt)$,
%   la relation de transition globale entre deux états du réseau
%   $\ANtrans \subset \ANl \times \ANl$ est définie par :
%   \begin{align*}
%     s \ANtrans s' \EQDEF \exists \ell \in \ANi, &\forall a_i \in \precond{\ell}, s(a) = a_i
%       \wedge \forall a_j \in \postcond{\ell}, s'(a) = a_j \\
%     \wedge &\forall b \in \ANs, \ANl_b \cap \precond{\ell} = \emptyset \Rightarrow s(b) = s'(b)
%   \end{align*}
\end{definition}

\begin{remark}
  Nous notons que les réseaux d'automates synchronisés ainsi définis sont non-déterministes,
  tant au niveau global du modèle qu'au niveau local des automates.
  Cette vision s'oppose à d'autres sémantiques des réseaux d'automates
  comme celles de \citeasnoun{Richard10} ou de \citeasnoun{RRT08},
  qui définissent la dynamique de leurs modèles à l'aide de fonctions de transition locales,
  qui sont par définition déterministes.
  Ces fonctions ont en effet la forme : $f_a : \ANl \rightarrow \ANl_a$
  et associent donc à chaque état global du modèle un état local (unique) pour chaque automate.
  La définition des réseaux d'automates synchronisés que nous proposons ici (\defref{an})
  n'empêche en revanche pas l'existence de deux libellés $\ell_1, \ell_2 \in \ANi$
  tels que $\precond{\ell_1} = \precond{\ell_2}$ mais $\postcond{\ell_1} \neq \postcond{\ell_2}$.
  Cela implique notamment l'existence de deux transitions entre état locaux
  $\ANaction{a_i}{\ell_1}{a_j}$ et $\ANaction{a_i}{\ell_2}{a_k}$
  avec $a_j \neq a_k$, d'où un non-déterminisme au niveau des automates.
\end{remark}

Pour toutes Frappes de Processus avec actions plurielles $\PH$,
la \defref{phm2an} propose une traduction de $\PH$
en un réseau d'automates synchronisés $\phmtoan[\PH]$ équivalent,
et le \thmref{bisimulationphm2an} établit la bisimilarité entre les deux modèles.
La notation $\recouvre$ qui est utilisée dans la définition
qualifie le recouvrement d'un ensemble de processus de sortes distinctes
par un autre comprenant uniquement des processus issus des mêmes sortes
(\defref{recouvrementps}).
Cette notion de recouvrement est une extension
du recouvrement d'un état par un ensemble de processus
tel que précédemment donné à la \defref[vref]{recouvrement}.

\begin{definition}[Recouvrement ($\recouvre : \PHsublset \times \PHsublset \rightarrow \PHsublset$)]
\deflabel{recouvrementps}
  Étant donné un sous-état désordonné $ps \in \PHsublset$ et un processus $a_i \in \Proc$,
  tel que $a \in \sortes{ps}$, on définit :
  $(ps \recouvre a_i) = (ps \setminus \PHl_a) \cup \{ a_i \}$.
  On étend de plus cette définition
  au recouvrement par un ensemble de processus de sortes distinctes
  $ps' \in \PHsublset$ tel que $\sortes{ps'} \subset \sortes{ps}$
  comme étant le recouvrement de $ps$ par chaque processus de $ps'$ :
  $ps \recouvre ps' = ps \underset{a_i \in ps'}{\recouvre} a_i$.
\end{definition}

\begin{definition}[Réseau d'automates équivalent ($\phmtoansymbol$)]
\deflabel{phm2an}
  Le réseau d'automates synchronisé équivalent aux Frappes de Processus
  avec actions plurielles $\PH = (\PHs; \PHl; \PHh)$
  est défini par : $\phmtoan = (\PHs; \PHl; \ANi; \ANt)$, où :
  \begin{itemize}
    \item $\ANi = \{ \ell_h \mid h \in \PHh \}$ ; % est l'ensemble des libellés de transitions ;
    \item $\ANt = \{ \ANaction{a_i}{\ell_h}{a_j} \mid
      h \in \PHh \wedge h = \PHfrappemult{A}{B} \wedge a_i \in A \wedge a_j \in A \recouvre B \}$.
      % est l'ensemble des transitions locales.
  \end{itemize}
\end{definition}

\begin{theorem}[$\PH \approx \phmtoan$]
\thmlabel{bisimulationphm2an}
  Soient $\PH = (\PHs; \PHl; \PHh)$ des Frappes de Processus avec actions plurielles.
  On a :
  \[\forall s, s' \in \PHl, s \PHtrans s' \Longleftrightarrow s \trans{\phmtoan} s' \enspace.\]
\end{theorem}

\begin{proof}
  Soient $s, s' \in \PHl$.
  On pose : $\phmtoan = (\ANs; \ANl; \ANi; \ANt)$.
  
  ($\Rightarrow$) Supposons que $s \PHtrans s'$, c'est-à-dire qu'il existe une action $h \in \PHh$
    telle que $s' = s \play h$.
    Posons : $h = \PHfrappemult{A}{B}$.
    D'après la \defref{phm2an},
    l'existence de cette action dans $\PH$ implique celle d'un libellé $\ell_h$ dans $\phmtoan$
    ainsi que de l'ensemble de transitions
    $\ANt_h = \{ a_i \xrightarrow{\ell_h} a_j \mid a_i \in A \wedge a_j \in A \recouvre B \}$.
    Autrement dit, $\precond{\ell_h} = A$, donc $\ell_h$ est jouable dans $s$
    si et seulement si $A \subseteq s$.
    De plus, $\postcond{\ell_h} = \invariant{h} \cup B$, donc
    $(s \play \ell_h) = s \recouvre (\invariant{h} \cup B) = s \recouvre B = s'$
    car $\invariant{h} \subseteq A \subseteq s$.
  
  ($\Leftarrow$) Supposons que $s \trans{\phmtoan} s'$,
    c'est-à-dire qu'il existe un libellé $\ell \in \ANi$ et un ensemble de transitions
    ayant ce libellé : $\ANt_\ell = \{ a_i \xrightarrow{\ell} a_j \in \ANt \}$,
    tels que $s' = s \play \ell$.
    D'après la \defref{phm2an}, cela signifie notamment qu'il existe une action
    $h = \PHfrappemult{A}{B} \in \PHh$ telle que $\ell = \ell_h$, et que :
    $\ANt_\ell = \{ a_i \xrightarrow{\ell} a_j \mid a_i \in A \wedge a_j \in A \recouvre B \}$.
    Étant donné que $\invariant{h}$ et $\cible{h}$ forment une partition de $A$,
    $\ANt_\ell$ peut être découpé en deux ensembles, selon les invariants et les cibles de $h$ :
    $\ANt_\ell = \{ a_i \xrightarrow{\ell} a_i \mid a_i \in \invariant{h} \} \cup
      \{ a_i \xrightarrow{\ell} a_j \mid a_i \in \cible{h} \wedge a_j \in B \}$.
    Ainsi, $s' = s \recouvre (\invariant{h} \cup B) = s \recouvre B = s \play h$.
\end{proof}

Pour finir, nous proposons à la \defref{an2phm} la traduction inverse
d'un réseau d'automates synchronisé $\AN$
en des Frappes de Processus avec actions plurielles équivalentes $\antophm$.
Le \thmref{bisimulationan2phm} stipule que le modèle obtenu est bien bisimilaire
au modèle d'origine.
Enfin, le \thmref{equivphman} résume les résultats de cette section
en statuant l'équivalence d'expressivité entre les Frappes de Processus avec
actions plurielles et les réseaux d'automates synchronisés.

\begin{definition}[Frappes de Processus équivalentes ($\antophmsymbol$)]
\deflabel{an2phm}
  Les Frappes de Processus avec actions plurielles
  équivalentes au réseau d'automates synchronisé $\AN = (\PHs, \PHl, \ANi, \ANt)$
  sont définies par $\antophm = (\ANs, \ANl, \PHh)$, où :
%   $\PHh = \{ \PHfrappemult{\precond{\ell}}{(\postcond{\ell} \setminus \invcond{\ell})}
%     \mid \ell \in \ANi \}$.
  \[\PHh = \{ \PHfrappemult{\precond{\ell}}{B} \mid \ell \in \ANi \wedge
    B = \postcond{\ell} \setminus \{ a_i \in \ANProc \mid \ANaction{a_i}{\ell}{a_i} \in \ANt \}
    \}\]
\end{definition}

\begin{theorem}[$\AN \approx \antophm$]
\thmlabel{bisimulationan2phm}
  Soit $\AN = (\ANs; \ANl; \ANi; \ANt)$ un réseau d'automates synchronisés.
  On a :
  \[\forall s, s' \in \ANl, s \ANtrans s' \Longleftrightarrow s \trans{\antophm} s' \enspace.\]
\end{theorem}

\begin{proof}
  Soient $s, s' \in \PHl$.
  On pose : $\antophm = (\ANs; \ANl; \PHh)$.
  
  ($\Rightarrow$) Supposons que $s \ANtrans s'$,
    c'est-à-dire qu'il existe un libellé $\ell \in \ANi$ et un ensemble de transitions
    ayant ce libellé : $\ANt_\ell = \{ a_i \xrightarrow{\ell} a_j \in \ANt \}$,
    tels que $s' = s \play \ell$.
    D'après la traduction donnée à la \defref{an2phm}, il existe donc une action
    $h = \PHfrappemult{A}{B} \in \PHh$ telle que $A = \precond{\ell}$ et
    $B = \postcond{\ell} \setminus \{ a_i \in \ANProc \mid \ANaction{a_i}{\ell}{a_i} \in \ANt \}$.
    Or $s' = s \recouvre \postcond{\ell}
      = s \recouvre (B \cup \{ a_i \in \ANProc \mid \ANaction{a_i}{\ell}{a_i} \in \ANt \})
      = s \recouvre B$
    car $\{ a_i \in \ANProc \mid \ANaction{a_i}{\ell}{a_i} \in \ANt \} \subseteq s$.
    Ainsi, $h$ est jouable dans $s$ et $s' = s \play h$.
  
  ($\Leftarrow$) Supposons que $s \trans{\antophm} s'$,
    c'est-à-dire qu'il existe une action $h = \PHfrappemult{A}{B} \in \PHh$
    telle que $s' = s \play h$.
    D'après la traduction de la \defref{an2phm},
    cela signifie qu'il existe un libellé $\ell \in \ANi$ et un ensemble de transitions
    ayant ce libellé : $\ANt_\ell = \{ a_i \xrightarrow{\ell} a_j \in \ANt \}$,
    tels que : $A = \precond{\ell}$ et
    $B = \postcond{\ell} \setminus \{ a_i \in \ANProc \mid \ANaction{a_i}{\ell}{a_i} \in \ANt \}$.
    Comme $h$ est jouable dans $s$, alors $A \subseteq s$, donc $\ell$ est aussi jouable dans $s$.
    De plus, $s' = s \play h = s \recouvre B = s \recouvre (B \cup \invariant{h})
      = s \recouvre \postcond{\ell}$.
\end{proof}

\begin{theorem}[Équivalence entre réseaux d'automates synchronisés
  et Frappes de Processus avec actions plurielles]
\thmlabel{equivphman}
  Les Frappes de Processus avec actions plurielles sont aussi expressives
  que les réseaux d'automates synchronisés.
\end{theorem}

\begin{proof}
  D'après les \defref{phm2an,an2phm} et les \thmref{bisimulationphm2an,bisimulationan2phm}
  associés, tout modèle de Frappes de Processus avec actions plurielles peut être représenté
  à l'aide d'un réseau d'automates synchronisés, et inversement.
\end{proof}



\subsection{Traduction vers les Frappes de Processus avec 4 classes de priorités}
\seclabel{phm2php}

Nous proposons dans cette section une traduction des Frappes de Processus
avec actions plurielles en Frappes de Processus avec 4 classes de priorités (\defref{phm2php})
et nous montrons que les modèles obtenus de cette façon sont faiblement bisimilaires
(\thmref{phmbisimphp}).
Nous en déduisons que ces deux formalismes ont une expressivité équivalente,
%(\thmref{phmequivphp}).
car les Frappes de Processus avec 4 classes de priorités peuvent à leur tour être
traduites en Frappes de Processus canoniques (comme expliqué à la \vsecref{phm-aplatissement}),
celles-ci pouvant à leur tour être traduites en Frappes de Processus avec actions plurielles
\TODO.

La traduction proposée permet d'obtenir un modèle de
Frappes de Processus pseudo-canoniques avec 4 classes de priorités,
telles que définies plus tard, à la \secref{phcanonique}.
De façon informelle, ce modèle comporte 4 classes de priorités
et repose sur l'utilisation de deux types de sortes particulières :
\begin{itemize}
  \item des sortes coopératives pour vérifier la présence de tous les processus du frappeur,
  \item des \emph{sortes de réaction} permettant de modéliser le déclenchement d'une \emph{réaction},
    ou son arrêt.
\end{itemize}
Une réaction modélise le fait qu'un ensemble d'actions (standards) est en train
de simuler le jeu d'une action plurielle.
À toute action plurielle $h$ du modèle d'origine correspond une sorte coopérative $\scf{h}$
entre les sortes des processus de $A$
et une sorte de réaction $\sr{h}$ dans le modèle résultant de cette traduction.
La sorte coopérative comporte notamment un processus $\scf{h}_\mypi$ qui représente le sous-état
où tous les processus de $A$ sont présents ; une action de priorité 4 de la forme
$\PHhit{\scf{h}_\mypi}{\sr{h}_0}{\sr{h}_1}$ permet d'activer la sorte de réaction.
Une auto-action de priorité 3 de la forme $\PHhit{\sr{h}_1}{\sr{h}_1}{\sr{h}_0}$
permet de plus la désactivation de la sorte de réaction
une fois que toutes les actions de priorité 2 auront été jouées.
Les actions de priorité 2 ont la forme $\PHhit{\sr{h}_1}{b_j}{b_k}$
avec $b_j \in A$ et $b_k \in B$,
ce qui permet d'effectuer tous les bonds nécessaires à l'activation des processus de $B$.
Enfin, les sortes coopératives sont toutes mises à jour par des actions de priorité 1,
afin d'éviter les problèmes d'entrelacement
et de correspondre à la définition de Frappes de Processus pseudo-canoniques
(\defref{phpseudocanonique})
L'agencement de ces classes de priorités permet ainsi de simuler des actions plurielles
tout en empêchant l'entrelacement entre réactions
---~car deux réactions ayant lieu en même temps pourraient potentiellement
amener le système dans un état normalement inaccessible.

\begin{definition}
\deflabel{phm2php}
  Soient $\PH = (\PHs; \PHl; \PHh)$ des Frappes de Processus avec actions plurielles,
  et $\phmtophp = (\PHs'; \PHl'; \PHh'^{\langle 4 \rangle})$
  leur traduction en Frappes de Processus avec $4$ classes de priorités, où :
  \begin{itemize}
    \item $\PHs' = \PHs \cup \{ \sr{h} \mid h \in \PHh \} \cup \{ \scf{h} \mid h \in \PHh \}$ ;
    \item $\PHl' = \PHl \times \left( \bigtimes{h \in \PHh} \PHl_{\sr{h}} \right)
      \times \left( \bigtimes{h \in \PHh} \PHl_{\scf{h}} \right)$, où :
      \begin{itemize}
        \item $\forall h \in \PHh, \PHl_{\sr{h}} = \{ \sr{h}_0 , \sr{h}_1 \}$,
        \item $\forall h \in \PHh, \PHl_{\scf{h}} = \{ \scf{h}_\mysigma \mid
          \mysigma \in \PHsubl_{\sortes{\frappeur{h}}} \}$ ;
      \end{itemize}
    \item $\PHh'^{\langle 4 \rangle} = ( \PHh'^{(1)} ; \PHh'^{(2)} ; \PHh'^{(3)} ; \PHh'^{(4)} )$,
      où :
      \begin{itemize}
        \item $\PHh'^{(1)} = \{ \PHhit{a_i}{\scf{h}_\mysigma}{\scf{h}_{\mysigma'}} \mid
          h \in \PHh \wedge a \in \sortes{\frappeur{h}} \wedge a_i \in \PHl_a \wedge
          \scf{h}_\mysigma , \scf{h}_{\mysigma'} \in \PHl_{\scf{h}} \wedge
          \PHget{\mysigma}{a} \neq a_i \wedge \mysigma' = \mysigma \Cap a_i \}$,
        \item $\PHh'^{(2)} = \{ \PHhit{\sr{h}_1}{b_j}{b_k} \mid
          h \in \PHh \wedge b \in \sortes{\cible{h}} \wedge b_j, b_k \in \PHl_b \wedge
          b_j \in \cible{h} \wedge b_k \in \bond{h} \}$,
        \item $\PHh'^{(3)} = \{ \PHhit{\sr{h}_1}{\sr{h}_1}{\sr{h}_0} \mid h \in \PHh \}$,
        \item $\PHh'^{(4)} = \{ \PHhit{\scf{h}_\mypi}{\sr{h}_0}{\sr{h}_1} \mid
          h \in \PHh \wedge \scf{h}_{\mypi} \in \PHl_{\scf{h}} \wedge
%          \Feval{\Fopphm{h}}{\mypi} \}$.
          \frappeur{h} \subseteq \mypi \}$.
      \end{itemize}
  \end{itemize}
  Pour tout état $s \in \PHl$ de $\PH$,
  on note $\tophp{s}$ l'état correspondant dans $\phmtophp$ :
  \begin{itemize}
    \item $\forall a \in \PHs, \PHget{\tophp{s}}{a} = \PHget{s}{a}$,
    \item $\forall h \in \PHh, \PHget{\tophp{s}}{\sr{h}} = \sr{h}_0$,
    \item $\forall h \in \PHh, \PHget{\tophp{s}}{\scf{h}} = \scf{h}_\mysigma$,
      tel que $\forall a \in \sortes{\frappeur{h}}, \PHget{\mysigma}{a} = \PHget{\tophp{s}}{a}$.
  \end{itemize}
  À l'inverse, pour tout état $s' \in \PHl'$ de $\phmtophp$,
  on note $\tophm{s'}$ l'état correspondant dans $\PH$ :
  $\forall a \in \PHs, \PHget{\tophm{s'}}{a} = \PHget{s'}{a}$.
\end{definition}

\begin{theorem}[$\PH \approx \phmtophp$]
\thmlabel{phmbisimphp}
  Soient $\PH = (\PHs; \PHl; \PHh)$ des Frappes de Processus avec actions plurielles,
  et posons : $\phmtophp = (\PHs'; \PHl'; \PHh'^{\langle 4 \rangle})$.
  On a :
%   \[\forall s, s' \in \PHl, s \PHtrans s' \Longleftrightarrow
%     \tophp{s} \mtrans{\phmtophp} \tophp{s'} \enspace.\]
% 
  \[\forall s, s' \in \PHl, \exists h \in \PHh, s' = s \play h \Longleftrightarrow
    \exists \delta \in \Sce(\tophp{s}), \left( \tophp{s'} = \tophp{s} \play \delta
    \wedge \card{\toset{\delta} \cap \PHh'^{(4)}} = 1 \right) \enspace.\]
\end{theorem}

\begin{proof}
  Posons $\phmtophp = (\PHs'; \PHl'; \PHh'^{\langle 4 \rangle})$.
  Soient $s, s' \in \PHl$.
  
  ($\Rightarrow$) On suppose qu'il existe une action $h \in \PHh$ telle que $s' = s \play h$.
    D'après la \defref{phm2php},
    cela implique notamment l'existence de sortes $\sr{h}$ et $\scf{h}$ dans $\PHs'$,
    et des actions suivantes :
    \begin{itemize}
      \item $\PHh'^{(1)}_h = \{ \PHhit{a_i}{\scf{h}_\mysigma}{\scf{h}_{\mysigma'}} \mid
        a \in \sortes{\frappeur{h}} \wedge a_i \in \PHl_a \wedge
        \scf{h}_\mysigma , \scf{h}_{\mysigma'} \in \PHl_{\scf{h}} \wedge
        \PHget{\mysigma}{a} \neq a_i \wedge \mysigma' = \mysigma \Cap a_i \} \subset \PHh'^{(1)}$,
      \item $\PHh'^{(2)}_h = \{ \PHhit{\sr{h}_1}{b_j}{b_k} \mid
        b \in \sortes{\bond{h}} \wedge b_j, b_k \in \PHl_b \wedge
        b_j \in \frappeur{h} \wedge b_k \in \cible{h} \} \subset \PHh'^{(2)}$,
      \item $h_3 = \PHhit{\sr{h}_1}{\sr{h}_1}{\sr{h}_0} \in \PHh'^{(3)}$,
      \item $h_4 = \PHhit{\scf{h}_\mypi}{\sr{h}_0}{\sr{h}_1} \in \PHh'^{(4)}$ avec $\mypi$ tel que
        $\Feval{\Fopphm{h}}{\mypi}$.
    \end{itemize}
    Comme $h$ est jouable dans $s$, alors $\frappeur{h} \subseteq s$,
%    d'où $\frappeur{h} \subseteq \tophp{s}$.
    d'où $\scf{h}_{\mypi} \in \tophp{s}$, avec $\toset{\mypi} = \frappeur{h}$.
    Ainsi, $\PHhit{\scf{h}_\mypi}{\sr{h}_0}{\sr{h}_1}$ est jouable dans $\tophp{s}$.
    Toutes les actions de $\PHh'^{(2)}_h$ sont jouables dans $\tophp{s} \play h_4$.
    Elles peuvent être jouées successivement et alternativement avec des actions de
    $\PHh'^{(1)}$ car celles-ci modifient uniquement l'état de sortes coopératives,
    ce qui n'influe donc pas sur la jouabilité des actions de $\PHh'^{(2)}_h$.
    Soit $\PHh'^{(2)}_h = \{ h_2^i \}_{i \in \segm{1}{n}}$
    un étiquetage des actions de $\PHh'^{(2)}_h$
    avec $n = \card{\PHh'^{(2)}_h}$.
    On peut jouer depuis l'état $\tophp{s} \play h_4$ une séquence d'actions de la forme :
    $\delta_h =
      h_2^1 \cons \delta^1 \cons h_2^2 \cons \delta^2 \cons
      \ldots \cons
      h_2^n \cons \delta^n$
    où toutes les séquences $\delta^i$ avec $i \in \segm{1}{n}$
    sont des séquences d'actions de $\PHh^{(1)}$ mettant à jour des sortes coopératives.
    Après avoir joué cette séquence, le modèle se trouve dans un état
    $\tophp{s} \play h_4 \play \delta_h$ tel que :
    $\forall h_i \in \PHh'^{(2)}_h, \bond{h_i} \in (\tophp{s} \play h_4 \play \delta_h)$,
    c'est-à-dire : $\bond{h} \subseteq (\tophp{s} \play h_4 \play \delta_h)$.
    De plus, toutes les sortes coopératives sont mises à jour, ce qui signifie qu'aucune
    action de $\PHh'^{(1)}_h \cup \PHh'^{(2)}_h$ n'est plus jouable.
    Il est donc possible de jouer pour finir l'auto-action $h_3$,
    et on a :
    $\tophp{s} \play h_4 \play \delta_h \play h_3 = \tophp{s'}$,
    avec : $\toset{\delta} \cap \PHh'^{(4)} = \{ h_4 \}$.
  
  ($\Leftarrow$) On suppose qu'il existe un scénario $\delta \in \Sce(\tophp{s})$
    tel que $\tophp{s'} = \tophp{s} \play \delta$ et $\card{\toset{\delta} \cap \PHh'^{(4)}} = 1$.
    D'après la forme de $\phmtophp$ (\defref{phm2php}),
    et en s'inspirant du raisonnement précédent,
    on constate que la seule action jouable dans $\tophp{s}$ est une action de $\PHh'^{(4)}$,
    puis que dans l'état résultat, on ne peut jouer
    qu'une alternance entre une action de $\PHh'^{(2)}$ et des actions de $\PHh'^{(1)}$,
    et que l'état résultant une fois toutes les actions de $\PHh'^{(1)} \cup \PHh'^{(2)}$
    jouées ne permet que le jeu d'une action de $\PHh'^{(3)}$.
    On en déduit que ce scénario a nécessairement la forme suivante :
    $\delta = h_4 \cons \delta_{12} \cons h_3$,
    avec $h_4 \in \PHh'^{(4)}$, $h_3 \in \PHh'^{(3)}$, et
    $\delta_{12} = h_2^1 \cons \delta^1 \cons h_2^2 \cons \delta^2 \cons
      \ldots \cons h_2^n \cons \delta^n$
    où pour tout $i \in \segm{1}{n}$, $h_2^i \in \PHh'^{(2)}$
    et $\delta^i \in \Sce$ est un scénario composé uniquement d'actions de $\PHh'^{(1)}$.
    De plus, les seules actions jouables dans $\tophp{s} \play \delta$ sont à nouveau des actions
    de $\PHh'^{(4)}$, donc $\delta$ ne peut pas être plus grand car
    $\card{\toset{\delta} \cap \PHh'^{(4)}} = 1$.
    Posons :
    \begin{itemize}
      \item $\PHh'^{(1)}_\delta = \toset{\delta} \cap \PHh'^{(1)}$,
      \item $\PHh'^{(2)}_\delta = \toset{\delta} \cap \PHh'^{(2)}
        = \{ h_2^i \mid i \in \segm{1}{n} \}$,
      \item $h_3 = \PHhit{\sr{}_1}{\sr{}_1}{\sr{}_0}$ et
      \item $h_4 = \PHhit{\scf{}_\mypi}{\sr{}_0}{\sr{}_1}$.
    \end{itemize}
    D'après la construction de $\phmtophp$ donné à la \defref{phm2php},
    il existe donc nécessairement une action $h \in \PHh$
    dont $\sr{}$ est la sorte de réaction et $\scf{}$ est la sorte coopérative correspondante.
    De plus, toujours grâce à cette définition, on a :
    $\mypi \in \PHsubl[\phmtophp]$ et $\frappeur{h} \subseteq \mypi$.
    Autrement dit : $\frappeur{h} = \toset{\mypi}$,
    donc que $h$ est jouable dans $s$, par définition de $\tophp{s}$.
    De plus, toujours d'après la \defref{phm2php}, on a :
    $\forall i \in \segm{1}{n}, h_2^i = \PHfrappe{\sr{}_1}{b_j^i}{b_k^i}
      \wedge b_j^i \in \frappeur{h} \wedge b_k \in \bond{h}$.
    Ainsi, $\bond{h} = \{ b_k = \bond{h_2^i} \mid i \in \segm{1}{n} \}$.
    Or on constate immédiatement que toutes les sortes coopératives sont nécessairement
    mises à jour
    dans $s \play \delta$ car la dernière action jouée est une action de $\PHh'^{(3)}$
    et que son jeu ne rend aucune action de $\PHh'^{(1)}$ jouable.
    De plus, $\PHget{s \play \delta}{\sr{}} = \sr{}_0$ et
    $\tophp{s} \play \delta = \tophp{s} \recouvre \{ \bond{h_2^i} \mid i \in \segm{1}{n} \}$.
    Ainsi, $\tophp{s'} = \tophp{s} \play \delta = \tophp{s} \recouvre \bond{h}
      = \tophp{s \recouvre \bond{h}} = \tophp{s \play h}$.
\end{proof}



\todo{Exemple}



\subsection{Équivalence avec les Frappes de Processus avec classes de priorités}
\seclabel{equivphmphp}

\todo{Formaliser l'équivalence PHplurielles / PHp de façon formelle ?}

Cette section permet de formaliser l'équivalence d'expressivité entre les Frappes de Processus
avec actions plurielles d'une part, et les Frappes de Processus avec $k$ classes de priorités
d'autres part, si $k \in \sNN$.
Le \thmref{phmequivphp} formalise cette conclusion,
bien que sa démonstration s'appuie sur des résultats qui seront
présentés aux \vsecref{phm-aplatissement,phcanonique2phm} \storef.
En effet, elle s'appuie sur plusieurs résultats plus forts qui stipulent que des
Frappes de Processus avec actions plurielles
et les Frappes de Processus avec $k$ classes de priorités
peuvent toujours être représentées
par une classe particulière de Frappes de Processus avec 2 classes de priorités,
qui elle-même peut aussi être représentée par des
Frappes de Processus avec actions plurielles.

\begin{theorem}[Équivalence]
\thmlabel{phmequivphp}
  Pour tout $k \in \sNN$,
  les Frappes de Processus avec actions plurielles
  et les Frappes de Processus avec $k$ classes de priorités
  ont une expressivité équivalente.
\end{theorem}

\begin{proof}
  Soit $k \in \sNN$.
  D'après \storef, il est toujours possible de traduire des Frappes de Processus avec actions
  plurielles en Frappes de Processus canoniques, qui sont \textit{a fortiori}
  des Frappes de Processus avec $k$ classes de priorités.
  Inversement, toutes Frappes de Processus avec $k$ classes de priorités
  peuvent être aplaties en Frappes de Processus canoniques \storef,
  qui elles-mêmes peuvent être traduites en Frappes de Processus avec action plurielles \storef.
\end{proof}



\subsection{Réutilisation des outils existants}
\seclabel{phm-outils}

\todo{Revoir le titre de cette section}

\todo{Faudrait-il fusionner cette section avec ses semblables des \secref{php} et \secref{phan} ?}

\subsubsection{Points fixes}
\seclabel{phm-outils-pf}

Du fait de la complexité des actions plurielles, il n'existe pas à l'heure actuelle de méthode
générique efficace permettant de déduire les points fixes de Frappes de Processus avec
actions plurielles.
Cependant, à l'instar des Frappes de Processus avec arcs neutralisants
(\secref{phan-outils-pf})
il est possible d'obtenir ces points fixes les cherchant sur
le modèle aplati (tel que défini à la \secref{phm-aplatissement}).
En effet, sa dynamique étant équivalente, aux sortes coopératives près, le résultat
peut être transposé au modèle d'origine.

\subsubsection{Analyse statique}
\seclabel{phm-outils-as}

Du fait de la forme particulière des actions, les méthodes d'analyse statique ne s'appliquent pas
directement aux Frappes de Processus avec actions plurielles.
Cependant, moyennant l'utilisation de la traduction vers des Frappes de Processus
avec 4 classes de priorités donnée à la \defref{phm2php},
il est possible de les appliquer sur un modèle équivalent.

\subsubsection{Paramètres stochastiques}
\seclabel{phm-outils-stocha}

Il est théoriquement possible d'associer des paramètres stochastiques à chaque action plurielle
d'un modèle de Frappes de Processus avec actions plurielles.
Un tel ajout aurait notamment l'avantage d'éviter l'utilisation de la valeur «~infinie~»
d'absorption de stochasticité, qui avait principalement pour but de simuler des actions
successives instantanées, afin notamment de pallier le fait qu'un seul processus
ne peut évoluer pour chaque jeu d'action des Frappes de Processus standards.

Une autre alternative consisterait à utiliser l'aplatissement de la \secref{phm-aplatissement},
et d'attribuer aux actions correspondant à des activations de réaction
dans le modèle obtenu les paramètres stochastiques voulus.



% Représentation canonique et analyse du PH

\chapter{Représentation canonique pour l'analyse des Frappes de Processus}
\chaplabel{phcanonique}
\chaplabel{as}

\chapeau{%
  Une analyse statique efficace des questions d'atteignabilité avait précédemment été développée
  sur les Frappes de Processus standards par \citeasnoun{PMR12-MSCS}.
  Elle ne peut cependant pas être appliquée aux
  formalismes alternatifs de Frappes de Processus présentés au \chapref{sem},
  qui possèdent une expressivité supérieure.
  
  L'objectif de ce chapitre est de présenter les travaux ayant permis
  de développer une nouvelle forme de sous-approximation pour l'analyse statique.
  Celle-ci s'applique à une classe restreinte de Frappes de Processus avec 2 classes de priorités,
  appelées \emph{Frappes de Processus canoniques}.
  Nous montrons par ailleurs que toutes les sémantiques alternatives de Frappes de Processus
  introduites au \chapref{sem} peuvent être traduites en Frappes de Processus canoniques,
  prouvant d'une part la large expressivité de cette classe de modèles,
  et permettant d'autre part d'appliquer la sous-approximation développée dans la suite
  à tous ces formalismes.
}



Les Frappes de Processus standards (\vdefref{ph}) sont très atomiques,
et leurs actions ont une forme particulière :
un processus peut déclencher le bond d'un autre processus dans une autre sorte.
Autrement dit, le changement du processus actif d'une sorte est conditionné par la présence
d'au plus un autre processus actif.
Dans le cas d'une auto-action $\PHfrappe{b_j}{b_j}{b_k}$, par exemple,
le frappeur et la cible $b_j$ sont confondus ;
il n'y a donc formellement aucun prérequis pour que $a_i$ bondisse vers $a_k$.
Dans le cas plus général d'une action $\PHfrappe{a_i}{b_j}{b_k}$, où $a_i \neq b_j$,
le seul prérequis pour que $b_j$ bondisse en $b_k$ est la présence de $a_i$.

Cette restriction sur la forme des actions permet une analyse efficace des questions d'atteignabilité
au sein des Frappes de Processus standards, qui s'expriment de la façon suivante :
\begin{center}
  «~Partant d'un état donné, est-il possible, en jouant un nombre quelconque d'actions,\\
  d'atteindre un état dans lequel un processus donné est actif ?~»
\end{center}
En termes plus formels, l'atteignabilité d'un processus $b_k$ depuis un état $s$
se définit comme l'existence d'un scénario $\delta \in \Sce(s)$, tel que
$\PHget{(s \play \delta)}{b} = b_k$.

En effet, comme expliqué plus haut,
la forme particulière des actions impose au plus un prérequis pour pouvoir faire bondir
le processus actif d'une sorte vers un autre niveau.
Partant de ce constat, une analyse statique permettant de répondre efficacement aux
questions d'atteignabilité avait été précédemment développée par \citeasnoun{PMR12-MSCS}.
Si on considère l'exemple de l'atteignabilité d'un processus $b_k$ depuis un état $s$
tel que $\PHget{s}{b} = b_j$, avec $b_j \neq b_k$,
la méthode utilisée repose sur l'analyse de la dynamique locale de $b$, ou, autrement dit,
sur l'analyse des bonds entre processus de $b$ produits par les actions frappant cette sorte.
En particulier, si on reprend les deux exemples d'actions donnés ci-dessus,
\begin{itemize}
  \item s'il existe une auto-action $\PHfrappe{b_j}{b_j}{b_k}$, alors la réponse est immédiate
    car $b_k$ est accessible en jouant cette action qui n'a pas de prérequis;
  \item s'il existe une action $\PHfrappe{a_i}{b_j}{b_k}$, alors l'atteignabilité de $b_k$
    est conditionnée par l'atteignabilité de $a_i$ ---~à la condition près que cette
    seconde atteignabilité laisse $b_j$ intact.
\end{itemize}
Dans le cas général, faire bondir un processus de $b_j$ à $b_k$ peut en fait nécessiter plusieurs
actions, ou aucune si le processus est déjà présent dans l'état initial.
Ainsi, l'atteignabilité d'un processus peut nécessiter l'atteignabilité de plusieurs
autres processus, ou d'aucun autre processus.
Cette approche permet donc de résoudre un problème d'atteignabilité de façon récursive,
les cas terminaux étant ceux pour lesquels la résolution ne nécessite aucune action,
ou ne nécessite que des auto-actions.

\myskip

Cependant, cette méthode d'analyse statique ne s'applique pas totalement aux nouveau formalismes
de Frappes de Processus présentés au \chapref{sem}.
En effet, ces formalismes augmentent l'expressivité des Frappes de Processus,
ce qui invalide, au moins en partie, la méthode développée.
Nous proposons dans ce chapitre une nouvelle approche par analyse statique qui permet
de répondre à des questions d'atteignabilité sur une classe particulière des Frappes de Processus
avec 2 classes de priorités, appelées \emph{Frappes de Processus canoniques}.
La méthode que nous proposons possède une efficacité proche de la méthode d'origine,
le calcul du résultat étant de complexité polynomiale selon le nombre de sortes dans le modèle.
Cela permet notamment de traiter de grands modèles efficacement ;
en effet, l'implémentation qui en a été fait répond toujours en moins d'une seconde
sur des modèles comportant une centaine de composants,
comme nous le montrerons par la suite au \chapref{applications}.

Les Frappes de Processus canoniques
ne comportent que 2 classes de priorités ainsi qu'une restriction dans
l'utilisation des actions de priorité 1 (c'est-à-dire les plus prioritaires) :
celles-ci ne doivent servir qu'à mettre à jour des sortes coopératives,
et tous les autres types d'actions doivent donc être de priorité 2.
Ce formalisme permet ainsi de représenter de façon simple les modèles de Thomas (avec paramètres)
ou les modèles booléens (avec portes logiques),
à l'aide de traductions de complexité polynomiale.
Nous montrons de plus qu'elles forment une classe de modèles qui est aussi expressive
que toutes les Frappes de Processus avec $k$ classes de priorités,
les Frappes de Processus avec arcs neutralisants
et les Frappes de Processus avec actions plurielles.
Nous donnons par ailleurs les traductions appropriées pour pouvoir aussi traiter ces modèles,
qui peuvent cependant s'avérer avoir une complexité exponentielle
selon certaines caractéristiques des modèles.

\myskip

Nous définissons dans un premier temps les Frappes de Processus canoniques
de façon formelle dans la \secref{phcanonique}
et donne un certain nombre de résultats à leur propos.
La \secref{phcanonique-equiv} aborde ensuite
la question du lien entre les Frappes de Processus canoniques
et les autres formalismes présentés au \chapref{sem},
et propose les traductions nécessaires.
Pour terminer, nous présentons formellement
les méthodes d'analyse statiques développées spécifiquement
pour les Frappes de Processus canoniques à la \secref{as}.

Les travaux présentés dans ce chapitre ont été publiés dans
\cite*{FPMR13-CS2Bio}.

\vfill

% \section{Approche informelle}
% \TODO
% 
% \todoplustard{Explications informelles}

% Définition

\subsection{Définition}
\seclabel{phcanonique-def}

Nous donnons dans cette section la définition des Frappes de Processus canoniques,
qui sont un sous-ensemble des Frappes de Processus avec classes de priorités.
Un modèle de Frappes de Processus est dit \emph{canonique} s'il comporte 2 classes de priorités
et qu'il respecte un certain nombre de critères concernant l'utilisation des actions prioritaires.
Il faut notamment que celles-ci soient uniquement utilisées pour la mise à jour des sortes
coopératives, et que cette mise à jour soit correctement formée (pas d'action manquantes ou en trop).
Cette forme particulière des Frappes de Processus assure un certain nombre de bonnes propriétés
qui seront explicitées par la suite et qui permettent l'utilisation des méthodes d'analyse statique
développées à la section \todo{suivante}.
Dans la suite de cette section, nous travaillons sur des modèles de Frappes de Processus avec $k$
classes de priorités, pour $k \in \sN^*$.



Nous définissons pour commencer la \emph{réduction} d'un modèle donné de Frappes de Processus
comme le modèle équivalent dont on a retiré toutes les actions qui ne sont pas de priorité 1
(\defref{reduction}).
Cette définition permet dans la suite de considérer le comportement des actions les plus prioritaires
d'un modèle afin d'en contraindre la forme.

\begin{definition}[Réduction ($\reductionsymbol{\PH}$)]
\deflabel{reduction}
  Si $\PH = (\PHs; \PHl; \PHh^{\langle k \rangle})$ sont des Frappes de Processus avec $k$
  classes de priorités, on note $\reduction[\PH] = (\PHs; \PHl; \PHh^{(1)})$
  la \emph{réduction} de $\PH$.
  $\reduction$ est un modèle de Frappes de Processus standard
  (ou encore : avec 1 classe de priorité).
  Si $s \in \PHl$, on note de plus : $\reduction[\Sce](s)$ l'ensemble des scénarios
  depuis l'état $s$ dans les Frappes de Processus $\reduction$.
\end{definition}

\begin{example}
  La \figref{ph-livelock} donne un exemple de Frappes de Processus avec 2 classes de priorités avec :
  \begin{align*}
    \PHs &= \{ a, b, c, ab \} \enspace, \\
    \PHl_a &= \{ a_0, a_1 \} \enspace, & \PHl_b &= \{ b_0, b_1 \} \enspace, \\
    \PHl_c &= \{ c_0, c_1 \} \enspace, & \PHl_{ab} &= \{ ab_{00}, ab_{01}, ab_{10}, ab_{11} \}
    \enspace.
  \end{align*}
  On note que :
  $\{ \PHhit{ab_{11}}{c_0}{c_1}, \PHhit{a_1}{a_1}{a_0}, \PHhit{a_0}{b_0}{b_1} \} \subseteq \PHh^{(2)}$.

\begin{figure}[tb]
  \centering
  \scalebox{1.2}{
  \begin{tikzpicture}
    \TSort{(0,0)}{a}{2}{l}
    \TSort{(0,4)}{b}{2}{l}
    \TSort{(7,2.5)}{c}{2}{r}

    \TSetTick{ab}{0}{00}
    \TSetTick{ab}{1}{01}
    \TSetTick{ab}{2}{10}
    \TSetTick{ab}{3}{11}
    \TSort{(2,2.5)}{ab}{4}{t}

    \THit{a_0}{prio}{ab_3}{.south}{ab_1}
    \THit{a_0}{prio}{ab_2}{.south}{ab_0}
    \THit{a_1}{prio}{ab_1}{.south}{ab_3}
    \THit{a_1}{prio}{ab_0}{.south}{ab_2}

    \THit{b_0}{prio}{ab_3}{.north}{ab_2}
    \THit{b_0}{prio}{ab_1}{.north}{ab_0}
    \THit{b_1}{prio}{ab_2}{.north}{ab_3}
    \THit{b_1}{prio}{ab_0}{.north}{ab_1}
    
    \THit{a_1}{selfhit}{a_1}{.west}{a_0}
    \THit{b_1}{selfhit}{b_1}{.west}{b_0}
    \THit{a_0.north}{bend left}{b_0}{.west}{b_1}
    \THit{b_0.south}{bend right=60}{a_0}{.west}{a_1}

    \THit{ab_3}{}{c_0}{.west}{c_1}

  \path[bounce, bend right=55]
      \TBounce{ab_0}{}{ab_2}{.west}
      \TBounce{ab_1}{}{ab_3}{.west}
  ;
  \path[bounce, bend left=20]
      \TBounce{ab_3}{}{ab_1}{.south east}
      \TBounce{ab_2}{}{ab_0}{.south east}
  ;
    \path[bounce, bend right=20]
      \TBounce{ab_3}{}{ab_2}{.north east}
      \TBounce{ab_1}{}{ab_0}{.north east}
    ;
    \path[bounce, bend left=30]
      \TBounce{ab_2}{}{ab_3}{.west}
      \TBounce{ab_0}{}{ab_1}{.west}
    ;
    \path[bounce, bend right]
      \TBounce{a_1}{}{a_0}{.north west}
      \TBounce{b_1}{}{b_0}{.north west}
    ;
    \path[bounce, bend left]
      \TBounce{a_0}{}{a_1}{.south west}
      \TBounce{b_0}{}{b_1}{.south west}
    ;
    \path[bounce, bend left]
      \TBounce{c_0}{}{c_1}{.south west}
    ;
    \TState{a_1, b_0, ab_2, c_0}
  \end{tikzpicture}
  }
  \caption{
  \figlabel{ph-livelock}
    Un exemple de Frappes de Processus canonique.
    Les actions de priorité 1 sont dessinées en traits doubles tandis que les actions
    de priorité 2 sont représentées avec des traits simples.
    Les processus grisés présentent un exemple d'état de départ :
    $\PHstate{a_1, b_0, c_0, ab_{10}}$.
  }
\end{figure}

\end{example}



\subsection{Restrictions}
\seclabel{restrictions}

Les Frappes de Processus canoniques consistent en une classe particulière de Frappes de Processus
avec 2 classes de priorités, et cette section a pour objectif d'en donner la définition.
Les restrictions permettant de définit des Frappes de Processus canoniques portent sur
les actions les plus prioritaires mettant à jour les sortes coopératives.
Cependant, afin de poser un cadre plus général et permettre une réutilisation de ces définitions
\stodo{ref},
nous considérerons le cas de Frappes de Processus avec $k$ classes de priorités dans les définitions
et critères qui suivent.

Dans la suite, nous appelons \emph{actions primaires} les actions de l'ensemble $\PHh^{(1)}$,
c'est-à-dire les actions les plus prioritaires.
Ces actions auront pour unique but de mettre à jour des sortes coopératives, permettant
à celles-ci de modéliser des portes logiques sans décalage temporel \stodo{ref}.
De fait, on peut dans certains cas considérer ces actions comme «~non biologiques~»,
ou «~instantanées~»,
car elles sont présentes principalement pour des raisons de modélisation.
À l'inverse, les actions de l'ensemble $\PHh \setminus \PHh^{(1)}$, qui ne sont pas de priorité 1,
seront appelées \emph{actions secondaires}.
Comme elles permettent de représenter les différentes réactions et régulations intervenant au sein du
système, elles peuvent par conséquent être qualifiées d'actions «~biologiques~» ou «~lentes~».

La définition de la forme canonique des Frappes de Processus début avec la
\crref{tf} qui stipule que la dynamique ne doit pas contenir de séquence infinie
d'actions primaires.
En effet, pour que ces actions effectuent une mise à jour des sortes coopératives sans perturber
le reste du système, il est nécessaire qu'elles ne puissent pas préempter les actions secondaires
indéfiniment.
Sans cela, le modèle serait victime du paradoxe de Zénon,
où une suite infinie d'actions peut être jouée en un temps nul et ainsi
bloquer l'évolution du système.
Il est donc nécessaire de postuler que tous les scénarios d'actions primaires sont finis,
ce qui a par ailleurs pour conséquence que $\reductionsce$ est un ensemble fini.

\begin{condition}[Terminaison finie]
\crlabel{tf}
  La dynamique de $\reduction$ ne contient aucun cycle :
  $\exists N \in \sN, \forall s \in \PHl, \forall \delta \in \reductionsce(s),
    |\delta| \leq N$.
\end{condition}

Étant donnée une sorte $a \in \PHs$ et un état $s \in \PHl$,
on note $\pfp_s(a)$ (\defref{pfp}) l'ensemble des processus de la sorte $a$ qui peuvent être présents
après avoir joué tous les scénarios d'actions de priorité 1 depuis l'état $s$.
Cet ensemble est toujours défini du fait de la \crref{tf}.

\todo{Redéfinir $\pfp$ avec $\compin$ de façon à simplifier et rendre locale la définition ?}

\begin{definition}[$\pfp : \PHl \times \PHs \rightarrow \powerset(\PHproc)$]
\deflabel{pfp}
  Pour tout état $s \in \PHl$ et pour toute sorte $a \in \PHs$,
  \begin{align*}
    \pfp_s(a) = \{ \get{(s\play\delta)}{a} \in \PHl_a &\mid \delta \in \reductionsce(s)
          \wedge\nexists h\in\PHh^{(1)}, (\delta; h) \in \reductionsce(s) \}
  \end{align*}
\end{definition}

Nous définissons dans la suite la notion de \emph{composant bien formé} (\defref{component})
et de \emph{sorte coopérative bien formée} (\defref{cs}).
Un composant bien formé n'est frappé que par des actions secondaires,
car il ne subit que des influences «~biologiques~».
Une sorte coopérative bien formée n'est frappée que par des actions primaires «~non biologiques~»
qui convergent toujours vers le même processus en fonction de l'état des sortes qui l'influencent
(\defref{comp}), afin qu'elle représente chaque configuration de ces sortes par un unique processus.
La \crref{part} stipule alors que l'ensemble des sortes des Frappes de Processus canoniques
($\PHs$) est une partition entre un ensemble de composants bien formés (noté $\components$)
et un ensemble de sortes coopératives bien formées (noté $\cs$).

\begin{definition}[Composant bien formé ($\components$)]
\deflabel{component}
  Une sorte $a \in \PHs$ est un \emph{composant bien formé} si et seulement si :
    \[\forall h \in \PHh, \PHsort(\target{h}) = a \Rightarrow \prio(h) \geq 2 \enspace.\]
\end{definition}

\begin{definition}[Influence ($\compin : \PHs \to \powerset(\components)$)]
\deflabel{comp}
  Pour toute sorte $a \in \PHs$, on définit : $\compin(a) \DEF \conn(a) \cap \components$ où
  $\conn(a) \subset \PHs$ est le plus petit ensemble de sortes satisfaisant :
  \begin{align*}
    a \in \conn(a) & \\
    \forall h \in \PHh^{(1)},
      \sort{\target{h}} \in \conn(a) & \Rightarrow \sort{\hitter{h}} \in \conn(a)
  \end{align*}
\end{definition}

\begin{definition}[Sorte coopérative bien formée ($\cs$)]
\deflabel{cs}
  Une sorte $a \in \PHs$ est une \emph{sorte coopérative bien formée} si et seulement si :
  \begin{enumerate}[(i)]
    \item $\forall h \in \PHh, \sort{\target{h}} = a \Rightarrow \prio(h) = 1$,
    \item \label{csai} $\forall s \in \PHl, \card{\pfp_s(a)} = 1$,
    \item \label{css} $\forall a_i \in \PHl_a, \exists s \in \PHl, a_i \in \pfp_s(a)$,
    \item $\forall \mysigma \in \PHsubl[\PHl]_{\compin(a)}, \forall s, s' \in \PHl,
        (\mysigma \subseteq s \wedge \mysigma \subseteq s') \Rightarrow \pfp_s(a) = \pfp_{s'}(a)$.
  \end{enumerate}
\end{definition}

\begin{condition}[Partition des composants et des sortes coopératives]
\crlabel{part}
  \[\PHs = \components \cup \cs \wedge \components \cap \cs = \emptyset\]
\end{condition}

Il est à noter qu'une sorte $a$ qui n'est frappée par aucune action,
c'est-à-dire telle que $\forall h \in \PHh, \sort{\target{h}} \neq a$,
est en accord à la fois avec la définition de composant bien formé
et avec celle de sorte coopérative bien formée.
Cette sorte peut être arbitrairement et indifféremment affectée à $\components$ ou à $\cs$.
Un tel cas peut se produire pour des composants ayant un rôle d'«~entrée~»,
c'est-à-dire dont le niveau d'expression est uniquement déterminé par l'état initial,
ou par certaines constructions particulières de sortes coopératives créées par la traduction
\stodo{ref}.

Pour toute sorte $a \in \PHs$ et tout état $s \in \PHl$,
étant donné le point (\ref{csai}) de la \defref{cs}, on a toujours :
$\exists a_i \in \PHl_a, \pfp_s(a) = \{ a_i \}$.
On notera donc dans la suite : $\pfp_s(a) = a_i$.
De plus, du fait du point (\ref{css}) de la \defref{cs}, on introduit la notation
$\csState(a_i)$ permettant de caractériser l'ensemble des sous-états représentés par le
processus $a_i$ de toute sorte coopérative bien formée $a$ (\defref{csState}).

\todo{$\csState : \PHproc \rightarrow \powerset(\components)$ ?}
\begin{definition}[$\csState : \PHproc \rightarrow \powerset(\PHproc)$]
\label{def:csState}
  Pour tout $a \in \cs$ et $a_i \in \PHl_a$, 
    \[\csState(a_i) \DEF \{ \toset{ps} \mid ps \in \PHsubl[\PHl]_{\compin(a)} \wedge
      \exists s \in \PHl, (ps \subseteq s \wedge \pfp_s(a) = a_i) \}\]
\end{definition}

Dans la suite, on écrira simplement «~composant~» (\resp «~sorte coopérative~»)
en lieu et place de «~composant bien formé~» (\resp «~sorte coopérative bien formée~»).

\begin{example}
  Les Frappes de Processus de la \figref{ph-livelock} comprennent trois composants ($a$, $b$ et $c$)
  et une sorte coopérative ($ab$).
  La sorte coopérative $ab$ modélise la coopération entre $a$ et $b$ car $\compin(ab) = \{ a, b \}$.
\end{example}

Enfin, la \defref{phcanonique} définit la notion de Frappes de Processus canoniques.
La forme particulière de ces Frappes de Processus permettra de développer une méthode efficace
d'analyse statique pour le problème d'atteignabilité défini à la \secref{sa},
grâce notamment aux résultats développés dans la section suivante.

\begin{definition}[Frappes de Processus canoniques]
\deflabel{phcanonique}
  Les \emph{Frappes de Processus canoniques}
  sont des Frappes de Processus avec 2 classes de priorités
  respectant les \allcr.
\end{definition}



\subsection{Conséquences des restrictions}

Dans cette section, nous donnons plusieurs résultats généraux qui peuvent être dérivés des
restrictions de la \secref{restrictions}.
Nous considérons donc dans la suite des Frappes de Processus avec $k$ classes de priorités
$\PH = (\PHs; \PHl; \PHa^{\langle k \rangle})$, avec $k \in \sN^*$,
respectant les \allcr.
\todo{Avec 2 classes de priorités ?}
Ces résultats permettront de construire la méthode d'analyse statique de la \stodo{section suivante}.

Pour tout état $s \in \PHl$, on appelle $\update(s)$ l'état dans lequel tous les composants
ont le même processus actif que dans $s$
mais où sortes coopératives ont été mises à jour en fonction (\defref{update}).
Cet état est unique du fait des propriétés de $\pfp$ données dans la section précédente.
Le \lemref{update} stipule ensuite que depuis tout état $s$, il existe un scénario d'actions primaires
mettant à jour toutes les sortes coopératives de façon à arriver dans $\update(s)$.
%ghtrjy <= Raphaël was here!

\begin{definition}[$\update : \PHl \rightarrow \PHl$]
\deflabel{update}
  Pour tout $s \in \PHl$, on définit :
  \begin{align*}
    \update(s) = s \Cap \{ \pfp_{s}(a) \mid a \in \cs \} \enspace.
  \end{align*}
\end{definition}

\begin{lemma}
\lemlabel{update}
  $\forall s \in \PHl, \exists \delta \in \reductionsce(s), s \PHplay \delta = \update(s)$
\end{lemma}

\begin{demo}[\Lemref{update}]
  Soit $s \in \PHl$ un état.
  Soit $a \in \cs$ une sorte coopérative telle que $\PHget{s}{a} \neq \pfp_s(a)$.
  Étant donnée la définition de $\pfp_s(a)$, il existe un scénario $\delta$ mettant à jour $a$
  dans $s$ tel que :
  $\forall \delta' \in \reductionsce(s \PHplay \delta),
    \PHget{(s \PHplay \delta \PHplay \delta')}{a} = \pfp_s(a)$.
  \todo{À revoir}
\end{demo}

Le \lemref{hcompcomp} stipule que pour un état donné $s \in \PHl$, et pour toute action secondaire
$h = \PHhit{a_i}{b_j}{b_k} \in \PHh$ où $a$ et $b$ sont des composants,
si $\PHget{s}{a} = a_i$ et $\PHget{s}{b} = b_j$, alors
$h$ peut toujours être jouée après une série d'actions primaires,
et ces actions n'empêchent pas son jeu.

De façon complémentaire, le \lemref{hcscomp} énonce le même résultat si $a$ est une sorte coopérative,
sous la condition que $a$ soit déjà à jour dans $s$.

\begin{lemma}
\lemlabel{hcompcomp}
  $\forall s \in \PHl, \forall a,b \in \components, \forall h = \PHhit{a_i}{b_j}{b_k} \in \PHh,$\\
  $(\PHget{s}{a} = a_i \wedge \PHget{s}{b} = b_j) \Rightarrow
    (\exists \delta \in \reductionsce(s),
%    (s \PHplay \delta) \PHtrans (s \PHplay \delta \PHplay h))$
    \Feval{\Fopphp{h}}{s \PHplay \delta})$
\end{lemma}

\begin{demo}[\Lemref{hcompcomp}]
  Soient $s \in \PHl$ un état, $a,b \in \components$ deux composants et
  $h = \PHhit{a_i}{b_j}{b_k} \in \PHh$ une action.
  D'après le \lemref{update}, il existe un scénario $\delta$ tel que :
  $(s \PHplay \delta) = \update(s)$.
  Étant donné que $a$ et $b$ sont des composants,
  $a_i \in (s \PHplay \delta)$ et $b_j \in (s \PHplay \delta)$.
  De plus, par définition de $\update(s)$, aucune action primaire n'est jouable
  dans $(s \PHplay \delta)$.
  $h$ est donc jouable dans $(s \PHplay \delta)$.
\end{demo}

\begin{lemma}
\lemlabel{hcscomp}
  $\forall s \in \PHl, \forall a \in \cs, \forall b \in \components,
    \forall h = \PHhit{a_i}{b_j}{b_k} \in \PHh,$\\
  $(\PHget{s}{a} = a_i \wedge \PHget{s}{b} = b_j \wedge \pfp_s(a) = a_i) \Rightarrow
    (\exists \delta \in \reductionsce(s),
%    (s \PHplay \delta) \PHtrans (s \PHplay \delta \PHplay h))$
    \Feval{\Fopphp{h}}{s \PHplay \delta})$
\end{lemma}

\begin{demo}[\Lemref{hcscomp}]
  Similaire à la preuve du \lemref{hcompcomp} ;
  étant donné que $a_i \in \pfp_s(a)$, on a : $a_i \in (s \PHplay \delta)$.
\end{demo}


%Équivalence avec priorités quelconques
\section{Équivalence avec les autres formalismes de Frappes de Processus}
\seclabel{phcanonique-equiv}

Cette section vise à tracer des liens entre les différentes sémantiques des Frappes de Processus
présentées au \chapref{sem}.
Son principal apport est l'\emph{aplatissement} des Frappes de Processus avec $k$
classes de priorités, donné à la \secref{aplatissement},
et qui permet de les traduire en Frappes de Processus canoniques.
Cette traduction permet donc, à partir d'un modèle de Frappes de Processus comprenant un nombre
arbitraire de priorités, d'obtenir un modèle canonique respectant la même dynamique.
Par la suite, le cas des Frappes de Processus avec arcs neutralisants est aussi traité,
de façon analogue, dans la \secref{phan-aplatissement}.
Enfin, la \secref{phm-aplatissement} réutilise ces résultats pour proposer aussi une traduction
depuis les Frappes de Processus avec actions plurielles.

Ces différentes traductions et propriétés de bisimulation (faible) qu'elles proposent
permettent d'établir que les différentes sémantiques de Frappes de Processus sont
aussi expressives que les Frappes de Processus canoniques.
Notamment, cela nous permet d'assurer que les Frappes de Processus avec plusieurs
classes de priorités sont équivalentes ;
autrement dit, considérer plus de deux classes de priorités n'augmente pas l'expressivité
---~bien que cela puisse faciliter la modélisation.



\subsection{Aplatissement des Frappes de Processus avec $k$ classes de priorités}
\seclabel{aplatissement}

Le but de cette section est de montrer qu'un modèle de Frappes de Processus avec $k$ classes
de priorités peut être \emph{aplati}, c'est-à-dire traduit en un autre modèle ne comportant
que 2 classes de priorités.
Ce dernier modèle est assuré de posséder la même dynamique,
car il lui est faiblement bisimilaire,
comme établi par le \vthmref{bisimulaplatissement}.
De plus, les actions de priorité 1 (les plus prioritaires) ne sont utilisées que pour mettre
à jour les sortes coopératives ;
il s'agit en fait d'un modèle de Frappes de Processus canoniques telles que définies
à la \secref{phcanonique-def}.
La forme particulière de ces modèles permet d'y appliquer les méthodes d'analyse statique
développées à la \secref{as}.
Cela nous permet de plus de montrer que les Frappes de Processus avec $k$ classes de priorités
sont aussi expressives entre elles pour tout $k \in \sN^*$,
car elles sont toutes aussi expressives que les Frappes de Processus canoniques.

Étant donné que les propriétés de jouabilité n'utilisent que des opérateurs de logique booléenne
standards, il est possible de calculer la forme normale disjonctive (FND) de toute propriété de
jouabilité. Pour toute action $h \in \PHh$, cette FND est de la forme :
\[\Fopphp{h} \equiv \bigvee_{i \in \segm{1}{\n}}
  \left( \bigwedge_{j \in \segm{1}{\m}} p_{i,j} \right)\]
où $\n \in \sN$ et $\forall i \in \segm{1}{\n}, \m \in \sN^*$.
Si $\n = 0$, alors $\Fopphp{h} \equiv \bot$, ce qui signifie que l'action $h$ ne peut jamais
être jouée car elle est préemptée dans dans tous les états où son frappeur et sa cible
sont présents.
Une telle action peut être retirée du modèle sans en changer le comportement.
En revanche, si $\n > 0$, alors $\Fopphp{h} \not\equiv \bot$ ;
dans ce cas, $\Fopphp{h}$ est une disjonction de $\n$ conjonctions d'atomes,
et peut donc être vue comme une disjonction de $\n$ propriétés de jouabilité plus petites.
Ces $\n$ conjonctions d'atomes peuvent être traduites en autant de sortes
coopératives priorisées, afin d'obtenir une dynamique équivalente avec un nombre réduit
de classes de priorités utilisées.
Dans ce second cas, on note, pour tout $i \in \segm{1}{\n}$ :
$\PHdep{i}{h} = \{ \PHsort(p_{i,j}) \mid j \in \segm{1}{\m} \}$.

L'opérateur d'aplatissement $\Fopsymbol_\Fopaplatsubsymbol$ donné à la \defref{fopaplat}
permet de caractériser la jouabilité d'une action sans prendre en compte les actions primaires ;
en d'autres termes, une action $h$ est jouable dans un état $s$ si et seulement si
$\Feval{\Fopaplat{h}}{s}$ et aucune action de priorité 1 n'est jouable.
Le \lemref{ppplaysubset} permet alors de caractériser la jouabilité d'une action dans un état
à l'aide d'un sous-état correspondant à l'une des conjonctions de sa propriété
d'aplatissement une fois traduite en FND.
Enfin, la \defref{aplatissement} donne la construction de l'\emph{aplatissement} de $\PH$ :
pour chaque action $h \in \PHh$, plusieurs sortes coopératives $f^{h,i}$
permettent de refléter chaque conjonction de $\Fopphp{h}$,
c'est-à-dire une pour chaque indice $i \in \segm{1}{\n}$.
Cette construction permet d'obtenir la même dynamique que pour $\PH$ en reproduisant
les préemptions possibles par d'autres actions plus prioritaires,
comme établi par le \thmref{bisimulaplatissement}.

Il est à noter que la \defref{aplatissement} ne s'applique qu'à des Frappes de Processus
pseudo-canoniques. Cependant, des Frappes de Processus avec $k$ classes de priorités quelconques
sont \textit{a fortiori} des Frappes de Processus pseudo-canoniques à condition d'ajouter
une classe vide d'actions de priorité supérieure.
En d'autres termes, il est possible d'ignorer le cas particulier des actions primaires
(mettant à jour les sortes coopératives) si les Frappes de Processus ne sont pas
pseudo-canoniques.

%\todo{À supprimer ? On peut l'intégrer dans la démo du \thmref{bisimulaplatissement}.}

\begin{definition}[Opérateur d'aplatissement
  ($\Fopsymbol_\Fopaplatsubsymbol : \PHh \rightarrow \F$)]
\deflabel{fopaplat}
  L'opérateur d'aplatissement des Frappes de Processus avec $k$ classes de priorités
  est défini par :
  \[\forall h \in \PHh, \Fopaplat{h} \equiv \hitter{h} \wedge \target{h} \wedge
    \left( \bigwedge_{\substack{g \in \PHh^{(n)}\\1 < n < \prio(h)}}
    \neg \left( \hitter{g} \wedge \target{g} \right) \right)\]
\end{definition}

\begin{lemma}
\lemlabel{ppplaysubset}
  Soient $h \in \PHh \setminus \PHh^{(1)}$ et $s \in \PHl$.
  \[\Feval{\Fopphp{h}}{s} \Leftrightarrow
    \big(\exists \mysigma \subseteq s, \Feval{\Fopaplat{h}}{\mysigma}\big)
    \wedge \big(\forall g \in \PHh^{(1)}, \Feval{\neg \Fopphp{g}}{s}\big)
    \enspace.\]
\end{lemma}
%
\begin{proof}
  ($\Rightarrow$)
    Supposons $\Feval{\Fopphp{h}}{s}$.
    Il n'existe donc aucune action primaire jouable dans $s$.
    Par ailleurs, $\Fopphp{h} \Rightarrow \Fopaplat{h}$ donc
    $\Fopaplat{h} \not\equiv \bot$ et, par propriété d'une FND,
    au moins l'une des conjonctions de $\Fopaplat{h}$ est vraie dans $s$.
    On suppose que la $i$\textsuperscript{e} conjonction est vraie dans $s$,
    avec $i \in \segm{1}{\n}$;
    on a alors : $\forall j \in \segm{1}{\m}, p_{i,j} \in s$.
    Soit $\mysigma \in \PHsubl_{\PHdep{i}{h}}$
    avec $\forall b \in \PHdep{i}{h}, \PHget{\mysigma}{b} = \PHget{s}{b}$.
    On a alors immédiatement : $\mysigma \subseteq s$,
    et, par construction de $\PHdep{i}{h}$, $\Feval{\Fopaplat{h}}{\mysigma}$.
  
  ($\Leftarrow$)
    Supposons qu'il existe $\mysigma \subseteq s$ tel que $\Feval{\Fopaplat{h}}{\mysigma}$,
    et qu'aucune action primaire n'est jouable dans $s$.
    On a alors immédiatement $\Feval{\Fopaplat{h}}{s}$
    car ajouter des processus au sous-état
    d'évaluation ne peut pas rendre la propriété fausse.
    De plus, comme aucune action primaire n'est jouable dans $s$, alors
    $\Feval{\left( \bigwedge_{g \in \PHh^{(1)}} \neg (\hitter{g} \wedge \target{g}) \right)}{s}$,
    donc $\Feval{\Fopphp{h}}{s}$.
\end{proof}

\begin{definition}[Aplatissement ($\PHflat$)]
\deflabel{aplatissement}
  Si $k \in \sNN$ et $\PH = (\PHs; \PHl; \PHa^{\angles{k}})$
  sont des Frappes de Processus pseudo-canoniques avec $k$ classes de priorités,
  on note $\PHflat(\PH) = \oPH = (\ov{\PHs}; \ov{\PHl}; (\ov{\PHa}^{(1)}; \ov{\PHa}^{(2)}))$
  l'\emph{aplatissement} de $\PH$, où :
  \begin{itemize}
    \item $\ov{\PHs} = \PHs \cup \PHs_f$
      où $\PHs_f = \{ f^{h,i} \mid h \in \PHh \wedge \n \geq 1 \wedge i \in \segm{1}{\n} \}$;
    \item $\ov{\PHl} = \left( \bigtimes{a \in \PHs} \PHl_{a} \right) \times
      \left(\bigtimes{f^{h,i} \in \PHs_f} \PHl_{f^{h,i}} \right)$, où
      $\forall f^{h,i} \in \PHs_f, \PHl_{f^{h,i}} =
        \{ f^{h,i}_\mysigma \mid \mysigma \in \PHsubl_{\PHdep{i}{h}} \}$;
    \item $\ov{\PHh}^{(1)} = \PHh^{(1)} \cup
      \{ \PHhit{a_k}{f^{h,i}_\mysigma}{f^{h,i}_{\mysigma'}} \mid
      h \in \PHh \wedge f^{h,i} \in \PHs_f \wedge
      a \in \PHdep{i}{h} \wedge a_k \in \PHl_a \wedge
      f^{h,i}_\mysigma , f^{h,i}_{\mysigma'} \in \PHl_{f^{h,i}} \wedge
      \PHget{\mysigma}{a} \neq a_k \wedge \mysigma' = \mysigma \Cap \{ a_k \} \}$;
    \item $\ov{\PHh}^{(2)}=\{ \PHhit{f^{h,i}_\mysigma}{\target{h}}{\bounce{h}} \mid
      h \in \PHh \setminus \PHh^{(1)} \wedge f^{h,i} \in \PHs_f \wedge
      f^{h,i}_\mysigma \in \PHl_{f^{h,i}} \wedge \Feval{\Fopaplat{h}}{\mysigma} \}$.
  \end{itemize}
  De plus, étant donné un état $\os \in \ov{\PHl}$,
  on note $\unflats{\os} = s$ l'état correspondant dans $\PHl$ :
  $\forall a \in \PHs, \PHget{s}{a} = \PHget{\os}{a}$.
  À l'inverse, étant donné un état $s \in \PHl$,
  $\flats{s} = \os$ est l'état correspondant dans $\ov{\PHl}$ :
  $\forall a \in \PHs, \PHget{\os}{a} = \PHget{s}{a}$
  et $\forall f^{h,i} \in \PHs_f, \PHget{\os}{f^{h,i}} = f^{h,i}_\mysigma$
  avec $f^{h,i}_\mysigma \in \PHl_{f^{h,i}}$
  et $\forall b \in \PHdep{i}{h}, \PHget{\mysigma}{b} = \PHget{s}{b}$.
\end{definition}

Nous notons que l'aplatissement $\PHflat(\PH)$ de toutes Frappes de Processus avec $k$
classes de priorités $\PH = (\PHs, \PHl, \PHh^{\angles{k}})$
sont des Frappes de Processus canoniques.
En effet, une partie des sortes coopératives générées lors de cette traduction proviennent
des Frappes de Processus d'origine, qui sont pseudo-canoniques et sont déjà contraintes de la
même manière.
L'autre partie constitue les sortes coopératives de l'ensemble $\PHs_f$ et leur définition
respecte les \allcr.

\begin{theorem}[$\PH \approx \PHflat(\PH)$]
\thmlabel{bisimulaplatissement}
  Soient $\PH = (\PHs; \PHl; \PHa^{\angles{k}})$ des Frappes de Processus avec $k$
  classes de priorités,
  et $\oPH = \PHflat(\PH) = (\ov{\PHs}; \ov{\PHl}; \ov{\PHa}^{\angles{2}})$ leur aplatissement.
  \begin{enumerate}
    \item \label{php2ph} $\forall s, s' \in \PHl$,
      $s \trans{\PH} s' \Longrightarrow \flats{s} \mtrans{\oPH} \flats{s'}$,
      où $\mtrans{\oPH}$ est une séquence finie de transitions $\trans{\oPH}$.
    \item \label{ph2php} $\forall \os, \os' \in \ov{\PHl}$,
      $\os \trans{\oPH} \os' \Longrightarrow \unflats{\os} = \unflats{\os'} \vee
      \unflats{\os} \trans{\PH} \unflats{\os'} \enspace.$
  \end{enumerate}
\end{theorem}

\begin{proof}
  (\ref{php2ph}) Soient $s, s' \in \PHl$ tels que $s \trans{\PH} s'$.
    Posons $\os = \flats{s}$.
    D'après la dynamique des Frappes de Processus (\defref{play}),
    si $s \trans{\PH} s'$, alors il existe une action $h \in \PHh$ jouable dans $s$,
    telle que $s' = s \PHplay h$.
    On a alors : $\Feval{\Fopphp{h}}{s}$.
    Par construction de $\oPH$ (\defref{aplatissement}) :
    \begin{itemize}
      \item Si $h \in \PHh^{(1)}$, alors il existe $g = h \in \ov{\PHh}^{(1)}$,
        et on a : $\hitter{g} \in \os$ et $\target{g} \in \os$.
      \item Si, au contraire, $h \in \PHh \setminus \PHh^{(1)}$, alors il existe
        $g = \PHhit{f^{h,i}_\mysigma}{\target{h}}{\bounce{h}} \in \ov{\PHh}^{(2)}$.
        De plus, d'après le \lemref{ppplaysubset}, il existe $\mysigma \subseteq s$
        tel que $\Feval{\Fopaplat{h}}{\mysigma}$
        et, par construction de $\os$ (\defref{aplatissement}),
        $\PHget{\os}{f^{h,i}} = f^{h,i}_\mysigma$.
    \end{itemize}
    Dans les deux cas, $g$ est jouable dans $\os$.
    Par la suite, dans l'état $\os \PHplay g$, les seules actions jouables sont celles dans
    $\PHh^{(1)}$ qui mettent à jour les sortes coopératives dans lesquelles
    $\bounce{h} = \bounce{g}$ est impliqué, directement ou indirectement,
    permettant donc d'accéder à l'état $\flats{s'}$ en un nombre fini d'actions.
    Ainsi, $\flats{s} \mtrans{\oPH} \flats{s'}$.
  
  (\ref{ph2php}) soient $\os, \os' \in \ov{\PHl}$ tels que $\os \trans{\oPH} \os'$.
    Posons $s = \unflats{\os}$.
    D'après la dynamique des Frappes de Processus (\defref{play}),
    si $\os \trans{\PH} \os'$, alors il existe une action $g \in \ov{\PHh}$ jouable dans $\os$,
    telle que $\os' = \os \PHplay h$.
    \begin{itemize}
      \item Si $g \in \ov{\PHh}^{(1)} \setminus \PHh^{(1)}$,
        alors $\unflats{\os} = \unflats{\os'}$ car seul le processus actif d'une sorte
        coopérative dans $\PHs_f$ a évolué. % qui n'est pas dans $\PHs$ a évolué.
      \item Sinon, si il existe $h = g \in \PHh^{(1)}$,
        alors $h$ est jouable dans $s$ et : $\unflats{\os} \trans{\PH} \unflats{\os'}$
        avec $\unflats{\os'} = s \play h$.
      \item Autrement, si $g \in \ov{\PHh}^{(2)}$,
        on note : $g = \PHhit{f^{h,i}_\mysigma}{b_j}{b_k}$.
        Par construction de l'aplatissement (\defref{aplatissement}), il existe
        $h \in \PHh$ tel que $\Feval{\Fopaplat{h}}{\mysigma}$.
        Comme $g$ est jouable, cela signifie qu'aucune action de $\PHh^{(1)}$ n'est jouable,
        et notamment que la sorte coopérative $f^{h,i}$ est déjà mise à jour,
        ce qui a pour conséquence que : $\mysigma \subseteq s$.
        Ainsi, d'après \lemref{ppplaysubset}, $h$ est jouable dans $s$,
        car $\Feval{\Fopphp{h}}{s}$, et $\unflats{\os} \trans{\PH} \unflats{\os'}$.
    \end{itemize}
\end{proof}

Nous notons que l'aplatissement donné à la \defref{aplatissement}
est, pour chaque action, exponentiel dans le nombre d'actions non primaires de
priorité supérieure.
En effet, il est nécessaire pour chaque action de convertir la propriété de jouabilité
donnée par la \defref{fopphp} en une FND.
Or la majorité des cas pratiques pour cette conversion sont proches du pire cas,
qui est de complexité exponentielle en fonction du nombre d'atomes dans la propriété.

Pour finir, il est intéressant de noter que l'aplatissement de la \defref{aplatissement}
n'est pas optimal en nombre d'actions et de processus créés dans le modèle final.
En effet, il est possible de simplifier le modèle $\PHflat(\PH)$ de différentes manières,
qui n'ont pas été prises en compte ici pour ne pas alourdir la définition.
Nous donnons dans la suite quelques pistes pour obtenir des modèles plus simples
mais au comportement équivalent, mais n'en donnons pas les preuves.

\subsubsection*{Simplification des propriétés d'aplatissement}

La propriété d'aplatissement $\Fopaplat{h}$ d'une action $h = \PHhit{a_i}{b_j}{b_k}$ peut être
simplifiée à l'aide des propriétés suivantes, permettant d'éviter la création
de certains éléments inutiles, comme des actions qui ne sont jamais jouables
ou des coopérations qui sont toujours vraies :
\begin{itemize}
  \item il n'est pas nécessaire de faire apparaître la cible d'une action dans
    sa propriété d'aplatissement car sa présence sera vérifiée par ailleurs,
    au moment du tir effectif de l'action,
    d'où : $b_j \equiv \top$ ;
  \item Tout processus $b_l \neq b_j$ de la même sorte que la cible empêche toujours la jouabilité
    de l'action, donc : $b_l \equiv \bot$ ;
  \item si $c_p, c_q$ sont des processus différents ($c_p \neq c_q$) de la même sorte $c$,
    alors $c_p \wedge c_q \equiv \bot$.
\end{itemize}

\subsubsection*{Suppression des sortes coopératives superflues}

Il existe deux cas pour lesquels il est possible de supprimer la sorte coopérative $f^{h,i}$
créée pour une action $h$ dans le modèle aplati :
\begin{itemize}
  \item si $\Fopaplat{h} \equiv \top$, alors l'action $h$ peut être traduite comme étant
    une auto-action (car elle est toujours jouable dès lors que la cible est présente) ;
  \item Si la $i$\textsuperscript{e} conjonction de $\Fopaplat{h}$ consiste
    en un unique élément $p$, alors cette conjonction peut être traduite par une action simple
    de la forme : $\PHhit{p}{\target{h}}{\bounce{h}}$ sans avoir recours à une sorte coopérative
    (étant donné qu'en dehors de la cible, un seul processus, $p$, est requis).
\end{itemize}

\begin{example}
  Les Frappes de Processus avec 3 classes de priorités de la \figref{metazoan-php-bis}
  représentent un modèle de segmentation métazoaire inspiré de la \vfigref{metazoan-php}
  donc les deux actions suivantes ont été retirées :
    \[\PHfrappe{f_1}{f_1}{f_0} \quad \text{ et } \quad \PHfrappe{f_0}{c_1}{c_0}\]
  Cela permet de s'intéresser uniquement à l'état stationnaire du système,
  qui consiste en l'oscillation alternative des sortes $a$ et $c$.
  Ce modèle peut être aplati par la méthode donnée à la \vdefref{aplatissement}
  et les simplifications détaillées ci-dessus.
  Le résultat est le modèle de Frappes de Processus canoniques donné à la
  \figref{metazoan-phcanonique}.
  Le modèle complet (de la \figref{metazoan-php}) peut naturellement aussi être
  aplati à l'aide de la même méthode, mais le résultat est plus complexe du fait des
  deux actions «~peu urgentes~» supplémentaires.
  
  \begin{figure}[ht]
  \begin{center}
  \begin{tikzpicture}
    \TSort{(0,4)}{c}{2}{l}
    \TSort{(1,0)}{f}{2}{l}
    \TSort{(7,4)}{a}{2}{r}
    
    \TSetTick{fc}{0}{00}
    \TSetTick{fc}{1}{01}
    \TSetTick{fc}{2}{10}
    \TSetTick{fc}{3}{11}
    \TSort{(4,1)}{fc}{4}{r}
    
    \TAction{c_1}{a_1.west}{a_0.north west}{}{right}
    \TAction{f_1}{c_0.west}{c_1.south west}{bend left=30, in=90}{left}
    \TAction{c_1}{c_1.west}{c_0.north west}{selfhit}{right}
%     \TAction{f_1.north east}{f_1.south east}{f_0.north east}%
%       {selfhit, min distance=30, bend left, out=150, in=90}{left}
%     \TAction{f_0.east}{c_1.south east}{c_0.north east}{bend right=60, in=-140}{left}

    \TAction{fc_2}{a_0.west}{a_1.south west}{}{left}
    \path (1.8, 0.5) edge[coopupdate] (3.2, 2);
    \path (0.8, 4.5) edge[coopupdate] (3.2, 3);
    
    \node[labelprio1] at (2.55,3.85) {$1$}; % c => fc
    \node[labelprio1] at (2.75,1) {$1$};    % f => fc
    \node[labelprio2] at (5.5,3.85) {$2$};  % fc_10 -> a_0 / 1
    \node[labelprio2] at (3.5,5.3) {$2$};   % c_1 -> a_1 / 0
    \node[labelprio3] at (0,2.5) {$3$};     % f_1 -> c_0 / 1
    \node[labelprio3] at (0.8,5.8) {$3$};   % c_1 -> c_1 / 0
  \end{tikzpicture}
  \end{center}
  \caption{\figlabel{metazoan-php-bis}%
    Exemple de Frappes de Processus avec 3 classes de priorités,
    inspiré du modèle de la \figref{metazoan-ph},
    dont deux actions ont été retirées pour supprimer
    l'interruption de l'avancée du front d'onde $f$.
    Les étiquettes numérotées (de 1 à 3) placées contre les flèches représentant les actions
    symbolisent leur appartenance à une classe de priorités donnée.
  }
  \end{figure}

  \begin{figure}[ht]
  \begin{center}
  \begin{tikzpicture}
    %\path[use as bounding box] (-5.75,0) rectangle (5.75,5.5);
    \TSort{(-6,5)}{c}{2}{l}
    \TSort{(0,1)}{f}{2}{l}
    \TSort{(6,5)}{a}{2}{r}

    \TSetTick{fc}{0}{00}
    \TSetTick{fc}{1}{01}
    \TSetTick{fc}{2}{10}
    \TSetTick{fc}{3}{11}
    \TSort{(3,0)}{fc}{4}{r}
    
    \TSetTick{fa}{0}{00}
    \TSetTick{fa}{1}{01}
    \TSetTick{fa}{2}{10}
    \TSetTick{fa}{3}{11}
    \TSort{(-3,0)}{fa}{4}{l}
    
    \THit{fc_2}{}{a_0}{.south west}{a_1}
    \path[bounce, bend left=60]
      \TBounce{a_0}{}{a_1}{.south west};
    
    \THit{c_1.north east}{}{a_1}{.west}{a_0}
    \path[bounce, bend right=50]
      \TBounce{a_1}{}{a_0}{.north west};
    
    \THit{a_0.west}{}{c_1}{.east}{c_0}
    \path[bounce, bend left=50]
      \TBounce{c_1}{}{c_0}{.north east};
    
    \THit{fa_3}{}{c_0}{.east}{c_1}
    \path[bounce, bend right=50]
      \TBounce{c_0}{}{c_1}{.south east};
    
    \path (0.8, 1.5) edge[coopupdate] (2.2, 1.5);
    \path (-5.3, 5.5) edge[coopupdate] (2.2, 2.5);
    \path (-0.8, 1.5) edge[coopupdate] (-2.2, 1.5);
    \path (5.3, 5.5) edge[coopupdate] (-2.2, 2.5);
    
    \node[labelprio1] at (-1.5,3.2) {$1$};
    \node[labelprio1] at (1.5,3.2) {$1$};
    \node[labelprio1] at (-1.5,1.9) {$1$};
    \node[labelprio1] at (1.5,1.9) {$1$};
    
    \node[labelprio2] at (0,6.4) {$2$};
    \node[labelprio2] at (1,5.7) {$2$};
    \node[labelprio2] at (-4.8,3.9) {$2$};
    \node[labelprio2] at (5,3.5) {$2$};
  \end{tikzpicture}
  \end{center}
  \caption{\figlabel{metazoan-phcanonique}%
    Frappes de Processus canoniques issues de l'aplatissement du modèle
    de la \vfigref{metazoan-php}.
    Les étiquettes numérotées (1 et 2) placées contre les flèches représentant les actions
    symbolisent leur appartenance à une classe de priorités donnée.
  }
  \end{figure}
\end{example}



\subsection{Aplatissement des Frappes de Processus avec arcs neutralisants}
\seclabel{phan-aplatissement}

À l'instar de la \secref{aplatissement}, il est possible de traduire les Frappes de Processus
avec arcs neutralisants en Frappes de Processus canoniques.
Le procédé est le même car il consiste, pour chaque action,
en un calcul de propriété d'aplatissement
qui est ici identique à la propriété de jouabilité.
En effet, l'opérateur d'aplatissement de cette traduction est égal à l'opérateur de jouabilité
donné à la \vdefref{fopphan} : $\Fopsymbol_\Fopaplatsubsymbol = \Fopsymbol_\Fopphansubsymbol$.
Par la suite, une fois cette propriété traduite en FND, il est possible de réutiliser
la traduction de la \defref{aplatissement} afin d'obtenir un modèle canonique
ayant la même dynamique, comme assuré par le \thmref{bisimulaplatissement}.

Cette traduction est elle aussi de complexité exponentielle dans le nombre d'actions
préemptant chaque action.
Cependant, on note que l'utilisation d'arcs neutralisants peut rendre cet aplatissement
beaucoup plus efficace.
En effet, contrairement aux Frappes de Processus avec classes de priorités,
les Frappes de Processus avec arcs neutralisants permettent une définition beaucoup plus fine
des préemptions entre actions.
Une des conséquences sur les modèles créés est un nombre bien moins important de relations
préempteur/préempté entre les actions, rendant la traduction plus efficace.



\subsection{Aplatissement des Frappes de Processus avec actions plurielles}
\seclabel{phm-aplatissement}

Il est possible de traduire les Frappes de Processus avec actions plurielles
en Frappes de Processus canoniques à l'aide d'outils précédemment développés.
En effet, la \vdefref{phm2php} offre une traduction des Frappes de Processus avec action
plurielles
en Frappes de Processus avec 4 classes de priorités, celles-ci pouvant être à leur tour traduites
en Frappes de Processus canoniques à l'aide de la \vdefref{aplatissement}.
Globalement, cette traduction est donc exponentielle dans le nombre d'actions dans le modèle
initial, car la traduction de chacune d'entre elles crée une sorte coopérative et plusieurs
actions de priorités différentes, qui doivent par la suite être aplaties.

\todoplustard{Conversion directe lorsque $\forall h \in \PHh, \card{\bond{h}} = 1$ ?}



\subsection{Représentation en Frappes de Processus avec actions plurielles}
\seclabel{phcanonique2phm}

Les Frappes de Processus canoniques permettent notamment de représenter des coopérations
entre processus.
Il peut être intéressant d'observer ces coopérations du point de vue des
Frappes de Processus avec actions plurielles, dont le formalisme est particulièrement
adapté à la représentation des coopérations.
Pour toutes Frappes de Processus canoniques $\PH$, nous proposons à la \defref{phcanonique2phm}
une représentation alternative $\PHmult(\PH)$ de ce modèle en Frappes de Processus avec
actions plurielles, et nous montrons au \thmref{bisimulphm} que $\PHmult(\PH)$
possède bien la même dynamique que $\PH$, aux mises à jour de sortes coopératives près.
Cette traduction se base sur l'interprétation de chaque action de priorité 2 du modèle $\PH$ :
\begin{itemize}
  \item Si le frappeur de l'action est un processus de composant ($\components$),
    la traduction est alors triviale ;
  \item Si à l'inverse, le frappeur est un processus de sorte coopérative ($\cs$),
    alors la sorte coopérative est étudiée et il y a autant d'actions multiples
    créées qu'il y a de configurations représentées par ce frappeur.
\end{itemize}
Autrement dit, les actions plurielles permettent d'avantageusement représenter
les sortes coopératives en représentant les ensembles de processus requis pour activer
un processus donné d'une sorte coopérative.
La \figref{livelock-phm} est la traduction en Frappes de Processus avec actions plurielles
du modèle simple de Frappes de Processus canoniques de la \figref{livelock}.

\begin{definition}
\deflabel{phcanonique2phm}
  Soient $\PH = (\PHs; \PHl; (\PHh^{(1)}; \PHh^{(2)}))$ des Frappes de Processus canoniques.
  On pose : $\PHmult(\PH) = (\ov{\PHs}; \ov{\PHl}; \ov{\PHh})$ les Frappes de Processus
  avec actions plurielles telles que :
  \begin{itemize}
    \item $\ov{\PHs} = \components$ ;
    \item $\ov{\PHl} = \bigtimes{a \in \PHs'} \PHl_a$ ;
    \item $\ov{\PHh} = \{ \PHfrappemult{\big( ps \cup \{ \target{h} \} \big)}{\bounce{h}}
      \mid h \in \PHh^{(2)} \wedge ps \in \virtualhitters(h) \}$
    avec, si on note $\hitter{h} = a_i$ :
    \[\virtualhitters(h) =
      \begin{cases}
        \{ \{ a_i \} \} & \text{si } a \in \components \\
        \csState(a_i) & \text{si } a \in \cs
      \end{cases}\]
  \end{itemize}
  De plus, pour tout état $s \in \PHl$,
  $\flats{s} = \os$ est l'état correspondant dans $\ov{\PHl}$ :
  $\forall a \in \ov{\PHs}, \PHget{\os}{a} = \PHget{s}{a}$.
  À l'inverse, étant donné un état $\os \in \ov{\PHl}$,
  on note $\unflats{\os} = s$ l'état correspondant dans $\PHl$ :
  $\forall a \in \components, \PHget{s}{a} = \PHget{\os}{a}$
  et $\forall f \in \cs, \PHget{s}{f} = f_\mysigma$
  avec $f_\mysigma \in \PHl_f$
  et $\forall b \in \compin(b), \PHget{\mysigma}{b} = \PHget{s}{b}$.
\end{definition}

\begin{theorem}[$\PH \approx \PHmult(\PH)$]
\thmlabel{bisimulphm}
  Soient $\PH = (\PHs; \PHl; (\PHh^{(1)}; \PHh^{(2)}))$ des Frappes de Processus canoniques,
  et $\PHmult(\PH) = (\ov{\PHs}; \ov{\PHl}; \ov{\PHh})$.
  \begin{enumerate}
    \item \label{bisimulph2phm} $\forall s, s' \in \PHl$,
      $s \trans{\PH} s' \Longrightarrow
      \flats{s} = \flats{s'} \vee \flats{s} \trans{\PHmult(\PH)} \flats{s'}$,
    \item \label{bisimulphm2ph} $\forall \os, \os' \in \ov{\PHl}$,
      $\os \trans{\PHmult(\PH)} \os' \Longrightarrow
      \unflats{\os} \mtrans{\PH} \unflats{\os'}$,
      où $\mtrans{\PH}$ est une séquence finie de transitions $\trans{\PH}$.
  \end{enumerate}
\end{theorem}

\begin{proof}
  (\ref{bisimulph2phm}) Soient $s, s' \in \PHl$ tels que $s \trans{\PH} s'$.
    Il existe donc une action $h \in \PHh$ telle que $s' = s \play h$.
    \begin{itemize}
      \item Si $h \in \PHh^{(1)}$, alors $\flats{s} = \flats{s'}$.
      \item Sinon, $h \in \PHh^{(2)}$ ; cela signifie qu'aucune action de priorité 1 n'est
        jouable, et qu'il existe $ps \in \virtualhitters(h)$ tel que
        $ps \cup \{ \target{h} \} \subseteq s$.
        D'après la \vdefref{phcanonique2phm}, il existe une action $g \in \ov{\PHh}$
        telle que $g = \PHfrappemult{ps \cup \{ \target{h} \}}{\bounce{h}}$.
        Ainsi, $g$ est jouable dans $\flats{s}$ et :
        $\flats{s} \play g = \flats{s} \recouvre \frappeur{h} = \flats{s'}$.
        Donc $\flats{s} \trans{\PHmult(\PH)} \flats{s'}$.
    \end{itemize}
    
  (\ref{bisimulphm2ph}) Soient $\os, \os' \in \ov{\PHl}$ tels que $\os \trans{\PHmult(\PH)} \os'$.
    Il existe donc une action $g \in \ov{\PHh}$ telle que $\os' = \os \play g$.
    De plus, cela signifie que $\frappeur{g} \subseteq \os$,
    donc $\frappeur{g} \subseteq \unflats{\os}$.
    Par construction de $\PHmult(\PH)$, il existe une action $h \in \PHh^{(2)}$
    telle que $\frappeur{g} = ps \cup \{ \cible{h} \}$ et $\bond{h} = \bond{g}$,
    avec $ps \in \virtualhitters(h)$.
    Par définition de $\unflats{\os}$, toutes les sortes coopératives sont mises à jour
    dans $\unflats{\os}$, ce qui fait que $h$ est jouable dans $\unflats{\os}$.
    Par ailleurs, d'après le \vlemref{update},
    il existe un scénario $\delta \in \Sce(\unflats{\os} \play h)$
    tel que $\unflats{\os} \play h \play \delta = \update(\unflats{\os} \play h)$.
    Enfin, par définition de $\update$ (\vdefref{update})
    on a : $\unflats{\os} \play h \play \delta = \unflats{\os'}$.
    Ainsi : $\unflats{\os} \mtrans{\PH} \unflats{\os'}$.
\end{proof}

\begin{figure}
  \centering
  \scalebox{1.2}{
  \begin{tikzpicture}[apdotsimple/.style={apdot}]
    \TSort{(0,0)}{a}{2}{l}
    \TSort{(0,3)}{b}{2}{l}
    \TSort{(4,2)}{c}{2}{r}
    
%     \THit{ab_3}{}{c_0}{.west}{c_1}
    
    \TActionPlur{a_1, b_1}{c_0.west}{c_1.south west}{}{2,2}{left}
    
%     \path[bounce, bend left]
%       \TBounce{c_0}{}{c_1}{.south west}
%     ;
    
    \TAction{a_1}{a_1.west}{a_0.north west}{selfhit}{right}
    \TAction{b_1}{b_1.west}{b_0.north west}{selfhit}{right}
    \TAction{a_0.south west}{b_0.west}{b_1.south west}{bend left=90}{left}
    \TAction{b_0}{a_0.west}{a_1.south west}{bend right=50}{left}
    
    \TState{a_0, b_0, c_0}
  \end{tikzpicture}
  }
  \caption{\figlabel{livelock-phm}%
    Frappes de Processus avec actions plurielles $\PHmult(\PH)$ issues de la traduction
    des Frappes de Processus canoniques $\PH$ de la \vfigref{livelock}
    d'après la \defref{phcanonique2phm}.
    Chaque action plurielle est représentée par un point relié aux frappeurs invariants
    par des arcs (sans flèche) et aux cibles et bonds par un couple de flèches
    (respectivement en trait plein puis en trait pointillé).
    Ici, l'action plurielle $\PHfrappemults{a_1, b_1, c_0}{c_1}$
    remplace la sorte coopérative $ab$
    et l'action $\PHfrappe{ab_{11}}{c_0}{c_1}$ de la \figref{livelock},
    qui permettaient de modéliser la coopération.
    Les processus grisés présentent un exemple d'état de départ :
    $\PHstate{a_0, b_0, c_0}$.
  }
\end{figure}



% Analyse statique

\section{Analyse statique}
\seclabel{as}

L'objectif de cette section est de définir le problème de l'\emph{atteignabilité} dans des
Frappes de Processus,
aussi appelée \emph{accessibilité},
et de proposer une sous-approximation permettant de la résoudre efficacement
dans les Frappes de Processus canoniques.

\myskip

Le problème de l'atteignabilité dans les Frappes de Processus consiste à rechercher l'existence
d'un scénario qui permette d'activer un ou plusieurs processus donné(s).
Il peut se résumer à la question suivante :
«~Étant donné un état initial, existe-t-il un scénario partant de cet état et
qui permette d'activer un processus donné ?~»
ou, de façon plus générale pour un ensemble de processus :
«~Étant donné un état initial $\ctx$ et une séquence de processus $\w$ donnés,
existe-t-il un scénario $\delta$ jouable dans $\ctx$ et permettant d'activer successivement
chacun des processus de $\w$ dans l'ordre ?~»
Ce problème d'atteignabilité peut parfois être résolu à l'aide des outils
de \textit{model checking} classiques.
Cependant, de telles méthodes reposent généralement sur l'analyse de la dynamique complète
du modèle.
Pour de grands modèles, ces méthodes se heurtent donc à l'explosion combinatoire inhérente
au calcul du graphe des états.

La méthode proposée ici repose en revanche sur une sous-approximation du modèle analysé.
Cela permet d'éviter la complexité exponentielle de l'analyse exhaustive de la dynamique,
car notre méthode possède une complexité polynomiale dans la taille du modèle
sous la condition que chaque sorte du modèle possède un nombre restreint de processus
(une sorte de quatre processus ou moins satisfaisant ce critère).
Cette méthode repose sur une succession d'analyses locales d'atteignabilités
qui se concentrent sur les sortes plutôt que sur le modèle complet.
Chaque atteignabilité est résolue sur une sorte en observant les actions qui permettent
de faire bondir le processus actif vers le processus recherché.
Comme ces actions sont éventuellement conditionnées par la présence d'un autre processus
d'une autre sorte, cela crée d'autres atteignabilités locales qui doivent être résolues.
Le problème est donc résolu récursivement, la condition d'arrêt étant soit
une atteignabilité locale impossible (ce qui peut empêcher de conclure),
soit un processus requis qui fait partie de l'état initial (ce qui consiste en une
atteignabilité locale \emph{triviale}).
Cette méthode est inspirée du travail de \citeasnoun{PMR12-MSCS},
qui portait sur l'atteignabilité dans les Frappes de Processus standards.
Les \defrangeref{obj}{maxCont} sont issues de la thèse de
\citefullname{Pauleve11}{Loïc}.
La contribution spécifique à cette thèse débute à la \defref{glc}
et comprend cette définition et tous les résultats suivants.

Un certain nombre d'outils préliminaires nécessaires à la résolution du problème
de l'atteignabilité dans des Frappes de Processus canoniques sont
définis à la \secref{sa-defs}.
Le mécanisme de résolution des atteignabilités locales mentionné au paragraphe précédent
est alors à son tour formalisé
à la \secref{ua} sous la forme d'un \emph{graphe de causalité locale} :
si ce graphe possède certaines propriétés, le \vthmref{approxinf} permet alors de conclure
quand à un problème d'atteignabilité donné.
Nous discutons aussi dans la suite du problèmes de l'atteignabilité
simultanée d'un ensemble de processus (\secref{as-etat}).
Nous proposons enfin une méthode permettant de raffiner cette approximation
dans le cas d'atteignabilités successives (\secref{approxinf-ordonnee}).

On considère dans toute la suite de cette section un modèle de Frappes de Processus canoniques
$\PH = (\PHs; \PHl; (\PHa^{(1)};\PHa^{(2)}))$
telles que définies à la \defref{phcanonique}.



\subsection{Définitions préliminaires}
\seclabel{sa-defs}

L'atteignabilité d'un processus $a_j$ d'une sorte $a$ donnée depuis un autre processus $a_i$
de la même sorte est le fait, depuis un état où $a_i$ est actif,
de pouvoir jouer un scénario menant dans un état où $a_j$ est actif.
La question de l'existence d'un tel scénario possède naturellement un intérêt particulier
dans la résolution d'une atteignabilité locale ; c'est pourquoi on la représente
sous la forme d'un \emph{objectif}, noté $\PHobjp{a}{i}{j}$ (\defref{obj}).
De plus, on appelle \emph{séquence d'objectifs} toute séquence dans laquelle
la cible de chaque objectif est égale au bond de l'objectif précédent de la même sorte
dans la séquence, s'il existe (\defref{os}).

\begin{definition}[Objectif ($\Obj$)]
\deflabel{obj}
  Si $a \in \components$, l'atteignabilité d'un processus $a_j$ depuis un processus $a_i$
  est appelé un \emph{objectif}, noté $\PHobj{a_i}{a_j}$.
  L'ensemble de tous les objectifs est noté
  $\Obj \DEF \{ \PHobj{a_i}{a_j} \mid
    a \in \components \wedge (a_i, a_j) \in \PHl_a \times \PHl_a \}$.
  Pour tout objectif $P = \PHobj{a_i}{a_j} \in \Obj$, on note
  $\sort{P} \DEF a$ la sorte de l'objectif $P$,
  $\target{P} \DEF a_i$ sa cible et $\bounce{P} \DEF a_j$ son bond.
  Enfin, $P$ est dit \emph{trivial} si $a_i = a_j$.
\end{definition}

\begin{definition}[Séquence d'objectif ($\Obj$)]
\deflabel{os}
  Une \emph{séquence d'objectifs} est une séquence $\w = P_1 \cons \ldots \cons P_{\card{\w}}$,
  où $\forall n \in \indexes{\w}, \w_n \in \Obj$
  et $\cible{\w_n} = a_i \Rightarrow \der{a}{\w_{1 \ldots n-1}} \in \{ \varnothing, a_i \}$.
  L'ensemble des séquences d'objectifs est référé par $\OS$.
%  Les définitions de $\premsymbol_a$ (\eqref{prem}), $\dersymbol_a$ (\eqref{der}),
%  $\suppsymbol$ (\eqref{supp}) et $\finsymbol$ (\eqref{fin}) sont étendues aux
  Les définitions de $\premsymbol_a$, $\dersymbol_a$,  $\suppsymbol$ et $\finsymbol$
  (\vdefref{premder}) sont étendues aux séquences d'objectifs en omettant
  de spécifier le cas des frappeurs.
\end{definition}

La \defref{ctx} introduit la notion de \emph{contexte} qui étend celle d'état
afin de pouvoir représenter un ensemble d'états initiaux possibles :
plutôt que d'attribuer un seul processus actif à chaque sorte, comme pour un état,
un contexte permet d'en attribuer plusieurs.
La notion de recouvrement, précédemment définie sur un états (\vdefref{recouvrement})
est étendue au cas d'un contexte dans la \defref{ctxrecouvrement}.
Il permettra à la \vdefref{glc} de saturer le contexte initial d'analyse
avec des processus supplémentaires.

\begin{definition}[Contexte ($\Ctx$)]
\deflabel{ctx}
  Un \emph{contexte} $\ctx$ associe à chaque sorte dans $\PHs$ un sous-ensemble non vide
  de ses processus :
  $\forall a \in \PHs, \PHget{\ctx}{a} \subseteq \PHl_a \wedge \PHget{\ctx}{a} \neq \emptyset$.
  On note $\Ctx$ l'ensemble de tous les contextes.
\end{definition}

\begin{definition}[Recouvrement ($\recouvre : \Ctx \times \powerset(\PHproc) \rightarrow \Ctx$)]
\deflabel{ctxrecouvrement}
  Pour tout contexte $\ctx \in \Ctx$ et tout ensemble de processus $ps \subset \Proc$,
  le recouvrement de $\ctx$ par $ps$ est noté $\ctx \recouvre ps$ et est défini par :
    \[ \forall a \in \PHs, \PHget{(\ctx \recouvre ps)}{a} \DEF
      \begin{cases}
        \{ p \in ps \mid \PHsort(p)=a \} & \text{si } \exists p \in ps, \PHsort(p)=a,\\
        \PHget{\ctx}{a} & \text{sinon.}
      \end{cases} \]
\end{definition}

Pour tout contexte $\ctx \in \Ctx$ et tout processus $a_i \in \Proc$, on note :
$a_i \in \ctx \EQDEF a_i \in \PHget{\ctx}{a}$,
et pour tout état $ps \in \PHl$ ou ensemble de processus $ps \subset \Proc$, on note :
$ps \subseteq \ctx \EQDEF \forall a_i \in ps, a_i \in \ctx$.
De plus, une séquence d'actions $\delta$ est \emph{jouable} dans un contexte $\ctx$
si et seulement si $\exists s \subseteq \ctx, \delta \in \Sce(s)$ ;
on note alors : $\delta \in \Sce(\ctx)$,
et le jeu de $\delta$ dans $\ctx$ est : $\ctx \PHplay \delta = \ctx \Cap \fin{\delta}$.

Finalement, une séquence de bonds sur une sorte $a$ (\defref{bs}) est une séquence d'actions
frappant $a$ dans laquelle le bond de chaque action est égal à la cible de l'action suivante,
en ignorant donc le frappeur de chaque action.
Les séquences de bonds sont utilisées pour trouver les solutions locales d'un objectif donné.
Une séquence de bonds sur $a$ peut de plus être abstraite par l'ensemble de tous les frappeurs
de ses actions qui ne sont pas dans $a$ (\defref{aBS}).
Cette abstraction permet de déplacer un objectif qui concerne une sorte $a$
vers d'autres objectifs sur d'autres sortes.
On note dans la suite : $\Sol = \powerset(\PHproc)$.

\begin{definition}[Séquence de bonds ($\BS$)]
\deflabel{bs}
  Une \emph{séquence de bonds} $\zeta$ est une séquence d'actions telle que
  $\forall n \in \indexes{\zeta}, n < \card{\zeta}, \PHbounce(\zeta_{n}) = \PHtarget(\zeta_{n+1})$.
  L'ensemble de toutes les séquences de bonds est appelé $\BS$,
  et on note $\BS(P)$ l'ensemble de toutes les séquences de bonds \emph{résolvant}
  un objectif $P$, appelé $\BS(P)$, qui est défini par :
    \[ \BS(\PHobj{a_i}{a_j}) \DEF \{ \zeta \in \BS \mid
      \PHtarget(\zeta_1) = a_i \wedge \PHbounce(\zeta_{\card{\zeta}}) = a_j \} \enspace. \]
\end{definition}

\noindent
On remarque que pour tout objectif $\obj{a_i}{a_j} \in \Obj$,
$\BS(\obj{a_i}{a_j}) = \emptyset$ s'il n'existe aucun moyen d'atteindre $a_j$ depuis $a_i$.
À l'inverse, la séquence vide appartient toujours à
l'ensemble des séquences de bonds résolvant un objectif trivial :
$\forall a_i \in \Proc, \emptyseq \in \BS(\obj{a_i}{a_i})$.

\begin{definition}[Séquence de bonds abstraite ($\aBS:\Obj \rightarrow \powerset(\Sol)$)]
\deflabel{aBS}
  \[
    \aBS(P) \DEF \{ \abstr{\zeta} \in \Sol \mid \zeta \in \BS(P), \nexists \zeta' \in \BS(P), \abstr{\zeta'} \subsetneq \abstr{\zeta} \} \enspace,
  \]
  où $\abstr{\zeta} \DEF \{ \PHhitter(\zeta_n) \mid  n \in \indexes{\zeta} \wedge \PHsort(\PHhitter(\zeta_n)) \neq \PHsort(P) \}$.
\end{definition}



\subsection{Sous-approximation}
\seclabel{ua}

On note $\concr(\w)$ l'ensemble des scénarios concrétisant
une séquence d'objectifs $\w$ dans le contexte $\ctx$ (\defref{concr})
et $\uconcr(\w)$ est défini comme étant égal à $\concr(\w)$ si et seulement si,
pour chaque état $s \subseteq \ctx$,
$\concr(\w) \cap \Sce(s) \neq \emptyset$ (\defref{uconcr}).

\begin{definition}[$\concr : \OS \to \powerset(\Sce)$]
\deflabel{concr}
  Pour toute séquence d'objectifs $\w \in \OS$, $\concr(\w)$ est l'ensemble des
  scénarios minimaux concrétisant $\w$ dans le contexte $\ctx$.
  Il est défini comme le plus grand ensemble satisfaisant les conditions suivantes :
  \begin{enumerate}[(i)]
  \item $\forall \delta \in \concr(\w), \exists s \subseteq \ctx, \delta \in \Sce(s)$,
  \item $\forall \delta \in \concr(\w), \exists \phi : \indexes{\w} \to \indexes{\delta},
      (\forall n, m \in \indexes{\w}, n < m \Leftrightarrow \phi(n) \leq \phi(m)),
      \forall n \in \indexes{\w}, \PHbounce(\w_n) \in \ctx \play \delta_{1 \ldots \phi(n)}$,
  \item $\forall \delta, \delta' \in \concr(\w),
      \card{\delta} \leq \card{\delta'} \Rightarrow \delta \neq \delta'_{1 \ldots \card{\delta}}$.
  \end{enumerate}
\end{definition}

\begin{definition}[$\uconcr : \OS \to \powerset(\Sce)$]
\deflabel{uconcr}
  \[ \uconcr(\w) \DEF
    \begin{cases}
      \concr(\w) & \text{si } \forall s \in \PHl, s \subseteq \ctx, \exists \delta \in \concr(\w),
        \delta \in \Sce(s) \\
      \emptyset & \text{sinon.}
    \end{cases} \]
\end{definition}

\begin{lemma}
\lemlabel{uconcr-ctx}
  $\ctx \subseteq \ctx' \wedge \muconcr_{\ctx'}(\w) \neq \emptyset \Rightarrow
    \muconcr_{\ctx}(\w) \neq \emptyset$.
\end{lemma}

Pour tout objectif $P \in \Obj$ et tout contexte $\ctx \in \Ctx$, la \defref{maxCont}
permet d'obtenir $\gCont_\ctx(\PHsort(P), P)$
qui est l'ensemble des processus de $\sorte{P}$ requis pour résoudre $P$ dans $\ctx$.
%appelé $\gCont_\ctx(\PHsort(P), P)$.
Cette définition sera utile pour correctement résoudre les atteignabilités locales
qui nécessitent indirectement un processus de leur propre sorte,
c'est-à-dire autrement que par une auto-action.

\begin{definition}[$\gCont_\ctx : \Sigma \times \Obj \to \powerset(\PHproc)$]
\deflabel{maxCont}
  \begin{align*}
    \gCont_\ctx(a,P) \DEF
    \{ p \in \PHproc &\mid \exists ps \in \aBS(P), \exists b_i \in ps, b = a \wedge p = b_i \\
      & \vee b \neq a \wedge p \in \gCont_\ctx(a, \PHobj{b_j}{b_i}) \wedge b_j \in \PHget{\ctx}{b} \}
    \enspace.
  \end{align*}
\end{definition}

Pour une séquence d'objectifs $\w$ et un contexte $\ctx$ donnés,
le \emph{graphe de causalité locale} $\cwB$ (\defref{glc}) représente une sous-approximation de
l'atteignabilité de cette séquence d'objectifs dans $\ctx$.
Pour cela, il relie les objectifs à des solutions à
l'aide des séquences de bonds abstraites de la \defref{aBS}, ce qui produit de nouveaux objectifs
résolus récursivement.
Il s'agit donc d'un graphe dont les nœuds sont des éléments de $\Proc \cup \Obj \cup \Sol$,
c'est-à-dire des processus, des objectifs et des \emph{solutions}
(c'est-à-dire des ensembles de processus) :
\begin{itemize}
  \item Un nœud dans $\Obj$ représente un objectif requis pour la résolution de $\w$,
    soit faisant directement partie de la séquence d'objectifs $\w$,
    soit indirectement nécessaire à sa résolution ;
  \item Un nœud dans $\Sol$ représente un ensemble de processus nécessaires pour résoudre
    un objectif, c'est-à-dire un élément parmi ses séquences de bonds abstraites ;
  \item Un nœud dans $\Proc$ représente un processus requis pour la résolution,
    c'est-à-dire mentionné dans un nœud solution.
\end{itemize}

Un objectif $P \in \Obj$ est soluble si tous les processus contenus dans au moins une de ses
abstractions de séquences de bonds $\aBS(P) \in \Sol$ (\cf \defref{aBS})
peuvent être activés (\eqref{ESol1}).
Une telle solution représente donc un ensemble de processus qui doivent être activés
pour la résolution de $P$ (\eqref{ESol2}).
Si $a \in \components$, l'atteignabilité d'un de ses processus $a_i$ est approximée par
la possibilité de résoudre tous les objectifs de la forme $\PHobjp{a}{j}{i} \in \Obj$
pour tout $a_j$ dans le contexte initial $\ctx$ (\eqref{EReq}) ;
si $a \in \cs$, l'atteignabilité de $a_i$ est possible si tous les processus du sous-état
$\csState(a_i)$ (\cf \defref{csState}) qu'il représente sont atteignables (\eqref{EPrio}).
La résolution d'un objectif $P$ peut nécessiter un processus $p$ de $\PHsort(P)$,
autrement dit : $\gCont(\PHsort(P), P) \neq \emptyset$ (\cf \defref{maxCont}) ;
dans ce cas, $P$ est \emph{re-centré} en $p$ (\eqref{ECont})
afin de s'assurer que la résolution intermédiaire de $\PHobj{\cible{P}}{p}$ n'interfère pas.
Enfin, les \eqref{Vw,Vproc,VE} assurent que l'ensemble des nœuds est complet.

Étant donné que le processus actif de chaque sorte peut évoluer au cours de la résolution,
le graphe de causalité locale $\cwB$ est obtenu par \emph{saturation} avec tous les processus
qu'il contient, \cad en recouvrant le contexte initial $\ctx$ par $\allprocs(\V, \E)$, défini par:
  \[\allprocs(\V, \E) = (V \cap \Proc) \cup
    \{ \PHtarget(P), \PHbounce(P) \mid P \in \V \cap \Obj \} \enspace.\]
Ce recouvrement est effectué autant de fois que nécessaire ;
le graphe de causalité locale est donc re-calculé avec cette saturation
jusqu'à ce qu'il n'évolue plus ---~autrement dit, jusqu'à atteindre un point fixe.

\begin{definition}
\deflabel{glc}
  Le graphe de causalité locale $\cwB \DEF (\Bv, \Be)$ est défini par :
  $\cwB \DEF \lfp{\aB^\w_\ctx}{\myB}{\aB^\w_{\ctx \Cap \allprocs(\myB)}}$,
  où $\myB \DEF (\cwV, \cwE)$ est le plus petit graphe respectant
  $\cwV \subseteq \Proc \cup \Obj \cup \Sol$ et $\cwE \subseteq \cwV \times \cwV$
  tel que :
  \begin{align}
    \w &\subseteq \cwV \label{eq:Vw} \\
    P \in \cwVObj &\Rightarrow \PHbounce(P) \in \cwV \label{eq:Vproc} \\
    (x, y) \in \cwE &\Rightarrow y \in \cwV \label{eq:VE} \\
    P \in \cwVObj \wedge ps \in \BS(P) &\Rightarrow (P, ps) \in \cwE \label{eq:ESol1} \\
    ps \in \cwVSol \wedge a_i \in ps &\Rightarrow (ps, a_i) \in \cwE \label{eq:ESol2} \\
    a \in \components \wedge a_i \in \cwVProc \wedge a_j \in \ctx &\Rightarrow (a_i, \PHobjp{a}{j}{i}) \in \cwE \label{eq:EReq} \\
    a \in \cs \wedge a_i \in \cwVProc \wedge ps \in \csState(a_i) &\Rightarrow (a_i, ps) \in \cwE \label{eq:EPrio} \\
    P \in \cwVObj \wedge q \in \gCont_\ctx(\PHsort(P), P) &\Rightarrow (P, \PHobj{q}{\PHbounce(P)}) \in \cwE \label{eq:ECont} \!
  \end{align}
\end{definition}

\begin{example}
\exlabel{livelock-glc}
  La \vfigref{glc-livelock} représente le graphe de causalité locale
  associé au modèle de Frappes de Processus canoniques de la \vfigref{livelock},
  pour la question de l'accessibilité de $c_1$ depuis l'état
  $\PHstate{a_1, b_0, c_0, ab_{10}}$.
  Nous discutons \vexpageref{livelock-as} des conclusions qui peuvent en être tirées.
\end{example}

Au sein de ce graphe de causalité locale, un arc $(p, ps) \in \Proc \times \Sol$
est dit \emph{cohérent} (\defref{coherent}) si aucun des processus dans $ps$
n'est «~compromis~» par un processus successeur du nœud $ps$,
%n'entre en conflit avec les processus successeur du nœud $ps$,
\cad si, pour tout processus de $ps$,
il n'existe pas de processus différent de la même sorte parmi tous les successeurs de $ps$.
%
%s'il n'existe pas un processus dans $ps$ et un processus parmi les successeurs de $ps$
%qui soient différents mais appartenant à la même sorte.
Si tous les arcs du graphe sont cohérents, alors le \thmref{approxinf}
donne une condition suffisante pour la concrétisation de la séquence d'objectifs $\w$
dans le contexte $\ctx$, qui est basée directement sur ce graphe $\cwB$.

\begin{definition}[Arc cohérent]
\deflabel{coherent}
  Un arc $(x, y) \in \cwE$ est dit \emph{cohérent} si et seulement si
  $(x, y) \in \Be \cap (\Proc \times \Sol) \Rightarrow y$ n'a aucun successeur
  $a_j \in \Bv \cap \Proc$
  tel que $\exists a_i \in y, \sorte{a_i} = \sorte{a_j} \wedge a_i \neq a_j$.
\end{definition}

\begin{theorem}[Sous-Approximation]
\thmlabel{approxinf}
  Étant données des Frappes de Processus canoniques $\PH = (\PHs; \PHl; (\PHa^{(1)};\PHa^{(2)}))$,
  un contexte $\ctx \in \Ctx$ et une séquence d'objectifs $\w \in \OS$,
  si le graphe $\cwB$ ne contient aucun cycle,
  que tous ses nœuds objectifs possèdent au moins une solution
  et que tous ses arcs sont cohérents,
  alors $\uconcr(\w) \neq \emptyset$.
\end{theorem}

\begin{proof} %[\Thmref{approxinf}]
  On note dans la suite :
  $\Bee{X}{Y} = \Be \cap (X \times Y)$, avec $X, Y$ parmi $\PHproc$, $\Obj$ et $\Sol$,
  et : $max\ctx = \ctx \Cap \allprocs(\cwB)$ le contexte accepté par $\cwB$.
  
  Soit $(a_i, ps) \in \Bee{\Proc}{\Sol}$ un arc liant un processus requis de sorte coopérative à
  une solution et supposons que tous les enfants de $ps$ sont concrétisables.
  On étiquette tous les processus de $ps$ par un entier : $ps = \{ p_n \}_{n \in \indexes{ps}}$.
  Montrons par récurrence que pour tout $n \in \segm{0}{\card{ps}}$,
  il existe un scénario $\delta_n$ tel que :
  $\forall i \in \segm{1}{n}, \PHget{(s \PHplay \delta_n)}{\PHsort(p_i)} = p_i$.
  \begin{itemize}
    \item Le cas $\delta_0 = \varepsilon$ est immédiat.
    \item Soit $n \in \segm{0}{\card{ps} - 1}$.
      On suppose qu'il existe $\delta_n$ tel que décrit ci-dessus.
      Posons $q = \PHget{(s \PHplay \delta_n)}{\PHsort(p_{n+1})}$.
      Par hypothèse, $(a_i, ps)$ est cohérent (\defref{coherent}) et tous les processus
      de $ps$ sont des processus de composants ;
      cela signifie qu'aucun des processus requis pour résoudre $p_{n+1}$ n'est un autre processus
      de la même sorte qu'un processus de $ps$.
      Par conséquent, il existe un scénario
      $\delta' \in \muconcr_{s \PHplay \delta_n}(\PHobj{q}{p_{n+1}})$
      tel que $\forall i \in \segm{1}{n+1},
        \PHget{(s \PHplay \delta_n \PHplay \delta')}{\PHsort(p_{i})} = p_{i}$.
      Finalement, d'après le \lemref{update}, il existe un scénario
      $\delta'' \in \reductionsce(s \PHplay \delta_n \PHplay \delta')$
      tel que $\update(s \PHplay \delta_n \PHplay \delta') = s \PHplay \delta_{n+1}$
      avec $\delta_{n+1} = \delta_n \PHplay \delta' \PHplay \delta''$,
      et d'après le \lemref{hcscomp} :
      $\forall i \in \segm{1}{n+1}, \PHget{(s \PHplay \delta_{n+1})}{\PHsort(p_i)} = p_i$
  \end{itemize}
  Ainsi, $\delta = \delta_{|ps|}$ existe, et étant données ses propriétés, on a immédiatement :
  $\PHget{(s \PHplay \delta)}{a} = a_i$ et $\update(s \PHplay \delta) = s \PHplay \delta$.
  
  Soit un état $s \in L$ tel que $s \subseteq max\ctx$.
  Étant donné qu'il n'y a aucun cycle dans $\cwB$, montrons par récurrence que
  pour tout objectif $P \in \Bv \cap \Obj$ tel que $\PHtarget(P) \in s$,
  $\exists \delta \in \muconcr_s(P)$.
  \begin{itemize}
    \item Si $(P, \emptyset) \in \Bee{\Obj}{\Sol}$,
      soit on a $\PHtarget(P) = \PHbounce(P)$ et $\delta = \emptyseq$,
      soit on a $\forall \zeta \in \BS(P), \zeta \in \Sce(s) \wedge \PHsort(\zeta) = \{ \PHsort(P) \}$
      et dans ce cas $\delta = \delta_1 \PHplay \zeta_1 \PHplay \dots \PHplay
        \delta_{|\zeta|} \PHplay \zeta_{|\zeta|}$
      est une séquence valide donnée par le \lemref{hcompcomp}.
    \item Supposons que tous les objectifs qui sont les successeurs de $P$ sont concrétisables.
      Si $\exists (P, Q) \in \Bee{\Obj}{\Obj}$, alors, par hypothèse,
        $\muconcr_{s}(\obj{\PHtarget(P)}{\PHtarget(Q)} \concat Q) \neq \emptyset$, et donc
        $\muconcr_{s}(P) \neq \emptyset$.
      Sinon, d'après la \defref{maxCont}, la concrétisation des successeurs de $P$ ne requiert
        aucun processus de la sorte $\PHsort(P)$.
        De plus, il existe $\zeta \in \BS(P)$ tel que $(P, \aZ) \in \Bee{\Obj}{\Sol}$.
        Montrons par récurrence que pour tout $n \in \indexes{\zeta}$, il existe un scénario
        $\delta_n$ tel que $\PHget{(s \PHplay \delta_n)}{\PHsort(P)} = \PHbounce(\zeta_n)$.
        \begin{itemize} %\item[$\circ$]
          \item[] Supposons que $\delta_n$ existe et posons $\zeta_n = \PHhit{b_i}{a_j}{a_k}$.
            Par hypothèse, il existe $\delta' \in \muconcr_{s \PHplay \delta_n}(\PHobj{\any}{b_i})$
            avec $\PHsort(P) \notin \PHsort(\delta')$ (\defref{maxCont}).
            D'après le \lemref{update}, il existe
            $\delta'' \in \reductionsce(s \PHplay \delta')$ tel que
            $\update(s \PHplay \delta') = s \PHplay \delta' \PHplay \delta''$.
            De plus, $\PHget{(s \PHplay \delta' \PHplay \delta'')}{b} = b_j$
            (D'après le \lemref{hcompcomp} si $b \in \components$
            ou le \lemref{hcscomp} si $b \in \cs$).
            Ainsi, $\delta_{n+1} = \delta_n \PHplay \delta' \PHplay \delta'' \PHplay \zeta_n$.
        \end{itemize}
      On a donc :$\delta_{|\zeta|} \in \muconcr_s(P)$. % and $\ceil(\delta) \subseteq max\ctx$.
  \end{itemize}
  Finalement, étant donné $\muconcr_{max\ctx}(\w) \neq \emptyset$,
  et d'après le \lemref{uconcr-ctx},
  on a : $\uconcr(\w) \neq \emptyset$.
\end{proof}



\begin{remark}
\label{subsetsolution}
  Le \thmref{approxinf} peut s'appliquer à tout graphe de causalité locale
  $\widehat{\cwB}$
  construit à partir d'un graphe $\widehat{\myB} = (\widehat{\cwV}, \widehat{\cwE})$
  où $\widehat{\cwV} \cap \Sol \subset \cwV \cap \Sol$.
  En effet, cela revient à réduire l'ensemble initial des solutions,
  ce qui réduit aussi l'ensemble de nœuds processus et objectifs utilisés.
  La solution est alors davantage contrainte, mais le résultat est toujours valable.
  Cela revient en fait à s'interdire certaines solutions,
  c'est-à-dire à calculer l'atteignabilité sur un graphe privé de certaines actions.
  Ainsi, si la sous-approximation est non-conclusive, il est possible de la tester
  sur tous les graphes comportant un sous-ensemble des nœuds solutions,
  ce qui permet notamment de supprimer certains cycles
  et parfois d'obtenir un graphe de causalité locale sur lequel il est possible de conclure.
  Cette recherche exhaustive est cependant exponentielle dans le nombre de nœuds solutions,
  mais il est possible de l'orienter de façon à trouver rapidement un sous-ensemble permettant
  de conclure,
  par exemple en retirant en priorité les solutions qui forment un cycle.
\end{remark}



\begin{remark}
  \citeasnoun{PMR12-MSCS} ont proposé une méthode de sur-approximation qui se base sur
  un graphe de causalité locale construit de façon similaire,
  et permet de réfuter une atteignabilité au sein d'un modèle de
  Frappes de Processus standards.
  Il est intéressant de noter que cette sur-approximation
  est toujours valable sur les Frappes de Processus canoniques à condition de l'appliquer
  sur la version fusionnée du modèle considéré (cf. \vdefref{aplatissement}).
  Cela permet d'obtenir un résultat supplémentaire en concluant
  dans certains cas quant à l'impossibilité d'atteindre un processus donné.
\end{remark}



\begin{example}
\exlabel{livelock-as}
  En ce qui concerne le modèle de Frappes de Processus canoniques de la \vfigref{livelock},
  la sous-approximation développée au \thmref{approxinf}
  ne conclut pas quant à l'accessibilité de $c_1$ depuis l'état
  $\PHstate{a_1, b_0, c_0, ab_{10}}$.
  En effet, comme le montre la \figref{glc-livelock},
  l'arc représenté en double trait liant le nœud processus $ab_{11}$ à son unique solution
  n'est pas cohérent selon la
  \defref{coherent}, ce qui empêche l'application du théorème.
  
  De même, la sur-approximation de \citeasnoun{PMR12-MSCS} (appliquée à $\PH'$)
  renvoie aussi un résultat non-conclusif,
  du fait que les deux approches ne peuvent pas être conclusives en même temps
  pour des raisons de cohérence mathématique.
  La méthode d'analyse statique ne répond donc globalement pas sur cet exemple,
  et de façon plus générale sur tous les exemples dont l'atteignabilité recherchée
  est rendue impossible par simple ajout de classes de priorités.
  
  On note pour finir que le \thmref{approxinf} est conclusif sur l'atteignabilité de $c_1$
  depuis $\PHstate{a_1, b_0, c_0, ab_{10}}$ dans les Frappes de Processus canoniques $\PH''$, où :
  \begin{align*}
  \PH'' &= (\PHs; \PHl; (\PHa^{(1)};\PHa'^{(2)})) \\
  \text{avec : } \quad
  \PHa'^{(2)} &= (\PHa^{(2)} \setminus \{ \PHhit{a_0}{b_0}{b_1}, \PHhit{b_0}{a_0}{a_1} \})
    \cup \{ \PHhit{a_0}{a_0}{a_1}, \PHhit{b_0}{b_0}{b_1} \}
  \end{align*}

  En effet, dans ce cas les processus $a_0$ et $b_0$ du graphe de la \figref{glc-livelock}
  sont permutés, ce qui rend tous les arcs cohérents.
  
  \begin{figure}[tp]
    \centering
    \begin{tikzpicture}[aS]
      \node[Aproc] (c1) {$c_1$};
      \node[Aobj,below of=c1] (c01) {$\PHobj{c_0}{c_1}$};
      \node[Asol,below of=c01] (c01s) {};

      \node[AprocPrio,below of=c01s] (ab11) {$ab_{11}$};
      \node[AsolPrio,below of=ab11] (ab11s) {};

      \node[Aproc,below left of=ab11s] (a1) {$a_1$};
      \node[Aobj,below of=a1] (a11) {$\PHobj{a_1}{a_1}$};
      \node[Asol,below of=a11] (a11s) {};
      \node[Aobj,below left of=a1] (a01) {$\PHobj{a_0}{a_1}$};
      \node[Asol,below of=a01] (a01s) {};
      \node[Aproc,below of=a01s] (b0) {$b_0$};
      \node[Aobj,below of=b0] (b00) {$\PHobj{b_0}{b_0}$};
      \node[Asol,below of=b00] (b00s) {};
      \node[Aobj,below left of=b0] (b10) {$\PHobj{b_1}{b_0}$};
      \node[Asol,below of=b10] (b10s) {};

      \node[Aproc,below right of=ab11s] (b1) {$b_1$};
      \node[Aobj,below of=b1] (b11) {$\PHobj{b_1}{b_1}$};
      \node[Asol,below of=b11] (b11s) {};
      \node[Aobj,below right of=b1] (b01) {$\PHobj{b_0}{b_1}$};
      \node[Asol,below of=b01] (b01s) {};
      \node[Aproc,below of=b01s] (a0) {$a_0$};
      \node[Aobj,below of=a0] (a00) {$\PHobj{a_0}{a_0}$};
      \node[Asol,below of=a00] (a00s) {};
      \node[Aobj,below right of=a0] (a10) {$\PHobj{a_1}{a_0}$};
      \node[Asol,below of=a10] (a10s) {};

      \path
      (c1) edge (c01)
      (c01) edge (c01s)
      (c01s) edge (ab11)
      (ab11) edge[aSPrio] (ab11s)
      (ab11s) edge (a1) edge (b1)

      (a1) edge (a01) edge (a11)
      (a01) edge (a01s)
      (a01s) edge (b0)
      (a11) edge (a11s)
      (a0) edge (a10) edge (a00)
      (a10) edge (a10s)
      (a00) edge (a00s)

      (b0) edge (b10) edge (b00)
      (b10) edge (b10s)
      (b00) edge (b00s)
      (b1) edge (b01) edge (b11)
      (b01) edge (b01s)
      (b01s) edge (a0)
      (b11) edge (b11s)
      ;
      \end{tikzpicture}
    \caption{\figlabel{glc-livelock}%
      Le graphe de causalité locale des Frappes de Processus de la \figref{livelock}
      pour l'objectif $\w = \PHobj{c_0}{c_1}$
      et le contexte initial $\ctx = \PHstate{a_1, b_0, c_0, ab_{10}}$.
      Les nœuds rectangulaires représentent les éléments de $\Proc$,
      les nœuds sans bordure sont les éléments de $\Obj$
      et les cercles sont les éléments de $\Sol$.
      Le processus $ab_{11}$, ainsi que son unique solution et l'arc qui les relie,
      sont mis en valeur avec des traits doubles car il s'agit du principal ajout de la méthode
      présentée à la \secref{ua}.
      Il est à noter que l'arc dessiné avec un trait double n'est pas cohérent
      au sens de la \defref{coherent}.
      En effet, sa cible est la solution $\{ a_1, b_1 \}$,
      or l'un de ses successeurs indirects est $a_0$, qui est un autre processus de la même sorte
      que $a_1$ (et le même raisonnement fonctionne pour $b_0$).
    }
  \end{figure}
\end{example}



Comme nous l'avons vu, l'analyse statique développée dans cette section est une approximation,
et peut retourner un résultat non-conclusif ;
Le modèle de la \figref{livelock} traité \vpageref[ci-dessus]{ex:livelock-as} en est un exemple.
Une partie de ces cas non-conclusifs apparaissent notamment pour un motif particulier,
mis en valeur par la notion de cohérence de la \defref{coherent}.
Cela est dû notamment au fait que la méthode de sur-approximation n'a pas été raffinée
dans le présent travail, ce qui mène à des cas non-conclusifs lorsque l'ajout de priorités
empêche certains comportements.

D'autres situations peuvent aussi empêcher de conclure : c'est notamment le cas des
atteignabilités nécessitant des «~allers-retours~»,
c'est-à-dire l'activation d'un processus $p$ plusieurs fois pendant la résolution.
Si d'autres requis sont intercalés entre les différentes occurrences de $p$,
le graphe de causalité locale va alors présenter un cycle, ce qui empêche l'utilisation
du \thmref{approxinf}.
L'une des alternatives est alors de détecter et d'expliciter cette séquentialité,
ce qui permet par exemple d'utiliser le résultat qui sera présenté à la
\vsecref{approxinf-ordonnee}.



\subsection{Atteignabilité d'un sous-état}
\seclabel{as-etat}

\newcommand{\uastotal}{\tau}
\newcommand{\uasreach}{\rho}
\newcommand{\uasps}{{ps}}

La propriété d'atteignabilité développée à la \secref{ua}
sur les Frappes de Processus canoniques
ne traite l'atteignabilité d'un ensemble de processus que de façon séquentielle.
Cependant, il est possible de vérifier l'atteignabilité d'un sous-état
(autrement dit, l'atteignabilité simultanée d'un ensemble de processus)
à l'aide d'une sorte coopérative.

En effet, soient $\PH = (\PHs, \PHl, (\PHh^{(1)}, \PHh^{(2)}))$ des Frappes de Processus
canoniques et supposons que l'ont cherche à étudier l'atteignabilité d'un sous-état
$\uasps \in \PHsubl_S$, avec $S \subset \PHs$.
On pose alors : $\PH' = (\PHs', \PHl', (\PHh'^{(1)}, \PHh'^{(2)}))$
les Frappes de Processus canoniques telles que :
\begin{itemize}
  \item $\PHs' = \PHs \cup \{ \uastotal, \uasreach \}$,
  \item $\PHl' = \PHl \times \PHl_\uastotal \times \PHl_\uasreach$, où
    $\PHl_\uastotal = \PHsubl_S$ et $\PHl_\uasreach = \{ \uasreach_0, \uasreach_1 \}$,
  \item $\PHh'^{(1)} = \PHh^{(1)} \cup
    \{ \PHfrappe{a_i}{\uastotal_\mysigma}{\uastotal_{\mysigma'}} \mid
    a \in S, \mysigma, \mysigma' \in \PHl_\uastotal,
    \PHget{\mysigma}{a} \neq a_i \wedge \mysigma' = \mysigma \recouvre a_i \}$,
  \item $\PHh'^{(2)} = \PHh^{(2)} \cup
    \{ \PHfrappe{\uastotal_\uasps}{\uasreach_0}{\uasreach_1} \}$.
\end{itemize}
Cette transformation consiste donc à ajouter au modèle
une sorte coopérative $\uastotal$ sur toutes les sortes de $S$,
et un composant $\uasreach$ qui ne puisse changer de processus que lorsque le sous-état $\uasps$
est présent (ce qui est déterminé par $\uastotal$).
Ainsi, l'atteignabilité du sous-état $\uasps$ depuis un contexte initial $\ctx$ dans $\PH$
est équivalente à celle du processus $\uasreach_1$ depuis le contexte
$\ctx \cup \{ \uasreach_0 \}$ dans $\PH'$
(l'état initial de $\uastotal$ n'a pas d'importance et peut être arbitrairement choisi),
qui peut être traitée grâce au \thmref{approxinf}.

Nous pouvons donc répondre à des questions d'atteignabilité simultanée de plusieurs processus
directement à l'aide des Frappes de Processus canoniques (\defref{phcanonique})
et de l'analyse statique développée pour ce formalisme (\thmref{approxinf}).
Cette méthode peut naturellement être adaptée pour répondre quant à
l'accessibilité d'un ensemble de sous-états.
Il est à noter cependant que le nombre de processus de la sorte coopérative $\uastotal$
croît exponentiellement avec le nombre de sortes dans $S$, ce qui peut fortement impacter
la vitesse de résolution de l'analyse statique.
Pour pallier cela, il est possible de «~factoriser~» cette sorte coopérative comme expliqué
\vpageref{factorisation-coop}.

\todoplustard{Exemple ! Lequel ?}



\subsection{Raffinement de la sous-approximation séquentielle}
\seclabel{approxinf-ordonnee}

Dans cette section, nous donnons une alternative à la condition suffisante du \thmref{approxinf}
qui permet de prendre en compte la séquentialité des objectifs plutôt que de les considérer
simultanément, tel que cela est fait dans la version actuelle.
Comme les objectifs sont pris en compte individuellement, une telle approche ne prend en compte
qu'un sous-ensemble des scénarios possibles.
Cependant, en se concentrant à chaque itération sur une plus petite partie du réseau, 
cette sous-approximation séquentielle peut s'avérer plus souvent conclusive.

Définissions une séquence d'objectifs $\w = \obj{a_i}{a_j} \concat \w'$ avec
$a_i \neq a_j$ et un état $s \in \PHl$ tel que $\get{s}{a} = a_i$.
On peut remarquer que tout scénario atteignant $a_j$ inclut nécessairement l'une des séquences
de bonds dans $\BS(\obj{a_i}{a_j})$ et, en particulier,
tout scénario minimal atteignant $a_j$ termine dans un état où son présents à la fois
$a_j$ et le frappeur $a_k$ de la dernière action d'une des séquences de bonds
dans $\BS(\obj{a_i}{a_j})$.
Si la sorte $b$ d'un tel frappe est de surcroît une sorte coopérative ($b \in \cs$),
cela signifie alors aussi que l'un des sous-états dans $\csState(b_k)$
est inclus dans l'état final.
La \defref{lastprocs} définit $\derprocs(\obj{a_i}{a_j})$ comme étant l'ensemble des ensembles
de processus qui peuvent être présents juste après avoir atteint $a_j$.

D'après le \thmref{approxinf}, on peut déduire que pour tout scénario $\delta \in \uconcr(P)$,
il existe un ensemble de processus $ps \in \derprocs(P)$ tel que $ps \subset (s \play \delta)$.
Donc, si $\muconcr_{\ctx' \Cap ps}(\w') \neq \emptyset$,
avec $\ctx' = \ctx \Cap \procs(\mycwB{\ctx}{P})$,
il existe alors un scénario $\delta'$ concrétisant $\w'$ depuis l'état $(s \play \delta)$.
Ainsi, le scénario  $\delta \concat \delta'$ concrétise $\w$.

\begin{definition}[$\derprocs : \Obj \to \powerset(\powerset(\Proc))$]
\deflabel{lastprocs}
  Pour tout objectif $\obj{a_i}{a_j} \in \Obj$, $\derprocs(\obj{a_i}{a_j})$
  est défini comme le plus grand ensemble tel que :
  $\forall lps \in \derprocs(\obj{a_i}{a_j}), lps \in \powerset(\Proc)$,
  \begin{enumerate}
    \item $a_j \in lps$,
    \item $\exists \zeta \in \BS(\obj{a_i}{a_j}),
        \sorte{\hitter{\zeta_{\card{\zeta}}}} \neq \sorte{\bounce{\zeta_{\card{\zeta}}}}
        \Rightarrow \hitter{\zeta_{\card{\zeta}}} \in lps$,
    \item $\forall b_j \in lps, b \in \cs \Rightarrow \exists ps \in \csState(b_j), ps \subset lps$,
    \item $\nexists lps'\in \derprocs(\obj{a_i}{a_j}), lps' \varsubsetneq lps$.
  \end{enumerate}
\end{definition}

\begin{theorem}[Sous-approximation séquentielle]
\thmlabel{approxinf-ordonnee}
  Pour toutes Frappes de Processus canoniques $(\PHs; \PHl; \PHa^{\langle 2 \rangle})$,
  tout contexte $\ctx \in \Ctx$ et toute séquence d'objectifs $\w = P \concat \w' \in \OS$,
  $\uconcr(P) \neq \emptyset \wedge \forall ps \in \derprocs(P),
    \muconcr_{\ctx' \Cap ps}(\w') \neq \emptyset
    \Rightarrow \uconcr(\w) \neq \emptyset$,
  où $\ctx' = \ctx \Cap \procs(\mycwB{\ctx}{P})$.
\end{theorem}

\begin{proof} %[\Thmref{approxinf-ordonnee}]
  Si $\uconcr(P) \neq \emptyset$, alors pour tout état $s \in \PHl, s \subset \ctx$,
  il existe un scénario $\delta \in \uconcr(P) \cap \Sce(s)$.
  D'après la \defref{lastprocs},
  et en s'inspirant de la démonstration du \thmref{approxinf},
  il existe $ps \in \derprocs(P)$ tel que $(s \play \delta) \subset \ctx' \Cap ps$.
  Ainsi, si $\muconcr_{\ctx' \Cap ps}(\w') \neq \emptyset$,
  alors il existe un scénario $\delta' \in \muconcr_{\ctx' \Cap ps}(\w')$ tel que
  $\delta' \in \Sce(s \play \delta)$.
  Par conséquent, $\delta \concat \delta'$ est un scénario jouable dans $s$.
  Donc, pour tout $s \in \PHl, s \subset \ctx$, il existe un scénario concrétisant $\w$.
  D'où : $\uconcr(\w) \neq \emptyset$.
\end{proof}

\todoplustard{Appliquer au modèle de segmentation métazoaire}



% Expressivité du PH et positionnement par rapport à d'autres formalismes
%% TODO : remplacer par le fichier définitif
% Expressivité du PH et positionnement par rapport à d'autres formalismes

\chapter{Expressivité du PH et positionnement par rapport à d'autres formalismes}
\chaplabel{expressivite}

\chapeau{
  Nous montrons dans ce chapitre que les différentes sémantiques de Frappes de Processus
  sont équivalentes à un certain nombre de formalismes répandus.
  Nous prouvons notamment que les Frappes de Processus canoniques permettent de
  représenter tout réseau discret asynchrone, un formalisme très répandu
  dans la représentation des réseaux de régulation biologiques.
  À l'inverse, nous proposons une méthode pour inférer les modèles de Thomas sous-jacents à
  des Frappes de Processus canoniques :
  il est possible d'inférer tous les types d'influence et les paramètres discrets,
  et d'énumérer tous les modèles compatibles avec ces résultats.
  Par ailleurs, les Frappes de Processus avec actions plurielles s'avèrent être
  équivalentes aux réseaux d'automates synchronisés.
  Enfin, nous proposons des traductions vers les réseaux de Petri et le formalisme Biocham.
}



\section{Réseaux discrets asynchrones}
\seclabel{trad-rda}

\section{Modèle de Thomas}
\seclabel{trad-thomas}

% Équivalence avec les réseaux d'automates synchronisés
% Équivalence avec les réseaux d'automates synchronisés

\section{Équivalence avec les réseaux d'automates synchronisés}
\seclabel{phm2an}

Nous nous intéressons ici au lien entre les Frappes de Processus avec actions plurielles
et les réseaux d'automates synchronisés.
Nous montrons notamment que ces deux formalismes sont équivalents
et nous exhibons pour cela deux traductions d'un formalisme vers l'autre.
%(\defref{phm2an,an2phm} \vdefpageref{phm2an}).
%(\defref[s]{phm2an} et \defref*[vref]{an2phm}).
Cette équivalence est intéressante car elle montre clairement le lien entre ce formalisme
de Frappes de Processus et celui plus répandu des réseaux d'automates synchronisés.
En effet, chaque action plurielle $\PHfrappemult{A}{B}$
possède la même dynamique qu'un ensemble de transitions synchronisées
partant chacune d'un processus de l'ensemble $A$ et
arrivant dans le processus de la même sorte de l'ensemble $A \recouvre B$%
\footnote{La notation $A \recouvre B$, formalisée à la \vdefref{recouvrementps}, représente
l'ensemble où chaque processus de $A$ a été remplacé par
le processus de $B$ de la même sorte, s'il existe.}.

\myskip

Nous rappelons tout d'abord la définition d'un réseau d'automates synchronisés (\defref{an})
ainsi que la relation de transition entre deux états d'un tel modèle (\defref{an-sem})
ce qui permet d'en définir la dynamique.

\begin{definition}[Réseau d'automates synchronisés]
\deflabel{an}
  Un \emph{réseau d'automates synchronisés} est un quadruplet $\AN = (\ANs; \ANl; \ANi; \ANt)$
  où :
  \begin{itemize}
    \item $\ANs \DEF \{a, b, \dots\}$ est l'ensemble fini et dénombrable des \emph{automates} ;
    \item $\ANl \DEF \bigtimes{a \in \ANs} \ANl_a$ est l'ensemble fini des \emph{états},
      où $\ANl_a = \{a_0, \ldots, a_{l_a}\}$ est l'ensemble fini et dénombrable
      des \emph{états locaux} de l'automate $a \in \ANs$ et $l_a \in \sN^*$,
      chaque état local appartenant à un unique automate :
      $\forall (a_i; b_j) \in \ANl_a \times \ANl_b, a \neq b \Rightarrow a_i \neq b_j$ ;
    \item $\ANi \DEF \{\ell_1, \dots, \ell_m\}$ est l'ensemble fini des
      \emph{libellés} de transitions ;
    \item $\ANt \DEF \{ \ANaction{a_i}{\ell}{a_j} \mid a \in \ANs \wedge a_i \in \ANl_a \wedge
      \ell \in \ANi \}$ est l'ensemble fini des \emph{transitions} entre états locaux.
  \end{itemize}
  Pour tout libellé $\ell \in \ANi$, on note
  $\precond{\ell} \DEF \{ a_i \mid \ANaction{a_i}{\ell}{a_j} \in \ANt \}$
  et $\postcond{\ell} \DEF \{ a_j \mid \ANaction{a_i}{\ell}{a_j} \in \ANt \}$.
%   et $\invcond{\ell} \DEF \{ a_i \mid \ANaction{a_i}{\ell}{a_i} \in \ANt \}$.
  L'ensemble des états locaux des automates est dénoté par
  $\ANProc \DEF \bigcup_{a \in \ANs} \ANl_a$.
  Enfin, étant donné un état global $s \in \ANl$, $s(a) = a_i \in \ANl_a$
  fait référence à l'état local de l'automate $a \in \ANs$.
\end{definition}

\begin{definition}[Sémantique des réseaux d'automates ($\ANtrans$)]
\deflabel{an-sem}
  Étant donné un réseau d'automates synchronisés $\AN = (\ANs; \ANl; \ANi; \ANt)$,
  un libellé $\ell$ est dit \emph{jouable} dans un état $s \in \ANl$ si et seulement si :
  $\forall a_i \in \precond{\ell}, s(a) = a_i$.
  Dans ce cas, on note $(s \play \ell)$ l'état résultant du jeu de toutes les transitions
  libellées par $\ell$, défini par :
  $s \play \ell = s \recouvre \postcond{\ell}$.
%   $\forall a_j \in \postcond{\ell}, (s \play \ell)(a) = a_j \wedge
%     \forall b \in \ANs, \ANl_b \cap \precond{\ell} = \emptyset \Rightarrow
%     (s \play \ell)(b) = s(b)$.
  De plus, on note alors : $s \ANtrans (s \play \ell)$.
%   Étant donné un réseau d'automates synchronisés $\AN = (\ANs; \ANl; \ANi; \ANt)$,
%   la relation de transition globale entre deux états du réseau
%   $\ANtrans \subset \ANl \times \ANl$ est définie par :
%   \begin{align*}
%     s \ANtrans s' \EQDEF \exists \ell \in \ANi, &\forall a_i \in \precond{\ell}, s(a) = a_i
%       \wedge \forall a_j \in \postcond{\ell}, s'(a) = a_j \\
%     \wedge &\forall b \in \ANs, \ANl_b \cap \precond{\ell} = \emptyset \Rightarrow s(b) = s'(b)
%   \end{align*}
\end{definition}

\begin{remark}
  Nous notons que les réseaux d'automates synchronisés ainsi définis sont non-déterministes,
  tant au niveau global du modèle qu'au niveau local des automates.
  Cette vision s'oppose à d'autres sémantiques des réseaux d'automates
  comme celles de \citeasnoun{Richard10} ou de \citeasnoun{RRT08},
  qui définissent la dynamique de leurs modèles à l'aide de fonctions de transition locales,
  qui sont par définition déterministes.
  Ces fonctions ont en effet la forme : $f_a : \ANl \rightarrow \ANl_a$
  et associent donc à chaque état global du modèle un état local (unique) pour chaque automate.
  La définition des réseaux d'automates synchronisés que nous proposons ici (\defref{an})
  n'empêche en revanche pas l'existence de deux libellés $\ell_1, \ell_2 \in \ANi$
  tels que $\precond{\ell_1} = \precond{\ell_2}$ mais $\postcond{\ell_1} \neq \postcond{\ell_2}$.
  Cela implique notamment l'existence de deux transitions entre état locaux
  $\ANaction{a_i}{\ell_1}{a_j}$ et $\ANaction{a_i}{\ell_2}{a_k}$
  avec $a_j \neq a_k$, d'où un non-déterminisme au niveau des automates.
\end{remark}

Pour tout modèle de Frappes de Processus avec actions plurielles $\PH$,
la \defref{phm2an} propose une traduction de $\PH$
en un réseau d'automates synchronisés $\phmtoan[\PH]$ équivalent,
et le \thmref{bisimulationphm2an} établit la bisimilarité entre les deux modèles.
La notation $\recouvre$ qui est utilisée dans la définition
qualifie le recouvrement d'un ensemble de processus de sortes distinctes
par un autre comprenant uniquement des processus issus des mêmes sortes
(\defref{recouvrementps}).
Cette notion de recouvrement est une extension
du recouvrement d'un état par un ensemble de processus
tel que précédemment formalisé à la \vdefref{recouvrement}.

\begin{definition}[Recouvrement ($\recouvre : \PHsublset \times \PHsublset \rightarrow \PHsublset$)]
\deflabel{recouvrementps}
  Étant donné un sous-état désordonné $ps \in \PHsublset$ et un processus $a_i \in \Proc$,
  tel que $a \in \sortes{ps}$, on définit :
  $(ps \recouvre a_i) = (ps \setminus \PHl_a) \cup \{ a_i \}$.
  On étend de plus cette définition
  au recouvrement par un ensemble de processus de sortes distinctes
  $ps' \in \PHsublset$ tel que $\sortes{ps'} \subset \sortes{ps}$
  comme étant le recouvrement de $ps$ par chaque processus de $ps'$ :
  $ps \recouvre ps' = ps \underset{a_i \in ps'}{\recouvre} a_i$.
\end{definition}

\begin{definition}[Réseau d'automates équivalent ($\phmtoansymbol$)]
\deflabel{phm2an}
  Le réseau d'automates synchronisés équivalent aux Frappes de Processus
  avec actions plurielles $\PH = (\PHs; \PHl; \PHh)$
  est défini par : $\phmtoan = (\PHs; \PHl; \ANi; \ANt)$, où :
  \begin{itemize}
    \item $\ANi = \{ \ell_h \mid h \in \PHh \}$ ; % est l'ensemble des libellés de transitions ;
    \item $\ANt = \{ \ANaction{a_i}{\ell_h}{a_j} \mid
      h \in \PHh \wedge h = \PHfrappemult{A}{B} \wedge a_i \in A \wedge a_j \in A \recouvre B \}$.
      % est l'ensemble des transitions locales.
  \end{itemize}
\end{definition}

\begin{theorem}[$\PH \approx \phmtoan$]
\thmlabel{bisimulationphm2an}
  Soient $\PH = (\PHs; \PHl; \PHh)$ des Frappes de Processus avec actions plurielles.
  On a :
  \[\forall s, s' \in \PHl, s \PHtrans s' \Longleftrightarrow s \trans{\phmtoan} s' \enspace.\]
\end{theorem}

\begin{proof}
  Soient $s, s' \in \PHl$.
  On pose : $\phmtoan = (\ANs; \ANl; \ANi; \ANt)$.
  
  ($\Rightarrow$) Supposons que $s \PHtrans s'$, c'est-à-dire qu'il existe une action $h \in \PHh$
    telle que $s' = s \play h$.
    Posons : $h = \PHfrappemult{A}{B}$.
    D'après la \defref{phm2an},
    l'existence de cette action dans $\PH$ implique celle d'un libellé $\ell_h$ dans $\phmtoan$
    ainsi que de l'ensemble de transitions
    $\ANt_h = \{ a_i \xrightarrow{\ell_h} a_j \mid a_i \in A \wedge a_j \in A \recouvre B \}$.
    Autrement dit, $\precond{\ell_h} = A$, donc $\ell_h$ est jouable dans $s$
    si et seulement si $A \subseteq s$.
    De plus, $\postcond{\ell_h} = \invariant{h} \cup B$, donc
    $(s \play \ell_h) = s \recouvre (\invariant{h} \cup B) = s \recouvre B = s'$
    car $\invariant{h} \subseteq A \subseteq s$.
  
  ($\Leftarrow$) Supposons que $s \trans{\phmtoan} s'$,
    c'est-à-dire qu'il existe un libellé $\ell \in \ANi$ et un ensemble de transitions
    ayant ce libellé : $\ANt_\ell = \{ a_i \xrightarrow{\ell} a_j \in \ANt \}$,
    tels que $s' = s \play \ell$.
    D'après la \defref{phm2an}, cela signifie notamment qu'il existe une action
    $h = \PHfrappemult{A}{B} \in \PHh$ telle que $\ell = \ell_h$, et que :
    $\ANt_\ell = \{ a_i \xrightarrow{\ell} a_j \mid a_i \in A \wedge a_j \in A \recouvre B \}$.
    Étant donné que $\invariant{h}$ et $\cible{h}$ forment une partition de $A$,
    $\ANt_\ell$ peut être découpé en deux ensembles, selon les invariants et les cibles de $h$ :
    $\ANt_\ell = \{ a_i \xrightarrow{\ell} a_i \mid a_i \in \invariant{h} \} \cup
      \{ a_i \xrightarrow{\ell} a_j \mid a_i \in \cible{h} \wedge a_j \in B \}$.
    Ainsi, $s' = s \recouvre (\invariant{h} \cup B) = s \recouvre B = s \play h$.
\end{proof}

Pour finir, nous proposons à la \defref{an2phm} la traduction inverse
d'un réseau d'automates synchronisés $\AN$
en des Frappes de Processus avec actions plurielles équivalentes $\antophm$.
Le \thmref{bisimulationan2phm} stipule que le modèle obtenu est bien bisimilaire
au modèle d'origine.
Enfin, le \thmref{equivphman} résume les résultats de cette section
en statuant l'équivalence d'expressivité entre les Frappes de Processus avec
actions plurielles et les réseaux d'automates synchronisés.

\begin{definition}[Frappes de Processus équivalentes ($\antophmsymbol$)]
\deflabel{an2phm}
  Les Frappes de Processus avec actions plurielles
  équivalentes au réseau d'automates synchronisés $\AN = (\PHs, \PHl, \ANi, \ANt)$
  sont définies par $\antophm = (\ANs, \ANl, \PHh)$, où :
%   $\PHh = \{ \PHfrappemult{\precond{\ell}}{(\postcond{\ell} \setminus \invcond{\ell})}
%     \mid \ell \in \ANi \}$.
  \[\PHh = \{ \PHfrappemult{\precond{\ell}}{B} \mid \ell \in \ANi \wedge
    B = \postcond{\ell} \setminus \{ a_i \in \ANProc \mid \ANaction{a_i}{\ell}{a_i} \in \ANt \}
    \}\]
\end{definition}

\begin{theorem}[$\AN \approx \antophm$]
\thmlabel{bisimulationan2phm}
  Soit $\AN = (\ANs; \ANl; \ANi; \ANt)$ un réseau d'automates synchronisés.
  On a :
  \[\forall s, s' \in \ANl, s \ANtrans s' \Longleftrightarrow s \trans{\antophm} s' \enspace.\]
\end{theorem}

\begin{proof}
  Soient $s, s' \in \PHl$.
  On pose : $\antophm = (\ANs; \ANl; \PHh)$.
  
  ($\Rightarrow$) Supposons que $s \ANtrans s'$,
    c'est-à-dire qu'il existe un libellé $\ell \in \ANi$ et un ensemble de transitions
    ayant ce libellé : $\ANt_\ell = \{ a_i \xrightarrow{\ell} a_j \in \ANt \}$,
    tels que $s' = s \play \ell$.
    D'après la traduction donnée à la \defref{an2phm}, il existe donc une action
    $h = \PHfrappemult{A}{B} \in \PHh$ telle que $A = \precond{\ell}$ et
    $B = \postcond{\ell} \setminus \{ a_i \in \ANProc \mid \ANaction{a_i}{\ell}{a_i} \in \ANt \}$.
    Or $s' = s \recouvre \postcond{\ell}
      = s \recouvre (B \cup \{ a_i \in \ANProc \mid \ANaction{a_i}{\ell}{a_i} \in \ANt \})
      = s \recouvre B$
    car $\{ a_i \in \ANProc \mid \ANaction{a_i}{\ell}{a_i} \in \ANt \} \subseteq s$.
    Ainsi, $h$ est jouable dans $s$ et $s' = s \play h$.
  
  ($\Leftarrow$) Supposons que $s \trans{\antophm} s'$,
    c'est-à-dire qu'il existe une action $h = \PHfrappemult{A}{B} \in \PHh$
    telle que $s' = s \play h$.
    D'après la traduction de la \defref{an2phm},
    cela signifie qu'il existe un libellé $\ell \in \ANi$ et un ensemble de transitions
    ayant ce libellé : $\ANt_\ell = \{ a_i \xrightarrow{\ell} a_j \in \ANt \}$,
    tels que : $A = \precond{\ell}$ et
    $B = \postcond{\ell} \setminus \{ a_i \in \ANProc \mid \ANaction{a_i}{\ell}{a_i} \in \ANt \}$.
    Comme $h$ est jouable dans $s$, alors $A \subseteq s$, donc $\ell$ est aussi jouable dans $s$.
    De plus, $s' = s \play h = s \recouvre B = s \recouvre (B \cup \invariant{h})
      = s \recouvre \postcond{\ell}$.
\end{proof}

\begin{theorem}[Équivalence entre réseaux d'automates synchronisés
  et Frappes de Processus avec actions plurielles]
\thmlabel{equivphman}
  Les Frappes de Processus avec actions plurielles sont aussi expressives
  que les réseaux d'automates synchronisés.
\end{theorem}

\begin{proof}
  D'après les \defref{phm2an,an2phm} et les \thmref{bisimulationphm2an,bisimulationan2phm}
  associés, tout modèle de Frappes de Processus avec actions plurielles peut être représenté
  à l'aide d'un réseau d'automates synchronisés, et inversement.
\end{proof}

%\todo{Exemple}

% \section{Équivalence avec les réseaux d'automates synchronisés}
% \seclabel{phm2an}

\section{Réseaux de Petri}
  \subsection{Réseaux de Petri saufs (une place par processus)}
  \subsection{Réseaux de Petri bornés (une place par sorte)}

\section{Biocham}

% Expressivité du PH et positionnement par rapport à d'autres formalismes

\chapter{Expressivité des Frappes de Processus
  et positionnement par rapport à d'autres formalismes}
\chaplabel{expressivite}

\chapeaupublis{
  Nous montrons dans ce chapitre que les différentes sémantiques de Frappes de Processus
  sont équivalentes à un certain nombre de formalismes répandus.
  Nous prouvons notamment que les Frappes de Processus canoniques permettent de
  représenter tout réseau discret asynchrone, un formalisme très répandu
  dans la représentation des réseaux de régulation biologique.
  À l'inverse, nous proposons une méthode pour inférer les modèles de Thomas sous-jacents à
  des Frappes de Processus canoniques :
  il est possible d'inférer tous les types d'influence et les paramètres discrets,
  et d'énumérer tous les modèles compatibles avec ces résultats.
  Par ailleurs, les Frappes de Processus avec actions plurielles s'avèrent être
  équivalentes aux réseaux d'automates synchronisés.
  Enfin, nous proposons des traductions vers les réseaux de Petri et le formalisme Biocham.
}{%
  La traduction vers le modèle de Thomas a fait l'objet de deux publications :
  \cite*{FPIMR12-LDSSB} et \cite*{FPIMR12-CMSB}
  et d'une soumission en journal en cours de \textit{review}.
  La traduction depuis les réseaux discrets asynchrones a,
  quant à elle, été publiée dans \cite*{FPMR13-CS2Bio}.
}



Le sujet du chapitre courant est l'expressivité des Frappes de Processus,
dans les différentes sémantiques présentées au \chapref{sem},
par rapport à d'autres modélisations couramment utilisées.
Ce travail de positionnement comporte un intérêt évident lorsqu'il s'agit d'interagir
avec d'autres types de modèles.
Ainsi, les Frappes de Processus peuvent bénéficier des capacités d'analyse d'autres
modélisations, moyennant une traduction, qui peuvent s'avérer efficaces pour
certains problèmes.
À l'inverse, traduire un formalisme en Frappes de Processus permet par exemple
de pouvoir y appliquer les méthodes de représentation et d'analyse proposées,
et notamment l'analyse statique détaillée au \chapref{as}.

Nous présentons ainsi à la \secref{trad-rda} une traduction des réseaux discrets asynchrones
en Frappes de Processus canoniques.
Cette traduction est fondamentale pour l'étude des réseaux de régulation biologique
car elle permet la traduction d'un modèle très utilisé directement en un formalisme
de Frappes de Processus qu'il est possible d'étudier à l'aide de l'analyse statique.
La \secref{trad-thomas} propose, à l'inverse, une méthode pour inférer l'ensemble des modèles
de Thomas sous-jacents à un modèle de Frappes de Processus canoniques.
L'inférence du graphe des interactions, par exemple,
conserve certaines propriétés comme la présence de cycles
exploitables par les résultats relatifs aux conjectures de Thomas.
Par ailleurs, l'inférence des paramètres discrets permet de restreindre l'espace des possibles
en matière de paramétrisations compatibles avec la dynamique du modèle étudié,
facilitant ainsi la recherche de la paramétrisation représentant le comportement recherche.
Les modèles de Thomas compatibles avec ces deux inférences sont ceux dont la paramétrisation
est complétée lorsque certains paramètres n'ont pas pu être inférés ;
nous proposons une méthode pour les énumérer en accord toujours avec la dynamique
de modèle de Frappes de Processus canoniques étudié.

Nous nous intéressons par ailleurs aux liens avec d'autres modélisations
qui ne sont pas totalement asynchrones,
à l'aide des Frappes de Processus avec actions plurielles.
Nous montrons notamment à la \secref{phm2an} que ce type de Frappes de Processus
est équivalent aux plus classiques automates synchronisés.
Nous donnons pour cela les deux traductions adéquates et exhibons les preuves correspondantes.
Par ailleurs, nous proposons aussi à la \secref{trad-petri}
une traduction des Frappes de Processus avec actions
plurielles vers les réseaux de Petri utilisant des arcs de lecture et des arcs inhibiteurs.
Cette traduction permet de montrer le lien d'inclusion entre les deux formalismes,
et ouvre des perspectives dans l'utilisation des réseaux de Petri pour l'analyse des réseaux
de régulation biologique représentés sous la forme de Frappes de Processus.
Pour finir, nous proposons à la \secref{trad-biocham} une traduction des systèmes d'équations tels
que définis dans le formalisme de Biocham vers les Frappes de Processus avec actions plurielles.
Nous nous intéressons naturellement à la sémantique booléenne de Biocham,
qui est par ailleurs asynchrone et qui ajoute une couche supplémentaire d'indéterminisme
sur chaque réaction (en autorisant la consommation d'une partie arbitraire des réactifs).
Cette traduction tire naturellement parti des similitudes entre Biocham et les Frappes
de Processus avec actions plurielles, et permet d'affirmer
qu'un tel système d'équations biochimiques peut toujours être exprimé
en Frappes de Processus avec actions plurielles,
et donc être analysé avec les outils adéquats.

Les résultats de ce chapitre sont représentés par les flèches fines reliant
les différentes formes de Frappes de Processus
aux autres formalismes discrets représentés à la \vfigref{contrib-equivalences}.



% Traduction depuis les réseaux discrets asynchrones}
% Traduction depuis les réseaux discrets asynchrones}

\section{Traduction depuis les réseaux discrets asynchrones}
\seclabel{trad-rda}

Nous proposons dans cette section une traduction des réseaux discrets asynchrones
en Frappes de Processus canoniques, et nous en montrons la validité
par une preuve de bisimulation faible.
Les réseaux discrets asynchrones ont été préalablement définis à la \vdefref{rda-def}
du \chapref{etatdelart}.
Il s'agit de modèles proches du modèle de Thomas, mais en comportant pas de restriction unitaire
de la dynamique, et présentant des fonctions d'évolution en lieu et place de paramètres discrets.
Un réseau discret asynchrone se présente sous la forme d'un couple $\RDA = (\GI; F)$
où $\GI = (\components; E)$ est un graphe des interactions
et $F$ est un ensemble de fonctions $f_x : \RRBreg{x} \rightarrow \segm{0}{l_x}$
pour tout composant $x \in \components$,
où $\RRBreg{x}$ est l'ensemble des prédécesseurs de $x$ dans le graphe des interactions.
La dynamique d'un tel réseau est la suivante : il existe une transition $\RRBtransrda{s}{s'}$
si et seulement si un unique composant $x$ évolue entre $s$ et $s'$
de façon à ce que : $\RRBget{s'}{x} = f_x(s)$.
Les Frappes de Processus canoniques, définies à la \vsecref{phcanonique},
permettent une représentation presque immédiate des réseaux discrets asynchrones,
à condition de créer les sortes coopératives adéquates.

\myskip

% Nous complétons dans un premier temps la définition d'un réseau discret asynchrone par
% la notion de dépendances d'un composant.
% En effet, même si les fonctions d'évolution de chaque composant sont sur $\RRBstates$,
% elles ne dépendent généralement 
% pour tout composant $x \in \components$, $\RDAdep(x)$ est l'ensemble des composants dont
% la valeur de $f_x$ dépend véritablement,
% 
% \begin{definition}[Dépendance]
%   Soit $\RDA = (\GI; F)$ un réseau discret asynchrone, avec $\GI = (\components; E)$.
%   Pour tout composant $x \in \components$,
%   on note $\DNdep(x) \subseteq \components$ l'ensemble des composants dont
%   la valeur de $f_x$ dépend véritablement,
%   c'est-à-dire l'ensemble minimal tel que :
%   \[\forall s, s' \in \RRBstates,
%     \big(\forall y \in \DNdep(f_x), \get{s}{y} = \get{s'}{y}\big) \Longrightarrow
%     f_x(s) = f_x(s')\]
% \end{definition}

La traduction proposée à la \defref{rda2ph}
associe deux sortes à chaque composant $a$ dans $\RDA$ :
\begin{itemize}
  \item une sorte du même nom pour représenter ce composant,
  \item une sorte coopérative $f^a$ représentant sa fonction d'évolution $f_a$,
    et dont les états sont donc une combinaison des états de ses régulateurs.
\end{itemize}
De plus, les actions primaires sont définies de façon à correctement mettre à jour
les sortes coopératives,
et les actions secondaires le sont de façon à ce que chaque sorte coopérative $f^a$
interagisse avec son composant $a$ de la façon dont la fonction d'évolution
correspondante le permet.
Nous montrons de plus au \thmref{bisimulationrda2ph} que le modèle obtenu
est faiblement bisimilaire au réseau discret asynchrone d'origine.

\begin{definition}[Frappes de Processus équivalentes ($\rdatophsymbol$)]
\deflabel{rda2ph}
  Soit $\RDA = (\GI; F)$ un réseau discret asynchrone, avec $\GI = (\components; E)$.
  On note $\rdatoph = (\PHs; \PHl; (\PHh^{(1)}; \PHh^{(2)}))$
  les Frappes de Processus canoniques équivalentes
  à $\RDA$, définies par :
  \begin{itemize}
    \item $\PHs = \components \cup \{ f^a \mid a \in \components \}$,
    \item $\PHl = \bigtimes{a \in \components} \PHl_{a} \times
      \bigtimes{a \in \components} \PHl_{f^a}$ l'ensemble des états , où :
      \begin{align*}
      \forall a \in \components&, \PHl_{a} = \{ a_i \mid i \in \segm{0}{l_a} \} \\
      \forall a \in \components&, \PHl_{f^a} = \begin{cases}
          \PHsubl_{\RRBreg{a}} & \text{ si } \RRBreg{a} \neq \emptyset \\
          \{ f^a_\emptyset \}  & \text{ sinon }
        \end{cases} \enspace,
      \end{align*}
    \item $\PHh^{(1)} = \{ \PHfrappe{b_k}{f^a_\mysigma}{f^a_{\mysigma'}} \mid
      a \in \components \wedge b \in \RRBreg{a} \wedge
      b_k \in \PHl_{b} \wedge f^a_\mysigma \in \PHl_{f^a} \wedge
      \get{\mysigma}{b} \neq b_k \wedge \mysigma' = \mysigma \recouvre b_k \}$,
    \item $\PHh^{(2)} = \{ \PHfrappe{f^a_\mysigma}{a_j}{a_{k}} \mid
      a \in \components \wedge f^a_\mysigma \in \PHl_{f^a} \wedge
      a_j, a_k \in \PHl_{a} \wedge j \neq k \wedge f_a(\decode \mysigma) = k \}$.
  \end{itemize}
  
  Pour tout état $s \in \RRBstates$ de $\RDA$,
  $\encode{s} = \os$ est l'état correspondant dans $\rdatoph$, défini par :
  $\forall a \in \components, \get{s}{a} = k \Rightarrow \get{\os}{a} = a_k$
  et
  $\forall a \in \components, \get{\os}{f^a}=f^a_\mysigma$
  avec $f^a_\mysigma \in \PHl_{f^a}$
  et $\forall b \in \RRBreg{a}, \get{\mysigma}{b} = \get{\os}{b}$.

  À l'inverse, pour tout état $\os \in \PHsubl$ de $\rdatoph$,
  $\decode{\os} = s$ est l'état correspondant dans $\RDA$ avec :
  $\forall a \in \sortes{\os}, \get{\os}{a} = a_k \Rightarrow \get{s}{a} = k$.
\end{definition}

\begin{theorem}[$\RDA \approx \rdatoph$]
\thmlabel{bisimulationrda2ph}
  Soit $\RDA = (\GI; F)$ un réseau discret asynchrone.
  On a :
  \begin{enumerate}
    \item \label{rda2ph} $\forall s, s' \in \RRBstates$,
      $s \RDAtrans s' \Longrightarrow \encode{s} \mtrans{\PH} \encode{s'}$,
      où $\mtrans{\PH}$ est une séquence finie de transitions $\trans{\PH}$.
    \item \label{ph2rda} $\forall \os, \os' \in \PHl$,
      $\os \trans{\PH} \os' \Longrightarrow
        \decode{\os} = \decode{\os'} \vee \decode{\os} \RDAtrans \decode{\os'}$
  \end{enumerate}
\end{theorem}

\begin{proof}
  On pose : $\rdatoph = (\PHs; \PHl; (\PHh^{(1)}; \PHh^{(2)}))$.
  
  (\ref{rda2ph}) Soient $s, s' \in \RRBstates$ tels que $s \RDAtrans s'$.
    Cela signifie qu'il existe un composant $a \in \components$ tel que :
    $\RRBget{s'}{a} = f_a(\mysigma)$ et
    $\forall b \in \components, b \neq a \Rightarrow \RRBget{s}{b} = \RRBget{s'}{b}$,
    où $\mysigma \in \PHsubl_{\RRBreg{a}}$ tel que $\mysigma \subseteq \encode{s}$.
    Posons : $j = \RRBget{s}{a}$ et $k = \RRBget{s'}{a}$ ;
    d'après la \defref{rda2ph}, il existe une action
    $\PHfrappe{f^a_\mysigma}{a_j}{a_k} \in \PHh^{(2)}$
    avec $f_a(\mysigma) = k$
    Par définition de $\encode{s}$, on a $a_k \in \encode{s}$
    et $f^a_\mysigma \in \encode{s}$,
    et aucune action de $\PHh^{(1)}$ n'est jouable dans $\encode{s}$ ;
    ainsi, $h$ est jouable dans $\encode{s}$, d'où : $\encode{s} \PHtrans \encode{s} \play h$.
    De plus, d'après le \vlemref{update},
    $\encode{s} \play h \mtrans{\PH} \update(\encode{s} \play h)$.
    Enfin, comme $\update$ met à jour des sortes coopératives, on a :
    $\update(\encode{s} \play h) = \encode{s'}$.
  
  (\ref{ph2rda}) Soient $\os, \os' \in \PHl$ tels que $\os \trans{\PH} \os'$.
    Cela signifie qu'il existe une action $h \in \PHh$ telle que : $s' = s \play h$.
    Si $h \in \PHh^{(1)}$, alors $\decode{\os} = \decode{\os'}$, par définition
    de $\decode{\os}$.
    En revanche, si $h \in \PHh^{(2)}$, alors d'après la \defref{rda2ph},
    il existe $a \in \components$, $\mysigma \in \PHsubl_{\RRBreg{a}}$
    et $a_j, a_k \in \PHl_{a}$, tels que :
    $h = \PHfrappe{f^a_\mysigma}{a_j}{a_{k}}$, avec $f_a(\decode \mysigma) = k$ et $j \neq k$.
    Ainsi, $\forall b \in \RRBreg{a}$,
    $\PHget{\mysigma}{b} = b_i \Rightarrow \RRBget{\decode{\os}}{b} = i$.
    D'où : $\decode{\os} \RDAtrans \decode{\os'}$
    car $f_a(\decode \mysigma) = k$.
\end{proof}


% Inférence du modèle de Thomas
% Inférence du modèle de Thomas

\section{Inférence du modèle de Thomas}
\seclabel{trad-thomas}

\TODO

\todo{Nous notons pour finir que la complexité de cette méthode est exponentielle dans le nombre
de régulateurs de chaque composant, et linéaire dans le nombre total de composants.
\footnote{Ce travail a été réalisé dans le cadre d'une collaboration avec Katsumi Inoue.
Cette collaboration a débuté par un stage doctoral de trois mois dans l'Inoue Laboratory,
au National Institute of Informatics (Tokyo, Japon).}}

\todo{Citer FPIMR-10}

Dans toute cette section, nous considérons un modèle de Frappes de Processus canoniques
$\PH = (\PHs; \PHl; (\PHh^{(1)}; \PHh^{(2)}))$.

\subsection{Inférence du graphe des interactions}
\seclabel{trad-thomas-gi}

Un graphe des interactions (\vdefref{thomas-gi}) est une représentation abstraite des
influences directes, positives ou négatives, entre les composants d'un système.
Comme discuté à la \vsecref{thomas-analyse}, le graphe des interactions permet
de caractériser efficacement les propriétés dynamiques globales du système,
à l'aide notamment de résultats comme les conjectures de Thomas,
qui apportent des résultats sur la présence d'oscillations ou d'états stables multiples.

Dans le cas d'un processus de modélisation d'un réseau de régulation biologique, le modèle
de Thomas est le point de départ e la spécification du modèle.
Cependant, il est courant que le graphe des interactions initialement conçu contienne des
influences qui n'ont pas d'impact sur la dynamique.
La méthode que nous proposons dans la suite s'appuie directement sur la dynamique d'un modèle
de Frappes de Processus canoniques, ce qui produit des graphes des interactions minimaux,
et permet d'affiner les conclusions de telles méthodes d'analyse statique.

L'intuition de cette inférence est que seuls les composants (les sortes dans $\components$)
figureront dans le graphe des interactions ;
les sortes coopératives (dans $\cs$) sont uniquement étudiées pour comprendre les actions
«~indirectes~» entre composants.



\subsubsection{Frappes de Processus bien-formées}

Nous notons que dans cette section, les indices des processus de composants
possèdent une importance particulière,
notamment pour contraindre le fait que la dynamique doit être unitaire (\crref{unitaire}).
Autrement dit, si on suppose que ces indices représentent des niveaux d'expression discrets
ordonnés,
par exemple si $b_0$, $b_1$ et $b_2$ représentent le fait que le composant $b$
est présent respectivement en faible, moyenne et forte concentration,
alors une action de la forme $\PHfrappe{a_1}{b_0}{b_2}$ n'est pas autorisée ;
en revanche, deux actions $\PHfrappe{a_1}{b_0}{b_1}$ et $\PHfrappe{a_1}{b_1}{b_2}$ le sont.
Naturellement, toute autre relations d'ordre entre les indices est admissible,
à condition qu'une contrainte d'unicité similaire puisse être définie.

\begin{equation}
\components = \{a \in \PHs \mid \nexists \PHfrappe{b_i}{a_j}{a_k} \in \PHa, |j - k| > 1\} \\
\eqlabel{PH-components}
\end{equation}

\begin{critere}[Dynamique unitaire]
\crlabel{unitaire}
  Toutes les actions secondaires de $\PH$ ne font pas de bond
  à plus d'un processus d'écart :
  $\forall \PHfrappe{a_i}{b_j}{b_k} \in \PHh^{(2)}, \card{j - k} = 1$
\end{critere}

\begin{example}
  Les Frappes de Processus canoniques représentées à la \figref{infer-ex-ph}
  possèdent une dynamique unitaire, et sont donc compatibles
  avec l'inférence du graphe des interactions proposée à la section suivante.
\end{example}

\begin{remark}
  Le \crref{unitaire} est naturellement vérifié pour tout modèle booléen,
  c'est-à-dire tel que : $\forall a \in \components, \card{\PHl_a} = 2$.
\end{remark}

\begin{remark}
  Il est possible de ne pas prendre en compte le \crref{unitaire},
  à condition de s'affranchir de l'aspect unitaire de la dynamique du modèle de Thomas.
  Les résultats de cette section restent alors théoriquement applicables.
\end{remark}

Nous considérons dans la suite que les Frappes de Processus canoniques $\PH$ respectent
le \crref{unitaire}.



\subsection{Inférence des interactions}
\seclabel{infer-gi}


L'inférence de cette section est directement inspirée des travaux de \citeasnoun{Richard10},
qui déduit les influences d'un graphe des interactions en fonction des évolutions des différents
composants, selon l'état de ses régulateurs.

\newcommand{\myupsilon}{\upsilon}

Pour tout composant $a$,
les \emph{prédécesseurs} de $a$, notés $\pred(a)$,
sont toutes les sortes ayant au moins une action frappant $a$.
Les \emph{régulateurs} de $a$, en revanche, notés $\reg(a)$, sont tous les composants
qui influent sur $a$, soit directement, soit à travers une sorte coopérative.
Il est à noter que les régulateurs définis de cette manière seront potentiellement
des régulateurs de $a$ dans le modèle de Thomas inféré,
tels que définis \vpageref{regulateurs},
ce qui explique pourquoi ces deux définitions sont proches.

\begin{align*}
  \forall a \in \components, \pred(a) &\DEF \{ b \in \PHs \mid
    \exists h \in \PHh, \sorte{\frappeur{h}} = b \wedge
    \sorte{\cible{h}} = a \} \\
  \forall a \in \components, \reg(a) &\DEF \{ \compin(b) \mid
    b \in \pred(a) \}
\end{align*}
Où $\compin(b)$ fait référence aux composants qui régulent la sorte coopérative $b$,
autrement dit aux sortes que $b$ représente
(cf.~\vdefref{comp}).

L'étude des influences d'un composant $b$ régulant un autre composant $a$ nécessite d'étudier
le groupe de régulateurs de $b$ qui vont influencer conjointement $a$.
Ces groupes de régulateurs sont aisément déterminés en observant les sortes coopératives.
Nous proposons ici de définir les groupes de régulateurs comme étant les composants connexes
d'un graphe reliant tous les régulateurs de $a$ qui sont représentés par une même sorte
coopérative :
\[
  \forall a \in \components,
  X(a) \DEF \mathcal{C} \big( (\reg(a),
  \{ \{b,c\} \subset \compin(\myupsilon) \mid
  \myupsilon \in \pred(a) \cap \cs \}) \big)
\]
Où $\mathcal{C}(G)$ représente l'ensemble des composantes connexes du graphe non orienté $G$.

Pour étudier l'influence d'un groupe de régulateurs $g$ sur un composant $a$,
nous effectuons une analyse exhaustive de toutes les configurations possibles de $g$.
Pour cela, il est nécessaire de définir un sous-état $\sigma$ sur les sortes de $g$,
et et de compléter ce sous-état par les processus de sorte coopérative
qui représentent l'état des composants dans $g$.
Nous définissons pour cela l'ensemble $\allFocals{a}{g}{\sigma}$
qui contient l'état des sortes de $g$, celui de $a$, et celui de toutes les sortes
coopératives frappant $a$.

%allFocals
\begin{align*}
  &\forall a \in \components, \forall g \in X(a), \forall \sigma \in \PHsubl_{g \cup \{ a \}}, \\
  & \quad
  \allFocals{a}{g}{\sigma} = \{ \PHget{\sigma}{b} \mid b \in \pred(a) \cap \components \}
  \cup \{ \pfp_\sigma(b) \mid b \in \pred(a) \cap \cs \}
\end{align*}
Avec :
\[
  \pfp_\sigma(b) = \pfp_{s \recouvre \sigma}(b)
\]
où le choix de $s \in \PHl$ est indifférent d'après le point (\ref{csss}) de la \vdefref{cs}.

Enfin, il est possible d'étudier localement la dynamique de $a$ en fonction du sous-état
$\sigma$ d'un groupe de régulateurs $g$ donné ;
cette dynamique locale se concentre donc uniquement sur les actions frappant $a$.
En effet, en faisant varier l'un des composants $b \in g$ et en observant le résultat
sur l'évolution de $a$ (tendance à l'augmentation ou à la diminution de son niveau d'expression),
il est possible d'en déduire l'influence locale de $b$ sur $a$ pour un niveau d'expression
de $b$ donné.
Pour cela, nous appelons $\irB_a(\sigma)$ l'ensemble des processus vers lesquels $a$ peut évoluer
depuis le sous-état $\sigma$ ;
naturellement, si aucune action ne frappe $a$ dans $\sigma$,
alors $\irB_a(\sigma) = \PHget{\sigma}{a}$.

\begin{align*}
  &\forall g \in X(a), \forall \sigma \in \PHsubl_{g \cup \{ a \}},
  \irB_a(\sigma) \DEF 
  \begin{cases}
    \irF_a(\sigma)
      & \text{ si } \irF_a(\sigma) \neq \emptyset\\
    \{ \PHget{\sigma}{a} \}
      & \text{ si } \irF_a(\sigma) = \emptyset
  \end{cases}\\
  &\text{où : } \irF_a(\sigma) \DEF \{ a_k \in \PHl_a \mid
    \exists b \in \PHs, \exists \PHfrappe{b_i}{a_j}{a_k} \in \PHa,
  \{ b_i, a_j \} \subset \allFocals{a}{g}{\sigma} \}\\
\end{align*}


La \propref{inference-edges} détaille l'inférence de toutes les influences locales existant
entre les composants, c'est-à-dire celles qui se produisent pour un seuil donné $t$.
L'idée principale derrière cette inférence est la suivante :
s'il existe une une influence
positive (\resp négative) d'un composant $b$ sur un autre composant $a$,
alors augmenter le niveau d'expression de $b$
va potentiellement faire faire augmenter (\resp diminuer) le niveau d'expression de $a$,
au moins dans certaines configurations (\eqref{edges-inference}).
Ainsi, ces influences locales se séparent en influences positives et négatives,
ce qui représente de potentiels arcs dans le graphe des interactions final.
De plus, l'étude des influences sur les groupes de régulateurs d'un composant $a$
permet aussi d'étudier les auto-influences de $a$
(\eqref{edges-inference-auto})
ce qui permettra potentiellement d'inférer des auto-arcs.
Finalement, il est nécessaire d'étudier le cas particulier où $a$ ne possède pas de régulateurs
(\eqref{edges-inference-noreg}).
Nous notons que cette méthode ignore naturellement tous les cas où il n'est pas possible
de distinguer une influence d'un composant sur un autre.

\begin{proposition}[Inférence des influences]
\proplabel{inference-edges}
  Nous définissons l'ensemble $\hat{E}_+$ (\resp $\hat{E}_-$)
  des \emph{influences locales positives} (\resp \emph{négatives})
  pour tout composant $a \in \components$ par :
  % Arcs a -> b, a ≠ b
  \begin{align}
    \begin{split}\eqlabel{edges-inference}
      \forall b \in \reg(a), \forall s \in \{ +, - \}, \\
      b \xrightarrow{t+1} a \in \hat{E}_s \Longleftrightarrow\ & \exists g \in X(a), b \in g,
      \exists \sigma \in \PHsubl_{g \cup \{ a \}}, \\
        &\qquad \{ b_t, b_{t+1} \} \subset \PHl_b \wedge b_t \in \sigma,\\
        &\qquad \exists a_j \in \irB_a(\sigma), \exists a_k \in \irB_a(\sigma\{b_{t+1}\}), \\
        &\qquad s = \f{signe}(k - j)
    \end{split}
    \end{align}
    % Auto-arcs depuis les groupes de régulateurs
    \begin{align}
    \begin{split}\eqlabel{edges-inference-auto}
      \forall s \in \{ +, - \}, \quad\qquad\qquad \\
      a \xrightarrow{t+1} a \in \hat{E}_s \Longleftrightarrow\ & \exists g \in X(a),
      \exists \sigma \in \PHsubl_{g \cup \{ a \}}, \\
        &\qquad \{ a_t, a_{t+1} \} \subset \PHl_a \wedge a_t \in \sigma,\\
        &\qquad \exists a_j \in \irB_a(\sigma), \exists a_k \in \irB_a(\sigma\{a_{t+1}\}), \\
        &\qquad s = \f{signe}(k - j)
    \end{split}
    \end{align}
    % Auto-arcs des composants sans prédécesseurs
    \begin{align}
    \begin{split}\eqlabel{edges-inference-noreg}
      \forall s \in \{ +, - \}, \quad\qquad\qquad \\
      a \xrightarrow{t+1} a \in \hat{E}_s \Longleftrightarrow\ & \reg(a) = \emptyset \wedge
        \{ a_t, a_{t+1} \} \subset \PHl_a, \\
        &\qquad \exists a_j \in \irB_a(\etat{a_t}), \exists a_k \in \irB_a(\etat{a_{t+1}}), \\
        &\qquad s = \f{signe}(k - j)
    \end{split}
  \end{align}
  où : $\f{signe}(x) = \begin{cases}
    + & \text{ si $x > 0$} \\
    - & \text{ si $x < 0$} \\
    0 & \text{ if $x = 0$}
  \end{cases}$
\end{proposition}

Nous sommes alors en mesure d'inférer les arcs du graphe des interactions final,
à partir de ces ensembles d'influences locales positives et négatives.
En effet, nous pouvons inférer une influence (globale) positive ou négative
d'un composant vers un autre
s'il n'existe que des influences locales correspondantes du même signe.
Une influence non-signée est inférée si, à l'inverse, il existe au moins deux influences
locales correspondantes de signes différents.
Enfin, le seuil de chaque influence (quel que soit son signe)
est égal au seuil minimum pour lequel une influence locale a été trouvée.
Nous formalisons cette inférence dans la \propref{inference-gi}.

\begin{proposition}[Inférence du graphe des interactions]
\proplabel{inference-gi}
  Nous inférons $\GI = (\components; E)$ à l'aide de la \propref{inference-edges} comme suit :
  \begin{align*}
    E_+ &= \{ \arc{a}{+}{t}{b} \mid \nexists a \xrightarrow{t'} b \in \hat{E}_-
      \wedge t = \min \{ r \mid a \xrightarrow{r} b \in \hat{E}_+ \}\} \\
    E_- &= \{ \arc{a}{-}{t}{b} \mid \nexists a \xrightarrow{t'} b \in \hat{E}_+
      \wedge t = \min \{ r \mid a \xrightarrow{r} b \in \hat{E}_- \}\} \\
    E_\uns &= \{ \arc{a}{\uns}{t}{b} \mid \exists a \xrightarrow{t'} b \in \hat{E}_+ \wedge
      \exists a \xrightarrow{t''} b \in \hat{E}_- \\
      & \qquad\qquad\qquad \wedge t = \min \{ r \mid
      a \xrightarrow{r} b \in \hat{E}_- \cup \hat{E}_+ \} \}
  \end{align*}
\end{proposition}




\begin{figure}[ht]
\centering
\scalebox{1.3}{
\begin{tikzpicture}
  \path[use as bounding box] (-4,-1.9) rectangle (4.5,3.9);

  \TSort{(0,0)}{a}{3}{l}
  \TSort{(3, 3)}{b}{2}{t}
  \TSort{(3,-1)}{c}{2}{b}

  \TSetTick{bc}{0}{00}
  \TSetTick{bc}{1}{01}
  \TSetTick{bc}{2}{10}
  \TSetTick{bc}{3}{11}
  % \TSetSortLbcel{bc}{$\neg a\wedge b$}
  \TSort{(-3,-0.5)}{bc}{4}{l}

  \THit{bc_3}{}{a_1}{.north west}{a_2}
  \THit{bc_0}{}{a_1}{.south west}{a_0}
  \path[bounce]
  \TBounce{a_1}{bend left}{a_2}{.south west}
  \TBounce{a_1}{bend right}{a_0}{.north west}
  ;

  \THit{b_0}{}{a_2}{.east}{a_1}
  \THit{b_1}{}{a_0}{.north east}{a_1}
  \path[bounce]
  \TBounce{a_2}{bend left}{a_1}{.north east}
  \TBounce{a_0}{bend right=20}{a_1}{.south}
  ;

  \THit{c_0}{bend right}{a_2}{.south east}{a_1}
  \THit{c_1}{bend right}{a_0}{.east}{a_1}
  \path[bounce]
  \TBounce{a_2}{bend left=20}{a_1}{.north}
  \TBounce{a_0}{bend right=30}{a_1}{.south east}
  ;

  \path[]
    (1.9,-1.3) edge[bend left=10,coopupdate] (-2.2,-0.7)
    (1.9, 3.3) edge[bend right=10,coopupdate] (-2.2,3)
  ;

  \THit{a_2}{bend left,out=40,in=80}{b_1}{.north west}{b_0}
  \path[bounce, bend right]
  \TBounce{b_1}{}{b_0}{.east}
  ;
\end{tikzpicture}
}
\caption{\figlabel{infer-ex-ph}%
  Exemple de Frappes de Processus canoniques avec trois composants ($a$, $b$ et $c$)
  et une sorte coopérative ($bc$).
  La dynamique de ce modèle est unitaire car il respecte bien le \vcrref{unitaire}.
  L'inférence du graphe des interactions peut donc être effectuée sur ce modèle.
}
\end{figure}



\begin{example}
  L'application de l'inférence du graphe des interactions aux Frappes de Processus canoniques
  de la \figref{infer-ex-ph} donne le graphe représenté à la \figref{infer-ig},
  contenant les arcs suivants :
  \begin{align*}
    E_+ &= \{\arcf{b}{+}{1}{a}, \arcf{c}{+}{1}{a}, \arcf{a}{+}{1}{a},
      \arcf{b}{+}{1}{b}, \arcf{c}{+}{1}{c}\} \\
    E_- &= \{\arcf{a}{-}{2}{b}\} \qquad\qquad\qquad\qquad\qquad
    E_\uns = \emptyset
  \end{align*}
  Ce graphe des interactions est proche de celui qui avait été proposé
  \vfigref{thomas}(gauche) bien qu'il ne soit pas équivalent,
  car chaque composant comporte une auto-action positive.
  Les auto-actions sur $b$ et $c$ sont la conséquence d'une stabilité globale
  sur plusieurs sous-états : en effet, $c$ n'évolue jamais,
  et $b$ n'évolue pas non plus lorsque $a_2$ n'est pas actif.
  L'auto-action sur $a$ est principalement causée par sa nature multi-valuée.
  
  \begin{figure}[ht]
  \centering
  \scalebox{1.2}{
  \begin{tikzpicture}[grn]
    \path[use as bounding box] (-1.3,-0.75) rectangle (3.5,1.5);
    \node[inner sep=0] (a) at (2,0) {a};
    \node[inner sep=0] (b) at (0,0) {b};
    \node[inner sep=0] (c) at (2,1.2) {c};
    \path[->]
      (b) edge[bend right] node[elabel, below=-2pt] {$+1$} (a)
      (c) edge node[elabel, right=-2pt] {$+1$} (a)
      (a) edge[bend right] node[elabel, above=-5pt] {$-2$} (b)
      (b) edge[in=-15+180, out=15+180, loop] node[elabel, left=-2pt] {+1} (b)
      (c) edge[in=15, out=-15, loop] node[elabel, right=-2pt] {+1} (c)
      (a) edge[in=15, out=-15, loop] node[elabel, right=-2pt] {+1} (a);
  \end{tikzpicture}
  }
  \caption{\figlabel{infer-ig}%
    Graphe des interactions inféré depuis les Frappes de Processus de la \figref{infer-ex-ph}.
  }
  \end{figure}
\end{example}

\begin{example}
  Si on remplace la sorte coopérative $bc$ des Frappes de Processus de la \figref{infer-ex-ph}
  par quatre actions
  \[ \PHfrappe{b_0}{a_1}{a_0} ; \PHfrappe{b_1}{a_1}{a_2} ;
     \PHfrappe{c_0}{a_1}{a_0} ; \PHfrappe{c_1}{a_1}{a_2} \]
  on obtient à nouveau le graphe des interactions donné à la \figref{infer-ig}.
\end{example}

\begin{example}
  L'ajout d'une action $\PHfrappe{a_2}{b_0}{b_1}$ aux Frappes de Processus canoniques de la
  \figref{infer-ex-ph} modifie le résultat de l'inférence.
  En effet, dans ce cas deux arcs non-signés vers $b$ sont inférés en lieu et place
  des arcs signés précédents :
  \begin{align*}
    E_+ &= \{\arcf{b}{+}{1}{a}, \arcf{c}{+}{1}{a}, \arcf{a}{+}{1}{a}, \arcf{c}{+}{1}{c}\}\\
    E_- &= \emptyset \qquad\qquad\qquad\qquad
    E_\uns = \{\arcf{a}{\uns}{2}{b}, \arcf{b}{\uns}{1}{b}\}
  \end{align*}
  Cela est dû au fait que les actions $\PHfrappe{a_2}{b_1}{b_0}$ et $\PHfrappe{a_2}{b_0}{b_1}$
  introduisent des oscillations causées uniquement par le processus $a_2$,
  ce qui implique une influence locale à la fois positive et négative,
  et est impossible à représenter au sein d'un modèle de Thomas.
\end{example}








\subsection{Inférence de la paramétrisation}
\seclabel{infer-param}

Une fois obtenu le graphe des interactions inféré selon la méthode proposée à la section
précédente, il est ensuite possible d'inférer une partie des paramètres discrets
propres à un modèle de Thomas, en fonction de la dynamique des Frappes de Processus canoniques
d'origine.
Cette inférence repose à nouveau sur une exploration exhaustive des comportements possibles
du modèle en fonction de l'état des prédécesseurs de chaque composant.
Cependant, cette inférence peut être partielle si le comportement modélisé
ne peut pas être représenté à l'aide d'un modèle de Thomas.
Dans ce cas, il est possible d'inférer une partie seulement des paramètres,
puis d'énumérer toutes les modèles compatibles avec cette paramétrisation partielle,
la dynamique du modèle et certaines contraintes de modélisation sur les paramètres.

\subsection{Inférence des paramètres}
\seclabel{infer-params}

Cette sous-section présente l'inférence des paramètres discrets indépendants
à partir d'un modèle de Frappes de Processus donné.
Ces résultats sont équivalents à ceux présentés par \citeasnoun{PMR10-TCSB},
auxquels nous ajoutons la notion de \emph{Frappes de Processus bien formées
pour l'inférence des paramètres}, définie au \crref{infer-params-ok},
et qui stipule que pour toute régulation de $a$ par $b$,
tous les processus de $\levels{b}{a}$ (\resp $\ulevels{b}{a}$)
possèdent la même influence sur $a$.

\begin{critere}[Frappes de Processus bien formées pour l'inférence des paramètres]
\crlabel{infer-params-ok}
  Des Frappes de Processus canoniques sont \emph{bien formées pour l'inférence des paramètres}
  si et seulement si leur dynamique est unitaire (\crref{unitaire})
  et si le graphe des interactions $(\components; E)$ inféré par \propref{inference-gi}
  vérifie :
  \begin{align*}
    \begin{split}
      \forall a \in \components &, \forall b \in \RRBreg{a},
        \forall N \in \{ \levels{b}{a}, \ulevels{b}{a} \}, \forall i,j \in N, \\
        %\forall (i, j \in \levels{b}{a} \vee i, j \in \ulevels{b}{a}), \\
      & \forall c \in \PHs, ( (b \neq a \wedge c = a) \vee
        (\exists \myupsilon \in \pred(a), c \in \conn(\myupsilon) \wedge b \in \compin(c)), \\
%        (c \in \PHpredec{a} \setminus \components \wedge b \in \PHdirectpredec{c})), \\
      & \qquad \PHfrappe{b_i}{c_k}{c_l}\in\PHa \Leftrightarrow \PHfrappe{b_j}{c_k}{c_l}\in\PHa
    \end{split}
  \end{align*}
\end{critere}

On souhaite dans la suite inférer le paramètre discret $K_{a,\omega}$,
pour un composant $a \in \components$ et un ensemble $\omega \subset \RRBreg{a}$
de ressources donnés.
Cette inférence se base, à l'instar de l'inférence du graphe des interactions, sur une analyse
exhaustive des sous-états des régulateurs de $a$.
Pour chaque sorte $b \in \RRBreg{a}$, on définit un \todo{\emph{contexte}} $C^b_{a,\omega}$
(\eqref{param-context}) qui recense tous les processus qui interagissent avec $a$
dans tous les sous-états représentés par l'ensemble de ressources $\omega$.
Le contexte d'une sorte coopérative $\myupsilon$ régulant $a$ est l'ensemble des processus
focaux correspondant à ces sous-états (\eqref{param-context-coop}).
Enfin, $C_{a,\omega}$ fait référence à l'union de tous ces contextes
(\eqref{param-context-total}).

\begin{align}
\eqlabel{param-context}
  \forall b \in \components,~
  C_{a,\omega}^b & \DEF \begin{cases}
    \levels{b}{a}  & \text{si $b \in \omega$,}\\
    \ulevels{b}{a} & \text{si $b \notin \omega$,}\\
    L_b            & \text{sinon}\\
  \end{cases}
  \\
\eqlabel{param-context-coop}
  \forall \myupsilon \in \pred(a) \cap \cs,
    C_{a,\omega}^\myupsilon & \DEF \{ \pfp_{\sigma}(\myupsilon) \mid
    \sigma \in \bigtimes{b \in \compin(\myupsilon)} C_{a,\omega}^b \} \\
\eqlabel{param-context-total}
  C_{a,\omega} & \DEF \bigcup_{b \in \pred(a)} C^b_{a,\omega}
\end{align}

Pour inférer le paramètre recherché, nous calculons les \emph{processus focaux} de $a$,
qui sont les processus vers lesquels tend le niveau d'expression de $a$ en présence
de certains autres processus (\defref{focals}).
Ainsi, $\focals(a, S, T)$ donne l'ensemble des processus de $a$ accessibles
en partant de n'importe quel processus dans $S$, et à condition de ne jouer que des
actions dont le frappeur est dans $T$.
Cette notion se base sur un graphe recensant tous les bonds que peuvent faire les processus
de $a$ ;
si ce graphe est acyclique, alors l'ensemble des processus focaux est l'ensemble des processus
de $a$ qui ne sont pas frappés ---~et vers lesquels $a$ va avoir tendance à évoluer.

\begin{definition}[$\focals(a,S,T)$]
\deflabel{focals}
  L'ensemble des \emph{processus focaux} de $a \in \components$ depuis $S \subset \PHl_a$
  pour le sous-état $T \subset \Proc$ est donné par :
  \[
    \focals(a, S, T) \DEF
    \begin{cases}
      \{ a_i \in V \mid \nexists (a_i,a_j)\in E\} & \text{si $(V,E)$ est acyclique},\\
      \emptyset & \text{sinon}\\
    \end{cases}
  \]
  où $(V,E)$ est le graphe orienté suivant :
  \begin{align*}
    E & \DEF \{ (a_j; a_k) \in (\PHl_a \times \PHl_a) \mid
      \exists \PHfrappe{b_i}{a_j}{a_k} \in \PHh^{(2)}, b_i \in T \wedge a_j \in S \} \\
    V & \DEF S \cup \{ a_k \in \PHl_a \mid \exists (a_j; a_k) \in E \}
    %\eqlabel{bounce-graph}
  \end{align*}
\end{definition}

Le paramètre $K_{a,\omega}$ détermine les niveaux d'expression vers lesquels tend $a$
en présence du contexte $C_{a,\omega}$.
Cette valeur peut être calculée à l'aide de $\focals$ qui permet justement
de retrouver les processus focaux en présence de certaines ressources.
Ainsi, on peut en conclure que $K_{a,\omega} = \focals(a,C^a_{a,\omega},C_{a,\omega})$
dans tous les cas où cette valeur est un intervalle non vide (\propref{inference-param}).

\begin{proposition}[Inférence des paramètres]
\proplabel{inference-param}
  Soient $\PH = (\PHs, \PHl, \PHh)$ des Frappes de Processus bien formées pour l'inférence des
  paramètres, $\GI = (\components, E)$ le graphe des interactions inféré pour $\PH$
  et $\omega \subset \RRBres{a}$ un ensemble de ressources de $a$.
  Si $\focals(a,C^a_{a,\omega},C_{a,\omega})$ est un intervalle non vide,
  avec $\focals(a,C^a_{a,\omega},C_{a,\omega}) = \segm{a_i}{a_j}$,
  alors $K_{a,\omega} = \segm{i}{j}$.
\end{proposition}



\begin{example}
%\label{ex:infer-param-runningPH-1}
  Si on l'applique aux Frappes de Processus de la \figref{infer-ex-ph},
  la méthode d'inférence des paramètres donnée dans cette section est conclusive sur tous
  les paramètres et donne :
  \begin{align*}
    K_{a, \emptyset} &= \segm{0}{0} &
    K_{b, \emptyset} &= \segm{0}{0} \\
    K_{a, \{a\}} &= \segm{0}{0} &
    K_{b, \{a\}} &= \segm{0}{0} \\
    K_{a, \{c\}} &= \segm{1}{1} &
    K_{b, \{b\}} &= \segm{1}{1} \\
    K_{a, \{b\}} &= \segm{1}{1} &
    K_{b, \{a,b\}} &= \segm{0}{0} \\
    K_{a, \{b,c\}} &= \segm{1}{1} &
    K_{c, \emptyset} &= \segm{0}{0} \\
    K_{a, \{a,b,c\}} &= \segm{2}{2} &
    K_{c, \{c\}} &= \segm{1}{1} \\
    K_{a,\{a,b\}} &= \segm{1}{1} &
    K_{a,\{a,c\}} &= \segm{1}{1}
  \end{align*}
\end{example}

\todo{Exemple partiellement conclusif}

En observant la \propref{inference-param}, on constate que l'inférence
de certains paramètres peut ne pas être possible.
Cela peut être notamment dû à des coopérations mal définies entre les régulateurs d'un composant :
lorsque deux régulateurs frappent un même composant de façon indépendante, leurs actions peuvent
avoir des effets opposés, créant des oscillations dans la dynamique.
Un tel indéterminisme ne peut pas être représenté à l'aide d'un modèle de Thomas étant donné
que dans une configuration de ressources données, un composant possède un unique attracteur,
représenté par le paramètre discret correspondant,
et ne peut donc évoluer que dans une seule direction.
Il est possible de résoudre ces cas non conclusifs
(autrement dit, de supprimer ces comportements oscillants)
en raffinant le modèle à l'aide de suppressions d'actions ou
en s'assurant que les coopérations sont correctement définies à l'aide de sortes coopératives
afin d'éviter des influences opposées depuis des régulateurs concurrents.



\subsection{Énumération des paramétrisations admissibles}
\seclabel{enum-param}

Lors de la construction d'un modèle de Thomas, trouver la paramétrisation compatible avec
le comportement désiré est nécessaire pour obtenir un modèle complet.
Cependant, cette étape possède une complexité inhérente à ce type de formalisme,
car le nombre de paramètres que contient le modèle croît exponentiellement dans
la taille du graphe des interactions
(plus précisément, dans le nombre de régulations vers chaque composant).
La méthode d'inférence des paramètres présentée précédemment permet cependant d'obtenir
certaines informations sur ces paramètres en fonction de la dynamique des Frappes de Processus
canoniques étudiées.
Ces informations permettent donc de restreindre l'espace des paramétrisations possibles,
et donc d'obtenir plus facilement le modèle recherché.

En d'autres termes,
lors de l'inférence d'un modèle de Thomas selon la méthode décrite précédemment,
il arrive que certains paramètres ne puissent pas être inférés.
Le modèle obtenu est alors partiel, et correspond à un ensemble
plus ou moins large de modèles complets.
En énumérant les valeurs possibles de chaque paramètre, il est envisageable de retrouver
l'ensemble des modèles \emph{compatibles} avec ces valeurs.

Nous délimitons tout d'abord la validité d'un paramètre (\crref{params-valide}) afin d'assurer
que toutes les transitions dans le modèle de Thomas résultant
sont permises par la dynamique des Frappes de Processus canoniques étudiées.
Cette propriété est vérifiée en s'assurant,
pour chaque configurations de ressources possibles,
de l'existence d'une frappe faisant bondir
le processus d'une sorte vers le paramètres correspondant.
Ainsi, conjointement avec le fait que les Frappes de Processus étudiées
sont bien formées pour l'inférence des paramètres,
nous assurons que pour toute transition dans le modèle de Thomas inféré,
il existe une transition équivalente dans les Frappes de Processus d'origine.
Nous remarquons par ailleurs que les paramètres inférés à l'aide de la \vpropref{inference-param}
vérifient déjà cette propriété.



\begin{critere}[Validité d'un paramètre]
\crlabel{params-valide}
  Un paramètre $K_{a,\omega}$ est \emph{valide} pour les Frappes de Processus $\PH$
  si et seulement si :
  \begin{align*}
    \forall a_i\in C^a_{a,\omega}, a_i \notin K_{a,\omega} \Longrightarrow
      (& \exists \PHfrappe{c_k}{a_i}{a_j}\in\PHa, c_k \in C^c_{a,\omega} \\
      & \wedge a_i < K_{a,\omega} \Rightarrow j > i \wedge  a_i > K_{a,\omega} \Rightarrow j <i )
  \end{align*}
\end{critere}

Nous utilisons de plus plusieurs contraintes de modélisation \citeaffixed{BernotSemBRN}{tirées de}
afin d'assurer une cohérence des paramètres avec les signes des régulations du graphe
des interactions préalablement inféré.
L'\emph{hypothèse des valeurs extrêmes} (\crref{param-enum-extreme})
stipule que les niveaux extrêmes d'un composant $a$ (c'est-à-dire $0$ et $l_a$)
doivent chacun apparaître dans au moins un paramètre.
L'\emph{hypothèse d'activité} (\crref{param-enum-activity})
stipule en outre que toutes les régulations doivent être fonctionnelles,
c'est-à-dire que pour chaque régulateur d'un composant,
il existe au moins une configuration dans laquelle la présence ou l'absence de ce régulateur
modifie le paramètre considéré.
Enfin, l'\emph{hypothèse de monotonicité} (\crref{param-enum-monotonicity})
stipule qu'ajouter un activateur (\resp inhibiteur) aux ressources d'un composant
ne peut qu'augmenter (\resp diminuer) la valeur du paramètre considéré.
La relation d'ordre $\leqsegm$ entre deux paramètres discrets s'applique à des segments
et est définie à la \vsecref{notations}.

\begin{critere}[Hypothèse des valeurs extrêmes]
\crlabel{param-enum-extreme}
  Soit $\GI = (\components, E)$ un graphe des interactions.
  Une paramétrisation $K$ sur $\GI$ satisfait l'\emph{hypothèse des valeurs extrêmes}
  si et seulement si :
  \begin{align*}
    \forall b \in \components, \RRBreg{b} \neq \emptyset \Longrightarrow
    \exists \omega \subset \RRBreg{b}, 0 \in K_{b,\omega} \wedge
    \exists \omega' \subset \RRBreg{b}, l_b \in K_{b,\omega'}
  \end{align*}
\end{critere}

\begin{critere}[Hypothèse d'activité]
\crlabel{param-enum-activity}
  Soit $\GI = (\components, E)$ un graphe des interactions.
  Une paramétrisation $K$ sur $\GI$ satisfait l'\emph{hypothèse d'activité}
  \begin{align*}
    \forall b \in \components, \forall a \in \RRBreg{b}, \exists \omega \subset \RRBreg{b},
    K_{b,\omega} \neq K_{b,\omega \cup \{ a \}}
  \end{align*}
\end{critere}

\begin{critere}[Hypothèse de monotonicité]
\crlabel{param-enum-monotonicity}
  Soit $\GI = (\components, E)$ un graphe des interactions.
  Une paramétrisation $K$ sur $\GI$ satisfait l'\emph{hypothèse de monotonicité}
  si et seulement si :
  \begin{align*}
    \forall b \in \components,
    \forall A^+ \subset \{ a \in \components \mid \arc{a}{+}{t}{b} \in E_+ \}&,
    \forall A^- \subset \{ a \in \components \mid \arc{a}{-}{t}{b} \in E_- \},\\
    K_{b,\omega \cup A^-} & \leqsegm K_{b,\omega \cup A^+}
  \end{align*}
\end{critere}

\todo{Ajouter exemple avec cas non conclusifs.}

% 
% \begin{example}\label{ex:enum-param-runningPH-1}
% The parametrization inferred in \pref{ex:infer-param-runningPH-1} was partial because $K_{a,\{a,b\}}$ and $K_{a,\{a,c\}}$ could not be inferred.
% It is however possible to enumerate all complete and admissible parametrizations
% compatible with both the inferred parameters, and the properties of this subsection.
% This enumeration gives 9 different parametrizations which correspond to the 3 possible values
% for each of the two parameters that could not be inferred:
% \begin{align*}
%   K_{a,\{a,b\}} &\in \{ \segm{1}{1}, \segm{1}{2}, \segm{2}{2} \} \\
%   K_{a,\{a,c\}} &\in \{ \segm{1}{1}, \segm{1}{2}, \segm{2}{2} \}
% \end{align*}
% We note that for all solutions, $0 \notin K_{a,\{a,b\}} \wedge 0 \notin K_{a,\{a,c\}}$.
% This is due to the monotonicity assumption (\pref{pro:param-enum-monotonicity}) which especially states that:
% \begin{align*}
%   K_{a,\{b\}} \leqsegm K_{a,\{a,b\}} \wedge
%   K_{a,\{c\}} \leqsegm K_{a,\{a,c\}}
% \end{align*}
% 
% Finally, we note that $\segm{1}{1}$ belongs to the possible values for both parameters.
% Therefore this enumeration allows, from the model in \pref{fig:runningPH-1},
% to find the behavior of the model refined with a cooperative sort described in \pref{fig:runningPH-2}.
% \end{example}

% Équivalence avec les réseaux d'automates synchronisés
% Équivalence avec les réseaux d'automates synchronisés

\section{Équivalence avec les réseaux d'automates synchronisés}
\seclabel{phm2an}

Nous nous intéressons ici au lien entre les Frappes de Processus avec actions plurielles
et les réseaux d'automates synchronisés.
Nous montrons notamment que ces deux formalismes sont équivalents
et nous exhibons pour cela deux traductions d'un formalisme vers l'autre.
%(\defref{phm2an,an2phm} \vdefpageref{phm2an}).
%(\defref[s]{phm2an} et \defref*[vref]{an2phm}).
Cette équivalence est intéressante car elle montre clairement le lien entre ce formalisme
de Frappes de Processus et celui plus répandu des réseaux d'automates synchronisés.
En effet, chaque action plurielle $\PHfrappemult{A}{B}$
possède la même dynamique qu'un ensemble de transitions synchronisées
partant chacune d'un processus de l'ensemble $A$ et
arrivant dans le processus de la même sorte de l'ensemble $A \recouvre B$%
\footnote{La notation $A \recouvre B$, formalisée à la \vdefref{recouvrementps}, représente
l'ensemble où chaque processus de $A$ a été remplacé par
le processus de $B$ de la même sorte, s'il existe.}.

\myskip

Nous rappelons tout d'abord la définition d'un réseau d'automates synchronisés (\defref{an})
ainsi que la relation de transition entre deux états d'un tel modèle (\defref{an-sem})
ce qui permet d'en définir la dynamique.

\begin{definition}[Réseau d'automates synchronisés]
\deflabel{an}
  Un \emph{réseau d'automates synchronisés} est un quadruplet $\AN = (\ANs; \ANl; \ANi; \ANt)$
  où :
  \begin{itemize}
    \item $\ANs \DEF \{a, b, \dots\}$ est l'ensemble fini et dénombrable des \emph{automates} ;
    \item $\ANl \DEF \bigtimes{a \in \ANs} \ANl_a$ est l'ensemble fini des \emph{états},
      où $\ANl_a = \{a_0, \ldots, a_{l_a}\}$ est l'ensemble fini et dénombrable
      des \emph{états locaux} de l'automate $a \in \ANs$ et $l_a \in \sN^*$,
      chaque état local appartenant à un unique automate :
      $\forall (a_i; b_j) \in \ANl_a \times \ANl_b, a \neq b \Rightarrow a_i \neq b_j$ ;
    \item $\ANi \DEF \{\ell_1, \dots, \ell_m\}$ est l'ensemble fini des
      \emph{libellés} de transitions ;
    \item $\ANt \DEF \{ \ANaction{a_i}{\ell}{a_j} \mid a \in \ANs \wedge a_i \in \ANl_a \wedge
      \ell \in \ANi \}$ est l'ensemble fini des \emph{transitions} entre états locaux.
  \end{itemize}
  Pour tout libellé $\ell \in \ANi$, on note
  $\precond{\ell} \DEF \{ a_i \mid \ANaction{a_i}{\ell}{a_j} \in \ANt \}$
  et $\postcond{\ell} \DEF \{ a_j \mid \ANaction{a_i}{\ell}{a_j} \in \ANt \}$.
%   et $\invcond{\ell} \DEF \{ a_i \mid \ANaction{a_i}{\ell}{a_i} \in \ANt \}$.
  L'ensemble des états locaux des automates est dénoté par
  $\ANProc \DEF \bigcup_{a \in \ANs} \ANl_a$.
  Enfin, étant donné un état global $s \in \ANl$, $s(a) = a_i \in \ANl_a$
  fait référence à l'état local de l'automate $a \in \ANs$.
\end{definition}

\begin{definition}[Sémantique des réseaux d'automates ($\ANtrans$)]
\deflabel{an-sem}
  Étant donné un réseau d'automates synchronisés $\AN = (\ANs; \ANl; \ANi; \ANt)$,
  un libellé $\ell$ est dit \emph{jouable} dans un état $s \in \ANl$ si et seulement si :
  $\forall a_i \in \precond{\ell}, s(a) = a_i$.
  Dans ce cas, on note $(s \play \ell)$ l'état résultant du jeu de toutes les transitions
  libellées par $\ell$, défini par :
  $s \play \ell = s \recouvre \postcond{\ell}$.
%   $\forall a_j \in \postcond{\ell}, (s \play \ell)(a) = a_j \wedge
%     \forall b \in \ANs, \ANl_b \cap \precond{\ell} = \emptyset \Rightarrow
%     (s \play \ell)(b) = s(b)$.
  De plus, on note alors : $s \ANtrans (s \play \ell)$.
%   Étant donné un réseau d'automates synchronisés $\AN = (\ANs; \ANl; \ANi; \ANt)$,
%   la relation de transition globale entre deux états du réseau
%   $\ANtrans \subset \ANl \times \ANl$ est définie par :
%   \begin{align*}
%     s \ANtrans s' \EQDEF \exists \ell \in \ANi, &\forall a_i \in \precond{\ell}, s(a) = a_i
%       \wedge \forall a_j \in \postcond{\ell}, s'(a) = a_j \\
%     \wedge &\forall b \in \ANs, \ANl_b \cap \precond{\ell} = \emptyset \Rightarrow s(b) = s'(b)
%   \end{align*}
\end{definition}

\begin{remark}
  Nous notons que les réseaux d'automates synchronisés ainsi définis sont non-déterministes,
  tant au niveau global du modèle qu'au niveau local des automates.
  Cette vision s'oppose à d'autres sémantiques des réseaux d'automates
  comme celles de \citeasnoun{Richard10} ou de \citeasnoun{RRT08},
  qui définissent la dynamique de leurs modèles à l'aide de fonctions de transition locales,
  qui sont par définition déterministes.
  Ces fonctions ont en effet la forme : $f_a : \ANl \rightarrow \ANl_a$
  et associent donc à chaque état global du modèle un état local (unique) pour chaque automate.
  La définition des réseaux d'automates synchronisés que nous proposons ici (\defref{an})
  n'empêche en revanche pas l'existence de deux libellés $\ell_1, \ell_2 \in \ANi$
  tels que $\precond{\ell_1} = \precond{\ell_2}$ mais $\postcond{\ell_1} \neq \postcond{\ell_2}$.
  Cela implique notamment l'existence de deux transitions entre état locaux
  $\ANaction{a_i}{\ell_1}{a_j}$ et $\ANaction{a_i}{\ell_2}{a_k}$
  avec $a_j \neq a_k$, d'où un non-déterminisme au niveau des automates.
\end{remark}

Pour tout modèle de Frappes de Processus avec actions plurielles $\PH$,
la \defref{phm2an} propose une traduction de $\PH$
en un réseau d'automates synchronisés $\phmtoan[\PH]$ équivalent,
et le \thmref{bisimulationphm2an} établit la bisimilarité entre les deux modèles.
La notation $\recouvre$ qui est utilisée dans la définition
qualifie le recouvrement d'un ensemble de processus de sortes distinctes
par un autre comprenant uniquement des processus issus des mêmes sortes
(\defref{recouvrementps}).
Cette notion de recouvrement est une extension
du recouvrement d'un état par un ensemble de processus
tel que précédemment formalisé à la \vdefref{recouvrement}.

\begin{definition}[Recouvrement ($\recouvre : \PHsublset \times \PHsublset \rightarrow \PHsublset$)]
\deflabel{recouvrementps}
  Étant donné un sous-état désordonné $ps \in \PHsublset$ et un processus $a_i \in \Proc$,
  tel que $a \in \sortes{ps}$, on définit :
  $(ps \recouvre a_i) = (ps \setminus \PHl_a) \cup \{ a_i \}$.
  On étend de plus cette définition
  au recouvrement par un ensemble de processus de sortes distinctes
  $ps' \in \PHsublset$ tel que $\sortes{ps'} \subset \sortes{ps}$
  comme étant le recouvrement de $ps$ par chaque processus de $ps'$ :
  $ps \recouvre ps' = ps \underset{a_i \in ps'}{\recouvre} a_i$.
\end{definition}

\begin{definition}[Réseau d'automates équivalent ($\phmtoansymbol$)]
\deflabel{phm2an}
  Le réseau d'automates synchronisés équivalent aux Frappes de Processus
  avec actions plurielles $\PH = (\PHs; \PHl; \PHh)$
  est défini par : $\phmtoan = (\PHs; \PHl; \ANi; \ANt)$, où :
  \begin{itemize}
    \item $\ANi = \{ \ell_h \mid h \in \PHh \}$ ; % est l'ensemble des libellés de transitions ;
    \item $\ANt = \{ \ANaction{a_i}{\ell_h}{a_j} \mid
      h \in \PHh \wedge h = \PHfrappemult{A}{B} \wedge a_i \in A \wedge a_j \in A \recouvre B \}$.
      % est l'ensemble des transitions locales.
  \end{itemize}
\end{definition}

\begin{theorem}[$\PH \approx \phmtoan$]
\thmlabel{bisimulationphm2an}
  Soient $\PH = (\PHs; \PHl; \PHh)$ des Frappes de Processus avec actions plurielles.
  On a :
  \[\forall s, s' \in \PHl, s \PHtrans s' \Longleftrightarrow s \trans{\phmtoan} s' \enspace.\]
\end{theorem}

\begin{proof}
  Soient $s, s' \in \PHl$.
  On pose : $\phmtoan = (\ANs; \ANl; \ANi; \ANt)$.
  
  ($\Rightarrow$) Supposons que $s \PHtrans s'$, c'est-à-dire qu'il existe une action $h \in \PHh$
    telle que $s' = s \play h$.
    Posons : $h = \PHfrappemult{A}{B}$.
    D'après la \defref{phm2an},
    l'existence de cette action dans $\PH$ implique celle d'un libellé $\ell_h$ dans $\phmtoan$
    ainsi que de l'ensemble de transitions
    $\ANt_h = \{ a_i \xrightarrow{\ell_h} a_j \mid a_i \in A \wedge a_j \in A \recouvre B \}$.
    Autrement dit, $\precond{\ell_h} = A$, donc $\ell_h$ est jouable dans $s$
    si et seulement si $A \subseteq s$.
    De plus, $\postcond{\ell_h} = \invariant{h} \cup B$, donc
    $(s \play \ell_h) = s \recouvre (\invariant{h} \cup B) = s \recouvre B = s'$
    car $\invariant{h} \subseteq A \subseteq s$.
  
  ($\Leftarrow$) Supposons que $s \trans{\phmtoan} s'$,
    c'est-à-dire qu'il existe un libellé $\ell \in \ANi$ et un ensemble de transitions
    ayant ce libellé : $\ANt_\ell = \{ a_i \xrightarrow{\ell} a_j \in \ANt \}$,
    tels que $s' = s \play \ell$.
    D'après la \defref{phm2an}, cela signifie notamment qu'il existe une action
    $h = \PHfrappemult{A}{B} \in \PHh$ telle que $\ell = \ell_h$, et que :
    $\ANt_\ell = \{ a_i \xrightarrow{\ell} a_j \mid a_i \in A \wedge a_j \in A \recouvre B \}$.
    Étant donné que $\invariant{h}$ et $\cible{h}$ forment une partition de $A$,
    $\ANt_\ell$ peut être découpé en deux ensembles, selon les invariants et les cibles de $h$ :
    $\ANt_\ell = \{ a_i \xrightarrow{\ell} a_i \mid a_i \in \invariant{h} \} \cup
      \{ a_i \xrightarrow{\ell} a_j \mid a_i \in \cible{h} \wedge a_j \in B \}$.
    Ainsi, $s' = s \recouvre (\invariant{h} \cup B) = s \recouvre B = s \play h$.
\end{proof}

Pour finir, nous proposons à la \defref{an2phm} la traduction inverse
d'un réseau d'automates synchronisés $\AN$
en des Frappes de Processus avec actions plurielles équivalentes $\antophm$.
Le \thmref{bisimulationan2phm} stipule que le modèle obtenu est bien bisimilaire
au modèle d'origine.
Enfin, le \thmref{equivphman} résume les résultats de cette section
en statuant l'équivalence d'expressivité entre les Frappes de Processus avec
actions plurielles et les réseaux d'automates synchronisés.

\begin{definition}[Frappes de Processus équivalentes ($\antophmsymbol$)]
\deflabel{an2phm}
  Les Frappes de Processus avec actions plurielles
  équivalentes au réseau d'automates synchronisés $\AN = (\PHs, \PHl, \ANi, \ANt)$
  sont définies par $\antophm = (\ANs, \ANl, \PHh)$, où :
%   $\PHh = \{ \PHfrappemult{\precond{\ell}}{(\postcond{\ell} \setminus \invcond{\ell})}
%     \mid \ell \in \ANi \}$.
  \[\PHh = \{ \PHfrappemult{\precond{\ell}}{B} \mid \ell \in \ANi \wedge
    B = \postcond{\ell} \setminus \{ a_i \in \ANProc \mid \ANaction{a_i}{\ell}{a_i} \in \ANt \}
    \}\]
\end{definition}

\begin{theorem}[$\AN \approx \antophm$]
\thmlabel{bisimulationan2phm}
  Soit $\AN = (\ANs; \ANl; \ANi; \ANt)$ un réseau d'automates synchronisés.
  On a :
  \[\forall s, s' \in \ANl, s \ANtrans s' \Longleftrightarrow s \trans{\antophm} s' \enspace.\]
\end{theorem}

\begin{proof}
  Soient $s, s' \in \PHl$.
  On pose : $\antophm = (\ANs; \ANl; \PHh)$.
  
  ($\Rightarrow$) Supposons que $s \ANtrans s'$,
    c'est-à-dire qu'il existe un libellé $\ell \in \ANi$ et un ensemble de transitions
    ayant ce libellé : $\ANt_\ell = \{ a_i \xrightarrow{\ell} a_j \in \ANt \}$,
    tels que $s' = s \play \ell$.
    D'après la traduction donnée à la \defref{an2phm}, il existe donc une action
    $h = \PHfrappemult{A}{B} \in \PHh$ telle que $A = \precond{\ell}$ et
    $B = \postcond{\ell} \setminus \{ a_i \in \ANProc \mid \ANaction{a_i}{\ell}{a_i} \in \ANt \}$.
    Or $s' = s \recouvre \postcond{\ell}
      = s \recouvre (B \cup \{ a_i \in \ANProc \mid \ANaction{a_i}{\ell}{a_i} \in \ANt \})
      = s \recouvre B$
    car $\{ a_i \in \ANProc \mid \ANaction{a_i}{\ell}{a_i} \in \ANt \} \subseteq s$.
    Ainsi, $h$ est jouable dans $s$ et $s' = s \play h$.
  
  ($\Leftarrow$) Supposons que $s \trans{\antophm} s'$,
    c'est-à-dire qu'il existe une action $h = \PHfrappemult{A}{B} \in \PHh$
    telle que $s' = s \play h$.
    D'après la traduction de la \defref{an2phm},
    cela signifie qu'il existe un libellé $\ell \in \ANi$ et un ensemble de transitions
    ayant ce libellé : $\ANt_\ell = \{ a_i \xrightarrow{\ell} a_j \in \ANt \}$,
    tels que : $A = \precond{\ell}$ et
    $B = \postcond{\ell} \setminus \{ a_i \in \ANProc \mid \ANaction{a_i}{\ell}{a_i} \in \ANt \}$.
    Comme $h$ est jouable dans $s$, alors $A \subseteq s$, donc $\ell$ est aussi jouable dans $s$.
    De plus, $s' = s \play h = s \recouvre B = s \recouvre (B \cup \invariant{h})
      = s \recouvre \postcond{\ell}$.
\end{proof}

\begin{theorem}[Équivalence entre réseaux d'automates synchronisés
  et Frappes de Processus avec actions plurielles]
\thmlabel{equivphman}
  Les Frappes de Processus avec actions plurielles sont aussi expressives
  que les réseaux d'automates synchronisés.
\end{theorem}

\begin{proof}
  D'après les \defref{phm2an,an2phm} et les \thmref{bisimulationphm2an,bisimulationan2phm}
  associés, tout modèle de Frappes de Processus avec actions plurielles peut être représenté
  à l'aide d'un réseau d'automates synchronisés, et inversement.
\end{proof}

%\todo{Exemple}


% Traduction en réseaux de Petri
% Traduction en réseaux de Petri

% \section{Réseaux de Petri}
%   \subsection{Réseaux de Petri saufs (une place par processus)}
%   \subsection{Réseaux de Petri bornés (une place par sorte)}

\section{Traduction en réseaux de Petri}
\seclabel{trad-petri}

\todo{Glu partout}

Dépliage (avec arcs inhibiteurs) \cite{baldan00}

\todo{Pour cette traduction, les indices des processus ont un sens particulier
et doivent être choisis dans $\sN$.}

\begin{definition}[Réseau de Petri ($\PT$)]
\deflabel{pt}
  Un \emph{réseau de Petri} est un 6-uplet
  $\PT = (\PTp; \PTt; \PTPre; \PTPost; \PTLect; \PTInh)$ où :
  \begin{itemize}
    \item $\PTp$ est l'ensemble fini des \emph{places},
    \item $\PTt$ est l'ensemble fini des \emph{transitions},
    \item $\PTPre \subset \PTp \times \sN \times \PTt$
%    \item $\PTPre : \PTp \times \PTt \rightarrow \sN$
      est l'ensemble des arcs entrant dans une transition,
    \item $\PTPost \subset \PTt \times \sN \times \PTp$
      est l'ensemble des arcs sortant d'une transition,
    \item $\PTLect \subset \PTp \times \sN \times \PTt$ est l'ensemble des arcs de lecture,
    \item $\PTInh \subset \PTp \times \sN \times \PTt$ est l'ensemble des arcs inhibiteurs,
  \end{itemize}
  avec : $\PTp \cap \PTt = \emptyset$ et $\PTp \cup \PTt \neq \emptyset$.
  De plus, chaque arc est unique ; autrement dit, pour chacun des ensembles
  $X \in \{ \PTPre, \PTPost, \PTLect, \PTInh \}$, on a :
  \[\forall (a; i; b) \in X, \forall (a'; i'; b') \in X,
    (a = a' \wedge b = b') \Rightarrow i = i'\]
  On note par ailleurs, pour tout couple d'éléments $a, b \in \PTp \cup \PTt$ :
%  $\forall X \in \{ \PTPre, \PTPost, \PTLect, \PTInh \},
  \[X(p, t) = \begin{cases}
              i & \text{ si } (a; i; b) \in X \\
              0 & \text{ sinon}
            \end{cases}\]
%   
%   place $p \in \PTp$ et toute transition $t \in \PTt$,
%   $\PTPost(t, p) = $
%   $\forall X \in \{ \PTPre, \PTPost, \PTLect, \PTInh \},
%     \forall p \in \PTp, \forall t \in \PTt$,
%   $X(p, t) \in \{ \PTPre, \PTPost, \PTLect, \PTInh \},$
  
  Un \emph{marquage} d'un réseau de Petri est une fonction $M : \PTp \rightarrow \sN$.
  Un \emph{réseau de Petri avec marquage initial} est un couple $(\PT; M_0)$
  où $\PT$ est un réseau de Petri et $M_0$ est un marquage de $\PT$.
%   
%   Pour toute transition $t \in \PTt$, on note de plus :
%   $\PTPre(t) = $
\end{definition}

\begin{definition}[Sémantique des réseaux de Petri ($\PTtrans$)]
\deflabel{pt-sem}
  Étant donnés un réseau de Petri $\PT = (\PTp; \PTt; \PTPre; \PTPost; \PTLect; \PTInh)$
  et un marquage $M$,
  une transition $t \in \PTt$ est dite \emph{jouable} dans $M$ si et seulement si :
  \[\forall p \in \PTp, M(p) \geq \PTPre(p, t) \wedge M(p) \geq \PTLect(p, t)
    \wedge M(p) < \PTInh(p, t)\]
  Dans ce cas, on note $(M \play t)$ le marquage résultant du jeu de cette transition depuis $M$,
  défini par :
  \[\forall p \in \PTp, (M \play t)(p) = M(p) - \PTPre(p, t) + \PTPost(t, p)\]
  De plus, on note alors : $M \PTtrans (M \play t)$.
\end{definition}

\begin{definition}[Réseau de Petri équivalent ($\phmtoptsymbol$)]
\deflabel{phm2pt}
  Le réseau de Petri équivalent aux Frappes de Processus
  avec actions plurielles $\PH = (\PHs; \PHl; \PHh)$
  est défini par : $\phmtoan = (\PTp; \PTt; \PTPre; \PTPost; \PTLect; \PTInh)$, où :
  \begin{itemize}
    \item $\PTp = \PHs$,
    \item $\PTt = \PHh$,
    \item $\PTPre = \{ (b, j-k, h) \mid h \in \PHh \wedge b_j \in \cible{h} \wedge
      b_k \in \bond{h} \wedge k-j < 0 \}$
    \item $\PTPost = \{ (h, k-j, b) \mid h \in \PHh \wedge b_j \in \cible{h} \wedge
      b_k \in \bond{h} \wedge k-j > 0 \}$
    \item $\PTLect = \{ (a, i, h) \mid h \in \PHh \wedge a_i \in \frappeur{h} \wedge i > 0 \}$
    \item $\PTInh = \{ (a, i+1, h) \mid h \in \PHh \wedge a_i \in \frappeur{h} \wedge i < l_a \}$
  \end{itemize}
  De plus, pour tout état $s \in \PHl$, on note
  $M^s$ le marquage correspondant, défini par :
  $\forall a \in \PTp, M^s(a) = i \text{ tel que } \PHget{s}{a} = a_i$.
\end{definition}

\begin{theorem}[$\PH \approx \phmtopt$]
\thmlabel{bisimulationphm2pt}
  Soient $\PH = (\PHs; \PHl; \PHh)$ des Frappes de Processus avec actions plurielles.
  On a :
  \[\forall s, s' \in \PHl, s \PHtrans s' \Longleftrightarrow
    M^s \trans{\phmtopt} M^{s'} \enspace.\]
\end{theorem}

\begin{proof}
  Soient $s, s' \in \PHl$.
  On pose : $\phmtopt = (\PTp; \PTt; \PTPre; \PTPost; \PTLect; \PTInh)$.
  Pour toute action $h \in \PHh$, on notera dans la suite :
  \begin{itemize}
    \item $\PTPre_h = \{ (b, j-k, h) \mid b_j \in \cible{h} \wedge
      b_k \in \bond{h} \wedge k-j < 0 \}$,
    \item $\PTPost_h = \{ (h, k-j, b) \mid b_j \in \cible{h} \wedge
      b_k \in \bond{h} \wedge k-j > 0 \}$,
    \item $\PTLect_h = \{ (a, i, h) \mid a_i \in \frappeur{h} \wedge i > 0 \}$,
    \item $\PTInh_h = \{ (a, i+1, h) \mid a_i \in \frappeur{h} \wedge i < l_a \}$.
  \end{itemize}
  
  ($\Rightarrow$) Supposons que $s \PHtrans s'$, c'est-à-dire qu'il existe une action $h \in \PHh$
    telle que $s' = s \play h$.
    Posons : $h = \PHfrappemult{A}{B}$.
    D'après la \defref{phm2pt},
    l'existence de cette action dans $\PH$ implique celle d'une transition $h \in \PTt$
    dans $\phmtopt$, ainsi que des arcs suivants :
    $\PTPre_h \subset \PTPre$, $\PTPost_h \subset \PTPost$,
    $\PTLect_h \subset \PTLect$ et $\PTInh_h \subset \PTInh$.
    Or, étant donné que $h$ est jouable dans $s$, on a :
    $\forall a_i \in \frappeur{h}, a_i \in s$, d'où :
    $\forall a \in \sortes{\frappeur{h}}, M^s(a) = i$.
    De plus, comme $\cible{h} \subset \frappeur{h}$, on en déduit :
    $\forall b \in \sortes{\cible{h}}, M^s(b) = j > j-k$.
    Ainsi, $h$ est jouable dans $\PT$ d'après la \defref{pt-sem}.
    De plus, d'après cette même définition,
    $\forall b \in \sortes{\bond{h}}, (M^s \play h)(b) = j + (k - j) = k$
    et $\forall a \in \PTp \setminus \sortes{\bond{h}}, (M^s \play h)(a) = M^s(a)$.
    Ainsi, $(M^s \play h) = M^{s \play h} = M^{s'}$.
  
  ($\Leftarrow$) Supposons que $M^s \trans{\phmtopt} M^{s'}$,
    c'est-à-dire qu'il existe une transition $h \in \PTt$ telle que $M^{s'} = M^s \play h$.
    D'après la \defref{phm2pt}, il existe alors
    une action $h = \PHfrappemult{A}{B} \in \PHh$ dans $\PH$,
    ainsi que les ensembles d'arcs suivants dans $\phmtopt$ :
    $\PTPre_h \subset \PTPre$, $\PTPost_h \subset \PTPost$,
    $\PTLect_h \subset \PTLect$ et $\PTInh_h \subset \PTInh$.
    Comme $h$ est jouable dans $M$, cela signifie alors, d'après la \defref{pt-sem}, que :
%     \[\forall p \in \PTp, M^s(p) \geq \PTPre(p, t) \wedge M^s(p) \geq \PTLect(p, t)
%       \wedge M^s(p) < \PTInh(p, t)\]
    $\forall p \in \frappeur{h}, M^s(p) = \PHget{s}{p}$
    (ainsi que : $\forall p \in \cible{h}, M^s(p) = \PHget{s}{p} \geq \PTPre(p, t)$).
    Ainsi, $h$ est jouable dans $s$ car $a \subseteq s$.
    Par ailleurs, $s \play h = s \recouvre B$ ;
    or, $\forall b \in \sortes{\bond{h}}, (M^s \play h)(b) = j + (k - j) = k$
    et $\forall a \in \PTp \setminus \sortes{\bond{h}}, (M^s \play h)(a) = M^s(a)$.
    Ainsi, $s \play h = s'$.
\end{proof}

\todo{Exemple}


% Traduction de modèles booléens Biocham
% Traduction de modèles booléens Biocham

\section{Traduction depuis la sémantique booléenne Biocham}
\seclabel{trad-biocham}

Nous proposons dans cette section une traduction des modèles Biocham dans la sémantique booléenne
en Frappes de Processus avec actions plurielles.
Biocham \citeaffixed{fages2004modelling}{pour \textit{Biochemical Abstract Machine}, voir}
est un environnement logiciel pour la modélisation et l'analyse de systèmes biochimiques.
Il propose notamment une syntaxe basée sur des règles de réactions pour représenter
un système de réactions biochimiques,
et un simulateur booléen permettant d'exécuter le système.
Ce simulateur possède la particularité d'interpréter le modèle comme étant un modèle booléen,
où les composants peuvent être présents ou absents,
sans prendre en compte les coefficients stœchiométriques ou les paramètres cinétiques
éventuels contenus dans les équations.

Notre traduction s'appuie sur les ressemblances entre le formalisme booléen de Biocham
et les Frappes de Processus avec actions plurielles, comme discutées au début de la
\secref{phm}.
Pour chaque équation de Biocham, cette traduction fait correspondre
un ensemble d'actions qui reproduit toutes les dynamiques (non-déterministes) possibles.
Nous montrons enfin que cette traduction produit effectivement un modèle équivalent,
ce qui montre que les Frappes de Processus sont au moins aussi expressives
que la sémantique booléenne de Biocham.

\myskip

Nous définissons dans la suite la notion de système d'équations biochimiques (\defref{bc})
et sa dynamique dans la sémantique booléenne de Biocham (\defref{bc-sem}).
Une équation biochimique est un triplet de la forme $X \xrightarrow{Y} Z$
où $X$, $Y$ et $Z$ sont des ensembles de composants ayant respectivement le rôle
de réactifs, catalyseurs et produits.
La signification intuitive équation est la suivante :
si, dans un état donné du système, tous les réactifs et tous les catalyseurs sont présents,
alors il est possible de créer tous les produits et de consommer une partie des catalyseurs.
Il est à noter que nous cherchons à garder l'expressivité la plus large de cette sémantique,
qui autorise de jouer une équation même si certains produits (dans $Z$) sont déjà présents
dans l'état considéré.
Par ailleurs, seule une partie des réactifs (dans $X$) est consommée, et ce de façon
non-déterministe, afin de conserver l'ensemble des dynamiques possibles d'un modèle non booléen
(et comportant donc des coefficients stœchiométriques).

% Nous proposons ensuite une traduction vers les Frappes de Processus avec actions plurielles
% (\defref{bc2phm})
% et nous montrons qu'elle possède la même dynamique (\thmref{bisimulationbc2phm}).
% Cette traduction fit correspondre, à chaque équation du système,
% un ensemble d'actions qui reproduit toutes les dynamiques non-déterministes possibles.

\begin{definition}[Système d'équations biochimiques]
\deflabel{bc}
  Un \emph{système d'équations biochimiques} tel que décrit par Biocham
  est un ensemble de réactions :
  \[\BCe = \{ X \xrightarrow{Y} Z \mid X \cap Y = Y \cap Z = X \cap Z = \emptyset \}\]
%   est un couple $(\BCc; \BCe)$ où :
%   \begin{itemize}
%     \item $\BCc$ est l'ensemble de composants booléens,
%     \item $\BCe = \{ X \xrightarrow{Y} Z \mid X, Y, Z \subset \BCc \wedge
%       X \cap Y = Y \cap Z = \emptyset \}$ est l'ensemble des équations biochimiques.
%   \end{itemize}
  Par ailleurs, on note $\BCc[\BCe]$
  l'ensemble de tous les composants mentionnés dans $\BCe$ :
  \[\BCc = \bigcup_{X \xrightarrow{Y} Z \in \BCe} X \cup Y \cup Z\]
  
  Pour toute équation biochimique $X \xrightarrow{Y} Z \in \BCe$,
  les éléments de $X$ sont appelés les \emph{réactifs}, ceux de $Y$ sont les \emph{catalyseurs}
  et ceux de $Z$ sont les \emph{produits}.
  
  Un \emph{état} d'un système d'équations biochimiques est un ensemble $S \subset \BCc$.
\end{definition}

\begin{remark}
\label{bc-nondeterminisme}
  La syntaxe véritable de Biocham permet d'intégrer des catalyseurs implicites,
  en relâchant la contrainte $X \cap Z = \emptyset$ ;
  autrement dit, il est possible d'avoir $X \cap Z \neq \emptyset$.
  Dans ce cas, les éléments de $X \cap Z$ sont aussi des catalyseurs,
  car ils conditionnent le jeu de la réaction mais ne sont pas modifiés par celle-ci.
  On peut réécrire de telles équations de la forme suivante,
  tout en assurant la même dynamique :
  $(X \setminus Z) \xrightarrow{Y \cup (X \cap Z)} (Z \setminus X)$.
  Par ailleurs, il est naturellement possible de représenter une réaction d'équilibre
  $X \overset{Y}{\longleftrightarrow} Z$ par deux réactions biochimiques
  $X \xrightarrow{Y} Z$ et $Z \xrightarrow{Y} X$.
\end{remark}

\begin{definition}[Sémantique booléenne d'un système d'équations biochimiques]
\deflabel{bc-sem}
  Soit $\BCe$ un système d'équations biochimiques
  et $S \subset \BCc$ un état de ce système.
  On note : $S \BCtrans S'$ si et seulement si :
  \[S \neq S' \wedge
    \exists X \xrightarrow{Y} Z \in \BCe,
    X \cup Y \subset S \wedge
    \exists X' \subset X, S' = (S \setminus X') \cup Z\]
%   Soient $\BCe$ un système d'équations biochimiques
%   et $S \subset \BCc$ un état de ce système.
%   Une équation $e = X \xrightarrow{Y} Z \in \BCe$ est dite \emph{jouable} dans $S$
%   si et seulement si $X \cup Y \subset S$.
%   On note alors $S \play e$ l'état résultant du jeu de $e$ dans $S$, défini par :
%   $S \play e = (S \setminus X) \cup Z$.
%   On notre par ailleurs : $S \BCtrans (S \play e)$.
\end{definition}

\begin{remark}
  Comme expliqué plus haut,
  cette définition prend compte de la sémantique non-déterministe de Biocham pour ce qui est
  de la consommation des réactifs.
  Il est possible de rendre cette définition déterministe (du point de vue de chaque réaction)
  en imposant $X' = X$ pour toutes les transitions.
  De même, cette définition ne prend pas en compte la présence éventuelle de produits
  dans le milieu avant de jouer une équation ;
  autrement dit, certains des produits d'une équation biochimique
  peuvent se trouver dans le milieu au moment où elle est jouée.
\end{remark}

La \defref{bc2phm} propose une traduction des systèmes d'équations biochimiques
en Frappes de Processus avec actions plurielles.
Le modèle obtenu comporte autant de sortes booléennes que de composants mentionnés ;
autrement dit, pour chaque composant $a$ mentionné, il existe une sorte avec deux processus
$a_0$ et $a_1$ dans le modèle obtenu.
De plus, pour chaque équation biochimique $X \xrightarrow{Y} Z$,
un ensemble d'actions est créé, chaque action étant de la forme :
$\PHfrappemult{(X_1 \cup Y_1 \cup Z'_0 \cup (Z \setminus Z')_1)}{(X'_0 \cup Z'_1)}$,
où $X$ et $Y$ sont les réactifs et catalyseurs nécessaires à l'initiation de la réaction,
$Z'$ représente les produits qui seront effectivement créés
(et $Z \setminus Z'$ représente ceux qui sont déjà présents) et
$X'$ représente le sous-ensemble des réactifs qui seront consommés
(défini de façon totalement non-déterministe).
Enfin, le \thmref{bisimulationbc2phm} stipule que le modèle obtenu
est fortement bisimilaire ---~et possède donc strictement la même dynamique.

\begin{definition}[Frappes de Processus équivalentes ($\bctophmsymbol$)]
\deflabel{bc2phm}
  Soit $\BCe$ un système d'équations biochimiques.
  On note $\bctophm = (\PHs, \PHl; \PHh)$ les Frappes de Processus avec actions plurielles
  équivalentes, définies par :
  \begin{itemize}
    \item $\PHs = \BCc$,
    \item $\PHl = \bigtimes{a \in \PHs} \PHl_a$, où $\forall a \in \PHs, \PHl_a = \{ a_0, a_1 \}$,
%    \item $\PHh = \{ \PHfrappemult{(X_1 \cup Y_1 \cup Z_0)}{(X_0 \cup Z_1)} \mid
    \item $\PHh = \{
      \PHfrappemult{(X_1 \cup Y_1 \cup Z'_0 \cup (Z \setminus Z')_1)}{(X'_0 \cup Z'_1)} \mid
      X \xrightarrow{Y} Z \in \BCe \wedge X' \subset X \wedge Z' \subset Z \wedge
      (X' \cup Z') \neq \emptyset \}$,
%     
%     \item $\PHh = \{ \PHfrappemult{A}{B} \mid \exists X \xrightarrow{Y} Z \in \BCe,
%       A = \{ a_1 \mid a \in X \cup Y \} \cup \{ a_0 \mid a \in Z \} \wedge
%       B = \{ a_0 \mid a \in X \} \cup \{ a_1 \mid a \in Z \} \}$.
%     
%     \item $\PHh = \{ \PHfrappemults{a_1 \mid a \in X \cup Y}{a_0}
%       \mid A = \{  \} X \cup Y \wedge B =  \}$
%     \item $\PHh = \{ \PHfrappemult{A}{B} \mid A = \{  \} X \cup Y \wedge B =  \}$
  \end{itemize}
  où, pour tout $i \in \{ 0, 1 \}$ et tout $N \subset \BCc$, on note :
  $N_i = \{ a_i \mid a \in N \}$.
%   où, pour toute équation $X \xrightarrow{Y} Z \in \BCe$, on note
%   $N_i = \{ a_i \mid a \in N \}$
%   avec $i \in \{ 0, 1 \}$ et $N \in \{ X, Y, Z \}$.
  
  Enfin, pour tout état $S \subset \BCc$, on note
  $\tophm{S} = S_1 \cup (\BCc \setminus S)_0$
  %\etat{a_1 \mid a \in S} \cup \etat{a_0 \mid a \notin S}$
  l'état correspondant dans $\bctophm$.
\end{definition}

\begin{remark}
  Pour supprimer le non-déterminisme de Biocham,
  tel qu'expliqué à la remarque \vpageref{bc-nondeterminisme},
  il est possible de modifier la \defref{bc2phm}
  afin d'imposer pour chaque action : $X' = X$,
  ou de faire les simplifications correspondantes.
\end{remark}

\begin{theorem}[$\PH \approx \bctophm$]
\thmlabel{bisimulationbc2phm}
  Soit $\BCe$ un système d'équations biochimiques.
  On a :
  \[\forall S, S' \subset \BCc, S \BCtrans S' \Longleftrightarrow
    \tophm{S} \trans{\bctophm} \tophm{S'} \enspace.\]
\end{theorem}

\begin{proof}
  Posons : $\bctophm = (\PHs, \PHl; \PHh)$.
  Soient $S, S' \subset \BCc$.
  
  ($\Rightarrow$) Supposons que $S \BCtrans S'$.
    D'après la \defref{bc-sem},
    cela signifie, outre $S \neq S'$,
    qu'il existe une équation $X \xrightarrow{Y} Z \in \BCe$
    telle que $X \cup Y \subset S$,
    et qu'il existe un ensemble $X' \subset X$ tel que $S' = (S \setminus X') \cup Z$.
    Posons $Z' = Z \cap S$
    et $h = \PHfrappemult{(X_1 \cup Y_1 \cup Z'_0 \cup (Z \setminus Z')_1)}{(X'_0 \cup Z'_1)}$.
    Comme $S \neq S'$, on note que $X' \neq \emptyset \vee Z' \neq \emptyset$.
    D'après la \defref{bc2phm}, on a donc : $h \in \PHh$.
    Par ailleurs, on a alors : $X_1 \cup Y_1 \subseteq \tophm{S}$,
    et par définition de $Z'$, $Z'_0 \subseteq \tophm{S}$
    et $(Z \setminus Z')_1 \subseteq \tophm{S}$,
    donc $h$ est jouable dans $\tophm{S}$.
%     Enfin, $\tophm{S} \play h = \tophm{S} \recouvre (X'_0 \cup Z'_1) =
%       \tophm{(S \setminus X') \cup Z'}$.
    Enfin, $\tophm{S'} = \tophm{(S \setminus X') \cup Z} = \tophm{(S \setminus X') \cup Z'} =
      \tophm{S \setminus X'} \recouvre Z'_1 = (\tophm{S} \recouvre Z'_1) \recouvre X'_0 =
      \tophm{S} \recouvre (Z'_1 \cup X'_0)$
    car $X$, $Y$ et $Z$ sont disjoints, et $X'$ et $Z'$ aussi.
  
  ($\Leftarrow$) Supposons que $\tophm{S} \trans{\bctophm} \tophm{S'}$.
    Cela signifie qu'il existe une action $h \in \PHh$
    telle que $\tophm{S'} = \tophm{S} \play h$.
    Par ailleurs, l'existence d'une telle action implique, d'après la \defref{bc2phm},
    l'existence d'une équation $X \xrightarrow{Y} Z \in \BCe$ telle que :
    $h = \PHfrappemult{(X_1 \cup Y_1 \cup Z'_0 \cup (Z \setminus Z')_1)}{(X'_0 \cup Z'_1)}$,
    avec $X' \subset X$, $Z' \subset Z$ et $(X' \cup Z') \neq \emptyset$.
    Comme $h$ est jouable dans $\tophm{S}$, cela signifie notamment que
    $X_1 \subseteq \tophm{S}$ et $Y_1 \subseteq \tophm{S}$
    car $X$, $Y$ et $Z$ sont disjoints ; % (et $Z'$ et $(Z \setminus Z')$ aussi).
    autrement dit : $X \subset S$ et $Y \subset S$.
    Enfin, avec le même raisonnement que ci-dessus, on obtient :
    $\tophm{S'} = \tophm{S} \recouvre (Z'_1 \cup X'_0) = \tophm{(S \setminus X') \cup Z}$,
    d'où : $S' = (S \setminus X') \cup Z$.
\end{proof}

%\todo{Exemple}




% \section{$\pi$-calcul ?}
% 
% \section{Automates finis communicants ?}


\section{Bilan}
Nous avons présenté dans ce chapitre plusieurs traductions :
\begin{itemize}
  \item depuis les Frappes de Processus,
    vers le modèle de Thomas (\secref{trad-thomas})
    et les réseaux de Petri (\secref{trad-petri}) ;
  \item vers les Frappes de Processus,
    depuis les réseaux discrets asynchrones (\secref{trad-rda})
    et les systèmes d'équations biochimiques de la sémantique booléenne de Biocham
    (\secref{trad-biocham}).
\end{itemize}
Par ailleurs, nous avons donné une preuve d'équivalence entre les Frappes de Processus
et les automates synchronisés (\secref{phm2an}) avec les traductions adéquates.
Ces différentes traductions sont résumées au sein de la \vfigref{contrib-equivalences}.

Nous insistons ici sur le fait que si chacune des traductions exposées ici
s'effectue depuis ou vers un formalisme particulier des Frappes de Processus,
il est néanmoins possible de naviguer entre les différentes sémantiques de Frappes
de Processus sans pertes d'expressivité, comme l'ont montré
les différents résultats du \chapref{sem}.
Ainsi, les résultats particuliers proposés dans le présent chapitre
se généralisent à toutes les sémantiques de Frappes de Processus,
ce qui permet non seulement d'élargir le cadre de ces traductions
à ces différentes sémantiques,
mais aussi de conclure quant aux rapports d'expressivité entre les Frappes de Processus
et les autres types de modélisation abordés.
Il est à noter qu'il n'est pas impropre de parler ici de «~Frappes de Processus~»,
sans préciser la sémantique, lorsqu'il est question d'expressivité,
car les résultats du \chapref{sem} ont bien montré leur équivalence au niveau
des capacités de représentation ---~à l'exception notable des Frappes de Processus standards
qui ont probablement moins d'expressivité que les autres sémantiques.


% Applications
% Applications

\chapter{Applications sur des exemples de grande taille}
\chaplabel{applications}

\chapeau{
  Nous proposons dans ce chapitre d'appliquer les différents résultats
  vus au cours de cette thèse
  %présentées aux \chapref{as} et \chapref{expressivite}
  à des modèles de grande taille.
  Les deux principales applications sont le suivantes :
  \begin{itemize}
    \item la dynamique d'un modèle de récepteur de lymphocytes T de 94 composants
      est étudiée à l'aide des méthodes d'analyse statique développées
      au \chapref{as},
    \item la traduction en modèle de Thomas
      d'un modèle de récepteur des cellules de croissance épithéliale de 104 composants 
      représenté en Frappes de Processus canoniques
      à l'aide de la traduction adéquate définie au \chapref{expressivite}.
  \end{itemize}
}

Nous avons présenté au \chapref{phcanonique} de cette thèse le formalisme de Frappes de Processus
canoniques qui permet d'unifier les autres formalismes de Frappes de Processus
à l'aide d'une représentation stricte.
Nous proposons ensuite dans ce même chapitre des méthodes que nous qualifions d'efficaces pour
analyser la dynamique de cette classe de modèles % sous l'aspect d'accessibilités locales,
et au \chapref{expressivite} nous donnons notamment une traduction vers le modèle de Thomas
de tout modèle décrit dans ce formalisme.

Afin de montrer que les méthodes développées dans le cadre de cette thèse sont efficaces,
nous proposons dans cette section d'en illustrer l'utilisation sur des modèles
de grande taille.
Nous considérons comme des modèles de grande taille les modèles comportant plusieurs
dizaines de composants
---~en somme, les modèles dont la taille ne permet actuellement pas d'en étudier
le comportement à l'aide de \textit{model checkers} classiques
qui doivent calculer l'intégralité de la dynamique.
Les exemples que nous donnons ici montrent que nos méthodes peuvent être
appliquées à des modèles de grande taille.
De plus, nous détaillons la nature et l'utilisation de l'implémentation
des différentes méthodes mises en pratique dans la suite.
Ces implémentations sont intégrées à la bibliothèque \Pint%
\footnote{\Pint{} est librement disponible à \url{http://loicpauleve.name/pint/}
accompagné de tous les modèles mentionnés dans ce chapitre.}
précédemment existante, principalement développée et maintenue par Loïc Paulevé,
et qui regroupe plusieurs utilitaires et outils propres aux Frappes de Processus.

\myskip

Nous détaillons à la \secref{appli-as-tcrsig} l'application de la méthode
d'analyse statique présentée
au \chapref{as} à un modèle du récepteur de lymphocyte~T comprenant 94 composants.
Nous appliquons de plus à la \secref{appli-thomas-egfr} notre méthode d'inférence
du modèle de Thomas à partir de Frappes de Processus canoniques proposée \vsecref{trad-thomas} ;
nous nous intéressons dans le détail à plusieurs modèles
de récepteur de facteur de croissance épidermique contenant 20 composants,
puis plus généralement aux résultats fournis sur différents modèles.



\section{Analyse statique du récepteur de lymphocyte~T}
\seclabel{appli-as-tcrsig}

Nous souhaitons démontrer dans cette section la puissance et l'adaptabilité
de l'analyse statique par sous-approximation développée à la \secref{as}.
Nous l'appliquons pour cela à un réseau booléen asynchrone
comportant 94 composants, ce qui peut être considéré comme un grand modèle,
et modélisant le récepteur de lymphocyte~T (\textit{T-cell receptor}).
Ce modèle est traduit en Frappes de Processus standards
et nous montrons que notre méthode reste efficace malgré la taille du modèle
et nous observons de plus que toutes les analyses effectuées sont conclusives.

\myskip

Nous nous intéressons au modèle de récepteur de lymphocyte~T
proposé par \citeasnoun{SaezRodriguez2007},
qui se présente sous la forme d'un réseau booléen asynchrone comportant 94 composants.
On y distingue notamment des composants d'entrée, c'est-à-dire qui ne sont régulés
par aucun autre composant, et dont la valeur de départ est donc fixe,
et des composants de sortie qui à l'inverse ne régulent aucun autre composant.
Dans l'état initial, les composants d'entrée sont généralement
indépendamment choisis actifs ou inactifs selon ce qu'on souhaite observer,
et tous les autres composants le sont en fonction des conditions initiales habituelles
observées expérimentalement ---~la majorité étant inactifs.
Cela permet d'observer la propagation du signal d'entrée caractérisée par une (dés)activation
successive des autres composants du modèle.

Le but de cette étude est de tester la possibilité d'activer chaque composant de sortie
en fonction de l'état initial (activé ou non) de chaque composant d'entrée.
Cela permet notamment de vérifier que le modèle est fonctionnel,
autrement dit de vérifier que
tous les composants de sortie peuvent être activés si toutes les entrées le sont
---~et qu'elles ne peuvent pas l'être si aucune entrée c'est active.
De plus, tous les autres comportements intermédiaires peuvent être analysés
afin de prévoir le comportement du système réel dans ces situations.

Le modèle comporte trois composants d'entrée (appelés CD4, CD28 et TCRlig)
et quatre composants de sortie (SRE, AP1, NFkB et NFAT).
Il a été traduit automatiquement en Frappes de Processus canoniques%
\footnote{tous les fichiers sont disponibles à
\url{http://maxime.folschette.name/underapprox-tcrsig94.zip}.}
depuis le format CellNetAnalyzer \cite{klamt2007structural}.
%à l'aide d'un programme basé sur \storef.
Les états initiaux sont construits en définissant tous les composants
comme étant inactifs sauf pour certains d'entre eux
(à savoir : CD45, CARD11, Bcl10, Malt1, Rac1r, Lckr, cCblr et IkB)
et pour les composants d'entrée choisis comme initialement actifs.
Pour chacun des états 8 initiaux ainsi obtenus,
nous testons l'atteignabilité des processus $x_1$ de chaque composant $x$ de sortie
indépendamment à l'aide des deux analyses statiques suivantes :
\begin{itemize}
  \item La sous-approximation développée à la \vsecref{as},
    et notamment le \thmref{approxinf},
    qui permet de répondre «~Oui~» ou «~Non-conclusif~» ;
  \item La sur-approximation proposée par \citeasnoun{PMR12-MSCS},
    appliquée au modèle fusionné d'après la \vdefref{fusion},
    qui permet de répondre «~Non~» ou «~Non-conclusif~».
\end{itemize}
L'utilisation de ces deux approches conjointement permet d'obtenir un résultat dans un
plus grand nombre de cas,
l'analyse par sous-approximation présentée à la \vsecref{as} de cette thèse
ne permettant pas de répondre «~Non~».

Il s'avère finalement que cette méthode est conclusive pour tous les cas testés ;
autrement dit, les deux méthodes ne répondent jamais «~Non-conclusif~»
pour une même question d'atteignabilité.
Sur les 32 cas de figure testés,
nous avons notamment pu répondre «~Oui~» dans 12 cas
grâce au \vthmref{approxinf}.
Lorsque tous les composants d'entrée sont inactifs, aucun composant de sortie ne peut
être activé ;
à l'inverse, lorsque tous les composants d'entrée sont actifs,
SRE, AP1 et NFAT peuvent être activés mais pas NFkB,
ce qui permet de corriger le modèle de façon à la rendre actif en ajoutant une action manquante.
Les temps de calcul sont de l'ordre de la seconde sur un ordinateur
% quelques centièmes de seconde
de bureau classique\footnote{Testé sur une machine comportant
un processeur de 2,4~Ghz avec 2~Go de mémoire vivre.},
ce qui permet d'effectuer un grand nombre de tests sur un même modèle.
À titre de comparaison, les mêmes analyses ont été effectuées 
sur une traduction en réseau de Petri du modèle étudié,
à l'aide du \textit{model checker} symbolique libDDD \cite{Kordon09libddd},
connu pour ses bonnes performances.
Cependant, du fait de la taille du modèle, le programme effectue un dépassement de mémoire
pour chacun des cas testés.

\myskip

Pour terminer, nous testons en complément l'atteignabilité simultanée
de l'activation des quatre composants de sortie (SRE, AP1, NFkB et NFAT).
Ce résultat est utile car les propriétés précédemment étudiées consistaient uniquement
en des atteignabilités indépendantes, et il demeure toujours la possibilité
que l'une des atteignabilité n'en entrave une autre.
Pour ce résultat supplémentaire, nous modifions le modèle comme expliqué à la
\vsecref{as-etat}, qui consiste à ajouter les éléments suivants :
\begin{itemize}
  \item une sorte coopérative $\uastotal$ entre les quatre composants considérés,
  \item toutes les actions primaires nécessaires à la mise à jour de $\uastotal$,
  \item une sorte $\uasreach$ comprenant deux processus ($\uasreach_0$ et $\uasreach_1$),
  \item une action $\PHfrappe{\uastotal_{1111}}{\uasreach_0}{\uasreach_1}$ de priorité basse.
\end{itemize}
Nous pouvons alors calculer l'atteignabilité de $\uasreach_1$, qui permettra de déterminer
si les quatre composants peuvent être simultanément actifs depuis l'état initial.

L'implémentation répond alors «~Oui~», ce qui permet d'affirmer que les quatre composants
peuvent être activés simultanément,
ce qui permet de conclure définitivement que le modèle est fonctionnel.
La réponse a été obtenue en \todo{48 secondes}.
Cet écart de temps part rapport aux résultats de la section précédente est causé
par deux facteurs :
\begin{itemize}
  \item L'intégralité du modèle doit être représenté par le graphe de causalité locale,
    contrairement aux atteignabilités précédentes qui n'en représentaient qu'une portion ;
  \item L'implémentation cherche à appliquer le \thmref{approxinf} de sous-approximation sur
    différents sous-ensembles de solutions du graphe, or l'énumération de ces sous-ensembles
    est exponentielle dans le nombre de solutions du graphe de causalité locale.
\end{itemize}
Malgré cela, notre méthode reste conclusive en un temps inférieur à la minute,
là où des implémentations jugées efficaces mais nécessitant le calcul
exhaustif de la dynamique ne terminent pas, comme nous l'avons montré plus haut.

\myskip

Nous avons démontré avec cet exemple la puissance de l'analyse statique que nous proposons
à la \vsecref{as}, applicable aux réseaux booléens asynchrones.
Bien que la dynamique soit approximée, tous les cas étudiés dans cet exemple restent conclusifs,
ce qui permet d'étudier la dynamique d'un grand modèle de 94 composants,
chose impossible avec les outils de \textit{model checking} classiques.

\subsubsection*{Implémentation}

L'analyse statique développée à la \vsecref{as}
a été implémentée par Loïc Paulevé, sur la base de celle existante
portant sur les Frappes de Processus standards.
L'outil en question s'appelle \texttt{ph-reach}, et il est nécessaire de spécifier l'option
\texttt{-{}-coop-priority}
pour que les actions de mise à jour des sortes coopératives soient priorisées,
et que le modèle soit interprété comme des Frappes de Processus canoniques.
Voici un exemple de ligne de commande :
\cl{ph-reach -{}-coop-priority -i tcrsig94.ph\\
  \clspace\clspace-{}-initial-state "cd4 1, cd45 1, cd28 1, tcrlig 1" ap1 1}
Cette ligne de commande utilise le modèle du fichier \texttt{tcrsig94.ph} ;
elle définit l'état initial du modèle comme étant
$s \recouvre \{ \text{CD4}_1, \text{CD45}_1, \text{CD28}_1, \text{TCRlig}_1 \}$
(où $s$ est défini préalablement dans le fichier \texttt{tcrsig94.ph})
et calcule l'atteignabilité de $\text{AP1}_1$ à partir de cet état.
Après exécution, l'outil retourne \texttt{True}, ce qui signifie que l'atteignabilité
testée à l'aide de la méthode de la \vsecref{as} est possible.
Les autres réponses sont \texttt{False},
qui est calculée d'après les travaux précédents de \citeasnoun{PMR12-MSCS},
et \texttt{Inconc}.



\section{Traduction vers le modèle de Thomas}
\seclabel{appli-thomas}


\section{Application au récepteur de facteur de croissance épidermique}
\seclabel{appli-thomas-egfr}

Nous nous intéressons ici à l'étude d'un modèle de récepteur de facteur de croissant épidermique
\cite{Sahin09}.
Ce modèle est représenté par un graphe des interactions contenant 20 composants et 52 arcs.
Une protéine appelée EGF peut être considérée comme un composant d'entrée
car elle ne comporte aucun régulateur,
et une chaîne e réactions permet l'activation de la protéine pRB qui est responsable
de la régulation de la division cellulaire.
Celle-ci est donc essentielle pour la prévention du développement de cancers.

Trois modèles de Frappes de Processus canoniques
sont créés à partir du graphe des interactions original, avec différents niveaux
de précision dans la modélisation des coopérations entre composants.
\begin{itemize}
  \item Le modèle (1) représente la dynamique généralisée du graphe des interactions,
    telle que théorisée par \citeasnoun{PMR10-TCSB},
    c'est-à-dire sans aucune information concernant les règles booléennes des coopérations,
    et ne contient donc aucune sorte coopérative (ni aucune action primaire) ;
  \item Le modèle (2) est partiellement raffiné : il intègre quelques-unes des règles
    booléennes en question en tant que sortes coopératives,
    choisies d'après des expériences de \textit{knockdown} menées sur le système ;
  \item Le modèle (3) est le modèle totalement raffiné contenant toutes les coopérations.
\end{itemize}
Les modèles (2) et (3) consistent donc en des raffinements successifs du modèle (1)
qui est le plus général au niveau de la dynamique.
La constitution de chaque modèle est détaillée dans la suite,
et les résultats de l'inférence du graphe des interactions et de l'inférence des paramètres
sur ces trois modèles sont donnés dans la \tabref{appli-thomas-egfr}
et discutés plus bas.
Nous ne nous intéressons ici qu'à des paramètres entiers,
autrement dit à des segments réduits à une seule valeur.

\begin{table}[ht]
  \ZifferAn
  \begin{center}
  \begin{tabular}{r|l|l|l|l|l} %m{2cm}|m{2.5cm}|m{1.5cm}}
    \textbf{Modèle} & $\mathbf{|E|}$ & $\mathbf{|K|}$ & \textbf{Paramètres inférés} &
      \textbf{Modèles possibles} & \textbf{Points fixes}
  \\\hline\hline
    (1) & $52$ & $196$ & $20$ & $2^{176}\simeq 9,6\cdot10^{52}$ & $0$   % v1_0.ph
  \\\hline
    (2) & $51$ & $192$ & $98$ & $2^{94}\simeq 2,0\cdot10^{28}$ & $0$    % v2_1.ph
  \\\hline
    (3) & $51$ & $192$ & $192$ & $1$ & $3$                              % ori.ph
  \\\hline
  \end{tabular}
  \end{center}
  \caption{\tablabel{appli-thomas-egfr}%
    Résultats de l'inférence du graphe des interactions et des paramètres
    sur les trois modèles dérivés du récepteur EGF %\cite{Sahin09}
    avec différentes précisions dans la définition des coopérations.
    Le modèle (1) ne contient aucune coopération entre les composants.
    Certaines coopérations ont été incluses dans le modèle (2) sous la forme de 14
    sortes coopératives et le modèle (3) contient toutes les coopérations entre composants
    sous la forme de 22 sortes coopératives.
    La deuxième colonne donne le nombre d'arcs dans le graphe des interactions inféré
    à l'aide de \storef (le nombre de nœuds étant toujours celui du modèle, c'est-à-dire 20).
    La troisième colonne donne le nombre total de paramètres à définir
    (calculé à partir du graphe des interactions),
    la quatrième colonne donne le nombre de paramètres entiers qui ont pu être inférés
    en utilisant \storef,
    et la cinquième colonne donne le nombre de modèles compatibles avec le modèle
    de Frappes de Processus canoniques étudié,
    qui dépend de façon exponentielle du nombre de paramètres n'ayant pu être inférés.
    Finalement, la dernière colonne donne le nombre de points fixes du modèle,
    calculé à l'aide du \vthmref{php-pf}.
  }
  \ZifferAus
\end{table}

\todoplustard{GI visuel ?}

Le modèle (1) comprend uniquement des interactions individuelles entre composants,
c'est-à-dire des activations ou inhibitions indépendantes d'un composant à l'autre,
obtenues d'après les régulations contenues dans le graphe des interactions.
Ainsi, le graphe des interactions inféré d'après le modèle (1) est exactement identique
au graphe des interactions utilisé pour créer ce modèle,
à l'exception d'une auto-activation sur l'entrée EGF,
qui est due à son absence de régulateurs,
comme expliqué à la \vsecref{trad-thomas}.
Les seuls paramètres ayant pu être inférés sont ceux qui concernent les configurations extrêmes
des ressources de chaque régulation,
à savoir dans le cas où tous les activateurs sont présents et tous les inhibiteurs absents,
et dans le cas inverse.
Ce premier modèle abstrait donc un grand nombre de modèle de Thomas
(de l'ordre de $9\cdot10^{52}$)
étant donné que la quasi-totalité des paramètres restent indéterminés.

Afin de construire le modèle (2), 14 sortes coopératives ont été ajoutées dans le but de
modéliser les fonctions booléennes de plusieurs composants (consistant en des portes ET et OU).
Pour ce faire, les composants suivants ont été retenus du fait de leur importance dans la
chaîne de réactions : CDK4, CDK6, CycD1, ER\nbd \textalpha{} and c\nbd MYC.
En effet, d'après les expériences par \textit{knockdown} menées par \citeasnoun{Sahin09}
empêcher ces composants de s'exprimer menait à une réduction significative de la production
de pRB.
On peut en conclure que ces composants sont impliqués dans les fonctions booléennes
d'autres composants d'une façon à ce que le \textit{knockdown} des premiers
empêche la production des seconds (ce qui est typique des portes de type ET).
Afin de reproduire ces requis, les fonctions booléennes des successeurs de ces composants
ont été modélisées sous la forme de sortes coopératives le cas échéant, c'est-à-dire celles de
CDK4, CDK6, prB, p21, p27, IGF1R, MYC, CycD1 et CycE1.
En théorie, 9 sortes coopératives auraient suffi, mais la factorisation des sortes coopératives
décrite \vpageref{factorisation-coop} a été utilisée afin de réduire la taille de celles-ci
---~en nombre de processus~--- lorsque c'était possible.
Les sortes coopératives ainsi ajoutées ont permis d'inférer environ la moitié des paramètres ;
cependant, le nombre de modèles de Thomas compatibles avec ce modèle reste très élevé
du fait des nombreux paramètres qui restent indéterminés.
De plus, on note que le graphe des interactions inféré comporte un arc de moins que
le graphe des interactions d'origine.
Cela est du au fait que la fonction booléenne de pRB intégrée sous la forme
d'une sorte coopérative pouvait en fait être simplifiée
d'une façon telle que le composant CDK2 n'y apparaissait plus.
Aucun arc n'a alors été inféré entre CDK2 et pRB par notre méthode.

Enfin, le modèle (3) a été construit en utilisant toutes les fonctions booléennes
fournies par \citeasnoun{Sahin09}.
Ces fonctions prennent la forme de 22 sortes coopératives qui permettent de retrouver
le comportement attendu du système.
Toutes ces coopérations étant correctement définies dans le modèle,
tous les paramètres peuvent être inférés et un seul modèle de Thomas est compatible
avec la dynamique des Frappes de Processus canoniques utilisées.
Nous notons de plus que ce modèle est le seul à contenir des points fixes,
calculés à l'aide du \vthmref{php-pf},
et qui comprennent notamment les deux états stables correspondant à une propagation
totale du signal pour les deux cas initiaux,
c'est-à-dire dans le cas où EGF est initialement actif et dans le cas où il ne l'est pas.
Les deux autres modèles ne contiennent pas de point fixe car les coopérations manquantes
créent des oscillations qui sont la conséquence du comportement non-déterministe
du formalisme.




\section{Temps d'exécution sur quelques modèles de grande taille}
\seclabel{appli-thomas-large}

L'implémentation de la méthode proposée \storef permet de traiter des modèles de
Frappes de Processus canoniques de grande taille et obtenus à partir de données
trouvées dans la littérature.
Ont notamment été traités :
\begin{itemize}
  \item les modèles du récepteur de croissance épidermique présentés
    à la \vsecref{appli-thomas-egfr} \cite{Sahin09},
    qui contiennent 20 composants et jusqu'à 22 sortes coopératives,
  \item un modèle de récepteur de lymphocyte~T \cite{Klamt06},
    contenant 40 composants et 14 sortes coopératives.
\end{itemize}
Pour chacun de ces modèles, les inférences du graphe des interactions et des paramètres
sont effectuées en moins d'une seconde sur un ordinateur de bureau
standard\footnote{Pour tous les exemples de cette section,
le logiciel a été lancé sur une machine comportant
un processeur de 3,4~Ghz avec 8~Go de mémoire vivre.}

D'autres modèles de plus grande taille portant sur les mêmes systèmes ont aussi été testés :
\begin{itemize}
  \item un modèle du récepteur de lymphocyte~T contenant 94 composants,
    déjà présenté à la section précédente \cite{SaezRodriguez2007},
  \item un modèle du récepteur de croissance avec 104 composants \cite{Samaga2009}.
\end{itemize}
Ces deux modèles ont été obtenus par \citeasnoun{PMR12-MSCS} à l'aide d'une traduction
automatique depuis le format CellNetAnalyzer \cite{klamt2007structural}
vers le Frappes de Processus canoniques.

La composition de ces modèles et les résultats de l'inférence du modèle de Thomas
cons résumés dans la \tabref{appli-thomas-large}.
Nous notons notamment que, du fait de la très grande taille des paramétrisations
des modèles de 94 et 104 composants, l'inférence des paramètres n'a pas pu être réalisée.
Ce nombre anormal de paramètres est dû à la présence d'un très grand nombre de régulateurs
pour un même composant dans ces deux modèles, faisant croître la taille de la paramétrisation
de façon exponentielle.
Il est à noter que ces régulations sont principalement dues à des contraintes de modélisation ;
notre méthode devrait néanmoins être en mesure de traiter des modèles d'aussi grande taille
à condition que chaque composant ne possède qu'un nombre limité de régulateurs.
Cela indique de plus que de tels modèles seraient plus efficacement étudiés
en tant que Frappes de Processus qu'en tant que modèles de Thomas.

\newcommand{\egfr}{5}
\newcommand{\tcrsig}{6}

\begin{table}[ht]
  \ZifferAn
  \begin{center}
  \begin{tabular}{r|c|c|c|c|c}
    \textbf{Modèle} & $\mathbf{\card{\Gamma}}$ & $\mathbf{\card{\cs}}$ &
      \textbf{Inférence GI} & \textbf{Inférence K} & $\mathbf{\card{K}}$
  \\\hline\hline
    RFCE\footnotemark[\egfr] \cite{Sahin09} &
      $20$ & $22$ & < 1 s & < 1 s & 192
  \\\hline
    RLT\footnotemark[\tcrsig] \cite{Klamt06} &
      $40$ & $14$ & < 1 s & < 1 s & 143
  \\\hline
    RLT\footnotemark[\tcrsig] \cite{SaezRodriguez2007} &
      $94$ & $39$ & 10 s & --- & $2,1\cdot10^{9}$
  \\\hline
    RFCE\footnotemark[\egfr] \cite{Samaga2009} &
      $104$ & $89$ & 3 min 20 s & --- & $4,2\cdot10^{6}$
  \end{tabular}
  \end{center}
  \caption{\tablabel{appli-thomas-large}%
    Temps d'exécution de notre implémentation et informations relatives aux modèles
    utilisés dans l'inférence du graphe des interactions et de la paramétrisation
    de quatre modèles de réseaux de régulation biologiques.
    La deuxième colonne donne le nombre de composants de chaque modèle
    et la troisième donne le nombre de sortes coopératives utilisées pour modéliser
    des coopérations entre composants.
    La quatrième (\resp cinquième) colonne donne le temps de calcul de l'inférence
    du graphe des interactions (\resp de la paramétrisation) pour chaque modèle,
    et la dernière colonne renseigne sur le nombre de paramètres que contient chaque modèle.
    Du fait d'un trop grand nombre de paramètres, l'inférence de la paramétrisation
    n'a pas pu être effectuée sur les modèles de $94$ et $104$ composants par manque de mémoire.
  }
  \ZifferAus
\end{table}

\footnotetext[\egfr]{Récepteur de facteur de croissance épidermique.}
\footnotetext[\tcrsig]{Récepteur de lymphocyte~T.}

\myskip

\subsubsection*{Implémentation}

La traduction de Frappes de Processus canoniques en modèle de Thomas développée
à la \vsecref{trad-thomas} a été implémentée et intégrée à \Pint.
L'implémentation consiste principalement en trois programmes
écrits en \textit{Answer Set Programming} (ASP),
un paradigme de programmation logique permettant de se concentrer sur la modélisation
du problème plutôt que sur sa résolution.
Ce paradigme est notamment pris en charge par le \textit{grounder-solver} \tool{Clingo},
et qui a déjà montré ses capacités dans la résolution de problèmes de complexité NP.
Les trois programmes de cette implémentation se chargent respectivement
de réaliser l'inférence
du graphe des interactions, des paramètres et de tous les modèles de Thomas compatibles.
Des compléments écrits en OCaml assurent de plus la traduction du modèle en ASP
et la récupération des résultats de \textit{solver}.
Cette traduction peut être appelée par l'outil \texttt{ph2thomas}, par exemple avec la
ligne de commande suivante :
\cl{ph2thomas -i ERBB\_G1-S.ph}
pour effectuer l'inférence des paramètres sur le modèle de Frappes de Processus canoniques
du fichier \texttt{ERBB\_G1-S.ph}.
Il est possible de plus d'exporter le graphe des interactions inféré dans un fichier
\texttt{ERBB\_G1-S.dot} à l'aide de l'option \texttt{-{}-dot ERBB\_G1-S.dot}
ou de lancer l'énumération des modèles de Thomas compatibles
avec \texttt{-{}-full-enumerate} (ou \texttt{-{}-enumerate} pour se restreindre
à des paramètres entiers).


% Conclusion
% Conclusion

\chapter{Conclusion et perspectives}
\chaplabel{conclusion}

\chapeau{
  Nous revenons dans ce chapitre sur les apports de la présente thèse,
  en insistant notamment sur la façon donc nous avons répondu aux deux problématiques
  principales proposées initialement :
  renforcer la puissance de la modélisation d'une part,
  et les capacités d'analyse d'autre part,
  notamment en ce qui concerne les modèles de grande taille.
  Plusieurs nouvelles perspectives scientifiques sont par ailleurs évoquées.
}



L'étude des réseaux de régulation biologique peut s'effectuer sous de nombreuses formes,
allant des systèmes d'équations différentielles à la programmation logique,
en passant par des modèles probabilistes.
L'utilisation de modèles discrets permet une abstraction puissante de la complexité inhérente
de ces systèmes, en conservant certaines propriétés intéressantes.
On peut considérer que l'utilisation de tels modèles a été initiée par
\citefullname{kauffman69}{Stuart A.}, plus tard suivi par \citefullname{Thomas73}{René}
qui y ajoute une dynamique asynchrone
et par \citefullname{Snoussi89}{El Houssine}
qui propose la notion de paramètres discrets pour définir totalement la dynamique.
Ces modèles discrets ont été abondamment étudiés,
comme nous l'avons vu au \chapref{etatdelart} :
beaucoup de résultats permettent d'optimiser l'étude de leur dynamique,
mais aussi de déduire la présence de certains comportements
simplement à l'aide de la structure du modèle.

Cependant, la simplicité apparente de leur définition masque deux écueils importants.
Tout d'abord, ces modèles possèdent une dynamique souvent très complexe,
l'espace des états ayant en effet une taille exponentielle dans la taille du modèle.
Les méthodes de compression ou de réduction des modèles ne permettent pas toujours
de restreindre suffisamment cet espace des états pour en permettre une étude exhaustive.
L'analyse dynamique des modèles de régulation reste donc souvent cantonnée
à des modèles de petite ou moyenne taille ---~rarement plus de quelques dizaines de composants.
De plus, certains paramètres peuvent être difficiles voire impossibles à intégrer dans le modèle.
En effet, le modèle de Thomas ne permet pas de faire figurer des notions de précédence,
de préemption ou de synchronisme entre les différentes interactions.
De même, son étude requiert d'avoir correctement défini tous les paramètres discrets
propres au système étudié, paramètres qui sont parfois inconnus à priori,
alors que les comportements attendus le sont (même partiellement).
Pour finir, il peut s'avérer difficile de raffiner un modèle lorsqu'il n'est pas
possible d'étudier l'impact de chaque modification
à cause des difficultés inhérente à l'analyse des grands modèles.

Les Frappes de Processus, introduites par \citeasnoun*{PMR10-TCSB},
offrent un outil alternatif puissant de représentation et d'analyse des
réseaux de régulation biologiques.
Le formalisme proposé, totalement asynchrone, est basé sur une représentation
atomique des interactions entre composants,
où le changement d'état d'un composant ne peut être déclenché que par l'état d'au plus
un autre composant.
Cette forme particulière a notamment permis le développement de méthodes efficaces
d'analyse de la dynamique, qui permettent notamment d'en réduire la complexité \cite{PMR12-MSCS}.
Cependant, les Frappes de Processus ne permettent pas de représenter fidèlement
un réseau de régulation biologique sous la forme d'un réseau discret asynchrone.
De plus, certaines contraintes de modélisation ne peuvent pas toujours être introduites
efficacement dans ces modèles standards de Frappes de Processus.

Le travail proposé dans cette thèse avait pour ambition de répondre à cette double problématique
de modélisation et d'analyse des grands modèles de réseaux de régulation biologiques,
en partant du formalisme Frappes de Processus précédemment proposé,
et des différents travaux correspondants.
Nous proposons un bref rappel des résultats obtenus sur ces deux points à la \secref{resultats}
et nous ouvrons un certain nombre de perspectives dans la lignée de nos travaux
à la \secref{perspectives}.



\section{Retour sur les résultats de cette thèse}
\seclabel{resultats}

Nous mettons en perspective dans la suite l'apport de nos travaux aux deux questions
que sont la modélisation et l'analyse d'un système.

\subsubsection*{Modélisation}

La représentation d'un système réel (ou fictif) sous forme de modèle est fondamentale
et doit être adaptée aux besoins de l'analyse.

\myskip

Nous avons proposé au \chapref{sem} plusieurs représentations alternatives
permettant d'enrichir la modélisation d'un système en Frappes de Processus.

À la \secref{php}, nous avons défini des classes de priorités sur les actions
d'un modèle de Frappes de Processus.
Ces classes de priorités permettent de spécifier des contraintes de préemption globales
entre plusieurs ensembles d'actions ---~chaque action ne pouvant être jouée que si
aucune action plus prioritaire ne l'est.
Nous avons montré comment ces classes de priorités permettent de représenter de façon
discrète des contraintes temporelles (éventuellement continues).
Pour cela, nous avons pris comme exemple les modèles de Frappes de Processus intégrant
des paramètres stochastiques pour chaque action,
paramètres qui peuvent se traduire en intervalles de tirs,
et nous avons montré comment modéliser une dynamique semblable grâce à des classes de priorités.

L'introduction d'arcs neutralisants à la \secref{phan} a en revanche permis de définir
des contraintes de préemption locales entre les actions.
Leur plus grande granularité permet d'intégrer plus finement certaines informations
relatives entre les actions, comme des contraintes de tir ou des durées de réaction.
Cette granularité est notamment nécessaire pour éviter des relations de préemption
non désirées qui peuvent bloquer des parties du modèle,
par exemple en abstrayant certaines phénomènes d'accumulation,
provoquant des paradoxes de Zénon.

Enfin, les actions plurielles définies à la \secref{phm} introduisent de la
synchronicité entre les actions.
Celle-ci est nécessaire pour modéliser certains phénomènes comme des réactions biochimiques,
qui nécessitent de représenter la formation de plusieurs produits ou la consommation de plusieurs
réactifs, voire les deux simultanément.
Ce formalisme offre donc à fortiori la possibilité de représenter du synchronisme au niveau des
pré-conditions de tir des actions, et par conséquent permet de compresser les modèles
à l'aide d'une notion naturelle de coopération.

\myskip

Nous avons de plus délimité au \chapref{phcanonique} une sous-classe des Frappes de Processus
avec classes de priorités, dite \emph{canonique},
qui ne comporte que deux classes de priorités et une forme particulière pour les actions
très prioritaires.
Celle classe de modèles présente un intérêt particulier pour la représentation,
car elle corrige les problèmes de décalage temporel présents
dans les coopérations modélisées par des Frappes de Processus standards.
Nous avons de plus mis en évidence le fait qu'elle est suffisante pour représenter tous
les autres formalismes de Frappes de Processus alternatifs rappelés ci-dessus ;
nous avons démontré par ailleurs grâce à cela que tous ces formalismes sont équivalents,
et nous utilisons les Frappes de Processus canoniques pour proposer
les traductions adéquates.

\myskip

Enfin, le \chapref{expressivite} a été l'occasion de comparer l'expressivité des
Frappes de Processus avec d'autres formalismes courants.

Nous avons notamment démontré à la \secref{trad-rda} que les Frappes de Processus canoniques
permettent de représenter fidèlement et naturellement les réseaux discrets asynchrones.
Cette équivalence est particulièrement utile pour pouvoir étudier ce type de modèles
efficacement à l'aide des outils détaillés plus bas.

Nous nous sommes aussi penché sur le cas du modèle de Thomas à la \secref{trad-thomas},
qui présente un intérêt supplémentaire du fait de sa dynamique unitaire
et de ses paramètres discrets.
Nous avons donné les condition sous lesquelles les Frappes de Processus canoniques
peuvent être traduites en modèle de Thomas de façon plus ou moins complète,
ce qui se manifeste par des arcs non-signés ou des paramètres impossibles à inférer.
Cependant, en cas de réponse partielle,
un ensemble de modèles compatibles peut être énuméré,
et le modèle d'origine peut être itérativement raffiné afin d'obtenir
le modèle de Thomas attendu.

% Liens avec plusieurs autres formalismes :
%   vers le modèle de Thomas :
%     formalisme historique et mieux connu, plus global (influences)
%     passage « atomique -> influences globales » difficile
%     inférence de toutes les influences et paramètres + énumération si incomplet
%     toujours compatible avec la dynamique du PH
%   depuis les réseaux discrets asynchrones :
%     équivalence du formalisme
%     permet l'étude des RDA
%   Biocham :
%     représentation possible
%   RdP :
%     formalisme répandu et analyses puissantes

\myskip

Ces différents formalismes de Frappes de Processus
peuvent directement bénéficier aux biologistes désirant
vérifier le bon fonctionnement d'un réseau discret asynchrone ou d'un modèle de Thomas,
ou effectuer des prédictions sur ces modèles.
Ils peuvent naturellement tout aussi bien être utilisés à d'autres fins,
comme l'inférence ou la correction de modèles, à l'aide des différentes équivalences proposées.
Cependant, il est aussi envisageable pour des informaticiens ou bioinformaticiens
d'en faire usage dans un cadre
plus large, à la condition de pouvoir représenter le système informatique souhaité
sous la forme de Frappes de Processus.



\subsubsection*{Analyse}

Un formalisme adapté à la représentation d'une certaine classe de systèmes présente cependant
un intérêt limité s'il n'est pas accompagné de méthodes d'analyse efficaces.

C'est pourquoi, en sus de la définition des Frappes de Processus canoniques,
nous avons proposé au \chapref{as} une méthode efficace d'analyse d'atteignabilité,
qui peut cependant aussi être appliquée aux autres formalismes
de Frappes de Processus, moyennant les traductions adéquates.
Elle s'appuie sur une sous-approximation de la dynamique du modèle étudié,
qui s'appuie sur la forme particulière des actions et de la définition
des Frappes de Processus canoniques.
Il est possible grâce à cela de répondre à des questions d'atteignabilité locale
avec une complexité polynomiale dans la taille du modèle étudié,
évitant ainsi l'explosion combinatoire inhérente au calcul exhaustif de la dynamique.
Cette analyse peut de plus se décliner sous plusieurs formes :
atteignabilité d'un état local, d'une succession d'états locaux
et même l'atteignabilité simultanée d'un ensemble d'états locaux,
ce qui était auparavant impossible.

\myskip

Enfin, le \chapref{applications} nous a permis d'illustrer l'étendue
des différents résultats proposés dans cette thèse
par quelques applications à des modèles biologiques.

Nous avons appliqué à la \secref{appli-as-tcrsig} la méthode d'analyse des
Frappes de Processus canoniques résumée ci-dessus
à un modèle de 94 composants et pouvant donc être considéré comme de grande taille.
Cette application a permis de répondre avec succès à plusieurs questions d'atteignabilité
avec des temps de calcul de l'ordre de quelques centièmes de seconde,
ce qui à notre connaissance est inégalé.

L'étude de comportements peut néanmoins aussi être abordée par d'autres formalismes.
Les différents modèles de Frappes de Processus traduits en modèle de Thomas
à la \secref{appli-thomas} ont illustré cela :
chaque modèle était issu d'un raffinement différent d'un même système,
et la traduction en modèle de Thomas a permis de déterminer
les parties de modèle dont le comportement est déterministe.
De plus, l'analyse efficace de la dynamique sur les Frappes de Processus
peut s'avérer utile pour effectuer des raffinements successifs d'un même modèle
avant de le traduire vers le formalisme souhaité.
La conservation de la dynamique %et de certaines bonnes propriétés sur la structure
pour toutes les traductions présentées
permet en effet de généraliser les résultats obtenus sur un modèle à tous les formalismes
dans lesquels il peut être traduit.

\myskip

Les possibilités offertes par l'utilisation conjointe des traductions présentées
dans cette thèse avec les méthodes d'analyse développées
rendent accessible l'étude de modèles de grande taille, voire de très grande taille.
Ces capacités d'analyse devraient notamment être suffisantes pour pouvoir analyser
en un temps raisonnable des propriétés dynamiques sur des bases de données métaboliques
comme PID \cite{schaefer09pid}
ou des regroupements de telles bases de données comme hiPathDB \cite{yu12hipathdb}.



\section{Perspectives de travail}
\seclabel{perspectives}

Les travaux présentés au cours de cette thèse ouvrent de nombreuses pistes de travail
permettant d'étendre et d'exploiter les résultats proposés.

% Application à des très grands réseaux (> 1000 composants)
% 
% Raffiner les traductions pour les grands modèles / pour prendre en compte de nouvelles sémantiques



\subsubsection*{Raffinement de l'analyse statique}

L'analyse d'atteignabilité présentée au \chapref{as}
repose sur une approximation de la dynamique ;
cette approximation fait chuter la complexité de l'analyse,
au risque cependant de ne pas pouvoir conclure dans certains cas.
Si le besoin s'en fait ressentir sur certains modèles,
il pourrait naturellement être intéressant de raffiner cette analyse tant au niveau
de la sous-approximation que nous proposons qu'au niveau
de la sur-approximation définie par \citeasnoun{PMR12-MSCS},
afin de pouvoir conclure dans un plus grand nombre de cas.
Un tel raffinement serait basé soit sur la construction du graphe de causalité locale,
soit sur les conditions nécessaires et suffisantes qui l'exploitent.
D'autres méthodes dérivées peuvent être envisagées,
permettant par exemple de guider une analyse plus exhaustive,
comme l'analyse des scénarios permis par le graphe de causalité locale.

De même, les méthodes d'analyse en question pourraient être adaptées à d'autres formalismes
que les Frappes de Processus canoniques.
Leur application aux Frappes de Processus avec actions plurielles, par exemple,
permettrait d'éviter la traduction coûteuse d'un modèle vers les Frappes de Processus canoniques.

Par ailleurs, l'analyse d'atteignabilité ne permet actuellement que de vérifier des propriétés
qui s'expriment en logiques temporelles sous la forme $\mathbf{EF}(P)$ ;
il pourrait être intéressant de l'enrichir avec d'autres propriétés comme
$\mathbf{AF}(P)$, assurant par exemple qu'un état local est toujours atteignable,
ou encore $(P\:\mathbf{U}\:Q)$, qui permettrait d'observer des commutations,
par exemple avec la formule :
$(\mathbf{EF}(P) \wedge \mathbf{EF}(Q))\:\mathbf{U}\:(\neg\mathbf{EF}(P) \wedge \mathbf{EF}(Q))$.
L'utilisation de logiques intégrant des compteurs,
ou de logiques non-régulières,
augmenterait aussi les capacités d'analyse.

% Raffinement de l'analyse statique :
%   répondre OUI sur de plus nombreux cas
%   répondre NON sur de plus nombreux cas
%   nouvelles questions (cf. logiques temporelles) tq AF ou U
%   utilisation de compteurs, de logiques non-régulières
% Moyens :
%   revoir la construction du GLC ou la condition nécessaire/suffisante
%   analyse exhaustive du GLC
%   nouvelles approches statiques (directement adaptées aux actions plurielles ?)
%   approches exhaustives (ASP ?)



\subsubsection*{Extensions de la modélisation et de l'analyse}

Dans un cadre plus général, au niveau de la modélisation,
la représentation de certains systèmes biologiques peut nécessiter de nouveaux ajouts au niveau du
formalisme,
comme des classes de priorités dynamiques ou des actions gardées.
À l'instar des réseaux d'automates, l'ajout de telles notions crée une forte complexité,
mais l'approche par interprétation abstraite pourrait à nouveau être plus efficace
qu'une recherche naïvement exhaustive.

En ce qui concerne l'analyse,
de nombreux autres résultats peuvent être recherchés sur un modèle de Frappes
de Processus, comme la prévision d'oscillations, la recherche
de domaines piège et d'attracteurs, etc.
Il est difficile de prévoir quels types d'analyse ces résultats peuvent requérir.
Le cas le plus souhaitable est une simple observation de la structure du modèle
(à l'instar des conjectures de Thomas) impliquant une complexité polynomiale, voire linéaire.
De telles observations peuvent porter sur la forme et l'agencement des actions,
sur la recherche d'éléments comme les $(n-k)$\nbd cliques dans le graphe sans-frappe,
etc.
Le cas le plus coûteux en temps de calcul, en revanche, serait la nécessite d'effectuer
une analyse exhaustive de la dynamique pour parvenir à un résultat.
De telles analyses peuvent cependant bénéficier d'avancées notables
comme l'utilisation d'un paradigme de programmation logique
telle que la programmation par ensembles de réponse ou \textit{Answer Set Programming}
\citeaffixed{benabdallah14}{voir p.~ex.}.
Il est de plus possible de trouver des résultats intermédiaires
permettant de guider la recherche et de trouver plus rapidement une solution.



\subsubsection*{Autres pistes d'application des capacités d'analyse}

L'inférence et la correction de modèles suscitent un engouement justifié,
alors que les données d'expression deviennent toujours plus accessibles.
L'utilisation de données de puce à ADN offre en effet la possibilité de corriger des modèles
existants ; l'automatisation de ce procédé présente alors un intérêt certain,
notamment pour les systèmes complexes, qui comportent beaucoup de mesures
et qui sont représentés par des modèles de grande taille
\citeaffixed{Guziolowski12072013}{voir p.~ex.}.
Les Frappes de Processus pourraient être avantageusement utilisées pour ce type de problème,
étant données leur définition atomique et les analyses efficaces de la dynamique
dont elles font l'objet.

Dans le même ordre d'idées,
la question de la résilience d'un système biologique peut aussi être soulevée
et abordée de plusieurs manières grâce aux Frappes de Processus.
Tout d'abord, la validité d'un modèle peut être testée sous l'angle de sa robustesse,
à savoir sa capacité à supporter des comportements biologiques supplémentaires.
Par ailleurs, l'ajout de comportements incontrôlables \cite{schwind13resilience}
poserait aussi de nouvelles questions :
le modèle court-il le risque d'être forcé à rentrer dans un état non acceptable ?
Est-il capable de revenir à tout moment dans un état acceptable ?
Et à quel prix ?
Répondre à ces questions pourra nécessiter d'élargir le cadre des propriétés vérifiables
de manière efficace.
D'un autre point de vue, il serait intéressant d'observer les conséquences sur la dynamique
de perturbations ponctuelles totalement aléatoires, comme l'ajout ou la suppression d'actions
dans un modèle.
Cette approche pourrait bénéficier d'autres moyens détournés comme le calcul des
\textit{cut-sets} afin de détecter des parties sensibles des modèles,
qui font office de «~goulots d'étranglement~» pour les bonnes propriétés de résilience.


% Création/inférence/adbuction de modèles :
%   selon un modèle de départ (requis) et des résultats expérimentaux (connaissance),
%     adapter le requis à la connaissance + minimalité
% Moyens :
%   
% Études de résilience :
%   en fonction de perturbations (aléatoires ?) mesurer la capacité de résilience du modèle
% Moyens :
%   analyse statique pour comparer les comportements
%   connaissance des processus clefs ou cut-sets pour connaître les « goulots de résilience »
% 
% Études dynamiques :
%   détecter d'autres comportements (oscillations, etc)
% Méthodes :
%   analyse statique ? pas nécessairement efficace
%   déceler les motifs menant à ces comportements (structure des actions)
%   déceler des propriétés permettant ces comportements ($n-k$) cliques, etc.
%   méthodes exhaustives (ASP)

  
%\chapter*{Références}
%\bibliographystyle{alpha}
\cleardoublepage
\addcontentsline{toc}{chapter}{Bibliographie}
\bibliography{biblio}



\end{document}
