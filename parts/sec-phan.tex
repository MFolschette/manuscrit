\subsection{Frappes de Processus avec arcs neutralisants}
\seclabel{phan}

Nous introduisons ici la notion d'\emph{arc neutralisant} dans les Frappes de Processus
afin de représenter la préemption d'une action par une seule autre.
Les \emph{Frappes de Processus avec arcs neutralisants} (\defref{php})
permettent notamment une modélisation plus atomique
par rapport aux classes de priorités présentées à la \secref{php}.

Un arc neutralisant est un couple d'actions noté $\PHan{h_1}{h_2}$,
où $h_1$ est appelée \emph{action bloquante},
et peut préempter $h_2$, appelée \emph{action bloquée},
dans certaines situations.
Avec la présence d'arcs neutralisants, une action est dite \emph{activée} dans un état donné si
son frappeur et sa cible y sont présents ;
une action est donc activée pour les Frappes de Processus avec arcs neutralisants
là où elle était immédiatement jouable pour les Frappes de Processus standards (\defref{fopph}).
Une action est \emph{jouable} pour les Frappes de Processus avec arcs neutralisants
si et seulement si elle est activée,
et que pour tout arc neutralisant la bloquant, son action bloquante n'est pas activée.
Une action activée mais qui n'est pas jouable est dite \emph{neutralisée}.

Il est à noter que la neutralisation d'une action par une autre ne dépend donc pas de la jouabilité
de l'action bloquante, mais uniquement de son activation.
Cela permet d'avoir un modèle cohérent, sans quoi certaines situations pourraient ne pas être
définies, notamment dans le cas d'un interblocage.
Ainsi, faire reposer la neutralisation d'une action bloquée sur la jouabilité de l'action bloquante
devient inextricable dans un cas comme le suivant :
$\PHan{h_1}{h_2}$, $\PHan{h_2}{h_3}$ et $\PHan{h_3}{h_1}$,
car si les trois actions $h1$, $h2$ et $h3$ sont actives, leur jouabilité reste indéterminée.
En revanche, si cette neutralisation repose sur l'état activé d'une action,
la situation précédente se résout immédiatement car aucune des trois actions n'est jouable.
On constate par ailleurs qu'une action peut en neutraliser une autre
même si elle-même est neutralisée.
Nous ne nous avancerons cependant pas sur la signification biologique de ce fait.

\todo{Schéma exemple}

\begin{definition}[Frappes de Processus avec arcs neutralisants]
\deflabel{phan}
  Les \emph{Frappes de Processus avec arcs neutralisants} sont définies par
  un quadruplet $\PH = (\PHs ; \PHl ; \PHh ; \PHn)$, où :
  \begin{itemize}
    \item $\PHs \DEF \{a, b, \dots\}$ est l'ensemble fini et dénombrable des \emph{sortes} ;
    \item $\PHl \DEF \bigtimes{a \in \PHs} \PHl_a$ est l'ensemble fini des \emph{états},
      où $\PHl_a = \{a_0, \ldots, a_{l_a}\}$ est l'ensemble fini et dénombrable
      des \emph{processus} de la sorte $a \in \PHs$ et $l_a \in \sN^*$.
      Chaque processus appartient à une unique sorte :
      $\forall (a_i; b_j) \in \PHl_a \times \PHl_b, a \neq b \Rightarrow a_i \neq b_j$ ;
    \item $\PHh \DEF \{\PHfrappe{a_i}{b_j}{b_l} \mid (a; b) \in \PHs^2 \wedge
      (a_i; b_j; b_l) \in \PHl_a \times \PHl_b \times \PHl_b \wedge
      b_j \neq b_l \wedge a = b \Rightarrow a_i = b_j \}$ est l'ensemble fini des actions ;
    \item $\PHn = \{\PHan{h_1}{h_2} \mid (h_1 ; h_2) \in \PHh \times \PHh\}$
      est l'ensemble fini des arcs neutralisants.
  \end{itemize}
\end{definition}

Un arc neutralisant $u = \PHan{h_1}{h_2} \in \PHn$ est donc un couple d'actions.
On note $\PHbloquant(u) = h_1$ la première action du couple $u$
et $\PHbloque(u) = h_2$ sa seconde action.
On réutilise par ailleurs les autres notations définies à la \secref{ph}.

L'opérateur de jouabilité des frappes de Processus avec arcs neutralisants (\defref{fopphan})
se concentre sur la présence du frappeur et de la cible de l'action considérée,
et sur celle de toutes ses actions bloquantes.
En ce sens, il est semblable à celui des Frappes de Processus avec $k$ classes de priorités
(\defref{fopphp}).

\begin{definition}[Opérateur de jouabilité ($\Fopsymbol_\Fopphansubsymbol : \PHh \rightarrow \F$)]
\deflabel{fopphan}
  L'opérateur de jouabilité des Frappes de Processus avec arcs neutralisants est défini par :
  \[\forall h \in \PHh, \Fopphan{h} \equiv \hitter{h} \wedge \target{h} \wedge
    \left( \bigwedge_{\substack{u \in \PHn\\u = \PHan{g}{h}}}
    \neg \left( \hitter{g} \wedge \target{g} \right) \right)\]
\end{definition}

\todo{Fonctionne encore à condition de prendre en compte les préemptions : analyse stochastique}

\todo{Ne fonctionne plus : points fixes}

\todo{Fonctionne avec traduction (PHcanonique) : analyse statique}

\todo{Traduction vers PHcanonique}
