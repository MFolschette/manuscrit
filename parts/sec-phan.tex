\section{Frappes de Processus avec arcs neutralisants}
\seclabel{phan}

Nous introduisons ici la notion d'\emph{arc neutralisant} dans les Frappes de Processus
afin de représenter la préemption d'une action par une seule autre.
Les \emph{Frappes de Processus avec arcs neutralisants} (\defref{php})
permettent notamment une modélisation plus atomique
par rapport aux classes de priorités présentées à la \secref{php}.

Un arc neutralisant est un couple d'actions noté $\PHan{h_1}{h_2}$,
où $h_1$ est appelée \emph{action bloquante},
et peut préempter $h_2$, appelée \emph{action bloquée},
dans certaines situations.
Avec la présence d'arcs neutralisants, une action est dite \emph{activée} dans un état donné si
son frappeur et sa cible y sont présents ;
une action est donc activée pour les Frappes de Processus avec arcs neutralisants
là où elle était immédiatement jouable pour les Frappes de Processus standards (\defref{fopph}).
Une action est \emph{jouable} pour les Frappes de Processus avec arcs neutralisants
si et seulement si elle est activée,
et que pour tout arc neutralisant la bloquant, son action bloquante n'est pas activée.
Une action activée mais qui n'est pas jouable est dite \emph{neutralisée}.

Il est à noter que la neutralisation d'une action par une autre ne dépend donc pas de la jouabilité
de l'action bloquante, mais uniquement de son activation.
Cela permet d'avoir un modèle cohérent, sans quoi certaines situations pourraient ne pas être
définies, notamment dans le cas d'un interblocage.
Ainsi, faire reposer la neutralisation d'une action bloquée sur la jouabilité de l'action bloquante
devient inextricable dans un cas comme le suivant :
$\PHan{h_1}{h_2}$, $\PHan{h_2}{h_3}$ et $\PHan{h_3}{h_1}$,
car si les trois actions $h1$, $h2$ et $h3$ sont actives, leur jouabilité reste indéterminée.
En revanche, si cette neutralisation repose sur l'état activé d'une action,
la situation précédente se résout immédiatement car aucune des trois actions n'est jouable.
On constate par ailleurs qu'une action peut en neutraliser une autre
même si elle-même est neutralisée.
Nous ne nous avancerons cependant pas sur la signification biologique de ce fait.

Enfin, il semble nécessaire de faire un parallèle entre les arcs inhibiteurs développés ici
et les arcs du même nom utilisés dans les réseaux de Petri \toref.
Leur rôle est effectivement proche, leur but étant la préemption d'une action en fonction
d'une condition extérieure à son déclenchement.
Malgré tout, les arcs inhibiteurs des réseaux de Petri se différencient car ils
reposent sur la présence d'un certain nombre de jetons dans une place
---~ce qui se traduirait, en Frappes de Processus,
par la présence d'un certain processus actif d'une sorte donnée~---
là où les arcs inhibiteurs des Frappes de Processus ne permettent l'inhibition
qu'en fonction de l'activité d'une autre action.
Cependant, un tel choix de conception peut aisément se pallier dans un sens comme dans l'autre.
En effet, représenter l'activité d'une action revient à créer deux arcs inhibiteurs en
réseaux de Petri ---~l'un pour le frappeur et l'autre pour la cible.
À l'inverse, il est possible qu'une action $h$ se bloque elle-même, à l'aide d'un 
«~auto-arc neutralisant~» $\PHan{h}{h}$,
permettant ainsi de bloquer une autre action $g$
en fonction de la présence du processus $\frappeur{h}$
à l'aide d'un deuxième arc neutralisant $\PHan{h}{g}$.

\todo{Schéma exemple}

\subsection{Définition}
\seclabel{phan-def}

\begin{definition}[Frappes de Processus avec arcs neutralisants]
\deflabel{phan}
  Les \emph{Frappes de Processus avec arcs neutralisants} sont définies par
  un quadruplet $\PH = (\PHs ; \PHl ; \PHh ; \PHn)$, où :
  \begin{itemize}
    \item $\PHs \DEF \{a, b, \dots\}$ est l'ensemble fini et dénombrable des \emph{sortes} ;
    \item $\PHl \DEF \bigtimes{a \in \PHs} \PHl_a$ est l'ensemble fini des \emph{états},
      où $\PHl_a = \{a_0, \ldots, a_{l_a}\}$ est l'ensemble fini et dénombrable
      des \emph{processus} de la sorte $a \in \PHs$ et $l_a \in \sN^*$.
      Chaque processus appartient à une unique sorte :
      $\forall (a_i; b_j) \in \PHl_a \times \PHl_b, a \neq b \Rightarrow a_i \neq b_j$ ;
    \item $\PHh \DEF \{\PHfrappe{a_i}{b_j}{b_l} \mid (a; b) \in \PHs^2 \wedge
      (a_i; b_j; b_l) \in \PHl_a \times \PHl_b \times \PHl_b \wedge
      b_j \neq b_l \wedge a = b \Rightarrow a_i = b_j \}$ est l'ensemble fini des actions ;
    \item $\PHn = \{\PHan{h_1}{h_2} \mid (h_1 ; h_2) \in \PHh \times \PHh\}$
      est l'ensemble fini des arcs neutralisants.
  \end{itemize}
\end{definition}

Un arc neutralisant $u = \PHan{h_1}{h_2} \in \PHn$ est donc un couple d'actions.
On note $\PHbloquant(u) = h_1$ la première action du couple $u$
et $\PHbloque(u) = h_2$ sa seconde action.
On réutilise par ailleurs les autres notations définies à la \secref{ph}.

L'opérateur de jouabilité des frappes de Processus avec arcs neutralisants (\defref{fopphan})
se concentre sur la présence du frappeur et de la cible de l'action considérée,
et sur celle de toutes ses actions bloquantes.
En ce sens, il est semblable à celui des Frappes de Processus avec $k$ classes de priorités
(\defref{fopphp}).

\begin{definition}[Opérateur de jouabilité ($\Fopsymbol_\Fopphansubsymbol : \PHh \rightarrow \F$)]
\deflabel{fopphan}
  L'opérateur de jouabilité des Frappes de Processus avec arcs neutralisants est défini par :
  \[\forall h \in \PHh, \Fopphan{h} \equiv \hitter{h} \wedge \target{h} \wedge
    \left( \bigwedge_{\substack{g \in \PHh\\\PHan{g}{h} \in \PHn}}
    \neg \left( \hitter{g} \wedge \target{g} \right) \right)\]
\end{definition}



\subsection{Équivalence avec les Frappes de Processus avec $k$ classes de priorités}
\seclabel{phan-equiv-php}

Les Frappes de Processus avec arcs neutralisants ont une expressivité équivalente
aux Frappes de Processus avec $k$ classes de priorité, où $k \in \sN^*$.
En effet, la \thmref{phpbisimphan} propose une traduction des Frappes de Processus
avec $k$ classes de priorités en Frappes de Processus avec arcs neutralisants.
À l'inverse, les Frappes de Processus avec arcs neutralisants peuvent être représentée
en Frappes de Processus canoniques, qui sont une sous-classe des Frappes de Processus avec
2 classes de priorités, comme démontré à la \secref{phan-aplatissement}.

\begin{theorem}[Équivalence avec les classes de priorités]
\thmlabel{phpbisimphan}
  Soient $\PH = (\PHs, \PHl, \PHh^{\angles{k}})$ des Frappes de Processus avec $k$ classes
  de priorités, où $k \in \sN^*$.
  Il existe un modèle $\oPH$ de Frappes de Processus avec arcs neutralisants tel que :
  \[\forall s, s' \in \PHl, s \trans{\PH} s' \Longleftrightarrow s \trans{\oPH} s' \]
\end{theorem}

\begin{proof}
  On pose : $\oPH = (\PHs, \PHl, \PHh, \PHn)$ les Frappes de Processus avec arcs neutralisants
  dont l'ensemble des actions est l'union de toutes les classes de priorités de $\PH$, et :
  $\PHn = \{ \PHan{g}{h} \mid g \in \PHh^{(n)}, h \in \PHh^{(k)}, n < k \}$.
  Soit $h \in \PHh$. et
  D'après la définition de $\oPH$, on constate que :
  $\{ g \in \PHh \mid \PHan{g}{h} \in \PHn \} = \{ g \in \PHh \mid n < \prio(h) \}$.
  Ainsi, $\forall h \in \PHh, \Fopphp{h} = \Fopphan{h}$,
  ce qui implique que les action jouables dans l'état $s$ du le modèle $\PH$
  sont exactement les actions jouables dans l'état $s$ du modèle $\oPH$.
  D'où : $s \trans{\PH} s' \Longleftrightarrow s \trans{\oPH} s'$.
\end{proof}



\subsection{Réutilisation des outils existants}
\seclabel{phan-outils}

\todo{Revoir le titre de cette section}

\todo{Faudrait-il fusionner cette section avec ses semblables des \secref{php} et \secref{phm} ?}

\subsubsection{Points fixes}
\seclabel{phan-outils-pf}

De façon analogue à la \secref{php-outils-pf}, nous définissions ici
la \emph{fusion} d'un modèle de Frappes de Processus avec arcs neutralisants
comme étant le même modèle en Frappes de Processus standards,
dont les arcs neutralisants ont été ignorés (\defref{phan-fusion}).
Cette définition nous permet d'avancer le résultat suivant :
les points fixes de $\PHmergean(\PH)$ sont des points fixes de $\PH$,
d'après la contraposée du \thmref{php-pf}.
Il n'est en effet pas possible d'avancer de résultat plus précis de façon immédiate,
car les arcs neutralisants peuvent former des boucles d'actions qui se neutralisent entre elles,
rajoutant ainsi des points fixes supplémentaires par rapport au modèle fusionné.
Ce cas ne se présente pas pour les Frappes de Processus avec $k$ classes de priorités.

\begin{definition}[Fusion ($\PHmergean$)]
\deflabel{phan-fusion}
  Soient $\PH = (\PHs, \PHl, \PHh, \PHn)$ des Frappes de Processus avec arcs neutralisants.
  On note $\PHmergean(\PH) = (\PHs, \PHl, \PHh)$
  les Frappes de Processus standards appelées \emph{fusion} de $\PH$,
  dont on a retiré les arcs neutralisants.
\end{definition}

\begin{theorem}[Points fixes des Frappes de Processus avec arcs neutralisants]
\thmlabel{phan-pf}
  Soient $\PH = (\PHs, \PHl, \PHh, \PHn)$ des Frappes de Processus avec arcs neutralisants,
  et $r \in \PHl$ :
  \[\exists s \in \PHl, r \trans{\PH} s \Longrightarrow
    \exists s' \in \PHl, r \trans{\PHmergean(\PH)} s'\]
\end{theorem}

\begin{proof}
  Si une action est jouable dans $r$, cela signifie notamment
  que son frappeur et sa cible sont présents dans $r$.
  Cette condition est suffisante pour que cette même action soit jouable dans $\PHmergean(\PH)$.
\end{proof}

\todo{Les cycles d'actions sont-ils suffisants pour trouver tous les points fixes ?}

Pour finir, il est toujours envisageable d'aplatir des Frappes de Processus avec
arcs neutralisants en Frappes de Processus canoniques,
comme décrit à la \secref{phan-aplatissement},
afin d'effectuer une recherche de points fixes à l'aide de la méthode
proposée à la \secref{php-outils-pf}.
Comme le modèle aplati de Frappes de Processus avec arcs neutralisants possède la même dynamique,
il est possible de retrouver les points fixes du modèle original de cette façon,
en supprimant simplement les sortes coopératives des résultats.

\subsubsection{Analyse statique}
\seclabel{phan-outils-as}

De même qu'à la \secref{php-outils-as},
étant donné que les Frappes de Processus avec arcs neutralisants permettent de restreindre
la dynamique par rapport aux Frappes de Processus standards, il est toujours possible
d'utiliser l'analyse statique par sur-approximation,
mais l'analyse statique par sous-approximation n'est plus valable.

Cependant, nous proposons au \chapref{phcanonique}
une traduction des Frappes de Processus avec arcs neutralisants en Frappes de Processus canoniques,
qui sont des Frappes de Processus avec 2 classes de priorités avec des contraintes précises,
ainsi qu'une nouvelle approche d'analyse statique par sous-approximation
qui s'applique à cette classe particulière de modèles.
L'analyse statique par sous-approximation est donc possible sur les Frappes de Processus
avec arcs neutralisants, au prix de cette traduction.

\subsubsection{Paramètres stochastiques}

Enfin, à l'instar de la \secref{php-outils-stocha},
il n'est pas possible d'utiliser directement les outils développés pour la simulation stochastique
car ceux-ci ne prennent pas en compte la présence d'arcs neutralisants.
Malgré cela, une telle utilisation reste théoriquement possible à condition de prendre
en compte les neutralisations entre actions.

Cependant, à nouveau, un parallèle intéressant peut être tracé entre les arcs neutralisants
et l'introduction de paramètres stochastiques dans le modèle.
En effet, étant donné un ensemble de paramètres stochastiques,
si les intervalles de tir de deux actions sont disjoints,
alors il est possible de modéliser cela par un arc neutralisant
dont l'action bloquante est l'action la plus «~rapide~»
(\cad donc l'intervalle est le plus proche de zéro)
et l'action bloquée est la plus «~lente~»
(\cad donc l'intervalle est le plus éloigné de zéro).

Cette représentation est avantageuse par rapport à celle proposée à la
\secref{php-outils-stocha}.
En effet, il n'est pas nécessaire pour créer le modèle
de distinguer des classes globales d'actions selon leurs
intervalles de tir, mais seulement de déterminer si les intervalles de tir de toutes les actions
sont, deux à deux, disjoints (voire assez éloignés selon des critères qui peuvent être
fixés arbitrairement).
Cela permet de plus d'obtenir des relations plus fines entre actions,
là où les Frappes de Processus ne permettent que la distinction de classes de priorités
globales, qui sont parfois trop grossières pour certaines modélisations.
Cependant, le problème de la représentation de l'accumulation persiste :
une partie du modèle évoluant «~plus rapidement~» peut totalement préempter une
action «~plus lente~» si celle-ci est sans cesse neutralisées.
Or en pratique, un tel cas devrait en définitive autoriser l'action «~lente~» à s'exécuter
après un certain temps.
Cela peut être corrigé en supprimant quelques arcs neutralisants bien choisis,
ou encore en ne permettant pas la création d'arcs neutralisants entre les parties indépendantes
du modèle, mais cela nécessite une analyse préalable assez poussée de la dynamique
du système.
