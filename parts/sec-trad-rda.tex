% Traduction depuis les réseaux discrets asynchrones}

\section{Traduction depuis les réseaux discrets asynchrones}
\seclabel{trad-rda}

Nous proposons dans cette section une traduction des réseaux discrets asynchrones
en Frappes de Processus canoniques, et nous en montrons la validité
par une preuve de bisimulation faible.
Les réseaux discrets asynchrones ont été préalablement définis à la \vdefref{rda-def}
du \chapref{etatdelart}.
Il s'agit de modèles proches du modèle de Thomas, mais en comportant pas de restriction unitaire
de la dynamique, et présentant des fonctions d'évolution en lieu et place de paramètres discrets.
Un réseau discret asynchrone se présente sous la forme d'un couple $\RDA = (\GI; F)$
où $\GI = (\components; E)$ est un graphe des interactions
et $F$ est un ensemble de fonctions $f_x : \RRBreg{x} \rightarrow \segm{0}{l_x}$
pour tout composant $x \in \components$,
où $\RRBreg{x}$ est l'ensemble des prédécesseurs de $x$ dans le graphe des interactions.
La dynamique d'un tel réseau est la suivante : il existe une transition $\RRBtransrda{s}{s'}$
si et seulement si un unique composant $x$ évolue entre $s$ et $s'$
de façon à ce que : $\RRBget{s'}{x} = f_x(s)$.
Les Frappes de Processus canoniques, définies à la \vsecref{phcanonique},
permettent une représentation presque immédiate des réseaux discrets asynchrones,
à condition de créer les sortes coopératives adéquates.

\myskip

% Nous complétons dans un premier temps la définition d'un réseau discret asynchrone par
% la notion de dépendances d'un composant.
% En effet, même si les fonctions d'évolution de chaque composant sont sur $\RRBstates$,
% elles ne dépendent généralement 
% pour tout composant $x \in \components$, $\RDAdep(x)$ est l'ensemble des composants dont
% la valeur de $f_x$ dépend véritablement,
% 
% \begin{definition}[Dépendance]
%   Soit $\RDA = (\GI; F)$ un réseau discret asynchrone, avec $\GI = (\components; E)$.
%   Pour tout composant $x \in \components$,
%   on note $\DNdep(x) \subseteq \components$ l'ensemble des composants dont
%   la valeur de $f_x$ dépend véritablement,
%   c'est-à-dire l'ensemble minimal tel que :
%   \[\forall s, s' \in \RRBstates,
%     \big(\forall y \in \DNdep(f_x), \get{s}{y} = \get{s'}{y}\big) \Longrightarrow
%     f_x(s) = f_x(s')\]
% \end{definition}

La traduction proposée à la \defref{rda2ph}
associe deux sortes à chaque composant $a$ dans $\RDA$ :
\begin{itemize}
  \item une sorte du même nom pour représenter ce composant,
  \item une sorte coopérative $f^a$ représentant sa fonction d'évolution $f_a$,
    et dont les états sont donc une combinaison des états de ses régulateurs.
\end{itemize}
De plus, les actions primaires sont définies de façon à correctement mettre à jour
les sortes coopératives,
et les actions secondaires le sont de façon à ce que chaque sorte coopérative $f^a$
interagisse avec son composant $a$ de la façon dont la fonction d'évolution
correspondante le permet.
Nous montrons de plus au \thmref{bisimulationrda2ph} que le modèle obtenu
est faiblement bisimilaire au réseau discret asynchrone d'origine.

\begin{definition}[Frappes de Processus équivalentes ($\rdatophsymbol$)]
\deflabel{rda2ph}
  Soit $\RDA = (\GI; F)$ un réseau discret asynchrone, avec $\GI = (\components; E)$.
  On note $\rdatoph = (\PHs; \PHl; (\PHh^{(1)}; \PHh^{(2)}))$
  les Frappes de Processus canoniques équivalentes
  à $\RDA$, définies par :
  \begin{itemize}
    \item $\PHs = \components \cup \{ f^a \mid a \in \components \}$,
    \item $\PHl = \bigtimes{a \in \components} \PHl_{a} \times
      \bigtimes{a \in \components} \PHl_{f^a}$ l'ensemble des états , où :
      \begin{align*}
      \forall a \in \components&, \PHl_{a} = \{ a_i \mid i \in \segm{0}{l_a} \} \\
      \forall a \in \components&, \PHl_{f^a} = \begin{cases}
          \PHsubl_{\RRBreg{a}} & \text{ si } \RRBreg{a} \neq \emptyset \\
          \{ f^a_\emptyset \}  & \text{ sinon }
        \end{cases} \enspace,
      \end{align*}
    \item $\PHh^{(1)} = \{ \PHfrappe{b_k}{f^a_\mysigma}{f^a_{\mysigma'}} \mid
      a \in \components \wedge b \in \RRBreg{a} \wedge
      b_k \in \PHl_{b} \wedge f^a_\mysigma \in \PHl_{f^a} \wedge
      \get{\mysigma}{b} \neq b_k \wedge \mysigma' = \mysigma \recouvre b_k \}$,
    \item $\PHh^{(2)} = \{ \PHfrappe{f^a_\mysigma}{a_j}{a_{k}} \mid
      a \in \components \wedge f^a_\mysigma \in \PHl_{f^a} \wedge
      a_j, a_k \in \PHl_{a} \wedge j \neq k \wedge f_a(\decode \mysigma) = k \}$.
  \end{itemize}
  
  Pour tout état $s \in \RRBstates$ de $\RDA$,
  $\encode{s} = \os$ est l'état correspondant dans $\rdatoph$, défini par :
  $\forall a \in \components, \get{s}{a} = k \Rightarrow \get{\os}{a} = a_k$
  et
  $\forall a \in \components, \get{\os}{f^a}=f^a_\mysigma$
  avec $f^a_\mysigma \in \PHl_{f^a}$
  et $\forall b \in \RRBreg{a}, \get{\mysigma}{b} = \get{\os}{b}$.

  À l'inverse, pour tout état $\os \in \PHsubl$ de $\rdatoph$,
  $\decode{\os} = s$ est l'état correspondant dans $\RDA$ avec :
  $\forall a \in \sortes{\os}, \get{\os}{a} = a_k \Rightarrow \get{s}{a} = k$.
\end{definition}

\begin{theorem}[$\RDA \approx \rdatoph$]
\thmlabel{bisimulationrda2ph}
  Soit $\RDA = (\GI; F)$ un réseau discret asynchrone.
  On a :
  \begin{enumerate}
    \item \label{rda2ph} $\forall s, s' \in \RRBstates$,
      $s \RDAtrans s' \Longrightarrow \encode{s} \mtrans{\PH} \encode{s'}$,
      où $\mtrans{\PH}$ est une séquence finie de transitions $\trans{\PH}$.
    \item \label{ph2rda} $\forall \os, \os' \in \PHl$,
      $\os \trans{\PH} \os' \Longrightarrow
        \decode{\os} = \decode{\os'} \vee \decode{\os} \RDAtrans \decode{\os'}$
  \end{enumerate}
\end{theorem}

\begin{proof}
  On pose : $\rdatoph = (\PHs; \PHl; (\PHh^{(1)}; \PHh^{(2)}))$.
  
  (\ref{rda2ph}) Soient $s, s' \in \RRBstates$ tels que $s \RDAtrans s'$.
    Cela signifie qu'il existe un composant $a \in \components$ tel que :
    $\RRBget{s'}{a} = f_a(\mysigma)$ et
    $\forall b \in \components, b \neq a \Rightarrow \RRBget{s}{b} = \RRBget{s'}{b}$,
    où $\mysigma \in \PHsubl_{\RRBreg{a}}$ tel que $\mysigma \subseteq \encode{s}$.
    Posons : $j = \RRBget{s}{a}$ et $k = \RRBget{s'}{a}$ ;
    d'après la \defref{rda2ph}, il existe une action
    $\PHfrappe{f^a_\mysigma}{a_j}{a_k} \in \PHh^{(2)}$
    avec $f_a(\mysigma) = k$
    Par définition de $\encode{s}$, on a $a_k \in \encode{s}$
    et $f^a_\mysigma \in \encode{s}$,
    et aucune action de $\PHh^{(1)}$ n'est jouable dans $\encode{s}$ ;
    ainsi, $h$ est jouable dans $\encode{s}$, d'où : $\encode{s} \PHtrans \encode{s} \play h$.
    De plus, d'après le \vlemref{update},
    $\encode{s} \play h \mtrans{\PH} \update(\encode{s} \play h)$.
    Enfin, comme $\update$ met à jour des sortes coopératives, on a :
    $\update(\encode{s} \play h) = \encode{s'}$.
  
  (\ref{ph2rda}) Soient $\os, \os' \in \PHl$ tels que $\os \trans{\PH} \os'$.
    Cela signifie qu'il existe une action $h \in \PHh$ telle que : $s' = s \play h$.
    Si $h \in \PHh^{(1)}$, alors $\decode{\os} = \decode{\os'}$, par définition
    de $\decode{\os}$.
    En revanche, si $h \in \PHh^{(2)}$, alors d'après la \defref{rda2ph},
    il existe $a \in \components$, $\mysigma \in \PHsubl_{\RRBreg{a}}$
    et $a_j, a_k \in \PHl_{a}$, tels que :
    $h = \PHfrappe{f^a_\mysigma}{a_j}{a_{k}}$, avec $f_a(\decode \mysigma) = k$ et $j \neq k$.
    Ainsi, $\forall b \in \RRBreg{a}$,
    $\PHget{\mysigma}{b} = b_i \Rightarrow \RRBget{\decode{\os}}{b} = i$.
    D'où : $\decode{\os} \RDAtrans \decode{\os'}$
    car $f_a(\decode \mysigma) = k$.
\end{proof}
