% Introduction

\chapter{Introduction}

\section{Contexte \& Motivations}

Le principal défi posé par l'étude des systèmes dynamiques réels,
qu'ils soient biologiques ou non,
repose dans la modélisation qui en est faite.
Un modèle permet d'abstraire les comportements du système
pour s'intéresser uniquement à ceux qui présentent un intérêt à l'étude,
tout en permettant leur analyse à l'aide d'outils préalablement développés.

Un modèle doit donc dans l'idéal :
\begin{itemize}
  \item être cohérent avec la réalité du système qu'il représente,
  \item reproduire les comportements présentant un intérêt
    et abstraire ceux qui surchargent inutilement le modèle,
%   \item abstraire les comportements qui rendent la représentation et l'analyse inutilement
%     complexe,
  \item faciliter la lecture pour le modélisateur,
  \item permettre l'analyse par des outils appropriés,
  \item permettre la traduction depuis ou vers d'autres formalismes
    ou l'ouverture à d'autres méthodes d'analyse.
\end{itemize}

Nous nous intéressons dans cette thèse aux propriétés dynamiques d'un modèle.
Elles se distinguent des propriétés statiques qui permettent de caractériser le modèle
en fonction de sa taille,
des liens entre les éléments qui le composent,
ou de toute autre propriété portant sur sa structure ou ses attributs,
bien que celles-ci apportent parfois des résultats très généraux concernant la dynamique.
À l'inverse, les propriétés dynamiques portent sur l'évolution et les comportements
possibles d'un modèles, par exemple :
\begin{itemize}
  \item Étant donné un certain état des entrées du système, le modèle est-il capable
    d'en reproduire ou d'en prédire les sorties ?
  \item Y retrouve-t-on des comportements qui s'apparentent à des oscillations
    ou des états stables ?
  \item Les comportements recherchés sont-ils accessibles depuis tous les états ?
    Si non, depuis lesquels ?
  \item Peut-on modifier le modèle pour voir apparaître un comportement donné, et comment ?
\end{itemize}
Répondre ces questions nécessite une analyse détaillée de la dynamique.

Ainsi, l'utilisation d'un modèle pose un double défi : sa conception et son analyse.

Il est nécessaire de proposer des outils permettant une modélisation cohérente et juste.
C'est pourquoi nous proposons dans cette thèse plusieurs formalismes nouveaux permettant
des représentations efficaces et complémentaires des systèmes dynamiques étudiés.
L'une des pistes d'enrichissement consiste pour cela
en l'introduction de contraintes dynamiques afin de filtrer les comportements non désirés,
par exemple sous la forme 
de relations de prévalence entre les différentes évolutions possibles d'un modèle.
D'autres pistes consistent à intégrer des paramètres temporels dans les modèles
en se basant sur des données de durée de réaction ou de sensibilisation.
Enfin, l'ajout d'outils de synchronisation à des formalismes purement asynchrones
permet d'enrichir leur sémantique.

% Des outils peuvent être proposés pour affiner la dynamique complète du modèle,
% afin d'éliminer des comportements non désirés.
% L'objectif est donc de pouvoir exprimer des contraintes supplémentaires
% qui ne l'étaient pas auparavant.
% Ainsi, il est possible par exemple
% d'élaborer un système décrivant une relation de prévalence entre les
% différentes évolutions d'un modèle.
% De cette manière, on peut parvenir à intégrer des notions temporelles dans les modèles,
% en se basant sur des données chronométriques.
% Il est aussi possible de pallier certains manques des modèles,
% par exemple en ajoutant des outils de synchronisme à des formalismes purement asynchrones.

Un autre problème courant est le lien entre plusieurs formalismes.
En effet, des formalismes différents peuvent permettre des approches variées,
et l'application d'outils adaptés à des problèmes précis.
Il s'avère donc nécessaire de pouvoir faire le lien entre différentes versions
d'un formalisme, ou encore de créer des ponts vers d'autres formalismes répandus
dans le cadre des systèmes dynamiques.
Aussi, cette thèse offre une part importante à l'étude des relations entre différentes
représentations complémentaires des modèles dynamiques étudiés,
ainsi qu'avec les modèles classiques pour ce type de représentations.

Cependant, vérifier de telles propriétés dynamiques possède un coût
en termes de temps d'exécution et de taille mémoire,
lié à la taille du modèle considéré.
Et même si, dans les meilleurs cas, ce coût varie de façon polynomiale
en fonction de la taille du modèle,
la plupart des méthodes formelles permettant de vérifier des propriétés dynamiques
font face à une explosion combinatoire qui empêche l'analyse de modèles de grande taille.
C'est pourquoi des méthodes alternatives peuvent être explorées,
comme la vérification par interprétation abstraite,
qui consiste à approcher la dynamique pour diminuer la complexité et simplifier les calculs,
comme nous le verrons dans la suite de ce manuscrit.

% ce qui permet d'analyser des modèles de très grande taille avec des ressources raisonnables.
% Cependant, il arrive que ces analyses soient de complexité exponentielle, voire plus,
% en fonction de la taille de modèle.
% Dans ce cas, des modèles de petite taille peuvent être analysés à condition d'avoir
% les capacités de calcul et de stockage nécessaires,
% mais la complexité croît trop vite avec la taille du modèle pour que
% des réseaux de grande taille puissent être abordés par ces méthodes.

%\todo{+ rapide retour sur les contributions}

La \secref{intro-rrb} propose une rapide vue d'ensemble des types de modèles permettant
la représentation et l'étude des systèmes d'interactions,
afin de mieux situer le travail proposé dans cette thèse,
dont les principaux résultats sont ensuite présentés à la section \secref{intro-contrib}.
La section \secref{intro-plan} présente la façon dont ce manuscrit est organisé,
et la section \secref{notations} introduit les différentes notations qui y sont utilisées.



\section{Les réseaux de régulation biologiques}
\seclabel{intro-rrb}

%\stodo{[Thomas'73], [Paulevé'11 et '12]}

\tocite[Citer Fages \& Soliman, 2008a ?]
\\\todo{Ajouter des figures ?
\begin{itemize}
  \item Graphique figurant l'abstraction continu $\rightarrow$ discret
  \item Exemple de graphe des interactions
  \item Exemple de dynamique
  \item ... ?
\end{itemize}
}

L'étude de la machine cellulaire nécessite de s'intéresser aux éléments interagissant
qui la composent :
%des interactions entre les différents composants de la machine cellulaire
gènes, protéines, ARN messager, métabolites, etc.
% 
% Plusieurs approches permettent de représenter et d'analyser le fonctionnement de la machine
% cellulaire.
% L'une d'entre elles consiste à étudier les interactions qui y ont lieu
% entre les différents éléments qui s'y trouvent :
% gènes, protéines, ARN messager, métabolites, etc.
À ce niveau, il est déjà possible d'abstraire dans une certaine mesure
une partie de composants.
%les différents composants peuvent être plus ou moins abstraits.
C'est ainsi qu'un gène, la protéine qu'il code et l'ARN messager correspondant sont
souvent tous trois modélisés par un unique élément,
car la concentration de la protéine et de l'ARN
dépendent directement du niveau d'activation du gène.
Dans la suite nous nous intéresserons particulièrement à ce cas de figure,
et nous confondrons souvent une protéine avec son gène codant.
%bien que d'autres peuvent être représentés de la même manière.

Cette simplification permet de mettre en évidence les phénomènes d'interaction entre
les différents éléments entrant en jeu.
En effet, la présence d'un composant (protéine, catalyseur…)
en quantité suffisante peut déclencher
l'activation (une hausse d'activité) ou l'inhibition (une baisse de l'activité)
d'un ou plusieurs autres éléments, y compris l'élément déclencheur lui-même.
C'est par ce mécanisme que se crée une cascade de réaction entre gènes :
%C'est par ce mécanisme qu'un gène, selon son degré d'activité,
en effet, l'un d'eux, selon son degré d'activité,
permettra la production d'une concentration plus ou moins importante de la protéine qu'il code ;
celle-ci aura alors éventuellement un rôle activateur ou inhibiteur sur un certain
nombre d'autres gènes, et ainsi de suite.
L'ensemble de ces régulations peut être représenté par un graphe des interactions,
où les composants sont représentés par des nœuds et leurs interactions mutuelles par des arcs.

% À ce niveau, les différents composants peuvent être plus ou moins abstraits.
% Par exemple, un gène, la protéine qu'il code et l'ARN messager correspondant sont
% souvent modélisés par un unique nœud,
% car la concentration de la protéine dépend directement du niveau d'activation du gène.
% Dans la suite nous nous intéresserons particulièrement à ce cas de figure,
% bien que d'autres peuvent être représentés de la même manière.

Afin de représenter la dynamique du modèle,
une valeur est associée à chaque élément afin de modéliser
son état courant
(niveau d'activité pour un gène,
concentration dans le milieu pour une protéine, etc.).
% À chacun des éléments de modèle est alors associée une valeur modélisant
% son niveau d'activité ou
% sa concentration dans le milieu (pour une protéine).
Dans le cadre de formalismes utilisant des équations différentielles
il s'agit d'une valeur continue,
ce qui permet par exemple de lier entre elles
les concentrations des différentes protéines et leurs dérivées par rapport au temps,
qui représentent alors leurs vitesses d'évolution \cite{tyson1978dynamics}.
Cependant, le manque de données expérimentales précises et fiables peut limiter de telles
approches.
De plus, la résolution analytique ou numérique des équations différentielles est parfois
très complexe.

Une autre approche consiste donc à symboliser l'activité d'un élément par
une valeur discrète au sein d'un ensemble fini.
Cette approche avait été théorisée notamment par Stuart A. Kauffman puis par René Thomas
\cite{kauffman69,Thomas73}
dans le cadre d'un formalisme booléen,
c'est-à-dire restreint à deux niveaux discrets par composant, généralement notés “0” et et “1”.
Elle a cependant été plus tard étendue à un formalisme multivalué,
où chaque élément peut posséder avoir plus de deux niveaux d'expression,
généralement représentés par des entiers consécutifs.
Cette approche a l'avantage d'abstraire les valeurs des seuils de concentration,
qui sont généralement mal connues
mais qui permettent de représenter le niveau de concentration
à partir duquel la protéine va influencer un autre composant.
% quel niveau de concentration chaque composant
% possède une influence sur un autre.
Ainsi, à chaque niveau d'expression d'un composant est associé un ensemble de régulations
sur d'autres composants.

Le dernier élément permettant de caractériser la dynamique est l'ajout d'une
dimension temporelle.
Dans le cadre des réseaux de régulation discrets, ce temps prend la forme d'une série
infinie de pas de temps discrets permettant de représenter les évolutions
successives du modèles au cours du temps, sans indication des durées réelles
entre ces pas de temps.
La question du synchronisme des formalismes développés se pose alors :
en effet, s'il existe des systèmes purement synchrones, où entre chaque pas de temps,
toutes les réactions sensibilisées ont lieu,
l'hypothèse de René Thomas \cite{Thomas73}
est au contraire d'abstraire les systèmes biologiques à l'aide
de réseaux de régulation purement asynchrones.
En effet, en l'absence de données temporelles du système étudié
(vitesse des réactions, durée des dégradations...)
il est plus raisonnable de considérer que deux réactions n'auront jamais lieu simultanément.

Ce sont sur les hypothèse précédentes qu'ont été formalisés
les versions actuelles du modèle de Thomas \cite{Richard06}
et les Frappes de Processus \cite{PMR10-TCSB},
qui permettent toux deux
la représentation des réseaux de régulation biologiques discrets et asynchrones.
Les deux permettent en effet de représenter les interactions entre différents composants
sous la forme de l'évolution séquentielle de niveaux d'expression discrets.
Cependant, ces deux formalismes se distinguent principalement au niveau de la représentation
des interactions entre composants.
En effet, le modèle de Thomas considère les interactions entre composants du point
de vue d'«~influences~» globales, c'est-à-dire d'interactions qui vont globalement avoir
le rôle d'activer ou inhiber les composants entre eux,
tandis que les Frappes de Processus font usage d'«~actions~»,
qui décrivent le saut d'un état local d'un composant à un autre.
De plus, une restriction particulière porte sur ces actions,
qui ne permettent la modification du niveau local d'un composant
que par au plus un autre composant.
Cette différence de point de vue en apporte une autre en ce qui concerne la représentation
des coopérations entre composants :
là où le modèle de Thomas fait usage de paramètres décrivant
les états focaux d'un élément en fonction de l'activité de l'ensemble de ses régulateurs
\cite{Snoussi89},
les Frappes de Processus introduisent un composant supplémentaire propre à la modélisation
et qui joue le rôle de porte logique.

Les Frappes de Processus ont fait l'objet de travaux précédents.
Ceux-ci ont notamment permis de montrer qu'elles permettent de représenter une «~superposition~»
de modèles de Thomas, afin de représenter des ensembles de paramètres partiellement connus.
De plus, un travail approfondi a permis d'élaborer de puissantes analyses statiques
portant sur la recherche d'états stables,
mais aussi sur des questions dynamiques comme l'atteignabilité d'un état local \cite{PMR12-MSCS}
qui peut être traitée de façon très efficace, et donc appliquée à de très grands modèles.




\section{Contributions}
\seclabel{intro-contrib}

Les apports de cette thèse se déclinent en trois points principaux :
\begin{itemize}
  \item enrichissement des Frappes de Processus par l'introduction de notions
    de prévalence et de synchronisme entre les différentes évolutions possibles,
  \item développement de méthodes d'analyse de la dynamique efficaces
    et adaptées aux enrichissements précédemment mentionnés,
  \item description des liens formels entre les différentes sémantiques de Frappes de Processus
    proposées, ainsi qu'avec d'autres formalismes courants pour la représentation
    des réseaux de régulation biologiques.
\end{itemize}

L'objectif principal de cette thèse est donc de répondre à la double problématique
de la conception et de l'analyse d'un modèle,
en nous concentrant particulièrement sur l'utilisation et l'amélioration
des Frappes de Processus des des outils associés.
Les trois aspects précédents sont détaillés dans la suite
afin d'insister sur leur rôle dans cette démarche.

\subsubsection{Enrichissement des Frappes de Processus}

Cette thèse se concentre sur plusieurs \bemph{formalisations alternatives des
Frappes de Processus dans le but d'enrichir celles-ci}.
Pour cela, nous proposons trois approches qui gravitent autour des notions
de \bemph{préemption} et de \bemph{synchronisme} entre les différentes
évolutions possibles du modèle :
\begin{itemize}
  \item les \bemph{classes de priorités} permettent de définir des règles globales de prévalence
    entre des ensembles d'actions, afin d'affiner la dynamique et d'obtenir une expressivité
    équivalente à celle des réseaux booléens,
  \item les \bemph{arcs neutralisants} proposent de raffiner la notion précédente,
    en définissant les prévalences de façon plus atomique entre les actions individuelles,
  \item les \bemph{actions plurielles} permettent enfin d'introduire des comportements
    synchrones entre les actions, dans le but de modéliser des phénomènes simultanés
    comme la création des produits d'une réaction.
\end{itemize}
Ces ajouts offrent de plus la possibilité de prendre en compte des paramètres qui était
auparavant abstraits,
permettant par exemple \bemph{l'intégration de données temporelles}
comme les durées relatives des réactions biochimiques.

\subsubsection{Outils efficaces d'analyse de la dynamique}

Nous développons par ailleurs des méthodes permettant d'analyser la dynamique
des modélisations précédentes.
Ces méthodes sont basées sur de l'analyse statique par interprétation abstraite :
%c'est-à-dire permettant d'approcher la dynamique afin de réduire la complexité de l'analyse.
leur fonctionnement repose sur l'abstraction de la dynamique globale du système
au profit des dynamiques locales de chaque composant.
Elles ont l'avantage de posséder \bemph{une complexité polynomiale en la taille du modèle},
ce qui permet de \bemph{traiter efficacement de très grands modèles,
de l'ordre de centaines de composants, en quelques dixièmes de seconde}.
Néanmoins, étant donné qu'elles sont basées sur une approximation de la dynamique,
il est possible qu'elles terminent sans pouvoir conclure,
bien que cela ne soit jamais arrivé sur les exemples testés et présentés dans cette thèse.

\subsubsection{Équivalences entre les Frappes de Processus}

L'analyse statique mentionnée précédemment doit être appliquée
aux \bemph{Frappes de Processus canoniques} ;
une classe de modèles qui consiste en une restriction de l'un des formalismes.
Nous montrons par ailleurs que cette classe est
\bemph{aussi expressive que tous les formalismes
de Frappes de Processus} développés dans cette thèse,
et nous donnons les traductions correspondantes.
Cela assure notamment qu'il est toujours possible de naviguer entre les formalismes,
et ainsi de pouvoir appliquer les analyses statiques mentionnées à tous les types
de modélisation, moyennant une traduction du modèle.

\subsubsection{Équivalences avec d'autres formalismes}

Enfin, nous nous intéressons aux \bemph{liens formels entre les différentes
sémantiques de Frappes de Processus
et les autres modélisations classiques pour la représentation des réseaux de régulations
biologiques}.
Nous nous intéressons notamment au \bemph{modèle de Thomas} et aux \bemph{réseaux booléens},
deux formalismes proches
développés spécifiquement pour la représentation de réseaux de régulations,
et nous montrons qu'il est possible de représenter ces deux formalismes de façon exacte
grâce à l'ajout de priorités dans les Frappes de Processus.
Nous abordons aussi la question de la traduction vers les \bemph{réseaux de Petri},
un formalisme plus généraliste mais objet d'un regain d'intérêt du fait des importantes
capacités d'expressivité et d'analyse qu'il offre.
Ces liens permettent d'une part de comprendre la position des Frappes de Processus au sein
de l'ensemble des modélisation discrètes asynchrones,
et offrent d'autre part des outils de traduction depuis et vers ces autres formalismes,
donnant la possibilité d'utiliser les outils d'analyse spécifiquement développés pour ceux-ci.



\section{Organisation du manuscrit}
\seclabel{intro-plan}

Le présent manuscrit est organisé de la manière suivante.

Le \chapref{etatdelart} offre un état des lieux en matière de modélisation et d'analyse
des réseaux de régulation biologiques discrets et asynchrones.
Nous nous intéresserons notamment au modèle de Thomas et aux réseaux booléens,
deux types de réseaux très répandus pour des raisons historiques.
Nous définirons aussi les Frappes de Processus standard,
telles que définies par le travail de Loïc Paulevé.

Le \chapref{semantiques} propose plusieurs ajout aux Frappes de Processus afin de l'enrichir
et d'en augmenter l'expressivité.
Les trois formalismes abordés reposent sur les notions de classes de priorités,
d'arcs neutralisants et d'actions plurielles,
afin de filtrer les comportements désirés du modèles
ou d'ajouter des comportements nécessaires à la modélisation de certains phénomènes.
Nous discutons aussi dans ce chapitre des applications possibles de ces formalismes,
et traitons en partie la question des équivalences et des traductions entre ces formalismes.

Le \chapref{as} se concentre sur une classe particulière de Frappes de Processus avec classes
de priorités, dite «~canonique~».
Cette classe sert de base au développement de méthodes d'analyse statique permettant
de vérifier des propriétés d'atteignabilités locales dans un modèle.
Ces méthodes utilisent un mécanisme d'abstraction pour éviter l'explosion combinatoire
inhérente à toute étude de la dynamique d'un modèle discret, et ainsi rester efficaces.
Nous montrons aussi, traduction à l'appui,
que les Frappes de Processus canoniques sont en réalité équivalentes
aux trois extensions des Frappes de Processus proposées,
ce qui permet d'étendre la portée des résultats précédents à ces formalismes.

Le \chapref{expressivite} dépasse le cadre des Frappes de Processus et s'intéresse aux liens
formels entre les extensions proposées plus haut et plusieurs formalismes classiques discrets
permettant la modélisation des réseaux de régulation biologiques.
Nous y prouvons que les extensions des Frappes de Processus permettent d'obtenir la même
expressivité que le modèle de Thomas et les réseaux booléens,
ce qui étend encore la portée de nos résultats.
Nous proposons aussi une traduction vers une sous-classe des réseaux de Petri.

Le \chapref{applications} porte sur l'application des modèles et méthodes proposées
dans cette thèse à des réseaux de régulation biologique de grande taille.
Nous y présentons plusieurs systèmes biologiques classiquement étudiés,
la façon dont ils sont représentés en Frappes de Processus,
et les résultats de nos méthodes de traduction et d'analyse.

Enfin, le \chapref{conclusion} nous permet de conclure ce manuscrit
et d'en discuter les résultats.
