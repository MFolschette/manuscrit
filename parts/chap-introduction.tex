% Introduction

\chapter{Introduction}

\section{Contexte \& Motivations}

Le principal défi posé par l'étude des systèmes dynamiques réels,
qu'ils soient biologiques ou non,
repose dans la modélisation qui en est faite.
Le modèle permet d'abstraire les comportements du système
pour s'intéresser à ceux qui présentent un intérêt à l'étude,
tout en permettant leur analyse à l'aide d'outils préalablement développés.

Un modèle doit donc dans l'idéal :
\begin{itemize}
  \item être cohérent avec la réalité du système qu'il représente,
  \item reproduire les comportements présentant un intérêt,
  \item abstraire les comportements qui rendent la représentation et l'analyse inutilement
    complexe,
  \item faciliter la lecture pour le modélisateur,
  \item permettre l'analyse par des outils appropriés.
\end{itemize}

Les propriétés qui peuvent être extraites d'un modèle sont de plusieurs ordres.
Les propriétés statiques permettent de caractériser le modèle
en fonction de sa taille,
des liens entre les éléments qui le composent,
ou de toute autre propriété portant sur sa structure ou ses attributs,
qui amènent parfois à des résultats portant sur sa dynamique.
Nous nous intéresserons davantage dans cette thèse aux propriétés dynamiques d'un modèle,
qui permettent de déterminer certaines propriétés sur les comportements :
\begin{itemize}
  \item Étant donné un certain état des entrées du système, le modèle est-il capable
    d'en reproduire ou d'en prédire les sorties ?
  \item Quels types de comportements retrouve-t-on : oscillations, états stables ?
  \item Les comportements recherchés sont-ils accessibles depuis tous les états ?
    Si non, depuis lesquels ?
  \item Quels changements peut-on apporter au modèle pour voir apparaître un comportement donné ?
\end{itemize}
\todo{Revoir}

Nous nous intéressons dans cette thèse principalement aux modèles asynchrones,
bien que nous proposons aussi des outils permettant d'introduire des comportements synchrones.

Ainsi, l'utilisation d'un modèle pose un double défi : sa conception et son analyse.

Il est nécessaire de proposer des outils permettant une modélisation cohérente.
Ainsi, il est nécessaire de rendre les outils de modélisation suffisamment expressifs pour
permettre une représentation efficace des systèmes étudiés.
Des outils peuvent être proposés pour affiner la dynamique complète du modèle,
afin d'éliminer des comportements non désirés.
L'objectif est donc de pouvoir exprimer des contraintes supplémentaires
qui ne l'étaient pas auparavant.
Ainsi, il est possible par exemple
d'élaborer un système décrivant une relation de prévalence entre les
différentes évolutions d'un modèle.
De cette manière, on peut parvenir à intégrer des notions temporelles dans les modèles,
en se basant sur des données chronométriques.
Il est aussi possible de pallier certains manques des modèles,
par exemple en ajoutant des outils de synchronisme à des formalismes purement asynchrones.

Cependant, de telles propriétés dynamiques ont un coût de calcul,
en termes de temps d'exécution et de taille mémoire,
qui est lié à la taille du modèle considéré (nœuds ou arcs).
Dans les meilleurs des cas ce coût varie de façon polynomiale en fonction de la taille du modèle,
ce qui permet d'analyser des modèles de très grande taille avec des ressources raisonnables.
Cependant, il arrive que ces analyses soient de complexité exponentielle, voire plus,
en fonction de la taille de modèle.
Dans ce cas, des modèles de petite taille peuvent être analysés à condition d'avoir
les capacités de calcul et de stockage nécessaires,
mais la complexité croît trop vite avec la taille du modèle pour que
des réseaux de grande taille puissent être abordés par ces méthodes.

\todo{+ rapide retour sur les contributions}



\section{Les réseaux de régulation biologiques}

\todo{Explications à la Loïc sur :
\begin{itemize}
  \item la modélisation de la machine cellulaire
  \item les modèles discrets
  \item les modèles asynchrones
  \item Thomas + PH
\end{itemize}}

\stodo{[Thomas'73], [Paulevé'11 et '12]}



\section{Contributions}

Afin de répondre à cette double problématique,
nous proposons dans cette thèse plusieurs approches permettant de compléter l'expressivité
des Frappes de Processus, un formalisme récemment proposé par Loïc Paulevé dans \toref.
Les Frappes de Processus permettent de représenter la dynamique d'un ensemble de composants
en modélisant chacun d'eux par plusieurs niveaux locaux discrets.
De plus, les interactions entre composants sont limitées à une syntaxe précise,
en ne permettant la modification du niveau local d'un composant que par au plus un autre composant.
\todo{À déplacer dans la section précédente.}
Ce formalisme purement asynchrone possède plusieurs avantages dans la
représentation des réseaux de régulation biologiques :
il offre une expressivité proche des réseaux booléens asynchrones ou du modèle de Thomas,
il permet notamment de représenter une «~superposition~» des différentes portes logiques possibles
pour un réseau,
et offre une capacité d'analyse très efficaces à l'aide de méthodes d'approximation
qui évitent l'explosion combinatoire habituellement inhérente à l'étude de tels réseaux.
Cependant, son expressivité ne permet pas de représenter fidèlement les réseaux booléens,
et ne possède aucune caractéristique véritablement synchrone,
ce qui fait qu'en général, les modèles sont légèrement approximés.
Nous proposons dans cette thèse plusieurs formalismes alternatifs pour les Frappes de Processus
qui permettent de pallier ces manques d'expressivité de plusieurs manières différentes.
Nos introduisons pour cela trois notions différentes :
\begin{itemize}
  \item les classes de priorités permettent de définir des règles globales de prévalence
    entre des ensembles d'actions, afin d'affiner la dynamique et d'obtenir une expressivité
    équivalente à celle des réseaux booléens,
  \item les arcs neutralisants proposent de raffiner la notion précédente,
    et offrent un point de vue plus atomique sur les définitions des prévalences,
   \item les actions plurielles permettent enfin d'introduire des comportements synchrones
    entre les actions, dans le but de modéliser de façon exacte certaines classes
    de modèles construits à partir d'ensembles de réactions biochimiques instantanées.
\end{itemize}

Cependant, une plus grande puissance de formalisation ne revêt que peu d'intérêt si elle n'est
pas accompagnée d'une plus grande puissance d'analyse.
C'est pourquoi nous avons en parallèle développé des outils efficaces
afin d'étudier la dynamique des modèles produits dans ces formalismes.
Ces outils s'inspirent de l'analyse statique développée pour les Frappes de Processus
dans \toref,
et possèdent une complexité polynomiale dans la taille du modèles considéré.
Ils permettent d'étudier l'activation d'un niveau local d'un composant pour un état initial donné,
aussi appelée «~atteignabilité~» de ce niveau local,
et répondent de façon formelle.
Leur fonctionnement repose sur l'abstraction de la dynamique globale du système
au profit des dynamiques locales de chaque composant.
Cela leur permet de terminer très rapidement (au plus quelques dixièmes de secondes sur des
modèles de plusieurs centaines de composants),
mais au prix de ne pas toujours être en mesure de conclure après exécution.
Malgré cela, la totalité des atteignabilités testées dans le cadre de cette thèse ont été
conclusives, ce qui nous a permis de conclure notamment sur le bon fonctionnement
d'un modèle tiré d'un réseau booléen de 94 composants,
résultat à notre connaissance actuellement inégalé.



\section{Organisation du manuscrit}

\TODO
