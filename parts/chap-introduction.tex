% Introduction

\chapter{Introduction}

\section{Contexte \& Motivations}

Le principal défi posé par l'étude des systèmes dynamiques réels,
qu'ils soient biologiques ou non,
repose dans la modélisation qui en est faite.
Un modèle permet d'abstraire les comportements du système
pour s'intéresser uniquement à ceux qui présentent un intérêt à l'étude,
tout en permettant leur analyse à l'aide d'outils préalablement développés.

Un modèle doit donc dans l'idéal :
\begin{itemize}
  \item être cohérent avec la réalité du système qu'il représente,
  \item reproduire les comportements présentant un intérêt
    et abstraire ceux qui surchargent inutilement le modèle,
%   \item abstraire les comportements qui rendent la représentation et l'analyse inutilement
%     complexe,
  \item faciliter la lecture pour le modélisateur,
  \item permettre l'analyse par des outils appropriés,
  \item permettre la traduction vers d'autres formalismes
    ou l'ouverture à d'autres méthodes d'analyse.
\end{itemize}

Nous nous intéressons dans cette thèse aux propriétés dynamiques d'un modèle.
Elles se distinguent des propriétés statiques qui permettent de caractériser le modèle
en fonction de sa taille,
des liens entre les éléments qui le composent,
ou de toute autre propriété portant sur sa structure ou ses attributs,
bien qu'ils apportent parfois des résultats très généraux concernant sa dynamique.
À l'inverse, les propriétés dynamiques portent sur l'évolution et les comportements
possibles d'un modèles, par exemple :
\begin{itemize}
  \item Étant donné un certain état des entrées du système, le modèle est-il capable
    d'en reproduire ou d'en prédire les sorties ?
  \item Y retrouve-t-on des comportements qui s'apparentent à des oscillations
    ou des états stables ?
  \item Les comportements recherchés sont-ils accessibles depuis tous les états ?
    Si non, depuis lesquels ?
  \item Peut-on modifier le modèle pour voir apparaître un comportement donné, et comment ?
\end{itemize}
Répondre ces questions nécessite une analyse détaillée de la dynamique.

Ainsi, l'utilisation d'un modèle pose un double défi : sa conception et son analyse.

Il est nécessaire de proposer des outils permettant une modélisation cohérente et juste.
C'est pourquoi nous proposons dans cette thèse plusieurs formalismes nouveaux permettant
des représentations efficaces et complémentaires des systèmes dynamiques étudiés.
L'une des pistes d'enrichissement consiste pour cela
en l'introduction de contraintes dynamiques afin de filtrer les comportements non désirés,
par exemple sous la forme 
de relations de prévalence entre les différentes évolutions possibles d'un modèle.
D'autres pistes consistent à intégrer des paramètres temporels dans les modèles
en se basant sur des données de durée de réaction ou de sensibilisation.
Enfin, l'ajout d'outils de synchronisation à des formalismes purement asynchrones
permet d'enrichir leur sémantique.

% Des outils peuvent être proposés pour affiner la dynamique complète du modèle,
% afin d'éliminer des comportements non désirés.
% L'objectif est donc de pouvoir exprimer des contraintes supplémentaires
% qui ne l'étaient pas auparavant.
% Ainsi, il est possible par exemple
% d'élaborer un système décrivant une relation de prévalence entre les
% différentes évolutions d'un modèle.
% De cette manière, on peut parvenir à intégrer des notions temporelles dans les modèles,
% en se basant sur des données chronométriques.
% Il est aussi possible de pallier certains manques des modèles,
% par exemple en ajoutant des outils de synchronisme à des formalismes purement asynchrones.

Un autre problème courant est le lien entre plusieurs formalismes.
En effet, des formalismes différents peuvent permettre des approches variées,
et l'application d'outils adaptés à des problèmes précis.
Il s'avère donc nécessaire de pouvoir faire le lien entre différentes versions
d'un formalisme, ou encore de créer des ponts vers d'autres formalismes répandus
dans le cadre des systèmes dynamiques.
Aussi, cette thèse offre une part importante à l'étude des relations entre différentes
représentations complémentaires des modèles dynamiques étudiés,
ainsi qu'avec les modèles classiques pour ce type de représentations.

Cependant, vérifier de telles propriétés dynamiques possède un coût
en termes de temps d'exécution et de taille mémoire,
lié à la taille du modèle considéré.
Et même si, dans les meilleurs cas, ce coût varie de façon polynomiale
en fonction de la taille du modèle,
la plupart des méthodes formelles permettant de vérifier des propriétés dynamiques
font face à une explosion combinatoire qui empêche l'analyse de modèles de grande taille.
C'est pourquoi des méthodes alternatives peuvent être explorées,
comme la vérification par interprétation abstraite,
qui consiste à approcher la dynamique pour diminuer la complexité et simplifier les calculs,
comme nous le verrons dans la suite de ce manuscrit.

% ce qui permet d'analyser des modèles de très grande taille avec des ressources raisonnables.
% Cependant, il arrive que ces analyses soient de complexité exponentielle, voire plus,
% en fonction de la taille de modèle.
% Dans ce cas, des modèles de petite taille peuvent être analysés à condition d'avoir
% les capacités de calcul et de stockage nécessaires,
% mais la complexité croît trop vite avec la taille du modèle pour que
% des réseaux de grande taille puissent être abordés par ces méthodes.

%\todo{+ rapide retour sur les contributions}

La \secref{intro-rrb} propose une rapide vue d'ensemble des types de modèles permettant
la représentation et l'étude des systèmes d'interactions,
afin de mieux situer le travail proposé dans cette thèse,
dont les principaux résultats sont présentés à la section \secref{intro-contrib}.
La section \secref{intro-plan} présente l'organisation de ce manuscrit,
et la section \secref{notations} introduit les différentes notations qui y sont utilisées.



\section{Les réseaux de régulation biologiques}
\seclabel{intro-rrb}

\todo{Explications à la Loïc sur :
\begin{itemize}
  \item la modélisation de la machine cellulaire
  \item les modèles discrets
  \item les modèles asynchrones
  \item Thomas + PH
\end{itemize}}

\stodo{[Thomas'73], [Paulevé'11 et '12]}

L'étude de la machine cellulaire nécessite de s'intéresser aux éléments interagissant
qui la composent :
%des interactions entre les différents composants de la machine cellulaire
gènes, protéines, ARN messager, métabolites, etc.
% 
% Plusieurs approches permettent de représenter et d'analyser le fonctionnement de la machine
% cellulaire.
% L'une d'entre elles consiste à étudier les interactions qui y ont lieu
% entre les différents éléments qui s'y trouvent :
% gènes, protéines, ARN messager, métabolites, etc.
À ce niveau, il est déjà possible d'abstraire dans une certaine mesure
une partie de composants.
%les différents composants peuvent être plus ou moins abstraits.
C'est ainsi qu'un gène, la protéine qu'il code et l'ARN messager correspondant sont
souvent tous trois modélisés par un unique élément,
car la concentration de la protéine et de l'ARN
dépendent directement du niveau d'activation du gène.
Dans la suite nous nous intéresserons particulièrement à ce cas de figure,
et nous confondrons souvent une protéine avec son gène codant.
%bien que d'autres peuvent être représentés de la même manière.

Cette simplification permet de mettre en évidence les phénomènes d'interaction entre
les différents éléments entrant en jeu.
En effet, la présence d'un composant (protéine, catalyseur…)
en quantité suffisante peut déclencher
l'activation (une hausse d'activité) ou l'inhibition (une baisse de l'activité)
d'un ou plusieurs autres éléments, y compris l'élément déclencheur lui-même.
C'est par ce mécanisme qu'un gène, selon son degré d'activité,
permettra la production d'une concentration plus ou moins importante de la protéine qu'il code ;
celle-ci aura alors éventuellement un rôle activateur ou inhibiteur sur un certain
nombre d'autres gènes, et ainsi de suite.
L'ensemble de ces régulations forme une cascade
qui peut être représentée par un graphe des interactions,
où les composants sont représentés par des nœuds et leurs interactions mutuelles par des arcs.

% À ce niveau, les différents composants peuvent être plus ou moins abstraits.
% Par exemple, un gène, la protéine qu'il code et l'ARN messager correspondant sont
% souvent modélisés par un unique nœud,
% car la concentration de la protéine dépend directement du niveau d'activation du gène.
% Dans la suite nous nous intéresserons particulièrement à ce cas de figure,
% bien que d'autres peuvent être représentés de la même manière.

Afin de représenter la dynamique du modèle,
une valeur est associée à chaque élément afin de modéliser
son état courant
(son niveau d'activité pour un gène,
concentration dans le milieu pour une protéine, etc.).
% À chacun des éléments de modèle est alors associée une valeur modélisant
% son niveau d'activité ou
% sa concentration dans le milieu (pour une protéine).
Dans le cadre de formalismes utilisant des équations différentielles,
il s'agit d'une valeur continue,
ce qui permet par exemple de lier entre elles
les concentrations des différentes protéines et leurs dérivées par rapport au temps
(qui représentent leurs vitesses d'évolution). \toref
Cependant, le manque de données expérimentales précises et fiables peut limiter de telles
approches.
De plus, la résolution analytique ou numérique des équations différentielles est parfois
très complexe.

Une autre approche consiste donc à symboliser l'activité d'un élément par
une valeur discrète au sein d'un ensemble fini.
Cette approche avait été théorisée notamment par René Thomas \stodo{Thomas'73}
dans le cadre d'un formalisme booléen (restreint à deux niveaux discrets par composant),
qui a plus tard été étendu à un formalisme multivalué
(chaque élément pouvant avoir plus de deux niveaux d'expression).
Cette approche a l'avantage d'abstraire les valeurs des seuils de concentration,
qui sont généralement mal connues
mais qui permettent de représenter le niveau de concentration
à partir duquel la protéine va influencer un autre composant.
% quel niveau de concentration chaque composant
% possède une influence sur un autre.
Ainsi, à chaque niveau d'expression d'un composant est associé un ensemble de régulations
sur d'autres composants.

Le dernier élément permettant de caractériser la dynamique est l'ajout d'une
dimension temporelle.
Dans le cadre des réseaux de régulation discrets, ce temps prend la forme d'une série
infinie de pas de temps discrets permettant de représenter les évolutions
successives du modèles au cours du temps, sans indication des durées réelles
entre ces pas de temps.
La question du synchronisme des formalismes développés se pose alors :
en effet, s'il existe des systèmes purement synchrones, où entre chaque pas de temps,
toutes les réactions sensibilisées ont lieu,
l'hypothèse de René Thomas \toref
est au contraire d'abstraire les systèmes biologiques à l'aide
de réseaux de régulation purement asynchrones.
En effet, en l'absence de données temporelles du système étudié
(vitesse des réactions, durée des dégradations...)
il est plus raisonnable de considérer que deux réactions n'auront jamais lieu simultanément.

Ce sont sur les hypothèse précédentes qu'ont été formalisés
le modèle de Thomas \toref
et les Frappes de Processus \toref,
qui permettent la représentation des réseaux de régulation biologiques discrets et asynchrones.
Les deux permettent en effet de représenter les interactions entre différents composants
sous la forme de l'évolution séquentielle de niveaux d'expression discrets.
Cependant, ces deux formalismes se distinguent principalement au niveau de la représentation
des coopérations entre composants.
Ainsi, là où le modèle de Thomas fait entrer en jeu des paramètres pour représenter
les états focaux d'un élément en fonction de l'activité de ses régulateurs, \stodo{Ref Snoussi}
les Frappes de Processus introduisent un composant supplémentaire qui joue le rôle
de porte logique.

Cette thèse se concentrera principalement sur des formalisation alternatives des
Frappes de Processus dans le but d'enrichir celle-ci.
Il sera notamment question d'introduire des notions de prévalence entre les différentes
évolutions possibles selon les états du modèle, mais aussi d'intégrer
des comportements synchrones pour les modèles où cela s'avère nécessaire.
Ces ajouts permettent notamment de prendre en compte des paramètres qui était
auparavant abstraits, comme les durées relatives des réactions biochimiques,
tout en proposant des méthodes d'analyse plus efficaces
que les méthodes de vérification des modèles discrets ou continus.
Enfin, l'un des objectifs de cette thèse est d'examiner
les rapports entre cette représentation et le modèle de Thomas \TODO



\section{Contributions}
\seclabel{intro-contrib}

Afin de répondre à cette double problématique,
nous proposons dans cette thèse plusieurs approches permettant de compléter l'expressivité
des Frappes de Processus, un formalisme récemment proposé par Loïc Paulevé dans \toref.
Les Frappes de Processus permettent de représenter la dynamique d'un ensemble de composants
en modélisant chacun d'eux par plusieurs niveaux locaux discrets.
De plus, les interactions entre composants sont limitées à une syntaxe précise,
en ne permettant la modification du niveau local d'un composant que par au plus un autre composant.
\todo{À déplacer dans la section précédente.}
Ce formalisme purement asynchrone possède plusieurs avantages dans la
représentation des réseaux de régulation biologiques :
il offre une expressivité proche des réseaux booléens asynchrones ou du modèle de Thomas,
il permet notamment de représenter une «~superposition~» des différentes portes logiques possibles
pour un réseau,
et offre une capacité d'analyse très efficaces à l'aide de méthodes d'approximation
qui évitent l'explosion combinatoire habituellement inhérente à l'étude de tels réseaux.
Cependant, son expressivité ne permet pas de représenter fidèlement les réseaux booléens,
et ne possède aucune caractéristique véritablement synchrone,
ce qui fait qu'en général, les modèles sont légèrement approximés.
Nous proposons dans cette thèse plusieurs formalismes alternatifs pour les Frappes de Processus
qui permettent de pallier ces manques d'expressivité de plusieurs manières différentes.
Nos introduisons pour cela trois notions différentes :
\begin{itemize}
  \item les classes de priorités permettent de définir des règles globales de prévalence
    entre des ensembles d'actions, afin d'affiner la dynamique et d'obtenir une expressivité
    équivalente à celle des réseaux booléens,
  \item les arcs neutralisants proposent de raffiner la notion précédente,
    et offrent un point de vue plus atomique sur les définitions des prévalences,
   \item les actions plurielles permettent enfin d'introduire des comportements synchrones
    entre les actions, dans le but de modéliser de façon exacte certaines classes
    de modèles construits à partir d'ensembles de réactions biochimiques instantanées.
\end{itemize}

Cependant, une plus grande puissance de formalisation ne revêt que peu d'intérêt si elle n'est
pas accompagnée d'une plus grande puissance d'analyse.
C'est pourquoi nous avons en parallèle développé des outils efficaces
afin d'étudier la dynamique des modèles produits dans ces formalismes.
Ces outils s'inspirent de l'analyse statique développée pour les Frappes de Processus
dans \toref,
et possèdent une complexité polynomiale dans la taille du modèles considéré.
Ils permettent d'étudier l'activation d'un niveau local d'un composant pour un état initial donné,
aussi appelée «~atteignabilité~» de ce niveau local,
et répondent de façon formelle.
Leur fonctionnement repose sur l'abstraction de la dynamique globale du système
au profit des dynamiques locales de chaque composant.
Cela leur permet de terminer très rapidement (au plus quelques dixièmes de secondes sur des
modèles de plusieurs centaines de composants),
mais au prix de ne pas toujours être en mesure de conclure après exécution.
Malgré cela, la totalité des atteignabilités testées dans le cadre de cette thèse ont été
conclusives, ce qui nous a permis de conclure notamment sur le bon fonctionnement
d'un modèle tiré d'un réseau booléen de 94 composants,
résultat à notre connaissance actuellement inégalé.



\section{Organisation du manuscrit}
\seclabel{intro-plan}

\TODO
