% Introduction

\chapter{Introduction}

\section{Contexte \& Motivations}

Le principal défi posé par l'étude des systèmes dynamiques réels,
qu'ils soient biologiques ou non,
repose dans la modélisation qui en est faite.
Un modèle permet d'abstraire les comportements du système
pour s'intéresser uniquement à ceux qui présentent un intérêt à l'étude,
tout en permettant leur analyse à l'aide d'outils préalablement développés.

Un modèle doit donc dans l'idéal :
\begin{itemize}
  \item être cohérent avec la réalité du système qu'il représente,
  \item reproduire les comportements présentant un intérêt
    et abstraire ceux qui surchargent inutilement le modèle,
%   \item abstraire les comportements qui rendent la représentation et l'analyse inutilement
%     complexe,
  \item faciliter la lecture pour le modélisateur,
  \item permettre l'analyse par des outils appropriés,
  \item permettre la traduction depuis ou vers d'autres formalismes
    et favoriser l'ouverture à d'autres méthodes d'analyse.
\end{itemize}

Nous nous intéressons dans cette thèse aux propriétés dynamiques d'un modèle.
Elles se distinguent des propriétés statiques qui permettent de caractériser le modèle
en fonction de sa taille,
des liens entre les éléments qui le composent,
ou de toute autre propriété portant sur sa structure ou ses attributs,
bien que celles-ci apportent parfois aussi des résultats très généraux concernant la dynamique.
À l'inverse, les propriétés dynamiques portent sur l'évolution et les comportements
possibles d'un modèle, par exemple :
\begin{itemize}
  \item Étant donné un certain état des entrées du système, le modèle est-il capable
    d'en reproduire ou d'en prédire les sorties ?
  \item Y retrouve-t-on des comportements qui s'apparentent à des oscillations
    ou des états stables ?
  \item Les comportements recherchés sont-ils accessibles depuis tous les états ?
    Si non, depuis lesquels ?
  \item Peut-on modifier le modèle pour voir apparaître un comportement donné, et comment ?
\end{itemize}
Répondre à ces questions nécessite une analyse détaillée de la dynamique.

Ainsi, l'utilisation d'un modèle pose un double défi :
sa \bemph{conception} et son \bemph{analyse}.

Il est nécessaire de proposer des outils permettant une modélisation cohérente et juste.
C'est pourquoi nous proposons dans cette thèse plusieurs formalismes nouveaux permettant
des représentations efficaces et complémentaires des systèmes dynamiques étudiés.
L'une des pistes d'enrichissement consiste
en l'introduction de contraintes dynamiques afin de filtrer les comportements non désirés,
par exemple sous la forme 
de relations de prévalence entre les différentes évolutions possibles d'un modèle.
Cela ajoute de plus la possibilité de contraindre les comportements du modèle
en fonction de paramètres temporels issus
par exemple de données relatives aux durées de réaction ou de sensibilisation,
à des mesures de retards entre phénomènes biologiques, etc.
Enfin, l'ajout d'outils de synchronisation à des formalismes purement asynchrones
peut s'avérer nécessaire pour représenter certains comportements simultanés.

% Des outils peuvent être proposés pour affiner la dynamique complète du modèle,
% afin d'éliminer des comportements non désirés.
% L'objectif est donc de pouvoir exprimer des contraintes supplémentaires
% qui ne l'étaient pas auparavant.
% Ainsi, il est possible par exemple
% d'élaborer un système décrivant une relation de prévalence entre les
% différentes évolutions d'un modèle.
% De cette manière, on peut parvenir à intégrer des notions temporelles dans les modèles,
% en se basant sur des données chronométriques.
% Il est aussi possible de pallier certains manques des modèles,
% par exemple en ajoutant des outils de synchronisme à des formalismes purement asynchrones.

Un autre problème récurrent est de parvenir à faire le lien entre plusieurs formalismes.
En effet, des formalismes différents peuvent permettre des approches variées,
et l'application d'outils adaptés à des problèmes précis.
Il s'avère donc nécessaire de pouvoir faire le lien entre différentes versions
d'un formalisme, ou encore de créer des ponts vers d'autres formalismes répandus
dans le cadre des systèmes dynamiques.
Aussi, cette thèse offre une part importante à l'\bemph{étude des relations entre différentes
représentations} complémentaires des modèles dynamiques étudiés,
ainsi qu'avec les modèles classiques pour ce type de représentations.
L'intérêt d'une telle comparaison est de pouvoir mettre en valeur et d'exploiter
les avantages de chacune de ces approches.

Cependant, vérifier de telles propriétés dynamiques présente un coût
en termes de temps d'exécution et de taille mémoire,
lié à la taille du modèle considéré.
Et si, dans les meilleurs cas, ce coût varie de façon polynomiale
en fonction de la taille du modèle,
la plupart des méthodes formelles permettant de vérifier des propriétés dynamiques
font malheureusement face à une explosion combinatoire
qui empêche l'analyse de modèles de grande taille.
C'est pourquoi des méthodes alternatives peuvent être explorées,
comme la vérification par interprétation abstraite,
qui consiste à approcher de façon plus ou moins fine
la dynamique pour diminuer la complexité et simplifier les calculs.
Comme nous le verrons dans la suite de ce manuscrit,
il s'agit d'une des facettes de notre travail,
et nous soutenons que celle-ci constitue une contribution importante
pour l'étude des systèmes considérés.

% ce qui permet d'analyser des modèles de très grande taille avec des ressources raisonnables.
% Cependant, il arrive que ces analyses soient de complexité exponentielle, voire plus,
% en fonction de la taille de modèle.
% Dans ce cas, des modèles de petite taille peuvent être analysés à condition d'avoir
% les capacités de calcul et de stockage nécessaires,
% mais la complexité croît trop vite avec la taille du modèle pour que
% des réseaux de grande taille puissent être abordés par ces méthodes.

%\todo{+ rapide retour sur les contributions}

La \secref{intro-rrb} propose une rapide vue d'ensemble des types de modèles permettant
la représentation et l'étude des systèmes d'interactions,
afin de mieux situer le travail proposé dans cette thèse.
Les principaux résultats de notre travail sont ensuite présentés à la \secref{intro-contrib},
dans un cadre de collaborations précisées à la \secref{intro-collaborations}.
La \secref{intro-plan} présente la façon dont ce manuscrit est organisé,
et la \secref{notations} introduit les différentes notations qui y sont utilisées.



\section{Les réseaux de régulation biologique}
\seclabel{intro-rrb}

%\stodo{[Thomas'73], [Paulevé'11 et '12]}

\tocite[Citer Fages \& Soliman, 2008a ?]

\todoplustard{Ajouter des figures ?
\begin{itemize}
  \item Graphique figurant l'abstraction continu $\rightarrow$ discret
  \item Exemple de graphe des interactions
  \item Exemple de dynamique
  \item ... ?
\end{itemize}
OU :\\
Présenter comment le travail s'insère dans le domaine par rapport à d'autres formalismes
concernant :
\begin{itemize}
  \item les aspects temporels,
  \item l'abstraction,
  \item les grands systèmes,
  \item ...
\end{itemize}
}

L'étude de la machine cellulaire nécessite de s'intéresser aux éléments interagissant
qui la composent :
%des interactions entre les différents composants de la machine cellulaire
gènes, protéines, ARN messager, métabolites, etc.
% 
% Plusieurs approches permettent de représenter et d'analyser le fonctionnement de la machine
% cellulaire.
% L'une d'entre elles consiste à étudier les interactions qui y ont lieu
% entre les différents éléments qui s'y trouvent :
% gènes, protéines, ARN messager, métabolites, etc.
À ce niveau, il est déjà possible d'abstraire dans une certaine mesure
une partie des composants.
%les différents composants peuvent être plus ou moins abstraits.
C'est ainsi qu'un gène, la protéine qu'il code et l'ARN messager correspondant sont
souvent tous trois modélisés par un unique élément,
car la concentration de la protéine et de l'ARN
dépendent directement du niveau d'activation du gène.
Si cette simplification paraît poussée face à la réalité biologique qu'elle représente,
elle permet néanmoins d'abstraire de façon cohérente le mécanisme de création d'une protéine
pour s'intéresser aux relations
entre la présence d'une protéine et son influence sur la production d'une autre.
Comme ce cas de figure est particulièrement répandu dans l'étude des réseaux de régulation,
%Dans la suite nous nous intéresserons particulièrement à ce cas de figure,
nous assimilerons souvent dans la suite une protéine avec son gène codant.
%bien que d'autres peuvent être représentés de la même manière.

Cette simplification permet de mettre en évidence les phénomènes d'interaction entre
les différents éléments entrant en jeu.
En effet, la présence d'un composant (protéine, catalyseur...)
en quantité suffisante peut déclencher
l'activation (une hausse de l'activité) ou l'inhibition (une baisse de l'activité)
d'un ou plusieurs autres éléments, y compris l'élément déclencheur lui-même.
C'est par ce mécanisme que se crée une cascade de réactions entre gènes :
%C'est par ce mécanisme qu'un gène, selon son degré d'activité,
en effet, l'un d'eux, selon son degré d'activité,
permettra la production d'une concentration plus ou moins importante de la protéine qu'il code ;
celle-ci aura alors éventuellement un rôle activateur ou inhibiteur sur un certain
nombre d'autres gènes, et ainsi de suite.
L'ensemble de ces régulations peut être représenté par un graphe des interactions,
où les composants sont modélisés par des nœuds et leurs interactions mutuelles par des arcs.

% À ce niveau, les différents composants peuvent être plus ou moins abstraits.
% Par exemple, un gène, la protéine qu'il code et l'ARN messager correspondant sont
% souvent modélisés par un unique nœud,
% car la concentration de la protéine dépend directement du niveau d'activation du gène.
% Dans la suite nous nous intéresserons particulièrement à ce cas de figure,
% bien que d'autres peuvent être représentés de la même manière.

Afin de représenter la dynamique du modèle,
une valeur abstraite est associée à chaque élément afin de modéliser
son état courant
(niveau d'activité pour un gène,
concentration dans le milieu pour une protéine, etc.).
% À chacun des éléments de modèle est alors associée une valeur modélisant
% son niveau d'activité ou
% sa concentration dans le milieu (pour une protéine).
Dans le cadre de formalismes utilisant des équations différentielles
il s'agit d'une valeur continue,
ce qui permet par exemple de lier entre elles
les concentrations des différentes protéines et leurs dérivées par rapport au temps,
qui représentent alors leurs vitesses d'évolution \cite{tyson1978dynamics}.
Cependant, le manque de données expérimentales précises et fiables peut limiter de telles
approches.
De plus, la résolution analytique ou numérique des équations différentielles est parfois
très complexe, quand elle n'est pas impossible.

Une autre approche consiste donc à symboliser l'activité d'un élément par
une \bemph{valeur discrète} au sein d'un ensemble fini.
Elle se justifie par le fait que la courbe d'évolution de la concentration
d'une protéine forme généralement une sigmoïde lorsque sa production augmente ou diminue.
Cela avait été théorisé notamment par \citefullname{kauffman69}{Stuart A.}
puis par \citefullname{Thomas73}{René}
dans le cadre de formalismes booléens,
c'est-à-dire restreints à deux niveaux discrets par composant, généralement notés
«~0~» et «~1~».
Cette approche a naturellement été étendue par la suite à des formalismes multivalués,
où chaque élément peut posséder plus de deux niveaux d'expression,
généralement représentés par des entiers consécutifs.
Cette approche a l'avantage d'abstraire les valeurs des seuils de concentration,
qui sont généralement mal connues
mais qui permettent de représenter le niveau de concentration
à partir duquel la protéine va influencer un autre composant.
% quel niveau de concentration chaque composant
% possède une influence sur un autre.
Ainsi, à chaque niveau d'expression d'un composant est associé un ensemble de régulations
sur d'autres composants.

Le dernier élément permettant de caractériser la dynamique est l'ajout d'une
\bemph{dimension temporelle}.
Dans le cadre des réseaux de régulation discrets, ce temps prend la forme d'une série
infinie de pas de temps discrets permettant de représenter les évolutions
successives du modèle au cours du temps, sans indication des durées réelles
entre ces pas de temps.
La question du synchronisme des formalismes développés se pose alors :
en effet, s'il existe des systèmes purement synchrones, où entre chaque pas de temps,
toutes les réactions sensibilisées ont lieu,
l'hypothèse de \citefullname{Thomas73}{René}
est au contraire d'abstraire les systèmes biologiques à l'aide
de réseaux de régulation purement \bemph{asynchrones}.
En effet, en l'absence de données temporelles sur le système étudié
(vitesse des réactions, durée des dégradations...)
il n'est pas possible d'assurer que deux protéines dont la production est activée au même moment
seront produites en mêmes concentrations simultanément,
ou que leurs concentrations passeront en même temps
un seuil d'expression (représenté par un niveau discret).

C'est à partir de ces hypothèses
ayant pour but la représentation de réseaux de régulation biologique discrets et asynchrones
%permettant de représenter une abstraction des composants
%entrant en jeu à l'aide de niveau discrets
qu'a été formalisée
la version actuelle du modèle de Thomas \cite{Richard06},
et c'est en s'inspirant de celle-ci qu'ont été conçues par la suite
les Frappes de Processus \cite{PMR10-TCSB}.
%qui permettent toutes deux
Ces deux formalismes permettent en effet
de représenter les interactions entre différents composants
sous la forme de l'évolution séquentielle de niveaux d'expression discrets.
Cependant, ils se distinguent principalement au niveau de la représentation
des interactions entre composants.
En effet, le modèle de Thomas considère les interactions entre composants du point
de vue d'«~influences~» globales, c'est-à-dire d'interactions qui vont globalement avoir
le rôle d'activer ou inhiber les composants entre eux,
tandis que les Frappes de Processus font usage d'«~actions~»,
qui décrivent le saut d'un état local d'un composant à un autre.
De plus, une restriction particulière porte sur ces actions,
qui ne permettent la modification du niveau local d'un composant
que par au plus un autre composant.
Cette différence de point de vue en implique une autre en ce qui concerne la représentation
des coopérations entre composants :
là où le modèle de Thomas fait usage de paramètres décrivant
les états focaux d'un élément en fonction de l'activité de l'ensemble de ses régulateurs
\cite{Snoussi89},
les Frappes de Processus introduisent un composant supplémentaire propre à la modélisation
et qui joue le rôle de porte logique.

Les Frappes de Processus ont fait précédemment l'objet de plusieurs travaux.
Ceux-ci ont notamment permis de montrer qu'elles permettent de représenter une «~superposition~»
de modèles de Thomas, afin de représenter des ensembles de paramètres partiellement connus.
De plus, un travail approfondi a permis d'élaborer de puissantes analyses statiques
portant sur la recherche d'états stables,
mais aussi sur des questions dynamiques comme l'atteignabilité d'un état local \cite{PMR12-MSCS}
qui peut être traitée de façon très efficace, et donc \bemph{appliquée à de très grands modèles}.

\todo{Mentionner et se positionner par rapport au travail de Vincent Danos ($\kappa$ calcul).}

\todo{Se positionner par rapport aux réseaux de Petri (cf. Heiner, Gilbert, Chaouiya).}



\section{Contributions}
\seclabel{intro-contrib}

Les apports de cette thèse se déclinent en trois points principaux :
\begin{itemize}
  \item l'enrichissement des Frappes de Processus par l'introduction de notions
    de prévalence et de synchronisme entre les différentes évolutions possibles ;
  \item le développement de méthodes d'analyse de la dynamique efficaces
    et adaptées aux enrichissements précédemment mentionnés,
    et leur application concluante à des réseaux de régulation biologique de grande taille ;
  \item la description des liens formels entre les différentes sémantiques de Frappes de Processus
    proposées, ainsi qu'avec d'autres formalismes courants pour la représentation
    des réseaux de régulation biologique.
\end{itemize}
L'objectif principal de cette thèse est donc de répondre à la double problématique
de la conception et de l'analyse d'un modèle,
en nous concentrant particulièrement sur l'utilisation et l'amélioration
des Frappes de Processus et des outils associés.
Les trois aspects précédents sont détaillés dans la suite
afin d'insister sur leur rôle dans cette démarche.

\subsubsection*{Enrichissement des Frappes de Processus}

Cette thèse se concentre sur plusieurs \bemph{alternatives d'extension du formalisme des
Frappes de Processus dans le but d'enrichir celles-ci}.
L'un des objectifs de cet enrichissement est \emph{la prise en compte de données temporelles},
qui étaient généralement abstraites auparavant,
comme des durées relatives de réactions biochimiques
ou des notions de retard entre les évolutions de certains éléments.
afin d'étudier leur influence sur la dynamique.
Cependant, plutôt que d'intégrer des données continues dans le modèle,
nous proposons de traduire ces données par des formes nouvelles de dynamique.
Pour cela, nous proposons trois approches qui gravitent autour des notions
de \bemph{préemption} et de \bemph{synchronisme} entre les différentes
évolutions possibles du modèle :
\begin{itemize}
  \item les \bemph{classes de priorités} permettent de définir des règles globales de prévalence
    entre des ensembles d'actions, afin d'affiner la dynamique et d'obtenir une expressivité
    équivalente à celle des réseaux booléens,
  \item les \bemph{arcs neutralisants} proposent de raffiner la notion précédente,
    en définissant les prévalences de façon plus atomique entre les actions individuelles,
  \item les \bemph{actions plurielles}, enfin, permettent d'introduire des comportements
    synchrones entre les actions, dans le but de modéliser des phénomènes simultanés
    comme la création des produits d'une réaction.
\end{itemize}
Nous montrons par ailleurs comment les différentes données temporelles peuvent être
intégrées à l'aide de ces nouveaux outils.

\subsubsection*{Outils efficaces d'analyse de la dynamique}

Nous développons par ailleurs des méthodes permettant d'analyser la dynamique
des modélisations proposées ci-dessus.
Ces méthodes sont basées sur de l'analyse statique par interprétation abstraite
précédemment proposée par \citeasnoun{PMR12-MSCS} :
%c'est-à-dire permettant d'approcher la dynamique afin de réduire la complexité de l'analyse.
leur fonctionnement repose sur l'abstraction de la dynamique globale du système
au profit des dynamiques locales de chaque composant.
Nous enrichissons cette approche afin d'adapter les méthodes développées
aux nouvelles formes de dynamique que nous proposons.
Elles ont l'avantage de posséder \bemph{une complexité polynomiale en la taille du modèle},
ce qui permet de \bemph{traiter efficacement de très grands modèles,
de l'ordre de centaines de composants, en quelques dixièmes de seconde}.
Néanmoins, étant donné qu'elles sont basées sur une approximation de la dynamique,
il est possible qu'elles terminent sans pouvoir conclure,
bien que cela ne soit jamais arrivé sur les exemples testés.
%notamment ceux présentés dans cette thèse.
Nous appliquons notamment ces méthodes à un réseau booléen de quatre-vingt-quatorze (94) composants
préalablement traduit en Frappes de Processus,
et nous montrons que cela permet de conclure
en moins d'une seconde sur plusieurs questions dynamiques.

\subsubsection*{Équivalences entre les Frappes de Processus}

L'analyse statique mentionnée précédemment nécessite l'utilisation
de \bemph{Frappes de Processus canoniques},
une classe particulière
consistant en une restriction de l'un des formalismes mentionnés précédemment.
Celle classe présente d'autres intérêts :
nous montrons notamment qu'elle est
\bemph{aussi expressive que tous les formalismes
de Frappes de Processus} développés dans cette thèse,
et nous donnons les traductions correspondantes.
Cela assure notamment qu'il est toujours possible de naviguer entre les formalismes,
afin de profiter de leurs avantages respectifs,
et notamment de l'utilisation des analyses statiques mentionnées ci-dessus
à tous les autres types de Frappes de Processus, moyennant une traduction du modèle.

\subsubsection*{Équivalences avec d'autres formalismes}

Enfin, nous nous intéressons également aux \bemph{liens formels entre les différentes
sémantiques de Frappes de Processus
et les autres modélisations classiques pour la représentation des réseaux de régulation
biologique}.
Nous nous intéressons notamment au \bemph{modèle de Thomas} et aux \bemph{réseaux booléens},
deux formalismes proches
développés spécifiquement pour la représentation de réseaux de régulations,
et nous montrons qu'il est possible de représenter ces deux formalismes de façon exacte
grâce à l'ajout de priorités dans les Frappes de Processus.
Nous abordons aussi la question de la traduction vers les \bemph{réseaux de Petri},
un formalisme plus généraliste mais qui est l'objet d'un intérêt important du fait des
capacités d'expressivité et d'analyse qu'il offre,
et les intérêts qu'il présente aussi pour la modélisation des réseaux biologiques
\cite{petri1962kommunikation,Chaouiya07petrinet}.
Ces liens permettent d'une part de comprendre la position des Frappes de Processus au sein
de l'ensemble des modélisations discrètes asynchrones,
et offrent d'autre part des outils de traduction depuis et vers ces autres formalismes,
donnant la possibilité d'utiliser les outils d'analyse spécifiquement développés pour ceux-ci.

\todo{Citer : Heiner, Gilbert, Chaouiya}



% Collaborations
% Collaborations

\section{Collaborations}
\seclabel{intro-collaborations}

Une partie du travail de cette thèse a été réalisé dans le cadre d'un stage doctoral
dans l'équipe de recherche de Katsumi Inoue,
au National Institute of Informatics (Tokyo, Japon).
Le sujet de ce stage était :
«~Raisonnement automatique et recherche d'hypothèses pour la biologie des systèmes.~»
Ont ainsi participé financièrement à ce travail le National Institute of Informatics,
\textit{via} l'Inoue Laboratory,
ainsi que la Fondation Centrale Initiatives.

Ce travail s'inscrit par ailleurs dans le projet de recherche ANR blanc BioTempo%
\footnote{Le site du projet est disponible à \url{http://biotempo.genouest.org/}}
dont l'intitulé était : «~Représentations à l'aide de langage, de temps et de modèles hybrides
pour l'analyse de modèles incomplets en biologie moléculaire~»,
et qui s'est étendu de mars 2011 à août 2014.
Le présent travail répond notamment à certains objectifs de la tâche 3 du projet,
qui concerne l'introduction de synchronisations et de données chronométriques
dans les modèles chronologiques.




\section{Organisation du manuscrit}
\seclabel{intro-plan}

Le présent manuscrit est organisé de la manière suivante.

Le \chapref{etatdelart} offre un état des lieux en matière de modélisation et d'analyse
des réseaux de régulation biologique discrets et asynchrones.
Nous nous intéresserons notamment aux réseaux booléens et au modèle de Thomas,
deux types de réseaux très répandus pour des raisons historiques
comme nous l'avons déjà mentionné.
Nous définirons aussi les Frappes de Processus standards,
c'est-à-dire telles que définies par le travail de \citefullname{Pauleve11}{Loïc},
car elles constituent le point de départ de ce travail.

Le \chapref{semantiques} propose plusieurs ajouts aux Frappes de Processus afin de l'enrichir
et d'en augmenter l'expressivité.
Les trois formalismes abordés reposent sur les notions de classes de priorités,
d'arcs neutralisants et d'actions plurielles,
afin de filtrer les comportements désirés du modèle
ou d'ajouter des comportements nécessaires à la modélisation de certains phénomènes.
Nous discutons aussi dans ce chapitre des applications possibles de ces formalismes,
et traitons en partie la question des équivalences et des traductions entre ces formalismes.

Le \chapref{as} se concentre sur une classe particulière de Frappes de Processus avec classes
de priorités, dite «~canonique~».
Cette classe sert de base au développement de méthodes d'analyse statique permettant
de vérifier des propriétés d'atteignabilité locales dans un modèle.
Ces méthodes utilisent un mécanisme d'abstraction pour éviter l'explosion combinatoire
inhérente à toute étude de la dynamique d'un modèle discret, et ainsi rester efficaces.
Nous montrons aussi, traduction à l'appui,
que les Frappes de Processus canoniques sont en réalité équivalentes
aux trois extensions des Frappes de Processus proposées,
ce qui permet d'étendre la portée des résultats précédents à ces formalismes.

Le \chapref{expressivite} dépasse le cadre des Frappes de Processus et s'intéresse aux liens
formels entre les extensions proposées plus haut et plusieurs formalismes classiques discrets
permettant la modélisation des réseaux de régulation biologique.
Nous y prouvons que les extensions des Frappes de Processus permettent d'obtenir la même
expressivité que le modèle de Thomas et les réseaux booléens,
ce qui étend encore la portée de nos résultats.
Nous montrons aussi que les Frappes de Processus sont aussi expressives que le
formalisme classique des automates synchronisés.
Enfin, nous proposons une traduction vers les réseaux de Petri
avec arcs de lecture et arcs inhibiteurs,
et, à l'inverse, une traduction depuis le systèmes d'équations biochimiques
de la sémantique booléenne de Biocham.

Le \chapref{applications} porte sur l'application des modèles et méthodes proposées
dans cette thèse à des réseaux de régulation biologique de grande taille.
Nous y présentons plusieurs systèmes biologiques classiquement étudiés,
la façon dont ils sont représentés en Frappes de Processus,
et les résultats de nos méthodes de traduction et d'analyse.

Enfin, le \chapref{conclusion} nous permet de conclure ce manuscrit
et d'en discuter les résultats afin d'ouvrir sur des perspectives de travaux futurs.
