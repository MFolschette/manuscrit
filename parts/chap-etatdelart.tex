% État de l'art de la modélisation

\chapter{État de l'art de la modélisation}
\chaplabel{etatdelart}

\chapeau{%
  Nous proposons dans ce chapitre un état de l'art des formalisations discrètes
  et asynchrones des réseaux de régulation biologique utilisées dans cette thèse.
  Nous y rappelons le principe et la définition du \emph{modèle de Thomas}
  et nous revenons brièvement sur les méthodes d'analyse qui existent sur ce modèle.
  Nous rappelons aussi le formalisme des \emph{Frappes de Processus standards},
  qui seront la base des travaux présentés dans cette thèse,
  et nous faisons une rapide revue des principaux travaux qui ont concerné ce formalisme.
}



Les outils de modélisation des réseaux de régulation biologique se déclinent en de nombreuses
formes permettant de représenter différents comportement des systèmes étudiés.
Au cours de cette thèse, nous nous intéressons tout particulièrement aux modèles discrets
et asynchrones.

Les modèles discrets permettent de représenter des systèmes plus complexes,
comme des systèmes d'équations différentielles,
en abstrayant une partie de la dynamique.
Cette abstraction permet de simplifier le modèle pour en faciliter l'analyse,
à condition de rester cohérente avec la représentation initiale.
Le caractère asynchrone des formalismes vient quant à lui de la constatation suivante :
il est biologiquement très improbable que plusieurs entités d'un système
évoluent en parfaite simultanéité.
Le parallèle avec les systèmes d'équations différentielles est le suivant :
il est rare d'observer plusieurs composants passer un seuil simultanément
au cours d'une évolution continue de leur état.
Ces hypothèses ont notamment été théorisées par \citefullname{Thomas73}{René},
qui en a dérivé le modèle qui porte son nom,
plus tard enrichi par plusieurs travaux successifs
comme l'ajout de paramètres discrets par \citefullname{Snoussi89}{El Houssine}
pour représenter les «~états focaux~» des différents
composants en fonction de l'état du modèle.

Dans ce chapitre, nous rappellerons tout d'abord la définition du modèle de Thomas
à la \secref{thomas}.
Nous dresserons par ailleurs un rapide état de l'art des méthodes d'analyse
de la dynamique de ce modèle qui, bien que facilitées par
l'abstraction discrète et asynchrone qui en est inhérente,
restent l'objet d'une explosion combinatoire non négligeable.
Plusieurs travaux apportent des résultats qui s'appuient uniquement sur les données
du modèles et non sur le calcul de sa dynamique,
ce qui évite de calculer celle-ci explicitement.
Cependant, ces résultats sont très généraux
car ils se focalisent le plus souvent sur la présence ou l'absence de comportements
dans le modèle (comme des oscillations ou des points fixes).
Il est donc rare de pouvoir totalement faire l'impasse
sur une analyse plus détaillée de la dynamique,
qui nécessite cependant de calculer l'espace des états du modèle,
ou une version compressée de celui-ci.

Nous définirons ensuite le formalisme des Frappes de Processus
à la \secref{ph}, tel qu'il a été proposé par \citeasnoun{PMR10-TCSB}.
Il a été conçu comme une alternative complémentaire au modèle de Thomas
avec lequel il partage certaines hypothèses, à savoir la discrétisation des états
et l'asynchronisme de la dynamique,
%basé sur certaines hypothèse les mêmes hypothèses d'états discrets et d'asynchronisme,
bien qu'il soit plus atomique par ailleurs dans sa représentation.
Il permet notamment de modéliser tout modèle de Thomas,
avec cependant une légère sur-approximation de la dynamique.
Bien que cette sur-approximation permette de représenter une classe de modèles de façon
abstraite, elle est cependant indésirable lorsque l'on souhaite modéliser
fidèlement certains comportements.
Des outils ont de surcroît été développés afin de calculer les points fixes d'un modèle de
Frappes de Processus ou encore d'y intégrer des paramètres continus
sous forme de probabilités.
Cependant, l'atout principal des Frappes de Processus réside dans les puissantes
méthodes d'analyse statique par interprétation abstraite qui y sont associées,
permettant ainsi d'effectuer des calculs d'atteignabilité locale
avec une complexité polynomiale dans la taille du modèle considéré.
Cela permet notamment d'étudier la dynamique de grands modèles
---~jusqu'à plusieurs centaines de composants, voire au-delà.

\todoplustard{Processus de modélisation des biologistes :
\begin{itemize}
  \item types de données en entrée,
  \item type de questions pour lesquelles ils attendent des résultats,
  \item repositionnement par rapport à cela.
\end{itemize}}



% Le modèle de Thomas
% Le modèle de Thomas

\begin{comment}
The expression of fixed points in a given regulatory network is potentially one of the most crucial behaviors to include in a model and is often the 
aspect of greatest interest for a biologist. For this reason, a great amount of research has gone toward the topic of fixed point analysis. We have 
already mentioned that a positive circuit is a proven necessary condition for multiple fixed points, as shown in~\cite{Richard2010378}, but this does 
not speak to the exact number of fixed points present in the system. In the field of Boolean networks, analysis of the interaction graph can give an 
upper bound of the possible number of fixed points~\cite{aracena2008maximum} and the topological fixed points independent of logical 
functions~\cite{PR10-CRAS}, but these still do not give a complete enumeration.
In \cite{Naldi07}, the authors propose an efficient method for enumerating all the fixed point by relying on the
encoding of logical functions using decision diagrams.
\end{comment}


% Les Frappes de Processus
\section{Les Frappes de Processus standard}

\subsection{Définition}
\seclabel{ph}

Les Frappes de Processus permettent une modélisation atomique des interactions entre composants.
\Defref{ph}
\todo{Description}

\begin{definition}[Frappes de Processus]
\deflabel{ph}
  Les \emph{Frappes de Processus} sont définies
  par un triplet $\PH = (\PHs; \PHl; \PHh)$, où :
  \begin{itemize}
    \item $\PHs \DEF \{a, b, \dots\}$ est l'ensemble fini et dénombrable des \emph{sortes} ;
    \item $\PHl \DEF \underset{a \in \PHs}{\times} \PHl_a$ est l'ensemble fini des \emph{états},
      où $\PHl_a = \{a_0, \ldots, a_{l_a}\}$ est l'ensemble fini et dénombrable
      des \emph{processus} de la sorte $a \in \PHs$ et $l_a \in \sN^*$.
      Chaque processus appartient à une unique sorte :
      $\forall (a_i; b_j) \in \PHl_a \times \PHl_b, a \neq b \Rightarrow a_i \neq b_j$ ;
    \item $\PHh \DEF \{\PHfrappe{a_i}{b_j}{b_l} \mid (a; b) \in \PHs \times \PHs \wedge
      (a_i; b_j; b_l) \in \PHl_a \times \PHl_b \times \PHl_b \wedge
      b_j \neq b_l \wedge a = b \Rightarrow a_i = b_j \}$ est l'ensemble fini des \emph{actions}.
  \end{itemize}
\end{definition}
%
\noindent
On note $\PHproc \DEF \bigcup_{a \in \PHs} \PHl_a$ l'ensemble de tous les processus.
La sorte d'un processus $a_i$ est donnée par $\PHsort(a_i) = a$.
Étant donné un état $s \in \PHl$, le processus de la sorte $a \in \PHs$ présent dans $s$ est donné
par $\PHget{s}{a}$, \cad la coordonnée correspondant à $a$ dans l'état $s$.
Si $a_i \in \PHl_a$, nous définissons la notation : $a_i \in s \EQDEF \PHget{s}{a} = a_i$.
Pour toute action $h = \PHfrappe{a_i}{b_j}{b_k} \in \PHh$,
$a_i$ est appelé le \emph{frappeur}, $b_j$ la \emph{cible} et $b_k$ le \emph{bond} de $h$,
et on note : $\hitter{h} = a_i$, $\target{h} = b_j$ et $\bounce{h} = b_k$.

La \defref{substate} établit la notion de sous-état sur un ensemble de sortes,
\cad un ensemble de processus qui sont deux à deux de sortes différentes,
ce qui permet de ne considérer qu'une partie d'un état complet.
Nous notons qu'un état est \textit{a fortiori} un sous-état : $\PHl \subset \PHsubl$.
Le recouvrement d'un état $s$ par un processus $a_i$ est formalisé à la \defref{statecap}
par un état identique à $s$, sauf pour le processus de $a$ qui a été remplacé par $a_i$,
ce qui permet de définir la dynamique des Frappes de Processus avec $k$ classes de priorités
dans la \defref{play}.
La définition de recouvrement est aussi étendue à un sous-ensemble,
autrement dit, un ensemble de processus contenant au plus un processus par sorte.

\begin{definition}[Sous-états ($\PHsublize{\PHl}$)]
\deflabel{substate}
  Si $S \subset \PHs$ est un ensemble de sortes, un sous-ensemble sur $S$ est un élément de :
  $\PHsubl[\PHl]_S \DEF \bigtimes{a \in S} \PHl_a$.
  L'ensemble de tous les sous-ensembles est noté :
  $\PHsubl[\PHl] \DEF \bigcup_{S \in\powerset(\PHs)} \PHsubl[\PHl]_S$.
  
  \noindent
  De plus, si $\mysigma \in \PHsubl[\PHl]$ et $s \in \PHl$, on note alors :
  \[\mysigma \subseteq s \EQDEF \forall a_i \in \Proc, a_i \in \mysigma \Rightarrow a_i \in s
    \enspace.\]
  
  \noindent
  Enfin, si $S \subset \PHs$, on note :
  $\PHsublset_S = \{ \toset{ps} \subset \Proc \mid ps \in \PHsubl_S \}$
  et
  $\PHsublset = \{ \toset{ps} \subset \Proc \mid ps \in \PHsubl \}$.
\end{definition}
%
\begin{definition}[$\Cap : \PHl \times \PHproc \rightarrow \PHl$]
\deflabel{statecap}
  Étant donné un état $s \in \PHl$ et un processus $a_i \in \PHproc$,
  $(s \Cap a_i)$ est l'état défini par :
  $\PHget{(s \Cap a_i)}{a} = a_i \wedge \forall b \neq a, \PHget{(s \Cap a_i)}{b} = \PHget{s}{b}$.
  On étend de plus cette définition à un tout ensemble de processus,
  à condition que tous les processus soient tous de sortes différentes,
  par le recouvrement de l'état par chaque processus :
  $\forall ps \in \PHsubl[\PHl], s \Cap \toset{ps} = s \underset{a_i \in \toset{ps}}{\Cap} a_i$.
\end{definition}

Une propriété de jouabilité telle que décrite à la \defref{ppl}
est équivalente à une formule booléenne dont les atomes sont des processus dans $\Proc$.
Le langage des propriétés de jouabilité permet de décrire la présence d'une configuration
de processus actifs dans un état donné.
Il permet notamment de décrire en termes formels la jouabilité d'une action,
ce qui est immédiatement mis en pratique dans la \defref{fopph}.

\begin{definition}[Propriété de jouabilité ($\F$)]
  \label{def:ppl}
  Une \emph{propriété de jouabilité} est un élément du langage $\F$ défini \todo{inductivement ?} par :
  \begin{itemize}
    \item $\top$ et $\bot$ appartiennent à $\F$ ;
    \item si $a \in \PHs$ et $a_i \in \PHl_a$, alors $a_i$ appartient à $\F$
      et est appelé un \emph{atome} ;
    \item si $P \in \F$ et $Q \in \F$,
      alors $\neg P$, $P \wedge Q$ et $P \vee Q$ appartiennent à $\F$.
  \end{itemize}
%
  Si $P \in \F$ est une propriété de jouabilité et $\mysigma \in \PHsubl$ est un sous-ensemble,
  on note $\Feval{P}{\mysigma}$ l'\emph{évaluation} de $P$ dans $\mysigma$:
  \begin{itemize}
    \item si $P = a_i \in \PHl_a$ est un atome, avec $a \in \PHs$,
      alors $\Feval{a_i}{\mysigma}$ est vraie si et seulement si $a_i \in \mysigma$ ;
    \item si $P$ n'est pas un atome, alors $\Feval{P}{\mysigma}$ est vraie si et seulement si
      on peut l'évaluer récursivement comme vraie en utilisant la sémantique habituelle des
      opérateurs $\neg$, $\wedge$ et $\vee$ et des constantes $\top$ et $\bot$.
  \end{itemize}
%
  Une fonction $\Fopsymbol : \PHh \rightarrow \F$ associant à toute action une propriété de jouabilité
  est appelée un \emph{opérateur de jouabilité}.
\end{definition}

Étant donné que ce langage n'utilise que des opérateur logiques classiques,
les propriétés de la logique booléenne sont applicables aux propriétés de jouabilité,
à savoir celles concernant la distributivité, l'associativité et la commutativité,
ainsi que les lois de De Morgan concernant la négation.

Il en résulte notamment la propriété suivante, permettant d'évaluer la négation d'un atome,
et qui dérive naturellement du fait que si un processus n'est pas actif dans un état donné,
cela signifie alors qu'un autre processus de la même sorte l'est :
\[\forall a \in \PHs, \forall a_i \in \PHl_a, \forall \mysigma \in \PHsubl,
  \Feval{\neg a_i}{\mysigma} \Leftrightarrow
  \Feval{\bigvee_{\substack{a_j \in \PHl_a\\a_j \neq a_i}} a_j}{\mysigma}\]

Enfin, on note dans la suite :
\[\forall A \in \PHsublset, \Fconj{A} \equiv \bigwedge_{p \in A} p \enspace.\]

\todo{Gluer}

\begin{definition}[Opérateur de jouabilité ($\Fopsymbol : \PHh \rightarrow \F$)]
\deflabel{fopph}
  L'opérateur de jouabilité des Frappes de Processus est défini par :
  \[\forall h \in \PHh, \Fop{h} \equiv \hitter{h} \wedge \target{h} \enspace.\]
\end{definition}

\begin{definition}[Dynamique des Frappes de Processus ($\PHtrans$)]
\deflabel{play}
  Une action $h \in \PHh$ est dite \emph{jouable}
  dans l'état $s \in \PHl$ si et seulement si :
  $\Feval{\Fop{h}}{s}$.
%  $\target{h} \in s \wedge \Feval{\Fop{h}}{s}$.
  Dans ce cas, $(s \PHplay h)$ est l'état résultant du jeu de l'action $h$ dans $s$,
  et on le définit par : $(s \PHplay h) = s \Cap \bounce{h}$.
  De plus, on note alors : $s \PHtrans (s \PHplay h)$.

  Si $s \in \PHl$, un \emph{scénario} $\delta$ dans $s$
  est une séquence d'actions de $\PHh$ qui peuvent être jouées successivement dans $s$.
  L'ensemble de tous les scénarios dans $s$ est noté $\Sce(s)$.
\end{definition}



\subsection{Analyse statique}

\todo{Rappels sur les résultats d'analyse statique (cf. MSCS'10)}

