
\section{Analyse statique}
\seclabel{as}

\todo{Revoir la glu}

L'objectif de cette section est de définir le problème de l'\emph{atteignabilité} dans des
Frappes de Processus,
aussi appelée \emph{accessibilité},
et de proposer une sous-approximation permettant de la résoudre efficacement
dans les Frappes de Processus canoniques.

La méthode développée ici est inspirée de \stodo{PMR12-MSCS}.  %~\cite{PMR12-MSCS}.

Le problème de l'atteignabilité dans les Frappes de Processus consiste à rechercher l'existence
d'un scénario qui permette d'activer un ou plusieurs processus donné(s).
Il peut se résumer à la question suivante :
«~Étant donné un état initial, existe-t-il un scénario partant de cet état et
qui permette d'activer un processus donné ?~»
ou, de façon plus générale pour un ensemble de processus :
«~Étant donné un état initial $\ctx$ et une séquence de processus $\w$ donnés,
existe-t-il un scénario $\delta$ jouable dans $\ctx$ et permettant d'activer successivement
chacun des processus de $\w$ dans l'ordre ?~»

La résolution du problème de l'atteignabilité peut parfois être résolu à l'aide des outils
de model checking classiques.
Cependant, de telles méthodes reposent généralement sur l'analyse de la dynamique du modèle.
Pour de grands modèles, ces méthodes se heurtent donc à l'explosion combinatoire inhérente
au calcul du graphe des états.
La méthode proposée ici repose en revanche sur une sous-approximation du modèle analysé.
Cela permet d'éviter une complexité exponentielle
et d'avoir à la place un problème de complexité polynomiale
sous la condition que chaque sorte du modèle possède un nombre restreint de processus
(une sorte de quatre processus ou moins satisfaisant ce critère).

On considère dans cette section un modèle de Frappes de Processus canoniques
$\PH = (\PHs; \PHl; (\PHa^{(1)};\PHa^{(2)}))$
tel que défini à la \defref{phcanonique}.






\subsection{Définitions préliminaires}
\seclabel{sa-defs}

L'atteignabilité d'un processus $a_j$ d'une sorte donnée $a$ depuis un autre processus $a_i$
de la même sorte est le fait, depuis un état où $a_i$ est actif,
de pouvoir jouer scénario menant dans un état où $a_j$ est actif.
La question de l'existence d'un tel scénario se représente sous la forme d'un \emph{objectif},
noté $\PHobjp{a}{i}{j}$ (\defref{obj}).
De plus, une \emph{séquence d'objectifs} est une séquence d'éléments de $\Obj$ telle que
la cible de chaque objectif est égale au bond de l'objectif précédent de la même sorte
dans la séquence, s'il existe (\defref{os}).

\begin{definition}[Objectif ($\Obj$)]
\deflabel{obj}
  Si $a \in \components$, l'atteignabilité d'un processus $a_j$ depuis un processus $a_i$
  est appelé un \emph{objectif}, noté $\PHobj{a_i}{a_j}$.
  L'ensemble de tous les objectifs est noté
  $\Obj \DEF \{ \PHobj{a_i}{a_j} \mid
    a \in \components \wedge (a_i, a_j) \in \PHl_a \times \PHl_a \}$.
  Pour tout objectif $P = \PHobj{a_i}{a_j} \in \Obj$, on note
  $\sort{P} \DEF a$ la sorte de l'objectif $P$,
  $\target{P} \DEF a_i$ sa cible et $\bounce{P} \DEF a_j$ son bond.
  Enfin, $P$ est dit \emph{trivial} si $a_i = a_j$.
\end{definition}

\begin{definition}[Séquence d'objectif ($\Obj$)]
\deflabel{os}
  Une \emph{séquence d'objectifs} est une séquence $\w = P_1 \cons \ldots \cons P_{\card{\w}}$,
  où $\forall n \in \indexes{\w}, \w_n \in \Obj$
  et $\cible{\w_n} = a_i \Rightarrow \der{a}{\w_{1 \ldots n-1}} \in \{ \varnothing, a_i \}$.
  L'ensemble des séquences d'objectifs est référé par $\OS$.
%  Les définitions de $\premsymbol_a$ (\eqref{prem}), $\dersymbol_a$ (\eqref{der}),
%  $\suppsymbol$ (\eqref{supp}) et $\finsymbol$ (\eqref{fin}) sont étendues aux
  Les définitions de $\premsymbol_a$, $\dersymbol_a$,  $\suppsymbol$ et $\finsymbol$
  (\defref{premder}) sont étendues aux séquences d'objectifs en omettant le cas des frappeurs.
\end{definition}

La \defref{ctx} introduit la notion de \emph{contexte} qui étend celle d'état
afin de pouvoir représenter un ensemble d'états initiaux possibles :
plutôt que d'attribuer un seul processus actif à chaque sorte, comme pour un état,
un contexte permet d'en attribuer plusieurs.
La notion de recouvrement est aussi étendue aux contextes dans la \defref{ctxrecouvrement}.

\begin{definition}[Contexte ($\Ctx$)]
\deflabel{ctx}
  Un \emph{contexte} $\ctx$ associe à chaque sorte dans $\PHs$ un sous-ensemble non vide
  de ses processus :
  $\forall a \in \PHs, \PHget{\ctx}{a} \subseteq \PHl_a \wedge \PHget{\ctx}{a} \neq \emptyset$.
  On note $\Ctx$ l'ensemble de tous les contextes.
\end{definition}

\begin{definition}[Recouvrement ($\recouvre : \Ctx \times \powerset(\PHproc) \rightarrow \Ctx$)]
\deflabel{ctxrecouvrement}
  Pour tout contexte $\ctx \in \Ctx$ et tout ensemble de processus $ps \subset \Proc$,
  le recouvrement de $\ctx$ par $ps$ est noté $\ctx \recouvre ps$ et est défini par :
    \[ \forall a \in \PHs, \PHget{(\ctx \recouvre ps)}{a} \DEF
      \begin{cases}
        \{ p \in ps \mid \PHsort(p)=a \} & \text{si } \exists p \in ps, \PHsort(p)=a,\\
        \PHget{\ctx}{a} & \text{sinon.}
      \end{cases} \]
\end{definition}

Pour tout contexte $\ctx \in \Ctx$ et tout processus $a_i \in \Proc$, on note :
$a_i \in \ctx \EQDEF a_i \in \PHget{\ctx}{a}$,
et pour tout état $ps \in \PHl$ ou ensemble de processus $ps \subset \Proc$, on note :
$ps \subseteq \ctx \EQDEF \forall a_i \in ps, a_i \in \ctx$.
De plus, une séquence d'actions $\delta$ est \emph{jouable} dans un contexte $\ctx$
si et seulement si $\exists s \subseteq \ctx, \delta \in \Sce(s)$ ;
on note alors : $\delta \in \Sce(\ctx)$,
et le jeu de $\delta$ dans $\ctx$ est : $\ctx \PHplay \delta = \ctx \Cap \fin{\delta}$.

Finalement, une séquence de bonds sur une sorte $a$ (\defref{bs}) est une séquence d'actions
frappant $a$ dans laquelle le bond de chaque action est égal au frappeur de l'action suivante.
Les séquences de bonds sont utilisées pour trouver les solutions locales d'un objectif donné.
Une séquence de bonds sur $a$ peut être abstraite par l'ensemble de ses frappeurs qui ne sont
pas dans $a$ (\defref{aBS}).
Cette abstraction permet de résoudre un objectif qui concerne une sorte $a$
à l'aide d'autres objectifs concernant d'autres sortes.
On note dans la suite : $\Sol = \powerset(\PHproc)$.

\begin{definition}[Séquence de bonds ($\BS$)]
\deflabel{bs}
  Une \emph{séquence de bonds} $\zeta$ est une séquence d'actions telle que
  $\forall n \in \indexes{\zeta}, n < \card{\zeta}, \PHbounce(\zeta_{n}) = \PHtarget(\zeta_{n+1})$.
  L'ensemble de toutes les séquences de bonds est appelé $\BS$,
  et l'ensemble de toutes les séquences de bonds \emph{résolvant} un objectif $P$, appelé $\BS(P)$,
  est défini par :
    \[ \BS(\PHobj{a_i}{a_j}) \DEF \{ \zeta \in \BS \mid
      \PHtarget(\zeta_1) = a_i \wedge \PHbounce(\zeta_{\card{\zeta}}) = a_j \} \enspace. \]
\end{definition}

On remarque que pour tout objectif $\obj{a_i}{a_j} \in \Obj$,
$\BS(\obj{a_i}{a_j}) = \emptyset$ s'il n'existe aucune moyen d'atteindre $a_j$ depuis $a_i$.
À l'inverse, la séquence vide appartient toujours à
l'ensemble des séquences de bonds résolvant un objectif trivial :
$\forall a_i \in \Proc, \emptyseq \in \BS(\obj{a_i}{a_i})$.

\begin{definition}[Séquence de bonds abstraite ($\aBS:\Obj \rightarrow \powerset(\Sol)$)]
\deflabel{aBS}
  \[
    \aBS(P) \DEF \{ \abstr{\zeta} \in \Sol \mid \zeta \in \BS(P), \nexists \zeta' \in \BS(P), \abstr{\zeta'} \subsetneq \abstr{\zeta} \} \enspace,
  \]
  où $\abstr{\zeta} \DEF \{ \PHhitter(\zeta_n) \mid  n \in \indexes{\zeta} \wedge \PHsort(\PHhitter(\zeta_n)) \neq \PHsort(P) \}$.
\end{definition}

\subsection{Sous-approximation}
\seclabel{ua}

On note $\concr(\w)$ (\defref{concr}) l'ensemble des scénarios concrétisant
une séquence d'objectifs $\w$ dans le contexte $\ctx$,
et $\uconcr(\w)$ (\defref{uconcr}) est défini comme étant égal à $\concr(\w)$ si et seulement si,
pour chaque état $s \subseteq \ctx$,
$\concr(\w) \cap \Sce(s) \neq \emptyset$.

\begin{definition}[$\concr : \OS \to \powerset(\Sce)$]
\deflabel{concr}
  Pour toute séquence d'objectifs $\w \in \OS$, $\concr(\w)$ est l'ensemble des
  de scénarios minimaux concrétisant $\w$ dans le contexte $\ctx$.
  Il est défini comme le plus grand ensemble satisfaisant les conditions suivantes :
  \begin{enumerate}[(i)]
  \item $\forall \delta \in \concr(\w), \exists s \subseteq \ctx, \delta \in \Sce(s)$,
  \item $\forall \delta \in \concr(\w), \exists \phi : \indexes{\w} \to \indexes{\delta},
      (\forall n, m \in \indexes{\w}, n < m \Leftrightarrow \phi(n) \leq \phi(m)),
      \forall n \in \indexes{\w}, \PHbounce(\w_n) \in \ctx \play \delta_{1 \ldots \phi(n)}$,
  \item $\forall \delta, \delta' \in \concr(\w),
      \card{\delta} \leq \card{\delta'} \Rightarrow \delta \neq \delta'_{1 \ldots \card{\delta}}$.
  \end{enumerate}
\end{definition}

\begin{definition}[$\uconcr : \OS \to \powerset(\Sce)$]
\deflabel{uconcr}
  \[ \uconcr(\w) \DEF
    \begin{cases}
      \concr(\w) & \text{si } \forall s \in \PHl, s \subseteq \ctx, \exists \delta \in \concr(\w),
        \delta \in \Sce(s) \\
      \emptyset & \text{sinon.}
    \end{cases} \]
\end{definition}

\begin{lemma}
\lemlabel{uconcr-ctx}
  $\ctx \subseteq \ctx' \wedge \muconcr_{\ctx'}(\w) \neq \emptyset \Rightarrow
    \muconcr_{\ctx}(\w) \neq \emptyset$.
\end{lemma}

Pour tout objectif $P \in \Obj$ et tout contexte $\ctx \in \Ctx$, la \defref{maxCont}
fournit l'ensemble des processus de la sorte $\sorte{P}$ requis pour résoudre $P$ dans $\ctx$,
appelé $\gCont_\ctx(\PHsort(P), P)$.
\begin{definition}[$\gCont_\ctx : \Sigma \times \Obj \to \powerset(\PHproc)$]
\deflabel{maxCont}
  \begin{align*}
    \gCont_\ctx(a,P) \DEF
    \{ p \in \PHproc &\mid \exists ps \in \aBS(P), \exists b_i \in ps, b = a \wedge p = b_i \\
      & \vee b \neq a \wedge p \in \gCont_\ctx(a, \PHobj{b_j}{b_i}) \wedge b_j \in \PHget{\ctx}{b} \}
    \enspace.
  \end{align*}
\end{definition}

Pour une séquence d'objectifs $\w$ et un contexte $\ctx$ donnés,
le \emph{graphe de causalité locale} $\cwB$ (\defref{glc}) représente une sous-approximation de
l'atteignabilité de cette séquence d'objectifs dans $\ctx$.
Pour cela, il relie les objectifs à des solutions à
l'aide des séquences de bonds abstraites de la \defref{aBS}, ce qui produit de nouveaux objectifs
résolus récursivement.
Il s'agit donc d'un graphe dont les nœuds sont des éléments de $\Proc \cup \Obj \cup \Sol$,
c'est-à-dire des processus, des objectifs et des ensembles de processus qui ont le rôle
de \emph{solutions} :
\begin{itemize}
  \item Un nœud dans $\Obj$ représente un objectif requis pour la résolution de $\w$,
    soit faisant directement partie de la séquence d'objectifs $\w$,
    soit indirectement nécessaire à sa résolution ;
  \item Un nœud dans $\Sol$ représente un ensemble de processus nécessaires pour résoudre
    un objectif, c'est-à-dire un élément parmi ses séquences de bonds abstraites ;
  \item Un nœud dans $\Proc$ représente un processus requis pour la résolution,
    c'est-à-dire mentionné dans un nœud solution.
\end{itemize}

Un objectif $P \in \Obj$ est soluble si tous les processus contenus dans au moins une de ses
abstractions de séquences de bonds $\aBS(P) \in \Sol$ (\cf \defref{aBS})
peuvent être activés (\eqref{ESol1}).
Une telle solution représente donc un ensemble de processus qui doivent être activés
pour la résolution de $P$ (\eqref{ESol2}).
Si $a \in \components$, l'atteignabilité d'un de ses processus $a_i$ est approximée par
la possibilité de résoudre tous les objectifs de la forme $\PHobjp{a}{j}{i} \in \Obj$
pour tout $a_j$ dans le contexte initial $\ctx$ (\eqref{EReq}) ;
si $a \in \cs$, l'atteignabilité de $a_i$ est possible si tous les processus du sous-état
$\csState(a_i)$ (\cf \defref{csState}) qu'il représente sont atteignables (\eqref{EPrio}).
La résolution d'un objectif $P$ peut nécessiter un processus de $\PHsort(P)$,
autrement dit : $\gCont(\PHsort(P), P) \neq \emptyset$ (\cf \defref{maxCont}) ;
dans ce cas, $P$ est re-centré (\eqref{ECont}).
Enfin, \eqref{Vw,Vproc,VE} assurent que l'ensemble des nœuds est complet.

Étant donné que le processus actif de chaque sorte peut évoluer au cours de la résolution,
le graphe de causalité locale $\cwB$ est obtenu par saturation avec tous les processus
qu'il contient, \cad en recouvrant le contexte initial $\ctx$ par $\allprocs(\V, \E)$, défini par:
  \[\allprocs(\V, \E) = (V \cap \Proc) \cup
    \{ \PHtarget(P), \PHbounce(P) \mid P \in \V \cap \Obj \} \enspace.\]

\begin{definition}
\deflabel{glc}
  Le graphe de causalité locale $\cwB \DEF (\Bv, \Be)$ est défini par :
  $\cwB \DEF \lfp{\aB^\w_\ctx}{\myB}{\aB^\w_{\ctx \Cap \allprocs(\myB)}}$,
  où $\myB \DEF (\cwV, \cwE)$ est le plus petit graphe respectant
  $\cwV \subseteq \Proc \cup \Obj \cup \Sol$ et $\cwE \subseteq \cwV \times \cwV$
  tel que :
  \begin{align}
    \w &\subseteq \cwV \label{eq:Vw} \\
    P \in \cwVObj &\Rightarrow \PHbounce(P) \in \cwV \label{eq:Vproc} \\
    (x, y) \in \cwE &\Rightarrow y \in \cwV \label{eq:VE} \\
    P \in \cwVObj \wedge ps \in \BS(P) &\Rightarrow (P, ps) \in \cwE \label{eq:ESol1} \\
    ps \in \cwVSol \wedge a_i \in ps &\Rightarrow (ps, a_i) \in \cwE \label{eq:ESol2} \\
    a \in \components \wedge a_i \in \cwVProc \wedge a_j \in \ctx &\Rightarrow (a_i, \PHobjp{a}{j}{i}) \in \cwE \label{eq:EReq} \\
    a \in \cs \wedge a_i \in \cwVProc \wedge ps \in \csState(a_i) &\Rightarrow (a_i, ps) \in \cwE \label{eq:EPrio} \\
    P \in \cwVObj \wedge q \in \gCont_\ctx(\PHsort(P), P) &\Rightarrow (P, \PHobj{q}{\PHbounce(P)}) \in \cwE \label{eq:ECont} \!
  \end{align}
\end{definition}

Au sein de ce graphe de causalité locale, un arc $(p, ps) \in \Proc \times \Sol$
est dit \emph{cohérent} (\defref{coherent}) si aucun des processus dans $ps$
n'est «~compromis~» par un processus successeur du nœud $ps$,
%n'entre en conflit avec les processus successeur du nœud $ps$,
\cad si, pour tout processus de $ps$,
il n'existe pas de processus différent de la même sorte parmi tous les successeurs de $ps$.
%
%s'il n'existe pas un processus dans $ps$ et un processus parmi les successeurs de $ps$
%qui soient différents mais appartenant à la même sorte.
Si tous les arcs du graphe sont cohérents, alors le \thmref{approxinf}
donne une condition suffisante pour la concrétisation de la séquence d'objectifs $\w$
dans le contexte $\ctx$, qui est basée directement sur ce graphe $\cwB$.

\begin{definition}[Arc cohérent]
\deflabel{coherent}
  Un arc $(x, y) \in \cwE$ est dit \emph{cohérent} si et seulement si
  $(x, y) \in \Be \cap (\Proc \times \Sol) \Rightarrow y$ n'a aucun successeur
  $a_j \in \Bv \cap \Proc$
  tel que $\exists a_i \in y, \sorte{a_i} = \sorte{a_j} \wedge a_i \neq a_j$.
\end{definition}

\begin{theorem}[Sous-Approximation]
\thmlabel{approxinf}
  Étant données des Frappes de Processus canoniques $\PH = (\PHs; \PHl; (\PHa^{(1)};\PHa^{(2)}))$,
  un contexte $\ctx \in \Ctx$ et une séquence d'objectifs $\w \in \OS$,
  si le graphe $\cwB$ ne contient aucun cycle,
  que tous ses nœuds objectifs possèdent au moins une solution
  et que tous ses arcs sont cohérents,
  alors $\uconcr(\w) \neq \emptyset$.
\end{theorem}

\begin{proof} %[\Thmref{approxinf}]
  On note dans la suite :
  $\Bee{X}{Y} = \Be \cap (X \times Y)$, avec $X, Y$ parmi $\PHproc$, $\Obj$ et $\Sol$,
  et : $max\ctx = \ctx \Cap \allprocs(\cwB)$ le contexte accepté par $\cwB$.
  
  Soit $(a_i, ps) \in \Bee{\Proc}{\Sol}$ un arc liant un processus requis de sorte coopérative à
  une solution et supposons que tous les enfants de $ps$ sont concrétisables.
  On étiquette tous les processus de $ps$ par un entier : $ps = \{ p_n \}_{n \in \indexes{ps}}$.
  Montrons par récurrence que pour tout $n \in \segm{0}{\card{ps}}$,
  il existe un scénario $\delta_n$ tel que :
  $\forall i \in \segm{1}{n}, \PHget{(s \PHplay \delta_n)}{\PHsort(p_i)} = p_i$.
  \begin{itemize}
    \item Le cas $\delta_0 = \varepsilon$ est immédiat.
    \item Soit $n \in \segm{0}{\card{ps} - 1}$.
      On suppose qu'il existe $\delta_n$ tel que décrit ci-dessus.
      Posons $q = \PHget{(s \PHplay \delta_n)}{\PHsort(p_{n+1})}$.
      Par hypothèse, $(a_i, ps)$ est cohérent (\defref{coherent}) et tous les processus
      de $ps$ sont des processus de composants \todo{???} ;
      cela signifie qu'aucun des processus requis pour résoudre $p_{n+1}$ n'est un autre processus
      de la même sorte qu'un processus de $ps$.
      Par conséquent, il existe un scénario
      $\delta' \in \muconcr_{s \PHplay \delta_n}(\PHobj{q}{p_{n+1}})$
      tel que $\forall i \in \segm{1}{n+1},
        \PHget{(s \PHplay \delta_n \PHplay \delta')}{\PHsort(p_{i})} = p_{i}$.
      Finalement, d'après le \lemref{update}, il existe un scénario
      $\delta'' \in \reductionsce(s \PHplay \delta_n \PHplay \delta')$
      tel que $\update(s \PHplay \delta_n \PHplay \delta') = s \PHplay \delta_{n+1}$
      avec $\delta_{n+1} = \delta_n \PHplay \delta' \PHplay \delta''$,
      et d'après le \lemref{hcscomp} :
      $\forall i \in \segm{1}{n+1}, \PHget{(s \PHplay \delta_{n+1})}{\PHsort(p_i)} = p_i$
  \end{itemize}
  Ainsi, $\delta = \delta_{|ps|}$ existe, et étant donné ses propriétés, on a immédiatement :
  $\PHget{(s \PHplay \delta)}{a} = a_i$ et $\update(s \PHplay \delta) = s \PHplay \delta$.
  
  Soit un état $s \in L$ tel que $s \subseteq max\ctx$.
  Étant donné qu'il n'y a aucun cycle dans $\cwB$, montrons par récurrence que
  pour tout objectif $P \in \Bv \cap \Obj$ tel que $\PHtarget(P) \in s$,
  $\exists \delta \in \muconcr_s(P)$.
  \begin{itemize}
    \item Si $(P, \emptyset) \in \Bee{\Obj}{\Sol}$,
      soit on a $\PHtarget(P) = \PHbounce(P)$ et $\delta = \emptyseq$,
      soit on a $\forall \zeta \in \BS(P), \zeta \in \Sce(s) \wedge \PHsort(\zeta) = \{ \PHsort(P) \}$
      et dans ce cas $\delta = \delta_1 \PHplay \zeta_1 \PHplay \dots \PHplay
        \delta_{|\zeta|} \PHplay \zeta_{|\zeta|}$
      est une séquence valide donnée par le \lemref{hcompcomp}.
    \item Supposons que tous les objectifs qui sont les successeurs de $P$ sont concrétisables.
      Si $\exists (P, Q) \in \Bee{\Obj}{\Obj}$, alors, par hypothèse,
        $\muconcr_{s}(\obj{\PHtarget(P)}{\PHtarget(Q)} \concat Q) \neq \emptyset$, et donc
        $\muconcr_{s}(P) \neq \emptyset$.
      Sinon, d'après la \defref{maxCont}, la concrétisation des successeurs de $P$ ne requiert
        aucun processus de la sorte $\PHsort(P)$.
        De plus, il existe $\zeta \in \BS(P)$ tel que $(P, \aZ) \in \Bee{\Obj}{\Sol}$.
        Montrons par récurrence que pour tout $n \in \indexes{\zeta}$, il existe un scénario
        $\delta_n$ tel que $\PHget{(s \PHplay \delta_n)}{\PHsort(P)} = \PHbounce(\zeta_n)$.
        \begin{itemize} %\item[$\circ$]
          \item[$\circ$] Supposons que $\delta_n$ existe et posons $\zeta_n = \PHhit{b_i}{a_j}{a_k}$.
            Par hypothèse, il existe $\delta' \in \muconcr_{s \PHplay \delta_n}(\PHobj{\any}{b_i})$
            avec $\PHsort(P) \notin \PHsort(\delta')$ (\defref{maxCont}).
            D'après le \lemref{update}, il existe
            $\delta'' \in \reductionsce(s \PHplay \delta')$ tel que
            $\update(s \PHplay \delta') = s \PHplay \delta' \PHplay \delta''$.
            De plus, $\PHget{(s \PHplay \delta' \PHplay \delta'')}{b} = b_j$
            (D'après le \lemref{hcompcomp} si $b \in \components$
            ou le \lemref{hcscomp} si $b \in \cs$).
            Ainsi, $\delta_{n+1} = \delta_n \PHplay \delta' \PHplay \delta'' \PHplay \zeta_n$.
        \end{itemize}
      On a donc :$\delta_{|\zeta|} \in \muconcr_s(P)$. % and $\ceil(\delta) \subseteq max\ctx$.
  \end{itemize}
  Finalement, étant donné $\muconcr_{max\ctx}(\w) \neq \emptyset$,
  et d'après le \lemref{uconcr-ctx},
  on a : $\uconcr(\w) \neq \emptyset$.
\end{proof}



\begin{example}
  Comparons les Frappes de Processus canoniques $\PH = (\PHs; \PHl; (\PHa^{(1)}; \PHa^{(2)}))$
  définies à la \figref{ph-livelock} avec 
  les Frappes de Processus standards $\PH' = (\PHs; \PHl; \PHh')$
  telles que $\PHh' = \PHh^{(1)} \cup \PHh^{(2)}$.
  En d'autres termes, $\PH'$ est la version sans priorités de $\PH$,
  c'est-à-dire où toutes les actions sont présentes mais où la notion de classes de priorités
  a été supprimée.
  
  Intéressons-nous à la dynamique de $\PH'$.
  Comme expliqué au \stodo{\S{} sur la modélisation des coopérations},
  ce modèle comporte une sorte coopérative $ab$ qui est définie de la façon
  dont le sont toutes les sortes coopératives en
  en Frappes de Processus standards, faute de priorités.
  Cette version sans priorités introduit des comportements supplémentaires qui ne sont pas
  désirables pour représenter une porte logique exacte.
  En effet, on pourrait s'attendre à ce que $ab_{11}$ représente la présence simultanée
  de $a$ et $b$ dans un état passé, c'est-à-dire :
  ou, en d'autres termes, que seule la présence simultanée de $a_1$ et $b_1$ pourrait
  permettre l'activation de $ab_{11}$.
  Or on constate que ce n'est pas le cas, car le scénario suivant permet par exemple d'activer
  ce processus, sans pour autant que $a_1$ et $b_1$ ne soient simultanément présents,
  et bien que le processus initial de $ab$ soit cohérent avec ceux de $a$ et $b$ :
  $\PHstate{a_1, b_0, c_0, ab_{10}} \PHtrans[\PH']
  \PHstate{a_0, b_0, c_0, ab_{10}} \PHtrans[\PH']
  \PHstate{a_0, b_1, c_0, ab_{10}} \PHtrans[\PH']
  \PHstate{a_0, b_1, c_0, ab_{11}}$.
  En effet, il se trouve que l'activation de $ab_{11}$ ne représente pas la présence simultanée
  de $a_1$ et $b_1$ dans un même état, mais leur présence consécutive dans
  différents états.
  En d'autres termes, la présence de $ab_{11}$ dans un état n'implique pas
  $a_1 \wedge b_1$ dans le même état, ni même
  $\tempop(a_1 \wedge b_1)$, mais plutôt : $\tempop(a_1) \wedge \tempop(b_1)$
  où $\tempop$ signifie : «~précédemment~».
  On constate d'ailleurs que l'analyse statique développée dans~\cite{PMR12-MSCS}
  répond positivement quand à l'accessibilité de $c_1$ depuis l'état
  $\PHstate{a_1, b_0, c_0, ab_{10}}$.
  
  Un tel comportement est indésirable si on souhaite utiliser les sortes coopératives en tant
  que portes logiques exactes, c'est-à-dire sans ce décalage temporel menant à une sur-approximation
  de la dynamique.
  Les Frappes de Processus canoniques permettent alors de modéliser un comportement (faiblement)
  bisimilaire au comportement voulu \stodo{ref?}, les actions primaires permettant la mise à jour
  sans délai des sortes coopératives, comme pour le modèle $\PH$ de la \figref{ph-livelock}.
  En effet, le processus $ab_{11}$ est impossible à activer si $a_1$ et $b_1$ ne sont
  pas simultanément actifs, et ce cas ne peut se présenter que dans l'état initial.
  Il s'agit d'un jardin d'Eden dans le sens développé dans \stodo{la thèse de Loïc}.
  La sous-approximation développée dans cette section au \thmref{approxinf}
  ne conclut d'ailleurs pas quant à l'accessibilité de $c_1$ depuis l'état
  $\PHstate{a_1, b_0, c_0, ab_{10}}$.
  En effet, comme le montre la \figref{glc-livelock},
  l'arc représenté en double trait liant le nœud processus $ab_{11}$ à son unique solution
  n'est pas cohérent selon la
  \defref{coherent}, ce qui empêche l'application du théorème.
  Cependant, il faut noter que la sur-approximation de~\cite{PMR12-MSCS}
  est toujours valable sur les Frappes de Processus canoniques à condition de l'appliquer
  sur une version sans priorités (\cad dans le cas présent, l'appliquer sur $\PH'$
  pour obtenir un résultat quant à $\PH$).
  Ainsi, cette sur-approximation renvoie un résultat lui aussi non-conclusif,
  ce qui fait que la méthode d'analyse statique ne répond globalement pas sur cet exemple,
  et de façon plus générale sur tous les exemples dont l'atteignabilité recherchée
  est rendue impossible par simple ajout de classes de priorités.
  
  On note pour finir que le \thmref{approxinf} est conclusif sur l'atteignabilité de $c_1$
  depuis $\PHstate{a_1, b_0, c_0, ab_{10}}$ dans les Frappes de Processus canoniques $\PH''$, où :
  $\PH'' = (\PHs; \PHl; (\PHa^{(1)};\PHa'^{(2)}))$ avec :
  \[\PHa'^{(2)} = (\PHa^{(2)} \setminus \{ \PHhit{a_0}{b_0}{b_1}, \PHhit{b_0}{a_0}{a_1} \})
    \cup \{ \PHhit{a_0}{a_0}{a_1}, \PHhit{b_0}{b_0}{b_1} \} \enspace.\]
  En effet, dans ce cas les processus $a_0$ et $b_0$ du graphe de la \figref{glc-livelock}
  sont permutés, ce qui tous les arcs cohérents.

\begin{figure}[tp]
  \centering
  \begin{tikzpicture}[aS]
    \node[Aproc] (c1) {$c_1$};
    \node[Aobj,below of=c1] (c01) {$\PHobj{c_0}{c_1}$};
    \node[Asol,below of=c01] (c01s) {};

    \node[AprocPrio,below of=c01s] (ab11) {$ab_{11}$};
    \node[AsolPrio,below of=ab11] (ab11s) {};

    \node[Aproc,below left of=ab11s] (a1) {$a_1$};
    \node[Aobj,below of=a1] (a11) {$\PHobj{a_1}{a_1}$};
    \node[Asol,below of=a11] (a11s) {};
    \node[Aobj,below left of=a1] (a01) {$\PHobj{a_0}{a_1}$};
    \node[Asol,below of=a01] (a01s) {};
    \node[Aproc,below of=a01s] (b0) {$b_0$};
    \node[Aobj,below of=b0] (b00) {$\PHobj{b_0}{b_0}$};
    \node[Asol,below of=b00] (b00s) {};
    \node[Aobj,below left of=b0] (b10) {$\PHobj{b_1}{b_0}$};
    \node[Asol,below of=b10] (b10s) {};

    \node[Aproc,below right of=ab11s] (b1) {$b_1$};
    \node[Aobj,below of=b1] (b11) {$\PHobj{b_1}{b_1}$};
    \node[Asol,below of=b11] (b11s) {};
    \node[Aobj,below right of=b1] (b01) {$\PHobj{b_0}{b_1}$};
    \node[Asol,below of=b01] (b01s) {};
    \node[Aproc,below of=b01s] (a0) {$a_0$};
    \node[Aobj,below of=a0] (a00) {$\PHobj{a_0}{a_0}$};
    \node[Asol,below of=a00] (a00s) {};
    \node[Aobj,below right of=a0] (a10) {$\PHobj{a_1}{a_0}$};
    \node[Asol,below of=a10] (a10s) {};

    \path
    (c1) edge (c01)
    (c01) edge (c01s)
    (c01s) edge (ab11)
    (ab11) edge[aSPrio] (ab11s)
    (ab11s) edge (a1) edge (b1)

    (a1) edge (a01) edge (a11)
    (a01) edge (a01s)
    (a01s) edge (b0)
    (a11) edge (a11s)
    (a0) edge (a10) edge (a00)
    (a10) edge (a10s)
    (a00) edge (a00s)

    (b0) edge (b10) edge (b00)
    (b10) edge (b10s)
    (b00) edge (b00s)
    (b1) edge (b01) edge (b11)
    (b01) edge (b01s)
    (b01s) edge (a0)
    (b11) edge (b11s)
    ;
    \end{tikzpicture}
  \caption{
  \figlabel{glc-livelock}
    Le graphe de causalité locale des Frappes de Processus de la \figref{ph-livelock}
    pour l'objectif $\w = \PHobj{c_0}{c_1}$
    et le contexte initial $\ctx = \PHstate{a_1, b_0, c_0, ab_{10}}$.
    Les nœuds rectangulaires représentent les éléments de $\Proc$,
    les nœuds sans bordure sont les éléments de $\Obj$
    et les cercles sont les éléments de $\Sol$.
    Le processus $ab_{11}$, ainsi que son unique solution et l'arc qui les relie,
    sont mis en valeur avec des traits doubles car il s'agit du principal ajout de la méthode
    présentée à la \secref{ua}.
    Il est à noter que l'arc dessiné avec un trait double n'est pas cohérent
    au sens de la \defref{coherent}
    En effet, sa cible est la solution $\{ a_1, b_1 \}$,
    or l'un de ses successeurs indirects est $a_0$, qui est un autre processus de la même sorte
    que $a_1$ (et le même raisonnement fonctionne pour $b_0$).
  }
\end{figure}
\end{example}

\todo{Mentionner les limites de l'analyse :
\begin{itemize}
  \item sur quels modèles l'analyse peut échouer ?
  \item Quels motifs ?
\end{itemize}}



\subsection{Atteignabilité d'un état}

\newcommand{\uastotal}{\tau}
\newcommand{\uasreach}{\rho}
\newcommand{\uasps}{{ps}}

La propriété d'atteignabilité développée à la \secref{ua}
sur les Frappes de Processus canoniques
ne traite l'atteignabilité d'un ensemble de processus que de façon séquentielle.
Cependant, il est possible de vérifier l'atteignabilité d'un sous-état
(autrement dit, l'atteignabilité simultanée d'un ensemble de processus)
à l'aide d'une sorte coopérative.

En effet, soient $\PH = (\PHs, \PHl, (\PHh^{(1)}, \PHh^{(2)}))$ des Frappes de Processus
canoniques et supposons que l'ont cherche à étudier l'atteignabilité d'un sous-état
$\uasps \in \PHsubl_S$, avec $S \subset \PHs$.
On pose alors : $\PH' = (\PHs', \PHl', (\PHh'^{(1)}, \PHh'^{(2)}))$
les Frappes de Processus canoniques telles que :
\begin{itemize}
  \item $\PHs' = \PHs \cup \{ \uastotal, \uasreach \}$,
  \item $\PHl' = \PHl \times \PHl_\uastotal \times \PHl_\uasreach$, où
    $\PHl_\uastotal = \PHsubl_S$ et $\PHl_\uasreach = \{ \uasreach_0, \uasreach_1 \}$,
  \item $\PHh'^{(1)} = \PHh^{(1)} \cup
      \{ \PHfrappe{a_i}{\uastotal_\mysigma}{\uastotal_{\mysigma'}} \mid
      a \in S, \mysigma, \mysigma' \in \PHl_\uastotal,
      \PHget{\mysigma}{a} \neq a_i \wedge \mysigma' = \mysigma \recouvre a_i \}$,
  \item $\PHh'^{(2)} = \PHh^{(2)} \cup \{ \PHfrappe{\uastotal_\uasps}{\uasreach_0}{\uasreach_1} \}$.
\end{itemize}
Cette transformation consiste donc à ajouter au modèle
une sorte coopérative $\uastotal$ sur toutes les sortes de $S$,
et un composant $\uasreach$ qui ne puisse changer de processus que lorsque le sous-état $\uasps$
est présent (ce qui est déterminé par $\uastotal$).
Ainsi, l'atteignabilité du sous-état $\uasps$ depuis un contexte initial $\ctx$ dans $\PH$
est équivalente à celle du processus $\uasreach_1$ depuis le contexte
$\ctx \cup \{ \uasreach_0 \}$ dans $\PH'$
(l'état initial de $\uastotal$ n'a pas d'importance et peut être arbitrairement choisi),
qui peut être traitée grâce au \thmref{approxinf}.

Nous pouvons donc répondre à des questions d'atteignabilité simultanée de plusieurs processus
directement à l'aide des Frappes de Processus canoniques (\defref{phcanonique})
et de l'analyse statique développée pour ce formalisme (\thmref{approxinf}).
Cette méthode peut naturellement être adaptée pour répondre quant à
l'accessibilité d'un ensemble de sous-états.
Il est à noter cependant que le nombre de processus de la sorte coopérative $\uastotal$
croît exponentiellement avec le nombre de sortes dans $S$, ce qui peut fortement impacter
la vitesse de résolution de l'analyse statique.
Pour pallier cela, il est possible de «~factoriser~» cette sorte coopérative comme expliqué
à la section \toref.



\subsection{Raffinement de la Sous-approximation séquentielle}
\seclabel{approxinf-ordonnee}

Dans cette section, nous donnons une alternative à la condition suffisante du \thmref{approxinf}
qui permet de prendre en compte la séquentialité des objectifs plutôt que de les considérer
simultanément, tel que cela est fait dans la version actuelle.
Comme les objectifs sont prix en compte individuellement, une telle approche ne prend en compte
qu'un sous-ensemble des scénarios possibles.
Cependant, en se concentrant à chaque itération sur une plus petite partie du réseau, 
cette sous-approximation séquentielle peut s'avérer plus souvent conclusive.

Définissions une séquence d'objectifs $\w = \obj{a_i}{a_j} \concat \w'$ avec
$a_i \neq a_j$ et un état $s \in \PHl$ tel que $\get{s}{a} = a_i$.
On peut remarquer que tout scénario atteignant $a_j$ inclut nécessairement l'une des séquences
de bonds dans $\BS(\obj{a_i}{a_j})$ et, en particulier,
tout scénario minimal atteignant $a_j$ termine dans un état où son présents à la fois
$a_j$ et le frappeur $a_k$ de la dernière action d'une des séquences de bonds
dans $\BS(\obj{a_i}{a_j})$.
Si la sorte $b$ d'un tel frappe est de surcroît une sorte coopérative ($b \in \cs$),
cela signifie alors aussi que l'un des sous-états dans $\csState(b_k)$
est inclus dans l'état final.
La \defref{lastprocs} définit $\derprocs(\obj{a_i}{a_j})$ comme étant l'ensemble des ensembles
de processus qui peuvent être présents juste après avoir atteint $a_j$.

D'après le \thmref{approxinf}, on peut déduire que pour tout scénario $\delta \in \uconcr(P)$,
il existe un ensemble de processus $ps \in \derprocs(P)$ tel que $ps \subset (s \play \delta)$.
Donc, si $\muconcr_{\ctx' \Cap ps}(\w') \neq \emptyset$,
avec $\ctx' = \ctx \Cap \procs(\mycwB{\ctx}{P})$,
il existe alors un scénario $\delta'$ concrétisant $\w'$ depuis l'état $(s \play \delta)$.
Ainsi, le scénario  $\delta \concat \delta'$ concrétise $\w$.

\begin{definition}[$\derprocs : \Obj \to \powerset(\powerset(\Proc))$]
\deflabel{lastprocs}
  Pour tout objectif $\obj{a_i}{a_j} \in \Obj$, $\derprocs(\obj{a_i}{a_j})$
  est défini comme le plus grand ensemble tel que :
  $\forall lps \in \derprocs(\obj{a_i}{a_j}), lps \in \powerset(\Proc)$,
  \begin{enumerate}
    \item $a_j \in lps$,
    \item $\exists \zeta \in \BS(\obj{a_i}{a_j}),
        \sorte{\hitter{\zeta_{\card{\zeta}}}} \neq \sorte{\bounce{\zeta_{\card{\zeta}}}}
        \Rightarrow \hitter{\zeta_{\card{\zeta}}} \in lps$,
    \item $\forall b_j \in lps, b \in \cs \Rightarrow \exists ps \in \csState(b_j), ps \subset lps$,
    \item $\nexists lps'\in \derprocs(\obj{a_i}{a_j}), lps' \varsubsetneq lps$.
  \end{enumerate}
\end{definition}

\begin{theorem}[Sous-approximation séquentielle]
\thmlabel{approxinf-ordonnee}
  Pour toutes Frappes de Processus canoniques $(\PHs; \PHl; \PHa^{\langle 2 \rangle})$,
  tout contexte $\ctx \in \Ctx$ et toute séquence d'objectifs $\w = P \concat \w' \in \OS$,
  $\uconcr(P) \neq \emptyset \wedge \forall ps \in \derprocs(P),
    \muconcr_{\ctx' \Cap ps}(\w') \neq \emptyset
    \Rightarrow \uconcr(\w) \neq \emptyset$,
  où $\ctx' = \ctx \Cap \procs(\mycwB{\ctx}{P})$.
\end{theorem}

\begin{proof} %[\Thmref{approxinf-ordonnee}]
  Si $\uconcr(P) \neq \emptyset$, alors pour tout état $s \in \PHl, s \subset \ctx$,
  il existe un scénario $\delta \in \uconcr(P) \cap \Sce(s)$.
  D'après la \defref{lastprocs},
  et en s'inspirant de la démonstration du \thmref{approxinf},
  il existe $ps \in \derprocs(P)$ tel que $(s \play \delta) \subset \ctx' \Cap ps$.
  Ainsi, si $\muconcr_{\ctx' \Cap ps}(\w') \neq \emptyset$,
  alors il existe un scénario $\delta' \in \muconcr_{\ctx' \Cap ps}(\w')$ tel que
  $\delta' \in \Sce(s \play \delta)$.
  Par conséquent, $\delta \concat \delta'$ est un scénario jouable dans $s$.
  Donc, pour tout $s \in \PHl, s \subset \ctx$, il existe un scénario concrétisant $\w$.
  D'où : $\uconcr(\w) \neq \emptyset$.
\end{proof}
