
\section{Analyse statique}
\seclabel{sa}

\todo{Revoir la glu}

L'objectif de cette section est de définir le problème de l'\emph{atteignabilité} dans des
Frappes de Processus,
aussi appelée \emph{accessibilité},
et de proposer une sous-approximation permettant de la résoudre efficacement
dans les Frappes de Processus canoniques.

La méthode développée ici est inspirée de \stodo{PMR12-MSCS}.  %~\cite{PMR12-MSCS}.

Le problème de l'atteignabilité dans les Frappes de Processus consiste à rechercher l'existence
d'un scénario qui permette d'activer un ou plusieurs processus donné(s).
Il peut se résumer à la question suivante :
«~Étant donné un état initial, existe-t-il un scénario partant de cet état et
qui permette d'activer un processus donné ?~»
ou, de façon plus générale pour un ensemble de processus :
«~Étant donné un état initial $\ctx$ et une séquence de processus $\w$ donnés,
existe-t-il un scénario $\delta$ jouable dans $\ctx$ et permettant d'activer successivement
chacun des processus de $\w$ dans l'ordre ?~»

La résolution du problème de l'atteignabilité peut parfois être résolu à l'aide des outils
de model checking classiques.
Cependant, de telles méthodes reposent généralement sur l'analyse de la dynamique du modèle.
Pour de grands modèles, ces méthodes se heurtent donc à l'explosion combinatoire inhérente
au calcul du graphe des états.
La méthode proposée ici repose en revanche sur une sous-approximation du modèle analysé.
Cela permet d'éviter une complexité exponentielle
et d'avoir à la place un problème de complexité polynomiale
sous la condition que chaque sorte du modèle possède un nombre restreint de processus
(une sorte de quatre processus ou moins satisfaisant ce critère).

On considère dans cette section un modèle de Frappes de Processus canoniques
$\PH = (\PHs; \PHl; (\PHa^{(1)};\PHa^{(2)}))$
tel que défini à la \defref{phcanonique}.






\subsection{Définitions préliminaires}
\seclabel{sa-defs}

L'atteignabilité d'un processus $a_j$ d'une sorte donnée $a$ depuis un autre processus $a_i$
de la même sorte est le fait, depuis un état où $a_i$ est actif,
de pouvoir jouer scénario menant dans un état où $a_j$ est actif.
La question de l'existence d'un tel scénario se représente sous la forme d'un \emph{objectif},
noté $\PHobjp{a}{i}{j}$ (\defref{obj}).
De plus, une \emph{séquence d'objectifs} est une séquence d'éléments de $\Obj$ telle que
la cible de chaque objectif est égale au bond de l'objectif précédent de la même sorte
dans la séquence, s'il existe (\defref{os}).

\begin{definition}[Objectif ($\Obj$)]
\deflabel{obj}
  Si $a \in \components$, l'atteignabilité d'un processus $a_j$ depuis un processus $a_i$
  est appelé un \emph{objectif}, noté $\PHobj{a_i}{a_j}$.
  L'ensemble de tous les objectifs est noté
  $\Obj \DEF \{ \PHobj{a_i}{a_j} \mid
    a \in \components \wedge (a_i, a_j) \in \PHl_a \times \PHl_a \}$.
  Pour tout objectif $P = \PHobj{a_i}{a_j} \in \Obj$, on note
  $\sort{P} \DEF a$ la sorte de l'objectif $P$,
  $\target{P} \DEF a_i$ sa cible et $\bounce{P} \DEF a_j$ son bond.
  Enfin, $P$ est dit \emph{trivial} si $a_i = a_j$.
\end{definition}

\begin{definition}[Séquence d'objectif ($\Obj$)]
\deflabel{os}
  Une \emph{séquence d'objectifs} est une séquence $\w = P_1 \cons \ldots \cons P_{\card{\w}}$,
  où $\forall n \in \indexes{\w}, \w_n \in \Obj$
  et $\cible{\w_n} = a_i \Rightarrow \der{a}{\w_{1 \ldots n-1}} \in \{ \varnothing, a_i \}$.
  L'ensemble des séquences d'objectifs est référé par $\OS$.
%  Les définitions de $\premsymbol_a$ (\eqref{prem}), $\dersymbol_a$ (\eqref{der}),
%  $\suppsymbol$ (\eqref{supp}) et $\finsymbol$ (\eqref{fin}) sont étendues aux
  Les définitions de $\premsymbol_a$, $\dersymbol_a$,  $\suppsymbol$ et $\finsymbol$
  (\defref{premder}) sont étendues aux séquences d'objectifs en omettant le cas des frappeurs.
\end{definition}

La \defref{ctx} introduit la notion de \emph{contexte} qui étend celle d'état
afin de pouvoir représenter un ensemble d'états initiaux possibles :
plutôt que d'attribuer un seul processus actif à chaque sorte, comme pour un état,
un contexte permet d'en attribuer plusieurs.
La notion de recouvrement est aussi étendue aux contextes dans la \defref{ctxcap}.

\begin{definition}[Contexte ($\Ctx$)]
\deflabel{ctx}
  Un \emph{contexte} $\ctx$ associe à chaque sorte dans $\PHs$ un sous-ensemble non vide
  de ses processus :
  $\forall a \in \PHs, \PHget{\ctx}{a} \subseteq \PHl_a \wedge \PHget{\ctx}{a} \neq \emptyset$.
  On note $\Ctx$ l'ensemble de tous les contextes.
\end{definition}

\begin{definition}[Recouvrement ($\recouvre : \Ctx \times \powerset(\PHproc) \rightarrow \Ctx$)]
\label{def:ctxcap}
  Pour tout contexte $\ctx \in \Ctx$ et tout ensemble de processus $ps \subset \Proc$,
  le recouvrement de $\ctx$ par $ps$ est noté $\ctx \recouvre ps$ et est défini par :
    \[ \forall a \in \PHs, \PHget{(\ctx \recouvre ps)}{a} \DEF
      \begin{cases}
        \{ p \in ps \mid \PHsort(p)=a \} & \text{si } \exists p \in ps, \PHsort(p)=a,\\
        \PHget{\ctx}{a} & \text{sinon.}
      \end{cases} \]
\end{definition}

Pour tout contexte $\ctx \in \Ctx$ et tout processus $a_i \in \Proc$, on note :
$a_i \in \ctx \EQDEF a_i \in \PHget{\ctx}{a}$,
et pour tout état $ps \in \PHl$ ou ensemble de processus $ps \subset \Proc$, on note :
$ps \subseteq \ctx \EQDEF \forall a_i \in ps, a_i \in \ctx$.
De plus, une séquence d'actions $\delta$ est \emph{jouable} dans un contexte $\ctx$
si et seulement si $\exists s \subseteq \ctx, \delta \in \Sce(s)$ ;
on note alors : $\delta \in \Sce(\ctx)$,
et le jeu de $\delta$ dans $\ctx$ est : $\ctx \PHplay \delta = \ctx \Cap \fin{\delta}$.

Finalement, une séquence de bonds sur une sorte $a$ (\defref{bs}) est une séquence d'actions
frappant $a$ dans laquelle le bond de chaque action est égal au frappeur de l'action suivante.
Les séquences de bonds sont utilisées pour trouver les solutions locales d'un objectif donné.
Une séquence de bonds sur $a$ peut être abstraite par l'ensemble de ses frappeurs qui ne sont
pas dans $a$ (\defref{aBS}).
Cette abstraction permet de résoudre un objectif qui concerne une sorte $a$
à l'aide d'autres objectifs concernant d'autres sortes.
On note dans la suite : $\Sol = \powerset(\PHproc)$.

\begin{definition}[Séquence de bonds ($\BS$)]
\deflabel{bs}
  Une \emph{séquence de bonds} $\zeta$ est une séquence d'actions telle que
  $\forall n \in \indexes{\zeta}, n < \card{\zeta}, \PHbounce(\zeta_{n}) = \PHtarget(\zeta_{n+1})$.
  L'ensemble de toutes les séquences de bonds est appelé $\BS$,
  et l'ensemble de toutes les séquences de bonds \emph{résolvant} un objectif $P$, appelé $\BS(P)$,
  est défini par :
    \[ \BS(\PHobj{a_i}{a_j}) \DEF \{ \zeta \in \BS \mid
      \PHtarget(\zeta_1) = a_i \wedge \PHbounce(\zeta_{\card{\zeta}}) = a_j \} \enspace. \]
\end{definition}

On remarque que pour tout objectif $\obj{a_i}{a_j} \in \Obj$,
$\BS(\obj{a_i}{a_j}) = \emptyset$ s'il n'existe aucune moyen d'atteindre $a_j$ depuis $a_i$.
À l'inverse, la séquence vide appartient toujours à
l'ensemble des séquences de bonds résolvant un objectif trivial :
$\forall a_i \in \Proc, \emptyseq \in \BS(\obj{a_i}{a_i})$.

\begin{definition}[Séquence de bonds abstraite ($\aBS:\Obj \rightarrow \powerset(\Sol)$)]
\deflabel{aBS}
  \[
    \aBS(P) \DEF \{ \abstr{\zeta} \in \Sol \mid \zeta \in \BS(P), \nexists \zeta' \in \BS(P), \abstr{\zeta'} \subsetneq \abstr{\zeta} \} \enspace,
  \]
  où $\abstr{\zeta} \DEF \{ \PHhitter(\zeta_n) \mid  n \in \indexes{\zeta} \wedge \PHsort(\PHhitter(\zeta_n)) \neq \PHsort(P) \}$.
\end{definition}

\subsection{Sous-approximation}
\seclabel{ua}

On note $\concr(\w)$ (\defref{concr}) l'ensemble des scénarios concrétisant
une séquence d'objectifs $\w$ dans le contexte $\ctx$,
et $\uconcr(\w)$ (\defref{uconcr}) est défini comme étant égal à $\concr(\w)$ si et seulement si,
pour chaque état $s \subseteq \ctx$,
$\concr(\w) \cap \Sce(s) \neq \emptyset$.

\begin{definition}[$\concr : \OS \to \powerset(\Sce)$]
\deflabel{concr}
  Pour toute séquence d'objectifs $\w \in \OS$, $\concr(\w)$ est l'ensemble des
  de scénarios minimaux concrétisant $\w$ dans le contexte $\ctx$.
  Il est défini comme le plus grand ensemble satisfaisant les conditions suivantes :
  \begin{enumerate}[(i)]
  \item $\forall \delta \in \concr(\w), \exists s \subseteq \ctx, \delta \in \Sce(s)$,
  \item $\forall \delta \in \concr(\w), \exists \phi : \indexes{\w} \to \indexes{\delta},
      (\forall n, m \in \indexes{\w}, n < m \Leftrightarrow \phi(n) \leq \phi(m)),
      \forall n \in \indexes{\w}, \PHbounce(\w_n) \in \ctx \play \delta_{1 \ldots \phi(n)}$,
  \item $\forall \delta, \delta' \in \concr(\w),
      \card{\delta} \leq \card{\delta'} \Rightarrow \delta \neq \delta'_{1 \ldots \card{\delta}}$.
  \end{enumerate}
\end{definition}

\begin{definition}[$\uconcr : \OS \to \powerset(\Sce)$]
\deflabel{uconcr}
  \[ \uconcr(\w) \DEF
    \begin{cases}
      \concr(\w) & \text{si } \forall s \in \PHl, s \subseteq \ctx, \exists \delta \in \concr(\w),
        \delta \in \Sce(s) \\
      \emptyset & \text{sinon.}
    \end{cases} \]
\end{definition}

\begin{lemma}
\lemlabel{uconcr-ctx}
  $\ctx \subseteq \ctx' \wedge \muconcr_{\ctx'}(\w) \neq \emptyset \Rightarrow
    \muconcr_{\ctx}(\w) \neq \emptyset$.
\end{lemma}

Pour tout objectif $P \in \Obj$ et tout contexte $\ctx \in \Ctx$, la \defref{maxCont}
fournit l'ensemble des processus de la sorte $\sorte{P}$ requis pour résoudre $P$ dans $\ctx$,
appelé $\gCont_\ctx(\PHsort(P), P)$.
\begin{definition}[$\gCont_\ctx : \Sigma \times \Obj \to \powerset(\PHproc)$]
\deflabel{maxCont}
  \begin{align*}
    \gCont_\ctx(a,P) \DEF
    \{ p \in \PHproc &\mid \exists ps \in \aBS(P), \exists b_i \in ps, b = a \wedge p = b_i \\
      & \vee b \neq a \wedge p \in \gCont_\ctx(a, \PHobj{b_j}{b_i}) \wedge b_j \in \PHget{\ctx}{b} \}
    \enspace.
  \end{align*}
\end{definition}

Pour une séquence d'objectifs $\w$ et un contexte $\ctx$ donnés,
le \emph{graphe de causalité locale} $\cwB$ (\defref{glc}) représente une sous-approximation de
l'atteignabilité de cette séquence d'objectifs dans $\ctx$.
Pour cela, il relie les objectifs à des solutions à
l'aide des séquences de bonds abstraites de la \defref{aBS}, ce qui produit de nouveaux objectifs
résolus récursivement.
Il s'agit donc d'un graphe dont les nœuds sont des éléments de $\Proc \cup \Obj \cup \Sol$,
c'est-à-dire des processus, des objectifs et des ensembles de processus qui ont le rôle
de \emph{solutions} :
\begin{itemize}
  \item Un nœud dans $\Obj$ représente un objectif requis pour la résolution de $\w$,
    soit faisant directement partie de la séquence d'objectifs $\w$,
    soit indirectement nécessaire à sa résolution ;
  \item Un nœud dans $\Sol$ représente un ensemble de processus nécessaires pour résoudre
    un objectif, c'est-à-dire un élément parmi ses séquences de bonds abstraites ;
  \item Un nœud dans $\Proc$ représente un processus requis pour la résolution,
    c'est-à-dire mentionné dans un nœud solution.
\end{itemize}

Un objectif $P \in \Obj$ est soluble si tous les processus contenus dans au moins une de ses
abstractions de séquences de bonds $\aBS(P) \in \Sol$ (\cf \defref{aBS})
peuvent être activés (\eqref{ESol1}).
Une telle solution représente donc un ensemble de processus qui doivent être activés
pour la résolution de $P$ (\eqref{ESol2}).
Si $a \in \components$, l'atteignabilité d'un de ses processus $a_i$ est approximée par
la possibilité de résoudre tous les objectifs de la forme $\PHobjp{a}{j}{i} \in \Obj$
pour tout $a_j$ dans le contexte initial $\ctx$ (\eqref{EReq}) ;
si $a \in \cs$, l'atteignabilité de $a_i$ est possible si tous les processus du sous-état
$\csState(a_i)$ (\cf \defref{csState}) qu'il représente sont atteignables (\eqref{EPrio}).
La résolution d'un objectif $P$ peut nécessiter un processus de $\PHsort(P)$,
autrement dit : $\gCont(\PHsort(P), P) \neq \emptyset$ (\cf \defref{maxCont}) ;
dans ce cas, $P$ est re-centré (\eqref{ECont}).
Enfin, \eqref{Vw,Vproc,VE} assurent que l'ensemble des nœuds est complet.

Étant donné que le processus actif de chaque sorte peut évoluer au cours de la résolution,
le graphe de causalité locale $\cwB$ est obtenu par saturation avec tous les processus
qu'il contient, \cad en recouvrant le contexte initial $\ctx$ par $\allprocs(\V, \E)$, défini par:
  \[\allprocs(\V, \E) = (V \cap \Proc) \cup
    \{ \PHtarget(P), \PHbounce(P) \mid P \in \V \cap \Obj \} \enspace.\]

\begin{definition}
\deflabel{glc}
  Le graphe de causalité locale $\cwB \DEF (\Bv, \Be)$ est défini par :
  $\cwB \DEF \lfp{\aB^\w_\ctx}{\myB}{\aB^\w_{\ctx \Cap \allprocs(\myB)}}$,
  où $\myB \DEF (\cwV, \cwE)$ est le plus petit graphe respectant
  $\cwV \subseteq \Proc \cup \Obj \cup \Sol$ et $\cwE \subseteq \cwV \times \cwV$
  tel que :
  \begin{align}
    \w &\subseteq \cwV \label{eq:Vw} \\
    P \in \cwVObj &\Rightarrow \PHbounce(P) \in \cwV \label{eq:Vproc} \\
    (x, y) \in \cwE &\Rightarrow y \in \cwV \label{eq:VE} \\
    P \in \cwVObj \wedge ps \in \BS(P) &\Rightarrow (P, ps) \in \cwE \label{eq:ESol1} \\
    ps \in \cwVSol \wedge a_i \in ps &\Rightarrow (ps, a_i) \in \cwE \label{eq:ESol2} \\
    a \in \components \wedge a_i \in \cwVProc \wedge a_j \in \ctx &\Rightarrow (a_i, \PHobjp{a}{j}{i}) \in \cwE \label{eq:EReq} \\
    a \in \cs \wedge a_i \in \cwVProc \wedge ps \in \csState(a_i) &\Rightarrow (a_i, ps) \in \cwE \label{eq:EPrio} \\
    P \in \cwVObj \wedge q \in \gCont_\ctx(\PHsort(P), P) &\Rightarrow (P, \PHobj{q}{\PHbounce(P)}) \in \cwE \label{eq:ECont} \!
  \end{align}
\end{definition}

Au sein de ce graphe de causalité locale, un arc $(p, ps) \in \Proc \times \Sol$
est dit \emph{cohérent} (\defref{coherent}) si aucun des processus dans $ps$
n'entre en conflit avec les processus fils du nœud $ps$,
\cad s'il n'existe pas un processus dans $ps$ et un processus parmi les fils de $ps$
qui soient différents mais appartenant à la même sorte.
Si tous les arcs du graphe sont cohérents, alors le \thmref{approxinf}
donne une condition suffisante pour la concrétisation de la séquence d'objectifs $\w$
dans le contexte $\ctx$, qui est basée directement sur ce graphe $\cwB$.

\begin{definition}[Arc cohérent]
\deflabel{coherent}
  Un arc $(x, y) \in \cwE$ est dit \emph{cohérent} si et seulement si :
  $(x, y) \in \Be \cap (\Proc \times \Sol) \Rightarrow y$ n'a aucun fils $a_j \in \Bv \cap \Proc$
  tel que $\exists a_i \in y, \sorte{a_i} = \sorte{a_j} \wedge a_i \neq a_j$.
\end{definition}

\begin{theorem}[Sous-Approximation]
\thmlabel{approxinf}
  Étant données des Frappes de Processus canoniques $\PH = (\PHs; \PHl; (\PHa^{(1)};\PHa^{(2)}))$,
  un contexte $\ctx \in \Ctx$ et une séquence d'objectifs $\w \in \OS$,
  si le graphe $\cwB$ ne contient aucun cycle,
  que tous ses nœuds objectifs possèdent au moins une solution
  et que tous ses arcs sont cohérents,
  alors $\uconcr(\w) \neq \emptyset$.
\end{theorem}


\todo{Recopier la preuve}



%\end{document}


\begin{comment}
\begin{example}
  Let $\PH' = (\PHs, \PHl, \PHh'^{\langle 1 \rangle})$ be the “merged” version of the PH in \pref{fig:ph-livelock}, that is: $\PHh'^{\langle 1 \rangle} = \PHh^{(1)} \cup \PHh^{(2)}$,
  which is how it would have been represented in the semantics without priorities.
  Due to spurious behaviours inherent to the cooperative sorts in this semantics, 
  the original under-approximation developed in~\cite{PMR12-MSCS} concludes that $c_1$ is reachable in $\PH'$ from $\ctx = \PHstate{a_1, b_0, c_0, ab_{10}}$.
  
  Such unwanted behaviours are palliated by the semantics of PH with priorities proposed in this paper.
  Indeed, the under-approximation given in \pref{th:approxinf} does not conclude regarding the reachability of $c_1$,
  as one of the edges of the resulting graph of local causality is not coherent (\pref{def:coherent}),
  as shown in in \pref{fig:sa-livelock}.
  (However, from the inconclusiveness of \pref{th:approxinf}, one cannot conclude about the unreachability of $c_1$.
  Such analysis should be driven for instance with over-approximation methods developed in~\cite{PMR12-MSCS}.)
  
  However, if $\PHhit{a_0}{b_0}{b_1}$ and $\PHhit{b_0}{a_0}{a_1}$ are replaced by the auto-actions
  $\PHhit{a_0}{a_0}{a_1}$ and $\PHhit{b_0}{b_0}{b_1}$,
  then \pref{th:approxinf} concludes that $c_1$ is reachable from $\ctx$.

\begin{figure}[tp]
  \centering
  \begin{tikzpicture}[aS]
    \node[Aproc] (c1) {$c_1$};
    \node[Aobj,below of=c1] (c01) {$\PHobj{c_0}{c_1}$};
    \node[Asol,below of=c01] (c01s) {};

    \node[AprocPrio,below of=c01s] (ab11) {$ab_{11}$};
    \node[AsolPrio,below of=ab11] (ab11s) {};

    \node[Aproc,below left of=ab11s] (a1) {$a_1$};
    \node[Aobj,below of=a1] (a11) {$\PHobj{a_1}{a_1}$};
    \node[Asol,below of=a11] (a11s) {};
    \node[Aobj,below left of=a1] (a01) {$\PHobj{a_0}{a_1}$};
    \node[Asol,below of=a01] (a01s) {};
    \node[Aproc,below of=a01s] (b0) {$b_0$};
    \node[Aobj,below of=b0] (b00) {$\PHobj{b_0}{b_0}$};
    \node[Asol,below of=b00] (b00s) {};
    \node[Aobj,below left of=b0] (b10) {$\PHobj{b_1}{b_0}$};
    \node[Asol,below of=b10] (b10s) {};

    \node[Aproc,below right of=ab11s] (b1) {$b_1$};
    \node[Aobj,below of=b1] (b11) {$\PHobj{b_1}{b_1}$};
    \node[Asol,below of=b11] (b11s) {};
    \node[Aobj,below right of=b1] (b01) {$\PHobj{b_0}{b_1}$};
    \node[Asol,below of=b01] (b01s) {};
    \node[Aproc,below of=b01s] (a0) {$a_0$};
    \node[Aobj,below of=a0] (a00) {$\PHobj{a_0}{a_0}$};
    \node[Asol,below of=a00] (a00s) {};
    \node[Aobj,below right of=a0] (a10) {$\PHobj{a_1}{a_0}$};
    \node[Asol,below of=a10] (a10s) {};

    \path
    (c1) edge (c01)
    (c01) edge (c01s)
    (c01s) edge (ab11)
    (ab11) edge[aSPrio] (ab11s)
    (ab11s) edge (a1) edge (b1)

    (a1) edge (a01) edge (a11)
    (a01) edge (a01s)
    (a01s) edge (b0)
    (a11) edge (a11s)
    (a0) edge (a10) edge (a00)
    (a10) edge (a10s)
    (a00) edge (a00s)

    (b0) edge (b10) edge (b00)
    (b10) edge (b10s)
    (b00) edge (b00s)
    (b1) edge (b01) edge (b11)
    (b01) edge (b01s)
    (b01s) edge (a0)
    (b11) edge (b11s)
    ;
    \end{tikzpicture}
  \caption{
  \label{fig:sa-livelock}
    The graph of local causality of the PH model in \pref{fig:ph-livelock}
    for the objective $\w = \PHobj{c_0}{c_1}$ and the initial context $\ctx = \PHstate{a_1, b_0, c_0, ab_{10}}$.
    Rectangular nodes containing a single process are elements in $\Proc$,
    borderless nodes containing a couple of processes are elements in $\Obj$
    and circle nodes are elements in $\Sol$.
    \pref{th:approxinf} is inconclusive on this example as edge $(ab_{11}, \{ a_1, b_1 \}) \in \Proc \times \Sol$ (here represented with a double line) is not coherent (\pref{def:coherent}).
    Indeed, $a_0 \in \Proc$ is a child of $\{ a_1, b_1 \}$, but $a_0 \neq a_1$ (and the same also stands for $b_0$).
  }
\end{figure}
\end{example}


\subsection{Reachability of a state}

\newcommand{\total}{\tau}
\newcommand{\reach}{\sigma}

The reachability property studied so far concerns a single process at a time.
However, we remark that the reachability of a partial or global state can be
addressed with the very same analysis by introducing a dedicated cooperative
sort in the scope of the PH semantics with $2$ classes of priorities.

Indeed, let $\PH = (\PHs, \PHl, (\PHh^{(1)}, \PHh^{(2)}))$ be a PH and suppose that we want to study the reachability of a state $s \in \PHl$.
Let $\PH' = (\PHs', \PHl', (\PHh'^{(1)}, \PHh'^{(2)}))$
with: $\PHs' = \PHs \cup \{ \total, \reach \}$ and $\PHl' = \PHl \times \PHl_\total \times \PHl_\reach$,
where $\total$ is a cooperative sort on all components $\components$ of $\PH$ (thus $\PHl_\total = \underset{a \in \components}{\times} \PHl_a$)
and $\reach$ is a component with $\PHl_\reach = \{ \reach_0, \reach_1 \}$;
furthermore, $\PHh'^{(1)}$ is the set $\PHh^{(1)}$ completed with all actions updating the cooperative sort $\total$,
and $\PHh'^{(2)} = \PHh^{(2)} \cup \{ \PHhit{\pfp_s(\total)}{\reach_0}{\reach_1} \}$.

Given an initial context $\ctx$, the reachability of $s$ in $\PH$ is equivalent to the concretisation of $\PHobjp{\reach}{0}{1}$ in $\PH'$ from the initial context $\ctx \cup \{ \reach_0 \}$ (the initial state of $\total$ does not matter), which can be efficiently under-approximated using \pref{th:approxinf}.
Indeed, the additional action $\PHhit{\pfp_s(\total)}{\reach_0}{\reach_1}$ in $\PHh'^{(2)}$ allows to conclude on the reachability of process $\pfp_s(\total)$, that is, on the reachability of the state $s$ (considering only the components).

It is also possible to compute the reachability of a set of states $S \subseteq \PHl$ by creating several actions $\PHhit{\total_s}{\reach_0}{\reach_1}$ in $\PHh^{(2)}$ for each state $s \in S$,
or on a sub-state $s \in \PHsubl[\PHl]_S$ of a set of components $S \subset \PHs$ by adapting the sort $\total$ with: $\PHl_\total = \PHsubl[\PHl]_S$


\subsection{Sequential under-approximation}
\label{ssec:ordered-ua}

In this section, we briefly explain an alternative sufficient condition that
progressively takes into account the successive objectives, instead of
considering all of them at a time, as it is done in the previous sections.
Because objectives are taken into account individually, such an approach
considers only a subset of scenarios.
However, because each iteration focuses on a smaller part of the network, this
sequential under-approximation may be more conclusive.

Let us define a sequence of objectives $\w=\obj{a_i}{a_j}\concat\w'$ with
$a_i\neq a_j$ and a state $s\in \PHl$ with $\get{s}{a}=a_i$.
One can remark that any scenario reaching $a_j$ necessarily includes one of the
bounce sequences in $\BS(\obj{a_i}{a_j})$, and, in particular,
any minimal scenario reaching $a_j$ ends in a state where $a_j$ is present but
also the hitter of the last bounce of one bounce sequence in $\BS(\obj{a_i}{a_j})$.
If such hitter $b_k$ is in a cooperative sort $b\in\cs$, it also means that one of the sub-states
in $\csState(b_k)$ is included in the ending state.
\pref{def:lastprocs} defines $\lastprocs(\obj{a_i}{a_j})$ as the set of set of
processes that may be present just after reaching $a_j$.

From \pref{th:approxinf}, we can deduce that
for any scenario $\delta$ in $\uconcr(P)$,
there exists a set of processes $ps\in\lastprocs(P)$
such that $ps \subset (s\play\delta)$.
Hence, if $\muconcr_{\ctx'\Cap ps}(\w')\neq\emptyset$,
with $\ctx'=\ctx\Cap\procs(\mycwB{\ctx}{P})$,
there exists a scenario $\delta'$ concretising $\w'$ from the
state $(s\play\delta)$.
Therefore, the scenario $\delta\concat\delta'$ concretises
$\w$.

\begin{definition}[$\lastprocs$]
\label{def:lastprocs}
Given an objective $\obj{a_i}{a_j}$, $\lastprocs:
\Obj\to\powerset(\powerset(\Proc))$ is
defined as the largest set such that, $\forall lps\in\lastprocs(\obj{a_i}{a_j})$, 
$lps\in\powerset(\Proc)$,
\begin{enumerate}
\item $a_j\in lps$;
\item $\exists\zeta\in\BS(\obj{a_i}{a_j}),
    \PHsort(\hitter{\zeta_{\card\zeta}}) \neq 
    \PHsort(\bounce{\zeta_{\card\zeta}})
     \Rightarrow
      \hitter{\zeta_{\card\zeta}}\in lps$;
\item $\forall b_j\in lps, b\in\cs, \exists ps\in\csState(b_j), ps\subset lps$;
\item $\nexists lps'\in\lastprocs(\obj{a_i}{a_j}), lps'\subset lps \wedge
                  lps'\neq lps$.
\end{enumerate}
\end{definition}

\begin{theorem}[Sequential under-approximation]
\label{thm:ordered-ua}
Given a Process Hitting $(\PHs; \PHl; \PHa^{\langle 2 \rangle})$
that satisfies \pref{cr:bounded} and \pref{cr:compcs},
a context $\ctx$ and an objective sequence $\w =
P\concat\w'\in\OS$,
$\uconcr(P)\neq\emptyset \wedge
  \forall ps \in\lastprocs(P),
  \muconcr_{\ctx'\Cap ps}(\w')\neq\emptyset
  \Longrightarrow \uconcr(\w)\neq\emptyset$,
where $\ctx' = \ctx\Cap\procs(\mycwB{\ctx}{P})$.
\end{theorem}
\begin{proof}
If $\uconcr(P)\neq\emptyset$,
for all $s\in \PHl, s\subset\ctx$,
there exists a scenario $\delta\in\uconcr(P)\cap\Sce(s)$;
from \pref{def:lastprocs} and proof of \pref{th:approxinf},
$\exists ps\in\lastprocs(P)$ such that
$(s\play\delta)\subset\ctx'\Cap ps$.
Hence, if $\muconcr_{\ctx'\Cap ps}(\w')\neq\emptyset$,
there exists a scenario $\delta'\in\muconcr_{\ctx'\Cap ps}(\w')$ such that
$\delta'\in\Sce(s\play\delta)$.
Hence, $\delta\concat\delta'$ is a scenario playable in $s$.
Therefore, for all $s\in \PHl, s\subset\ctx$, there exists a scenario
concretising $\w$.
Hence, $\uconcr(\w)\neq\emptyset$.
\end{proof}

\end{comment}
