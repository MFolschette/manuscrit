% Traduction en réseaux de Petri

\section{Traduction en réseaux de Petri}
\seclabel{trad-petri}

Nous présentons dans cette section une méthode de traduction des Frappes de Processus
avec actions plurielles en réseaux de Petri avec arcs inhibiteurs et arcs de lecture
\cite{Peterson77petrinets}.
Les réseaux de Petri, du nom de leur créateur \citefullname{petri1962kommunikation}{Carl Adam},
permettent une représentation compacte des systèmes discrets concurrents
sous la forme de places contenant des jetons qui représentent leur niveau d'expression,
et de transitions qui consomment et produisent ces jetons.

La traduction des Frappes de Processus en réseaux de Petri peut s'inscrire dans une volonté
plus générale d'utiliser ce formalisme pour représenter des processus de régulation
biologique \cite{Chaouiya07petrinet}.
De plus, cela peut permettre d'appliquer des méthodes de dépliage \cite{baldan00}
afin de réduire le parallélisme dans les modèles et faciliter en partie l'analyse
par \textit{model checking}.

La traduction proposée dans cette section fait correspondre, pour chaque sorte
des Frappes de Processus avec actions plurielles initiales, une place dans le réseau de Petri
produit.
Les jetons de cette place représentent alors le processus actif dans la sorte correspondante.
Les actions sont quand à elles représentées par une transition comportant les arcs ordinaires,
de lecture et inhibiteurs nécessaire à reproduire la dynamique recherchée.
Il est à noter que toutes les sémantiques de Frappes de Processus
(avec actions plurielles, arcs neutralisants ou classes de priorités,
mais aussi standards)
peuvent être traduites en réseaux de Petri à l'aide de la traduction proposée ici,
moyennant les traductions présentées aux \chapref{sem,phcanonique}.
Cela montre notamment que les Frappes de Processus sont expressivement incluses
dans les réseaux de Petri.
De plus, l'avantage de la traduction proposée ci-après est de ne représenter qu'une place
par sorte, d'être bornée
et d'utiliser le système d'accumulation des jetons pour représenter les processus,
même dans un cas multi-valué (non-booléen).
La bornitude que nous prouvons à la fin de cette section est intéressante car
cette propriété est indécidable dans le cas général \cite{Hack76} ;
de plus, cela permet de mieux délimiter l'expressivité de ce type de modèle
(les réseaux de Petri bornés avec arcs inhibiteurs n'étant pas plus expressifs
que ceux qui n'en comportent pas).
Cette traduction se distingue donc de celle proposée par \citeasnoun[p.~47]{Pauleve11}
qui fait correspondre une place à chaque processus,
et produit donc un réseau de Petri sauf et n'utilisant pas d'arcs inhibiteurs.

\myskip

Nous commençons par rappeler la notion de réseau de Petri avec des arcs de lecture
et des arcs inhibiteurs (\defref{pt}).
Un réseau de Petri est composé de \emph{places} et de \emph{transitions}.
Chaque place contient à tout instant un certain nombre de \emph{jetons} qui tiennent le rôle
de niveau d'expression courant de la place.
Les transitions, quant à elles, sont reliées aux places par des \emph{arcs ordinaires},
des \emph{arcs de lecture} et des \emph{arcs inhibiteurs} ;
ces arcs déterminent la condition de tir de la transition en fonction des places auxquelles
elle est reliée, ainsi que le nombre de jetons consommés et produits par son tir.
Les arcs ordinaires permettent d'établir le nombre de jetons consommés
et produits par le tir d'une transition, tandis que
les arcs de lecture permettent de s'assurer de la présence d'un certain nombre de jetons
dans une place et les arcs inhibiteurs permettent de s'assurer de la propriété
inverse.
Il est donc possible de passer d'un état (appelé \emph{marquage}) à un autre
en tirant une transition, et à la condition de respecter ces règles (\defref{pt-sem}).

\begin{definition}[Réseau de Petri ($\PT$)]
\deflabel{pt}
  Un \emph{réseau de Petri} est un 6-uplet
  $\PT = (\PTp; \PTt; \PTPre; \PTPost; \PTLect; \PTInh)$ où :
  \begin{itemize}
    \item $\PTp$ est l'ensemble fini des \emph{places},
    \item $\PTt$ est l'ensemble fini des \emph{transitions},
    \item $\PTPre \subset \PTp \times \sN \times \PTt$
%    \item $\PTPre : \PTp \times \PTt \rightarrow \sN$
      est l'ensemble des ordinaires arcs entrant dans une transition,
    \item $\PTPost \subset \PTt \times \sN \times \PTp$
      est l'ensemble des arcs ordinaires sortant d'une transition,
    \item $\PTLect \subset \PTp \times \sN \times \PTt$ est l'ensemble des arcs de lecture,
    \item $\PTInh \subset \PTp \times \sN \times \PTt$ est l'ensemble des arcs inhibiteurs,
  \end{itemize}
  avec : $\PTp \cap \PTt = \emptyset$ et $\PTp \cup \PTt \neq \emptyset$.
  De plus, chaque arc est unique ; autrement dit, pour chacun des ensembles
  $X \in \{ \PTPre, \PTPost, \PTLect, \PTInh \}$, on a :
  \[\forall (a; i; b) \in X, \forall (a'; i'; b') \in X,
    (a = a' \wedge b = b') \Rightarrow i = i'\]
  On note par ailleurs, pour tout couple d'éléments $a, b \in \PTp \cup \PTt$ :
  $X(p, t) = i \text{ si } (a; i; b) \in X \text{ ou } 0 \text{ sinon}$.
%   \[X(p, t) = \begin{cases}
%               i & \text{ si } (a; i; b) \in X \\
%               0 & \text{ sinon}
%             \end{cases}\]
  
  Un \emph{marquage} d'un réseau de Petri est une fonction $M : \PTp \rightarrow \sN$.
  Un \emph{réseau de Petri avec marquage initial} est un couple $(\PT; M_0)$
  où $\PT$ est un réseau de Petri et $M_0$ est un marquage de $\PT$.
%   
%   Pour toute transition $t \in \PTt$, on note de plus :
%   $\PTPre(t) = $
\end{definition}

\begin{definition}[Sémantique des réseaux de Petri ($\PTtrans$)]
\deflabel{pt-sem}
  Étant donnés un réseau de Petri $\PT = (\PTp; \PTt; \PTPre; \PTPost; \PTLect; \PTInh)$
  et un marquage $M$,
  une transition $t \in \PTt$ est dite \emph{jouable} dans $M$ si et seulement si :
  \[\forall p \in \PTp, M(p) \geq \PTPre(p, t) \wedge M(p) \geq \PTLect(p, t)
    \wedge M(p) < \PTInh(p, t)\]
  Dans ce cas, on note $(M \play t)$ le marquage résultant du jeu de cette transition depuis $M$,
  défini par :
  \[\forall p \in \PTp, (M \play t)(p) = M(p) - \PTPre(p, t) + \PTPost(t, p)\]
  De plus, on note alors : $M \PTtrans (M \play t)$.
\end{definition}

Nous proposons à la \defref{phm2pt} une traduction des Frappes de Processus avec actions
plurielles vers les réseaux de Petri tels que définis plus haut.
Cette définition fait correspondre, à chaque sorte du modèle initial,
une place dans le réseau de Petri produit,
dont le nombre de jetons figurera le niveau d'activité.
De même à chaque action correspond une transition, dont le déclenchement
est contraint par des arcs de lecture et des arcs inhibiteurs pour s'assurer
que le nombre de jetons des places correspondant aux frappeurs est le bon.
Enfin, des arcs ordinaires permettent de consommer et produire les jetons
nécessaires pour représenter tous les bonds.

Il est à noter que pour cette traduction, les indices des processus ont un sens particulier
car ils représentent le nombre de jetons utilisés pour représenter chaque processus.
Par exemple, une sorte hypothétique $a$ comprenant trois processus $a_0$, $a_1$ et $a_2$
sera représentée par une place pouvant structurellement contenir jusqu'à $2$ jetons
(cette contrainte n'étant pas inhérente au modèle mais à la forme des arcs de la traduction).
Ainsi, les indices des processus du modèle initial doivent être choisis dans $\sN$.

Nous montrons ensuite au \thmref{bisimulationphm2pt} que cette traduction
conserve la dynamique ;
autrement dit, que le modèle traduit est fortement bisimilaire au modèle original.

\begin{definition}[Réseau de Petri équivalent ($\phmtoptsymbol$)]
\deflabel{phm2pt}
  Le réseau de Petri équivalent aux Frappes de Processus
  avec actions plurielles $\PH = (\PHs; \PHl; \PHh)$
  est défini par : $\phmtoan = (\PTp; \PTt; \PTPre; \PTPost; \PTLect; \PTInh)$, où :
  \begin{itemize}
    \item $\PTp = \PHs$,
    \item $\PTt = \PHh$,
    \item $\PTPre = \{ (b, j-k, h) \mid h \in \PHh \wedge b_j \in \cible{h} \wedge
      b_k \in \bond{h} \wedge k-j < 0 \}$
    \item $\PTPost = \{ (h, k-j, b) \mid h \in \PHh \wedge b_j \in \cible{h} \wedge
      b_k \in \bond{h} \wedge k-j > 0 \}$
    \item $\PTLect = \{ (a, i, h) \mid h \in \PHh \wedge a_i \in \frappeur{h} \wedge i > 0 \}$
    \item $\PTInh = \{ (a, i+1, h) \mid h \in \PHh \wedge a_i \in \frappeur{h} \wedge i < l_a \}$
  \end{itemize}
  De plus, pour tout état $s \in \PHl$, on note
  $M^s$ le marquage correspondant, défini par :
  $\forall a \in \PTp, M^s(a) = i \text{ tel que } \PHget{s}{a} = a_i$.
\end{definition}

\begin{theorem}[$\PH \approx \phmtopt$]
\thmlabel{bisimulationphm2pt}
  Soient $\PH = (\PHs; \PHl; \PHh)$ des Frappes de Processus avec actions plurielles.
  On a :
  \[\forall s, s' \in \PHl, s \PHtrans s' \Longleftrightarrow
    M^s \trans{\phmtopt} M^{s'} \enspace.\]
\end{theorem}

\begin{proof}
  Soient $s, s' \in \PHl$.
  On pose : $\phmtopt = (\PTp; \PTt; \PTPre; \PTPost; \PTLect; \PTInh)$.
  Pour toute action $h \in \PHh$, on notera dans la suite :
  \begin{itemize}
    \item $\PTPre_h = \{ (b, j-k, h) \mid b_j \in \cible{h} \wedge
      b_k \in \bond{h} \wedge k-j < 0 \}$,
    \item $\PTPost_h = \{ (h, k-j, b) \mid b_j \in \cible{h} \wedge
      b_k \in \bond{h} \wedge k-j > 0 \}$,
    \item $\PTLect_h = \{ (a, i, h) \mid a_i \in \frappeur{h} \wedge i > 0 \}$,
    \item $\PTInh_h = \{ (a, i+1, h) \mid a_i \in \frappeur{h} \wedge i < l_a \}$.
  \end{itemize}
  
  ($\Rightarrow$) Supposons que $s \PHtrans s'$, c'est-à-dire qu'il existe une action $h \in \PHh$
    telle que $s' = s \play h$.
    Posons : $h = \PHfrappemult{A}{B}$.
    D'après la \defref{phm2pt},
    l'existence de cette action dans $\PH$ implique celle d'une transition $h \in \PTt$
    dans $\phmtopt$, ainsi que des arcs suivants :
    $\PTPre_h \subset \PTPre$, $\PTPost_h \subset \PTPost$,
    $\PTLect_h \subset \PTLect$ et $\PTInh_h \subset \PTInh$.
    Or, étant donné que $h$ est jouable dans $s$, on a :
    $\forall a_i \in \frappeur{h}, a_i \in s$, d'où :
    $\forall a \in \sortes{\frappeur{h}}, M^s(a) = i$.
    De plus, comme $\cible{h} \subset \frappeur{h}$, on en déduit :
    $\forall b \in \sortes{\cible{h}}, M^s(b) = j > j-k$.
    Ainsi, $h$ est jouable dans $\PT$ d'après la \defref{pt-sem}.
    De plus, d'après cette même définition,
    $\forall b \in \sortes{\bond{h}}, (M^s \play h)(b) = j + (k - j) = k$
    et $\forall a \in \PTp \setminus \sortes{\bond{h}}, (M^s \play h)(a) = M^s(a)$.
    Ainsi, $(M^s \play h) = M^{s \play h} = M^{s'}$.
  
  ($\Leftarrow$) Supposons que $M^s \trans{\phmtopt} M^{s'}$,
    c'est-à-dire qu'il existe une transition $h \in \PTt$ telle que $M^{s'} = M^s \play h$.
    D'après la \defref{phm2pt}, il existe alors
    une action $h = \PHfrappemult{A}{B} \in \PHh$ dans $\PH$,
    ainsi que les ensembles d'arcs suivants dans $\phmtopt$ :
    $\PTPre_h \subset \PTPre$, $\PTPost_h \subset \PTPost$,
    $\PTLect_h \subset \PTLect$ et $\PTInh_h \subset \PTInh$.
    Comme $h$ est jouable dans $M$, cela signifie alors, d'après la \defref{pt-sem}, que :
%     \[\forall p \in \PTp, M^s(p) \geq \PTPre(p, t) \wedge M^s(p) \geq \PTLect(p, t)
%       \wedge M^s(p) < \PTInh(p, t)\]
    $\forall p \in \frappeur{h}, M^s(p) = \PHget{s}{p}$
    (ainsi que : $\forall p \in \cible{h}, M^s(p) = \PHget{s}{p} \geq \PTPre(p, t)$).
    Ainsi, $h$ est jouable dans $s$ car $a \subseteq s$.
    Par ailleurs, $s \play h = s \recouvre B$ ;
    or, $\forall b \in \sortes{\bond{h}}, (M^s \play h)(b) = j + (k - j) = k$
    et $\forall a \in \PTp \setminus \sortes{\bond{h}}, (M^s \play h)(a) = M^s(a)$.
    Ainsi, $s \play h = s'$.
\end{proof}

Pour terminer, nous montrons au \thmref{bornephm2pt} que le réseau de Petri
obtenu par la traduction d'un modèle de Frappes de Processus avec actions plurielles
d'après la \defref{phm2pt} est borné si on lui associe un marquage
initial correspondant à un état possible des Frappes de Processus considérées.
En réalité, un tel réseau de Petri est toujours borné, car il n'est pas possible
d'ajouter des jetons à l'infini dans une place ;
cependant, nous limitons notre résultat aux marquages correspondant à des états possibles
pour des raisons de simplicité.
Ce résultat est important car il en découle que certaines propriétés deviennent décidables,
ce qui n'est pas le cas sur des réseaux de Petri non bornés s'ils comportent des arcs inhibiteurs
\cite{roux04}.

\begin{theorem}[Bornitude de $\phmtopt$]
\thmlabel{bornephm2pt}
  Pour toutes Frappes de Processus avec actions plurielles $\PH = (\PHs; \PHl; \PHh)$
  et tout état $s \in \PHl$,
  $(\phmtopt; M^s)$ est $k$\nbd borné,
  avec $k = \max_{a \in \PHs}(\card{\PHl_a})$.
\end{theorem}

\begin{proof}
  D'après le \thmref{bisimulationphm2pt}, $\PH$ et $\phmtopt$ sont bisimilaires.
  Autrement dit, le marquage de toute place $a \in \PTp$ ne peut pas dépasser
  son plafond $l_a$ dans $\PH$.
\end{proof}



\todo{Exemple}
