
\chapter{Enrichissement des Frappes de Processus}

\chapeau{%
La sémantique standard des Frappes de Processus de la \secref{ph} peut s'avérer insuffisante pour
prendre en compte certaines informations connues sur le système étudié, comme des informations
en terme de vitesse de réaction.
Ce chapitre propose d'étendre sa sémantique afin d'enrichir les modèles sur deux axes :
\begin{itemize}
  \item la préemption entre actions, qui permet d'empêcher le jeu d'une action sous certaines
    conditions,
  \item la simultanéité d'actions, qui permet de forcer le jeu simultané de plusieurs actions.
\end{itemize}
Ces deux axes ont pour but de permettre ou de faciliter l'intégration de telles informations au sein
des modèles.

Les trois extensions aux Frappes de Processus proposées dans ce chapitre prennent la forme
de classes de priorités, d'arcs neutralisants et d'actions plurielles.
Nous montrons de plus que les trois modèles qui en résultent sont (faiblement) bisimilaires.
Enfin, il est à noter que l'analyse statique développée au chapitre \todo{suivant}
peut s'appliquer à toutes ces extensions, les classes de priorités étant d'ailleurs réutilisées
pour définir la forme canonique des Frappes de Processus.
}

La sémantique standard des Frappes de Processus telle que rappelée à la \secref{ph}
ne permet pas d'inclure de contraintes dérivées d'informations biologiques au sein des modèles créés,
comme les vitesses de réaction, la connaissance d'inhibitions entre composants et réactions,
ou la connaissance des réactions précises ayant lieu au sein du système.
Pourtant, la connaissance de telles informations peut permettre de privilégier un chemin sur un autre
à d'un moment clef de l'évolution du système, empêchant ainsi une certaine évolution du système,
en favorisant l'apparition d'une autre.
Leur intégration permet donc d'affiner le modèle en réduisant les comportements possibles
afin d'obtenir un modèle plus proche du système étudié.
Ces connaissances peuvent être intégrées sous la forme de préemptions (la jouabilité d'une action
peut empêcher la jouabilité d'une autre action) ou de simultanéité
(plusieurs processus peuvent évoluer simultanément).

De même, au niveau de la modélisation, la représentation des coopérations avec des Frappes de
Processus standards souffre de certaines lacunes.
Ces coopérations sont modélisées à l'aide de sortes coopératives telles que décrites à la \secref{sc},
%dont la dynamique n'est pas strictement équivalente à celle d'une porte logique autorisant
%le jeu d'une action en présence de certains processus donnés.
et souffrent d'un décalage temporel entre les sortes à représenter et
la mise à jour du processus actif de la sortie coopérative,
qui peut entraîner l'existence de «~faux états~» pour la sorte coopérative.
%Ce décalage temporel implique que ce processus actif représente toujours
%une combinaison d'états passés des sortes en amont.
%S'il s'avère que cette combinaison coïncide dans la majeure partie des cas avec un état passé
%effectif,
%ou avec l'état présent, cela n'est pas toujours le cas, et il est possible qu'une sorte coopérative
%représente une combinaison d'états en pratique inaccessible pour les sortes en amont.
Plusieurs solutions sont proposées dans ce chapitre, qu'il s'agisse de rendre les actions de mise à
jour des sortes coopératives prioritaires ou plus simplement de les remplacer par une forme plus
complexe d'action.

L'une des pistes permettant d'affiner un modèle de Frappes de Processus consiste à y intégrer
des informations de préemption entre les actions afin d'affiner la dynamique.
Une telle approche permet de modéliser des contraintes temporelles,
toute action modélisant une réaction très rapide étant par exemple systématiquement jouée
avant les actions modélisant des réactions plus lentes.
D'autres contraintes peuvent aussi être prises en compte, comme la concurrence entre réactions,
mais aussi la représentation de processus propres à la modélisation et n'ayant pas nécessairement
de sens biologique.

Les formes alternatives de Frappes de Processus présentées dans ce chapitre se concentrent donc
sur les notions de préemption et de simultanéité d'une action par rapport à une autre.
La préemption permet à une action d'avoir priorité sur une autre ou, du point de vue inverse,
permet d'empêcher le jeu d'une action dans une situation où elle pourrait normalement être jouable
selon la dynamique des Frappes de Processus standards.
Une telle préemption peut être opérée de façon généralisée
%par une action sur un ensemble d'autres,
comme c'est le cas pour les Frappes de Processus avec classes de priorités (\secref{php}),
où chaque action peut bloquer l'ensemble des actions de priorité inférieure ;
ou de façon ponctuelle, comme au sein des Frappes de Processus avec arcs neutralisants (\secref{phan}),
qui permet de définir des relations plus fines de préemption entre actions individuelles.
À l'inverse, la simultanéité entre actions permet de s'assurer qu'un ensemble de frappes est joué
de façon simultanée, ou plus généralement qu'un ensemble de processus bondit en même temps,
comme permettent de le représenter les Frappes de Processus avec actions plurielles (\secref{phm}).
Nous montrons au cours de ce chapitre que ces différentes sémantiques sont deux à deux aussi
expressives \todo{ref}.

Les apports de ces formes alternatives de Frappes de Processus permettent de restreindre la dynamique
d'un modèle par rapport aux Frappes de Processus standards.
Elles se posent en alternatives à l'ajout de paramètres stochastiques dans les Frappes de Processus
\todo{ref},
qui permettent d'ajouter une dimension probabiliste dans ce formalisme.
Leur premier avantage est de simplifier l'écriture et la lecture d'un modèle selon les informations
que l'on souhaite y intégrer.
Leur second avantage est la possibilité d'appliquer les méthodes d'analyse statique
qui sont étendues et adaptées à ces formalismes au chapitre \todo{suivant}.



\section{Enrichissement par préemption des actions}

% Frappes de Processus avec classes de priorités
\section{Frappes de Processus avec classes de priorités}
\seclabel{php}

Pour tout entier naturel $k$ non nul,
les \emph{Frappes de Processus avec $k$ classes de priorités}
%(aussi appelées «~Frappes de Processus~» dans la suite, lorsque ce n'est pas ambigu),
sont des Frappes de Processus dont l'ensemble des actions est partitionné
en $k$ ensembles, chacun étant associé à une classe de priorité distincte.
Cela signifie qu'une action est jouable dans un état si et seulement si,
en plus de la condition de la présence du frappeur et de la cible,
il n'existe aucune autre action appartenant à une classe de priorité plus grande
qui soit aussi jouable dans cet état.
Il est à noter que les classes de priorités sont étiquetées de façon décroissante par les entiers de
l'ensemble $\segm{1}{k}$ en fonction de l'importance de la priorité ;
autrement dit, la classe de priorités 1 contient les actions les plus prioritaires,
ne pouvant jamais être préemptées,
tandis que la jouabilité d'une action de la classe de priorité $k$ ne peut pas empêcher le jeu
d'une autre action.
Un exemple de ce type de modèle est donné par la \figref{metazoan-php},
où les différentes priorités sont signifiées par des étiquettes numérotées sur les actions.

Cette modélisation permet notamment de distinguer les actions en fonction 
de différents critères comme
leur vitesse d'exécution (les actions les plus rapides étant jouées en priorité),
ou tout autre paramètre permettant de déterminer l'existence de la préemption d'une action
en fonction de la possibilité d'en jouer une autre.
L'application la plus poussée de cette utilisation consisterait à
classer les actions d'un modèle en fonction d'un tel critère,
et à attribuer à chaque classe de priorité une action unique en fonction de ce classement.
De cette manière,
les actions seraient arrangées selon un ordre total défini par leurs priorités.
%afin de rendre compte de la priorité de chaque action en fonction de chaque autre.

De même, ces classes de priorités permettent de prendre en compte des comportements non biologiques
inhérents à la modélisation.
Il est par exemple possible de donner une priorité différente aux
actions qui n'ont pas de sens biologique propre
---~mais dont ce sens émerge uniquement dans leur relation avec d'autres actions.
L'application la plus immédiate de ce cas est celle des actions de mise à jour
des sortes coopératives, comme cela est développé au \chapref{phcanonique},
où une classe de priorités supérieure offre l'avantage de supprimer les effets d'entrelacement,
et ainsi de simuler le comportement d'une véritable porte logique
sans le problème de décalage temporel soulevé à la \secref{sc}.

Cette représentation basée sur des classes de priorités permet de modéliser un système
dont les actions peuvent être distinguées en plusieurs classes en fonction de leur importance,
de leur vitesse d'exécution, ou encore d'autres facteurs leur donnant prévalence sur d'autres.
Chaque action peut donc en préempter un ensemble d'autres en fonction de sa classe de priorité.
Cela permet une représentation compacte des rapports de priorités entre actions
ou, autrement dit, de leur ordonnancement,
qui présente néanmoins quelques lacunes.
Les phénomènes d'accumulation, notamment, n'y sont pas représentés ;
un cycle d'actions prioritaires ne peut jamais être interrompu par une action moins prioritaire,
menant à un cycle infini et pouvant contredire la réalité biologique.
De plus, les classes de priorités définies pour un modèle sont invariables;
certains modèles pourraient cependant nécessiter l'évolution de certaines classes de priorités
en fonction de la présence ou de l'absence d'un composant dans un état donné.
Enfin, elles peuvent ne pas permettre la précision nécessaire à une représentation fidèle de
certains modèles, notamment lorsqu'il est nécessaire de définir des préemptions ponctuelles
comme le permettent les Frappes de Processus avec arcs neutralisants
présentées à la \secref{phan}.



\subsection{Définition}
\seclabel{php-def}

\begin{definition}[Frappes de Processus avec $k$ classes de priorités]
\deflabel{php}
  Si $k \in \sN^*$, les \emph{Frappes de Processus avec $k$ classes de priorités} sont définies
  par un triplet $\PH = (\PHs; \PHl; \PHh^{\langle k \rangle})$,
  où $\PHh^{\langle k \rangle} = (\PHh^{(1)}; \dots; \PHh^{(k)})$ est un $k$-uplet, et :
  \begin{itemize}
    \item $\PHs \DEF \{a, b, \dots\}$ est l'ensemble fini et dénombrable des \emph{sortes} ;
    \item $\PHl \DEF \bigtimes{a \in \PHs} \PHl_a$ est l'ensemble fini des \emph{états},
      où $\PHl_a = \{a_0, \ldots, a_{l_a}\}$ est l'ensemble fini et dénombrable
      des \emph{processus} de la sorte $a \in \PHs$ et $l_a \in \sN^*$.
      Chaque processus appartient à une unique sorte :
      $\forall (a_i; b_j) \in \PHl_a \times \PHl_b, a \neq b \Rightarrow a_i \neq b_j$ ;
    \item pour tout $n \in \llbracket 1; k \rrbracket$,
      $\PHh^{(n)} \DEF \{\PHfrappe{a_i}{b_j}{b_l} \mid (a; b) \in \PHs^2 \wedge
      (a_i; b_j; b_l) \in \PHl_a \times \PHl_b \times \PHl_b \wedge
      b_j \neq b_l \wedge a = b \Rightarrow a_i = b_j \}$ est l'ensemble fini
      des \emph{actions de priorité $n$}.
  \end{itemize}
  On note $\PHh \DEF \bigcup_{n \in \segm{1}{k}} \PHh^{(n)}$ l'ensemble de toutes les actions
  et, pour tout $n \in \sN^*$ et $h \in \PHh^{(n)}$, $\prio(h) \DEF n$.
\end{definition}
%
\noindent
On réutilise de surcroît les notations définies à la \secref{ph} concernant les états et
l'extraction de la sorte d'un processus.

%La sorte d'un processus $a_i$ est donnée par $\PHsort(a_i) = a$.
%Étant donné un état $s \in \PHl$, le processus de la sorte $a \in \PHs$ présent dans $s$ est donné
%par $\PHget{s}{a}$, \cad la coordonnée correspondant à $a$ dans l'état $s$.
%Si $a_i \in \PHl_a$, nous définissons la notation : $a_i \in s \EQDEF \PHget{s}{a} = a_i$.

À l'instar de la \secref{ph}, il faut définir un opérateur de jouabilité
pour déterminer la dynamique des Frappes de Processus avec $k$ classes de priorités.
Cependant, à l'inverse de celui des Frappes de Processus standards (\defref{fopph})
il faut ici prendre en compte la possible présence d'actions jouables appartenant à des
classes de priorités supérieures.
Pour cela, il est suffisant de vérifier que le frappeur et la cible de toute action
de priorité plus importante ne sont pas simultanément présents.
En effet, prenons deux actions $g, h \in \PHh$ avec : $\prio(g) < \prio(h)$, et
un état $s \in \PHl$ tel que $\frappeur{g} \in s \wedge \cible{g} \in s$ ;
Deux cas de figures sont alors possibles :
\begin{itemize}
  \item l'action $g$ est jouable dans $s$ --- autrement dit, aucune autre action de priorité plus
    importante ne la préempte -- et elle préempte $h$ en conséquence,
  \item l'action $g$ n'est pas jouable dans $s$, ce qui signifie qu'elle est préemptée par une
    action de priorité plus importante, qui préempte alors aussi l'action $h$.
\end{itemize}
Dans les deux cas, $h$ n'est pas jouable, ce qui montre qu'il est suffisant de n'observer
que la présence simultanée du frappeur et de la cible de chaque action de priorité supérieure
pour déterminer la jouabilité de $h$.
Nous obtenons alors l'opérateur de jouabilité donné à la \defref{fopphp}.

\begin{definition}[Opérateur de jouabilité ($\Fopsymbol_\Fopphpsubsymbol : \PHh \rightarrow \F$)]
\deflabel{fopphp}
  L'opérateur de jouabilité des Frappes de Processus avec $k$ classes de priorités est défini par :
  \[\forall h \in \PHh, \Fopphp{h} \equiv \hitter{h} \wedge \target{h} \wedge
    \left( \bigwedge_{\substack{g \in \PHh^{(n)}\\n < \prio(h)}}
    \neg \left( \hitter{g} \wedge \target{g} \right) \right)\]
\end{definition}



\begin{example}
\exlabel{metazoan-php}
  Nous illustrons les possibilités offertes par l'introduction des classes de priorités
  par la \figref{metazoan-php} qui représente un modèle de Frappes de Processus
  avec 3 classes de priorités $\PH = (\PHs, \PHl, \PHh^{\angles{3}})$.
  Celui-ci reprend la structure du modèle de Frappes de Processus standards
  de la \figref{metazoan-ph},
  en y ajoutant trois classes de priorités permettant de distinguer trois types d'actions :
  \begin{itemize}
    \item les actions de $\PHh^{(1)}$ permettent d'assigner une priorité maximale
      aux actions mettant à jour la sorte coopérative $fc$,
      et peuvent être considérées comme «~instantanées~» du point de vue du reste du modèle ;
    \item les actions de $\PHh^{(2)}$ assurent que la sorte $a$ est mise à jour immédiatement
      en fonction de l'évolution de $f$ et $c$,
      et peuvent être vues comme «~urgentes~» par rapport aux actions de $\PHh^{(3)}$ ;
    \item enfin, les actions restantes sont par conséquent considérées comme «~lentes~»
      ou «~peu urgentes~» en regard du reste du modèle ;
      il s'agit des actions de $\PHh^{(3)}$, qui représentent des processus biologiques
      plus lents.
  \end{itemize}
  Comme expliqué par la suite au \chapref{phcanonique},
  assigner la priorité maximale aux actions permettant la mise à jour des
  sortes coopératives permet d'éviter les comportements indésirables
  décrits à la page \expageref{metazoan-ph-nocoop}.
  De même, accorder aux actions de $\PHh^{(2)}$ le statut d'«~urgentes~»
  permet de s'assurer qu'elles seront jouées avant les actions de $\PHh^{(3)}$.
  Dans ce modèle, cela se traduit par le fait que l'activation ou la désactivation de $a$
  est forcée lorsque $c$ et $f$ évoluent,
  ce qui restreint la dynamique aux seuls comportements désirés.
  En effet, la seule dynamique possible, en partant de l'état initial
  $\etat{a_0, c_0, f_1, fc_{10}}$,
  consiste en un comportement stationnaire oscillant,
  où $c$ et $a$ sont alternativement activés et désactivés,
  interrompu par la désactivation de $f$ qui entraîne irrémédiablement celle de $c$,
  sans possibilité de le ré-activer par la suite,
  et provoque cette fois un comportement stationnaire constant
  (qui se traduit en Frappes de Processus par un point fixe
  où $a$ reste indéfiniment à sa dernière valeur (actif ou non).
  Le comportement stationnaire est donné par le scénario suivant,
  jouable dans l'état initial $\etat{a_0, c_0, f_1, fc_{10}}$ :
  \begin{align*}
    \PHfrappe{fc_{10}}{a_0}{a_1} &\cons
    \PHfrappe{f_1}{c_0}{c_1} \cons
    \PHfrappe{c_1}{fc_{10}}{fc_{11}} \cons \\
    &\PHfrappe{c_1}{a_1}{a_0} \cons
    \PHfrappe{c_1}{c_1}{c_0} \cons
    \PHfrappe{c_0}{fc_{11}}{fc_{10}}
  \end{align*}
  L'interruption de ce comportement stationnaire se fait grâce à l'auto-action
  $\PHfrappe{f_1}{f_1}{f_0}$, qui est de priorité 3, et donc jouable uniquement
  dans les deux états suivants :
  $\etat{a_1, c_0, f_1, fc_{10}}$ et $\etat{a_0, c_1, f_1, fc_{11}}$.
  Depuis le premier état, la désactivation est opérée par le scénario suivant :
  \[
    \PHfrappe{f_1}{f_1}{f_0} \cons
    \PHfrappe{f_0}{fc_{10}}{fc_{00}}
    \enspace,
  \]
  qui termine dans l'état $\etat{a_1, c_0, f_0, fc_{00}}$ et conserve donc le processus $a_1$,
  tandis que depuis le deuxième état, la désactivation est opérée par le scénario :
  \[
    \PHfrappe{f_1}{f_1}{f_0} \cons
    \PHfrappe{f_0}{fc_{11}}{fc_{01}} \cons
    \PHfrappe{f_0}{c_1}{c_0} \cons
    \PHfrappe{c_0}{fc_{01}}{fc_{00}}
    \enspace,
  \]
  qui termine en $\etat{a_0, c_0, f_0, fc_{00}}$ et conserve cette fois le processus $a_0$.
  
  \begin{figure}[ht]
  \begin{center}
  \begin{tikzpicture}
    \exmetazoan
    
    \node[labelprio1] at (2.55,3.85) {$1$}; % c => fc
    \node[labelprio1] at (2.75,1) {$1$};    % f => fc
    \node[labelprio2] at (5.5,3.85) {$2$};  % fc_10 -> a_0 / 1
    \node[labelprio2] at (3.5,5.3) {$2$};   % c_1 -> a_1 / 0
    \node[labelprio3] at (0,2.5) {$3$};     % f_1 -> c_0 / 1
    \node[labelprio3] at (0.8,5.8) {$3$};   % c_1 -> c_1 / 0
    \node[labelprio3] at (2.15,2.5) {$3$};  % f_0 -> c_1 / 0
    \node[labelprio3] at (1.5,1.8) {$3$};   % f_1 -> f_1 / 0
    
    \TState{f_1, a_0, c_0, fc_2}
  \end{tikzpicture}
  \caption{\figlabel{metazoan-php}%
    Exemple de Frappes de Processus avec 3 classes de priorités.
    Ce modèle est issu de celui de la \figref{metazoan-ph}
    auquel ont été rajoutées des classes de priorités.
    Les étiquettes numérotées (de 1 à 3) placées contre les flèches représentant les actions
    symbolisent leur appartenance à une classe de priorités donnée ;
    ainsi, on a notamment :
    $\PHh^{(2)} = \{ \PHfrappe{fc_{10}}{a_0}{a_1} ; \PHfrappe{c_1}{a_1}{a_0} \}$.
  }
  \end{center}
  \end{figure}
\end{example}



\subsection{Équivalences entre Frappes de Processus avec $k$ classes de priorités}

Nous montrons à la \secref{aplatissement} que les Frappes de Processus avec $k$ classes
de priorités sont aussi expressives que les Frappes de Processus avec $n$ classes de
priorités, pour tout $k, n \in \sNN$.
Nous donnons pour cela un résultat encore plus fort : tout modèle de Frappes de Processus avec $k$
classes de priorités peut être traduit en Frappes de Processus canoniques,
comme défini à la \secref{phcanonique-def},
qui sont des Frappes de Processus avec 2 classes de priorités
avec une forme particulière.
À l'inverse, les Frappes de Processus avec 2 classes de priorités sont
naturellement à fortiori
des Frappes de Processus avec $k$ classes de priorités, pour tout $k \in \sNN$.

Nous notons cependant que ce résultat n'inclut pas les Frappes de Processus
avec 1 classe de priorité (c'est-à-dire les Frappes de Processus standards).
En effet, il est intuité que leur expressivité est strictement moindre
que les Frappes de Processus possédant plusieurs classes de priorités,
mais nous ne démontrons pas ce résultat ici.

\begin{theorem}[Équivalences entres Frappes de Processus avec classes de priorités]
\thmlabel{equivphpkn}
  Pour tout $k, n \in \sNN$, les Frappes de Processus avec $k$ classes de priorités
  sont aussi expressives que les Frappes de Processus avec $n$ classes de priorités.
\end{theorem}

\begin{proof}
  Nous utilisons pour cette démonstration
  l'opérateur d'aplatissement $\PHflat$ donné à la \defref[vref]{fopaplat}
  et les résultats qui le concernent.
  Soient $k, n \in \sNN$,
  et soient $\PH$ des Frappes de Processus avec $k$ classes de priorités.
  D'après le \thmref{bisimulaplatissement},
  l'aplatissement $\PHflat(\PH)$ est faiblement bisimilaire à $\PH$.
  Posons :
  \begin{itemize}
    \item $\PHflat(\PH) = (\PHs; \PHl; \PHh^{\angles{2}})$
      avec $\PHh^{\angles{2}} = (\PHh^{(1)}; \PHh^{(2)})$,
    \item $\PH' = (\PHs; \PHl; \PHh'^{\angles{n}})$
      avec $\PHh'^{\angles{n}} = (\PHh^{(1)}; \PHh^{(2)}; \PHh'^{(3)}; \ldots; \PHh'^{(n)})$, \\
      \hspace*{1em} où $\forall i \in \segm{3}{n}, \PHh'^{(i)} = \emptyset$.
  \end{itemize}
  Autrement dit, $\PH'$ sont les Frappes de Processus avec $n$ classes de priorités
  identiques à $\PHflat(\PH)$, où des classes de priorités vides ont été
  artificiellement ajoutées.
  Cela est possible car $n \geq 2$.
  Ainsi, $\PH'$ possède la même dynamique que $\PHflat(\PH)$,
  et est donc faiblement bisimilaire à $\PH$
  (autrement dit, sa dynamique est équivalente au jeu des actions de priorité 1 près).
  Ainsi, pour tout $k, n \in \sNN$,
  toutes Frappes de Processus avec $k$ classes de priorités peuvent être représentées
  en Frappes de Processus avec $n$ classes de priorités.
\end{proof}



\subsection{Réutilisation des résultats existants}
\seclabel{php-outils}

Nous discutons dans cette section de la transposition des différents outils et
résultats concernant les Frappes de Processus standards
aux Frappes de Processus avec classes de priorités.
Nous aborderons la question de la recherche des points fixes d'un modèle (\secref{php-outils-pf}),
de l'analyse statique des modèles (\secref{php-outils-as})
et des possibles (ré)utilisations des paramètres stochastiques (\secref{php-outils-stocha}).

\subsubsection{Points fixes}
\seclabel{php-outils-pf}

Nous montrons dans cette sous-section que les points fixes des Frappes de Processus
avec classes de priorités peuvent être obtenus de façon similaire à ceux
des Frappes de Processus standards.
Pour cela, nous définissons la notion de \emph{fusion} d'un modèle de Frappes de Processus
%avec classes de priorités,
qui consiste à fusionner les classes de priorités en une seule,
afin de retrouver un modèle de Frappes de Processus standards.

\myskip

Pour tout modèle de Frappes de Processus avec $k$ classes de priorités, pour $k \in \sN^*$,
nous notons dans la suite $\PHmerge(\PH)$
la \emph{fusion} de $\PH$,
c'est-à-dire le même modèle dont les classes de priorités ont été fusionnées
(\defref{fusion}).
En d'autres termes, il s'agit d'un modèle de Frappes de Processus standards dont
l'ensemble des actions est l'union de toutes les classes de priorités de $\PH$.

On peut constater que pour un modèle donné, si l'ensemble de toutes les actions reste le même,
ajouter (ou retirer) des classes de priorités à des Frappes de Processus ne change pas
l'ensemble de ses points fixes.
En effet, le \thmref{php-pf} stipule que
l'ensemble des points fixes des Frappes de Processus avec $k$
classes de priorités $\PH$ est identique à l'ensemble des points fixes de
sa fusion $\PHmerge(\PH)$.
Cela se démontre simplement en constatant qu'il existe une action jouable dans un état donné du
modèle $\PH$ si et seulement si il en existe une dans le même état du modèle fusionné.
En effet, si une action est jouable dans un état donné de $\PHmerge(\PH)$,
alors soit elle est jouable car non préemptée dans le même état de $\PH$,
soit elle ne l'est pas car elle est préemptée par une autre action qui, elle, est jouable.
L'autre sens de la démonstration est immédiat car l'ajout de priorités restreint la dynamique
et n'ajoute aucun comportement supplémentaire possible.
Ce résultat permet d'appliquer aux Frappes de Processus avec $k$ classes de priorités
les méthodes de recherche de points fixes développée pour les Frappes de Processus standards.
L'une d'entre elles repose sur la recherche de $n$\nbd cliques (\vsecref{ph-pf})
et sa résolution jouit aujourd'hui de méthodes de résolutions performantes.
D'autres méthodes de recherche peuvent être envisagées,
par exemple par l'utilisation de programmation logique.
En effet, la formalisation du problème de recherche de points fixes est très simple,
et sa résolution est donc traitée efficacement par des méthodes SAT ou par ASP.
Enfin, ce résultat permet aussi de conclure quant aux ensembles de points fixes de deux Frappes
de Processus avec un nombre de classes de priorités différent, à condition que
leurs modèles fusionnés soient identiques.

\begin{definition}[Fusion ($\PHmerge$)]
\deflabel{fusion}
  Soient $\PH = (\PHs, \PHl, \PHh^{\angles{k}})$ des Frappes de Processus avec $k$
  classes de priorités, où $k \in \sN^*$.
  On note $\PHmerge(\PH) = (\PHs, \PHl, \PHh)$
  les Frappes de Processus standards appelées \emph{fusion} de $\PH$,
  dont l'ensemble des actions est l'union de toutes les classes de priorités de $\PH$.
\end{definition}

\begin{theorem}[Points fixes des Frappes de Processus avec classes de priorités]
\thmlabel{php-pf}
  Soient $\PH = (\PHs, \PHl, \PHh^{\angles{k}})$, où $k \in \sN^*$,
  des Frappes de Processus avec $k$ classes de priorités,
%  et $\PH' = (\PHs, \PHl, \PHh')$ les Frappes de Processus standards avec $\PHh' = \PHh$.
  et $r \in \PHl$ :
  \[\exists s \in \PHl, r \trans{\PH} s \Longleftrightarrow
    \exists s' \in \PHl, r \trans{\PHmerge(\PH)} s'\]
\end{theorem}

\begin{proof}
  On pose : $\PH' = \PHmerge(\PH)$.

  ($\Rightarrow$) Supposons qu'il existe $s \in \PHl$ tel que $r \trans{\PH} s$;
    cela signifie qu'une action $h \in \PHh$ est jouable dans $\PH$.
    Cette action est donc aussi jouable dans $\PH'$ car son frappeur et sa cible sont présents,
    d'où : $r \trans{\PH'} (r \play h)$.
    
  ($\Leftarrow$) Supposons qu'il existe $s' \in \PHl$ tel que $r \trans{\PH'} s'$ ;
    cela signifie qu'une action $g \in \PHh$ est jouable dans $\PH'$.
    Le frappeur et la cible de $g$ sont donc présents dans $r$.
    \begin{itemize}
      \item Si cette action n'est pas préemptée par une autre action dans $\PH$,
        elle est alors jouable et $r \trans{\PH} (r \play g)$ ;
      \item Si cette action est préemptée par une autre action $g'$ dans $\PH$,
        cela signifie que cette action $g'$ est jouable, et $r \trans{\PH} (r \play g')$.
        \qedhere
    \end{itemize}
\end{proof}

\subsubsection{Analyse statique}
\seclabel{php-outils-as}

Afin de permettre une étude efficace des Frappes de Processus standards de grande taille,
une analyse statique par interprétation abstraite avait été développée
par \citeasnoun{PMR12-MSCS}.
Son principe est rappelé à la \vsecref{ph-as}.
L'ajout de classes de priorités au formalisme a pour conséquence d'en restreindre la dynamique,
mais n'ajoute aucun comportement supplémentaire.
Ainsi, pour tout modèle de Frappes de Processus avec $k$ classes de priorités,
il est toujours possible de réutiliser l'analyse statique par sur-approximation
en l'appliquant au modèle $\PHmerge(\PH)$.
Bien que toujours exacte, cette analyse pourra néanmoins être moins conclusive,
n'ayant pas été spécifiquement adaptée aux modèles comportant des classes de priorités.

En revanche, l'analyse statique par sous-approximation n'est plus valable,
car elle ne prend pas en compte les possibles préemptions entre actions qui rendent
impossibles certaines atteignabilités.
C'est pourquoi une nouvelle version de l'analyse statique par sous-approximation sera développée
au \chapref{phcanonique}
sur une classe particulière de Frappes de Processus avec 2 classes de priorités, appelées
Frappes de Processus canoniques.
Cette classe particulière n'autorise les actions avec une forte priorité que pour la mise à jour
des sortes coopératives.
Cependant, nous montrons aussi à la \vsecref{phcanonique-equiv}
que toutes Frappes de Processus avec un nombre quelconque de classes de priorités
peuvent être traduites en Frappes de Processus canoniques équivalentes,
étendant ainsi l'analyse statique mentionnée aux modèles de Frappes de Processus
avec $k$ classes de priorités.
%, et nous donnons une telle traduction à la
%\defref{aplatissement}.

\subsubsection{Paramètres stochastiques}
\seclabel{php-outils-stocha}

Il est théoriquement toujours possible d'utiliser,
dans des Frappes de Processus avec $k$ classes de priorités,
des paramètres stochastiques tels que ceux
développés par \citeasnoun{PMR10-TCSB} et mentionnés à la \vsecref{ph-stocha}.
Il suffirait pour cela d'empêcher la sensibilisation de toute action préemptée par une action
de priorité plus importante.
Cependant, un autre parallèle intéressant peut être tracé entre l'approche
par définition de classes de priorités
et l'approche par introduction de paramètres stochastiques.

\myskip

L'ajout de paramètres stochastiques a pour but d'assigner un intervalle de tir temporel à
chaque action, afin de s'assurer (avec un certain niveau de confiance) que l'action sera
nécessairement tirée dans cet intervalle à partir du moment où elle est devenue jouable.
La simulation stochastique développée par \citeasnoun{PMR10-TCSB}
ne permet actuellement pas de prendre en compte des classes de priorités entre actions.
Il faudrait en effet pour cela raffiner la machine stochastique afin d'y intégrer des contraintes
supplémentaires concernant la jouabilité et la sensibilisation de chaque action.
Cependant, l'aplatissement proposé à la \vdefref{aplatissement}
permet théoriquement d'obtenir un modèle équivalent,
utilisable avec la simulation stochastique.
Nous détaillons ici le principe de ce procédé sans toutefois donner de preuve de sa fidélité.

L'aplatissement mentionné ci-dessus permet en effet d'obtenir un modèle de Frappes de Processus
avec 2 classes de priorités ayant une certaine forme qui permet de distinguer
les actions instantanées (de priorité 1) propres à la modélisation
des actions possédant une durée (de priorité 2) et permettant de représenter des processus
biologiques.
On peut ainsi, dans le modèle obtenu, attribuer à chaque action secondaire (\cad de priorité 2)
des paramètres stochastiques identiques à ceux de l'action originelle dont elle est issue,
et à chaque action primaire (\cad de priorité 1) une absorption de stochasticité infinie.
Le modèle résultant devrait alors posséder une dynamique identique à un potentiel modèle hybride
mêlant classes de priorités et paramètres stochastiques.

\myskip

Plutôt que d'intégrer des données stochastiques connues dans un modèle,
il est aussi possible de s'en servir pour l'obtention d'un modèle discret.
En effet, pour créer un modèle de Frappes de Processus avec classes de priorités,
il est nécessaire de connaître certaines relations entre les phénomènes modélisés
afin de répartir correctement les actions entre les différentes classes de priorités.
Il peut s'agit de données de préemption (un phénomène en bloque un autre),
de durée (un phénomène est beaucoup plus rapide qu'un autre),
de vitesse de déclenchement (un phénomène se déclenche toujours avant un autre), etc.
La piste que nous présentons dans la suite est l'utilisation d'un modèle de Frappes de Processus
standards enrichi à l'aide de paramètres stochastiques.
En effet, de tels paramètres stochastiques définis pour chaque action correspondent
à une fenêtre de tir avec un intervalle de confiance donné (généralement 95~\%),
qu'il est possible d'approximer \cite[p.~72]{Pauleve11},
et donc d'utiliser de façon interchangeable.
Ces paramètres stochastiques permettent donc au modélisateur de rendre une action
d'autant plus «~urgente~» que sa fenêtre de tir est proche de sont instant
de sensibilisation.

Ainsi, sous la condition que l'on peut distinguer les intervalles de tir en différentes classes
entre lesquelles les intervalles ne se recouvrent pas,
il est possible d'approximer la modélisation avec des paramètres stochastiques à
l'aide de classes de priorités.
En associant une classe de priorités à chaque ensemble d'intervalles de tir
---~la priorité la plus haute étant naturellement associée à la classe dont les intervalles
sont les plus proches de zéro,~---
on retrouve alors un modèle dont les caractéristiques dynamiques sont proches
du modèle initial.
En effet, le système de classes de priorités permet d'approcher une dynamique où
chaque intervalle est joué en priorité avant tous les suivants.

\begin{example}
\exlabel{paramsstocha-php}
  Le modèle de Frappes de Processus standards de la \vfigref{metazoan-ph}
  peut être enrichi à l'aide des paramètres proposés \vexpageref{metazoan-stocha-params},
  permettant ainsi
  de contraindre (à l'intervalle de confiance près) le jeu des actions entre elles,
  et favorisant le jeu des actions «~urgentes~».
  Cela permet notamment de rendre «~instantanées~»
  les actions de mise à jour des sortes coopératives,
  mais aussi de rendre «~urgentes~» les actions faisant bondir $a$,
  et de réguler l'évolution de $c$ afin de lui donner un rôle d'horloge.
  
  Si on observe les fenêtres de tir définies pour ce modèle, on constate qu'elles
  ne se recouvrent pas ; autrement dit, on peut clairement distinguer des classes
  d'actions qui seront tirées en priorité par rapport à d'autres classes.
  Nous proposons donc de traduire ces classes de fenêtres de tir en classes de priorités
  qui auront le même rôle : empêcher le jeu d'une action tant que d'autres actions
  plus «~urgentes~» sont jouables.
  Nous décidons de distinguer pour cet exemple trois classes de priorités :
  \begin{itemize}
    \item les actions instantanées car de taux infini (\stochainf) feront partie de la classe de
      plus forte priorité,
    \item les actions qui influencent $a$ (\stochaa) peuvent être considérés comme «~urgentes~»,
      et formeront la classe de priorité intermédiaire,
    \item les autres actions (\stochab{} et \stochac) seront regroupées dans la classe de priorité faible,
      car considérées comme «~peu urgentes~».
  \end{itemize}
  Nous obtenons ainsi le modèle donné à la \figref{metazoan-php},
  dont la dynamique a été précédemment discutée \vexpageref{metazoan-php}.
  Nous constatons que la dynamique du modèle en Frappes de Processus avec classes de priorités
  est «~strict~» : il n'y a pas d'intervalle de confiance et plus aucune notion de probabilités.
\end{example}

\myskip

Cependant, il n'est évidemment pas possible d'atteindre
avec cette méthode le même niveau de précision, en utilisant un nombre
discret de classes de priorités,
qu'avec des intervalles de tir définis sur une ligne temporelle continue.
Il n'est par exemple
pas possible de représenter fidèlement le recouvrement de deux intervalles de tir,
qui aurait pour conséquence de favoriser le tir d'une première action sans totalement
préempter la seconde action,
autrement qu'en mettant ces deux actions au même niveau de priorité.
À l'inverse, deux intervalles de tir qui ne se recouvrent pas devraient être associés à la même
classe de priorités s'ils sont tous deux recouverts par un troisième intervalle,
ce qui aura pour conséquence de mettre les trois actions sur un pied d'égalité, alors que les
paramètres stochastiques représentent une situation différente.

%\begin{example}
  \todoplustard{Exemple de trois intervalles qui se recouvrent et impossibles à traduire fidèlement en PHp ?}
%\end{example}

Enfin, il faut noter que
les phénomènes d'accumulation ou de retard ne sont pas pris en compte dans la modélisation
par classes de priorités.
En effet, si deux actions n'ont pas de sorte en commun, elles devraient dans l'idéal pouvoir
évoluer de façon indépendante, ce qui est notamment permis par la simulation stochastique.
En revanche, dans un formalisme avec classes de priorités, si l'une des actions est plus
prioritaire que l'autre, elle exercera tout de même sa préemption sur l'autre.
L'une des façons de pallier ce défaut est l'utilisation d'arcs neutralisants,
comme développé à la \v secref{phan}.

\begin{example}
\exlabel{paramsstocha-php-accumulation}
  L'exemple de la \vfigref{metazoan-php}
  peut être obtenu à l'aide de paramètres stochastiques,
  comme expliqué \vexpageref{paramsstocha-php}.
  Cependant, cette approche possède certaines limites ;
  il n'est pas possible par exemple de n'autoriser la désactivation de $f$
  qu'après un nombre donné d'oscillations de $a$ ou de $c$.
  Pour représenter cela, un modélisateur pourrait être tenté d'introduire une classe de
  priorité $4$ afin d'y intégrer l'action $\PHfrappe{f_1}{f_1}{f_0}$,
  en effectuant un parallèle avec les paramètres stochastiques proposés
  \vexpageref{metazoan-stocha-params}
  qui permettent de ne tirer cette action qu'après un certain nombre d'oscillations ;
  ou encore en constatant que les paramètres de cette action devraient permettre
  la création d'une telle classe de priorités, car son intervalle de tir n'en recouvre
  aucun autre.
  Cependant, un tel choix de conception aurait uniquement pour effet de rendre ladite action
  injouable, car sans cesse préemptée par des actions de priorité supérieure.
  En effet, une telle modélisation en Frappes de Processus avec $4$ classes de priorités
  fait abstraction du temps continu,
  ce qui signifie qu'il n'y a plus de notion d'«~accumulation~» du temps de sensibilisation,
  sur laquelle reposait le fait de pouvoir jouer l'action en question après une certaine
  durée de sensibilisation.
\end{example}

Nous avons ici montré l'une des limites de la modélisation par
Frappes de Processus avec classes de priorités, qui fait abstraction du temps
continu sur lequel se basent les paramètres stochastiques proposés par \citeasnoun{PMR10-TCSB}.
Le modélisateur doit effectivement être prudent pour ne pas rendre impossibles des
comportements qui sont seulement retardés par des modélisations chronométriques.
Cependant, à condition d'éviter ces écueils, nous avons montré qu'il est possible
de représenter un modèle avec des données temporelles à l'aide de ce formalisme,
et donc d'en étudier efficacement la dynamique à l'aide
de la traduction et des méthodes d'analyse statique proposés au \chapref{as}.


% Frappes de Processus avec arcs neutralisants
\section{Frappes de Processus avec arcs neutralisants}
\seclabel{phan}

Nous introduisons ici la notion d'\emph{arc neutralisant} dans les Frappes de Processus
afin de représenter la préemption d'une action par une seule autre.
Les \emph{Frappes de Processus avec arcs neutralisants} (\defref{php})
permettent notamment une modélisation plus atomique
par rapport aux classes de priorités présentées à la \secref{php}.

Un arc neutralisant est un couple d'actions noté $\PHan{h_1}{h_2}$,
où $h_1$ est appelée \emph{action bloquante},
et peut préempter $h_2$, appelée \emph{action bloquée},
dans certaines situations.
Avec la présence d'arcs neutralisants, une action est dite \emph{activée} dans un état donné si
son frappeur et sa cible y sont présents ;
une action est donc activée pour les Frappes de Processus avec arcs neutralisants
là où elle était immédiatement jouable pour les Frappes de Processus standards (\defref{fopph}).
Une action est \emph{jouable} pour les Frappes de Processus avec arcs neutralisants
si et seulement si elle est activée,
et que pour tout arc neutralisant la bloquant, son action bloquante n'est pas activée.
Une action activée mais qui n'est pas jouable est dite \emph{neutralisée}.

Il est à noter que la neutralisation d'une action par une autre ne dépend donc pas de la jouabilité
de l'action bloquante, mais uniquement de son activation.
Cela permet d'avoir un modèle cohérent, sans quoi certaines situations pourraient ne pas être
définies, notamment dans le cas d'un interblocage.
Ainsi, faire reposer la neutralisation d'une action bloquée sur la jouabilité de l'action bloquante
devient inextricable dans un cas comme le suivant :
$\PHan{h_1}{h_2}$, $\PHan{h_2}{h_3}$ et $\PHan{h_3}{h_1}$,
car si les trois actions $h1$, $h2$ et $h3$ sont actives, leur jouabilité reste indéterminée.
En revanche, si cette neutralisation repose sur l'état activé d'une action,
la situation précédente se résout immédiatement car aucune des trois actions n'est jouable.
On constate par ailleurs qu'une action peut en neutraliser une autre
même si elle-même est neutralisée.
Nous ne nous avancerons cependant pas sur la signification biologique de ce fait.

Enfin, il semble nécessaire de faire un parallèle entre les arcs inhibiteurs développés ici
et les arcs du même nom utilisés dans les réseaux de Petri \toref.
Leur rôle est effectivement proche, leur but étant la préemption d'une action en fonction
d'une condition extérieure à son déclenchement.
Malgré tout, les arcs inhibiteurs des réseaux de Petri se différencient car ils
reposent sur la présence d'un certain nombre de jetons dans une place
---~ce qui se traduirait, en Frappes de Processus,
par la présence d'un certain processus actif d'une sorte donnée~---
là où les arcs inhibiteurs des Frappes de Processus ne permettent l'inhibition
qu'en fonction de l'activité d'une autre action.
Cependant, un tel choix de conception peut aisément se pallier dans un sens comme dans l'autre.
En effet, représenter l'activité d'une action revient à créer deux arcs inhibiteurs en
réseaux de Petri ---~l'un pour le frappeur et l'autre pour la cible.
À l'inverse, il est possible qu'une action $h$ se bloque elle-même, à l'aide d'un 
«~auto-arc neutralisant~» $\PHan{h}{h}$,
permettant ainsi de bloquer une autre action $g$
en fonction de la présence du processus $\frappeur{h}$
à l'aide d'un deuxième arc neutralisant $\PHan{h}{g}$.

\todo{Schéma exemple}

\subsection{Définition}
\seclabel{phan-def}

\begin{definition}[Frappes de Processus avec arcs neutralisants]
\deflabel{phan}
  Les \emph{Frappes de Processus avec arcs neutralisants} sont définies par
  un quadruplet $\PH = (\PHs ; \PHl ; \PHh ; \PHn)$, où :
  \begin{itemize}
    \item $\PHs \DEF \{a, b, \dots\}$ est l'ensemble fini et dénombrable des \emph{sortes} ;
    \item $\PHl \DEF \bigtimes{a \in \PHs} \PHl_a$ est l'ensemble fini des \emph{états},
      où $\PHl_a = \{a_0, \ldots, a_{l_a}\}$ est l'ensemble fini et dénombrable
      des \emph{processus} de la sorte $a \in \PHs$ et $l_a \in \sN^*$.
      Chaque processus appartient à une unique sorte :
      $\forall (a_i; b_j) \in \PHl_a \times \PHl_b, a \neq b \Rightarrow a_i \neq b_j$ ;
    \item $\PHh \DEF \{\PHfrappe{a_i}{b_j}{b_l} \mid (a; b) \in \PHs^2 \wedge
      (a_i; b_j; b_l) \in \PHl_a \times \PHl_b \times \PHl_b \wedge
      b_j \neq b_l \wedge a = b \Rightarrow a_i = b_j \}$ est l'ensemble fini des actions ;
    \item $\PHn = \{\PHan{h_1}{h_2} \mid (h_1 ; h_2) \in \PHh \times \PHh\}$
      est l'ensemble fini des arcs neutralisants.
  \end{itemize}
\end{definition}

Un arc neutralisant $u = \PHan{h_1}{h_2} \in \PHn$ est donc un couple d'actions.
On note $\PHbloquant(u) = h_1$ la première action du couple $u$
et $\PHbloque(u) = h_2$ sa seconde action.
On réutilise par ailleurs les autres notations définies à la \secref{ph}.

L'opérateur de jouabilité des frappes de Processus avec arcs neutralisants (\defref{fopphan})
se concentre sur la présence du frappeur et de la cible de l'action considérée,
et sur celle de toutes ses actions bloquantes.
En ce sens, il est semblable à celui des Frappes de Processus avec $k$ classes de priorités
(\defref{fopphp}).

\begin{definition}[Opérateur de jouabilité ($\Fopsymbol_\Fopphansubsymbol : \PHh \rightarrow \F$)]
\deflabel{fopphan}
  L'opérateur de jouabilité des Frappes de Processus avec arcs neutralisants est défini par :
  \[\forall h \in \PHh, \Fopphan{h} \equiv \hitter{h} \wedge \target{h} \wedge
    \left( \bigwedge_{\substack{g \in \PHh\\\PHan{g}{h} \in \PHn}}
    \neg \left( \hitter{g} \wedge \target{g} \right) \right)\]
\end{definition}



\subsection{Équivalence avec les Frappes de Processus avec $k$ classes de priorités}
\seclabel{phan-equiv-php}

Les Frappes de Processus avec arcs neutralisants ont une expressivité équivalente
aux Frappes de Processus avec $k$ classes de priorité, où $k \in \sN^*$.
En effet, la \thmref{phpbisimphan} propose une traduction des Frappes de Processus
avec $k$ classes de priorités en Frappes de Processus avec arcs neutralisants.
À l'inverse, les Frappes de Processus avec arcs neutralisants peuvent être représentée
en Frappes de Processus canoniques, qui sont une sous-classe des Frappes de Processus avec
2 classes de priorités, comme démontré à la \secref{phan-aplatissement}.

\begin{theorem}[Équivalence avec les classes de priorités]
\thmlabel{phpbisimphan}
  Soient $\PH = (\PHs, \PHl, \PHh^{\angles{k}})$ des Frappes de Processus avec $k$ classes
  de priorités, où $k \in \sN^*$.
  Il existe un modèle $\oPH$ de Frappes de Processus avec arcs neutralisants tel que :
  \[\forall s, s' \in \PHl, s \trans{\PH} s' \Longleftrightarrow s \trans{\oPH} s' \]
\end{theorem}

\begin{proof}
  On pose : $\oPH = (\PHs, \PHl, \PHh, \PHn)$ les Frappes de Processus avec arcs neutralisants
  dont l'ensemble des actions est l'union de toutes les classes de priorités de $\PH$, et :
  $\PHn = \{ \PHan{g}{h} \mid g \in \PHh^{(n)}, h \in \PHh^{(k)}, n < k \}$.
  Soit $h \in \PHh$. et
  D'après la définition de $\oPH$, on constate que :
  $\{ g \in \PHh \mid \PHan{g}{h} \in \PHn \} = \{ g \in \PHh \mid n < \prio(h) \}$.
  Ainsi, $\forall h \in \PHh, \Fopphp{h} = \Fopphan{h}$,
  ce qui implique que les action jouables dans l'état $s$ du le modèle $\PH$
  sont exactement les actions jouables dans l'état $s$ du modèle $\oPH$.
  D'où : $s \trans{\PH} s' \Longleftrightarrow s \trans{\oPH} s'$.
\end{proof}



\subsection{Réutilisation des outils existants}
\seclabel{phan-outils}

\todo{Revoir le titre de cette section}

\todo{Faudrait-il fusionner cette section avec ses semblables des \secref{php} et \secref{phm} ?}

\subsubsection{Points fixes}
\seclabel{phan-outils-pf}

De façon analogue à la \secref{php-outils-pf}, nous définissions ici
la \emph{fusion} d'un modèle de Frappes de Processus avec arcs neutralisants
comme étant le même modèle en Frappes de Processus standards,
dont les arcs neutralisants ont été ignorés (\defref{phan-fusion}).
Cette définition nous permet d'avancer le résultat suivant :
les points fixes de $\PHmergean(\PH)$ sont des points fixes de $\PH$,
d'après la contraposée du \thmref{php-pf}.
Il n'est en effet pas possible d'avancer de résultat plus précis de façon immédiate,
car les arcs neutralisants peuvent former des boucles d'actions qui se neutralisent entre elles,
rajoutant ainsi des points fixes supplémentaires par rapport au modèle fusionné.
Ce cas ne se présente pas pour les Frappes de Processus avec $k$ classes de priorités.

\begin{definition}[Fusion ($\PHmergean$)]
\deflabel{phan-fusion}
  Soient $\PH = (\PHs, \PHl, \PHh, \PHn)$ des Frappes de Processus avec arcs neutralisants.
  On note $\PHmergean(\PH) = (\PHs, \PHl, \PHh)$
  les Frappes de Processus standards appelées \emph{fusion} de $\PH$,
  dont on a retiré les arcs neutralisants.
\end{definition}

\begin{theorem}[Points fixes des Frappes de Processus avec arcs neutralisants]
\thmlabel{phan-pf}
  Soient $\PH = (\PHs, \PHl, \PHh, \PHn)$ des Frappes de Processus avec arcs neutralisants,
  et $r \in \PHl$ :
  \[\exists s \in \PHl, r \trans{\PH} s \Longrightarrow
    \exists s' \in \PHl, r \trans{\PHmergean(\PH)} s'\]
\end{theorem}

\begin{proof}
  Si une action est jouable dans $r$, cela signifie notamment
  que son frappeur et sa cible sont présents dans $r$.
  Cette condition est suffisante pour que cette même action soit jouable dans $\PHmergean(\PH)$.
\end{proof}

\todo{Les cycles d'actions sont-ils suffisants pour trouver tous les points fixes ?}

Pur finir, il est toujours envisageable d'aplatir des Frappes de Processus avec
arcs neutralisants en Frappes de Processus canoniques,
comme décrit à la \secref{phan-aplatissement},
afin d'effectuer une recherche de points fixes à l'aide de la méthode
proposée à la \secref{php-outils-pf}.
Comme le modèle aplati de Frappes de Processus avec arcs neutralisants possède la même dynamique,
il est possible de retrouver les points fixes du modèle original de cette façon,
en supprimant simplement les sortes coopératives des résultats.

\subsubsection{Analyse statique}
\seclabel{phan-outils-as}

De même qu'à la \secref{php-outils-as},
étant donné que les Frappes de Processus avec arcs neutralisants permettent de restreindre
la dynamique par rapport aux Frappes de Processus standards, il est toujours possible
d'utiliser l'analyse statique par sur-approximation,
mais l'analyse statique par sous-approximation n'est plus valable.

Cependant, nous proposons au \chapref{phcanonique}
une traduction des Frappes de Processus avec arcs neutralisants en Frappes de Processus canoniques,
qui sont des Frappes de Processus avec 2 classes de priorités avec des contraintes précises,
ainsi qu'une nouvelle approche d'analyse statique par sous-approximation
qui s'applique à cette classe particulière de modèles.
L'analyse statique par sous-approximation est donc possible sur les Frappes de Processus
avec arcs neutralisants, au prix de cette traduction.

\subsubsection{Paramètres stochastiques}

Enfin, à l'instar de la \secref{php-outils-stocha},
il n'est pas possible d'utiliser directement les outils développés pour la simulation stochastique
car ceux-ci ne prennent pas en compte la présence d'arcs neutralisants.
Malgré cela, une telle utilisation reste théoriquement possible à condition de prendre
en compte les neutralisations entre actions.

Cependant, à nouveau, un parallèle intéressant peut être tracé entre les arcs neutralisants
et l'introduction de paramètres stochastiques dans le modèle.
En effet, étant donné un ensemble de paramètres stochastiques,
si les intervalles de tir de deux actions sont disjoints,
alors il est possible de modéliser cela par un arc neutralisant
dont l'action bloquante est l'action la plus «~rapide~»
(\cad donc l'intervalle est le plus proche de zéro)
et l'action bloquée est la plus «~lente~»
(\cad donc l'intervalle est le plus éloigné de zéro).

Cette représentation est avantageuse par rapport à celle proposée à la
\secref{php-outils-stocha}.
En effet, il n'est pas nécessaire pour créer le modèle
de distinguer des classes globales d'actions selon leurs
intervalles de tir, mais seulement de déterminer si les intervalles de tir de toutes les actions
sont, deux à deux, disjoints (voire assez éloignés selon des critères qui peuvent être
fixés arbitrairement).
Cela permet de plus d'obtenir des relations plus fines entre actions,
là où les Frappes de Processus ne permettent que la distinction de classes de priorités
globales, qui sont parfois trop grossières pour certaines modélisations.
Cependant, le problème de la représentation de l'accumulation persiste :
une partie du modèle évoluant «~plus rapidement~» peut totalement préempter une
action «~plus lente~» si celle-ci est sans cesse neutralisées.
Or en pratique, un tel cas devrait en définitive autoriser l'action «~lente~» à s'exécuter
après un certain temps.
Cela peut être corrigé en supprimant quelques arcs neutralisants bien choisis,
ou encore en ne permettant pas la création d'arcs neutralisants entre les parties indépendantes
du modèle, mais cela nécessite une analyse préalable assez poussée de la dynamique
du système.


% Frappes de Processus avec actions plurielles
% Enrichissement avec simultanéité des actions :
\section{Frappes de Processus avec actions plurielles}
\seclabel{phm}

Il peut être intéressant de vouloir représenter un système au niveau de ses réactions biochimiques,
c'est-à-dire des réactions entre les différents composants présents.
De telles réactions peuvent avoir différentes formes (transformation, complexation, dissociation...),
et il est fréquent qu'elles fassent intervenir plusieurs réactifs et plusieurs produits.
Biocham \todo{ref} propose par exemple de modéliser un tel système de réactions biochimiques à l'aide
d'un ensemble de règles de réaction de la forme :
$X \xrightarrow{Y} Z$,
ou encore :
$X + Y \rightarrow Y + Z$,
où $X$ est un ensemble de réactifs, $Y$ un ensemble de catalyseurs et $Z$ un ensemble de produits.

Les \emph{Frappes de Processus avec actions plurielles} permettent de représenter de telles réactions
mettant en jeu un nombre arbitraire de réactifs, de produits et de catalyseurs.
Ainsi, une réaction de la forme : $X \xrightarrow{Y} Z$
peut être représentée à l'aide de l'action $\PHfrappemult{A}{B}$
où $A$ et $B$ sont deux ensembles des processus,
$A$ regroupant tous les processus représentant les composants nécessaires à initier la réaction,
et $B$ tous les processus qui ont évolué pendant la réaction.
%$A$ regroupant tous les processus permettant de représenter les composants de $X \cup Y$,
%et $B$ ceux de $Z$.
Une telle action peut donc être jouée dans un état contenant tous les processus de $A$
%(c'est-à-dire les réactifs et les catalyseurs)
et fait évoluer celui-ci vers un état contenant tous les processus de $B$, % (les produits),
les autres processus restant inchangés. % (car n'intervenant pas ou en tant que catalyseurs).
Cela implique toutefois que pour tout processus de $B$, il existe un autre processus de la même
sorte dans $A$.
Les Frappes de Processus avec actions plurielles permettent donc de représenter
un nombre arbitraire de bonds simultanés
---~autrement dit, de changements simultanés de processus actifs~---
déclenchés par un nombre arbitraire de requis
---~sous la forme de processus actifs .

Un parallèle peut être tracé d'une part entre $A$ et l'ensemble des réactifs et catalyseurs,
et d'autre part entre $B$ et l'ensemble des produits.
Cependant, la modélisation par Frappes de Processus avec actions plurielles
nécessite aussi de donner explicitement les composants sont absents.
Par exemple, une réaction de complexation du type : $x + y \rightarrow c$ %x\!\!-\!\!y$
sera représentée en Frappes de Processus avec actions plurielles par trois sortes de deux processus
$x$, $y$ et $c$ représentant respectivement les deux réactifs et le complexe produit,
et par l'action $\PHfrappemults{x_1, y_1, c_0}{x_0, y_0, c_1}$.
%si on ne prend pas en compte la disparition des réactifs.
Autrement dit, il est nécessaire de décomposer chaque élément en fonction de sa présence
($x_1$ et $y_1$)
ou de son absence ($c_0$) au début comme à la fin de la réaction,
et pas uniquement d'indiquer les composants présents en tant que réactifs ou produits.

On note ainsi qu'une réaction de la forme $\PHfrappemult{\{ a_0, b_0, c_0 \}}{\{ a_1, b_1 \}}$
ne peut être jouée si l'une des deux sortes entrant en jeu, $a$ et $b$, est déjà au niveau $1$,
même si l'autre est au niveau $0$.
Un tel comportement a du sens lorsque les différents processus d'une sorte
($a_0$ et $a_1$, par exemple)
représentent différents états d'une même molécule :
la réaction ne peut alors pas être jouée pour des raisons de stœchiométrie.
Cependant, si ces différents processus représentent plutôt des niveaux de concentration
($a_1$ représentant par exemple un niveau de concentration de la molécule $a$ plus élevé que $a_0$),
cette restriction n'a plus de sens car une plus forte concentration d'une des entités
ne devrait pas empêcher la réaction d'avoir lieu et de produire la seconde entité.
Cela peut néanmoins être corrigé en ajoutant les actions
$\PHfrappemult{\{ a_1, b_0, c_0 \}}{\{ a_1, b_1 \}}$ et
$\PHfrappemult{\{ a_0, b_1, c_0 \}}{\{ a_1, b_1 \}}$,
ou encore en séparant la production de $a_1$ et de $b_1$ en deux actions (ou ensemble d'actions)
distinctes.

Cette forme des Frappes de Processus peut être aisément représentée à l'aide d'un réseau
d'automates synchronisés, chaque sorte ayant le rôle d'un automate et chaque action celui d'un
ensemble de transitions étiquetées avec le même libellé partant chacune d'un processus dans $A$ et
arrivant dans le processus de la même sorte dans $A \recouvre B$,
comme décrit à la \secref{phm2an}.
On peut aussi la représenter à l'aide de Frappes de Processus avec 4 classes de priorités,
comme détaillé à la \secref{phm2php} ;
cependant, cette représentation a l'inconvénient d'être moins claire car faisant intervenir
un nombre important d'actions et de sortes supplémentaires.

\todo{Schéma exemple}



\subsection{Définition}

La \defref{phm} formalise la notion de Frappes de Processus avec actions plurielles,
en accord avec la discussion informelle ci-dessus :
une action plurielle est constituée de deux ensembles de processus de sortes distinctes,
qui représentent l'ensemble des frappeurs et celui des bonds.
Cela permet de formaliser le déclenchement d'une action par une synchronisation exacte
entre un nombre arbitraire de frappeurs,
et une synchronisation exacte entre plusieurs bonds lorsque l'action est jouée.
% Nous étendons aussi la notion de recouvrement à la \defref{recouvrementps}
% au cas du recouvrement d'un sous-état désordonné par un autre,
% ce qui permet, conjointement à la définition de
Nous définissons de plus l'opérateur de jouabilité
des Frappes de Processus avec actions plurielles
à la \defref{fopphm},
afin de formaliser la dynamique de ce type de modèles
à l'aide de la sémantique donnée à la \defref[vref]{play}.

\begin{definition}[Frappes de Processus avec actions plurielles]
\deflabel{phm}
  Les \emph{Frappes de Processus avec actions plurielles} sont définies
  par un triplet $\PH = (\PHs; \PHl; \PHh)$, où :
  \begin{itemize}
    \item $\PHs \DEF \{a, b, \dots\}$ est l'ensemble fini et dénombrable des \emph{sortes} ;
    \item $\PHl \DEF \bigtimes{a \in \PHs} \PHl_a$ est l'ensemble fini des \emph{états},
      où $\PHl_a = \{a_0, \ldots, a_{l_a}\}$ est l'ensemble fini et dénombrable
      des \emph{processus} de la sorte $a \in \PHs$ et $l_a \in \sN^*$,
      chaque processus appartenant à une unique sorte :
      $\forall (a_i; b_j) \in \PHl_a \times \PHl_b, a \neq b \Rightarrow a_i \neq b_j$ ;
    \item $\PHh \DEF \{\PHfrappemult{A}{B} \mid A, B \in \PHsublset \setminus \emptyset \wedge
      \forall q \in B, \exists p \in A, (p \neq q \wedge \PHsort(p) = \PHsort(q)) \}$
      est l'ensemble fini des \emph{actions}.
  \end{itemize}
\end{definition}
%
\noindent
Pour toute action $h = \PHfrappemult{A}{B} \in \PHh$,
$A$ est appelé le \emph{frappeur} et $B$ le \emph{bond} de $h$,
et on note : $\hitter{h} = A$, $\bounce{h} = B$.
On note de plus :
$\target{h} = \{ p \in A \mid \exists q \in B, \PHsort(p) = \PHsort(q) \}$,
et : $\invariant{h} = \{ p \in A \mid \sort{p} \notin \sortes{B} \}$.

\begin{definition}[Opérateur de jouabilité ($\Fopsymbol_\Fopphmsubsymbol : \PHh \rightarrow \F$)]
\deflabel{fopphm}
  L'opérateur de jouabilité des Frappes de Processus avec actions plurielles est défini par :
  \[\forall h \in \PHh, \Fopphm{h} \equiv \bigwedge_{p \in \hitter{h}} p \enspace.\]
\end{definition}



\subsection{Équivalence avec les réseaux d'automates synchronisés}
\seclabel{phm2an}

\todo{À déplacer dans le chapitre 5 ?}

\todo{Glu}

Nous nous intéressons ici au lien entre les Frappes de Processus avec actions plurielles
et les réseaux d'automates synchronisés.
Nous montrons notamment que ces deux formalismes sont équivalents
et nous exhibons pour cela deux traductions d'un formalisme vers l'autre
(\defref{phm2an,an2phm}\vpageref{def:phm2an}).
%(\defref[s]{phm2an} et \defref*[vref]{an2phm}).
Cette équivalence est intéressante car elle montre clairement le lien entre ce formalisme
de Frappes de Processus et celui plus répandu des réseaux d'automates synchronisés.
De plus, les traductions permettent de naviguer entre les deux représentations
afin d'en utiliser les différents outils.

Nous rappelons tout d'abord la définition d'un réseau d'automates synchronisés (\defref{an})
ainsi que la relation de transition entre deux états d'un tel modèle (\defref{an-sem})
ce qui permet d'en définir la dynamique.

\begin{definition}[Réseau d'automates synchronisés]
\deflabel{an}
  Un \emph{réseau d'automates synchronisés} est un quadruplet $\AN = (\ANs; \ANl; \ANi; \ANt)$
  où :
  \begin{itemize}
    \item $\ANs \DEF \{a, b, \dots\}$ est l'ensemble fini et dénombrable des \emph{automates} ;
    \item $\ANl \DEF \bigtimes{a \in \ANs} \ANl_a$ est l'ensemble fini des \emph{états},
      où $\ANl_a = \{a_0, \ldots, a_{l_a}\}$ est l'ensemble fini et dénombrable
      des \emph{états locaux} de l'automate $a \in \ANs$ et $l_a \in \sN^*$,
      chaque état local appartenant à un unique automate :
      $\forall (a_i; b_j) \in \ANl_a \times \ANl_b, a \neq b \Rightarrow a_i \neq b_j$ ;
    \item $\ANi \DEF \{\ell_1, \dots, \ell_m\}$ est l'ensemble fini des
      \emph{libellés} de transitions ;
    \item $\ANt \DEF \{ \ANaction{a_i}{\ell}{a_j} \mid a \in \ANs \wedge a_i \in \ANl_a \wedge
      \ell \in \ANi \}$ est l'ensemble fini des \emph{transitions} entre états locaux.
  \end{itemize}
  Pour tout libellé $\ell \in \ANi$, on note
  $\precond{\ell} \DEF \{ a_i \mid \ANaction{a_i}{\ell}{a_j} \in \ANt \}$
  et $\postcond{\ell} \DEF \{ a_j \mid \ANaction{a_i}{\ell}{a_j} \in \ANt \}$.
%   et $\invcond{\ell} \DEF \{ a_i \mid \ANaction{a_i}{\ell}{a_i} \in \ANt \}$.
  L'ensemble des états locaux des automates est dénoté par
  $\ANProc \DEF \bigcup_{a \in \ANs} \ANl_a$.
  Enfin, étant donné un état global $s \in \ANl$, $s(a) = a_i \in \ANl_a$
  fait référence à l'état local de l'automate $a \in \ANs$.
\end{definition}

\begin{definition}[Sémantique des réseaux d'automates ($\ANtrans$)]
\deflabel{an-sem}
  Étant donné un réseau d'automates synchronisés $\AN = (\ANs; \ANl; \ANi; \ANt)$,
  un libellé $\ell$ est dit \emph{jouable} dans un état $s \in \ANl$ si et seulement si :
  $\forall a_i \in \precond{\ell}, s(a) = a_i$.
  Dans ce cas, on note $(s \play \ell)$ l'état résultant du jeu de toutes les transitions
  libellées par $\ell$, défini par :
  $s \play \ell = s \recouvre \postcond{\ell}$.
%   $\forall a_j \in \postcond{\ell}, (s \play \ell)(a) = a_j \wedge
%     \forall b \in \ANs, \ANl_b \cap \precond{\ell} = \emptyset \Rightarrow
%     (s \play \ell)(b) = s(b)$.
  De plus, on note alors : $s \ANtrans (s \play \ell)$.
%   Étant donné un réseau d'automates synchronisés $\AN = (\ANs; \ANl; \ANi; \ANt)$,
%   la relation de transition globale entre deux états du réseau
%   $\ANtrans \subset \ANl \times \ANl$ est définie par :
%   \begin{align*}
%     s \ANtrans s' \EQDEF \exists \ell \in \ANi, &\forall a_i \in \precond{\ell}, s(a) = a_i
%       \wedge \forall a_j \in \postcond{\ell}, s'(a) = a_j \\
%     \wedge &\forall b \in \ANs, \ANl_b \cap \precond{\ell} = \emptyset \Rightarrow s(b) = s'(b)
%   \end{align*}
\end{definition}

\begin{remark}
  Nous notons que les réseaux d'automates synchronisés ainsi définis sont non-déterministes,
  tant au niveau global du modèle qu'au niveau local des automates.
  Cette vision s'oppose à d'autres sémantiques des réseaux d'automates
  comme celles de \citeasnoun{Richard10} ou de \citeasnoun{RRT08},
  qui définissent la dynamique de leurs modèles à l'aide de fonctions de transition locales,
  qui sont par définition déterministes.
  Ces fonctions ont en effet la forme : $f_a : \ANl \rightarrow \ANl_a$
  et associent donc à chaque état global du modèle un état local (unique) pour chaque automate.
  La définition des réseaux d'automates synchronisés que nous proposons ici (\defref{an})
  n'empêche en revanche pas l'existence de deux libellés $\ell_1, \ell_2 \in \ANi$
  tels que $\precond{\ell_1} = \precond{\ell_2}$ mais $\postcond{\ell_1} \neq \postcond{\ell_2}$.
  Cela implique notamment l'existence de deux transitions entre état locaux
  $\ANaction{a_i}{\ell_1}{a_j}$ et $\ANaction{a_i}{\ell_2}{a_k}$
  avec $a_j \neq a_k$, d'où un non-déterminisme au niveau des automates.
\end{remark}

Pour toutes Frappes de Processus avec actions plurielles $\PH$,
la \defref{phm2an} propose une traduction de $\PH$
en un réseau d'automates synchronisés $\phmtoan[\PH]$ équivalent,
et le \thmref{bisimulationphm2an} établit la bisimilarité entre les deux modèles.
La notation $\recouvre$ qui est utilisée dans la définition
qualifie le recouvrement d'un ensemble de processus de sortes distinctes
par un autre comprenant uniquement des processus issus des mêmes sortes
(\defref{recouvrementps}).
Cette notion de recouvrement est une extension
du recouvrement d'un état par un ensemble de processus
tel que précédemment donné à la \defref[vref]{recouvrement}.

\begin{definition}[Recouvrement ($\recouvre : \PHsublset \times \PHsublset \rightarrow \PHsublset$)]
\deflabel{recouvrementps}
  Étant donné un sous-état désordonné $ps \in \PHsublset$ et un processus $a_i \in \Proc$,
  tel que $a \in \sortes{ps}$, on définit :
  $(ps \recouvre a_i) = (ps \setminus \PHl_a) \cup \{ a_i \}$.
  On étend de plus cette définition
  au recouvrement par un ensemble de processus de sortes distinctes
  $ps' \in \PHsublset$ tel que $\sortes{ps'} \subset \sortes{ps}$
  comme étant le recouvrement de $ps$ par chaque processus de $ps'$ :
  $ps \recouvre ps' = ps \underset{a_i \in ps'}{\recouvre} a_i$.
\end{definition}

\begin{definition}[Réseau d'automates équivalent ($\phmtoansymbol$)]
\deflabel{phm2an}
  Le réseau d'automates synchronisé équivalent aux Frappes de Processus
  avec actions plurielles $\PH = (\PHs; \PHl; \PHh)$
  est défini par : $\phmtoan = (\PHs; \PHl; \ANi; \ANt)$, où :
  \begin{itemize}
    \item $\ANi = \{ \ell_h \mid h \in \PHh \}$ ; % est l'ensemble des libellés de transitions ;
    \item $\ANt = \{ \ANaction{a_i}{\ell_h}{a_j} \mid
      h \in \PHh \wedge h = \PHfrappemult{A}{B} \wedge a_i \in A \wedge a_j \in A \recouvre B \}$.
      % est l'ensemble des transitions locales.
  \end{itemize}
\end{definition}

\begin{theorem}[$\PH \approx \phmtoan$]
\thmlabel{bisimulationphm2an}
  Soient $\PH = (\PHs; \PHl; \PHh)$ des Frappes de Processus avec actions plurielles.
  On a :
  \[\forall s, s' \in \PHl, s \PHtrans s' \Longleftrightarrow s \trans{\phmtoan} s' \enspace.\]
\end{theorem}

\begin{proof}
  Soient $s, s' \in \PHl$.
  On pose : $\phmtoan = (\ANs; \ANl; \ANi; \ANt)$.
  
  ($\Rightarrow$) Supposons que $s \PHtrans s'$, c'est-à-dire qu'il existe une action $h \in \PHh$
    telle que $s' = s \play h$.
    Posons : $h = \PHfrappemult{A}{B}$.
    D'après la \defref{phm2an},
    l'existence de cette action dans $\PH$ implique celle d'un libellé $\ell_h$ dans $\phmtoan$
    ainsi que de l'ensemble de transitions
    $\ANt_h = \{ a_i \xrightarrow{\ell_h} a_j \mid a_i \in A \wedge a_j \in A \recouvre B \}$.
    Autrement dit, $\precond{\ell_h} = A$, donc $\ell_h$ est jouable dans $s$
    si et seulement si $A \subseteq s$.
    De plus, $\postcond{\ell_h} = \invariant{h} \cup B$, donc
    $(s \play \ell_h) = s \recouvre (\invariant{h} \cup B) = s \recouvre B = s'$
    car $\invariant{h} \subseteq A \subseteq s$.
  
  ($\Leftarrow$) Supposons que $s \trans{\phmtoan} s'$,
    c'est-à-dire qu'il existe un libellé $\ell \in \ANi$ et un ensemble de transitions
    ayant ce libellé : $\ANt_\ell = \{ a_i \xrightarrow{\ell} a_j \in \ANt \}$,
    tels que $s' = s \play \ell$.
    D'après la \defref{phm2an}, cela signifie notamment qu'il existe une action
    $h = \PHfrappemult{A}{B} \in \PHh$ telle que $\ell = \ell_h$, et que :
    $\ANt_\ell = \{ a_i \xrightarrow{\ell} a_j \mid a_i \in A \wedge a_j \in A \recouvre B \}$.
    Étant donné que $\invariant{h}$ et $\cible{h}$ forment une partition de $A$,
    $\ANt_\ell$ peut être découpé en deux ensembles, selon les invariants et les cibles de $h$ :
    $\ANt_\ell = \{ a_i \xrightarrow{\ell} a_i \mid a_i \in \invariant{h} \} \cup
      \{ a_i \xrightarrow{\ell} a_j \mid a_i \in \cible{h} \wedge a_j \in B \}$.
    Ainsi, $s' = s \recouvre (\invariant{h} \cup B) = s \recouvre B = s \play h$.
\end{proof}

Pour finir, nous proposons à la \defref{an2phm} la traduction inverse
d'un réseau d'automates synchronisé $\AN$
en des Frappes de Processus avec actions plurielles équivalentes $\antophm$.
Le \thmref{bisimulationan2phm} stipule que le modèle obtenu est bien bisimilaire
au modèle d'origine.
Enfin, le \thmref{equivphman} résume les résultats de cette section
en statuant l'équivalence d'expressivité entre les Frappes de Processus avec
actions plurielles et les réseaux d'automates synchronisés.

\begin{definition}[Frappes de Processus équivalentes ($\antophmsymbol$)]
\deflabel{an2phm}
  Les Frappes de Processus avec actions plurielles
  équivalentes au réseau d'automates synchronisé $\AN = (\PHs, \PHl, \ANi, \ANt)$
  sont définies par $\antophm = (\ANs, \ANl, \PHh)$, où :
%   $\PHh = \{ \PHfrappemult{\precond{\ell}}{(\postcond{\ell} \setminus \invcond{\ell})}
%     \mid \ell \in \ANi \}$.
  \[\PHh = \{ \PHfrappemult{\precond{\ell}}{B} \mid \ell \in \ANi \wedge
    B = \postcond{\ell} \setminus \{ a_i \in \ANProc \mid \ANaction{a_i}{\ell}{a_i} \in \ANt \}
    \}\]
\end{definition}

\begin{theorem}[$\AN \approx \antophm$]
\thmlabel{bisimulationan2phm}
  Soit $\AN = (\ANs; \ANl; \ANi; \ANt)$ un réseau d'automates synchronisés.
  On a :
  \[\forall s, s' \in \ANl, s \ANtrans s' \Longleftrightarrow s \trans{\antophm} s' \enspace.\]
\end{theorem}

\begin{proof}
  Soient $s, s' \in \PHl$.
  On pose : $\antophm = (\ANs; \ANl; \PHh)$.
  
  ($\Rightarrow$) Supposons que $s \ANtrans s'$,
    c'est-à-dire qu'il existe un libellé $\ell \in \ANi$ et un ensemble de transitions
    ayant ce libellé : $\ANt_\ell = \{ a_i \xrightarrow{\ell} a_j \in \ANt \}$,
    tels que $s' = s \play \ell$.
    D'après la traduction donnée à la \defref{an2phm}, il existe donc une action
    $h = \PHfrappemult{A}{B} \in \PHh$ telle que $A = \precond{\ell}$ et
    $B = \postcond{\ell} \setminus \{ a_i \in \ANProc \mid \ANaction{a_i}{\ell}{a_i} \in \ANt \}$.
    Or $s' = s \recouvre \postcond{\ell}
      = s \recouvre (B \cup \{ a_i \in \ANProc \mid \ANaction{a_i}{\ell}{a_i} \in \ANt \})
      = s \recouvre B$
    car $\{ a_i \in \ANProc \mid \ANaction{a_i}{\ell}{a_i} \in \ANt \} \subseteq s$.
    Ainsi, $h$ est jouable dans $s$ et $s' = s \play h$.
  
  ($\Leftarrow$) Supposons que $s \trans{\antophm} s'$,
    c'est-à-dire qu'il existe une action $h = \PHfrappemult{A}{B} \in \PHh$
    telle que $s' = s \play h$.
    D'après la traduction de la \defref{an2phm},
    cela signifie qu'il existe un libellé $\ell \in \ANi$ et un ensemble de transitions
    ayant ce libellé : $\ANt_\ell = \{ a_i \xrightarrow{\ell} a_j \in \ANt \}$,
    tels que : $A = \precond{\ell}$ et
    $B = \postcond{\ell} \setminus \{ a_i \in \ANProc \mid \ANaction{a_i}{\ell}{a_i} \in \ANt \}$.
    Comme $h$ est jouable dans $s$, alors $A \subseteq s$, donc $\ell$ est aussi jouable dans $s$.
    De plus, $s' = s \play h = s \recouvre B = s \recouvre (B \cup \invariant{h})
      = s \recouvre \postcond{\ell}$.
\end{proof}

\begin{theorem}[Équivalence entre réseaux d'automates synchronisés
  et Frappes de Processus avec actions plurielles]
\thmlabel{equivphman}
  Les Frappes de Processus avec actions plurielles sont aussi expressives
  que les réseaux d'automates synchronisés.
\end{theorem}

\begin{proof}
  D'après les \defref{phm2an,an2phm} et les \thmref{bisimulationphm2an,bisimulationan2phm}
  associés, tout modèle de Frappes de Processus avec actions plurielles peut être représenté
  à l'aide d'un réseau d'automates synchronisés, et inversement.
\end{proof}



\subsection{Traduction vers les Frappes de Processus avec 4 classes de priorités}
\seclabel{phm2php}

Nous proposons dans cette section une traduction des Frappes de Processus
avec actions plurielles en Frappes de Processus avec 4 classes de priorités (\defref{phm2php})
et nous montrons que les modèles obtenus de cette façon sont faiblement bisimilaires
(\thmref{phmbisimphp}).
Nous en déduisons que ces deux formalismes ont une expressivité équivalente,
%(\thmref{phmequivphp}).
car les Frappes de Processus avec 4 classes de priorités peuvent à leur tour être
traduites en Frappes de Processus canoniques (comme expliqué à la \vsecref{phm-aplatissement}),
celles-ci pouvant à leur tour être traduites en Frappes de Processus avec actions plurielles
\TODO.

La traduction proposée permet d'obtenir un modèle de
Frappes de Processus pseudo-canoniques avec 4 classes de priorités,
telles que définies plus tard, à la \secref{phcanonique}.
De façon informelle, ce modèle comporte 4 classes de priorités
et repose sur l'utilisation de deux types de sortes particulières :
\begin{itemize}
  \item des sortes coopératives pour vérifier la présence de tous les processus du frappeur,
  \item des \emph{sortes de réaction} permettant de modéliser le déclenchement d'une \emph{réaction},
    ou son arrêt.
\end{itemize}
Une réaction modélise le fait qu'un ensemble d'actions (standards) est en train
de simuler le jeu d'une action plurielle.
À toute action plurielle $h$ du modèle d'origine correspond une sorte coopérative $\scf{h}$
entre les sortes des processus de $A$
et une sorte de réaction $\sr{h}$ dans le modèle résultant de cette traduction.
La sorte coopérative comporte notamment un processus $\scf{h}_\mypi$ qui représente le sous-état
où tous les processus de $A$ sont présents ; une action de priorité 4 de la forme
$\PHhit{\scf{h}_\mypi}{\sr{h}_0}{\sr{h}_1}$ permet d'activer la sorte de réaction.
Une auto-action de priorité 3 de la forme $\PHhit{\sr{h}_1}{\sr{h}_1}{\sr{h}_0}$
permet de plus la désactivation de la sorte de réaction
une fois que toutes les actions de priorité 2 auront été jouées.
Les actions de priorité 2 ont la forme $\PHhit{\sr{h}_1}{b_j}{b_k}$
avec $b_j \in A$ et $b_k \in B$,
ce qui permet d'effectuer tous les bonds nécessaires à l'activation des processus de $B$.
Enfin, les sortes coopératives sont toutes mises à jour par des actions de priorité 1,
afin d'éviter les problèmes d'entrelacement
et de correspondre à la définition de Frappes de Processus pseudo-canoniques
(\defref{phpseudocanonique})
L'agencement de ces classes de priorités permet ainsi de simuler des actions plurielles
tout en empêchant l'entrelacement entre réactions
---~car deux réactions ayant lieu en même temps pourraient potentiellement
amener le système dans un état normalement inaccessible.

\begin{definition}
\deflabel{phm2php}
  Soient $\PH = (\PHs; \PHl; \PHh)$ des Frappes de Processus avec actions plurielles,
  et $\phmtophp = (\PHs'; \PHl'; \PHh'^{\langle 4 \rangle})$
  leur traduction en Frappes de Processus avec $4$ classes de priorités, où :
  \begin{itemize}
    \item $\PHs' = \PHs \cup \{ \sr{h} \mid h \in \PHh \} \cup \{ \scf{h} \mid h \in \PHh \}$ ;
    \item $\PHl' = \PHl \times \left( \bigtimes{h \in \PHh} \PHl_{\sr{h}} \right)
      \times \left( \bigtimes{h \in \PHh} \PHl_{\scf{h}} \right)$, où :
      \begin{itemize}
        \item $\forall h \in \PHh, \PHl_{\sr{h}} = \{ \sr{h}_0 , \sr{h}_1 \}$,
        \item $\forall h \in \PHh, \PHl_{\scf{h}} = \{ \scf{h}_\mysigma \mid
          \mysigma \in \PHsubl_{\sortes{\frappeur{h}}} \}$ ;
      \end{itemize}
    \item $\PHh'^{\langle 4 \rangle} = ( \PHh'^{(1)} ; \PHh'^{(2)} ; \PHh'^{(3)} ; \PHh'^{(4)} )$,
      où :
      \begin{itemize}
        \item $\PHh'^{(1)} = \{ \PHhit{a_i}{\scf{h}_\mysigma}{\scf{h}_{\mysigma'}} \mid
          h \in \PHh \wedge a \in \sortes{\frappeur{h}} \wedge a_i \in \PHl_a \wedge
          \scf{h}_\mysigma , \scf{h}_{\mysigma'} \in \PHl_{\scf{h}} \wedge
          \PHget{\mysigma}{a} \neq a_i \wedge \mysigma' = \mysigma \Cap a_i \}$,
        \item $\PHh'^{(2)} = \{ \PHhit{\sr{h}_1}{b_j}{b_k} \mid
          h \in \PHh \wedge b \in \sortes{\cible{h}} \wedge b_j, b_k \in \PHl_b \wedge
          b_j \in \cible{h} \wedge b_k \in \bond{h} \}$,
        \item $\PHh'^{(3)} = \{ \PHhit{\sr{h}_1}{\sr{h}_1}{\sr{h}_0} \mid h \in \PHh \}$,
        \item $\PHh'^{(4)} = \{ \PHhit{\scf{h}_\mypi}{\sr{h}_0}{\sr{h}_1} \mid
          h \in \PHh \wedge \scf{h}_{\mypi} \in \PHl_{\scf{h}} \wedge
%          \Feval{\Fopphm{h}}{\mypi} \}$.
          \frappeur{h} \subseteq \mypi \}$.
      \end{itemize}
  \end{itemize}
  Pour tout état $s \in \PHl$ de $\PH$,
  on note $\tophp{s}$ l'état correspondant dans $\phmtophp$ :
  \begin{itemize}
    \item $\forall a \in \PHs, \PHget{\tophp{s}}{a} = \PHget{s}{a}$,
    \item $\forall h \in \PHh, \PHget{\tophp{s}}{\sr{h}} = \sr{h}_0$,
    \item $\forall h \in \PHh, \PHget{\tophp{s}}{\scf{h}} = \scf{h}_\mysigma$,
      tel que $\forall a \in \sortes{\frappeur{h}}, \PHget{\mysigma}{a} = \PHget{\tophp{s}}{a}$.
  \end{itemize}
  À l'inverse, pour tout état $s' \in \PHl'$ de $\phmtophp$,
  on note $\tophm{s'}$ l'état correspondant dans $\PH$ :
  $\forall a \in \PHs, \PHget{\tophm{s'}}{a} = \PHget{s'}{a}$.
\end{definition}

\begin{theorem}[$\PH \approx \phmtophp$]
\thmlabel{phmbisimphp}
  Soient $\PH = (\PHs; \PHl; \PHh)$ des Frappes de Processus avec actions plurielles,
  et posons : $\phmtophp = (\PHs'; \PHl'; \PHh'^{\langle 4 \rangle})$.
  On a :
%   \[\forall s, s' \in \PHl, s \PHtrans s' \Longleftrightarrow
%     \tophp{s} \mtrans{\phmtophp} \tophp{s'} \enspace.\]
% 
  \[\forall s, s' \in \PHl, \exists h \in \PHh, s' = s \play h \Longleftrightarrow
    \exists \delta \in \Sce(\tophp{s}), \left( \tophp{s'} = \tophp{s} \play \delta
    \wedge \card{\toset{\delta} \cap \PHh'^{(4)}} = 1 \right) \enspace.\]
\end{theorem}

\begin{proof}
  Posons $\phmtophp = (\PHs'; \PHl'; \PHh'^{\langle 4 \rangle})$.
  Soient $s, s' \in \PHl$.
  
  ($\Rightarrow$) On suppose qu'il existe une action $h \in \PHh$ telle que $s' = s \play h$.
    D'après la \defref{phm2php},
    cela implique notamment l'existence de sortes $\sr{h}$ et $\scf{h}$ dans $\PHs'$,
    et des actions suivantes :
    \begin{itemize}
      \item $\PHh'^{(1)}_h = \{ \PHhit{a_i}{\scf{h}_\mysigma}{\scf{h}_{\mysigma'}} \mid
        a \in \sortes{\frappeur{h}} \wedge a_i \in \PHl_a \wedge
        \scf{h}_\mysigma , \scf{h}_{\mysigma'} \in \PHl_{\scf{h}} \wedge
        \PHget{\mysigma}{a} \neq a_i \wedge \mysigma' = \mysigma \Cap a_i \} \subset \PHh'^{(1)}$,
      \item $\PHh'^{(2)}_h = \{ \PHhit{\sr{h}_1}{b_j}{b_k} \mid
        b \in \sortes{\bond{h}} \wedge b_j, b_k \in \PHl_b \wedge
        b_j \in \frappeur{h} \wedge b_k \in \cible{h} \} \subset \PHh'^{(2)}$,
      \item $h_3 = \PHhit{\sr{h}_1}{\sr{h}_1}{\sr{h}_0} \in \PHh'^{(3)}$,
      \item $h_4 = \PHhit{\scf{h}_\mypi}{\sr{h}_0}{\sr{h}_1} \in \PHh'^{(4)}$ avec $\mypi$ tel que
        $\Feval{\Fopphm{h}}{\mypi}$.
    \end{itemize}
    Comme $h$ est jouable dans $s$, alors $\frappeur{h} \subseteq s$,
%    d'où $\frappeur{h} \subseteq \tophp{s}$.
    d'où $\scf{h}_{\mypi} \in \tophp{s}$, avec $\toset{\mypi} = \frappeur{h}$.
    Ainsi, $\PHhit{\scf{h}_\mypi}{\sr{h}_0}{\sr{h}_1}$ est jouable dans $\tophp{s}$.
    Toutes les actions de $\PHh'^{(2)}_h$ sont jouables dans $\tophp{s} \play h_4$.
    Elles peuvent être jouées successivement et alternativement avec des actions de
    $\PHh'^{(1)}$ car celles-ci modifient uniquement l'état de sortes coopératives,
    ce qui n'influe donc pas sur la jouabilité des actions de $\PHh'^{(2)}_h$.
    Soit $\PHh'^{(2)}_h = \{ h_2^i \}_{i \in \segm{1}{n}}$
    un étiquetage des actions de $\PHh'^{(2)}_h$
    avec $n = \card{\PHh'^{(2)}_h}$.
    On peut jouer depuis l'état $\tophp{s} \play h_4$ une séquence d'actions de la forme :
    $\delta_h =
      h_2^1 \cons \delta^1 \cons h_2^2 \cons \delta^2 \cons
      \ldots \cons
      h_2^n \cons \delta^n$
    où toutes les séquences $\delta^i$ avec $i \in \segm{1}{n}$
    sont des séquences d'actions de $\PHh^{(1)}$ mettant à jour des sortes coopératives.
    Après avoir joué cette séquence, le modèle se trouve dans un état
    $\tophp{s} \play h_4 \play \delta_h$ tel que :
    $\forall h_i \in \PHh'^{(2)}_h, \bond{h_i} \in (\tophp{s} \play h_4 \play \delta_h)$,
    c'est-à-dire : $\bond{h} \subseteq (\tophp{s} \play h_4 \play \delta_h)$.
    De plus, toutes les sortes coopératives sont mises à jour, ce qui signifie qu'aucune
    action de $\PHh'^{(1)}_h \cup \PHh'^{(2)}_h$ n'est plus jouable.
    Il est donc possible de jouer pour finir l'auto-action $h_3$,
    et on a :
    $\tophp{s} \play h_4 \play \delta_h \play h_3 = \tophp{s'}$,
    avec : $\toset{\delta} \cap \PHh'^{(4)} = \{ h_4 \}$.
  
  ($\Leftarrow$) On suppose qu'il existe un scénario $\delta \in \Sce(\tophp{s})$
    tel que $\tophp{s'} = \tophp{s} \play \delta$ et $\card{\toset{\delta} \cap \PHh'^{(4)}} = 1$.
    D'après la forme de $\phmtophp$ (\defref{phm2php}),
    et en s'inspirant du raisonnement précédent,
    on constate que la seule action jouable dans $\tophp{s}$ est une action de $\PHh'^{(4)}$,
    puis que dans l'état résultat, on ne peut jouer
    qu'une alternance entre une action de $\PHh'^{(2)}$ et des actions de $\PHh'^{(1)}$,
    et que l'état résultant une fois toutes les actions de $\PHh'^{(1)} \cup \PHh'^{(2)}$
    jouées ne permet que le jeu d'une action de $\PHh'^{(3)}$.
    On en déduit que ce scénario a nécessairement la forme suivante :
    $\delta = h_4 \cons \delta_{12} \cons h_3$,
    avec $h_4 \in \PHh'^{(4)}$, $h_3 \in \PHh'^{(3)}$, et
    $\delta_{12} = h_2^1 \cons \delta^1 \cons h_2^2 \cons \delta^2 \cons
      \ldots \cons h_2^n \cons \delta^n$
    où pour tout $i \in \segm{1}{n}$, $h_2^i \in \PHh'^{(2)}$
    et $\delta^i \in \Sce$ est un scénario composé uniquement d'actions de $\PHh'^{(1)}$.
    De plus, les seules actions jouables dans $\tophp{s} \play \delta$ sont à nouveau des actions
    de $\PHh'^{(4)}$, donc $\delta$ ne peut pas être plus grand car
    $\card{\toset{\delta} \cap \PHh'^{(4)}} = 1$.
    Posons :
    \begin{itemize}
      \item $\PHh'^{(1)}_\delta = \toset{\delta} \cap \PHh'^{(1)}$,
      \item $\PHh'^{(2)}_\delta = \toset{\delta} \cap \PHh'^{(2)}
        = \{ h_2^i \mid i \in \segm{1}{n} \}$,
      \item $h_3 = \PHhit{\sr{}_1}{\sr{}_1}{\sr{}_0}$ et
      \item $h_4 = \PHhit{\scf{}_\mypi}{\sr{}_0}{\sr{}_1}$.
    \end{itemize}
    D'après la construction de $\phmtophp$ donné à la \defref{phm2php},
    il existe donc nécessairement une action $h \in \PHh$
    dont $\sr{}$ est la sorte de réaction et $\scf{}$ est la sorte coopérative correspondante.
    De plus, toujours grâce à cette définition, on a :
    $\mypi \in \PHsubl[\phmtophp]$ et $\frappeur{h} \subseteq \mypi$.
    Autrement dit : $\frappeur{h} = \toset{\mypi}$,
    donc que $h$ est jouable dans $s$, par définition de $\tophp{s}$.
    De plus, toujours d'après la \defref{phm2php}, on a :
    $\forall i \in \segm{1}{n}, h_2^i = \PHfrappe{\sr{}_1}{b_j^i}{b_k^i}
      \wedge b_j^i \in \frappeur{h} \wedge b_k \in \bond{h}$.
    Ainsi, $\bond{h} = \{ b_k = \bond{h_2^i} \mid i \in \segm{1}{n} \}$.
    Or on constate immédiatement que toutes les sortes coopératives sont nécessairement
    mises à jour
    dans $s \play \delta$ car la dernière action jouée est une action de $\PHh'^{(3)}$
    et que son jeu ne rend aucune action de $\PHh'^{(1)}$ jouable.
    De plus, $\PHget{s \play \delta}{\sr{}} = \sr{}_0$ et
    $\tophp{s} \play \delta = \tophp{s} \recouvre \{ \bond{h_2^i} \mid i \in \segm{1}{n} \}$.
    Ainsi, $\tophp{s'} = \tophp{s} \play \delta = \tophp{s} \recouvre \bond{h}
      = \tophp{s \recouvre \bond{h}} = \tophp{s \play h}$.
\end{proof}



\todo{Exemple}



\subsection{Équivalence avec les Frappes de Processus avec classes de priorités}
\seclabel{equivphmphp}

\todo{Formaliser l'équivalence PHplurielles / PHp de façon formelle ?}

Cette section permet de formaliser l'équivalence d'expressivité entre les Frappes de Processus
avec actions plurielles d'une part, et les Frappes de Processus avec $k$ classes de priorités
d'autres part, si $k \in \sNN$.
Le \thmref{phmequivphp} formalise cette conclusion,
bien que sa démonstration s'appuie sur des résultats qui seront
présentés aux \vsecref{phm-aplatissement,phcanonique2phm} \storef.
En effet, elle s'appuie sur plusieurs résultats plus forts qui stipulent que des
Frappes de Processus avec actions plurielles
et les Frappes de Processus avec $k$ classes de priorités
peuvent toujours être représentées
par une classe particulière de Frappes de Processus avec 2 classes de priorités,
qui elle-même peut aussi être représentée par des
Frappes de Processus avec actions plurielles.

\begin{theorem}[Équivalence]
\thmlabel{phmequivphp}
  Pour tout $k \in \sNN$,
  les Frappes de Processus avec actions plurielles
  et les Frappes de Processus avec $k$ classes de priorités
  ont une expressivité équivalente.
\end{theorem}

\begin{proof}
  Soit $k \in \sNN$.
  D'après \storef, il est toujours possible de traduire des Frappes de Processus avec actions
  plurielles en Frappes de Processus canoniques, qui sont \textit{a fortiori}
  des Frappes de Processus avec $k$ classes de priorités.
  Inversement, toutes Frappes de Processus avec $k$ classes de priorités
  peuvent être aplaties en Frappes de Processus canoniques \storef,
  qui elles-mêmes peuvent être traduites en Frappes de Processus avec action plurielles \storef.
\end{proof}



\subsection{Réutilisation des outils existants}
\seclabel{phm-outils}

\todo{Revoir le titre de cette section}

\todo{Faudrait-il fusionner cette section avec ses semblables des \secref{php} et \secref{phan} ?}

\subsubsection{Points fixes}
\seclabel{phm-outils-pf}

Du fait de la complexité des actions plurielles, il n'existe pas à l'heure actuelle de méthode
générique efficace permettant de déduire les points fixes de Frappes de Processus avec
actions plurielles.
Cependant, à l'instar des Frappes de Processus avec arcs neutralisants
(\secref{phan-outils-pf})
il est possible d'obtenir ces points fixes les cherchant sur
le modèle aplati (tel que défini à la \secref{phm-aplatissement}).
En effet, sa dynamique étant équivalente, aux sortes coopératives près, le résultat
peut être transposé au modèle d'origine.

\subsubsection{Analyse statique}
\seclabel{phm-outils-as}

Du fait de la forme particulière des actions, les méthodes d'analyse statique ne s'appliquent pas
directement aux Frappes de Processus avec actions plurielles.
Cependant, moyennant l'utilisation de la traduction vers des Frappes de Processus
avec 4 classes de priorités donnée à la \defref{phm2php},
il est possible de les appliquer sur un modèle équivalent.

\subsubsection{Paramètres stochastiques}
\seclabel{phm-outils-stocha}

Il est théoriquement possible d'associer des paramètres stochastiques à chaque action plurielle
d'un modèle de Frappes de Processus avec actions plurielles.
Un tel ajout aurait notamment l'avantage d'éviter l'utilisation de la valeur «~infinie~»
d'absorption de stochasticité, qui avait principalement pour but de simuler des actions
successives instantanées, afin notamment de pallier le fait qu'un seul processus
ne peut évoluer pour chaque jeu d'action des Frappes de Processus standards.

Une autre alternative consisterait à utiliser l'aplatissement de la \secref{phm-aplatissement},
et d'attribuer aux actions correspondant à des activations de réaction
dans le modèle obtenu les paramètres stochastiques voulus.

