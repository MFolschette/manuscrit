
\chapter{Enrichissement des Frappes de Processus pour l'aide à la modélisation}
\chaplabel{sem}
\chaplabel{semantiques}

\chapeau{%
  La sémantique standard des Frappes de Processus de la \secref{ph} peut s'avérer insuffisante pour
  prendre en compte certaines informations connues sur le système étudié, comme des informations
  en terme de vitesse de réaction.
  De plus, certains comportement non désirés apparaissent dans les Frappes de Processus standards
  dès que l'on cherche à synchroniser plusieurs processus.
  Ce chapitre propose d'étendre sa sémantique afin de pallier ces problèmes
  en enrichissant les modèles sur deux axes :
  \begin{itemize}
    \item la préemption entre actions, qui permet d'empêcher le jeu d'une action sous certaines
      conditions,
    \item la simultanéité d'actions, qui permet de forcer le jeu simultané de plusieurs actions.
  \end{itemize}
  Ces deux axes ont pour but de permettre ou de faciliter l'intégration de telles informations
  au sein des modèles.

  Du premier axe découlent deux extensions aux Frappes de Processus
  prenant la forme d'arcs neutralisants, qui modélisent la préemption d'une action par une autre,
  et de classes de priorités, qui modélisent la préemption d'un ensemble d'actions par un autre.
  Le second axe apporte à une troisième extension, reposant sur la notion d'actions plurielles.
%   Les trois extensions aux Frappes de Processus proposées dans ce chapitre prennent la forme
%   de classes de priorités, d'arcs neutralisants et d'actions plurielles.
  Après les avoir définis définies dans ce chapitre,
  nous montrons que ces trois modélisations sont (faiblement) bisimilaires.
  De plus, il est à noter que l'analyse statique développée au \chapref{phcanonique}
  peut s'appliquer à toutes ces extensions, moyennant une traduction.
%   Les classes de priorités sont d'ailleurs réutilisées
%   pour définir la forme canonique des Frappes de Processus.
}

Nous présentons dans ce chapitre les trois sémantiques de Frappes de Processus développées
pour enrichir l'expressivité de ce formalisme.
Nous les définissons et discutons de leurs avantages en termes de modélisation
pour les réseaux de régulation biologiques ou les réseaux de réactions biochimiques.
De plus, nous traçons des liens formels entre ces différents formalismes
afin de mieux comprendre leur complémentarité
et d'offrir une bonne souplesse de représentation et d'analyse.

\myskip

Nous proposons dans le présent chapitre des outils permettant
d'enrichir un modèle de Frappes de Processus à l'aide de
contraintes dérivées d'informations biologiques
comme les vitesses de réaction,
la connaissance d'inhibitions de réactions en présence de certains composants,
ou celle des réactions précises ayant lieu au sein du système, etc.
La connaissance de telles informations peut permettre de privilégier un chemin
sur un autre
à d'un moment clef de l'évolution du système, empêchant ainsi une certaine évolution du système,
en favorisant l'apparition d'une autre.
Leur intégration permet donc d'affiner le modèle en réduisant les comportements possibles
afin d'obtenir un modèle plus proche du système étudié.
Ces connaissances peuvent être intégrées sous la forme de préemptions (la jouabilité d'une action
peut empêcher la jouabilité d'une autre action) ou de simultanéité
(plusieurs processus peuvent évoluer simultanément).

Pourtant, la sémantique standard des Frappes de Processus
développée par \citeasnoun{PMR10-TCSB}
et rappelée à la \vsecref{ph}
ne permet pas de concilier l'introduction de contraintes dérivées d'informations biologiques
et une bonne capacité d'analyse des modèles ainsi créés.
En effet, s'il est possible d'y intégrer des informations temporelles
sous la forme de paramètres stochastiques, tel que mentionné à la \vsecref{ph-stocha},
en revanche les analyses puissantes de la dynamiques rappelées à la \vsecref{ph-as-pf}
ne sont alors plus valables.
En effet, celles-ci ne prennent pas en compte les fenêtres de tir introduites par les
paramètres stochastiques.
L'analyse des modèles doit alors être effectuée à l'aide d'outils de model checking
probabilistes, qui doivent faire face à l'explosion combinatoire provoquée par l'ajout
de la dimension temporelle continue.
Il n'est alors généralement plus possible d'envisager d'étudier
les modèles de plus de cinq composants avec une précision acceptable
\cite[p.~170]{Pauleve11}.

De plus, au niveau de la modélisation, la représentation des coopérations avec des Frappes de
Processus standards souffre de certaines lacunes.
Ces coopérations sont modélisées à l'aide de sortes coopératives, décrites à la \secref{sc},
%dont la dynamique n'est pas strictement équivalente à celle d'une porte logique autorisant
%le jeu d'une action en présence de certains processus donnés.
et souffrent d'un décalage temporel entre les sortes à représenter et
la mise à jour du processus actif de la sortie coopérative,
qui peut entraîner l'existence de «~faux états~» pour la sorte coopérative.
%Ce décalage temporel implique que ce processus actif représente toujours
%une combinaison d'états passés des sortes en amont.
%S'il s'avère que cette combinaison coïncide dans la majeure partie des cas avec un état passé
%effectif,
%ou avec l'état présent, cela n'est pas toujours le cas, et il est possible qu'une sorte coopérative
%représente une combinaison d'états en pratique inaccessible pour les sortes en amont.
Plusieurs solutions sont proposées dans ce chapitre, qu'il s'agisse de rendre les actions de mise à
jour des sortes coopératives prioritaires ou plus simplement de les remplacer par une forme plus
complexe d'action.

L'une des pistes permettant d'affiner un modèle de Frappes de Processus consiste à y intégrer
des informations de préemption entre les actions afin d'affiner la dynamique.
Une telle approche permet de modéliser des contraintes temporelles,
toute action modélisant une réaction très rapide étant par exemple systématiquement jouée
avant les actions modélisant des réactions plus lentes.
D'autres contraintes peuvent aussi être prises en compte, comme la concurrence entre réactions,
mais aussi la représentation de processus propres à la modélisation et n'ayant pas nécessairement
de sens biologique.

Les formes alternatives de Frappes de Processus présentées dans ce chapitre se concentrent donc
sur les notions de préemption et de simultanéité d'une action par rapport à une autre.
La préemption permet à une action d'avoir priorité sur une autre ou, du point de vue inverse,
permet d'empêcher le jeu d'une action dans une situation où elle pourrait normalement être jouable
selon la dynamique des Frappes de Processus standards.
Une telle préemption peut être opérée de façon généralisée
%par une action sur un ensemble d'autres,
comme c'est le cas pour les Frappes de Processus avec classes de priorités (\secref{php}),
où chaque action peut bloquer l'ensemble des actions de priorité inférieure ;
ou de façon ponctuelle, comme au sein des Frappes de Processus avec arcs neutralisants
(\secref{phan}),
qui permet de définir des relations plus fines de préemption entre actions individuelles.
À l'inverse, la simultanéité entre actions permet de s'assurer qu'un ensemble de frappes est joué
de façon simultanée, ou plus généralement qu'un ensemble de processus bondit en même temps,
comme permettent de le représenter les Frappes de Processus avec actions plurielles (\secref{phm}).
Nous montrons enfin au cours de ce chapitre que ces différentes sémantiques sont deux à deux aussi
expressives.

Les apports de ces formes alternatives de Frappes de Processus permettent de restreindre
la dynamique
d'un modèle par rapport aux Frappes de Processus standards.
Elles se posent en alternatives à l'ajout de paramètres stochastiques dans les Frappes de Processus
(cf. \vsecref{ph-stocha})
qui permettent d'ajouter une dimension probabiliste dans ce formalisme.
Leur principal atout est de renforcer la puissance d'expression des Frappes de Processus,
ce qui a pour conséquence de simplifier l'écriture et la lecture des modèles,
mais aussi d'offrir de nouvelles capacités de modélisation.
%selon les informations que l'on souhaite y intégrer.
Par ailleurs,
ces différents formalismes sont tous compatibles avec les méthodes d'analyse statique
développées au \chapref{phcanonique},
ce qui assure de pouvoir étudier efficacement la dynamique des modèles créés.

Ces différentes notions font naturellement écho aux problématiques plus générales d'enrichissement
dans les modèles discrets.
Ainsi, on peut par exemple rapprocher la notion d'arc neutralisant des Frappes de Processus
à celle propre aux réseaux de Petri,
à la différence que celle-ci permet de préempter des actions en fonction de l'activité
d'une place (correspondant ici à un processus ou à une sorte, selon le point de vue).
De même, la notion de priorités est aussi présente dans certaines sémantiques de réseaux
de Petri \toref.
Enfin, les actions plurielles permet de se rapprocher de la classe des modèles
à dynamique synchrone \todo{Biocham, graphes des états...}.

La définition du formalisme des Frappes de Processus avec classes de priorités
a été publiée dans \cite{FPMR13-CS2Bio}.



% Frappes de Processus avec classes de priorités
\subsection{Frappes de Processus avec priorités quelconques}

Les Frappes de Processus avec $k$ classes de priorités
(aussi appelées «~Frappes de Processus~» dans la suite, lorsque ce n'est pas ambigu),
consistent, comme formalisé par la \defref{php},
en une extension des Frappes de Processus contenant, à la place d'un ensemble d'actions,
$k$ ensembles d'actions, chacun étant associé à une classe de priorités distincte.
Avec cette sémantique, une action est jouable dans un état si et seulement si,
en plus de la condition de la présence du frappeur et de la cible,
il n'y a aucune autre action appartenant à une classe de priorité plus grande
qui soit aussi jouable dans cet état.

Cette représentation basée sur des classes de priorités permet de modéliser un système
dont les actions peuvent être distinguées en plusieurs classes en fonction de leur importance,
de leur vitesse d'exécution, ou encore d'autres facteurs leur donnant prévalence sur d'autres.

Elle offre notamment la possibilité de distinguer les actions en fonction 
de leur vitesse d'exécution (les actions les plus rapides étant jouées en priorité),
de leur fonction (biologique donc lente ou non-biologique donc instantanée)
ou d'autres paramètres permettant de déterminer l'existence de la préemption d'une action
en fonction de la jouabilité d'une autre.
L'application la plus poussée de cette modélisation consisterait à
classer les actions d'un modèle en fonction d'un critère telle que la vitesse de réaction,
et à attribuer à chaque classe de priorité une action unique en fonction de ce classement,
afin de rendre compte de la priorité de chaque action en fonction des autres.

Cette modélisation présente néanmoins quelques lacunes.
Les phénomènes d'accumulation, notamment, n'y sont pas représentés ;
un cycle d'actions prioritaires ne peut jamais être interrompu par une action moins prioritaire,
menant à un cycle infini et pouvant contredire la réalité biologique.
De plus, les classes de priorités définies pour un modèle sont invariables;
certains modèles pourraient cependant nécessiter l'évolution de certaines classes de priorités
en fonction de la présence ou de l'absence d'un composant dans un état donné.

\begin{definition}[Frappes de Processus avec $k$ classes de priorités]
\deflabel{php}
  Si $k \in \sN^*$, les \emph{Frappes de Processus avec $k$ classes de priorités} sont définies
  par un triplet $\PH = (\PHs; \PHl; \PHh^{\langle k \rangle})$,
  où $\PHh^{\langle k \rangle} = (\PHh^{(1)}; \dots; \PHh^{(k)})$ est un $k$-uplet, et :
  \begin{itemize}
    \item $\PHs \DEF \{a, b, \dots\}$ est l'ensemble fini et dénombrable des \emph{sortes} ;
    \item $\PHl \DEF \underset{a \in \PHs}{\times} \PHl_a$ est l'ensemble fini des \emph{états},
      où $\PHl_a = \{a_0, \ldots, a_{l_a}\}$ est l'ensemble fini et dénombrable
      des \emph{processus} de la sorte $a \in \PHs$ et $l_a \in \sN^*$.
      Chaque processus appartient à une unique sorte :
      $\forall (a_i; b_j) \in \PHl_a \times \PHl_b, a \neq b \Rightarrow a_i \neq b_j$ ;
    \item pour tout $n \in \llbracket 1; k \rrbracket$,
      $\PHh^{(n)} \DEF \{\PHfrappe{a_i}{b_j}{b_l} \mid (a; b) \in \PHs^2 \wedge
      (a_i; b_j; b_l) \in \PHl_a \times \PHl_b \times \PHl_b \wedge
      b_j \neq b_l \wedge a = b \Rightarrow a_i = b_j \}$ est l'ensemble fini
      des \emph{actions de priorité $n$}.
  \end{itemize}
  On appelle $\PHproc \DEF \bigcup_{a \in \PHs} \PHl_a$ l'ensemble de tous les processus,
  et $\PHh \DEF \bigcup_{n \in \segm{1}{k}} \PHh^{(n)}$ l'ensemble de toutes les actions.
  De plus, pour tout $n \in \sN^*$ et $h \in \PHh^{(n)}$, on note : $\prio(h) \DEF n$.
\end{definition}
%
\noindent
\todo{Reprendre les notation existantes}
La sorte d'un processus $a_i$ est donnée par $\PHsort(a_i) = a$.
Étant donné un état $s \in \PHl$, le processus de la sorte $a \in \PHs$ présent dans $s$ est donné
par $\PHget{s}{a}$, \cad la coordonnée correspondant à $a$ dans l'état $s$.
Si $a_i \in \PHl_a$, nous définissons la notation : $a_i \in s \EQDEF \PHget{s}{a} = a_i$.

\begin{definition}[Propriété de jouabilité des Frappes de Processus
  ($\Fopsymbol : \PHh \rightarrow \F$)]
\deflabel{fopphp}
    \[\forall h \in \PHh, \Fop{h} \equiv \hitter{h} \wedge
    \left( \bigwedge_{\substack{g \in \PHh^{(n)}\\n < \prio(h)}}
    \neg \left( \target{g} \wedge \hitter{g}\right) \right)\]
\end{definition}
\todo{Est-ce une bonne idée ? L'opérateur pour la traduction est différent.}


\begin{definition}[Dynamique des Frappes de Processus ($\PHtrans$)]
\deflabel{play}
  Une action $h = \PHhit{a_i}{b_j}{b_k} \in \PHh^{(n)}$ de priorité $n$ est dite \emph{jouable}
  dans $s \in \PHl$ si et seulement si $\PHget{s}{a} = a_i$, $\PHget{s}{b} = b_j$
  et $\forall m < n, \forall g \in \PHh^{(m)}, \PHhitter(g) \notin s \vee \PHtarget(g) \notin s$.
  Dans ce cas, $(s \PHplay h)$ est l'état résultant du jeu de l'action $h$ dans $s$,
  et on le définit par : $(s \PHplay h) = s \Cap b_k$.
  De plus, on note alors : $s \PHtrans (s \PHplay h)$.

  Si $s \in \PHl$, un \emph{scénario} $\delta$ dans $s$
  est une séquence d'actions de $\PHh$ qui peuvent être jouées successivement dans $s$.
  L'ensemble de tous les scénarios dans $s$ est noté $\Sce(s)$.
\end{definition}


% Frappes de Processus avec arcs neutralisants
\subsection{Frappes de Processus avec arcs neutralisants}
\seclabel{phan}

\todo{Ok avec la terminologie de cette section ?}

Nous introduisons ici la notion d'arc neutralisant dans les Frappes de Processus (\defref{php})
afin de représenter la préemption d'une action par une seule autre.
Une telle représentation permet notamment une modélisation plus atomique
par rapport aux classes de priorités présentées à la \secref{php}.

Un arc neutralisant est un couple d'actions noté $\PHan{h_1}{h_2}$,
où $h_1$ est appelée \emph{action bloquante},
et peut préempter $h_2$, qui appelée \emph{action bloquée},
dans certaines situations.
Avec la présence d'arcs neutralisants, une action est dite \emph{activée} dans un état donné si
son frappeur et sa cible y sont présents ;
le fait pour une action d'être activée remplace donc celui d'être jouable
par rapport à la sémantique standard des Frappes de Processus telle que décrite à la \secref{ph}.
Un action n'est ici \emph{jouable} que si elle est activée,
et que pour tout arc neutralisant la bloquant, son action bloquante n'est pas activée.
Une action activée mais qui n'est pas jouable est dite \emph{neutralisée}.

Il est à noter que la neutralisation d'une action par une autre ne dépend dont pas de la jouabilité
de l'action bloquante, mais uniquement de son activation.
Cela permet d'avoir un modèle cohérent, sans quoi certaines situations pourraient ne pas être
définies, notamment dans le cas d'un interblocage.
Ainsi, faire reposer la neutralisation d'une action bloquée sur la jouabilité de l'action bloquante
devient inextricable dans un cas comme le suivant :
$\PHan{h_1}{h_2}$, $\PHan{h_2}{h_3}$ et $\PHan{h_3}{h_1}$,
car si les trois actions $h1$, $h2$ et $h3$ sont actives, leur jouabilité resté indéterminée.
En revanche, si cette neutralisation repose sur l'état activé d'une action,
la situation précédente se résout immédiatement car aucune des trois actions n'est jouable.
On constate par ailleurs qu'une action peut en neutraliser une autre
même si elle-même est neutralisée.
Nous ne nous avancerons cependant pas sur la signification biologique de ce fait.

\begin{definition}[Frappes de Processus avec arcs neutralisants]
\deflabel{phan}
  Les \emph{Frappes de Processus avec arcs neutralisants} sont définies par
  un quadruplet $\PH = (\PHs ; \PHl ; \PHh ; \PHn)$, où :
  \begin{itemize}
    \item $\PHs \DEF \{a, b, \dots\}$ est l'ensemble fini et dénombrable des \emph{sortes} ;
    \item $\PHl \DEF \bigtimes{a \in \PHs} \PHl_a$ est l'ensemble fini des \emph{états},
      où $\PHl_a = \{a_0, \ldots, a_{l_a}\}$ est l'ensemble fini et dénombrable
      des \emph{processus} de la sorte $a \in \PHs$ et $l_a \in \sN^*$.
      Chaque processus appartient à une unique sorte :
      $\forall (a_i; b_j) \in \PHl_a \times \PHl_b, a \neq b \Rightarrow a_i \neq b_j$ ;
    \item $\PHh \DEF \{\PHfrappe{a_i}{b_j}{b_l} \mid (a; b) \in \PHs^2 \wedge
      (a_i; b_j; b_l) \in \PHl_a \times \PHl_b \times \PHl_b \wedge
      b_j \neq b_l \wedge a = b \Rightarrow a_i = b_j \}$ est l'ensemble fini des actions ;
    \item $\PHn = \{\PHan{h_1}{h_2} \mid (h_1 ; h_2) \in \PHh \times \PHh\}$
      est l'ensemble fini des arcs neutralisants.
  \end{itemize}
\end{definition}

Un arc neutralisant $u = \PHan{h_1}{h_2} \in \PHn$ est donc un couple d'actions.
On note $\PHbloquant(u) = h_1$ la première action du couple $u$
et $\PHbloque(u) = h_2$ sa seconde action.
On réutilise par ailleurs les autres notations définies à la \secref{ph}.

\begin{definition}[Opérateur de jouabilité ($\Fopsymbol_\Fopphansubsymbol : \PHh \rightarrow \F$)]
\deflabel{fophan}
  L'opérateur de jouabilité des Frappes de Processus avec arcs neutralisants est défini par :
  \[\forall h \in \PHh, \Fopphan{h} \equiv \hitter{h} \wedge \target{h} \wedge
    \left( \bigwedge_{\substack{u \in \PHn\\u = \PHan{g}{h}}}
    \neg \left( \target{g} \wedge \hitter{g} \right) \right)\]
\end{definition}

\todo{Fonctionne encore à condition de prendre en compte les préemptions : analyse stochastique}

\todo{Ne fonctionne plus : points fixes}

\todo{Fonctionne avec traduction (PHcanonique) : analyse statique}

\todo{Traduction vers PHcanonique}


% Frappes de Processus avec actions plurielles
\section{Enrichissement avec simultanéité des actions : Frappes de Processus avec actions plurielles}
\seclabel{phm}

\todo{Introduction}

Les Frappes de Processus avec actions plurielles permettent de représenter des systèmes
de réactions biochimiques de la forme : $X + Y \rightarrow Y + Z$,
c'est-à-dire comportant un ensemble de réactifs $X$, un ensemble de catalyseurs $Y$
et un ensemble de produits $Z$.
La représentation d'une telle réaction en Frappes de Processus avec actons plurielles
aura la forme suivante :
$\PHfrappemult{A}{B}$ où $A = X \cup Y$ et $B = Z$,
où $A$ et $B$ sont des ensembles des processus.

Une telle action peut donc être jouée dans un état contenant tous les processus de $A$
(c'est-à-dire les réactifs et les catalyseurs)
et fait évoluer celui-ci vers un état contenant tous les processus de $B$ (les produits),
les autres processus restant inchangés (car n'intervenant pas ou en tant que catalyseurs).
Cela implique toutefois que pour tout processus de $B$, on trouve un autre processus de la même
sorte dans $A$.

On note cependant qu'une réaction de la forme $\PHfrappemult{\{ a_0, b_0, c_0 \}}{\{ a_1, b_1 \}}$
ne peut être jouée si l'une des deux sortes entrant en jeu, $a$ et $b$, est déjà au niveau $1$,
même si l'autre est au niveau $0$.
Un tel comportement a du sens lorsque les différents processus d'une sorte
($a_0$ et $a_1$, par exemple)
représentent différents états d'une même molécule :
la réaction ne peut alors pas être jouée pour des raisons de stœchiométrie.
Cependant, si ces différents processus représentent plutôt des niveaux de concentration
($a_1$ représentant par exemple un niveau de concentration de la molécule $a$ plus élevé que $a_0$),
cette restriction n'a plus de sens car une plus forte concentration d'une des entités
ne devrait pas empêcher la réaction d'avoir lieu et de produire la seconde entité.
Cela peut néanmoins être corrigé en ajoutant les actions
$\PHfrappemult{\{ a_1, b_0, c_0 \}}{\{ a_1, b_1 \}}$ et
$\PHfrappemult{\{ a_0, b_1, c_0 \}}{\{ a_1, b_1 \}}$,
ou encore en séparant la production de $a_1$ et de $b_1$ en deux actions (ou ensemble d'actions)
distinctes.

On peut utiliser les Frappes de Processus avec actions plurielles pour représenter toute simultanéité
entre les bonds ou entre les processus requis (aussi appelés frappeurs).
Elle est particulièrement adaptée à la forme des réactions biochimiques,
comme expliqué ci-dessus.

Cette forme des Frappes de Processus peut être aisément représentée à l'aide d'un réseau
d'automates synchronisés, chaque sorte ayant le rôle d'un automate et chaque action celui d'un
ensemble de transition étiquetées avec le même libellé partant chacune d'un processus dans $A$ et
arrivant dans le processus de la même sorte dans $A \recouvre B$,
comme décrit à la \secref{phm2an}.
On peut aussi la représenter à l'aide de Frappes de Processus avec 4 classes de priorités,
comme détaillé à la \secref{phm2php} ;
cependant, cette représentation a l'inconvénient d'être moins claire car faisant intervenir
un nombre important d'actions et de sortes supplémentaires.



\subsection{Définition}

\todo{Gluer}

\begin{definition}[Frappes de Processus avec actions plurielles]
\deflabel{phm}
  Les \emph{Frappes de Processus avec actions plurielles} sont définies
  par un triplet $\PH = (\PHs; \PHl; \PHh)$, où :
  \begin{itemize}
    \item $\PHs \DEF \{a, b, \dots\}$ est l'ensemble fini et dénombrable des \emph{sortes} ;
    \item $\PHl \DEF \bigtimes{a \in \PHs} \PHl_a$ est l'ensemble fini des \emph{états},
      où $\PHl_a = \{a_0, \ldots, a_{l_a}\}$ est l'ensemble fini et dénombrable
      des \emph{processus} de la sorte $a \in \PHs$ et $l_a \in \sN^*$.
      Chaque processus appartient à une unique sorte :
      $\forall (a_i; b_j) \in \PHl_a \times \PHl_b, a \neq b \Rightarrow a_i \neq b_j$ ;
    \item $\PHh \DEF \{\PHfrappemult{A}{B} \mid A, B \in \PHsublset \setminus \emptyset \wedge
      \forall q \in B, \exists p \in A, (p \neq q \wedge \PHsort(p) = \PHsort(q)) \}$
      est l'ensemble fini des \emph{actions}.
  \end{itemize}
\end{definition}
%
\noindent
Pour toute action $h = \PHfrappemult{A}{B} \in \PHh$,
$A$ est appelé le \emph{frappeur}, et $b_k$ le \emph{bond} de $h$,
et on note : $\hitter{h} = A$, $\bounce{h} = B$.
On note de plus :
$\target{h} = \{ p \in A \mid \exists q \in B, \PHsort(p) = \PHsort(q) \}$.

\begin{definition}[Opérateur de jouabilité ($\Fopsymbol_\Fopphmsubsymbol : \PHh \rightarrow \F$)]
\deflabel{fopphm}
  L'opérateur de jouabilité des Frappes de Processus avec actions plurielles est défini par :
  \[\forall h \in \PHh, \Fopphm{h} \equiv \bigwedge_{p \in \hitter{h}} p \enspace.\]
\end{definition}



\subsection{Équivalence avec les automates synchronisés}
\seclabel{phm2an}

\todo{Gluer}
On note $\phmtoan$ l'automate équivalent aux frappes de processus $\PH$, défini par :

\begin{definition}[Automate équivalent ($\phmtoansymbol$)]
\deflabel{phm2an}
  Le réseau d'automates équivalent aux Frappes de Processus avec actions plurielles $\PH$
  est défini par : $\phmtoan = (\PHs, \PHl, \ANi, \ANt)$, où :
  \begin{itemize}
    \item $\ANi = \{ \ell_h \mid h \in \PHh \}$ est l'ensemble des libellés de transitions ;
    \item $\ANt = \{ p \xrightarrow{\ell_h} q \mid
      h \in \PHh \wedge h = \PHfrappemult{A}{B} \wedge p \in A \wedge q \in A \recouvre B \wedge
      \PHsort(p) = \PHsort(q) \}$ est l'ensemble des transitions locales.
  \end{itemize}
\end{definition}

\todo{Définir la dynamique des réseaux d'automates}

\begin{theorem}[$\PH \approx \phmtoan$]
\thmlabel{phmequivan}
  \[\forall s, s' \in \PHl, s \PHtrans s' \Longleftrightarrow s \trans{\phmtoan} s' \enspace.\]
\end{theorem}

\todo{Preuve}



\subsection{Équivalence avec les Frappes de Processus avec 4 classes de priorités}
\seclabel{phm2php}

\begin{comment}
\newcommand{\phmtophan}{\mathsf{phm2phan}}
\newcommand{\srsymbol}{r}
\newcommand{\sr}[1]{\srsymbol^{#1}}
\newcommand{\srgsymbol}{R}
\newcommand{\srg}{\srgsymbol}
\newcommand{\scfsymbol}{f}
\newcommand{\scf}[1]{\scfsymbol^{(#1)}}
\newcommand{\scfa}{\scf{A}}
\newcommand{\scfb}{\scf{A \recouvre B}}

\newcommand{\PHssr}[1][\PHs']{#1_{\srsymbol}}
\newcommand{\PHsscfa}[1][\PHs']{#1^A_{\scfsymbol}}
\newcommand{\PHsscfb}[1][\PHs']{#1^B_{\scfsymbol}}

\newcommand{\PHhsr}[1][\PHh']{#1_{\srsymbol}}
\newcommand{\PHhscfa}[1][\PHh']{#1^A_{\scfsymbol}}
\newcommand{\PHhscfb}[1][\PHh']{#1^B_{\scfsymbol}}

\begin{definition}[Représentation avec arcs neutralisants]
\deflabel{phm2phan}
  On note : $\forall A \in \PHsublset, \lozsc{x}{A} =
  \{ \mysc{x}{A}_\mysigma \mid \mysigma \in \PHsubl_{\sortes{A}} \}$
  où $\sortes{A} = \{ \sorte{p} \mid p \in A \}$.
  De plus, on note : $\mysc{x}{A}_\mypi \in \lozsc{x}{A}$
  le processus tel que $\Feval{\Fconj{A}}{\mypi}$.

  Soit $\PH = (\PHs; \PHl; \PHh)$ et $\phmtophan(\PH) = (\PHs'; \PHl'; \PHh'; \PHn')$
  sa traduction en Frappes de Processus avec arcs neutralisants.
  \begin{itemize}
    \item $\PHs' = \PHs \cup \PHs'_{\srsymbol} \cup \PHs'^A_{\scfsymbol} \cup
      \PHs'^B_{\scfsymbol} \cup \{ \srg \}$ où :
      \begin{itemize}
        \item $\PHssr = \{ \sr{h} \mid h \in \PHh \}$,
        \item $\PHsscfa = \{ \scf{A} \mid \exists h \in \PHh, A = \hitter{h} \}$,
        \item $\PHsscfb = \{ \scf{A \recouvre B} \mid
          \exists h \in \PHh, h = \PHfrappemult{A}{B} \}$ ;
      \end{itemize}
    \item $\PHl' = \PHl \times
      \left( \bigtimes{\sr{h} \in \PHssr} \PHl_{\sr{h}} \right) \times
      \left( \bigtimes{\scfa \in \PHsscfa} \PHl_{\scfa} \right) \times
      \left( \bigtimes{\scfb \in \PHsscfa} \PHl_{\scfb} \right) \times
      \PHl_\srg$ où :
      \begin{itemize}
        \item $\forall \sr{h} \in \PHssr, \PHl_{\sr{h}} = \{ \sr{h}_0, \sr{h}_1 \}$,
        \item $\forall \scfa \in \PHsscfa, \PHl_{\scfa} = \makesc{\scfa}$,
          %\{ \scfa_\mysigma \mid
          %\mysigma \in \PHsubl_{\sort{A}} \}$,
        \item $\forall \scfb \in \PHsscfb, \PHl_{\scfb} = \makesc{\scfb}$,
          %\{ \scfb_\mysigma \mid
          %\mysigma \in \PHsubl_{\sort{A \recouvre B}} \}$,
        \item $\PHl_\srg = \makesc{\srg^{(\theta)}}$
          avec $\theta = \{ \sr{h}_0 \mid \sr{h} \in \PHssr \}$ ;
          %\{ \srg_\mypi \mid
          %\mypi \in \PHsubl_{\{ \sr{h}_1 \mid \sr{h} \in \PHssr \}} \}$ ;
      \end{itemize}
    \item $\PHh'^{(1)} = \PHhscfa \cup \PHhscfb$ où :
      \begin{itemize}
        \item $\PHhsr = \{ \PHfrappe{a_i}{\scfa_\mysigma}{\scfa_{\mysigma'}} \mid
          \exists h = \PHfrappemult{A}{B} \in \PHh \wedge
          \scfa \in \PHsscfa \wedge
          \scfa_\mysigma , \scfa_{\mysigma'} \in \PHl_{\scfa} \wedge
          a \in \sortes{A} \wedge a_i \in \PHl_a \wedge
          \PHget{\mysigma}{a} \neq a_i \wedge \mysigma' = \mysigma \recouvre a_i \}$,
        \item $\PHhscfb = \{ \PHfrappe{b_j}{\scfb_\mysigma}{\scfb_{\mysigma'}} \mid
          \exists h = \PHfrappemult{A}{B} \in \PHh \wedge
          \scfb \in \PHsscfb \wedge
          \scfb_\mysigma , \scfb_{\mysigma'} \in \PHl_{\scfb} \wedge
          b \in \sortes{A \recouvre B} \wedge b_j \in \PHl_b \wedge
          \PHget{\mysigma}{b} \neq b_j \wedge \mysigma' = \mysigma \recouvre b_j \}$;
      \end{itemize}
    \item $\PHh'^{(2)} = \PHhsr \cup \PHh'_R$ où :
      \begin{itemize}
        \item $\PHhsr = ...$
      \end{itemize}


  \end{itemize}
  \todo{Revoir traduction avec 3 classes de priorités (plus simple)}
\end{definition}
\end{comment}



\begin{definition}
\deflabel{phm2php}
  Soient $\PH = (\PHs; \PHl; \PHh)$ des Frappes de Processus avec actions plurielles,
  et $\phmtophp = (\PHs'; \PHl'; \PHh'^{\langle 4 \rangle})$
  leur traduction en Frappes de Processus avec $4$ classes de priorités, où :
  \begin{itemize}
    \item $\PHs' = \PHs \cup \{ \sr{h} \mid h \in \PHh \} \cup \{ \scf{h} \mid h \in \PHh \}$ ;
    \item $\PHl' = \PHl \times \left( \bigtimes{h \in \PHh} \PHl_{\sr{h}} \right)
      \times \left( \bigtimes{h \in \PHh} \PHl_{\scf{h}} \right)$, où :
      \begin{itemize}
        \item $\forall h \in \PHh, \PHl_{\sr{h}} = \{ \sr{h}_0 , \sr{h}_1 \}$,
        \item $\forall h \in \PHh, \PHl_{\scf{h}} = \{ \scf{h}_\mysigma \mid
          \mysigma \in \PHsubl_{\sortes{\frappeur{h}}} \}$ ;
      \end{itemize}
    \item $\PHh'^{\langle 4 \rangle} = ( \PHh'^{(1)} ; \PHh'^{(2)} ; \PHh'^{(3)} ; \PHh'^{(4)} )$, où :
      \begin{itemize}
        \item $\PHh'^{(1)} = \{ \PHhit{a_i}{\scf{h}_\mysigma}{\scf{h}_{\mysigma'}} \mid
          h \in \PHh \wedge a \in \sortes{\frappeur{h}} \wedge a_i \in \PHl_a \wedge
          \scf{h}_\mysigma , \scf{h}_{\mysigma'} \in \PHl_{\scf{h}} \wedge
          \PHget{\mysigma}{a} \neq a_i \wedge \mysigma' = \mysigma \Cap a_i \}$,
        \item $\PHh'^{(2)} = \{ \PHhit{\sr{h}_1}{b_j}{b_k} \mid
          h \in \PHh \wedge b \in \sortes{\bond{h}} \wedge b_j, b_k \in \PHl_b \wedge
          b_j \in \frappeur{h} \wedge b_k \in \cible{h} \}$,
        \item $\PHh'^{(3)} = \{ \PHhit{\sr{h}_1}{\sr{h}_1}{\sr{h}_0} \mid h \in \PHh \}$,
        \item $\PHh'^{(4)} = \{ \PHhit{\scf{h}_\mypi}{\sr{h}_0}{\sr{h}_1} \mid
          h \in \PHh \wedge \Feval{\Fopphm{h}}{\mypi} \}$.
      \end{itemize}
  \end{itemize}
  Pour tout état $s \in \PHl$ de $\PH$,
  on note $\tophp{s}$ l'état correspondant dans $\phmtophp$ :
  \begin{itemize}
    \item $\forall a \in \PHs, \PHget{\tophp{s}}{a} = \PHget{s}{a}$,
    \item $\forall h \in \PHh, \PHget{\tophp{s}}{\sr{h}} = \sr{h}_0$,
    \item $\forall h \in \PHh, \PHget{\tophp{s}}{\scf{h}} = \scf{h}_\mysigma$,
      tel que $\forall a \in \sortes{\frappeur{h}}, \PHget{\mysigma}{a} = \PHget{\tophp{s}}{a}$.
  \end{itemize}
  À l'inverse, pour tout état $s' \in \PHl'$ de $\phmtophp$,
  on note $\tophm{s'}$ l'état correspondant dans $\PH$ :
  $\forall a \in \PHs, \PHget{\tophm{s'}}{a} = \PHget{s'}{a}$.
\end{definition}

\begin{theorem}[$\PH \approx \phmtophp$]
\thmlabel{phmequivphp}
  \[\forall s_1, s_2 \in \PHl, s \PHtrans s' \Longleftrightarrow
    \tophp{s_1} \trans{\phmtophp} \tophp{s_2} \enspace.\]
\end{theorem}

\todo{Preuve}




