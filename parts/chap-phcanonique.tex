
\chapter{Représentation canonique et analyse des Frappes de Processus}

\chapeau{%
  Une analyse statique efficace des questions d'atteignabilité avait précédemment été développée
  sur les Frappes de Processus standards dans~\stodo{PMR12-MSCS}.
  Elle ne peut cependant pas être appliquée directement aux
  formalismes alternatifs de Frappes de Processus présentés au \chapref{sem}.
  
  L'objectif de ce chapitre est de présenter les travaux ayant permis
  de développer une nouvelle forme de sous-approximation pour l'analyse statique.
  Celle-ci s'applique à une classe restreinte de Frappes de Processus avec 2 classes de priorités,
  appelées \emph{Frappes de Processus canoniques}.
  Nous montrons par ailleurs que toutes les sémantiques alternatives de Frappes de Processus
  introduites au \chapref{sem} peuvent être traduites en Frappes de Processus canoniques,
  prouvant d'une part la large expressivité de cette classe de modèles,
  et permettant d'autre part d'appliquer la sous-approximation développée dans la suite
  à tous ces formalismes.
}

Les Frappes de Processus standards (\defref{ph}) sont très atomiques,
et leurs actions ont une forme particulière :
un processus peut déclencher le bond d'un autre processus dans sa sorte.
Autrement dit, le changement du processus actif d'une sorte est conditionné par la présence
d'au plus un autre processus actif.
Dans le cas d'une auto-action $\PHfrappe{b_j}{b_j}{b_k}$, par exemple,
le frappeur et la cible $b_j$ sont confondus ;
il n'y a donc formellement aucun prérequis pour que $a_i$ bondisse vers $a_k$.
Dans le cas plus général d'une action $\PHfrappe{a_i}{b_j}{b_k}$, où $a_i \neq b_j$,
alors le seul prérequis pour que $b_j$ bondisse en $b_k$ est la présence de $a_i$.

Cette restriction sur la forme des actions permet une analyse efficace des questions d'atteignabilité
au sein des Frappes de Processus standards, qui s'expriment de la façon suivante :
\begin{center}
  «~Partant d'un état donné, est-il possible, en jouant un nombre quelconque d'actions,
  d'atteindre un état dans lequel un processus donné est actif ?~»
\end{center}
En termes plus formels, l'atteignabilité d'un processus $b_k$ depuis un état $s$
se définit comme l'existence d'un scénario $\delta \in \Sce(s)$, tel que
$\PHget{(s \play \delta)}{b} = b_k$.

En effet, la forme particulière des actions impose au plus un prérequis pour pouvoir faire bondir
le processus actif d'une sorte vers un autre niveau.
Partant de ce constat, une analyse statique permettant de répondre efficacement aux
questions d'atteignabilité a été développée par Loïc Paulevé dans~\stodo{PMR12-MSCS}.
Si on considère l'exemple de l'atteignabilité d'un processus $b_k$ depuis un état $s$
tel que $\PHget{s}{b} = b_j$, avec $b_j \neq b_k$,
la méthode utilisée repose sur l'analyse de la dynamique locale de $b$, ou, autrement dit,
sur l'analyse des bonds entre processus de $b$ produits par les actions frappant cette sorte.
En particulier, si on reprend les deux exemples d'actions donnés ci-dessus,
\begin{itemize}
  \item s'il existe une auto-action $\PHfrappe{b_j}{b_j}{b_k}$, alors la réponse est immédiate
    car $b_k$ est accessible en jouant cette action qui n'a pas de prérequis;
  \item s'il existe une action $\PHfrappe{a_i}{b_j}{b_k}$, alors l'atteignabilité de $b_k$
    est conditionnée par l'atteignabilité de $a_i$ --- à la condition près que cette
    seconde atteignabilité laisse $b_j$ intact.
\end{itemize}
Dans le cas général, faire bondir un processus de $b_j$ à $b_k$ peur nécessiter plusieurs
actions, ou aucune si le processus est déjà présent dans l'état initial.
Ainsi, l'atteignabilité d'un processus peut nécessiter l'atteignabilité de plusieurs
autres processus, ou d'aucun autre processus.
Cette approche permet donc de résoudre un problème d'atteignabilité de façon récursive,
les cas terminaux étant ceux pour lesquels la résolution ne nécessite aucune action,
ou ne nécessite que des auto-actions.

Cependant, cette méthode d'analyse statique ne s'applique pas totalement aux nouveau formalismes
de Frappes de Processus présentés au \chapref{sem}.
En effet, ces formalismes augmentent l'expressivité des Frappes de Processus,
ce qui invalide, au moins en partie, la méthode développée.
Nous proposons dans ce chapitre une nouvelle approche par analyse statique qui permet
de répondre à des questions d'atteignabilité sur une classe particulière des Frappes de Processus
avec 2 classes de priorités, appelées \emph{Frappes de Processus canoniques}.
Néanmoins, nous montrons aussi que cette classe particulière est aussi expressive
que toutes les Frappes de Processus avec $k$ classes de priorités,
les Frappes de Processus avec arcs neutralisants,
et les Frappes de Processus avec actions plurielles,
et nous donnons les traductions appropriées pour pouvoir aussi traiter ces modèles.
La méthode que nous proposons possède une efficacité proche de la méthode d'origine,
le calcul du résultat étant de complexité polynomiale selon le nombre de sortes dans le modèle.
Cependant, bien que le modèle traité par l'analyse statique permette de représenter de façon simple
les modèles booléens avec portes logiques,
les traductions depuis d'autres formalismes peut s'avérer coûteuse. \todo{complexité ?}

Nous définissons dans un premier temps les Frappes de Processus canoniques
de façon formelle dans la \secref{phcanonique}.
Cette section aborde aussi la question de la traduction vers les Frappes de Processus canoniques
depuis les autres formalismes présentés au \chapref{sem}.
Elle donne enfin un certain nombre de résultats à leurs propos,
qui sont par la suite réutilisés dans la \secref{as},
qui présente formellement la méthode d'analyse statique développée.



\section{Approche informelle}
\TODO

\todo{Explications informelles}

\section{Les Frappes de Processus canoniques}
\seclabel{phcanonique}
  % Définition
  
\subsection{Définition}
\seclabel{phcanonique-def}

Nous donnons dans cette section la définition des Frappes de Processus canoniques,
qui sont un sous-ensemble des Frappes de Processus avec classes de priorités.
Un modèle de Frappes de Processus est dit \emph{canonique} s'il comporte 2 classes de priorités
et qu'il respecte un certain nombre de critères concernant l'utilisation des actions prioritaires.
Il faut notamment que celles-ci soient uniquement utilisées pour la mise à jour des sortes
coopératives, et que cette mise à jour soit correctement formée (pas d'action manquantes ou en trop).
Cette forme particulière des Frappes de Processus assure un certain nombre de bonnes propriétés
qui seront explicitées par la suite et qui permettent l'utilisation des méthodes d'analyse statique
développées à la section \todo{suivante}.
Dans la suite de cette section, nous travaillons sur des modèles de Frappes de Processus avec $k$
classes de priorités, pour $k \in \sN^*$.



Nous définissons pour commencer la \emph{réduction} d'un modèle donné de Frappes de Processus
comme le modèle équivalent dont on a retiré toutes les actions qui ne sont pas de priorité 1
(\defref{reduction}).
Cette définition permet dans la suite de considérer le comportement des actions les plus prioritaires
d'un modèle afin d'en contraindre la forme.

\begin{definition}[Réduction ($\reductionsymbol{\PH}$)]
\deflabel{reduction}
  Si $\PH = (\PHs; \PHl; \PHh^{\langle k \rangle})$ sont des Frappes de Processus avec $k$
  classes de priorités, on note $\reduction[\PH] = (\PHs; \PHl; \PHh^{(1)})$
  la \emph{réduction} de $\PH$.
  $\reduction$ est un modèle de Frappes de Processus standard
  (ou encore : avec 1 classe de priorité).
  Si $s \in \PHl$, on note de plus : $\reduction[\Sce](s)$ l'ensemble des scénarios
  depuis l'état $s$ dans les Frappes de Processus $\reduction$.
\end{definition}

\begin{example}
  La \figref{ph-livelock} donne un exemple de Frappes de Processus avec 2 classes de priorités avec :
  \begin{align*}
    \PHs &= \{ a, b, c, ab \} \enspace, \\
    \PHl_a &= \{ a_0, a_1 \} \enspace, & \PHl_b &= \{ b_0, b_1 \} \enspace, \\
    \PHl_c &= \{ c_0, c_1 \} \enspace, & \PHl_{ab} &= \{ ab_{00}, ab_{01}, ab_{10}, ab_{11} \}
    \enspace.
  \end{align*}
  On note que :
  $\{ \PHhit{ab_{11}}{c_0}{c_1}, \PHhit{a_1}{a_1}{a_0}, \PHhit{a_0}{b_0}{b_1} \} \subseteq \PHh^{(2)}$.

\begin{figure}[tb]
  \centering
  \scalebox{1.2}{
  \begin{tikzpicture}
    \TSort{(0,0)}{a}{2}{l}
    \TSort{(0,4)}{b}{2}{l}
    \TSort{(7,2.5)}{c}{2}{r}

    \TSetTick{ab}{0}{00}
    \TSetTick{ab}{1}{01}
    \TSetTick{ab}{2}{10}
    \TSetTick{ab}{3}{11}
    \TSort{(2,2.5)}{ab}{4}{t}

    \THit{a_0}{prio}{ab_3}{.south}{ab_1}
    \THit{a_0}{prio}{ab_2}{.south}{ab_0}
    \THit{a_1}{prio}{ab_1}{.south}{ab_3}
    \THit{a_1}{prio}{ab_0}{.south}{ab_2}

    \THit{b_0}{prio}{ab_3}{.north}{ab_2}
    \THit{b_0}{prio}{ab_1}{.north}{ab_0}
    \THit{b_1}{prio}{ab_2}{.north}{ab_3}
    \THit{b_1}{prio}{ab_0}{.north}{ab_1}
    
    \THit{a_1}{selfhit}{a_1}{.west}{a_0}
    \THit{b_1}{selfhit}{b_1}{.west}{b_0}
    \THit{a_0.north}{bend left}{b_0}{.west}{b_1}
    \THit{b_0.south}{bend right=60}{a_0}{.west}{a_1}

    \THit{ab_3}{}{c_0}{.west}{c_1}

  \path[bounce, bend right=55]
      \TBounce{ab_0}{}{ab_2}{.west}
      \TBounce{ab_1}{}{ab_3}{.west}
  ;
  \path[bounce, bend left=20]
      \TBounce{ab_3}{}{ab_1}{.south east}
      \TBounce{ab_2}{}{ab_0}{.south east}
  ;
    \path[bounce, bend right=20]
      \TBounce{ab_3}{}{ab_2}{.north east}
      \TBounce{ab_1}{}{ab_0}{.north east}
    ;
    \path[bounce, bend left=30]
      \TBounce{ab_2}{}{ab_3}{.west}
      \TBounce{ab_0}{}{ab_1}{.west}
    ;
    \path[bounce, bend right]
      \TBounce{a_1}{}{a_0}{.north west}
      \TBounce{b_1}{}{b_0}{.north west}
    ;
    \path[bounce, bend left]
      \TBounce{a_0}{}{a_1}{.south west}
      \TBounce{b_0}{}{b_1}{.south west}
    ;
    \path[bounce, bend left]
      \TBounce{c_0}{}{c_1}{.south west}
    ;
    \TState{a_1, b_0, ab_2, c_0}
  \end{tikzpicture}
  }
  \caption{
  \figlabel{ph-livelock}
    Un exemple de Frappes de Processus canonique.
    Les actions de priorité 1 sont dessinées en traits doubles tandis que les actions
    de priorité 2 sont représentées avec des traits simples.
    Les processus grisés présentent un exemple d'état de départ :
    $\PHstate{a_1, b_0, c_0, ab_{10}}$.
  }
\end{figure}

\end{example}



\subsection{Restrictions}
\seclabel{restrictions}

Les Frappes de Processus canoniques consistent en une classe particulière de Frappes de Processus
avec 2 classes de priorités, et cette section a pour objectif d'en donner la définition.
Les restrictions permettant de définit des Frappes de Processus canoniques portent sur
les actions les plus prioritaires mettant à jour les sortes coopératives.
Cependant, afin de poser un cadre plus général et permettre une réutilisation de ces définitions
\stodo{ref},
nous considérerons le cas de Frappes de Processus avec $k$ classes de priorités dans les définitions
et critères qui suivent.

Dans la suite, nous appelons \emph{actions primaires} les actions de l'ensemble $\PHh^{(1)}$,
c'est-à-dire les actions les plus prioritaires.
Ces actions auront pour unique but de mettre à jour des sortes coopératives, permettant
à celles-ci de modéliser des portes logiques sans décalage temporel \stodo{ref}.
De fait, on peut dans certains cas considérer ces actions comme «~non biologiques~»,
ou «~instantanées~»,
car elles sont présentes principalement pour des raisons de modélisation.
À l'inverse, les actions de l'ensemble $\PHh \setminus \PHh^{(1)}$, qui ne sont pas de priorité 1,
seront appelées \emph{actions secondaires}.
Comme elles permettent de représenter les différentes réactions et régulations intervenant au sein du
système, elles peuvent par conséquent être qualifiées d'actions «~biologiques~» ou «~lentes~».

La définition de la forme canonique des Frappes de Processus début avec la
\crref{tf} qui stipule que la dynamique ne doit pas contenir de séquence infinie
d'actions primaires.
En effet, pour que ces actions effectuent une mise à jour des sortes coopératives sans perturber
le reste du système, il est nécessaire qu'elles ne puissent pas préempter les actions secondaires
indéfiniment.
Sans cela, le modèle serait victime du paradoxe de Zénon,
où une suite infinie d'actions peut être jouée en un temps nul et ainsi
bloquer l'évolution du système.
Il est donc nécessaire de postuler que tous les scénarios d'actions primaires sont finis,
ce qui a par ailleurs pour conséquence que $\reductionsce$ est un ensemble fini.

\begin{condition}[Terminaison finie]
\crlabel{tf}
  La dynamique de $\reduction$ ne contient aucun cycle :
  $\exists N \in \sN, \forall s \in \PHl, \forall \delta \in \reductionsce(s),
    |\delta| \leq N$.
\end{condition}

Étant donnée une sorte $a \in \PHs$ et un état $s \in \PHl$,
on note $\pfp_s(a)$ (\defref{pfp}) l'ensemble des processus de la sorte $a$ qui peuvent être présents
après avoir joué tous les scénarios d'actions de priorité 1 depuis l'état $s$.
Cet ensemble est toujours défini du fait de la \crref{tf}.

\todo{Redéfinir $\pfp$ avec $\compin$ de façon à simplifier et rendre locale la définition ?}

\begin{definition}[$\pfp : \PHl \times \PHs \rightarrow \powerset(\PHproc)$]
\deflabel{pfp}
  Pour tout état $s \in \PHl$ et pour toute sorte $a \in \PHs$,
  \begin{align*}
    \pfp_s(a) = \{ \get{(s\play\delta)}{a} \in \PHl_a &\mid \delta \in \reductionsce(s)
          \wedge\nexists h\in\PHh^{(1)}, (\delta; h) \in \reductionsce(s) \}
  \end{align*}
\end{definition}

Nous définissons dans la suite la notion de \emph{composant bien formé} (\defref{component})
et de \emph{sorte coopérative bien formée} (\defref{cs}).
Un composant bien formé n'est frappé que par des actions secondaires,
car il ne subit que des influences «~biologiques~».
Une sorte coopérative bien formée n'est frappée que par des actions primaires «~non biologiques~»
qui convergent toujours vers le même processus en fonction de l'état des sortes qui l'influencent
(\defref{comp}), afin qu'elle représente chaque configuration de ces sortes par un unique processus.
La \crref{part} stipule alors que l'ensemble des sortes des Frappes de Processus canoniques
($\PHs$) est une partition entre un ensemble de composants bien formés (noté $\components$)
et un ensemble de sortes coopératives bien formées (noté $\cs$).

\begin{definition}[Composant bien formé ($\components$)]
\deflabel{component}
  Une sorte $a \in \PHs$ est un \emph{composant bien formé} si et seulement si :
    \[\forall h \in \PHh, \PHsort(\target{h}) = a \Rightarrow \prio(h) \geq 2 \enspace.\]
\end{definition}

\begin{definition}[Influence ($\compin : \PHs \to \powerset(\components)$)]
\deflabel{comp}
  Pour toute sorte $a \in \PHs$, on définit : $\compin(a) \DEF \conn(a) \cap \components$ où
  $\conn(a) \subset \PHs$ est le plus petit ensemble de sortes satisfaisant :
  \begin{align*}
    a \in \conn(a) & \\
    \forall h \in \PHh^{(1)},
      \sort{\target{h}} \in \conn(a) & \Rightarrow \sort{\hitter{h}} \in \conn(a)
  \end{align*}
\end{definition}

\begin{definition}[Sorte coopérative bien formée ($\cs$)]
\deflabel{cs}
  Une sorte $a \in \PHs$ est une \emph{sorte coopérative bien formée} si et seulement si :
  \begin{enumerate}[(i)]
    \item $\forall h \in \PHh, \sort{\target{h}} = a \Rightarrow \prio(h) = 1$,
    \item \label{csai} $\forall s \in \PHl, \card{\pfp_s(a)} = 1$,
    \item \label{css} $\forall a_i \in \PHl_a, \exists s \in \PHl, a_i \in \pfp_s(a)$,
    \item $\forall \mysigma \in \PHsubl[\PHl]_{\compin(a)}, \forall s, s' \in \PHl,
        (\mysigma \subseteq s \wedge \mysigma \subseteq s') \Rightarrow \pfp_s(a) = \pfp_{s'}(a)$.
  \end{enumerate}
\end{definition}

\begin{condition}[Partition des composants et des sortes coopératives]
\crlabel{part}
  \[\PHs = \components \cup \cs \wedge \components \cap \cs = \emptyset\]
\end{condition}

Il est à noter qu'une sorte $a$ qui n'est frappée par aucune action,
c'est-à-dire telle que $\forall h \in \PHh, \sort{\target{h}} \neq a$,
est en accord à la fois avec la définition de composant bien formé
et avec celle de sorte coopérative bien formée.
Cette sorte peut être arbitrairement et indifféremment affectée à $\components$ ou à $\cs$.
Un tel cas peut se produire pour des composants ayant un rôle d'«~entrée~»,
c'est-à-dire dont le niveau d'expression est uniquement déterminé par l'état initial,
ou par certaines constructions particulières de sortes coopératives créées par la traduction
\stodo{ref}.

Pour toute sorte $a \in \PHs$ et tout état $s \in \PHl$,
étant donné le point (\ref{csai}) de la \defref{cs}, on a toujours :
$\exists a_i \in \PHl_a, \pfp_s(a) = \{ a_i \}$.
On notera donc dans la suite : $\pfp_s(a) = a_i$.
De plus, du fait du point (\ref{css}) de la \defref{cs}, on introduit la notation
$\csState(a_i)$ permettant de caractériser l'ensemble des sous-états représentés par le
processus $a_i$ de toute sorte coopérative bien formée $a$ (\defref{csState}).

\todo{$\csState : \PHproc \rightarrow \powerset(\components)$ ?}
\begin{definition}[$\csState : \PHproc \rightarrow \powerset(\PHproc)$]
\label{def:csState}
  Pour tout $a \in \cs$ et $a_i \in \PHl_a$, 
    \[\csState(a_i) \DEF \{ \toset{ps} \mid ps \in \PHsubl[\PHl]_{\compin(a)} \wedge
      \exists s \in \PHl, (ps \subseteq s \wedge \pfp_s(a) = a_i) \}\]
\end{definition}

Dans la suite, on écrira simplement «~composant~» (\resp «~sorte coopérative~»)
en lieu et place de «~composant bien formé~» (\resp «~sorte coopérative bien formée~»).

\begin{example}
  Les Frappes de Processus de la \figref{ph-livelock} comprennent trois composants ($a$, $b$ et $c$)
  et une sorte coopérative ($ab$).
  La sorte coopérative $ab$ modélise la coopération entre $a$ et $b$ car $\compin(ab) = \{ a, b \}$.
\end{example}

Enfin, la \defref{phcanonique} définit la notion de Frappes de Processus canoniques.
La forme particulière de ces Frappes de Processus permettra de développer une méthode efficace
d'analyse statique pour le problème d'atteignabilité défini à la \secref{sa},
grâce notamment aux résultats développés dans la section suivante.

\begin{definition}[Frappes de Processus canoniques]
\deflabel{phcanonique}
  Les \emph{Frappes de Processus canoniques}
  sont des Frappes de Processus avec 2 classes de priorités
  respectant les \allcr.
\end{definition}



\subsection{Conséquences des restrictions}

Dans cette section, nous donnons plusieurs résultats généraux qui peuvent être dérivés des
restrictions de la \secref{restrictions}.
Nous considérons donc dans la suite des Frappes de Processus avec $k$ classes de priorités
$\PH = (\PHs; \PHl; \PHa^{\langle k \rangle})$, avec $k \in \sN^*$,
respectant les \allcr.
\todo{Avec 2 classes de priorités ?}
Ces résultats permettront de construire la méthode d'analyse statique de la \stodo{section suivante}.

Pour tout état $s \in \PHl$, on appelle $\update(s)$ l'état dans lequel tous les composants
ont le même processus actif que dans $s$
mais où sortes coopératives ont été mises à jour en fonction (\defref{update}).
Cet état est unique du fait des propriétés de $\pfp$ données dans la section précédente.
Le \lemref{update} stipule ensuite que depuis tout état $s$, il existe un scénario d'actions primaires
mettant à jour toutes les sortes coopératives de façon à arriver dans $\update(s)$.
%ghtrjy <= Raphaël was here!

\begin{definition}[$\update : \PHl \rightarrow \PHl$]
\deflabel{update}
  Pour tout $s \in \PHl$, on définit :
  \begin{align*}
    \update(s) = s \Cap \{ \pfp_{s}(a) \mid a \in \cs \} \enspace.
  \end{align*}
\end{definition}

\begin{lemma}
\lemlabel{update}
  $\forall s \in \PHl, \exists \delta \in \reductionsce(s), s \PHplay \delta = \update(s)$
\end{lemma}

\begin{demo}[\Lemref{update}]
  Soit $s \in \PHl$ un état.
  Soit $a \in \cs$ une sorte coopérative telle que $\PHget{s}{a} \neq \pfp_s(a)$.
  Étant donnée la définition de $\pfp_s(a)$, il existe un scénario $\delta$ mettant à jour $a$
  dans $s$ tel que :
  $\forall \delta' \in \reductionsce(s \PHplay \delta),
    \PHget{(s \PHplay \delta \PHplay \delta')}{a} = \pfp_s(a)$.
  \todo{À revoir}
\end{demo}

Le \lemref{hcompcomp} stipule que pour un état donné $s \in \PHl$, et pour toute action secondaire
$h = \PHhit{a_i}{b_j}{b_k} \in \PHh$ où $a$ et $b$ sont des composants,
si $\PHget{s}{a} = a_i$ et $\PHget{s}{b} = b_j$, alors
$h$ peut toujours être jouée après une série d'actions primaires,
et ces actions n'empêchent pas son jeu.

De façon complémentaire, le \lemref{hcscomp} énonce le même résultat si $a$ est une sorte coopérative,
sous la condition que $a$ soit déjà à jour dans $s$.

\begin{lemma}
\lemlabel{hcompcomp}
  $\forall s \in \PHl, \forall a,b \in \components, \forall h = \PHhit{a_i}{b_j}{b_k} \in \PHh,$\\
  $(\PHget{s}{a} = a_i \wedge \PHget{s}{b} = b_j) \Rightarrow
    (\exists \delta \in \reductionsce(s),
%    (s \PHplay \delta) \PHtrans (s \PHplay \delta \PHplay h))$
    \Feval{\Fopphp{h}}{s \PHplay \delta})$
\end{lemma}

\begin{demo}[\Lemref{hcompcomp}]
  Soient $s \in \PHl$ un état, $a,b \in \components$ deux composants et
  $h = \PHhit{a_i}{b_j}{b_k} \in \PHh$ une action.
  D'après le \lemref{update}, il existe un scénario $\delta$ tel que :
  $(s \PHplay \delta) = \update(s)$.
  Étant donné que $a$ et $b$ sont des composants,
  $a_i \in (s \PHplay \delta)$ et $b_j \in (s \PHplay \delta)$.
  De plus, par définition de $\update(s)$, aucune action primaire n'est jouable
  dans $(s \PHplay \delta)$.
  $h$ est donc jouable dans $(s \PHplay \delta)$.
\end{demo}

\begin{lemma}
\lemlabel{hcscomp}
  $\forall s \in \PHl, \forall a \in \cs, \forall b \in \components,
    \forall h = \PHhit{a_i}{b_j}{b_k} \in \PHh,$\\
  $(\PHget{s}{a} = a_i \wedge \PHget{s}{b} = b_j \wedge \pfp_s(a) = a_i) \Rightarrow
    (\exists \delta \in \reductionsce(s),
%    (s \PHplay \delta) \PHtrans (s \PHplay \delta \PHplay h))$
    \Feval{\Fopphp{h}}{s \PHplay \delta})$
\end{lemma}

\begin{demo}[\Lemref{hcscomp}]
  Similaire à la preuve du \lemref{hcompcomp} ;
  étant donné que $a_i \in \pfp_s(a)$, on a : $a_i \in (s \PHplay \delta)$.
\end{demo}

  %Équivalence avec priorités quelconques
  \section{Équivalence avec les autres formalismes de Frappes de Processus}
\seclabel{phcanonique-equiv}

Cette section vise à tracer des liens entre les différentes sémantiques des Frappes de Processus
présentées au \chapref{sem}.
Son principal apport est l'\emph{aplatissement} des Frappes de Processus avec $k$
classes de priorités, donné à la \secref{aplatissement},
et qui permet de les traduire en Frappes de Processus canoniques.
Cette traduction permet donc, à partir d'un modèle de Frappes de Processus comprenant un nombre
arbitraire de priorités, d'obtenir un modèle canonique respectant la même dynamique.
Par la suite, le cas des Frappes de Processus avec arcs neutralisants est aussi traité,
de façon analogue, dans la \secref{phan-aplatissement}.
Enfin, la \secref{phm-aplatissement} réutilise ces résultats pour proposer aussi une traduction
depuis les Frappes de Processus avec actions plurielles.

Ces différentes traductions et propriétés de bisimulation (faible) qu'elles proposent
permettent d'établir que les différentes sémantiques de Frappes de Processus sont
aussi expressives que les Frappes de Processus canoniques.
Notamment, cela nous permet d'assurer que les Frappes de Processus avec plusieurs
classes de priorités sont équivalentes ;
autrement dit, considérer plus de deux classes de priorités n'augmente pas l'expressivité
---~bien que cela puisse faciliter la modélisation.



\subsection{Aplatissement des Frappes de Processus avec $k$ classes de priorités}
\seclabel{aplatissement}

Le but de cette section est de montrer qu'un modèle de Frappes de Processus avec $k$ classes
de priorités peut être \emph{aplati}, c'est-à-dire traduit en un autre modèle ne comportant
que 2 classes de priorités.
Ce dernier modèle est assuré de posséder la même dynamique,
car il lui est faiblement bisimilaire,
comme établi par le \vthmref{bisimulaplatissement}.
De plus, les actions de priorité 1 (les plus prioritaires) ne sont utilisées que pour mettre
à jour les sortes coopératives ;
il s'agit en fait d'un modèle de Frappes de Processus canoniques telles que définies
à la \secref{phcanonique-def}.
La forme particulière de ces modèles permet d'y appliquer les méthodes d'analyse statique
développées à la \secref{as}.
Cela nous permet de plus de montrer que les Frappes de Processus avec $k$ classes de priorités
sont aussi expressives entre elles pour tout $k \in \sN^*$,
car elles sont toutes aussi expressives que les Frappes de Processus canoniques.

Étant donné que les propriétés de jouabilité n'utilisent que des opérateurs de logique booléenne
standards, il est possible de calculer la forme normale disjonctive (FND) de toute propriété de
jouabilité. Pour toute action $h \in \PHh$, cette FND est de la forme :
\[\Fopphp{h} \equiv \bigvee_{i \in \segm{1}{\n}}
  \left( \bigwedge_{j \in \segm{1}{\m}} p_{i,j} \right)\]
où $\n \in \sN$ et $\forall i \in \segm{1}{\n}, \m \in \sN^*$.
Si $\n = 0$, alors $\Fopphp{h} \equiv \bot$, ce qui signifie que l'action $h$ ne peut jamais
être jouée car elle est préemptée dans dans tous les états où son frappeur et sa cible
sont présents.
Une telle action peut être retirée du modèle sans en changer le comportement.
En revanche, si $\n > 0$, alors $\Fopphp{h} \not\equiv \bot$ ;
dans ce cas, $\Fopphp{h}$ est une disjonction de $\n$ conjonctions d'atomes,
et peut donc être vue comme une disjonction de $\n$ propriétés de jouabilité plus petites.
Ces $\n$ conjonctions d'atomes peuvent être traduites en autant de sortes
coopératives priorisées, afin d'obtenir une dynamique équivalente avec un nombre réduit
de classes de priorités utilisées.
Dans ce second cas, on note, pour tout $i \in \segm{1}{\n}$ :
$\PHdep{i}{h} = \{ \PHsort(p_{i,j}) \mid j \in \segm{1}{\m} \}$.

L'opérateur d'aplatissement $\Fopsymbol_\Fopaplatsubsymbol$ donné à la \defref{fopaplat}
permet de caractériser la jouabilité d'une action sans prendre en compte les actions primaires ;
en d'autres termes, une action $h$ est jouable dans un état $s$ si et seulement si
$\Feval{\Fopaplat{h}}{s}$ et aucune action de priorité 1 n'est jouable.
Le \lemref{ppplaysubset} permet alors de caractériser la jouabilité d'une action dans un état
à l'aide d'un sous-état correspondant à l'une des conjonctions de sa propriété
d'aplatissement une fois traduite en FND.
Enfin, la \defref{aplatissement} donne la construction de l'\emph{aplatissement} de $\PH$ :
pour chaque action $h \in \PHh$, plusieurs sortes coopératives $f^{h,i}$
permettent de refléter chaque conjonction de $\Fopphp{h}$,
c'est-à-dire une pour chaque indice $i \in \segm{1}{\n}$.
Cette construction permet d'obtenir la même dynamique que pour $\PH$ en reproduisant
les préemptions possibles par d'autres actions plus prioritaires,
comme établi par le \thmref{bisimulaplatissement}.

Il est à noter que la \defref{aplatissement} ne s'applique qu'à des Frappes de Processus
pseudo-canoniques. Cependant, des Frappes de Processus avec $k$ classes de priorités quelconques
sont \textit{a fortiori} des Frappes de Processus pseudo-canoniques à condition d'ajouter
une classe vide d'actions de priorité supérieure.
En d'autres termes, il est possible d'ignorer le cas particulier des actions primaires
(mettant à jour les sortes coopératives) si les Frappes de Processus ne sont pas
pseudo-canoniques.

%\todo{À supprimer ? On peut l'intégrer dans la démo du \thmref{bisimulaplatissement}.}

\begin{definition}[Opérateur d'aplatissement
  ($\Fopsymbol_\Fopaplatsubsymbol : \PHh \rightarrow \F$)]
\deflabel{fopaplat}
  L'opérateur d'aplatissement des Frappes de Processus avec $k$ classes de priorités
  est défini par :
  \[\forall h \in \PHh, \Fopaplat{h} \equiv \hitter{h} \wedge \target{h} \wedge
    \left( \bigwedge_{\substack{g \in \PHh^{(n)}\\1 < n < \prio(h)}}
    \neg \left( \hitter{g} \wedge \target{g} \right) \right)\]
\end{definition}

\begin{lemma}
\lemlabel{ppplaysubset}
  Soient $h \in \PHh \setminus \PHh^{(1)}$ et $s \in \PHl$.
  \[\Feval{\Fopphp{h}}{s} \Leftrightarrow
    \big(\exists \mysigma \subseteq s, \Feval{\Fopaplat{h}}{\mysigma}\big)
    \wedge \big(\forall g \in \PHh^{(1)}, \Feval{\neg \Fopphp{g}}{s}\big)
    \enspace.\]
\end{lemma}
%
\begin{proof}
  ($\Rightarrow$)
    Supposons $\Feval{\Fopphp{h}}{s}$.
    Il n'existe donc aucune action primaire jouable dans $s$.
    Par ailleurs, $\Fopphp{h} \Rightarrow \Fopaplat{h}$ donc
    $\Fopaplat{h} \not\equiv \bot$ et, par propriété d'une FND,
    au moins l'une des conjonctions de $\Fopaplat{h}$ est vraie dans $s$.
    On suppose que la $i$\textsuperscript{e} conjonction est vraie dans $s$,
    avec $i \in \segm{1}{\n}$;
    on a alors : $\forall j \in \segm{1}{\m}, p_{i,j} \in s$.
    Soit $\mysigma \in \PHsubl_{\PHdep{i}{h}}$
    avec $\forall b \in \PHdep{i}{h}, \PHget{\mysigma}{b} = \PHget{s}{b}$.
    On a alors immédiatement : $\mysigma \subseteq s$,
    et, par construction de $\PHdep{i}{h}$, $\Feval{\Fopaplat{h}}{\mysigma}$.
  
  ($\Leftarrow$)
    Supposons qu'il existe $\mysigma \subseteq s$ tel que $\Feval{\Fopaplat{h}}{\mysigma}$,
    et qu'aucune action primaire n'est jouable dans $s$.
    On a alors immédiatement $\Feval{\Fopaplat{h}}{s}$
    car ajouter des processus au sous-état
    d'évaluation ne peut pas rendre la propriété fausse.
    De plus, comme aucune action primaire n'est jouable dans $s$, alors
    $\Feval{\left( \bigwedge_{g \in \PHh^{(1)}} \neg (\hitter{g} \wedge \target{g}) \right)}{s}$,
    donc $\Feval{\Fopphp{h}}{s}$.
\end{proof}

\begin{definition}[Aplatissement ($\PHflat$)]
\deflabel{aplatissement}
  Si $k \in \sNN$ et $\PH = (\PHs; \PHl; \PHa^{\angles{k}})$
  sont des Frappes de Processus pseudo-canoniques avec $k$ classes de priorités,
  on note $\PHflat(\PH) = \oPH = (\ov{\PHs}; \ov{\PHl}; (\ov{\PHa}^{(1)}; \ov{\PHa}^{(2)}))$
  l'\emph{aplatissement} de $\PH$, où :
  \begin{itemize}
    \item $\ov{\PHs} = \PHs \cup \PHs_f$
      où $\PHs_f = \{ f^{h,i} \mid h \in \PHh \wedge \n \geq 1 \wedge i \in \segm{1}{\n} \}$;
    \item $\ov{\PHl} = \left( \bigtimes{a \in \PHs} \PHl_{a} \right) \times
      \left(\bigtimes{f^{h,i} \in \PHs_f} \PHl_{f^{h,i}} \right)$, où
      $\forall f^{h,i} \in \PHs_f, \PHl_{f^{h,i}} =
        \{ f^{h,i}_\mysigma \mid \mysigma \in \PHsubl_{\PHdep{i}{h}} \}$;
    \item $\ov{\PHh}^{(1)} = \PHh^{(1)} \cup
      \{ \PHhit{a_k}{f^{h,i}_\mysigma}{f^{h,i}_{\mysigma'}} \mid
      h \in \PHh \wedge f^{h,i} \in \PHs_f \wedge
      a \in \PHdep{i}{h} \wedge a_k \in \PHl_a \wedge
      f^{h,i}_\mysigma , f^{h,i}_{\mysigma'} \in \PHl_{f^{h,i}} \wedge
      \PHget{\mysigma}{a} \neq a_k \wedge \mysigma' = \mysigma \Cap \{ a_k \} \}$;
    \item $\ov{\PHh}^{(2)}=\{ \PHhit{f^{h,i}_\mysigma}{\target{h}}{\bounce{h}} \mid
      h \in \PHh \setminus \PHh^{(1)} \wedge f^{h,i} \in \PHs_f \wedge
      f^{h,i}_\mysigma \in \PHl_{f^{h,i}} \wedge \Feval{\Fopaplat{h}}{\mysigma} \}$.
  \end{itemize}
  De plus, étant donné un état $\os \in \ov{\PHl}$,
  on note $\unflats{\os} = s$ l'état correspondant dans $\PHl$ :
  $\forall a \in \PHs, \PHget{s}{a} = \PHget{\os}{a}$.
  À l'inverse, étant donné un état $s \in \PHl$,
  $\flats{s} = \os$ est l'état correspondant dans $\ov{\PHl}$ :
  $\forall a \in \PHs, \PHget{\os}{a} = \PHget{s}{a}$
  et $\forall f^{h,i} \in \PHs_f, \PHget{\os}{f^{h,i}} = f^{h,i}_\mysigma$
  avec $f^{h,i}_\mysigma \in \PHl_{f^{h,i}}$
  et $\forall b \in \PHdep{i}{h}, \PHget{\mysigma}{b} = \PHget{s}{b}$.
\end{definition}

Nous notons que l'aplatissement $\PHflat(\PH)$ de toutes Frappes de Processus avec $k$
classes de priorités $\PH = (\PHs, \PHl, \PHh^{\angles{k}})$
sont des Frappes de Processus canoniques.
En effet, une partie des sortes coopératives générées lors de cette traduction proviennent
des Frappes de Processus d'origine, qui sont pseudo-canoniques et sont déjà contraintes de la
même manière.
L'autre partie constitue les sortes coopératives de l'ensemble $\PHs_f$ et leur définition
respecte les \allcr.

\begin{theorem}[$\PH \approx \PHflat(\PH)$]
\thmlabel{bisimulaplatissement}
  Soient $\PH = (\PHs; \PHl; \PHa^{\angles{k}})$ des Frappes de Processus avec $k$
  classes de priorités,
  et $\oPH = \PHflat(\PH) = (\ov{\PHs}; \ov{\PHl}; \ov{\PHa}^{\angles{2}})$ leur aplatissement.
  \begin{enumerate}
    \item \label{php2ph} $\forall s, s' \in \PHl$,
      $s \trans{\PH} s' \Longrightarrow \flats{s} \mtrans{\oPH} \flats{s'}$,
      où $\mtrans{\oPH}$ est une séquence finie de transitions $\trans{\oPH}$.
    \item \label{ph2php} $\forall \os, \os' \in \ov{\PHl}$,
      $\os \trans{\oPH} \os' \Longrightarrow \unflats{\os} = \unflats{\os'} \vee
      \unflats{\os} \trans{\PH} \unflats{\os'} \enspace.$
  \end{enumerate}
\end{theorem}

\begin{proof}
  (\ref{php2ph}) Soient $s, s' \in \PHl$ tels que $s \trans{\PH} s'$.
    Posons $\os = \flats{s}$.
    D'après la dynamique des Frappes de Processus (\defref{play}),
    si $s \trans{\PH} s'$, alors il existe une action $h \in \PHh$ jouable dans $s$,
    telle que $s' = s \PHplay h$.
    On a alors : $\Feval{\Fopphp{h}}{s}$.
    Par construction de $\oPH$ (\defref{aplatissement}) :
    \begin{itemize}
      \item Si $h \in \PHh^{(1)}$, alors il existe $g = h \in \ov{\PHh}^{(1)}$,
        et on a : $\hitter{g} \in \os$ et $\target{g} \in \os$.
      \item Si, au contraire, $h \in \PHh \setminus \PHh^{(1)}$, alors il existe
        $g = \PHhit{f^{h,i}_\mysigma}{\target{h}}{\bounce{h}} \in \ov{\PHh}^{(2)}$.
        De plus, d'après le \lemref{ppplaysubset}, il existe $\mysigma \subseteq s$
        tel que $\Feval{\Fopaplat{h}}{\mysigma}$
        et, par construction de $\os$ (\defref{aplatissement}),
        $\PHget{\os}{f^{h,i}} = f^{h,i}_\mysigma$.
    \end{itemize}
    Dans les deux cas, $g$ est jouable dans $\os$.
    Par la suite, dans l'état $\os \PHplay g$, les seules actions jouables sont celles dans
    $\PHh^{(1)}$ qui mettent à jour les sortes coopératives dans lesquelles
    $\bounce{h} = \bounce{g}$ est impliqué, directement ou indirectement,
    permettant donc d'accéder à l'état $\flats{s'}$ en un nombre fini d'actions.
    Ainsi, $\flats{s} \mtrans{\oPH} \flats{s'}$.
  
  (\ref{ph2php}) soient $\os, \os' \in \ov{\PHl}$ tels que $\os \trans{\oPH} \os'$.
    Posons $s = \unflats{\os}$.
    D'après la dynamique des Frappes de Processus (\defref{play}),
    si $\os \trans{\PH} \os'$, alors il existe une action $g \in \ov{\PHh}$ jouable dans $\os$,
    telle que $\os' = \os \PHplay h$.
    \begin{itemize}
      \item Si $g \in \ov{\PHh}^{(1)} \setminus \PHh^{(1)}$,
        alors $\unflats{\os} = \unflats{\os'}$ car seul le processus actif d'une sorte
        coopérative dans $\PHs_f$ a évolué. % qui n'est pas dans $\PHs$ a évolué.
      \item Sinon, si il existe $h = g \in \PHh^{(1)}$,
        alors $h$ est jouable dans $s$ et : $\unflats{\os} \trans{\PH} \unflats{\os'}$
        avec $\unflats{\os'} = s \play h$.
      \item Autrement, si $g \in \ov{\PHh}^{(2)}$,
        on note : $g = \PHhit{f^{h,i}_\mysigma}{b_j}{b_k}$.
        Par construction de l'aplatissement (\defref{aplatissement}), il existe
        $h \in \PHh$ tel que $\Feval{\Fopaplat{h}}{\mysigma}$.
        Comme $g$ est jouable, cela signifie qu'aucune action de $\PHh^{(1)}$ n'est jouable,
        et notamment que la sorte coopérative $f^{h,i}$ est déjà mise à jour,
        ce qui a pour conséquence que : $\mysigma \subseteq s$.
        Ainsi, d'après \lemref{ppplaysubset}, $h$ est jouable dans $s$,
        car $\Feval{\Fopphp{h}}{s}$, et $\unflats{\os} \trans{\PH} \unflats{\os'}$.
    \end{itemize}
\end{proof}

Nous notons que l'aplatissement donné à la \defref{aplatissement}
est, pour chaque action, exponentiel dans le nombre d'actions non primaires de
priorité supérieure.
En effet, il est nécessaire pour chaque action de convertir la propriété de jouabilité
donnée par la \defref{fopphp} en une FND.
Or la majorité des cas pratiques pour cette conversion sont proches du pire cas,
qui est de complexité exponentielle en fonction du nombre d'atomes dans la propriété.

Pour finir, il est intéressant de noter que l'aplatissement de la \defref{aplatissement}
n'est pas optimal en nombre d'actions et de processus créés dans le modèle final.
En effet, il est possible de simplifier le modèle $\PHflat(\PH)$ de différentes manières,
qui n'ont pas été prises en compte ici pour ne pas alourdir la définition.
Nous donnons dans la suite quelques pistes pour obtenir des modèles plus simples
mais au comportement équivalent, mais n'en donnons pas les preuves.

\subsubsection*{Simplification des propriétés d'aplatissement}

La propriété d'aplatissement $\Fopaplat{h}$ d'une action $h = \PHhit{a_i}{b_j}{b_k}$ peut être
simplifiée à l'aide des propriétés suivantes, permettant d'éviter la création
de certains éléments inutiles, comme des actions qui ne sont jamais jouables
ou des coopérations qui sont toujours vraies :
\begin{itemize}
  \item il n'est pas nécessaire de faire apparaître la cible d'une action dans
    sa propriété d'aplatissement car sa présence sera vérifiée par ailleurs,
    au moment du tir effectif de l'action,
    d'où : $b_j \equiv \top$ ;
  \item Tout processus $b_l \neq b_j$ de la même sorte que la cible empêche toujours la jouabilité
    de l'action, donc : $b_l \equiv \bot$ ;
  \item si $c_p, c_q$ sont des processus différents ($c_p \neq c_q$) de la même sorte $c$,
    alors $c_p \wedge c_q \equiv \bot$.
\end{itemize}

\subsubsection*{Suppression des sortes coopératives superflues}

Il existe deux cas pour lesquels il est possible de supprimer la sorte coopérative $f^{h,i}$
créée pour une action $h$ dans le modèle aplati :
\begin{itemize}
  \item si $\Fopaplat{h} \equiv \top$, alors l'action $h$ peut être traduite comme étant
    une auto-action (car elle est toujours jouable dès lors que la cible est présente) ;
  \item Si la $i$\textsuperscript{e} conjonction de $\Fopaplat{h}$ consiste
    en un unique élément $p$, alors cette conjonction peut être traduite par une action simple
    de la forme : $\PHhit{p}{\target{h}}{\bounce{h}}$ sans avoir recours à une sorte coopérative
    (étant donné qu'en dehors de la cible, un seul processus, $p$, est requis).
\end{itemize}

\begin{example}
  Les Frappes de Processus avec 3 classes de priorités de la \figref{metazoan-php-bis}
  représentent un modèle de segmentation métazoaire inspiré de la \vfigref{metazoan-php}
  donc les deux actions suivantes ont été retirées :
    \[\PHfrappe{f_1}{f_1}{f_0} \quad \text{ et } \quad \PHfrappe{f_0}{c_1}{c_0}\]
  Cela permet de s'intéresser uniquement à l'état stationnaire du système,
  qui consiste en l'oscillation alternative des sortes $a$ et $c$.
  Ce modèle peut être aplati par la méthode donnée à la \vdefref{aplatissement}
  et les simplifications détaillées ci-dessus.
  Le résultat est le modèle de Frappes de Processus canoniques donné à la
  \figref{metazoan-phcanonique}.
  Le modèle complet (de la \figref{metazoan-php}) peut naturellement aussi être
  aplati à l'aide de la même méthode, mais le résultat est plus complexe du fait des
  deux actions «~peu urgentes~» supplémentaires.
  
  \begin{figure}[ht]
  \begin{center}
  \begin{tikzpicture}
    \TSort{(0,4)}{c}{2}{l}
    \TSort{(1,0)}{f}{2}{l}
    \TSort{(7,4)}{a}{2}{r}
    
    \TSetTick{fc}{0}{00}
    \TSetTick{fc}{1}{01}
    \TSetTick{fc}{2}{10}
    \TSetTick{fc}{3}{11}
    \TSort{(4,1)}{fc}{4}{r}
    
    \TAction{c_1}{a_1.west}{a_0.north west}{}{right}
    \TAction{f_1}{c_0.west}{c_1.south west}{bend left=30, in=90}{left}
    \TAction{c_1}{c_1.west}{c_0.north west}{selfhit}{right}
%     \TAction{f_1.north east}{f_1.south east}{f_0.north east}%
%       {selfhit, min distance=30, bend left, out=150, in=90}{left}
%     \TAction{f_0.east}{c_1.south east}{c_0.north east}{bend right=60, in=-140}{left}

    \TAction{fc_2}{a_0.west}{a_1.south west}{}{left}
    \path (1.8, 0.5) edge[coopupdate] (3.2, 2);
    \path (0.8, 4.5) edge[coopupdate] (3.2, 3);
    
    \node[labelprio1] at (2.55,3.85) {$1$}; % c => fc
    \node[labelprio1] at (2.75,1) {$1$};    % f => fc
    \node[labelprio2] at (5.5,3.85) {$2$};  % fc_10 -> a_0 / 1
    \node[labelprio2] at (3.5,5.3) {$2$};   % c_1 -> a_1 / 0
    \node[labelprio3] at (0,2.5) {$3$};     % f_1 -> c_0 / 1
    \node[labelprio3] at (0.8,5.8) {$3$};   % c_1 -> c_1 / 0
  \end{tikzpicture}
  \end{center}
  \caption{\figlabel{metazoan-php-bis}%
    Exemple de Frappes de Processus avec 3 classes de priorités,
    inspiré du modèle de la \figref{metazoan-ph},
    dont deux actions ont été retirées pour supprimer
    l'interruption de l'avancée du front d'onde $f$.
    Les étiquettes numérotées (de 1 à 3) placées contre les flèches représentant les actions
    symbolisent leur appartenance à une classe de priorités donnée.
  }
  \end{figure}

  \begin{figure}[ht]
  \begin{center}
  \begin{tikzpicture}
    %\path[use as bounding box] (-5.75,0) rectangle (5.75,5.5);
    \TSort{(-6,5)}{c}{2}{l}
    \TSort{(0,1)}{f}{2}{l}
    \TSort{(6,5)}{a}{2}{r}

    \TSetTick{fc}{0}{00}
    \TSetTick{fc}{1}{01}
    \TSetTick{fc}{2}{10}
    \TSetTick{fc}{3}{11}
    \TSort{(3,0)}{fc}{4}{r}
    
    \TSetTick{fa}{0}{00}
    \TSetTick{fa}{1}{01}
    \TSetTick{fa}{2}{10}
    \TSetTick{fa}{3}{11}
    \TSort{(-3,0)}{fa}{4}{l}
    
    \THit{fc_2}{}{a_0}{.south west}{a_1}
    \path[bounce, bend left=60]
      \TBounce{a_0}{}{a_1}{.south west};
    
    \THit{c_1.north east}{}{a_1}{.west}{a_0}
    \path[bounce, bend right=50]
      \TBounce{a_1}{}{a_0}{.north west};
    
    \THit{a_0.west}{}{c_1}{.east}{c_0}
    \path[bounce, bend left=50]
      \TBounce{c_1}{}{c_0}{.north east};
    
    \THit{fa_3}{}{c_0}{.east}{c_1}
    \path[bounce, bend right=50]
      \TBounce{c_0}{}{c_1}{.south east};
    
    \path (0.8, 1.5) edge[coopupdate] (2.2, 1.5);
    \path (-5.3, 5.5) edge[coopupdate] (2.2, 2.5);
    \path (-0.8, 1.5) edge[coopupdate] (-2.2, 1.5);
    \path (5.3, 5.5) edge[coopupdate] (-2.2, 2.5);
    
    \node[labelprio1] at (-1.5,3.2) {$1$};
    \node[labelprio1] at (1.5,3.2) {$1$};
    \node[labelprio1] at (-1.5,1.9) {$1$};
    \node[labelprio1] at (1.5,1.9) {$1$};
    
    \node[labelprio2] at (0,6.4) {$2$};
    \node[labelprio2] at (1,5.7) {$2$};
    \node[labelprio2] at (-4.8,3.9) {$2$};
    \node[labelprio2] at (5,3.5) {$2$};
  \end{tikzpicture}
  \end{center}
  \caption{\figlabel{metazoan-phcanonique}%
    Frappes de Processus canoniques issues de l'aplatissement du modèle
    de la \vfigref{metazoan-php}.
    Les étiquettes numérotées (1 et 2) placées contre les flèches représentant les actions
    symbolisent leur appartenance à une classe de priorités donnée.
  }
  \end{figure}
\end{example}



\subsection{Aplatissement des Frappes de Processus avec arcs neutralisants}
\seclabel{phan-aplatissement}

À l'instar de la \secref{aplatissement}, il est possible de traduire les Frappes de Processus
avec arcs neutralisants en Frappes de Processus canoniques.
Le procédé est le même car il consiste, pour chaque action,
en un calcul de propriété d'aplatissement
qui est ici identique à la propriété de jouabilité.
En effet, l'opérateur d'aplatissement de cette traduction est égal à l'opérateur de jouabilité
donné à la \vdefref{fopphan} : $\Fopsymbol_\Fopaplatsubsymbol = \Fopsymbol_\Fopphansubsymbol$.
Par la suite, une fois cette propriété traduite en FND, il est possible de réutiliser
la traduction de la \defref{aplatissement} afin d'obtenir un modèle canonique
ayant la même dynamique, comme assuré par le \thmref{bisimulaplatissement}.

Cette traduction est elle aussi de complexité exponentielle dans le nombre d'actions
préemptant chaque action.
Cependant, on note que l'utilisation d'arcs neutralisants peut rendre cet aplatissement
beaucoup plus efficace.
En effet, contrairement aux Frappes de Processus avec classes de priorités,
les Frappes de Processus avec arcs neutralisants permettent une définition beaucoup plus fine
des préemptions entre actions.
Une des conséquences sur les modèles créés est un nombre bien moins important de relations
préempteur/préempté entre les actions, rendant la traduction plus efficace.



\subsection{Aplatissement des Frappes de Processus avec actions plurielles}
\seclabel{phm-aplatissement}

Il est possible de traduire les Frappes de Processus avec actions plurielles
en Frappes de Processus canoniques à l'aide d'outils précédemment développés.
En effet, la \vdefref{phm2php} offre une traduction des Frappes de Processus avec action
plurielles
en Frappes de Processus avec 4 classes de priorités, celles-ci pouvant être à leur tour traduites
en Frappes de Processus canoniques à l'aide de la \vdefref{aplatissement}.
Globalement, cette traduction est donc exponentielle dans le nombre d'actions dans le modèle
initial, car la traduction de chacune d'entre elles crée une sorte coopérative et plusieurs
actions de priorités différentes, qui doivent par la suite être aplaties.

\todoplustard{Conversion directe lorsque $\forall h \in \PHh, \card{\bond{h}} = 1$ ?}



\subsection{Représentation en Frappes de Processus avec actions plurielles}
\seclabel{phcanonique2phm}

Les Frappes de Processus canoniques permettent notamment de représenter des coopérations
entre processus.
Il peut être intéressant d'observer ces coopérations du point de vue des
Frappes de Processus avec actions plurielles, dont le formalisme est particulièrement
adapté à la représentation des coopérations.
Pour toutes Frappes de Processus canoniques $\PH$, nous proposons à la \defref{phcanonique2phm}
une représentation alternative $\PHmult(\PH)$ de ce modèle en Frappes de Processus avec
actions plurielles, et nous montrons au \thmref{bisimulphm} que $\PHmult(\PH)$
possède bien la même dynamique que $\PH$, aux mises à jour de sortes coopératives près.
Cette traduction se base sur l'interprétation de chaque action de priorité 2 du modèle $\PH$ :
\begin{itemize}
  \item Si le frappeur de l'action est un processus de composant ($\components$),
    la traduction est alors triviale ;
  \item Si à l'inverse, le frappeur est un processus de sorte coopérative ($\cs$),
    alors la sorte coopérative est étudiée et il y a autant d'actions multiples
    créées qu'il y a de configurations représentées par ce frappeur.
\end{itemize}
Autrement dit, les actions plurielles permettent d'avantageusement représenter
les sortes coopératives en représentant les ensembles de processus requis pour activer
un processus donné d'une sorte coopérative.
La \figref{livelock-phm} est la traduction en Frappes de Processus avec actions plurielles
du modèle simple de Frappes de Processus canoniques de la \figref{livelock}.

\begin{definition}
\deflabel{phcanonique2phm}
  Soient $\PH = (\PHs; \PHl; (\PHh^{(1)}; \PHh^{(2)}))$ des Frappes de Processus canoniques.
  On pose : $\PHmult(\PH) = (\ov{\PHs}; \ov{\PHl}; \ov{\PHh})$ les Frappes de Processus
  avec actions plurielles telles que :
  \begin{itemize}
    \item $\ov{\PHs} = \components$ ;
    \item $\ov{\PHl} = \bigtimes{a \in \PHs'} \PHl_a$ ;
    \item $\ov{\PHh} = \{ \PHfrappemult{\big( ps \cup \{ \target{h} \} \big)}{\bounce{h}}
      \mid h \in \PHh^{(2)} \wedge ps \in \virtualhitters(h) \}$
    avec, si on note $\hitter{h} = a_i$ :
    \[\virtualhitters(h) =
      \begin{cases}
        \{ \{ a_i \} \} & \text{si } a \in \components \\
        \csState(a_i) & \text{si } a \in \cs
      \end{cases}\]
  \end{itemize}
  De plus, pour tout état $s \in \PHl$,
  $\flats{s} = \os$ est l'état correspondant dans $\ov{\PHl}$ :
  $\forall a \in \ov{\PHs}, \PHget{\os}{a} = \PHget{s}{a}$.
  À l'inverse, étant donné un état $\os \in \ov{\PHl}$,
  on note $\unflats{\os} = s$ l'état correspondant dans $\PHl$ :
  $\forall a \in \components, \PHget{s}{a} = \PHget{\os}{a}$
  et $\forall f \in \cs, \PHget{s}{f} = f_\mysigma$
  avec $f_\mysigma \in \PHl_f$
  et $\forall b \in \compin(b), \PHget{\mysigma}{b} = \PHget{s}{b}$.
\end{definition}

\begin{theorem}[$\PH \approx \PHmult(\PH)$]
\thmlabel{bisimulphm}
  Soient $\PH = (\PHs; \PHl; (\PHh^{(1)}; \PHh^{(2)}))$ des Frappes de Processus canoniques,
  et $\PHmult(\PH) = (\ov{\PHs}; \ov{\PHl}; \ov{\PHh})$.
  \begin{enumerate}
    \item \label{bisimulph2phm} $\forall s, s' \in \PHl$,
      $s \trans{\PH} s' \Longrightarrow
      \flats{s} = \flats{s'} \vee \flats{s} \trans{\PHmult(\PH)} \flats{s'}$,
    \item \label{bisimulphm2ph} $\forall \os, \os' \in \ov{\PHl}$,
      $\os \trans{\PHmult(\PH)} \os' \Longrightarrow
      \unflats{\os} \mtrans{\PH} \unflats{\os'}$,
      où $\mtrans{\PH}$ est une séquence finie de transitions $\trans{\PH}$.
  \end{enumerate}
\end{theorem}

\begin{proof}
  (\ref{bisimulph2phm}) Soient $s, s' \in \PHl$ tels que $s \trans{\PH} s'$.
    Il existe donc une action $h \in \PHh$ telle que $s' = s \play h$.
    \begin{itemize}
      \item Si $h \in \PHh^{(1)}$, alors $\flats{s} = \flats{s'}$.
      \item Sinon, $h \in \PHh^{(2)}$ ; cela signifie qu'aucune action de priorité 1 n'est
        jouable, et qu'il existe $ps \in \virtualhitters(h)$ tel que
        $ps \cup \{ \target{h} \} \subseteq s$.
        D'après la \vdefref{phcanonique2phm}, il existe une action $g \in \ov{\PHh}$
        telle que $g = \PHfrappemult{ps \cup \{ \target{h} \}}{\bounce{h}}$.
        Ainsi, $g$ est jouable dans $\flats{s}$ et :
        $\flats{s} \play g = \flats{s} \recouvre \frappeur{h} = \flats{s'}$.
        Donc $\flats{s} \trans{\PHmult(\PH)} \flats{s'}$.
    \end{itemize}
    
  (\ref{bisimulphm2ph}) Soient $\os, \os' \in \ov{\PHl}$ tels que $\os \trans{\PHmult(\PH)} \os'$.
    Il existe donc une action $g \in \ov{\PHh}$ telle que $\os' = \os \play g$.
    De plus, cela signifie que $\frappeur{g} \subseteq \os$,
    donc $\frappeur{g} \subseteq \unflats{\os}$.
    Par construction de $\PHmult(\PH)$, il existe une action $h \in \PHh^{(2)}$
    telle que $\frappeur{g} = ps \cup \{ \cible{h} \}$ et $\bond{h} = \bond{g}$,
    avec $ps \in \virtualhitters(h)$.
    Par définition de $\unflats{\os}$, toutes les sortes coopératives sont mises à jour
    dans $\unflats{\os}$, ce qui fait que $h$ est jouable dans $\unflats{\os}$.
    Par ailleurs, d'après le \vlemref{update},
    il existe un scénario $\delta \in \Sce(\unflats{\os} \play h)$
    tel que $\unflats{\os} \play h \play \delta = \update(\unflats{\os} \play h)$.
    Enfin, par définition de $\update$ (\vdefref{update})
    on a : $\unflats{\os} \play h \play \delta = \unflats{\os'}$.
    Ainsi : $\unflats{\os} \mtrans{\PH} \unflats{\os'}$.
\end{proof}

\begin{figure}
  \centering
  \scalebox{1.2}{
  \begin{tikzpicture}[apdotsimple/.style={apdot}]
    \TSort{(0,0)}{a}{2}{l}
    \TSort{(0,3)}{b}{2}{l}
    \TSort{(4,2)}{c}{2}{r}
    
%     \THit{ab_3}{}{c_0}{.west}{c_1}
    
    \TActionPlur{a_1, b_1}{c_0.west}{c_1.south west}{}{2,2}{left}
    
%     \path[bounce, bend left]
%       \TBounce{c_0}{}{c_1}{.south west}
%     ;
    
    \TAction{a_1}{a_1.west}{a_0.north west}{selfhit}{right}
    \TAction{b_1}{b_1.west}{b_0.north west}{selfhit}{right}
    \TAction{a_0.south west}{b_0.west}{b_1.south west}{bend left=90}{left}
    \TAction{b_0}{a_0.west}{a_1.south west}{bend right=50}{left}
    
    \TState{a_0, b_0, c_0}
  \end{tikzpicture}
  }
  \caption{\figlabel{livelock-phm}%
    Frappes de Processus avec actions plurielles $\PHmult(\PH)$ issues de la traduction
    des Frappes de Processus canoniques $\PH$ de la \vfigref{livelock}
    d'après la \defref{phcanonique2phm}.
    Chaque action plurielle est représentée par un point relié aux frappeurs invariants
    par des arcs (sans flèche) et aux cibles et bonds par un couple de flèches
    (respectivement en trait plein puis en trait pointillé).
    Ici, l'action plurielle $\PHfrappemults{a_1, b_1, c_0}{c_1}$
    remplace la sorte coopérative $ab$
    et l'action $\PHfrappe{ab_{11}}{c_0}{c_1}$ de la \figref{livelock},
    qui permettaient de modéliser la coopération.
    Les processus grisés présentent un exemple d'état de départ :
    $\PHstate{a_0, b_0, c_0}$.
  }
\end{figure}





% Analyse statique

\section{Analyse statique}
\seclabel{as}

L'objectif de cette section est de définir le problème de l'\emph{atteignabilité} dans des
Frappes de Processus,
aussi appelée \emph{accessibilité},
et de proposer une sous-approximation permettant de la résoudre efficacement
dans les Frappes de Processus canoniques.

\myskip

Le problème de l'atteignabilité dans les Frappes de Processus consiste à rechercher l'existence
d'un scénario qui permette d'activer un ou plusieurs processus donné(s).
Il peut se résumer à la question suivante :
«~Étant donné un état initial, existe-t-il un scénario partant de cet état et
qui permette d'activer un processus donné ?~»
ou, de façon plus générale pour un ensemble de processus :
«~Étant donné un état initial $\ctx$ et une séquence de processus $\w$ donnés,
existe-t-il un scénario $\delta$ jouable dans $\ctx$ et permettant d'activer successivement
chacun des processus de $\w$ dans l'ordre ?~»
Ce problème d'atteignabilité peut parfois être résolu à l'aide des outils
de \textit{model checking} classiques.
Cependant, de telles méthodes reposent généralement sur l'analyse de la dynamique complète
du modèle.
Pour de grands modèles, ces méthodes se heurtent donc à l'explosion combinatoire inhérente
au calcul du graphe des états.

La méthode proposée ici repose en revanche sur une sous-approximation du modèle analysé.
Cela permet d'éviter la complexité exponentielle de l'analyse exhaustive de la dynamique,
car notre méthode possède une complexité polynomiale dans la taille du modèle
sous la condition que chaque sorte du modèle possède un nombre restreint de processus
(une sorte de quatre processus ou moins satisfaisant ce critère).
Cette méthode repose sur une succession d'analyses locales d'atteignabilités
qui se concentrent sur les sortes plutôt que sur le modèle complet.
Chaque atteignabilité est résolue sur une sorte en observant les actions qui permettent
de faire bondir le processus actif vers le processus recherché.
Comme ces actions sont éventuellement conditionnées par la présence d'un autre processus
d'une autre sorte, cela crée d'autres atteignabilités locales qui doivent être résolues.
Le problème est donc résolu récursivement, la condition d'arrêt étant soit
une atteignabilité locale impossible (ce qui peut empêcher de conclure),
soit un processus requis qui fait partie de l'état initial (ce qui consiste en une
atteignabilité locale \emph{triviale}).
Cette méthode est inspirée du travail de \citeasnoun{PMR12-MSCS},
qui portait sur l'atteignabilité dans les Frappes de Processus standards.
Les \defrangeref{obj}{maxCont} sont issues de la thèse de
\citefullname{Pauleve11}{Loïc}.
La contribution spécifique à cette thèse débute à la \defref{glc}
et comprend cette définition et tous les résultats suivants.

Un certain nombre d'outils préliminaires nécessaires à la résolution du problème
de l'atteignabilité dans des Frappes de Processus canoniques sont
définis à la \secref{sa-defs}.
Le mécanisme de résolution des atteignabilités locales mentionné au paragraphe précédent
est alors à son tour formalisé
à la \secref{ua} sous la forme d'un \emph{graphe de causalité locale} :
si ce graphe possède certaines propriétés, le \vthmref{approxinf} permet alors de conclure
quand à un problème d'atteignabilité donné.
Nous discutons aussi dans la suite du problèmes de l'atteignabilité
simultanée d'un ensemble de processus (\secref{as-etat}).
Nous proposons enfin une méthode permettant de raffiner cette approximation
dans le cas d'atteignabilités successives (\secref{approxinf-ordonnee}).

On considère dans toute la suite de cette section un modèle de Frappes de Processus canoniques
$\PH = (\PHs; \PHl; (\PHa^{(1)};\PHa^{(2)}))$
telles que définies à la \defref{phcanonique}.



\subsection{Définitions préliminaires}
\seclabel{sa-defs}

L'atteignabilité d'un processus $a_j$ d'une sorte $a$ donnée depuis un autre processus $a_i$
de la même sorte est le fait, depuis un état où $a_i$ est actif,
de pouvoir jouer un scénario menant dans un état où $a_j$ est actif.
La question de l'existence d'un tel scénario possède naturellement un intérêt particulier
dans la résolution d'une atteignabilité locale ; c'est pourquoi on la représente
sous la forme d'un \emph{objectif}, noté $\PHobjp{a}{i}{j}$ (\defref{obj}).
De plus, on appelle \emph{séquence d'objectifs} toute séquence dans laquelle
la cible de chaque objectif est égale au bond de l'objectif précédent de la même sorte
dans la séquence, s'il existe (\defref{os}).

\begin{definition}[Objectif ($\Obj$)]
\deflabel{obj}
  Si $a \in \components$, l'atteignabilité d'un processus $a_j$ depuis un processus $a_i$
  est appelé un \emph{objectif}, noté $\PHobj{a_i}{a_j}$.
  L'ensemble de tous les objectifs est noté
  $\Obj \DEF \{ \PHobj{a_i}{a_j} \mid
    a \in \components \wedge (a_i, a_j) \in \PHl_a \times \PHl_a \}$.
  Pour tout objectif $P = \PHobj{a_i}{a_j} \in \Obj$, on note
  $\sort{P} \DEF a$ la sorte de l'objectif $P$,
  $\target{P} \DEF a_i$ sa cible et $\bounce{P} \DEF a_j$ son bond.
  Enfin, $P$ est dit \emph{trivial} si $a_i = a_j$.
\end{definition}

\begin{definition}[Séquence d'objectif ($\Obj$)]
\deflabel{os}
  Une \emph{séquence d'objectifs} est une séquence $\w = P_1 \cons \ldots \cons P_{\card{\w}}$,
  où $\forall n \in \indexes{\w}, \w_n \in \Obj$
  et $\cible{\w_n} = a_i \Rightarrow \der{a}{\w_{1 \ldots n-1}} \in \{ \varnothing, a_i \}$.
  L'ensemble des séquences d'objectifs est référé par $\OS$.
%  Les définitions de $\premsymbol_a$ (\eqref{prem}), $\dersymbol_a$ (\eqref{der}),
%  $\suppsymbol$ (\eqref{supp}) et $\finsymbol$ (\eqref{fin}) sont étendues aux
  Les définitions de $\premsymbol_a$, $\dersymbol_a$,  $\suppsymbol$ et $\finsymbol$
  (\vdefref{premder}) sont étendues aux séquences d'objectifs en omettant
  de spécifier le cas des frappeurs.
\end{definition}

La \defref{ctx} introduit la notion de \emph{contexte} qui étend celle d'état
afin de pouvoir représenter un ensemble d'états initiaux possibles :
plutôt que d'attribuer un seul processus actif à chaque sorte, comme pour un état,
un contexte permet d'en attribuer plusieurs.
La notion de recouvrement, précédemment définie sur un états (\vdefref{recouvrement})
est étendue au cas d'un contexte dans la \defref{ctxrecouvrement}.
Il permettra à la \vdefref{glc} de saturer le contexte initial d'analyse
avec des processus supplémentaires.

\begin{definition}[Contexte ($\Ctx$)]
\deflabel{ctx}
  Un \emph{contexte} $\ctx$ associe à chaque sorte dans $\PHs$ un sous-ensemble non vide
  de ses processus :
  $\forall a \in \PHs, \PHget{\ctx}{a} \subseteq \PHl_a \wedge \PHget{\ctx}{a} \neq \emptyset$.
  On note $\Ctx$ l'ensemble de tous les contextes.
\end{definition}

\begin{definition}[Recouvrement ($\recouvre : \Ctx \times \powerset(\PHproc) \rightarrow \Ctx$)]
\deflabel{ctxrecouvrement}
  Pour tout contexte $\ctx \in \Ctx$ et tout ensemble de processus $ps \subset \Proc$,
  le recouvrement de $\ctx$ par $ps$ est noté $\ctx \recouvre ps$ et est défini par :
    \[ \forall a \in \PHs, \PHget{(\ctx \recouvre ps)}{a} \DEF
      \begin{cases}
        \{ p \in ps \mid \PHsort(p)=a \} & \text{si } \exists p \in ps, \PHsort(p)=a,\\
        \PHget{\ctx}{a} & \text{sinon.}
      \end{cases} \]
\end{definition}

Pour tout contexte $\ctx \in \Ctx$ et tout processus $a_i \in \Proc$, on note :
$a_i \in \ctx \EQDEF a_i \in \PHget{\ctx}{a}$,
et pour tout état $ps \in \PHl$ ou ensemble de processus $ps \subset \Proc$, on note :
$ps \subseteq \ctx \EQDEF \forall a_i \in ps, a_i \in \ctx$.
De plus, une séquence d'actions $\delta$ est \emph{jouable} dans un contexte $\ctx$
si et seulement si $\exists s \subseteq \ctx, \delta \in \Sce(s)$ ;
on note alors : $\delta \in \Sce(\ctx)$,
et le jeu de $\delta$ dans $\ctx$ est : $\ctx \PHplay \delta = \ctx \Cap \fin{\delta}$.

Finalement, une séquence de bonds sur une sorte $a$ (\defref{bs}) est une séquence d'actions
frappant $a$ dans laquelle le bond de chaque action est égal à la cible de l'action suivante,
en ignorant donc le frappeur de chaque action.
Les séquences de bonds sont utilisées pour trouver les solutions locales d'un objectif donné.
Une séquence de bonds sur $a$ peut de plus être abstraite par l'ensemble de tous les frappeurs
de ses actions qui ne sont pas dans $a$ (\defref{aBS}).
Cette abstraction permet de déplacer un objectif qui concerne une sorte $a$
vers d'autres objectifs sur d'autres sortes.
On note dans la suite : $\Sol = \powerset(\PHproc)$.

\begin{definition}[Séquence de bonds ($\BS$)]
\deflabel{bs}
  Une \emph{séquence de bonds} $\zeta$ est une séquence d'actions telle que
  $\forall n \in \indexes{\zeta}, n < \card{\zeta}, \PHbounce(\zeta_{n}) = \PHtarget(\zeta_{n+1})$.
  L'ensemble de toutes les séquences de bonds est appelé $\BS$,
  et on note $\BS(P)$ l'ensemble de toutes les séquences de bonds \emph{résolvant}
  un objectif $P$, appelé $\BS(P)$, qui est défini par :
    \[ \BS(\PHobj{a_i}{a_j}) \DEF \{ \zeta \in \BS \mid
      \PHtarget(\zeta_1) = a_i \wedge \PHbounce(\zeta_{\card{\zeta}}) = a_j \} \enspace. \]
\end{definition}

\noindent
On remarque que pour tout objectif $\obj{a_i}{a_j} \in \Obj$,
$\BS(\obj{a_i}{a_j}) = \emptyset$ s'il n'existe aucun moyen d'atteindre $a_j$ depuis $a_i$.
À l'inverse, la séquence vide appartient toujours à
l'ensemble des séquences de bonds résolvant un objectif trivial :
$\forall a_i \in \Proc, \emptyseq \in \BS(\obj{a_i}{a_i})$.

\begin{definition}[Séquence de bonds abstraite ($\aBS:\Obj \rightarrow \powerset(\Sol)$)]
\deflabel{aBS}
  \[
    \aBS(P) \DEF \{ \abstr{\zeta} \in \Sol \mid \zeta \in \BS(P), \nexists \zeta' \in \BS(P), \abstr{\zeta'} \subsetneq \abstr{\zeta} \} \enspace,
  \]
  où $\abstr{\zeta} \DEF \{ \PHhitter(\zeta_n) \mid  n \in \indexes{\zeta} \wedge \PHsort(\PHhitter(\zeta_n)) \neq \PHsort(P) \}$.
\end{definition}



\subsection{Sous-approximation}
\seclabel{ua}

On note $\concr(\w)$ l'ensemble des scénarios concrétisant
une séquence d'objectifs $\w$ dans le contexte $\ctx$ (\defref{concr})
et $\uconcr(\w)$ est défini comme étant égal à $\concr(\w)$ si et seulement si,
pour chaque état $s \subseteq \ctx$,
$\concr(\w) \cap \Sce(s) \neq \emptyset$ (\defref{uconcr}).

\begin{definition}[$\concr : \OS \to \powerset(\Sce)$]
\deflabel{concr}
  Pour toute séquence d'objectifs $\w \in \OS$, $\concr(\w)$ est l'ensemble des
  scénarios minimaux concrétisant $\w$ dans le contexte $\ctx$.
  Il est défini comme le plus grand ensemble satisfaisant les conditions suivantes :
  \begin{enumerate}[(i)]
  \item $\forall \delta \in \concr(\w), \exists s \subseteq \ctx, \delta \in \Sce(s)$,
  \item $\forall \delta \in \concr(\w), \exists \phi : \indexes{\w} \to \indexes{\delta},
      (\forall n, m \in \indexes{\w}, n < m \Leftrightarrow \phi(n) \leq \phi(m)),
      \forall n \in \indexes{\w}, \PHbounce(\w_n) \in \ctx \play \delta_{1 \ldots \phi(n)}$,
  \item $\forall \delta, \delta' \in \concr(\w),
      \card{\delta} \leq \card{\delta'} \Rightarrow \delta \neq \delta'_{1 \ldots \card{\delta}}$.
  \end{enumerate}
\end{definition}

\begin{definition}[$\uconcr : \OS \to \powerset(\Sce)$]
\deflabel{uconcr}
  \[ \uconcr(\w) \DEF
    \begin{cases}
      \concr(\w) & \text{si } \forall s \in \PHl, s \subseteq \ctx, \exists \delta \in \concr(\w),
        \delta \in \Sce(s) \\
      \emptyset & \text{sinon.}
    \end{cases} \]
\end{definition}

\begin{lemma}
\lemlabel{uconcr-ctx}
  $\ctx \subseteq \ctx' \wedge \muconcr_{\ctx'}(\w) \neq \emptyset \Rightarrow
    \muconcr_{\ctx}(\w) \neq \emptyset$.
\end{lemma}

Pour tout objectif $P \in \Obj$ et tout contexte $\ctx \in \Ctx$, la \defref{maxCont}
permet d'obtenir $\gCont_\ctx(\PHsort(P), P)$
qui est l'ensemble des processus de $\sorte{P}$ requis pour résoudre $P$ dans $\ctx$.
%appelé $\gCont_\ctx(\PHsort(P), P)$.
Cette définition sera utile pour correctement résoudre les atteignabilités locales
qui nécessitent indirectement un processus de leur propre sorte,
c'est-à-dire autrement que par une auto-action.

\begin{definition}[$\gCont_\ctx : \Sigma \times \Obj \to \powerset(\PHproc)$]
\deflabel{maxCont}
  \begin{align*}
    \gCont_\ctx(a,P) \DEF
    \{ p \in \PHproc &\mid \exists ps \in \aBS(P), \exists b_i \in ps, b = a \wedge p = b_i \\
      & \vee b \neq a \wedge p \in \gCont_\ctx(a, \PHobj{b_j}{b_i}) \wedge b_j \in \PHget{\ctx}{b} \}
    \enspace.
  \end{align*}
\end{definition}

Pour une séquence d'objectifs $\w$ et un contexte $\ctx$ donnés,
le \emph{graphe de causalité locale} $\cwB$ (\defref{glc}) représente une sous-approximation de
l'atteignabilité de cette séquence d'objectifs dans $\ctx$.
Pour cela, il relie les objectifs à des solutions à
l'aide des séquences de bonds abstraites de la \defref{aBS}, ce qui produit de nouveaux objectifs
résolus récursivement.
Il s'agit donc d'un graphe dont les nœuds sont des éléments de $\Proc \cup \Obj \cup \Sol$,
c'est-à-dire des processus, des objectifs et des \emph{solutions}
(c'est-à-dire des ensembles de processus) :
\begin{itemize}
  \item Un nœud dans $\Obj$ représente un objectif requis pour la résolution de $\w$,
    soit faisant directement partie de la séquence d'objectifs $\w$,
    soit indirectement nécessaire à sa résolution ;
  \item Un nœud dans $\Sol$ représente un ensemble de processus nécessaires pour résoudre
    un objectif, c'est-à-dire un élément parmi ses séquences de bonds abstraites ;
  \item Un nœud dans $\Proc$ représente un processus requis pour la résolution,
    c'est-à-dire mentionné dans un nœud solution.
\end{itemize}

Un objectif $P \in \Obj$ est soluble si tous les processus contenus dans au moins une de ses
abstractions de séquences de bonds $\aBS(P) \in \Sol$ (\cf \defref{aBS})
peuvent être activés (\eqref{ESol1}).
Une telle solution représente donc un ensemble de processus qui doivent être activés
pour la résolution de $P$ (\eqref{ESol2}).
Si $a \in \components$, l'atteignabilité d'un de ses processus $a_i$ est approximée par
la possibilité de résoudre tous les objectifs de la forme $\PHobjp{a}{j}{i} \in \Obj$
pour tout $a_j$ dans le contexte initial $\ctx$ (\eqref{EReq}) ;
si $a \in \cs$, l'atteignabilité de $a_i$ est possible si tous les processus du sous-état
$\csState(a_i)$ (\cf \defref{csState}) qu'il représente sont atteignables (\eqref{EPrio}).
La résolution d'un objectif $P$ peut nécessiter un processus $p$ de $\PHsort(P)$,
autrement dit : $\gCont(\PHsort(P), P) \neq \emptyset$ (\cf \defref{maxCont}) ;
dans ce cas, $P$ est \emph{re-centré} en $p$ (\eqref{ECont})
afin de s'assurer que la résolution intermédiaire de $\PHobj{\cible{P}}{p}$ n'interfère pas.
Enfin, les \eqref{Vw,Vproc,VE} assurent que l'ensemble des nœuds est complet.

Étant donné que le processus actif de chaque sorte peut évoluer au cours de la résolution,
le graphe de causalité locale $\cwB$ est obtenu par \emph{saturation} avec tous les processus
qu'il contient, \cad en recouvrant le contexte initial $\ctx$ par $\allprocs(\V, \E)$, défini par:
  \[\allprocs(\V, \E) = (V \cap \Proc) \cup
    \{ \PHtarget(P), \PHbounce(P) \mid P \in \V \cap \Obj \} \enspace.\]
Ce recouvrement est effectué autant de fois que nécessaire ;
le graphe de causalité locale est donc re-calculé avec cette saturation
jusqu'à ce qu'il n'évolue plus ---~autrement dit, jusqu'à atteindre un point fixe.

\begin{definition}
\deflabel{glc}
  Le graphe de causalité locale $\cwB \DEF (\Bv, \Be)$ est défini par :
  $\cwB \DEF \lfp{\aB^\w_\ctx}{\myB}{\aB^\w_{\ctx \Cap \allprocs(\myB)}}$,
  où $\myB \DEF (\cwV, \cwE)$ est le plus petit graphe respectant
  $\cwV \subseteq \Proc \cup \Obj \cup \Sol$ et $\cwE \subseteq \cwV \times \cwV$
  tel que :
  \begin{align}
    \w &\subseteq \cwV \label{eq:Vw} \\
    P \in \cwVObj &\Rightarrow \PHbounce(P) \in \cwV \label{eq:Vproc} \\
    (x, y) \in \cwE &\Rightarrow y \in \cwV \label{eq:VE} \\
    P \in \cwVObj \wedge ps \in \BS(P) &\Rightarrow (P, ps) \in \cwE \label{eq:ESol1} \\
    ps \in \cwVSol \wedge a_i \in ps &\Rightarrow (ps, a_i) \in \cwE \label{eq:ESol2} \\
    a \in \components \wedge a_i \in \cwVProc \wedge a_j \in \ctx &\Rightarrow (a_i, \PHobjp{a}{j}{i}) \in \cwE \label{eq:EReq} \\
    a \in \cs \wedge a_i \in \cwVProc \wedge ps \in \csState(a_i) &\Rightarrow (a_i, ps) \in \cwE \label{eq:EPrio} \\
    P \in \cwVObj \wedge q \in \gCont_\ctx(\PHsort(P), P) &\Rightarrow (P, \PHobj{q}{\PHbounce(P)}) \in \cwE \label{eq:ECont} \!
  \end{align}
\end{definition}

\begin{example}
\exlabel{livelock-glc}
  La \vfigref{glc-livelock} représente le graphe de causalité locale
  associé au modèle de Frappes de Processus canoniques de la \vfigref{livelock},
  pour la question de l'accessibilité de $c_1$ depuis l'état
  $\PHstate{a_1, b_0, c_0, ab_{10}}$.
  Nous discutons \vexpageref{livelock-as} des conclusions qui peuvent en être tirées.
\end{example}

Au sein de ce graphe de causalité locale, un arc $(p, ps) \in \Proc \times \Sol$
est dit \emph{cohérent} (\defref{coherent}) si aucun des processus dans $ps$
n'est «~compromis~» par un processus successeur du nœud $ps$,
%n'entre en conflit avec les processus successeur du nœud $ps$,
\cad si, pour tout processus de $ps$,
il n'existe pas de processus différent de la même sorte parmi tous les successeurs de $ps$.
%
%s'il n'existe pas un processus dans $ps$ et un processus parmi les successeurs de $ps$
%qui soient différents mais appartenant à la même sorte.
Si tous les arcs du graphe sont cohérents, alors le \thmref{approxinf}
donne une condition suffisante pour la concrétisation de la séquence d'objectifs $\w$
dans le contexte $\ctx$, qui est basée directement sur ce graphe $\cwB$.

\begin{definition}[Arc cohérent]
\deflabel{coherent}
  Un arc $(x, y) \in \cwE$ est dit \emph{cohérent} si et seulement si
  $(x, y) \in \Be \cap (\Proc \times \Sol) \Rightarrow y$ n'a aucun successeur
  $a_j \in \Bv \cap \Proc$
  tel que $\exists a_i \in y, \sorte{a_i} = \sorte{a_j} \wedge a_i \neq a_j$.
\end{definition}

\begin{theorem}[Sous-Approximation]
\thmlabel{approxinf}
  Étant données des Frappes de Processus canoniques $\PH = (\PHs; \PHl; (\PHa^{(1)};\PHa^{(2)}))$,
  un contexte $\ctx \in \Ctx$ et une séquence d'objectifs $\w \in \OS$,
  si le graphe $\cwB$ ne contient aucun cycle,
  que tous ses nœuds objectifs possèdent au moins une solution
  et que tous ses arcs sont cohérents,
  alors $\uconcr(\w) \neq \emptyset$.
\end{theorem}

\begin{proof} %[\Thmref{approxinf}]
  On note dans la suite :
  $\Bee{X}{Y} = \Be \cap (X \times Y)$, avec $X, Y$ parmi $\PHproc$, $\Obj$ et $\Sol$,
  et : $max\ctx = \ctx \Cap \allprocs(\cwB)$ le contexte accepté par $\cwB$.
  
  Soit $(a_i, ps) \in \Bee{\Proc}{\Sol}$ un arc liant un processus requis de sorte coopérative à
  une solution et supposons que tous les enfants de $ps$ sont concrétisables.
  On étiquette tous les processus de $ps$ par un entier : $ps = \{ p_n \}_{n \in \indexes{ps}}$.
  Montrons par récurrence que pour tout $n \in \segm{0}{\card{ps}}$,
  il existe un scénario $\delta_n$ tel que :
  $\forall i \in \segm{1}{n}, \PHget{(s \PHplay \delta_n)}{\PHsort(p_i)} = p_i$.
  \begin{itemize}
    \item Le cas $\delta_0 = \varepsilon$ est immédiat.
    \item Soit $n \in \segm{0}{\card{ps} - 1}$.
      On suppose qu'il existe $\delta_n$ tel que décrit ci-dessus.
      Posons $q = \PHget{(s \PHplay \delta_n)}{\PHsort(p_{n+1})}$.
      Par hypothèse, $(a_i, ps)$ est cohérent (\defref{coherent}) et tous les processus
      de $ps$ sont des processus de composants ;
      cela signifie qu'aucun des processus requis pour résoudre $p_{n+1}$ n'est un autre processus
      de la même sorte qu'un processus de $ps$.
      Par conséquent, il existe un scénario
      $\delta' \in \muconcr_{s \PHplay \delta_n}(\PHobj{q}{p_{n+1}})$
      tel que $\forall i \in \segm{1}{n+1},
        \PHget{(s \PHplay \delta_n \PHplay \delta')}{\PHsort(p_{i})} = p_{i}$.
      Finalement, d'après le \lemref{update}, il existe un scénario
      $\delta'' \in \reductionsce(s \PHplay \delta_n \PHplay \delta')$
      tel que $\update(s \PHplay \delta_n \PHplay \delta') = s \PHplay \delta_{n+1}$
      avec $\delta_{n+1} = \delta_n \PHplay \delta' \PHplay \delta''$,
      et d'après le \lemref{hcscomp} :
      $\forall i \in \segm{1}{n+1}, \PHget{(s \PHplay \delta_{n+1})}{\PHsort(p_i)} = p_i$
  \end{itemize}
  Ainsi, $\delta = \delta_{|ps|}$ existe, et étant données ses propriétés, on a immédiatement :
  $\PHget{(s \PHplay \delta)}{a} = a_i$ et $\update(s \PHplay \delta) = s \PHplay \delta$.
  
  Soit un état $s \in L$ tel que $s \subseteq max\ctx$.
  Étant donné qu'il n'y a aucun cycle dans $\cwB$, montrons par récurrence que
  pour tout objectif $P \in \Bv \cap \Obj$ tel que $\PHtarget(P) \in s$,
  $\exists \delta \in \muconcr_s(P)$.
  \begin{itemize}
    \item Si $(P, \emptyset) \in \Bee{\Obj}{\Sol}$,
      soit on a $\PHtarget(P) = \PHbounce(P)$ et $\delta = \emptyseq$,
      soit on a $\forall \zeta \in \BS(P), \zeta \in \Sce(s) \wedge \PHsort(\zeta) = \{ \PHsort(P) \}$
      et dans ce cas $\delta = \delta_1 \PHplay \zeta_1 \PHplay \dots \PHplay
        \delta_{|\zeta|} \PHplay \zeta_{|\zeta|}$
      est une séquence valide donnée par le \lemref{hcompcomp}.
    \item Supposons que tous les objectifs qui sont les successeurs de $P$ sont concrétisables.
      Si $\exists (P, Q) \in \Bee{\Obj}{\Obj}$, alors, par hypothèse,
        $\muconcr_{s}(\obj{\PHtarget(P)}{\PHtarget(Q)} \concat Q) \neq \emptyset$, et donc
        $\muconcr_{s}(P) \neq \emptyset$.
      Sinon, d'après la \defref{maxCont}, la concrétisation des successeurs de $P$ ne requiert
        aucun processus de la sorte $\PHsort(P)$.
        De plus, il existe $\zeta \in \BS(P)$ tel que $(P, \aZ) \in \Bee{\Obj}{\Sol}$.
        Montrons par récurrence que pour tout $n \in \indexes{\zeta}$, il existe un scénario
        $\delta_n$ tel que $\PHget{(s \PHplay \delta_n)}{\PHsort(P)} = \PHbounce(\zeta_n)$.
        \begin{itemize} %\item[$\circ$]
          \item[] Supposons que $\delta_n$ existe et posons $\zeta_n = \PHhit{b_i}{a_j}{a_k}$.
            Par hypothèse, il existe $\delta' \in \muconcr_{s \PHplay \delta_n}(\PHobj{\any}{b_i})$
            avec $\PHsort(P) \notin \PHsort(\delta')$ (\defref{maxCont}).
            D'après le \lemref{update}, il existe
            $\delta'' \in \reductionsce(s \PHplay \delta')$ tel que
            $\update(s \PHplay \delta') = s \PHplay \delta' \PHplay \delta''$.
            De plus, $\PHget{(s \PHplay \delta' \PHplay \delta'')}{b} = b_j$
            (D'après le \lemref{hcompcomp} si $b \in \components$
            ou le \lemref{hcscomp} si $b \in \cs$).
            Ainsi, $\delta_{n+1} = \delta_n \PHplay \delta' \PHplay \delta'' \PHplay \zeta_n$.
        \end{itemize}
      On a donc :$\delta_{|\zeta|} \in \muconcr_s(P)$. % and $\ceil(\delta) \subseteq max\ctx$.
  \end{itemize}
  Finalement, étant donné $\muconcr_{max\ctx}(\w) \neq \emptyset$,
  et d'après le \lemref{uconcr-ctx},
  on a : $\uconcr(\w) \neq \emptyset$.
\end{proof}



\begin{remark}
\label{subsetsolution}
  Le \thmref{approxinf} peut s'appliquer à tout graphe de causalité locale
  $\widehat{\cwB}$
  construit à partir d'un graphe $\widehat{\myB} = (\widehat{\cwV}, \widehat{\cwE})$
  où $\widehat{\cwV} \cap \Sol \subset \cwV \cap \Sol$.
  En effet, cela revient à réduire l'ensemble initial des solutions,
  ce qui réduit aussi l'ensemble de nœuds processus et objectifs utilisés.
  La solution est alors davantage contrainte, mais le résultat est toujours valable.
  Cela revient en fait à s'interdire certaines solutions,
  c'est-à-dire à calculer l'atteignabilité sur un graphe privé de certaines actions.
  Ainsi, si la sous-approximation est non-conclusive, il est possible de la tester
  sur tous les graphes comportant un sous-ensemble des nœuds solutions,
  ce qui permet notamment de supprimer certains cycles
  et parfois d'obtenir un graphe de causalité locale sur lequel il est possible de conclure.
  Cette recherche exhaustive est cependant exponentielle dans le nombre de nœuds solutions,
  mais il est possible de l'orienter de façon à trouver rapidement un sous-ensemble permettant
  de conclure,
  par exemple en retirant en priorité les solutions qui forment un cycle.
\end{remark}



\begin{remark}
  \citeasnoun{PMR12-MSCS} ont proposé une méthode de sur-approximation qui se base sur
  un graphe de causalité locale construit de façon similaire,
  et permet de réfuter une atteignabilité au sein d'un modèle de
  Frappes de Processus standards.
  Il est intéressant de noter que cette sur-approximation
  est toujours valable sur les Frappes de Processus canoniques à condition de l'appliquer
  sur la version fusionnée du modèle considéré (cf. \vdefref{aplatissement}).
  Cela permet d'obtenir un résultat supplémentaire en concluant
  dans certains cas quant à l'impossibilité d'atteindre un processus donné.
\end{remark}



\begin{example}
\exlabel{livelock-as}
  En ce qui concerne le modèle de Frappes de Processus canoniques de la \vfigref{livelock},
  la sous-approximation développée au \thmref{approxinf}
  ne conclut pas quant à l'accessibilité de $c_1$ depuis l'état
  $\PHstate{a_1, b_0, c_0, ab_{10}}$.
  En effet, comme le montre la \figref{glc-livelock},
  l'arc représenté en double trait liant le nœud processus $ab_{11}$ à son unique solution
  n'est pas cohérent selon la
  \defref{coherent}, ce qui empêche l'application du théorème.
  
  De même, la sur-approximation de \citeasnoun{PMR12-MSCS} (appliquée à $\PH'$)
  renvoie aussi un résultat non-conclusif,
  du fait que les deux approches ne peuvent pas être conclusives en même temps
  pour des raisons de cohérence mathématique.
  La méthode d'analyse statique ne répond donc globalement pas sur cet exemple,
  et de façon plus générale sur tous les exemples dont l'atteignabilité recherchée
  est rendue impossible par simple ajout de classes de priorités.
  
  On note pour finir que le \thmref{approxinf} est conclusif sur l'atteignabilité de $c_1$
  depuis $\PHstate{a_1, b_0, c_0, ab_{10}}$ dans les Frappes de Processus canoniques $\PH''$, où :
  \begin{align*}
  \PH'' &= (\PHs; \PHl; (\PHa^{(1)};\PHa'^{(2)})) \\
  \text{avec : } \quad
  \PHa'^{(2)} &= (\PHa^{(2)} \setminus \{ \PHhit{a_0}{b_0}{b_1}, \PHhit{b_0}{a_0}{a_1} \})
    \cup \{ \PHhit{a_0}{a_0}{a_1}, \PHhit{b_0}{b_0}{b_1} \}
  \end{align*}

  En effet, dans ce cas les processus $a_0$ et $b_0$ du graphe de la \figref{glc-livelock}
  sont permutés, ce qui rend tous les arcs cohérents.
  
  \begin{figure}[tp]
    \centering
    \begin{tikzpicture}[aS]
      \node[Aproc] (c1) {$c_1$};
      \node[Aobj,below of=c1] (c01) {$\PHobj{c_0}{c_1}$};
      \node[Asol,below of=c01] (c01s) {};

      \node[AprocPrio,below of=c01s] (ab11) {$ab_{11}$};
      \node[AsolPrio,below of=ab11] (ab11s) {};

      \node[Aproc,below left of=ab11s] (a1) {$a_1$};
      \node[Aobj,below of=a1] (a11) {$\PHobj{a_1}{a_1}$};
      \node[Asol,below of=a11] (a11s) {};
      \node[Aobj,below left of=a1] (a01) {$\PHobj{a_0}{a_1}$};
      \node[Asol,below of=a01] (a01s) {};
      \node[Aproc,below of=a01s] (b0) {$b_0$};
      \node[Aobj,below of=b0] (b00) {$\PHobj{b_0}{b_0}$};
      \node[Asol,below of=b00] (b00s) {};
      \node[Aobj,below left of=b0] (b10) {$\PHobj{b_1}{b_0}$};
      \node[Asol,below of=b10] (b10s) {};

      \node[Aproc,below right of=ab11s] (b1) {$b_1$};
      \node[Aobj,below of=b1] (b11) {$\PHobj{b_1}{b_1}$};
      \node[Asol,below of=b11] (b11s) {};
      \node[Aobj,below right of=b1] (b01) {$\PHobj{b_0}{b_1}$};
      \node[Asol,below of=b01] (b01s) {};
      \node[Aproc,below of=b01s] (a0) {$a_0$};
      \node[Aobj,below of=a0] (a00) {$\PHobj{a_0}{a_0}$};
      \node[Asol,below of=a00] (a00s) {};
      \node[Aobj,below right of=a0] (a10) {$\PHobj{a_1}{a_0}$};
      \node[Asol,below of=a10] (a10s) {};

      \path
      (c1) edge (c01)
      (c01) edge (c01s)
      (c01s) edge (ab11)
      (ab11) edge[aSPrio] (ab11s)
      (ab11s) edge (a1) edge (b1)

      (a1) edge (a01) edge (a11)
      (a01) edge (a01s)
      (a01s) edge (b0)
      (a11) edge (a11s)
      (a0) edge (a10) edge (a00)
      (a10) edge (a10s)
      (a00) edge (a00s)

      (b0) edge (b10) edge (b00)
      (b10) edge (b10s)
      (b00) edge (b00s)
      (b1) edge (b01) edge (b11)
      (b01) edge (b01s)
      (b01s) edge (a0)
      (b11) edge (b11s)
      ;
      \end{tikzpicture}
    \caption{\figlabel{glc-livelock}%
      Le graphe de causalité locale des Frappes de Processus de la \figref{livelock}
      pour l'objectif $\w = \PHobj{c_0}{c_1}$
      et le contexte initial $\ctx = \PHstate{a_1, b_0, c_0, ab_{10}}$.
      Les nœuds rectangulaires représentent les éléments de $\Proc$,
      les nœuds sans bordure sont les éléments de $\Obj$
      et les cercles sont les éléments de $\Sol$.
      Le processus $ab_{11}$, ainsi que son unique solution et l'arc qui les relie,
      sont mis en valeur avec des traits doubles car il s'agit du principal ajout de la méthode
      présentée à la \secref{ua}.
      Il est à noter que l'arc dessiné avec un trait double n'est pas cohérent
      au sens de la \defref{coherent}.
      En effet, sa cible est la solution $\{ a_1, b_1 \}$,
      or l'un de ses successeurs indirects est $a_0$, qui est un autre processus de la même sorte
      que $a_1$ (et le même raisonnement fonctionne pour $b_0$).
    }
  \end{figure}
\end{example}



Comme nous l'avons vu, l'analyse statique développée dans cette section est une approximation,
et peut retourner un résultat non-conclusif ;
Le modèle de la \figref{livelock} traité \vpageref[ci-dessus]{ex:livelock-as} en est un exemple.
Une partie de ces cas non-conclusifs apparaissent notamment pour un motif particulier,
mis en valeur par la notion de cohérence de la \defref{coherent}.
Cela est dû notamment au fait que la méthode de sur-approximation n'a pas été raffinée
dans le présent travail, ce qui mène à des cas non-conclusifs lorsque l'ajout de priorités
empêche certains comportements.

D'autres situations peuvent aussi empêcher de conclure : c'est notamment le cas des
atteignabilités nécessitant des «~allers-retours~»,
c'est-à-dire l'activation d'un processus $p$ plusieurs fois pendant la résolution.
Si d'autres requis sont intercalés entre les différentes occurrences de $p$,
le graphe de causalité locale va alors présenter un cycle, ce qui empêche l'utilisation
du \thmref{approxinf}.
L'une des alternatives est alors de détecter et d'expliciter cette séquentialité,
ce qui permet par exemple d'utiliser le résultat qui sera présenté à la
\vsecref{approxinf-ordonnee}.



\subsection{Atteignabilité d'un sous-état}
\seclabel{as-etat}

\newcommand{\uastotal}{\tau}
\newcommand{\uasreach}{\rho}
\newcommand{\uasps}{{ps}}

La propriété d'atteignabilité développée à la \secref{ua}
sur les Frappes de Processus canoniques
ne traite l'atteignabilité d'un ensemble de processus que de façon séquentielle.
Cependant, il est possible de vérifier l'atteignabilité d'un sous-état
(autrement dit, l'atteignabilité simultanée d'un ensemble de processus)
à l'aide d'une sorte coopérative.

En effet, soient $\PH = (\PHs, \PHl, (\PHh^{(1)}, \PHh^{(2)}))$ des Frappes de Processus
canoniques et supposons que l'ont cherche à étudier l'atteignabilité d'un sous-état
$\uasps \in \PHsubl_S$, avec $S \subset \PHs$.
On pose alors : $\PH' = (\PHs', \PHl', (\PHh'^{(1)}, \PHh'^{(2)}))$
les Frappes de Processus canoniques telles que :
\begin{itemize}
  \item $\PHs' = \PHs \cup \{ \uastotal, \uasreach \}$,
  \item $\PHl' = \PHl \times \PHl_\uastotal \times \PHl_\uasreach$, où
    $\PHl_\uastotal = \PHsubl_S$ et $\PHl_\uasreach = \{ \uasreach_0, \uasreach_1 \}$,
  \item $\PHh'^{(1)} = \PHh^{(1)} \cup
    \{ \PHfrappe{a_i}{\uastotal_\mysigma}{\uastotal_{\mysigma'}} \mid
    a \in S, \mysigma, \mysigma' \in \PHl_\uastotal,
    \PHget{\mysigma}{a} \neq a_i \wedge \mysigma' = \mysigma \recouvre a_i \}$,
  \item $\PHh'^{(2)} = \PHh^{(2)} \cup
    \{ \PHfrappe{\uastotal_\uasps}{\uasreach_0}{\uasreach_1} \}$.
\end{itemize}
Cette transformation consiste donc à ajouter au modèle
une sorte coopérative $\uastotal$ sur toutes les sortes de $S$,
et un composant $\uasreach$ qui ne puisse changer de processus que lorsque le sous-état $\uasps$
est présent (ce qui est déterminé par $\uastotal$).
Ainsi, l'atteignabilité du sous-état $\uasps$ depuis un contexte initial $\ctx$ dans $\PH$
est équivalente à celle du processus $\uasreach_1$ depuis le contexte
$\ctx \cup \{ \uasreach_0 \}$ dans $\PH'$
(l'état initial de $\uastotal$ n'a pas d'importance et peut être arbitrairement choisi),
qui peut être traitée grâce au \thmref{approxinf}.

Nous pouvons donc répondre à des questions d'atteignabilité simultanée de plusieurs processus
directement à l'aide des Frappes de Processus canoniques (\defref{phcanonique})
et de l'analyse statique développée pour ce formalisme (\thmref{approxinf}).
Cette méthode peut naturellement être adaptée pour répondre quant à
l'accessibilité d'un ensemble de sous-états.
Il est à noter cependant que le nombre de processus de la sorte coopérative $\uastotal$
croît exponentiellement avec le nombre de sortes dans $S$, ce qui peut fortement impacter
la vitesse de résolution de l'analyse statique.
Pour pallier cela, il est possible de «~factoriser~» cette sorte coopérative comme expliqué
\vpageref{factorisation-coop}.

\todoplustard{Exemple ! Lequel ?}



\subsection{Raffinement de la sous-approximation séquentielle}
\seclabel{approxinf-ordonnee}

Dans cette section, nous donnons une alternative à la condition suffisante du \thmref{approxinf}
qui permet de prendre en compte la séquentialité des objectifs plutôt que de les considérer
simultanément, tel que cela est fait dans la version actuelle.
Comme les objectifs sont pris en compte individuellement, une telle approche ne prend en compte
qu'un sous-ensemble des scénarios possibles.
Cependant, en se concentrant à chaque itération sur une plus petite partie du réseau, 
cette sous-approximation séquentielle peut s'avérer plus souvent conclusive.

Définissions une séquence d'objectifs $\w = \obj{a_i}{a_j} \concat \w'$ avec
$a_i \neq a_j$ et un état $s \in \PHl$ tel que $\get{s}{a} = a_i$.
On peut remarquer que tout scénario atteignant $a_j$ inclut nécessairement l'une des séquences
de bonds dans $\BS(\obj{a_i}{a_j})$ et, en particulier,
tout scénario minimal atteignant $a_j$ termine dans un état où son présents à la fois
$a_j$ et le frappeur $a_k$ de la dernière action d'une des séquences de bonds
dans $\BS(\obj{a_i}{a_j})$.
Si la sorte $b$ d'un tel frappe est de surcroît une sorte coopérative ($b \in \cs$),
cela signifie alors aussi que l'un des sous-états dans $\csState(b_k)$
est inclus dans l'état final.
La \defref{lastprocs} définit $\derprocs(\obj{a_i}{a_j})$ comme étant l'ensemble des ensembles
de processus qui peuvent être présents juste après avoir atteint $a_j$.

D'après le \thmref{approxinf}, on peut déduire que pour tout scénario $\delta \in \uconcr(P)$,
il existe un ensemble de processus $ps \in \derprocs(P)$ tel que $ps \subset (s \play \delta)$.
Donc, si $\muconcr_{\ctx' \Cap ps}(\w') \neq \emptyset$,
avec $\ctx' = \ctx \Cap \procs(\mycwB{\ctx}{P})$,
il existe alors un scénario $\delta'$ concrétisant $\w'$ depuis l'état $(s \play \delta)$.
Ainsi, le scénario  $\delta \concat \delta'$ concrétise $\w$.

\begin{definition}[$\derprocs : \Obj \to \powerset(\powerset(\Proc))$]
\deflabel{lastprocs}
  Pour tout objectif $\obj{a_i}{a_j} \in \Obj$, $\derprocs(\obj{a_i}{a_j})$
  est défini comme le plus grand ensemble tel que :
  $\forall lps \in \derprocs(\obj{a_i}{a_j}), lps \in \powerset(\Proc)$,
  \begin{enumerate}
    \item $a_j \in lps$,
    \item $\exists \zeta \in \BS(\obj{a_i}{a_j}),
        \sorte{\hitter{\zeta_{\card{\zeta}}}} \neq \sorte{\bounce{\zeta_{\card{\zeta}}}}
        \Rightarrow \hitter{\zeta_{\card{\zeta}}} \in lps$,
    \item $\forall b_j \in lps, b \in \cs \Rightarrow \exists ps \in \csState(b_j), ps \subset lps$,
    \item $\nexists lps'\in \derprocs(\obj{a_i}{a_j}), lps' \varsubsetneq lps$.
  \end{enumerate}
\end{definition}

\begin{theorem}[Sous-approximation séquentielle]
\thmlabel{approxinf-ordonnee}
  Pour toutes Frappes de Processus canoniques $(\PHs; \PHl; \PHa^{\langle 2 \rangle})$,
  tout contexte $\ctx \in \Ctx$ et toute séquence d'objectifs $\w = P \concat \w' \in \OS$,
  $\uconcr(P) \neq \emptyset \wedge \forall ps \in \derprocs(P),
    \muconcr_{\ctx' \Cap ps}(\w') \neq \emptyset
    \Rightarrow \uconcr(\w) \neq \emptyset$,
  où $\ctx' = \ctx \Cap \procs(\mycwB{\ctx}{P})$.
\end{theorem}

\begin{proof} %[\Thmref{approxinf-ordonnee}]
  Si $\uconcr(P) \neq \emptyset$, alors pour tout état $s \in \PHl, s \subset \ctx$,
  il existe un scénario $\delta \in \uconcr(P) \cap \Sce(s)$.
  D'après la \defref{lastprocs},
  et en s'inspirant de la démonstration du \thmref{approxinf},
  il existe $ps \in \derprocs(P)$ tel que $(s \play \delta) \subset \ctx' \Cap ps$.
  Ainsi, si $\muconcr_{\ctx' \Cap ps}(\w') \neq \emptyset$,
  alors il existe un scénario $\delta' \in \muconcr_{\ctx' \Cap ps}(\w')$ tel que
  $\delta' \in \Sce(s \play \delta)$.
  Par conséquent, $\delta \concat \delta'$ est un scénario jouable dans $s$.
  Donc, pour tout $s \in \PHl, s \subset \ctx$, il existe un scénario concrétisant $\w$.
  D'où : $\uconcr(\w) \neq \emptyset$.
\end{proof}

\todoplustard{Appliquer au modèle de segmentation métazoaire}

