% Conclusion

\chapter{Conclusion et perspectives}
\chaplabel{conclusion}

\chapeau{
  Nous revenons dans ce chapitre sur les apports de la présente thèse,
  en insistant notamment sur la façon dont nous avons répondu aux deux problématiques
  principales proposées initialement :
  renforcer la puissance de la modélisation d'une part,
  en permettant notamment l'intégration de données temporelles et de synchronisation,
  et les capacités d'analyse d'autre part,
  particulièrement en ce qui concerne les modèles de grande taille.
  Plusieurs nouvelles perspectives scientifiques sont par ailleurs évoquées.
}



L'étude des réseaux de régulation biologique peut s'effectuer sous de nombreuses formes,
allant des systèmes d'équations différentielles à la programmation logique,
en passant par des modèles probabilistes.
L'utilisation de modèles discrets permet une abstraction puissante de la complexité inhérente
de ces systèmes, en conservant certaines propriétés intéressantes.
On peut considérer que l'utilisation de tels modèles a été initiée par
\citefullname{kauffman69}{Stuart A.}, plus tard suivi par \citefullname{Thomas73}{René}
qui y ajoute une dynamique asynchrone
et par \citefullname{Snoussi89}{El Houssine}
qui propose la notion de paramètres discrets pour définir totalement la dynamique.
Ces modèles discrets ont été abondamment étudiés,
comme nous l'avons vu au \chapref{etatdelart} :
beaucoup de résultats permettent d'optimiser l'étude de leur dynamique,
mais aussi de déduire la présence de certains comportements
simplement à l'aide de la structure du modèle.

Cependant, la simplicité apparente de leur définition masque deux écueils importants.
Tout d'abord, ces modèles possèdent une dynamique souvent très complexe,
l'espace des états ayant en effet une taille exponentielle dans la taille du modèle.
Les méthodes de compression ou de réduction des modèles ne permettent pas toujours
de restreindre suffisamment cet espace des états pour en permettre une étude exhaustive.
L'analyse dynamique des modèles de régulation reste donc souvent cantonnée
à des modèles de petite ou moyenne taille ---~rarement plus d'une vingtaine de composants.
De plus, certains paramètres peuvent être difficiles, voire impossibles à intégrer dans le modèle.
En effet, le modèle de Thomas ne permet pas de faire figurer des notions de précédence,
de préemption ou de synchronisme entre les différentes interactions.
De même, son étude requiert d'avoir correctement défini tous les paramètres discrets
propres au système étudié, paramètres qui sont parfois inconnus à priori,
alors que les comportements attendus le sont (même partiellement).
Pour finir, il peut s'avérer difficile de raffiner un modèle lorsqu'il n'est pas
possible d'étudier l'impact de chaque modification
à cause des difficultés inhérentes à l'analyse des grands modèles.

Les Frappes de Processus, introduites par \citeasnoun*{PMR10-TCSB},
offrent un outil alternatif puissant de représentation et d'analyse des
réseaux de régulation biologique.
Le formalisme proposé, totalement asynchrone, est basé sur une représentation
atomique des interactions entre composants,
où le changement d'état d'un composant ne peut être déclenché que par l'état d'au plus
un autre composant.
Cette forme particulière a notamment permis le développement de méthodes efficaces
d'analyse de la dynamique, qui permettent notamment d'en réduire la complexité \cite{PMR12-MSCS}.
Cependant, les Frappes de Processus ne permettent pas de représenter fidèlement
un réseau de régulation biologique sous la forme d'un réseau discret asynchrone.
De plus, certaines contraintes de modélisation ne peuvent pas toujours être introduites
efficacement dans ces modèles standards de Frappes de Processus.

Le travail proposé dans cette thèse avait pour ambition de contribuer à répondre
à cette double problématique
de modélisation et d'analyse des grands modèles de réseaux de régulation biologique,
en partant du formalisme Frappes de Processus précédemment proposé,
et des différents travaux correspondants.
Nous proposons un bref rappel des résultats obtenus sur ces deux points à la \secref{resultats}
et nous ouvrons un certain nombre de perspectives dans la lignée de nos travaux
à la \secref{perspectives}.



\section{Retour sur les résultats de cette thèse}
\seclabel{resultats}

Nous mettons en perspective dans la suite l'apport de nos travaux aux deux questions
que sont la modélisation et l'analyse d'un système.

\subsubsection*{Modélisation}

La représentation d'un système réel (ou fictif) sous forme de modèle est fondamentale
et doit être adaptée aux besoins de l'analyse.

\myskip

Nous avons proposé au \chapref{sem} plusieurs représentations alternatives
permettant d'enrichir la modélisation d'un système en Frappes de Processus.

À la \secref{php}, nous avons défini des \bemph{classes de priorités} sur les actions
d'un modèle de Frappes de Processus.
Ces classes de priorités permettent de spécifier des contraintes de préemption globales
entre plusieurs ensembles d'actions ---~chaque action ne pouvant être jouée que si
aucune action plus prioritaire ne l'est.
Nous avons montré comment ces classes de priorités permettent de représenter de façon
discrète des \bemph{contraintes temporelles} (éventuellement continues).
Pour cela, nous avons pris comme exemple le cas de modèles de Frappes de Processus intégrant
des paramètres stochastiques pour chaque action,
paramètres qui peuvent se traduire en intervalles de tirs,
et nous avons montré comment modéliser une dynamique semblable grâce à des classes de priorités.

L'introduction d'\bemph{arcs neutralisants} à la \secref{phan} a en revanche permis de définir
des contraintes de préemption locales entre les actions.
Leur plus grande granularité permet d'intégrer plus finement certaines informations
relatives entre les actions, comme des contraintes de tir ou des durées de réaction.
Cette granularité est notamment nécessaire pour éviter des relations de préemption
non désirées qui peuvent bloquer des parties du modèle,
par exemple en abstrayant certaines phénomènes d'accumulation,
provoquant des \bemph{comportements Zénon}.

Enfin, les \bemph{actions plurielles} définies à la \secref{phm} introduisent de la
synchronicité entre les actions.
Celle-ci est nécessaire pour modéliser certains phénomènes comme des réactions biochimiques,
qui nécessitent de représenter la formation de plusieurs produits ou la consommation de plusieurs
réactifs, voire les deux simultanément.
Ce formalisme offre donc à fortiori la possibilité de représenter du \bemph{synchronisme}
au niveau des
pré-conditions de tir des actions, et par conséquent permet de compresser les modèles
à l'aide d'une notion naturelle de coopération.

\myskip

Nous avons de plus délimité au \chapref{phcanonique} une sous-classe des Frappes de Processus
avec classes de priorités, dites \bemph{Frappes de Processus canoniques},
dont les modèles ne comportent que deux classes de priorités et possèdent une forme particulière
pour les actions très prioritaires.
Cette classe de modèles présente un intérêt particulier pour la représentation,
car elle \bemph{corrige les problèmes de décalage temporel} présents
dans les coopérations modélisées par des Frappes de Processus standards.
Nous avons de plus mis en évidence le fait qu'elle est suffisante pour représenter tous
les autres formalismes de Frappes de Processus alternatifs rappelés ci-dessus ;
nous avons démontré par ailleurs grâce à cela que tous ces formalismes possèdent
une \bemph{expressivité équivalente},
et nous utilisons les Frappes de Processus canoniques pour proposer
les traductions adéquates.

\myskip

Enfin, le \chapref{expressivite} a été l'occasion de comparer l'expressivité des
Frappes de Processus avec d'autres formalismes courants.

Nous avons notamment démontré à la \secref{trad-rda} que les Frappes de Processus canoniques
permettent de représenter fidèlement et naturellement les \bemph{réseaux discrets asynchrones}.
Cette équivalence est particulièrement utile pour pouvoir étudier ce type de modèles
efficacement à l'aide des outils détaillés plus bas.

Nous nous sommes aussi penché sur le cas du \bemph{modèle de Thomas} à la \secref{trad-thomas},
qui présente un intérêt supplémentaire du fait de sa dynamique unitaire
et de ses paramètres discrets.
Nous avons donné les conditions sous lesquelles les Frappes de Processus canoniques
peuvent être traduites en modèle de Thomas de façon plus ou moins complète,
ce qui se manifeste par des arcs non-signés ou des paramètres impossibles à inférer.
Cependant, en cas de réponse partielle,
un ensemble de modèles compatibles peut être énuméré,
et le modèle d'origine peut être itérativement raffiné afin d'obtenir
le modèle de Thomas attendu.

% Liens avec plusieurs autres formalismes :
%   vers le modèle de Thomas :
%     formalisme historique et mieux connu, plus global (influences)
%     passage « atomique -> influences globales » difficile
%     inférence de toutes les influences et paramètres + énumération si incomplet
%     toujours compatible avec la dynamique du PH
%   depuis les réseaux discrets asynchrones :
%     équivalence du formalisme
%     permet l'étude des RDA
%   Biocham :
%     représentation possible
%   RdP :
%     formalisme répandu et analyses puissantes

\myskip

Ces différents formalismes de Frappes de Processus
peuvent directement bénéficier aux biologistes désirant
vérifier le bon fonctionnement d'un réseau discret asynchrone ou d'un modèle de Thomas,
ou effectuer des prédictions sur ces modèles.
Ils peuvent naturellement tout aussi bien être utilisés à d'autres fins,
comme l'inférence ou la correction de modèles, à l'aide des différentes équivalences proposées.
Cependant, il est aussi envisageable pour des informaticiens ou bioinformaticiens
d'en faire usage dans un cadre
plus large, à la condition de pouvoir représenter le système informatique souhaité
sous la forme de Frappes de Processus.



\subsubsection*{Analyse}

Un formalisme adapté à la représentation d'une certaine classe de systèmes présente cependant
un intérêt limité s'il n'est pas accompagné de méthodes d'analyse efficaces.

C'est pourquoi, en sus de la définition des Frappes de Processus canoniques,
nous avons proposé au \chapref{as} une méthode efficace
d'\bemph{analyse statique pour l'atteignabilité},
qui peut cependant aussi être appliquée aux autres formalismes
de Frappes de Processus, moyennant les traductions adéquates.
Elle s'inspire des résultats de \citeasnoun*{PMR12-MSCS}
qui avaient précédemment proposé une telle analyse sur les Frappes de Processus standards.
De la même façon, notre analyse
s'appuie sur une sous-approximation de la dynamique du modèle étudié,
qui exploite la forme particulière des actions et de la définition
des Frappes de Processus canoniques.
L'apport de notre travail est donc, grâce aux traductions proposées au \chapref{expressivite},
de \bemph{pouvoir répondre de façon formelle à des questions d'atteignabilité locale
sur des modèles de réseaux discrets asynchrones}, ce qui était impossible
en utilisant les Frappes de Processus standards.
De plus, elle reste efficace
du fait de sa complexité polynomiale dans la taille du modèle étudié,
évitant ainsi l'explosion combinatoire inhérente au calcul exhaustif de la dynamique.
À l'instar de la méthode d'origine, elle permet de vérifier des questions
d'atteignabilités individuelles successives ;
les question d'\bemph{atteignabilités simultanées d'états locaux}
peuvent cependant aussi être traitées,
ce qui est propre à notre approche.

\myskip

Enfin, le \chapref{applications} nous a permis d'illustrer l'étendue
des différents résultats proposés dans cette thèse
par quelques applications à des modèles biologiques.

L'étude de comportements peut être abordée par des formalismes différents.
Les différents modèles de Frappes de Processus traduits en modèle de Thomas
à la \secref{appli-thomas} ont illustré cela :
chaque modèle était issu d'un \bemph{raffinement} différent d'un même système,
et la traduction en modèle de Thomas a permis de déterminer
les parties de modèle dont le comportement est déterministe.
De plus, l'analyse efficace de la dynamique sur les Frappes de Processus
peut s'avérer utile pour effectuer des raffinements successifs d'un même modèle
avant de le traduire vers le formalisme souhaité.
La conservation de la dynamique %et de certaines bonnes propriétés sur la structure
pour toutes les traductions présentées
permet en effet de généraliser les résultats obtenus sur un modèle à tous les formalismes
dans lesquels il peut être traduit.

Nous avons appliqué à la \secref{appli-as-tcrsig} la méthode d'analyse des
Frappes de Processus canoniques résumée ci-dessus
à un réseau discret asynchrone de 94 composants
et pouvant donc être considéré comme de grande taille.
Cette application a permis de répondre avec succès à plusieurs questions d'atteignabilité
avec \bemph{des temps de calcul de l'ordre de quelques centièmes de seconde},
ce qui à notre connaissance est inégalé.

\myskip

Les possibilités offertes par l'utilisation conjointe des traductions présentées
dans cette thèse avec les méthodes d'analyse développées
rendent accessible l'étude de modèles de grande taille, voire de très grande taille.
Ces capacités d'analyse devraient notamment être suffisantes pour pouvoir analyser
en un temps raisonnable des propriétés dynamiques sur des bases de données métaboliques
comme PID \cite{schaefer09pid}
ou des regroupements de telles bases de données comme hiPathDB \cite{yu12hipathdb}.



\section{Perspectives de travail}
\seclabel{perspectives}

Les travaux présentés au cours de cette thèse ouvrent de nombreuses pistes de travail
permettant d'étendre et d'exploiter les résultats proposés.

% Application à des très grands réseaux (> 1000 composants)
% 
% Raffiner les traductions pour les grands modèles / pour prendre en compte de nouvelles sémantiques



\subsubsection*{Raffinement de l'analyse statique}

L'analyse d'atteignabilité présentée au \chapref{as}
repose sur une approximation de la dynamique ;
cette approximation fait chuter la complexité de l'analyse,
au risque cependant de ne pas pouvoir conclure dans certains cas.
Si le besoin s'en fait ressentir sur certains modèles,
il pourrait naturellement être intéressant de \bemph{raffiner cette analyse statique}
tant au niveau de la sous-approximation que nous proposons qu'au niveau
de la sur-approximation définie par \citeasnoun{PMR12-MSCS},
afin de pouvoir conclure dans un plus grand nombre de cas.
Un tel raffinement serait basé soit sur la construction du graphe de causalité locale,
soit sur les conditions nécessaires et suffisantes qui l'exploitent.
Des \bemph{méthodes dérivées} peuvent aussi être envisagées,
permettant par exemple de guider une analyse plus exhaustive,
comme l'analyse des scénarios permis par le graphe de causalité locale.

De même, les méthodes d'analyse statique en question pourraient être adaptées
à d'autres formalismes que les Frappes de Processus canoniques.
Leur application directe aux
Frappes de Processus avec classes de priorités ou aux Frappes de Processus avec actions
plurielles, par exemple,
permettrait d'éviter la traduction coûteuse d'un modèle vers les Frappes de Processus canoniques.

Par ailleurs, l'analyse d'atteignabilité ne permet actuellement de vérifier que des propriétés
qui s'expriment en logique CTL sous la forme $\mathbf{EF}(P)$ ;
il pourrait être intéressant de l'enrichir avec de \bemph{nouveaux types de propriétés} comme
$\mathbf{AF}(P)$, assurant par exemple qu'un état local est toujours atteignable,
ou encore $(P\:\mathbf{EU}\:Q)$, qui permettrait d'observer des commutations,
c'est-à-dire des bifurcations de la dynamique impossibles à inverser.
Par exemple, la formule suivante décrit un tel comportement,
en exprimant le fait qu'il existe un chemin tel que les deux propriétés $P$ et $Q$ sont
initialement atteignables, mais qu'un phénomène de commutation force l'atteignabilité de $Q$
tout en empêchant celle de $P$ :
\[(\mathbf{EF}(P) \wedge \mathbf{EF}(Q))\:\mathbf{EU}\:(\neg\mathbf{EF}(P) \wedge \mathbf{AF}(Q))\]
L'utilisation de logiques intégrant des \bemph{compteurs},
ou de \bemph{logiques non-régulières},
augmenterait aussi les capacités d'analyse.

% Raffinement de l'analyse statique :
%   répondre OUI sur de plus nombreux cas
%   répondre NON sur de plus nombreux cas
%   nouvelles questions (cf. logiques temporelles) tq AF ou U
%   utilisation de compteurs, de logiques non-régulières
% Moyens :
%   revoir la construction du GLC ou la condition nécessaire/suffisante
%   analyse exhaustive du GLC
%   nouvelles approches statiques (directement adaptées aux actions plurielles ?)
%   approches exhaustives (ASP ?)



\subsubsection*{Extensions de la modélisation et de l'analyse}

Dans un cadre plus général, au niveau de la modélisation,
la représentation de certains systèmes biologiques peut nécessiter de nouveaux ajouts au niveau du
formalisme,
comme des \bemph{classes de priorités dynamiques} ou des \bemph{actions gardées}.
À l'instar des réseaux d'automates, l'ajout de telles notions crée une forte complexité,
mais l'approche par interprétation abstraite pourrait à nouveau être plus efficace
qu'une recherche naïvement exhaustive.

En ce qui concerne l'analyse,
de nombreux autres résultats peuvent être recherchés sur un modèle de Frappes
de Processus, comme la \bemph{prévision d'oscillations}, la recherche
de \bemph{domaines pièges} et d'\bemph{attracteurs}, etc.
Il est difficile de prévoir quels types d'analyse ces résultats peuvent requérir.
Le cas le plus souhaitable est une simple observation de la structure du modèle
(à l'instar des conjectures de Thomas) impliquant une complexité polynomiale, voire linéaire.
De telles observations peuvent porter sur la forme et l'agencement des actions,
sur la recherche d'éléments comme les $(n-k)$\nbd cliques dans le graphe sans-frappe,
etc.
Le cas le plus coûteux en temps de calcul, en revanche, serait la nécessité d'effectuer
une \bemph{analyse exhaustive} de la dynamique pour parvenir à un résultat.
De telles analyses peuvent cependant bénéficier d'avancées notables
comme l'utilisation d'un paradigme de programmation logique
telle que la programmation par ensembles de réponse ou \textit{Answer Set Programming}
\citeaffixed{benabdallah14}{voir p.~ex.}.
Il est de plus possible de trouver des résultats intermédiaires
permettant de guider la recherche et de trouver plus rapidement une solution
(comme l'utilisation du graphe de causalité locale évoquée plus haut).



\subsubsection*{Autres pistes d'application des capacités d'analyse}

L'\bemph{inférence et la correction} de modèles pour les systèmes biologiques
suscitent un engouement justifié,
alors que les données d'expression deviennent toujours plus accessibles.
L'utilisation de \bemph{données de puce à ADN}
donne en effet accès une quantité inégalée de données d'expression
et offre la possibilité de corriger des modèles existants.
L'automatisation de ce procédé présente alors un intérêt certain,
notamment pour les systèmes complexes, qui comportent beaucoup de mesures
et qui sont représentés par des modèles de grande taille
\citeaffixed{Guziolowski12072013}{voir p.~ex.}.
Les Frappes de Processus pourraient être avantageusement utilisées pour ce type de problème,
étant données leur définition atomique et les analyses efficaces de la dynamique
dont elles font l'objet.

Dans le même ordre d'idées,
la question de la \bemph{résilience d'un système biologique},
c'est-à-dire sa capacité et sa propension à revenir vers un état acceptable après
une perturbation, peut aussi être soulevée
et abordée de plusieurs manières grâce aux Frappes de Processus.
% Tout d'abord, la validité d'un modèle peut être testée sous l'angle de sa robustesse,
% à savoir sa capacité à supporter des comportements biologiques supplémentaires.
Tout d'abord, l'ajout de \bemph{comportements incontrôlables} \cite{schwind13resilience}
poserait de nouvelles questions :
le modèle court-il le risque d'être forcé à rentrer dans un état non acceptable ?
Est-il capable de revenir à tout moment dans un état acceptable ?
Et à quel prix ?
Répondre à ces questions pourra nécessiter d'élargir le cadre des propriétés vérifiables
de manière efficace.
D'un autre point de vue, il serait intéressant d'observer les conséquences sur la dynamique
de \bemph{perturbations ponctuelles aléatoires}, comme l'ajout ou la suppression d'actions
dans un modèle.
Cette approche pourrait bénéficier d'autres moyens détournés comme le calcul des
\textit{cut-sets} afin de détecter des parties sensibles des modèles,
en révélant les parties du modèle les plus sensibles aux perturbations.


% Création/inférence/adbuction de modèles :
%   selon un modèle de départ (requis) et des résultats expérimentaux (connaissance),
%     adapter le requis à la connaissance + minimalité
% Moyens :
%   
% Études de résilience :
%   en fonction de perturbations (aléatoires ?) mesurer la capacité de résilience du modèle
% Moyens :
%   analyse statique pour comparer les comportements
%   connaissance des processus clefs ou cut-sets pour connaître les « goulots de résilience »
% 
% Études dynamiques :
%   détecter d'autres comportements (oscillations, etc)
% Méthodes :
%   analyse statique ? pas nécessairement efficace
%   déceler les motifs menant à ces comportements (structure des actions)
%   déceler des propriétés permettant ces comportements ($n-k$) cliques, etc.
%   méthodes exhaustives (ASP)
