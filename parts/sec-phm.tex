% Enrichissement avec simultanéité des actions :
\section{Frappes de Processus avec actions plurielles}
\seclabel{phm}

Il peut être intéressant de vouloir représenter un système au niveau de ses réactions biochimiques,
c'est-à-dire des réactions entre les différents composants présents.
De telles réactions peuvent avoir différentes formes
(transformation, complexation, dissociation...),
et il est fréquent qu'elles fassent intervenir plusieurs réactifs et plusieurs produits.
Biocham \cite{fages2004modelling}
propose par exemple de modéliser un tel système de réactions biochimiques à l'aide
d'un ensemble de règles de réaction de la forme :
$X \xrightarrow{Y} Z$,
ou encore :
$X + Y \rightarrow Y + Z$,
où $X$ est un ensemble de réactifs, $Y$ un ensemble de catalyseurs et $Z$ un ensemble de produits.

\myskip

Les \emph{Frappes de Processus avec actions plurielles} permettent de représenter de telles réactions
mettant en jeu un nombre arbitraire de réactifs, de produits et de catalyseurs.
Ainsi, une réaction de la forme : $X \xrightarrow{Y} Z$
peut être représentée à l'aide de l'action $\PHfrappemult{A}{B}$
où $A$ et $B$ sont deux ensembles des processus,
$A$ regroupant tous les processus représentant les composants nécessaires à initier la réaction,
et $B$ tous les processus qui ont évolué pendant la réaction.
%$A$ regroupant tous les processus permettant de représenter les composants de $X \cup Y$,
%et $B$ ceux de $Z$.
Une telle action peut donc être jouée dans un état contenant tous les processus de $A$
%(c'est-à-dire les réactifs et les catalyseurs)
et fait évoluer celui-ci vers un état contenant tous les processus de $B$, % (les produits),
les autres processus restant inchangés. % (car n'intervenant pas ou en tant que catalyseurs).
Cela implique toutefois que pour tout processus de $B$, il existe un autre processus de la même
sorte dans $A$.
Les Frappes de Processus avec actions plurielles permettent donc de représenter
un nombre arbitraire de bonds simultanés
---~autrement dit, de changements simultanés de processus actifs~---
déclenchés par un nombre arbitraire de prérequis
---~sous la forme de processus actifs dans l'état courant.
Les \figref{metazoan-phm,simple-phm},
plus loin dans cette section,
représentent des exemples de Frappes de Processus avec actions plurielles.

Un parallèle peut être tracé d'une part entre $A$ et l'ensemble des réactifs et catalyseurs,
et d'autre part entre $B$ et l'ensemble des produits.
Cependant, la modélisation par Frappes de Processus avec actions plurielles
nécessite aussi de donner explicitement les composants qui sont absents.
Par exemple, une réaction de complexation du type : $x + y \rightarrow c$ %x\!\!-\!\!y$
sera représentée en Frappes de Processus avec actions plurielles
à l'aide de trois sortes $x$, $y$ et $c$ contenant chacune deux processus et
représentant respectivement les deux réactifs et le complexe produit,
et par l'action $\PHfrappemults{x_1, y_1, c_0}{x_0, y_0, c_1}$.
%si on ne prend pas en compte la disparition des réactifs.
Autrement dit, il est nécessaire de décomposer chaque élément en fonction de sa présence
($x_1$ et $y_1$)
ou de son absence ($c_0$) au début comme à la fin de la réaction,
et pas uniquement d'indiquer les composants présents en tant que réactifs ou produits.

On note ainsi qu'une réaction de la forme $\PHfrappemult{\{ a_0, b_0, c_0 \}}{\{ a_1, b_1 \}}$
ne peut pas être jouée
si $a$ ou $b$ est déjà au niveau $1$,
comme c'est le cas par exemple dans l'état $\etat{a_1, b_0, c_0}$.
%si $a_1$ et $b_0$ sont actifs.
% l'une des deux sortes entrant en jeu, $a$ et $b$, est déjà au niveau $1$,
% même si l'autre est au niveau $0$.
Un tel comportement a du sens lorsque les différents processus d'une sorte
($a_0$ et $a_1$, par exemple)
représentent différents états d'une même molécule :
la réaction ne peut alors pas être jouée pour des raisons de stœchiométrie.
Cependant, si ces différents processus représentent plutôt des niveaux de concentration
($a_1$ représentant par exemple un niveau de concentration de la molécule $a$ plus élevé que $a_0$),
cette restriction n'a plus de sens car une plus forte concentration d'une des entités
ne devrait pas empêcher la réaction d'avoir lieu et de produire la seconde entité.
Cela peut néanmoins être corrigé en ajoutant les actions
$\PHfrappemult{\{ a_1, b_0, c_0 \}}{\{ a_1, b_1 \}}$ et
$\PHfrappemult{\{ a_0, b_1, c_0 \}}{\{ a_1, b_1 \}}$,
ou encore en séparant la production de $a_1$ et de $b_1$ en deux actions (ou ensemble d'actions)
distinctes.

\myskip

Cette forme des Frappes de Processus peut être aisément représentée à l'aide d'un réseau
d'automates synchronisés, car chaque sorte possède un rôle similaire à celui d'un automate
et chaque action pouvant être remplacée par un ensemble de transitions étiquetées
avec le même libellé.
La \vsecref{phm2an} formalise cette équivalence.
On peut aussi la représenter à l'aide de Frappes de Processus avec 4 classes de priorités,
comme détaillé à la \secref{phm2php} ;
cependant, cette représentation a l'inconvénient d'être moins claire car faisant intervenir
un nombre important d'actions et de sortes supplémentaires.



\subsection{Définition}

La \defref{phm} formalise la notion de Frappes de Processus avec actions plurielles,
en accord avec la discussion informelle ci-dessus :
une action plurielle est constituée de deux ensembles de processus de sortes distinctes,
qui représentent l'ensemble des frappeurs et celui des bonds.
Cela permet de formaliser le déclenchement d'une action par une synchronisation exacte
entre un nombre arbitraire de frappeurs,
et une synchronisation exacte entre plusieurs bonds lorsque l'action est jouée.
% Nous étendons aussi la notion de recouvrement à la \defref{recouvrementps}
% au cas du recouvrement d'un sous-état désordonné par un autre,
% ce qui permet, conjointement à la définition de
Nous définissons de plus l'opérateur de jouabilité
des Frappes de Processus avec actions plurielles
à la \defref{fopphm},
afin de formaliser la dynamique de ce type de modèles
à l'aide de la sémantique donnée à la \defref[vref]{play}.

\begin{definition}[Frappes de Processus avec actions plurielles]
\deflabel{phm}
  Les \emph{Frappes de Processus avec actions plurielles} sont définies
  par un triplet $\PH = (\PHs; \PHl; \PHh)$, où :
  \begin{itemize}
    \item $\PHs \DEF \{a, b, \dots\}$ est l'ensemble fini et dénombrable des \emph{sortes} ;
    \item $\PHl \DEF \bigtimes{a \in \PHs} \PHl_a$ est l'ensemble fini des \emph{états},
      où $\PHl_a = \{a_0, \ldots, a_{l_a}\}$ est l'ensemble fini et dénombrable
      des \emph{processus} de la sorte $a \in \PHs$ et $l_a \in \sN^*$,
      chaque processus appartenant à une unique sorte :
      $\forall (a_i; b_j) \in \PHl_a \times \PHl_b, a \neq b \Rightarrow a_i \neq b_j$ ;
    \item $\PHh \DEF \{\PHfrappemult{A}{B} \mid A, B \in \PHsublset \setminus \emptyset \wedge
      \forall q \in B, \exists p \in A, (p \neq q \wedge \PHsort(p) = \PHsort(q)) \}$
      est l'ensemble fini des \emph{actions}.
  \end{itemize}
\end{definition}

\noindent
Pour toute action $h = \PHfrappemult{A}{B} \in \PHh$,
$A$ est appelé le \emph{frappeur} et $B$ le \emph{bond} de $h$,
et on note : $\hitter{h} = A$, $\bounce{h} = B$.
On appelle par ailleurs \emph{frappeurs invariants} les processus de $A$
qui ne sont pas désactivés par l'action plurielle
(autrement dit, tels qu'il n'existe pas d'autre processus de la même sorte dans $B$)
et on appelle \emph{cibles} les autres processus de $A$
(pour lesquels il existe un autre processus de la même sorte dans $B$).
On note en conséquence :
$\target{h} = \{ p \in A \mid \exists q \in B, \PHsort(p) = \PHsort(q) \}$,
et $\invariant{h} = \{ p \in A \mid \sort{p} \notin \sortes{B} \}$.

\begin{definition}[Opérateur de jouabilité ($\Fopsymbol_\Fopphmsubsymbol : \PHh \rightarrow \F$)]
\deflabel{fopphm}
  L'opérateur de jouabilité des Frappes de Processus avec actions plurielles est défini par :
  \[\forall h \in \PHh, \Fopphm{h} \equiv \bigwedge_{p \in \hitter{h}} p \enspace.\]
\end{definition}

%\myskip

\subsubsection*{Modélisation des coopérations}

Nous notons que les Frappes de Processus avec actions plurielles proposent une alternative
très efficace à la représentation des coopérations entre actions.
Une coopération est représentée par une sorte coopérative en Frappes de Processus
standards,
qui peut présenter un défaut de modélisation provoquant des décalages
temporels, voire l'apparition de «~faux états~»,
comme expliqué à la \vsecref{sc}.
% par un mécanisme développé
% comme expliqué à la \vsecref{sc}.
% Or une telle sorte coopérative présente un défaut de modélisation qui provoque des décalages
% temporels, voire l'apparition de «~faux états~», par un mécanisme développé
% \vexpageref{metazoan-ph-decalagetemporel}.
De plus, cela nécessite l'utilisation d'un grand nombre d'actions pour assurer la mise à jour
du processus actif de la sorte coopérative.

\label{ph2phm}
Les Frappes de Processus avec classes de priorités permettent de pallier ce défaut,
comme cela sera expliqué à la \vsecref{phcanonique}.
Cependant, cette solution ne règle pas le problème du grand nombre d'actions
utilisée pour la mise à jour de la sorte coopérative,
qui peuvent nuire à la lecture du modèle.
Les Frappes de Processus avec actions plurielles peuvent alors se révéler utiles
pour la représentation des processus de coopération.
En effet, il suffit d'une action plurielle pour définir une coopération entre processus,
à condition de lui attribuer les frappeurs adéquats.
Ainsi, on peut obtenir une coopération depuis un modèle de Frappes de Processus standards
en remplaçant :
\begin{itemize}
  \item toute action de la forme $\PHfrappe{a_i}{b_j}{b_k}$,
    où $a$ n'est pas une sorte coopérative, par
    une action plurielle $\PHfrappemults{a_i, b_j}{b_k}$ ;
  \item toute action de la forme $\PHfrappe{f_{\mysigma}}{b_j}{b_k}$,
    où $f$ est une sorte coopérative, par
    une action plurielle $\PHfrappemult{(\toset{\mysigma} \cup \{ b_j \})}{b_k}$.
\end{itemize}
Le modèle obtenu corrige ainsi le problème de décalage temporel propre aux sortes
coopératives, tout en permettant une meilleur lisibilité
car chaque chaque sorte coopérative et toutes ses actions sont remplacées par
une seule action plurielle.

\begin{example}
  On peut prendre en exemple 
  la sorte coopérative $fc$ des Frappes de Processus standards de la \figref{metazoan-ph}.
  Le modèle corrigé en Frappes de Processus avec actions plurielles
  est donné à la \figref{metazoan-phm},
  et contient notamment l'ensemble d'actions plurielles suivant :
  \begin{align*}
    \PHh = \{\qquad
      \PHfrappemults{c_1, a_1}{a_0}\quad, && \PHfrappemults{c_0, f_1, a_0}{a_1}\quad,& \\
      \PHfrappemults{c_1}{c_0}\quad, && \PHfrappemults{f_1, c_0}{c_1}\quad,& \\
      \PHfrappemults{f_0, c_1}{c_0}\quad,&& \PHfrappemults{f_1}{f_0}\quad\; &
    \qquad\}
  \end{align*}
  Il a été obtenu en reprenant toutes les actions dont le frappeur et la cible
  n'étaient pas des processus de sortes coopératives,
  et en remplaçant la sorte coopérative $fc$ et ses actions par l'action plurielle :
  $\PHfrappemults{c_0, f_1, a_0}{a_1}$,
  selon le procédé décrit \vpageref[ci-dessus]{ph2phm}.
  
  \begin{figure}[ht]
  \begin{center}
  \begin{tikzpicture}[apdotsimple/.style={apdot}]
    \TSort{(0,4)}{c}{2}{l}
    \TSort{(1,0)}{f}{2}{l}
    \TSort{(7,4)}{a}{2}{r}
    
    \TAction{c_1}{a_1.west}{a_0.north west}{}{right}
    \TAction{f_1}{c_0.west}{c_1.south west}{bend left=30, in=90}{left}
    \TAction{f_0.east}{c_1.south east}{c_0.north east}{bend right=60, in=-140}{left}
    
    \TActionPlur{}{c_1.west}{c_0.north west}{}{-1,5.5}{right}
    \TActionPlur{}{f_1.south west}{f_0.north west}{}{0,0.5}{right}
    \TActionPlur{f_1, c_0}{a_0.west}{a_1.south west}{}{3.5,2.5}{left}
    
    \TState{f_1, a_0, c_0}
  \end{tikzpicture}
  \caption{\figlabel{metazoan-phm}%
    Exemple de Frappes de Processus avec actions plurielles.
    Chaque action plurielle est représentée par plusieurs arcs et plusieurs couples de flèches,
    partant tous d'un même point.
    Les arcs indiquent les frappeurs invariants
    tandis que les flèches pleines indiquent les cibles,
    et sont suivies par une flèche en pointillés pour indiquer le bond correspondant.
    Par exemple, l'action plurielle $\PHfrappemults{c_1, a_1}{a_0}$
    est représentée par un point d'où partent un arc vers $c_1$
    et un couple de flèches vers $a_1$ puis vers $a_0$.
    De même, l'action plurielle $\PHfrappemults{c_1}{c_0}$
    n'est représentée que par un couple de flèches vers $c_1$ puis vers $c_0$,
    car elle ne possède pas de frappeur invariant.
  }
  \end{center}
  \end{figure}
\end{example}



\subsection{Traduction vers les Frappes de Processus avec 4 classes de priorités}
\seclabel{phm2php}

Nous proposons dans cette section une traduction des Frappes de Processus
avec actions plurielles en Frappes de Processus avec 4 classes de priorités (\defref{phm2php})
et nous montrons que les modèles obtenus de cette façon sont faiblement bisimilaires
(\thmref{phmbisimphp}).
Les \figref{simple-phm,simple-php-trad} illustrent cette traduction.

La traduction proposée permet d'obtenir un modèle de
Frappes de Processus pseudo-canoniques avec 4 classes de priorités,
telles que définies à la \vsecref{phcanonique}.
De façon informelle, ce modèle comporte 4 classes de priorités
et repose sur l'utilisation de deux types de sortes particulières :
\begin{itemize}
  \item des sortes coopératives pour vérifier la présence de tous les processus du frappeur,
  \item des \emph{sortes de réaction} permettant de modéliser le déclenchement d'une \emph{réaction},
    ou son arrêt.
\end{itemize}
Une réaction modélise le fait qu'un ensemble d'actions (standards) est en train
de simuler le jeu d'une action plurielle.
À toute action plurielle $h$ du modèle d'origine correspond une sorte coopérative $\scf{h}$
entre les sortes des processus de $A$
et une sorte de réaction $\sr{h}$ dans le modèle résultant de cette traduction.
La sorte coopérative comporte notamment un processus $\scf{h}_\mypi$ qui représente le sous-état
où tous les processus de $A$ sont présents ; une action de priorité 4 de la forme
$\PHhit{\scf{h}_\mypi}{\sr{h}_0}{\sr{h}_1}$ permet d'activer la sorte de réaction.
Une auto-action de priorité 3 de la forme $\PHhit{\sr{h}_1}{\sr{h}_1}{\sr{h}_0}$
permet de plus la désactivation de la sorte de réaction
une fois que toutes les actions de priorité 2 auront été jouées.
Les actions de priorité 2 ont la forme $\PHhit{\sr{h}_1}{b_j}{b_k}$
avec $b_j \in A$ et $b_k \in B$,
ce qui permet d'effectuer tous les bonds nécessaires à l'activation des processus de $B$.
Enfin, les sortes coopératives sont toutes mises à jour par des actions de priorité 1,
afin d'éviter les problèmes d'entrelacement
et de correspondre à la définition de Frappes de Processus pseudo-canoniques
telle que donnée par la \vdefref{phpseudocanonique}.
L'agencement de ces classes de priorités permet ainsi de simuler des actions plurielles
tout en empêchant l'entrelacement entre réactions
---~car deux réactions ayant lieu en même temps pourraient potentiellement
amener le système dans un état normalement inaccessible.

\begin{definition}
\deflabel{phm2php}
  Soient $\PH = (\PHs; \PHl; \PHh)$ des Frappes de Processus avec actions plurielles,
  et $\phmtophp = (\PHs'; \PHl'; \PHh'^{\langle 4 \rangle})$
  leur traduction en Frappes de Processus avec $4$ classes de priorités, où :
  \begin{itemize}
    \item $\PHs' = \PHs \cup \{ \sr{h} \mid h \in \PHh \} \cup \{ \scf{h} \mid h \in \PHh \}$ ;
    \item $\PHl' = \PHl \times \left( \bigtimes{h \in \PHh} \PHl_{\sr{h}} \right)
      \times \left( \bigtimes{h \in \PHh} \PHl_{\scf{h}} \right)$, où :
      \begin{itemize}
        \item $\forall h \in \PHh, \PHl_{\sr{h}} = \{ \sr{h}_0 , \sr{h}_1 \}$,
        \item $\forall h \in \PHh, \PHl_{\scf{h}} = \{ \scf{h}_\mysigma \mid
          \mysigma \in \PHsubl_{\sortes{\frappeur{h}}} \}$ ;
      \end{itemize}
    \item $\PHh'^{\langle 4 \rangle} = ( \PHh'^{(1)} ; \PHh'^{(2)} ; \PHh'^{(3)} ; \PHh'^{(4)} )$,
      où :
      \begin{itemize}
        \item $\PHh'^{(1)} = \{ \PHhit{a_i}{\scf{h}_\mysigma}{\scf{h}_{\mysigma'}} \mid
          h \in \PHh \wedge a \in \sortes{\frappeur{h}} \wedge a_i \in \PHl_a \wedge
          \scf{h}_\mysigma , \scf{h}_{\mysigma'} \in \PHl_{\scf{h}} \wedge
          \PHget{\mysigma}{a} \neq a_i \wedge \mysigma' = \mysigma \Cap a_i \}$,
        \item $\PHh'^{(2)} = \{ \PHhit{\sr{h}_1}{b_j}{b_k} \mid
          h \in \PHh \wedge b \in \sortes{\cible{h}} \wedge b_j, b_k \in \PHl_b \wedge
          b_j \in \cible{h} \wedge b_k \in \bond{h} \}$,
        \item $\PHh'^{(3)} = \{ \PHhit{\sr{h}_1}{\sr{h}_1}{\sr{h}_0} \mid h \in \PHh \}$,
        \item $\PHh'^{(4)} = \{ \PHhit{\scf{h}_\mypi}{\sr{h}_0}{\sr{h}_1} \mid
          h \in \PHh \wedge \scf{h}_{\mypi} \in \PHl_{\scf{h}} \wedge
%          \Feval{\Fopphm{h}}{\mypi} \}$.
          \frappeur{h} \subseteq \mypi \}$.
      \end{itemize}
  \end{itemize}
  Pour tout état $s \in \PHl$ de $\PH$,
  on note $\tophp{s}$ l'état correspondant dans $\phmtophp$ :
  \begin{itemize}
    \item $\forall a \in \PHs, \PHget{\tophp{s}}{a} = \PHget{s}{a}$,
    \item $\forall h \in \PHh, \PHget{\tophp{s}}{\sr{h}} = \sr{h}_0$,
    \item $\forall h \in \PHh, \PHget{\tophp{s}}{\scf{h}} = \scf{h}_\mysigma$,
      tel que $\forall a \in \sortes{\frappeur{h}}, \PHget{\mysigma}{a} = \PHget{\tophp{s}}{a}$.
  \end{itemize}
  À l'inverse, pour tout état $s' \in \PHl'$ de $\phmtophp$,
  on note $\tophm{s'}$ l'état correspondant dans $\PH$ :
  $\forall a \in \PHs, \PHget{\tophm{s'}}{a} = \PHget{s'}{a}$.
\end{definition}

\begin{theorem}[$\PH \approx \phmtophp$]
\thmlabel{phmbisimphp}
  Soient $\PH = (\PHs; \PHl; \PHh)$ des Frappes de Processus avec actions plurielles,
  et posons : $\phmtophp = (\PHs'; \PHl'; \PHh'^{\langle 4 \rangle})$.
  On a :
%   \[\forall s, s' \in \PHl, s \PHtrans s' \Longleftrightarrow
%     \tophp{s} \mtrans{\phmtophp} \tophp{s'} \enspace.\]
% 
  \[\forall s, s' \in \PHl, \exists h \in \PHh, s' = s \play h \Longleftrightarrow
    \exists \delta \in \Sce(\tophp{s}), \left( \tophp{s'} = \tophp{s} \play \delta
    \wedge \card{\toset{\delta} \cap \PHh'^{(4)}} = 1 \right) \enspace.\]
\end{theorem}

\begin{proof}
  Posons $\phmtophp = (\PHs'; \PHl'; \PHh'^{\langle 4 \rangle})$.
  Soient $s, s' \in \PHl$.
  
  ($\Rightarrow$) On suppose qu'il existe une action $h \in \PHh$ telle que $s' = s \play h$.
    D'après la \defref{phm2php},
    cela implique notamment l'existence de sortes $\sr{h}$ et $\scf{h}$ dans $\PHs'$,
    et des actions suivantes :
    \begin{itemize}
      \item $\PHh'^{(1)}_h = \{ \PHhit{a_i}{\scf{h}_\mysigma}{\scf{h}_{\mysigma'}} \mid
        a \in \sortes{\frappeur{h}} \wedge a_i \in \PHl_a \wedge
        \scf{h}_\mysigma , \scf{h}_{\mysigma'} \in \PHl_{\scf{h}} \wedge
        \PHget{\mysigma}{a} \neq a_i \wedge \mysigma' = \mysigma \Cap a_i \} \subset \PHh'^{(1)}$,
      \item $\PHh'^{(2)}_h = \{ \PHhit{\sr{h}_1}{b_j}{b_k} \mid
        b \in \sortes{\bond{h}} \wedge b_j, b_k \in \PHl_b \wedge
        b_j \in \frappeur{h} \wedge b_k \in \cible{h} \} \subset \PHh'^{(2)}$,
      \item $h_3 = \PHhit{\sr{h}_1}{\sr{h}_1}{\sr{h}_0} \in \PHh'^{(3)}$,
      \item $h_4 = \PHhit{\scf{h}_\mypi}{\sr{h}_0}{\sr{h}_1} \in \PHh'^{(4)}$ avec $\mypi$ tel que
        $\Feval{\Fopphm{h}}{\mypi}$.
    \end{itemize}
    Comme $h$ est jouable dans $s$, alors $\frappeur{h} \subseteq s$,
%    d'où $\frappeur{h} \subseteq \tophp{s}$.
    d'où $\scf{h}_{\mypi} \in \tophp{s}$, avec $\toset{\mypi} = \frappeur{h}$.
    Ainsi, $\PHhit{\scf{h}_\mypi}{\sr{h}_0}{\sr{h}_1}$ est jouable dans $\tophp{s}$.
    Toutes les actions de $\PHh'^{(2)}_h$ sont jouables dans $\tophp{s} \play h_4$.
    Elles peuvent être jouées successivement et alternativement avec des actions de
    $\PHh'^{(1)}$ car celles-ci modifient uniquement l'état de sortes coopératives,
    ce qui n'influe donc pas sur la jouabilité des actions de $\PHh'^{(2)}_h$.
    Soit $\PHh'^{(2)}_h = \{ h_2^i \}_{i \in \segm{1}{n}}$
    un étiquetage des actions de $\PHh'^{(2)}_h$
    avec $n = \card{\PHh'^{(2)}_h}$.
    On peut jouer depuis l'état $\tophp{s} \play h_4$ une séquence d'actions de la forme :
    $\delta_h =
      h_2^1 \cons \delta^1 \cons h_2^2 \cons \delta^2 \cons
      \ldots \cons
      h_2^n \cons \delta^n$
    où toutes les séquences $\delta^i$ avec $i \in \segm{1}{n}$
    sont des séquences d'actions de $\PHh^{(1)}$ mettant à jour des sortes coopératives.
    Après avoir joué cette séquence, le modèle se trouve dans un état
    $\tophp{s} \play h_4 \play \delta_h$ tel que :
    $\forall h_i \in \PHh'^{(2)}_h, \bond{h_i} \in (\tophp{s} \play h_4 \play \delta_h)$,
    c'est-à-dire : $\bond{h} \subseteq (\tophp{s} \play h_4 \play \delta_h)$.
    De plus, toutes les sortes coopératives sont mises à jour, ce qui signifie qu'aucune
    action de $\PHh'^{(1)}_h \cup \PHh'^{(2)}_h$ n'est plus jouable.
    Il est donc possible de jouer pour finir l'auto-action $h_3$,
    et on a :
    $\tophp{s} \play h_4 \play \delta_h \play h_3 = \tophp{s'}$,
    avec : $\toset{\delta} \cap \PHh'^{(4)} = \{ h_4 \}$.
  
  ($\Leftarrow$) On suppose qu'il existe un scénario $\delta \in \Sce(\tophp{s})$
    tel que $\tophp{s'} = \tophp{s} \play \delta$ et $\card{\toset{\delta} \cap \PHh'^{(4)}} = 1$.
    D'après la forme de $\phmtophp$ (\defref{phm2php}),
    et en s'inspirant du raisonnement précédent,
    on constate que la seule action jouable dans $\tophp{s}$ est une action de $\PHh'^{(4)}$,
    puis que dans l'état résultat, on ne peut jouer
    qu'une alternance entre une action de $\PHh'^{(2)}$ et des actions de $\PHh'^{(1)}$,
    et que l'état résultant une fois toutes les actions de $\PHh'^{(1)} \cup \PHh'^{(2)}$
    jouées ne permet que le jeu d'une action de $\PHh'^{(3)}$.
    On en déduit que ce scénario a nécessairement la forme suivante :
    $\delta = h_4 \cons \delta_{12} \cons h_3$,
    avec $h_4 \in \PHh'^{(4)}$, $h_3 \in \PHh'^{(3)}$, et
    $\delta_{12} = h_2^1 \cons \delta^1 \cons h_2^2 \cons \delta^2 \cons
      \ldots \cons h_2^n \cons \delta^n$
    où pour tout $i \in \segm{1}{n}$, $h_2^i \in \PHh'^{(2)}$
    et $\delta^i \in \Sce$ est un scénario composé uniquement d'actions de $\PHh'^{(1)}$.
    De plus, les seules actions jouables dans $\tophp{s} \play \delta$ sont à nouveau des actions
    de $\PHh'^{(4)}$, donc $\delta$ ne peut pas être plus grand car
    $\card{\toset{\delta} \cap \PHh'^{(4)}} = 1$.
    Posons :
    \begin{itemize}
      \item $\PHh'^{(1)}_\delta = \toset{\delta} \cap \PHh'^{(1)}$,
      \item $\PHh'^{(2)}_\delta = \toset{\delta} \cap \PHh'^{(2)}
        = \{ h_2^i \mid i \in \segm{1}{n} \}$,
      \item $h_3 = \PHhit{\sr{}_1}{\sr{}_1}{\sr{}_0}$ et
      \item $h_4 = \PHhit{\scf{}_\mypi}{\sr{}_0}{\sr{}_1}$.
    \end{itemize}
    D'après la construction de $\phmtophp$ donnée à la \defref{phm2php},
    il existe donc nécessairement une action $h \in \PHh$
    dont $\sr{}$ est la sorte de réaction et $\scf{}$ est la sorte coopérative correspondante.
    De plus, toujours grâce à cette définition, on a :
    $\mypi \in \PHsubl[\phmtophp]$ et $\frappeur{h} \subseteq \mypi$.
    Autrement dit : $\frappeur{h} = \toset{\mypi}$,
    donc que $h$ est jouable dans $s$, par définition de $\tophp{s}$.
    De plus, toujours d'après la \defref{phm2php}, on a :
    $\forall i \in \segm{1}{n}, h_2^i = \PHfrappe{\sr{}_1}{b_j^i}{b_k^i}
      \wedge b_j^i \in \frappeur{h} \wedge b_k \in \bond{h}$.
    Ainsi, $\bond{h} = \{ b_k = \bond{h_2^i} \mid i \in \segm{1}{n} \}$.
    Or on constate immédiatement que toutes les sortes coopératives sont nécessairement
    mises à jour
    dans $s \play \delta$ car la dernière action jouée est une action de $\PHh'^{(3)}$
    et que son jeu ne rend aucune action de $\PHh'^{(1)}$ jouable.
    De plus, $\PHget{s \play \delta}{\sr{}} = \sr{}_0$ et
    $\tophp{s} \play \delta = \tophp{s} \recouvre \{ \bond{h_2^i} \mid i \in \segm{1}{n} \}$.
    Ainsi, $\tophp{s'} = \tophp{s} \play \delta = \tophp{s} \recouvre \bond{h}
      = \tophp{s \recouvre \bond{h}} = \tophp{s \play h}$.
\end{proof}



\begin{example}
  Afin d'illustrer la traduction définie dans cette section,
  nous nous intéressons aux Frappes de Processus avec actions plurielles
  $\PH = (\PHs; \PHl; \PHh)$ suivantes,
  représentées à la \figref{simple-phm} :
  \begin{itemize}
    \item $\PHs = {a, b, c}$,
    \item $\PHl_a = \{ a_0, a_0\} \quad,\quad \PHl_b = \{ b_0, b_0 \} \quad,\quad
      \PHl_c = \{ c_0, c_0\}$,
    \item $\PHh = \{ \PHfrappemults{a_1, b_1, c_1}{b_0, c_0} \}$.
  \end{itemize}
  On appelle $h = \PHfrappemults{a_1, b_1, c_1}{b_0, c_0}$
  l'unique action que comporte ce modèle.
  Celle-ci représente un cas intéressant d'action plurielle,
  car elle comporte plusieurs cibles (et donc plusieurs bonds).
  En effet : $\cible{h} = \{ b_1, c_1 \}$ ;
  par ailleurs : $\invariant{h} = \{ a_1 \}$.
  
  \begin{figure}[p]
  \begin{center}
  \begin{tikzpicture}
    \TSort{(1,3)}{a}{2}{t}
    \TSort{(0,0)}{b}{2}{l}
    \TSort{(3,0)}{c}{2}{r}
    
    \node[apdot] (dot) at (1.5,0.7) {};
    \draw (a_1) edge (dot);
    \TAction{dot}{b_1.east}{b_0.north east}{}{left}
    \TAction{dot}{c_1.west}{c_0.north west}{}{right}
    
    \TState{a_1, b_1, c_1}
  \end{tikzpicture}
  \caption{\figlabel{simple-phm}%
    Exemple de Frappes de Processus avec actions plurielles.
    L'unique action de ce modèle est $\PHfrappemults{a_1, b_1, c_1}{b_0, c_0}$
    et comporte plusieurs cibles ($b_1$ et $c_1$).
  }
  \end{center}
  \end{figure}
  
  La \defref{phm2php} nous permet de traduire ces Frappes de Processus avec actions plurielles
  en un modèle de Frappes de Processus avec 4 classes de priorités
  $\phmtophp = (\PHs'; \PHl'; \PHh'^{\langle 4 \rangle})$
  qui est présenté à la \figref{simple-php-trad}.
  Ce modèles comporte cinq sortes :
  $\PHs' = \{ a, b, c, \sr{h}, \scf{h} \}$,
  et notamment les classes d'actions suivantes :
  \begin{align*}
    \PHh'^{(2)} &= \{\quad \PHfrappe{\sr{h}_1}{b_1}{b_0} \quad,
      \qquad \PHfrappe{\sr{h}_1}{c_1}{c_0} \quad\} \\
    \PHh'^{(3)} &= \{\quad \PHfrappe{\sr{h}_1}{\sr{h}_1}{\sr{h}_0} \quad\} \qquad\qquad
      \PHh'^{(4)} = \{\quad \PHfrappe{\scf{h}_7}{\sr{h}_0}{\sr{h}_1} \quad\}
  \end{align*}
  
  \begin{figure}[p]
  \begin{center}
  \begin{tikzpicture}
    \TSort{(0,-3)}{a}{2}{l}
    \TSort{(0,0)}{b}{2}{l}
    \TSort{(0,3)}{c}{2}{l}
    \TSort{(4,3)}{\sr{h}}{2}{r}
    
    \TSetTick{\scf{h}}{0}{000}
    \TSetTick{\scf{h}}{1}{001}
    \TSetTick{\scf{h}}{2}{010}
    \TSetTick{\scf{h}}{3}{011}
    \TSetTick{\scf{h}}{4}{100}
    \TSetTick{\scf{h}}{5}{101}
    \TSetTick{\scf{h}}{6}{110}
    \TSetTick{\scf{h}}{7}{111}
    \TSort{(7,-3.5)}{\scf{h}}{8}{r}
    
    \TAction{\scf{h}_7}{\sr{h}_0.east}{\sr{h}_1.south}{bend right = 10}{right}
    \TAction{\sr{h}_1.north east}{\sr{h}_1.south east}{\sr{h}_0.north east}
      {selfhit, min distance=30, bend left, out=110, in=70}{left}
    \TAction{\sr{h}_1}{c_1.east}{c_0.north east}{}{left}
    \TAction{\sr{h}_1}{b_1.east}{b_0.north east}{}{left}
    
    \path (0.8, 3.5) edge[coopupdate] (6, 0.5);  % c
    \path (0.8, 0.5) edge[coopupdate] (6, 0);  % b
    \path (0.8, -2.5) edge[coopupdate] (6, -0.5);  % a
    
    \node[labelprio1] at (5,1.5) {$1$}; % c => f
    \node[labelprio1] at (4,0.6) {$1$}; % b => f
    \node[labelprio1] at (4.5,-1.5) {$1$}; % a => f
    \node[labelprio2] at (2,4.3) {$2$}; % r_1 -> c_1 / 0
    \node[labelprio2] at (2.9,3.5) {$2$}; % r_1 -> b_1 / 0
    \node[labelprio3] at (5,4.5) {$3$}; % r_1 -> r_1 / 0
    \node[labelprio4] at (5.9,3.75) {$4$}; % f_011 -> r_0 / 1
    
    \TState{a_1, b_1, c_1, \sr{h}_0, \scf{h}_7}
  \end{tikzpicture}
  \caption{\figlabel{simple-php-trad}%
    Traduction des Frappes de Processus avec actions plurielles
    de la \figref{simple-phm} en Frappes de Processus avec 4 classes de priorités.
    Le sortes $a$, $b$ et $c$ sont identiques à celles du modèle d'origine.
    Les deux autres sortes ont été ajoutées pour les besoins de la traduction de l'action $h$.
    $\sr{h}$ est une sorte de réaction
    (la présence de
    $\sr{h}_1$ modélise le fait que les actions représentant $h$ sont en train d'être jouées)
    et $\scf{h}$ est une sorte coopérative
    (permettant de vérifier la présence de tous les frappeurs).
  }
  \end{center}
  \end{figure}
\end{example}



\subsection{Équivalence avec les Frappes de Processus avec $k$ classes de priorités}
\seclabel{equivphmphp}

Cette section permet de formaliser l'équivalence d'expressivité entre les Frappes de Processus
avec actions plurielles d'une part, et les Frappes de Processus avec $k$ classes de priorités
d'autres part, si $k \in \sNN$.
Le \thmref{phmequivphp} formalise cette conclusion,
bien que sa démonstration s'appuie sur des résultats qui seront
présentés à la \vsecref{phcanonique2phm}.
En effet, elle s'appuie sur plusieurs résultats plus forts qui stipulent que des
Frappes de Processus avec actions plurielles
et les Frappes de Processus avec $k$ classes de priorités
peuvent toujours être représentées
par une classe particulière de Frappes de Processus avec 2 classes de priorités,
qui elle-même peut aussi être représentée par des
Frappes de Processus avec actions plurielles.

\begin{theorem}[Équivalence]
\thmlabel{phmequivphp}
  Pour tout $k \in \sNN$,
  les Frappes de Processus avec actions plurielles
  et les Frappes de Processus avec $k$ classes de priorités
  ont une expressivité équivalente.
\end{theorem}

\begin{proof}
  Soit $k \in \sNN$.
  D'après le \vthmref{phmbisimphp}, il est toujours possible de traduire
  des Frappes de Processus avec actions plurielles en Frappes de Processus
  avec 4 classes de priorités,
  et celles-ci peuvent être à leur tour traduites en
  Frappes de Processus avec 2 classes de priorités
  d'après le \vthmref{bisimulaplatissement}.
  Or des Frappes de Processus avec 2 classes de priorités sont
  à fortiori des Frappes de Processus avec $k$ classes de priorités.
  Inversement, toutes Frappes de Processus avec $k$ classes de priorités
  peuvent être aplaties en Frappes de Processus canoniques toujours
  d'après le \thmref{bisimulaplatissement},
  qui elles-mêmes peuvent être traduites en Frappes de Processus avec action plurielles
  d'après le \vthmref{bisimulphm}.
\end{proof}



\subsection{Réutilisation des résultats existants}
\seclabel{phm-outils}

Nous passons rapidement en revue dans la suite différentes méthodes permettant de réutiliser
certains résultats des Frappes de Processus standards sur les Frappes de Processus avec
actions plurielles.
Nous ne détaillons pas ces méthodes qui reprennent principalement des résultats
traités à d'autres endroits de cette thèse.

\subsubsection{Points fixes}
\seclabel{phm-outils-pf}

Du fait de la complexité des actions plurielles, il n'existe pas à l'heure actuelle de méthode
générique efficace permettant de déduire les points fixes de Frappes de Processus avec
actions plurielles.
Cependant, à l'instar des Frappes de Processus avec arcs neutralisants
(voir \vsecref{phan-outils-pf})
il est possible d'obtenir ces points fixes en les cherchant sur
le modèle aplati (tel que défini à la \vsecref{phm-aplatissement}).
En effet, sa dynamique étant équivalente, aux sortes coopératives près, le résultat
peut être transposé au modèle d'origine.

\subsubsection{Analyse statique}
\seclabel{phm-outils-as}

Du fait de la forme particulière des actions, les méthodes d'analyse statique ne s'appliquent pas
directement aux Frappes de Processus avec actions plurielles.
Cependant, moyennant l'utilisation de la traduction vers des Frappes de Processus
avec 4 classes de priorités donnée à la \defref{phm2php},
il est possible de les appliquer sur un modèle équivalent.

\subsubsection{Paramètres stochastiques}
\seclabel{phm-outils-stocha}

Il est théoriquement possible d'associer des paramètres stochastiques à chaque action plurielle
d'un modèle de Frappes de Processus avec actions plurielles.
Un tel ajout aurait notamment l'avantage d'éviter l'utilisation de la valeur «~infinie~»
d'absorption de stochasticité, qui avait principalement pour but de simuler des actions
successives instantanées, afin notamment de pallier le fait qu'un seul processus
ne peut évoluer pour chaque jeu d'action des Frappes de Processus standards.

Une autre alternative consisterait à utiliser l'aplatissement de la \secref{phm-aplatissement},
et d'attribuer aux actions correspondant à des activations de réaction
dans le modèle obtenu les paramètres stochastiques voulus.
