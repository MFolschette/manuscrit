\section{Frappes de Processus avec actions plurielles}
\seclabel{phm}

\todo{Introduction}

Les Frappes de Processus avec actions plurielles permettent de représenter des systèmes
de réactions biochimiques de la forme : $X + Y \rightarrow Y + Z$,
c'est-à-dire comportant un ensemble de réactifs $X$, un ensemble de catalyseurs $Y$
et un ensemble de produits $Z$.
La représentation d'une telle réaction en Frappes de Processus aura cependant la forme suivante :
$\PHfrappemult{A}{B}$ où $A = X \cup Y$ et $B = Z$,
où $A$ et $B$ sont des ensembles des processus.

Une telle action peut donc être jouée dans un état contenant tous les processus de $A$
(c'est-à-dire les réactifs et les catalyseurs)
et fait évoluer celui-ci vers un état contenant tous les processus de $B$ (les produits),
les autres processus restant inchangés (car n'intervenant pas ou en tant que catalyseurs).
Cela implique toutefois que pour tout processus de $B$, on trouve un autre processus de la même
sorte dans $A$.

On note cependant qu'une réaction de la forme $\PHfrappemult{\{ a_0, b_0, c_0 \}}{\{ a_1, b_1 \}}$
ne peut être jouée l'une des deux sortes entrant en jeu, $a$ et $b$, est déjà au niveau $1$,
même si l'autre est au niveau $0$.
Cela peut avoir un sens lorsque les différents processus d'une sorte ($a_0$ et $a_1$, par exemple)
représentent différents états d'une même molécule :
la réaction ne peut alors pas être jouée pour des raisons de stœchiométrie.
Cependant, si ces différents processus représentent plutôt des niveaux de concentration
($a_1$ représentant par exemple un niveau de concentration de la molécule $a$ plus élevé que $a_0$),
cela n'a plus de sens car une plus forte concentration d'une des entités ne devrait pas empêcher
la réaction d'avoir lieu et de produire la seconde entité.
Cela peut néanmoins être corrigé en ajoutant les actions
$\PHfrappemult{\{ a_1, b_0, c_0 \}}{\{ a_1, b_1 \}}$ et
$\PHfrappemult{\{ a_0, b_1, c_0 \}}{\{ a_1, b_1 \}}$,
ou encore en séparant la production de $a_1$ et de $b_1$ en deux actions (ou ensemble d'actions)
distinctes.

Cette représentation des Frappes de Processus peut être aisément représentée à l'aide d'un réseau
d'automates synchronisés, chaque sorte ayant le rôle d'un automate et chaque action celui d'un
ensemble de transition étiquetées avec le même libellé partant chacune d'un processus dans $A$ et
arrivant dans le processus de la même sorte dans $A \Cap B$.

\subsection{Définition}

\todo{Gluer}

\begin{definition}[Frappes de Processus avec actions plurielles]
\deflabel{phm}
  Les \emph{Frappes de Processus avec actions plurielles} sont définies
  par un triplet $\PH = (\PHs; \PHl; \PHh)$, où :
  \begin{itemize}
    \item $\PHs \DEF \{a, b, \dots\}$ est l'ensemble fini et dénombrable des \emph{sortes} ;
    \item $\PHl \DEF \underset{a \in \PHs}{\times} \PHl_a$ est l'ensemble fini des \emph{états},
      où $\PHl_a = \{a_0, \ldots, a_{l_a}\}$ est l'ensemble fini et dénombrable
      des \emph{processus} de la sorte $a \in \PHs$ et $l_a \in \sN^*$.
      Chaque processus appartient à une unique sorte :
      $\forall (a_i; b_j) \in \PHl_a \times \PHl_b, a \neq b \Rightarrow a_i \neq b_j$ ;
    \item $\PHh \DEF \{\PHfrappemult{A}{B} \mid A, B \in \PHsublset \setminus \emptyset \wedge
      \forall q \in B, \exists p \in A, (p \neq q \wedge \PHsort(p) = \PHsort(q)) \}$
      est l'ensemble fini des \emph{actions}.
  \end{itemize}
\end{definition}
%
\noindent
Pour toute action $h = \PHfrappemult{A}{B} \in \PHh$,
$A$ est appelé le \emph{frappeur}, et $b_k$ le \emph{bond} de $h$,
et on note : $\hitter{h} = A$, $\bounce{h} = B$.
On note de plus :
$\target{h} = \{ p \in A \mid \exists q \in B, \PHsort(p) = \PHsort(q) \}$.

\begin{definition}[Opérateur de jouabilité ($\Fopsymbol_\Fopphmsubsymbol : \PHh \rightarrow \F$)]
\deflabel{fopphm}
  L'opérateur de jouabilité des Frappes de Processus avec actions plurielles est défini par :
  \[\forall h \in \PHh, \Fopphm{h} \equiv \bigwedge_{p \in \hitter{h}} p \enspace.\]
\end{definition}



\subsection{Équivalence avec les automates synchronisés}

\todo{Gluer}
On note $\phmtoan$ l'automate équivalent aux frappes de processus $\PH$, défini par :

\begin{definition}[Automate équivalent ($\phmtoansymbol$)]
\deflabel{phm2an}
  Le réseau d'automates équivalent aux Frappes de Processus avec actions plurielles $\PH$
  est défini par : $\phmtoan = (\PHs, \PHl, \ANi, \ANt)$, où :
  \begin{itemize}
    \item $\ANi = \{ \ell_h \mid h \in \PHh \}$ est l'ensemble des libellés de transitions ;
    \item $\ANt = \{ p \xrightarrow{\ell_h} q \mid h \in \PHh \wedge p \in A \wedge q \in A \Cap B \wedge
      \PHsort(p) = \PHsort(q) \}$ est l'ensemble des transitions locales.
  \end{itemize}
\end{definition}

\begin{theorem}[$\PH \equiv \phmtoan$]
\thmlabel{phmequivan}
  \[\forall s, s' \in \PHl, s \PHtrans s' \Longleftrightarrow s \trans{\phmtoan} s' \enspace.\]
\end{theorem}

\todo{Preuve}



\subsection{Équivalence : cf. modélisation avec arcs neutralisants ?}

\TODO
\todo{Idée :
\begin{itemize}
  \item SC 1 entre tous les processus de $A$
  \item SC 2 entre toutes les sortes de réaction
  \item SC 3 entre tous les processus de $B$
  \item sorte de réaction activable ($sr_1$) si SC 1 et SC 2 ok
  \item actions depuis $sr_1$ activant tous les processus de $B$ \textsuperscript{*}
  \item sorte de réaction désactivable ($sr_0$) si SC 3 \textsuperscript{**}
\end{itemize}
OU : idem avec arcs neutralisants depuis les actions de \textsuperscript{*}
vers l'action de \textsuperscript{**}
}


