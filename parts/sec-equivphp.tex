\section{Équivalence avec les autres formalismes de Frappes de Processus}
\seclabel{phcanonique-equiv}

\TODO

\todo{Introduire une notion de forme canonique indépendante des critères ?\\
Avantage : forme canonique $\Rightarrow$ critères}

\todo{Traduction depuis les Frappes de Processus avec arcs neutralisants}

\todo{Traduction depuis les Frappes de Processus avec actions plurielles}


\subsection{Aplatissement des Frappes de Processus avec $k$ classes de priorités}
\seclabel{aplatissement}

Le but de cette section est de montrer qu'un modèle de Frappes de Processus avec $k$ classes
de priorités peut être \emph{aplati}, c'est-à-dire traduit en un autre modèle ne comportant
que 2 classes de priorités.
Ce dernier modèle est assuré de posséder la même dynamique,
car il lui est faiblement bisimilaire.
De plus, les actions de priorité 1 (les plus prioritaires) ne sont utilisées que pour mettre
à jour les sortes coopératives ;
il s'agit donc en fait d'un modèle de Frappes de Processus canoniques telles que définies
à la \secref{phcanonique-def}.
La forme particulière de ces modèles permet d'y appliquer les méthodes d'analyse statique
développées à la \secref{as}.
Cela nous permet de plus de montrer que les Frappes de Processus avec $k$ classes de priorités
sont aussi expressives entre elles pour tout $k \in \sN^*$,
car elles sont toutes aussi expressives que les Frappes de Processus canoniques.

Étant donné que les propriétés de jouabilité n'utilisent que des opérateurs de logique booléenne
standards, il est possible de calculer la forme normale disjonctive (FND) de toute propriété de
jouabilité. Pour toute action $h \in \PHh$, cette FND est de la forme :
\[\Fopphp{h} \equiv \bigvee_{i \in \segm{1}{\n}}
  \left( \bigwedge_{j \in \segm{1}{\m}} p_{i,j} \right)\]
où $\n \in \sN$ et $\forall i \in \segm{1}{\n}, \m \in \sN^*$.
Si $\n = 0$, alors $\Fopphp{h} \equiv \bot$, ce qui signifie que l'action $h$ ne peut jamais
être jouée car elle est préemptée dans dans tous les états où son frappeur et sa cible
sont présents.
Une telle action peut être retirée du modèle sans en changer le comportement.
En revanche, si $\n \geq 0$, alors $\Fopphp{h} \not\equiv \bot$ ;
dans ce cas, $\Fopphp{h}$ est une disjonction de $\n$ conjonctions d'atomes,
et peut donc être vue comme une disjonction de $\n$ propriétés de jouabilité plus petites.
Ces $\n$ conjonctions d'atomes peuvent être traduites en autant de sortes
coopératives priorisées, afin d'obtenir une dynamique équivalente avec un nombre réduit
de classes de priorités utilisées.
Dans ce second cas, on note, pour tout $i \in \segm{1}{\n}$ :
$\PHdep{i}{h} = \{ \PHsort(p_{i,j}) \mid j \in \segm{1}{\m} \}$.

Le \lemref{ppplaysubset} permet de caractériser la jouabilité d'une action dans un état
à l'aide d'un sous-état correspondant à l'une des conjonctions de sa propriété
de jouabilité en FND.
Enfin, la \defref{aplatissement} donne la construction de l'\emph{aplatissement} de $\PH$ :
pour chaque action $h \in \PHh$, plusieurs sortes coopératives $f^{h,i}$
permettent de refléter chaque conjonction de $\Fopphp{h}$,
c'est-à-dire une pour chaque indice $i \in \segm{1}{\n}$.
Cette construction permet d'obtenir la même dynamique que pour $\PH$ en reproduisant
les préemptions possibles par d'autres actions plus prioritaires,
comme établi par le \thmref{bisimulaplatissement}.

On note pour finir que la \defref{aplatissement} ne s'applique qu'à des Frappes de Processus
pseudo-canoniques. Cependant, des Frappes de Processus avec $k$ classes de priorités quelconques
sont \textit{a fortiori} des Frappes de Processus pseudo-canoniques à condition d'ajouter
une classe vide d'actions de priorité supérieure.
En d'autres termes, il est possible d'ignorer le cas particulier des actions primaires
(mettant à jour les sortes coopératives) si les Frappes de Processus ne sont pas
pseudo-canoniques.

\todo{À supprimer ? On peut l'intégrer dans la démo du \thmref{bisimulaplatissement}.}

\begin{lemma}
\lemlabel{ppplaysubset}
  Soient $h \in \PHh$ et $s \in \PHl$.
  %$h$ est jouable dans $s$ si et seulement si :
  %\[\exists \mysigma \subseteq s, \Feval{\Fopphp{h}}{\mysigma} \enspace.\]
  \[\Feval{\Fopphp{h}}{s} \Leftrightarrow
    \exists \mysigma \subseteq s, \Feval{\Fopphp{h}}{\mysigma} \enspace.\]
\end{lemma}
%
\begin{proof}
  ($\Rightarrow$)
    Si $h$ est jouable dans $s$, alors $\Feval{\Fopphp{h}}{s}$.
    Ainsi, $\Fopphp{h} \not\equiv \bot$ et, par propriété d'une FND,
    au moins l'une des conjonctions de $\Fopphp{h}$ est vraie dans $s$.
    On suppose que la $i$\textsuperscript{e} conjonction est vraie dans $s$,
    avec $i \in \segm{1}{\n}$;
    on a alors : $\forall j \in \segm{1}{\m}, p_{i,j} \in s$.
    Soit $\mysigma \in \PHsubl_{\PHdep{i}{h}}$
    avec $\forall b \in \PHdep{i}{h}, \PHget{\mysigma}{b} = \PHget{s}{b}$.
    On a alors immédiatement : $\mysigma \subseteq s$,
    et, par construction de $\PHdep{i}{h}$, $\Feval{\Fopphp{h}}{\mysigma}$.
  
  ($\Leftarrow$)
    Supposons qu'il existe $\mysigma \subseteq s$ tel que $\Feval{\Fopphp{h}}{\mysigma}$.
    On a alors immédiatement $\Feval{\Fopphp{h}}{s}$ car ajouter des processus au sous-état
    d'évaluation ne peut pas rendre la propriété fausse.
\end{proof}

\begin{definition}
\deflabel{aplatissement}
  Si $k \in \sNN$ et $\PH = (\PHs; \PHl; \PHa^{\angles{k}})$
  sont des Frappes de Processus pseudo-canoniques avec $k$ classes de priorités,
  on note $\PHflat(\PH) = \oPH = (\ov{\PHs}; \ov{\PHl}; (\ov{\PHa}^{(1)}; \ov{\PHa}^{(2)}))$
  l'\emph{aplatissement} de $\PH$, où :
  \begin{itemize}
    \item $\ov{\PHs} = \PHs \cup \PHs_f$
      où $\PHs_f = \{ f^{h,i} \mid h \in \PHh \wedge \n \geq 1 \wedge i \in \segm{1}{\n} \}$;
    \item $\ov{\PHl} = \left( \bigtimes{a \in \PHs} \PHl_{a} \right) \times
      \left(\bigtimes{f^{h,i} \in \PHs_f} \PHl_{f^{h,i}} \right)$, où
      $\forall f^{h,i} \in \PHs_f, \PHl_{f^{h,i}} =
        \{ f^{h,i}_\mysigma \mid \mysigma \in \PHsubl_{\PHdep{i}{h}} \}$;
    \item $\ov{\PHh}^{(1)} = \PHh^{(1)} \cup
      \{ \PHhit{a_k}{f^{h,i}_\mysigma}{f^{h,i}_{\mysigma'}} \mid
      h \in \PHh \wedge f^{h,i} \in \PHs_f \wedge
      a \in \PHdep{i}{h} \wedge a_k \in \PHl_a \wedge
      f^{h,i}_\mysigma , f^{h,i}_{\mysigma'} \in \PHl_{f^{h,i}} \wedge
      \PHget{\mysigma}{a} \neq a_k \wedge \mysigma' = \mysigma \Cap \{ a_k \} \}$;
    \item $\ov{\PHh}^{(2)}=\{ \PHhit{f^{h,i}_\mysigma}{\target{h}}{\bounce{h}} \mid
      h \in \PHh \setminus \PHh^{(1)} \wedge f^{h,i} \in \PHs_f \wedge
      f^{h,i}_\mysigma \in \PHl_{f^{h,i}} \wedge \Feval{\Fopphp{h}}{\mysigma} \}$.
  \end{itemize}
  De plus, étant donné un état $\os \in \ov{\PHl}$,
  on note $\unflats{\os} = s$ l'état correspondant dans $\PHl$ :
  $\forall a \in \PHs, \PHget{s}{a} = \PHget{\os}{a}$.
  À l'inverse, étant donné un état $s \in \PHl$,
  $\flats{s} = \os$ est l'état correspondant dans $\ov{\PHl}$ :
  $\forall a \in \PHs, \PHget{\os}{a} = \PHget{s}{a}$
  et $\forall f^{h,i} \in \PHs_f, \PHget{\os}{f^{h,i}} = f^{h,i}_\mysigma$
  avec $f^{h,i}_\mysigma \in \PHl_{f^{h,i}}$
  et $\forall b \in \PHdep{i}{h}, \PHget{\mysigma}{b} = \PHget{s}{b}$.
\end{definition}

\begin{theorem}[$\PH \approx \PHflat(\PH)$]
\thmlabel{bisimulaplatissement}
  Soient $\PH = (\PHs; \PHl; \PHa^{\angles{k}})$ des Frappes de Processus avec $k$
  classes de priorités,
  et $\oPH = \PHflat(\PH) = (\ov{\PHs}; \ov{\PHl}; \ov{\PHa}^{\angles{2}})$ leur aplatissement.
  \begin{enumerate}
    \item \label{php2ph} $\forall s, s' \in \PHl$,
      $s \trans{\PH} s' \Longrightarrow \flats{s} \mtrans{\oPH} \flats{s'}$,
      où $\mtrans{\oPH}$ est une séquence finie de transitions $\trans{\oPH}$.
    \item \label{ph2php} $\forall \os, \os' \in \ov{\PHl}$,
      $\os \trans{\oPH} \os' \Longrightarrow \unflats{\os} = \unflats{\os'} \vee
      \unflats{\os} \trans{\PH} \unflats{\os'} \enspace.$
  \end{enumerate}
\end{theorem}

\begin{proof}
  (\ref{php2ph}) Soit $\os = \flats{s}$.
    D'après la dynamique des Frappes de Processus (\defref{play}),
    si $s \trans{\PH} s'$, alors il existe une action $h \in \PHh$ jouable dans $s$,
    telle que $s' = s \PHplay h$.
    On a alors : $\Feval{\Fopphp{h}}{s}$, et d'après le \lemref{ppplaysubset},
    il existe $\mysigma \subseteq s$ tel que $\Feval{\Fopphp{h}}{\mysigma}$.
    Par construction de $\oPH$ (\defref{aplatissement}) :
    \begin{itemize}
      \item Si $h \in \PHh^{(1)}$, alors il existe $g = h \in \ov{\PHh}^{(1)}$,
        et on a : $\hitter{g} \in \os$ et $\target{g} \in \os$.
      \item Si, au contraire, $h \in \PHh^{(2)}$, alors il existe
        $g = \PHhit{f^{h,i}_\mysigma}{\target{h}}{\bounce{h}} \in \ov{\PHh}^{(2)}$.
        De plus, par construction de $\os$ (\defref{aplatissement}),
        $\PHget{\os}{f^{h,i}} = f^{h,i}_\mysigma$.
    \end{itemize}
    Dans les deux cas, $g$ est jouable dans $\os$.
    Par la suite, dans l'état $\os \PHplay g$, les seules actions jouables sont celles dans
    $\PHh^{(1)}$ ayant $\bounce{h} = \bounce{g}$ comme frappeur
    et mettant à jour des sortes coopératives,
    permettant donc d'accéder à l'état $\flats{s'}$ en un nombre fini d'actions.
    Ainsi, $\flats{s} \mtrans{\oPH} \flats{s'}$.
  
  (\ref{ph2php}) Soit $s = \unflats{\os}$.
    D'après la dynamique des Frappes de Processus (\defref{play}),
    si $\os \trans{\PH} \os'$, alors il existe une action $g \in \ov{\PHh}$ jouable dans $\os$,
    telle que $\os' = \os \PHplay h$.
    \begin{itemize}
      \item Si $g \in \ov{\PHh}^{(1)} \setminus \PHh^{(1)}$,
        alors $\unflats{\os} = \unflats{\os'}$ car seul le processus actif d'une sorte
        coopérative qui n'est pas dans $\PHs$ a évolué.
      \item Sinon, si il existe $h = g \in \PHh^{(1)}$,
        alors $h$ est jouable dans $s$ et : $\unflats{\os} \trans{\PH} \unflats{\os'}$
        avec $\unflats{\os'} = s \play h$.
      \item Autrement, si $g \in \ov{\PHh}^{(2)}$,
        on note : $g = \PHhit{f^{h,i}_\mysigma}{b_j}{b_k}$.
        Par construction de l'aplatissement (\defref{aplatissement}), il existe
        $h \in \PHh$ tel que $\Feval{\Fopphp{h}}{\mysigma}$.
        Comme $g$ est jouable, cela signifie qu'aucune action de $\PHh^{(1)}$ n'est jouable,
        et notamment que la sorte coopérative $f^{h,i}$ est déjà mise à jour,
        ce qui a pour conséquence que : $\mysigma \subseteq s$.
        Ainsi, d'après \lemref{ppplaysubset}, $h$ est jouable dans $s$,
        et $\unflats{\os} \trans{\PH} \unflats{\os'}$.
    \end{itemize}
\end{proof}

\todo{simplifications}

Pour finir, nous notons que l'aplatissement $\PHflat(\PH)$ de toutes Frappes de Processus avec $k$
classes de priorités $\PH = (\PHs, \PHl, \PHh^{\angles{k}})$
sont des Frappes de Processus canoniques.
En effet, une partie des sortes coopératives générées lors de cette traduction proviennent
des Frappes de Processus d'origine, qui sont pseudo-canoniques et sont déjà contraintes de la
même manière.
L'autre partie constitue les sortes coopératives de l'ensemble $\PHs_f$ et leur définition
respecte les \allcr.
