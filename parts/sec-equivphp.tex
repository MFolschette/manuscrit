\section{Équivalence avec les autres formalismes de Frappes de Processus}
\seclabel{phcanonique-equiv}

Cette section vise à tracer des liens entre les différentes sémantiques des Frappes de Processus
présentées au \chapref{sem}.
Son principal apport est l'\emph{aplatissement} des Frappes de Processus avec $k$
classes de priorités, donné à la \secref{aplatissement},
et qui permet de les traduire en Frappes de Processus canoniques.
Cette traduction permet donc, à partir d'un modèle de Frappes de Processus comprenant un nombre
arbitraire de priorités, d'obtenir un modèle canonique respectant la même dynamique.
Par la suite, le cas des Frappes de Processus avec arcs neutralisants est aussi traité,
de façon analogue, dans la \secref{phan-aplatissement}.
Enfin, la \secref{phm-aplatissement} réutilise ces résultats pour proposer aussi une traduction
depuis les Frappes de Processus avec actions plurielles.

Ces différentes traductions et propriétés de bisimulation (faible) qu'elles proposent
permettent d'établir que les différentes sémantiques de Frappes de Processus sont
aussi expressives que les Frappes de Processus canoniques.
Notamment, cela nous permet d'assurer que les Frappes de Processus avec plusieurs
classes de priorités sont équivalentes ;
autrement dit, considérer plus de deux classes de priorités n'augmente pas l'expressivité
---~bien que cela puisse faciliter la modélisation.



\subsection{Aplatissement des Frappes de Processus avec $k$ classes de priorités}
\seclabel{aplatissement}

Le but de cette section est de montrer qu'un modèle de Frappes de Processus avec $k$ classes
de priorités peut être \emph{aplati}, c'est-à-dire traduit en un autre modèle ne comportant
que 2 classes de priorités.
Ce dernier modèle est assuré de posséder la même dynamique,
car il lui est faiblement bisimilaire.
De plus, les actions de priorité 1 (les plus prioritaires) ne sont utilisées que pour mettre
à jour les sortes coopératives ;
il s'agit donc en fait d'un modèle de Frappes de Processus canoniques telles que définies
à la \secref{phcanonique-def}.
La forme particulière de ces modèles permet d'y appliquer les méthodes d'analyse statique
développées à la \secref{as}.
Cela nous permet de plus de montrer que les Frappes de Processus avec $k$ classes de priorités
sont aussi expressives entre elles pour tout $k \in \sN^*$,
car elles sont toutes aussi expressives que les Frappes de Processus canoniques.

Étant donné que les propriétés de jouabilité n'utilisent que des opérateurs de logique booléenne
standards, il est possible de calculer la forme normale disjonctive (FND) de toute propriété de
jouabilité. Pour toute action $h \in \PHh$, cette FND est de la forme :
\[\Fopphp{h} \equiv \bigvee_{i \in \segm{1}{\n}}
  \left( \bigwedge_{j \in \segm{1}{\m}} p_{i,j} \right)\]
où $\n \in \sN$ et $\forall i \in \segm{1}{\n}, \m \in \sN^*$.
Si $\n = 0$, alors $\Fopphp{h} \equiv \bot$, ce qui signifie que l'action $h$ ne peut jamais
être jouée car elle est préemptée dans dans tous les états où son frappeur et sa cible
sont présents.
Une telle action peut être retirée du modèle sans en changer le comportement.
En revanche, si $\n \geq 0$, alors $\Fopphp{h} \not\equiv \bot$ ;
dans ce cas, $\Fopphp{h}$ est une disjonction de $\n$ conjonctions d'atomes,
et peut donc être vue comme une disjonction de $\n$ propriétés de jouabilité plus petites.
Ces $\n$ conjonctions d'atomes peuvent être traduites en autant de sortes
coopératives priorisées, afin d'obtenir une dynamique équivalente avec un nombre réduit
de classes de priorités utilisées.
Dans ce second cas, on note, pour tout $i \in \segm{1}{\n}$ :
$\PHdep{i}{h} = \{ \PHsort(p_{i,j}) \mid j \in \segm{1}{\m} \}$.

L'opérateur d'aplatissement $\Fopsymbol_\Fopaplatsubsymbol$ donné à la \defref{fopaplat}
permet de caractériser la jouabilité d'une action sans prendre en compte les actions primaires ;
en d'autres termes, une action $h$ est jouable dans un état $s$ si et seulement si
$\Feval{\Fopaplat{h}}{s}$ et aucune action de priorité 1 n'est jouable.
Le \lemref{ppplaysubset} permet alors de caractériser la jouabilité d'une action dans un état
à l'aide d'un sous-état correspondant à l'une des conjonctions de sa propriété
d'aplatissement une fois traduite en FND.
Enfin, la \defref{aplatissement} donne la construction de l'\emph{aplatissement} de $\PH$ :
pour chaque action $h \in \PHh$, plusieurs sortes coopératives $f^{h,i}$
permettent de refléter chaque conjonction de $\Fopphp{h}$,
c'est-à-dire une pour chaque indice $i \in \segm{1}{\n}$.
Cette construction permet d'obtenir la même dynamique que pour $\PH$ en reproduisant
les préemptions possibles par d'autres actions plus prioritaires,
comme établi par le \thmref{bisimulaplatissement}.

Il est à noter que la \defref{aplatissement} ne s'applique qu'à des Frappes de Processus
pseudo-canoniques. Cependant, des Frappes de Processus avec $k$ classes de priorités quelconques
sont \textit{a fortiori} des Frappes de Processus pseudo-canoniques à condition d'ajouter
une classe vide d'actions de priorité supérieure.
En d'autres termes, il est possible d'ignorer le cas particulier des actions primaires
(mettant à jour les sortes coopératives) si les Frappes de Processus ne sont pas
pseudo-canoniques.

%\todo{À supprimer ? On peut l'intégrer dans la démo du \thmref{bisimulaplatissement}.}

\begin{definition}[Opérateur d'aplatissement
  ($\Fopsymbol_\Fopaplatsubsymbol : \PHh \rightarrow \F$)]
\deflabel{fopaplat}
  L'opérateur d'aplatissement des Frappes de Processus avec $k$ classes de priorités
  est défini par :
  \[\forall h \in \PHh, \Fopaplat{h} \equiv \hitter{h} \wedge \target{h} \wedge
    \left( \bigwedge_{\substack{g \in \PHh^{(n)}\\1 < n < \prio(h)}}
    \neg \left( \hitter{g} \wedge \target{g} \right) \right)\]
\end{definition}

\begin{lemma}
\lemlabel{ppplaysubset}
  Soient $h \in \PHh \setminus \PHh^{(1)}$ et $s \in \PHl$.
  \[\Feval{\Fopphp{h}}{s} \Leftrightarrow
    \big(\exists \mysigma \subseteq s, \Feval{\Fopaplat{h}}{\mysigma}\big)
    \wedge \big(\forall g \in \PHh^{(1)}, \Feval{\neg \Fopphp{g}}{s}\big)
    \enspace.\]
\end{lemma}
%
\begin{proof}
  ($\Rightarrow$)
    Supposons $\Feval{\Fopphp{h}}{s}$.
    Il n'existe donc aucune action primaire jouable dans $s$.
    Par ailleurs, $\Fopphp{h} \Rightarrow \Fopaplat{h}$ donc
    $\Fopaplat{h} \not\equiv \bot$ et, par propriété d'une FND,
    au moins l'une des conjonctions de $\Fopaplat{h}$ est vraie dans $s$.
    On suppose que la $i$\textsuperscript{e} conjonction est vraie dans $s$,
    avec $i \in \segm{1}{\n}$;
    on a alors : $\forall j \in \segm{1}{\m}, p_{i,j} \in s$.
    Soit $\mysigma \in \PHsubl_{\PHdep{i}{h}}$
    avec $\forall b \in \PHdep{i}{h}, \PHget{\mysigma}{b} = \PHget{s}{b}$.
    On a alors immédiatement : $\mysigma \subseteq s$,
    et, par construction de $\PHdep{i}{h}$, $\Feval{\Fopaplat{h}}{\mysigma}$.
  
  ($\Leftarrow$)
    Supposons qu'il existe $\mysigma \subseteq s$ tel que $\Feval{\Fopaplat{h}}{\mysigma}$,
    et qu'aucune action primaire n'est jouable dans $s$.
    On a alors immédiatement $\Feval{\Fopaplat{h}}{s}$
    car ajouter des processus au sous-état
    d'évaluation ne peut pas rendre la propriété fausse.
    De plus, comme aucune action primaire n'est jouable dans $s$, alors
    $\Feval{\left( \bigwedge_{g \in \PHh^{(1)}} \neg (\hitter{g} \wedge \target{g}) \right)}{s}$,
    donc $\Feval{\Fopphp{h}}{s}$.
\end{proof}

\begin{definition}
\deflabel{aplatissement}
  Si $k \in \sNN$ et $\PH = (\PHs; \PHl; \PHa^{\angles{k}})$
  sont des Frappes de Processus pseudo-canoniques avec $k$ classes de priorités,
  on note $\PHflat(\PH) = \oPH = (\ov{\PHs}; \ov{\PHl}; (\ov{\PHa}^{(1)}; \ov{\PHa}^{(2)}))$
  l'\emph{aplatissement} de $\PH$, où :
  \begin{itemize}
    \item $\ov{\PHs} = \PHs \cup \PHs_f$
      où $\PHs_f = \{ f^{h,i} \mid h \in \PHh \wedge \n \geq 1 \wedge i \in \segm{1}{\n} \}$;
    \item $\ov{\PHl} = \left( \bigtimes{a \in \PHs} \PHl_{a} \right) \times
      \left(\bigtimes{f^{h,i} \in \PHs_f} \PHl_{f^{h,i}} \right)$, où
      $\forall f^{h,i} \in \PHs_f, \PHl_{f^{h,i}} =
        \{ f^{h,i}_\mysigma \mid \mysigma \in \PHsubl_{\PHdep{i}{h}} \}$;
    \item $\ov{\PHh}^{(1)} = \PHh^{(1)} \cup
      \{ \PHhit{a_k}{f^{h,i}_\mysigma}{f^{h,i}_{\mysigma'}} \mid
      h \in \PHh \wedge f^{h,i} \in \PHs_f \wedge
      a \in \PHdep{i}{h} \wedge a_k \in \PHl_a \wedge
      f^{h,i}_\mysigma , f^{h,i}_{\mysigma'} \in \PHl_{f^{h,i}} \wedge
      \PHget{\mysigma}{a} \neq a_k \wedge \mysigma' = \mysigma \Cap \{ a_k \} \}$;
    \item $\ov{\PHh}^{(2)}=\{ \PHhit{f^{h,i}_\mysigma}{\target{h}}{\bounce{h}} \mid
      h \in \PHh \setminus \PHh^{(1)} \wedge f^{h,i} \in \PHs_f \wedge
      f^{h,i}_\mysigma \in \PHl_{f^{h,i}} \wedge \Feval{\Fopaplat{h}}{\mysigma} \}$.
  \end{itemize}
  De plus, étant donné un état $\os \in \ov{\PHl}$,
  on note $\unflats{\os} = s$ l'état correspondant dans $\PHl$ :
  $\forall a \in \PHs, \PHget{s}{a} = \PHget{\os}{a}$.
  À l'inverse, étant donné un état $s \in \PHl$,
  $\flats{s} = \os$ est l'état correspondant dans $\ov{\PHl}$ :
  $\forall a \in \PHs, \PHget{\os}{a} = \PHget{s}{a}$
  et $\forall f^{h,i} \in \PHs_f, \PHget{\os}{f^{h,i}} = f^{h,i}_\mysigma$
  avec $f^{h,i}_\mysigma \in \PHl_{f^{h,i}}$
  et $\forall b \in \PHdep{i}{h}, \PHget{\mysigma}{b} = \PHget{s}{b}$.
\end{definition}

Nous notons que l'aplatissement $\PHflat(\PH)$ de toutes Frappes de Processus avec $k$
classes de priorités $\PH = (\PHs, \PHl, \PHh^{\angles{k}})$
sont des Frappes de Processus canoniques.
En effet, une partie des sortes coopératives générées lors de cette traduction proviennent
des Frappes de Processus d'origine, qui sont pseudo-canoniques et sont déjà contraintes de la
même manière.
L'autre partie constitue les sortes coopératives de l'ensemble $\PHs_f$ et leur définition
respecte les \allcr.

\begin{theorem}[$\PH \approx \PHflat(\PH)$]
\thmlabel{bisimulaplatissement}
  Soient $\PH = (\PHs; \PHl; \PHa^{\angles{k}})$ des Frappes de Processus avec $k$
  classes de priorités,
  et $\oPH = \PHflat(\PH) = (\ov{\PHs}; \ov{\PHl}; \ov{\PHa}^{\angles{2}})$ leur aplatissement.
  \begin{enumerate}
    \item \label{php2ph} $\forall s, s' \in \PHl$,
      $s \trans{\PH} s' \Longrightarrow \flats{s} \mtrans{\oPH} \flats{s'}$,
      où $\mtrans{\oPH}$ est une séquence finie de transitions $\trans{\oPH}$.
    \item \label{ph2php} $\forall \os, \os' \in \ov{\PHl}$,
      $\os \trans{\oPH} \os' \Longrightarrow \unflats{\os} = \unflats{\os'} \vee
      \unflats{\os} \trans{\PH} \unflats{\os'} \enspace.$
  \end{enumerate}
\end{theorem}

\begin{proof}
  (\ref{php2ph}) Soient $s, s' \in \PHl$ tels que $s \trans{\PH} s'$.
    Posons $\os = \flats{s}$.
    D'après la dynamique des Frappes de Processus (\defref{play}),
    si $s \trans{\PH} s'$, alors il existe une action $h \in \PHh$ jouable dans $s$,
    telle que $s' = s \PHplay h$.
    On a alors : $\Feval{\Fopphp{h}}{s}$.
    Par construction de $\oPH$ (\defref{aplatissement}) :
    \begin{itemize}
      \item Si $h \in \PHh^{(1)}$, alors il existe $g = h \in \ov{\PHh}^{(1)}$,
        et on a : $\hitter{g} \in \os$ et $\target{g} \in \os$.
      \item Si, au contraire, $h \in \PHh \setminus \PHh^{(1)}$, alors il existe
        $g = \PHhit{f^{h,i}_\mysigma}{\target{h}}{\bounce{h}} \in \ov{\PHh}^{(2)}$.
        De plus, d'après le \lemref{ppplaysubset}, il existe $\mysigma \subseteq s$
        tel que $\Feval{\Fopaplat{h}}{\mysigma}$
        et, par construction de $\os$ (\defref{aplatissement}),
        $\PHget{\os}{f^{h,i}} = f^{h,i}_\mysigma$.
    \end{itemize}
    Dans les deux cas, $g$ est jouable dans $\os$.
    Par la suite, dans l'état $\os \PHplay g$, les seules actions jouables sont celles dans
    $\PHh^{(1)}$ qui mettent à jour les sortes coopératives dans lesquelles
    $\bounce{h} = \bounce{g}$ est impliqué, directement ou indirectement,
    permettant donc d'accéder à l'état $\flats{s'}$ en un nombre fini d'actions.
    Ainsi, $\flats{s} \mtrans{\oPH} \flats{s'}$.
  
  (\ref{ph2php}) soient $\os, \os' \in \ov{\PHl}$ tels que $\os \trans{\oPH} \os'$.
    Posons $s = \unflats{\os}$.
    D'après la dynamique des Frappes de Processus (\defref{play}),
    si $\os \trans{\PH} \os'$, alors il existe une action $g \in \ov{\PHh}$ jouable dans $\os$,
    telle que $\os' = \os \PHplay h$.
    \begin{itemize}
      \item Si $g \in \ov{\PHh}^{(1)} \setminus \PHh^{(1)}$,
        alors $\unflats{\os} = \unflats{\os'}$ car seul le processus actif d'une sorte
        coopérative dans $\PHs_f$ a évolué. % qui n'est pas dans $\PHs$ a évolué.
      \item Sinon, si il existe $h = g \in \PHh^{(1)}$,
        alors $h$ est jouable dans $s$ et : $\unflats{\os} \trans{\PH} \unflats{\os'}$
        avec $\unflats{\os'} = s \play h$.
      \item Autrement, si $g \in \ov{\PHh}^{(2)}$,
        on note : $g = \PHhit{f^{h,i}_\mysigma}{b_j}{b_k}$.
        Par construction de l'aplatissement (\defref{aplatissement}), il existe
        $h \in \PHh$ tel que $\Feval{\Fopaplat{h}}{\mysigma}$.
        Comme $g$ est jouable, cela signifie qu'aucune action de $\PHh^{(1)}$ n'est jouable,
        et notamment que la sorte coopérative $f^{h,i}$ est déjà mise à jour,
        ce qui a pour conséquence que : $\mysigma \subseteq s$.
        Ainsi, d'après \lemref{ppplaysubset}, $h$ est jouable dans $s$,
        car $\Feval{\Fopphp{h}}{s}$, et $\unflats{\os} \trans{\PH} \unflats{\os'}$.
    \end{itemize}
\end{proof}

Nous notons que l'aplatissement donné à la \defref{aplatissement}
est, pour chaque action, exponentiel dans le nombre d'actions non primaires de
priorité supérieure.
En effet, il est nécessaire pour chaque action de convertir la propriété de jouabilité
donnée par la \defref{fopphp} en une FND.
Or la majorité des cas pratiques pour cette conversion sont proches du pire cas,
qui est de complexité exponentielle en fonction du nombre d'atomes dans la propriété.

Pour finir, il est intéressant de noter que l'aplatissement de la \defref{aplatissement}
n'est pas optimal en nombre d'actions et de processus créés dans le modèle final.
En effet, il est possible de simplifier le modèle $\PHflat(\PH)$ de différentes manières,
qui n'ont pas été prises en compte ici pour ne pas alourdir la définition.
Nous donnons dans la suite quelques pistes pour obtenir des modèles plus simples
mais au comportement équivalent, mais n'en donnons pas les preuves.

\subsubsection{Simplification des propriétés d'aplatissement}

La propriété d'aplatissement $\Fopaplat{h}$ d'une action $h = \PHhit{a_i}{b_j}{b_k}$ peut être
simplifiée à l'aide des propriétés suivantes, permettant d'éviter la création
de certains éléments inutiles, comme des actions qui ne sont jamais jouables
ou des coopérations qui sont toujours vraies :
\begin{itemize}
  \item il n'est pas nécessaire de faire apparaître la cible d'une action dans
    sa propriété d'aplatissement car sa présence sera vérifiée par ailleurs,
    au moment du tir effectif de l'action,
    d'où : $b_j \equiv \top$ ;
  \item Tout processus $b_l \neq b_j$ de la même sorte que la cible empêche toujours la jouabilité
    de l'action, donc : $b_l \equiv \bot$ ;
  \item si $c_p, c_q$ sont des processus différents ($c_p \neq c_q$) de la même sorte $c$,
    alors $c_p \wedge c_q \equiv \bot$.
\end{itemize}

\subsubsection{Suppression des sortes coopératives superflues}

Il existe deux cas pour lesquels il est possible de supprimer la sorte coopérative $f^{h,i}$
créée pour une action $h$ dans le modèle aplati :
\begin{itemize}
  \item si $\Fopaplat{h} \equiv \top$, alors l'action $h$ peut être traduite comme étant
    une auto-action (car elle est toujours jouable dès lors que la cible est présente) ;
  \item Si la $i$\textsuperscript{e} conjonction de $\Fopaplat{h}$ consiste
    en un unique élément $p$, alors cette conjonction peut être traduite par une action simple
    de la forme : $\PHhit{p}{\target{h}}{\bounce{h}}$ sans avoir recours à une sorte coopérative
    (étant donné qu'en dehors de la cible, un seul processus, $p$, est requis).
\end{itemize}



\subsection{Aplatissement des Frappes de Processus avec arcs neutralisants}
\seclabel{phan-aplatissement}

À l'instar de la \secref{aplatissement}, il est possible de traduire les Frappes de Processus
avec arcs neutralisants en Frappes de Processus canoniques.
Le procédé est le même car il consiste, pour chaque action,
en un calcul de propriété d'aplatissement
qui est ici identique à la propriété de jouabilité.
En effet, l'opérateur d'aplatissement de cette traduction est égal à l'opérateur de jouabilité 
donné à la \defref{fopphan} : $\Fopsymbol_\Fopaplatsubsymbol = \Fopsymbol_\Fopphansubsymbol$.
Par la suite, une fois cette propriété traduite en FND, il est possible de réutiliser
la traduction de la \defref{aplatissement} afin d'obtenir un modèle canonique
ayant la même dynamique, comme assuré par le \thmref{bisimulaplatissement}.

Cette traduction est elle aussi de complexité exponentielle dans le nombre d'actions
préemptant chaque action.
Cependant, on note que l'utilisation d'arcs neutralisants peut rendre cet aplatissement
beaucoup plus efficace.
En effet, contrairement aux Frappes de Processus avec classes de priorités,
les Frappes de Processus avec arcs neutralisants permettent une définition beaucoup plus fine
des préemptions entre actions.
Une des conséquences sur les modèles créés est un nombre bien moins important de relations
préempteur/préempté entre les actions, rendant la traduction plus efficace.



\subsection{Aplatissement des Frappes de Processus avec actions plurielles}
\seclabel{phm-aplatissement}

Il est possible de traduire les Frappes de Processus avec actions plurielles
en Frappes de Processus canoniques à l'aide d'outils précédemment développés.
En effet, la \defref{phm2php} offre une traduction des Frappes de Processus avec action plurielles
en Frappes de Processus avec 4 classes de priorités, celles-ci pouvant être à leur tour traduites
en Frappes de Processus canoniques à l'aide de la \defref{aplatissement}.
Globalement, cette traduction est donc exponentielle dans le nombre d'actions dans le modèle
initial, car la traduction de chacune d'entre elles crée une sorte coopérative et plusieurs
actions de priorités différentes, qui doivent par la suite être aplaties.

\todoplustard{Conversion directe lorsque $\forall h \in \PHh, \card{\bond{h}} = 1$ ?}



\subsection{Représentation en Frappes de Processus avec actions plurielles}
\seclabel{phcanonique2phm}

Les Frappes de Processus canoniques permettent notamment de représenter des coopérations
entre processus.
Il peut être intéressant d'observer ces coopérations du point de vue des
Frappes de Processus avec actions plurielles, dont le formalisme est particulièrement
adapté à la représentation des coopérations.
Pour toutes Frappes de Processus canoniques $\PH$, nous proposons à la \defref{phcanonique2phm}
une représentation alternative $\PHmult(\PH)$ de ce modèle en Frappes de Processus avec
actions plurielles, et nous montrons au \thmref{bisimulphm} que $\PHmult(\PH)$
possède bien la même dynamique que $\PH$, aux mises à jour de sortes coopératives près.
Cette traduction se base sur l'interprétation de chaque action de priorité 2 du modèle $\PH$ :
\begin{itemize}
  \item Si le frappeur de l'action est un processus de composant ($\components$),
    la traduction est alors triviale ;
  \item Si à l'inverse, le frappeur est un processus de sorte coopérative ($\cs$),
    alors la sorte coopérative est étudiée et il y a autant d'actions multiples
    créées qu'il y a de configurations représentées par ce frappeur.
\end{itemize}
Autrement dit, les actions plurielles permettent d'avantageusement représenter
les sortes coopératives en représentant les ensembles de processus requis pour activer
un processus donné d'une sorte coopérative.
La \figref{livelock-phm} est la traduction en Frappes de Processus avec actions plurielles
du modèle simple de Frappes de Processus canoniques de la \figref{livelock}.

\begin{definition}
\deflabel{phcanonique2phm}
  Soient $\PH = (\PHs; \PHl; (\PHh^{(1)}; \PHh^{(2)}))$ des Frappes de Processus canoniques.
  On pose : $\PHmult(\PH) = (\ov{\PHs}; \ov{\PHl}; \ov{\PHh})$ les Frappes de Processus
  avec actions plurielles telles que :
  \begin{itemize}
    \item $\ov{\PHs} = \components$ ;
    \item $\ov{\PHl} = \bigtimes{a \in \PHs'} \PHl_a$ ;
    \item $\ov{\PHh} = \{ \PHfrappemult{\big( ps \cup \{ \target{h} \} \big)}{\bounce{h}}
      \mid h \in \PHh^{(2)} \wedge ps \in \virtualhitters(h) \}$
    avec, si on note $\hitter{h} = a_i$ :
    \[\virtualhitters(h) =
      \begin{cases}
        \{ \{ a_i \} \} & \text{si } a \in \components \\
        \csState(a_i) & \text{si } a \in \cs
      \end{cases}\]
  \end{itemize}
  De plus, pour tout état $s \in \PHl$,
  $\flats{s} = \os$ est l'état correspondant dans $\ov{\PHl}$ :
  $\forall a \in \ov{\PHs}, \PHget{\os}{a} = \PHget{s}{a}$.
  À l'inverse, étant donné un état $\os \in \ov{\PHl}$,
  on note $\unflats{\os} = s$ l'état correspondant dans $\PHl$ :
  $\forall a \in \components, \PHget{s}{a} = \PHget{\os}{a}$
  et $\forall f \in \cs, \PHget{s}{f} = f_\mysigma$
  avec $f_\mysigma \in \PHl_f$
  et $\forall b \in \compin(b), \PHget{\mysigma}{b} = \PHget{s}{b}$.
\end{definition}

\begin{theorem}[$\PH \approx \PHmult(\PH)$]
\thmlabel{bisimulphm}
  Soient $\PH = (\PHs; \PHl; (\PHh^{(1)}; \PHh^{(2)}))$ des Frappes de Processus canoniques,
  et $\PHmult(\PH) = (\ov{\PHs}; \ov{\PHl}; \ov{\PHh})$.
  \begin{enumerate}
    \item \label{bisimulph2phm} $\forall s, s' \in \PHl$,
      $s \trans{\PH} s' \Longrightarrow
      \flats{s} = \flats{s'} \vee \flats{s} \trans{\PHmult(\PH)} \flats{s'}$,
    \item \label{bisimulphm2ph} $\forall \os, \os' \in \ov{\PHl}$,
      $\os \trans{\PHmult(\PH)} \os' \Longrightarrow
      \unflats{\os} \mtrans{\PH} \unflats{\os'}$,
      où $\mtrans{\PH}$ est une séquence finie de transitions $\trans{\PH}$.
  \end{enumerate}
\end{theorem}

\begin{proof}
  (\ref{bisimulph2phm}) Soient $s, s' \in \PHl$ tels que $s \trans{\PH} s'$.
    Il existe donc une action $h \in \PHh$ telle que $s' = s \play h$.
    \begin{itemize}
      \item Si $h \in \PHh^{(1)}$, alors $\flats{s} = \flats{s'}$.
      \item Sinon, $h \in \PHh^{(2)}$ ; cela signifie qu'aucune action de priorité 1 n'est
        jouable, et qu'il existe $ps \in \virtualhitters(h)$ tel que
        $ps \cup \{ \target{h} \} \subseteq s$.
        D'après la \vdefref{phcanonique2phm}, il existe une action $g \in \ov{\PHh}$
        telle que $g = \PHfrappemult{ps \cup \{ \target{h} \}}{\bounce{h}}$.
        Ainsi, $g$ est jouable dans $\flats{s}$ et :
        $\flats{s} \play g = \flats{s} \recouvre \frappeur{h} = \flats{s'}$.
        Donc $\flats{s} \trans{\PHmult(\PH)} \flats{s'}$.
    \end{itemize}
    
  (\ref{bisimulphm2ph}) Soient $\os, \os' \in \ov{\PHl}$ tels que $\os \trans{\PHmult(\PH)} \os'$.
    Il existe donc une action $g \in \ov{\PHh}$ telle que $\os' = \os \play g$.
    De plus, cela signifie que $\frappeur{g} \subseteq \os$,
    donc $\frappeur{g} \subseteq \unflats{\os}$.
    Par construction de $\PHmult(\PH)$, il existe une action $h \in \PHh^{(2)}$
    telle que $\frappeur{g} = ps \cup \{ \cible{h} \}$ et $\bond{h} = \bond{g}$,
    avec $ps \in \virtualhitters(h)$.
    Par définition de $\unflats{\os}$, toutes les sortes coopératives sont mises à jour
    dans $\unflats{\os}$, ce qui fait que $h$ est jouable dans $\unflats{\os}$.
    Par ailleurs, d'après le \vlemref{update},
    il existe un scénario $\delta \in \Sce(\unflats{\os} \play h)$
    tel que $\unflats{\os} \play h \play \delta = \update(\unflats{\os} \play h)$.
    Enfin, par définition de $\update$ (\vdefref{update})
    on a : $\unflats{\os} \play h \play \delta = \unflats{\os'}$.
    Ainsi : $\unflats{\os} \mtrans{\PH} \unflats{\os'}$.
\end{proof}

\begin{figure}
  \centering
  \scalebox{1.2}{
  \begin{tikzpicture}[apdotsimple/.style={apdot}]
    \TSort{(0,0)}{a}{2}{l}
    \TSort{(0,3)}{b}{2}{l}
    \TSort{(4,2)}{c}{2}{r}
    
%     \THit{ab_3}{}{c_0}{.west}{c_1}
    
    \TActionPlur{a_1, b_1}{c_0.west}{c_1.south west}{}{2,2}{left}
    
%     \path[bounce, bend left]
%       \TBounce{c_0}{}{c_1}{.south west}
%     ;
    
    \TAction{a_1}{a_1.west}{a_0.north west}{selfhit}{right}
    \TAction{b_1}{b_1.west}{b_0.north west}{selfhit}{right}
    \TAction{a_0.south west}{b_0.west}{b_1.south west}{bend left=90}{left}
    \TAction{b_0}{a_0.west}{a_1.south west}{bend right=50}{left}
    
    \TState{a_0, b_0, c_0}
  \end{tikzpicture}
  }
  \caption{\figlabel{livelock-phm}%
    Frappes de Processus avec actions plurielles $\PHmult(\PH)$ issues de la traduction
    des Frappes de Processus canoniques $\PH$ de la \vfigref{livelock}
    d'après la \defref{phcanonique2phm}.
    Chaque action plurielle est représentée par un point relié aux frappeurs invariants
    par des arcs (sans flèche) et aux cibles et bonds par un couple de flèches
    (respectivement en trait plein puis en trait pointillé).
    Ici, l'action plurielle $\PHfrappemults{a_1, b_1, c_0}{c_1}$
    remplace la sorte coopérative $ab$
    et l'action $\PHfrappe{ab_{11}}{c_0}{c_1}$ de la \figref{livelock},
    qui permettaient de modéliser la coopération.
    Les processus grisés présentent un exemple d'état de départ :
    $\PHstate{a_0, b_0, c_0}$.
  }
\end{figure}

