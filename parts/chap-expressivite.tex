% Expressivité du PH et positionnement par rapport à d'autres formalismes

\chapter{Expressivité des Frappes de Processus
  et positionnement par rapport à d'autres formalismes}
\chaplabel{expressivite}

\chapeaupublis{
  Nous montrons dans ce chapitre que les différentes sémantiques de Frappes de Processus
  sont équivalentes à un certain nombre de formalismes répandus.
  Nous prouvons notamment que les Frappes de Processus canoniques permettent de
  représenter tout réseau discret asynchrone, un formalisme très répandu
  dans la représentation des réseaux de régulation biologique.
  À l'inverse, nous proposons une méthode pour inférer les modèles de Thomas sous-jacents à
  des Frappes de Processus canoniques :
  il est possible d'inférer tous les types d'influence et les paramètres discrets,
  et d'énumérer tous les modèles compatibles avec ces résultats.
  Par ailleurs, les Frappes de Processus avec actions plurielles s'avèrent être
  équivalentes aux réseaux d'automates synchronisés.
  Enfin, nous proposons des traductions vers les réseaux de Petri et le formalisme Biocham.
}{%
  La traduction vers le modèle de Thomas a fait l'objet de deux publications :
  \cite*{FPIMR12-LDSSB} et \cite*{FPIMR12-CMSB}
  et d'une soumission en journal en cours de \textit{review}.
  La traduction depuis les réseaux discrets asynchrones a,
  quant à elle, été publiée dans \cite*{FPMR13-CS2Bio}.
}



Le sujet du chapitre courant est l'expressivité des Frappes de Processus,
dans les différentes sémantiques présentées au \chapref{sem},
par rapport à d'autres modélisations couramment utilisées.
Ce travail de positionnement comporte un intérêt évident lorsqu'il s'agit d'interagir
avec d'autres types de modèles.
Ainsi, les Frappes de Processus peuvent bénéficier des capacités d'analyse d'autres
modélisations, moyennant une traduction, qui peuvent s'avérer efficaces pour
certains problèmes.
À l'inverse, traduire un formalisme en Frappes de Processus permet par exemple
de pouvoir y appliquer les méthodes de représentation et d'analyse proposées,
et notamment l'analyse statique détaillée au \chapref{as}.

Nous présentons ainsi à la \secref{trad-rda} une traduction des réseaux discrets asynchrones
en Frappes de Processus canoniques.
Cette traduction est fondamentale pour l'étude des réseaux de régulation biologique
car elle permet la traduction d'un modèle très utilisé directement en un formalisme
de Frappes de Processus qu'il est possible d'étudier à l'aide de l'analyse statique.
La \secref{trad-thomas} propose, à l'inverse, une méthode pour inférer l'ensemble des modèles
de Thomas sous-jacents à un modèle de Frappes de Processus canoniques.
L'inférence du graphe des interactions, par exemple,
conserve certaines propriétés comme la présence de cycles
exploitables par les résultats relatifs aux conjectures de Thomas.
Par ailleurs, l'inférence des paramètres discrets permet de restreindre l'espace des possibles
en matière de paramétrisations compatibles avec la dynamique du modèle étudié,
facilitant ainsi la recherche de la paramétrisation représentant le comportement recherche.
Les modèles de Thomas compatibles avec ces deux inférences sont ceux dont la paramétrisation
est complétée lorsque certains paramètres n'ont pas pu être inférés ;
nous proposons une méthode pour les énumérer en accord toujours avec la dynamique
de modèle de Frappes de Processus canoniques étudié.

Nous nous intéressons par ailleurs aux liens avec d'autres modélisations
qui ne sont pas totalement asynchrones,
à l'aide des Frappes de Processus avec actions plurielles.
Nous montrons notamment à la \secref{phm2an} que ce type de Frappes de Processus
est équivalent aux plus classiques automates synchronisés.
Nous donnons pour cela les deux traductions adéquates et exhibons les preuves correspondantes.
Par ailleurs, nous proposons aussi à la \secref{trad-petri}
une traduction des Frappes de Processus avec actions
plurielles vers les réseaux de Petri utilisant des arcs de lecture et des arcs inhibiteurs.
Cette traduction permet de montrer le lien d'inclusion entre les deux formalismes,
et ouvre des perspectives dans l'utilisation des réseaux de Petri pour l'analyse des réseaux
de régulation biologique représentés sous la forme de Frappes de Processus.
Pour finir, nous proposons à la \secref{trad-biocham} une traduction des systèmes d'équations tels
que définis dans le formalisme de Biocham vers les Frappes de Processus avec actions plurielles.
Nous nous intéressons naturellement à la sémantique booléenne de Biocham,
qui est par ailleurs asynchrone et qui ajoute une couche supplémentaire d'indéterminisme
sur chaque réaction (en autorisant la consommation d'une partie arbitraire des réactifs).
Cette traduction tire naturellement parti des similitudes entre Biocham et les Frappes
de Processus avec actions plurielles, et permet d'affirmer
qu'un tel système d'équations biochimiques peut toujours être exprimé
en Frappes de Processus avec actions plurielles,
et donc être analysé avec les outils adéquats.

Les résultats de ce chapitre sont représentés par les différentes flèches fines
entre les différentes formes de Frappes de Processus représentés 
et les autres formalismes discrets de la \vfigref{contrib-equivalences}.



% Traduction depuis les réseaux discrets asynchrones}
% Traduction depuis les réseaux discrets asynchrones}

\section{Traduction depuis les réseaux discrets asynchrones}
\seclabel{trad-rda}

Nous proposons dans cette section une traduction des réseaux discrets asynchrones
en Frappes de Processus canoniques, et nous en montrons la validité
par une preuve de bisimulation faible.
Les réseaux discrets asynchrones ont été préalablement définis à la \vdefref{rda-def}
du \chapref{etatdelart}.
Il s'agit de modèles proches du modèle de Thomas, mais en comportant pas de restriction unitaire
de la dynamique, et présentant des fonctions d'évolution en lieu et place de paramètres discrets.
Un réseau discret asynchrone se présente sous la forme d'un couple $\RDA = (\GI; F)$
où $\GI = (\components; E)$ est un graphe des interactions
et $F$ est un ensemble de fonctions $f_x : \RRBreg{x} \rightarrow \segm{0}{l_x}$
pour tout composant $x \in \components$,
où $\RRBreg{x}$ est l'ensemble des prédécesseurs de $x$ dans le graphe des interactions.
La dynamique d'un tel réseau est la suivante : il existe une transition $\RRBtransrda{s}{s'}$
si et seulement si un unique composant $x$ évolue entre $s$ et $s'$
de façon à ce que : $\RRBget{s'}{x} = f_x(s)$.
Les Frappes de Processus canoniques, définies à la \vsecref{phcanonique},
permettent une représentation presque immédiate des réseaux discrets asynchrones,
à condition de créer les sortes coopératives adéquates.

\myskip

% Nous complétons dans un premier temps la définition d'un réseau discret asynchrone par
% la notion de dépendances d'un composant.
% En effet, même si les fonctions d'évolution de chaque composant sont sur $\RRBstates$,
% elles ne dépendent généralement 
% pour tout composant $x \in \components$, $\RDAdep(x)$ est l'ensemble des composants dont
% la valeur de $f_x$ dépend véritablement,
% 
% \begin{definition}[Dépendance]
%   Soit $\RDA = (\GI; F)$ un réseau discret asynchrone, avec $\GI = (\components; E)$.
%   Pour tout composant $x \in \components$,
%   on note $\DNdep(x) \subseteq \components$ l'ensemble des composants dont
%   la valeur de $f_x$ dépend véritablement,
%   c'est-à-dire l'ensemble minimal tel que :
%   \[\forall s, s' \in \RRBstates,
%     \big(\forall y \in \DNdep(f_x), \get{s}{y} = \get{s'}{y}\big) \Longrightarrow
%     f_x(s) = f_x(s')\]
% \end{definition}

La traduction proposée à la \defref{rda2ph}
associe deux sortes à chaque composant $a$ dans $\RDA$ :
\begin{itemize}
  \item une sorte du même nom pour représenter ce composant,
  \item une sorte coopérative $f^a$ représentant sa fonction d'évolution $f_a$,
    et dont les états sont donc une combinaison des états de ses régulateurs.
\end{itemize}
De plus, les actions primaires sont définies de façon à correctement mettre à jour
les sortes coopératives,
et les actions secondaires le sont de façon à ce que chaque sorte coopérative $f^a$
interagisse avec son composant $a$ de la façon dont la fonction d'évolution
correspondante le permet.
Nous montrons de plus au \thmref{bisimulationrda2ph} que le modèle obtenu
est faiblement bisimilaire au réseau discret asynchrone d'origine.

\begin{definition}[Frappes de Processus équivalentes ($\rdatophsymbol$)]
\deflabel{rda2ph}
  Soit $\RDA = (\GI; F)$ un réseau discret asynchrone, avec $\GI = (\components; E)$.
  On note $\rdatoph = (\PHs; \PHl; (\PHh^{(1)}; \PHh^{(2)}))$
  les Frappes de Processus canoniques équivalentes
  à $\RDA$, définies par :
  \begin{itemize}
    \item $\PHs = \components \cup \{ f^a \mid a \in \components \}$,
    \item $\PHl = \bigtimes{a \in \components} \PHl_{a} \times
      \bigtimes{a \in \components} \PHl_{f^a}$ l'ensemble des états , où :
      \begin{align*}
      \forall a \in \components&, \PHl_{a} = \{ a_i \mid i \in \segm{0}{l_a} \} \\
      \forall a \in \components&, \PHl_{f^a} = \begin{cases}
          \PHsubl_{\RRBreg{a}} & \text{ si } \RRBreg{a} \neq \emptyset \\
          \{ f^a_\emptyset \}  & \text{ sinon }
        \end{cases} \enspace,
      \end{align*}
    \item $\PHh^{(1)} = \{ \PHfrappe{b_k}{f^a_\mysigma}{f^a_{\mysigma'}} \mid
      a \in \components \wedge b \in \RRBreg{a} \wedge
      b_k \in \PHl_{b} \wedge f^a_\mysigma \in \PHl_{f^a} \wedge
      \get{\mysigma}{b} \neq b_k \wedge \mysigma' = \mysigma \recouvre b_k \}$,
    \item $\PHh^{(2)} = \{ \PHfrappe{f^a_\mysigma}{a_j}{a_{k}} \mid
      a \in \components \wedge f^a_\mysigma \in \PHl_{f^a} \wedge
      a_j, a_k \in \PHl_{a} \wedge j \neq k \wedge f_a(\decode \mysigma) = k \}$.
  \end{itemize}
  
  Pour tout état $s \in \RRBstates$ de $\RDA$,
  $\encode{s} = \os$ est l'état correspondant dans $\rdatoph$, défini par :
  $\forall a \in \components, \get{s}{a} = k \Rightarrow \get{\os}{a} = a_k$
  et
  $\forall a \in \components, \get{\os}{f^a}=f^a_\mysigma$
  avec $f^a_\mysigma \in \PHl_{f^a}$
  et $\forall b \in \RRBreg{a}, \get{\mysigma}{b} = \get{\os}{b}$.

  À l'inverse, pour tout état $\os \in \PHsubl$ de $\rdatoph$,
  $\decode{\os} = s$ est l'état correspondant dans $\RDA$ avec :
  $\forall a \in \sortes{\os}, \get{\os}{a} = a_k \Rightarrow \get{s}{a} = k$.
\end{definition}

\begin{theorem}[$\RDA \approx \rdatoph$]
\thmlabel{bisimulationrda2ph}
  Soit $\RDA = (\GI; F)$ un réseau discret asynchrone.
  On a :
  \begin{enumerate}
    \item \label{rda2ph} $\forall s, s' \in \RRBstates$,
      $s \RDAtrans s' \Longrightarrow \encode{s} \mtrans{\PH} \encode{s'}$,
      où $\mtrans{\PH}$ est une séquence finie de transitions $\trans{\PH}$.
    \item \label{ph2rda} $\forall \os, \os' \in \PHl$,
      $\os \trans{\PH} \os' \Longrightarrow
        \decode{\os} = \decode{\os'} \vee \decode{\os} \RDAtrans \decode{\os'}$
  \end{enumerate}
\end{theorem}

\begin{proof}
  On pose : $\rdatoph = (\PHs; \PHl; (\PHh^{(1)}; \PHh^{(2)}))$.
  
  (\ref{rda2ph}) Soient $s, s' \in \RRBstates$ tels que $s \RDAtrans s'$.
    Cela signifie qu'il existe un composant $a \in \components$ tel que :
    $\RRBget{s'}{a} = f_a(\mysigma)$ et
    $\forall b \in \components, b \neq a \Rightarrow \RRBget{s}{b} = \RRBget{s'}{b}$,
    où $\mysigma \in \PHsubl_{\RRBreg{a}}$ tel que $\mysigma \subseteq \encode{s}$.
    Posons : $j = \RRBget{s}{a}$ et $k = \RRBget{s'}{a}$ ;
    d'après la \defref{rda2ph}, il existe une action
    $\PHfrappe{f^a_\mysigma}{a_j}{a_k} \in \PHh^{(2)}$
    avec $f_a(\mysigma) = k$
    Par définition de $\encode{s}$, on a $a_k \in \encode{s}$
    et $f^a_\mysigma \in \encode{s}$,
    et aucune action de $\PHh^{(1)}$ n'est jouable dans $\encode{s}$ ;
    ainsi, $h$ est jouable dans $\encode{s}$, d'où : $\encode{s} \PHtrans \encode{s} \play h$.
    De plus, d'après le \vlemref{update},
    $\encode{s} \play h \mtrans{\PH} \update(\encode{s} \play h)$.
    Enfin, comme $\update$ met à jour des sortes coopératives, on a :
    $\update(\encode{s} \play h) = \encode{s'}$.
  
  (\ref{ph2rda}) Soient $\os, \os' \in \PHl$ tels que $\os \trans{\PH} \os'$.
    Cela signifie qu'il existe une action $h \in \PHh$ telle que : $s' = s \play h$.
    Si $h \in \PHh^{(1)}$, alors $\decode{\os} = \decode{\os'}$, par définition
    de $\decode{\os}$.
    En revanche, si $h \in \PHh^{(2)}$, alors d'après la \defref{rda2ph},
    il existe $a \in \components$, $\mysigma \in \PHsubl_{\RRBreg{a}}$
    et $a_j, a_k \in \PHl_{a}$, tels que :
    $h = \PHfrappe{f^a_\mysigma}{a_j}{a_{k}}$, avec $f_a(\decode \mysigma) = k$ et $j \neq k$.
    Ainsi, $\forall b \in \RRBreg{a}$,
    $\PHget{\mysigma}{b} = b_i \Rightarrow \RRBget{\decode{\os}}{b} = i$.
    D'où : $\decode{\os} \RDAtrans \decode{\os'}$
    car $f_a(\decode \mysigma) = k$.
\end{proof}


% Inférence du modèle de Thomas
% Inférence du modèle de Thomas

\section{Inférence du modèle de Thomas}
\seclabel{trad-thomas}

\TODO

\todo{Nous notons pour finir que la complexité de cette méthode est exponentielle dans le nombre
de régulateurs de chaque composant, et linéaire dans le nombre total de composants.
\footnote{Ce travail a été réalisé dans le cadre d'une collaboration avec Katsumi Inoue.
Cette collaboration a débuté par un stage doctoral de trois mois dans l'Inoue Laboratory,
au National Institute of Informatics (Tokyo, Japon).}}

\todo{Citer FPIMR-10}

Dans toute cette section, nous considérons un modèle de Frappes de Processus canoniques
$\PH = (\PHs; \PHl; (\PHh^{(1)}; \PHh^{(2)}))$.

\subsection{Inférence du graphe des interactions}
\seclabel{trad-thomas-gi}

Un graphe des interactions (\vdefref{thomas-gi}) est une représentation abstraite des
influences directes, positives ou négatives, entre les composants d'un système.
Comme discuté à la \vsecref{thomas-analyse}, le graphe des interactions permet
de caractériser efficacement les propriétés dynamiques globales du système,
à l'aide notamment de résultats comme les conjectures de Thomas,
qui apportent des résultats sur la présence d'oscillations ou d'états stables multiples.

Dans le cas d'un processus de modélisation d'un réseau de régulation biologique, le modèle
de Thomas est le point de départ e la spécification du modèle.
Cependant, il est courant que le graphe des interactions initialement conçu contienne des
influences qui n'ont pas d'impact sur la dynamique.
La méthode que nous proposons dans la suite s'appuie directement sur la dynamique d'un modèle
de Frappes de Processus canoniques, ce qui produit des graphes des interactions minimaux,
et permet d'affiner les conclusions de telles méthodes d'analyse statique.

L'intuition de cette inférence est que seuls les composants (les sortes dans $\components$)
figureront dans le graphe des interactions ;
les sortes coopératives (dans $\cs$) sont uniquement étudiées pour comprendre les actions
«~indirectes~» entre composants.



\subsubsection{Frappes de Processus bien-formées}

Nous notons que dans cette section, les indices des processus de composants
possèdent une importance particulière,
notamment pour contraindre le fait que la dynamique doit être unitaire (\crref{unitaire}).
Autrement dit, si on suppose que ces indices représentent des niveaux d'expression discrets
ordonnés,
par exemple si $b_0$, $b_1$ et $b_2$ représentent le fait que le composant $b$
est présent respectivement en faible, moyenne et forte concentration,
alors une action de la forme $\PHfrappe{a_1}{b_0}{b_2}$ n'est pas autorisée ;
en revanche, deux actions $\PHfrappe{a_1}{b_0}{b_1}$ et $\PHfrappe{a_1}{b_1}{b_2}$ le sont.
Naturellement, toute autre relations d'ordre entre les indices est admissible,
à condition qu'une contrainte d'unicité similaire puisse être définie.

\begin{equation}
\components = \{a \in \PHs \mid \nexists \PHfrappe{b_i}{a_j}{a_k} \in \PHa, |j - k| > 1\} \\
\eqlabel{PH-components}
\end{equation}

\begin{critere}[Dynamique unitaire]
\crlabel{unitaire}
  Toutes les actions secondaires de $\PH$ ne font pas de bond
  à plus d'un processus d'écart :
  $\forall \PHfrappe{a_i}{b_j}{b_k} \in \PHh^{(2)}, \card{j - k} = 1$
\end{critere}

\begin{example}
  Les Frappes de Processus canoniques représentées à la \figref{infer-ex-ph}
  possèdent une dynamique unitaire, et sont donc compatibles
  avec l'inférence du graphe des interactions proposée à la section suivante.
\end{example}

\begin{remark}
  Le \crref{unitaire} est naturellement vérifié pour tout modèle booléen,
  c'est-à-dire tel que : $\forall a \in \components, \card{\PHl_a} = 2$.
\end{remark}

\begin{remark}
  Il est possible de ne pas prendre en compte le \crref{unitaire},
  à condition de s'affranchir de l'aspect unitaire de la dynamique du modèle de Thomas.
  Les résultats de cette section restent alors théoriquement applicables.
\end{remark}

Nous considérons dans la suite que les Frappes de Processus canoniques $\PH$ respectent
le \crref{unitaire}.



\subsection{Inférence des interactions}
\seclabel{infer-gi}


L'inférence de cette section est directement inspirée des travaux de \citeasnoun{Richard10},
qui déduit les influences d'un graphe des interactions en fonction des évolutions des différents
composants, selon l'état de ses régulateurs.

\newcommand{\myupsilon}{\upsilon}

Pour tout composant $a$,
les \emph{prédécesseurs} de $a$, notés $\pred(a)$,
sont toutes les sortes ayant au moins une action frappant $a$.
Les \emph{régulateurs} de $a$, en revanche, notés $\reg(a)$, sont tous les composants
qui influent sur $a$, soit directement, soit à travers une sorte coopérative.
Il est à noter que les régulateurs définis de cette manière seront potentiellement
des régulateurs de $a$ dans le modèle de Thomas inféré,
tels que définis \vpageref{regulateurs},
ce qui explique pourquoi ces deux définitions sont proches.

\begin{align*}
  \forall a \in \components, \pred(a) &\DEF \{ b \in \PHs \mid
    \exists h \in \PHh, \sorte{\frappeur{h}} = b \wedge
    \sorte{\cible{h}} = a \} \\
  \forall a \in \components, \reg(a) &\DEF \{ \compin(b) \mid
    b \in \pred(a) \}
\end{align*}
Où $\compin(b)$ fait référence aux composants qui régulent la sorte coopérative $b$,
autrement dit aux sortes que $b$ représente
(cf.~\vdefref{comp}).

L'étude des influences d'un composant $b$ régulant un autre composant $a$ nécessite d'étudier
le groupe de régulateurs de $b$ qui vont influencer conjointement $a$.
Ces groupes de régulateurs sont aisément déterminés en observant les sortes coopératives.
Nous proposons ici de définir les groupes de régulateurs comme étant les composants connexes
d'un graphe reliant tous les régulateurs de $a$ qui sont représentés par une même sorte
coopérative :
\[
  \forall a \in \components,
  X(a) \DEF \mathcal{C} \big( (\reg(a),
  \{ \{b,c\} \subset \compin(\myupsilon) \mid
  \myupsilon \in \pred(a) \cap \cs \}) \big)
\]
Où $\mathcal{C}(G)$ représente l'ensemble des composantes connexes du graphe non orienté $G$.

Pour étudier l'influence d'un groupe de régulateurs $g$ sur un composant $a$,
nous effectuons une analyse exhaustive de toutes les configurations possibles de $g$.
Pour cela, il est nécessaire de définir un sous-état $\sigma$ sur les sortes de $g$,
et et de compléter ce sous-état par les processus de sorte coopérative
qui représentent l'état des composants dans $g$.
Nous définissons pour cela l'ensemble $\allFocals{a}{g}{\sigma}$
qui contient l'état des sortes de $g$, celui de $a$, et celui de toutes les sortes
coopératives frappant $a$.

%allFocals
\begin{align*}
  &\forall a \in \components, \forall g \in X(a), \forall \sigma \in \PHsubl_{g \cup \{ a \}}, \\
  & \quad
  \allFocals{a}{g}{\sigma} = \{ \PHget{\sigma}{b} \mid b \in \pred(a) \cap \components \}
  \cup \{ \pfp_\sigma(b) \mid b \in \pred(a) \cap \cs \}
\end{align*}
Avec :
\[
  \pfp_\sigma(b) = \pfp_{s \recouvre \sigma}(b)
\]
où le choix de $s \in \PHl$ est indifférent d'après le point (\ref{csss}) de la \vdefref{cs}.

Enfin, il est possible d'étudier localement la dynamique de $a$ en fonction du sous-état
$\sigma$ d'un groupe de régulateurs $g$ donné ;
cette dynamique locale se concentre donc uniquement sur les actions frappant $a$.
En effet, en faisant varier l'un des composants $b \in g$ et en observant le résultat
sur l'évolution de $a$ (tendance à l'augmentation ou à la diminution de son niveau d'expression),
il est possible d'en déduire l'influence locale de $b$ sur $a$ pour un niveau d'expression
de $b$ donné.
Pour cela, nous appelons $\irB_a(\sigma)$ l'ensemble des processus vers lesquels $a$ peut évoluer
depuis le sous-état $\sigma$ ;
naturellement, si aucune action ne frappe $a$ dans $\sigma$,
alors $\irB_a(\sigma) = \PHget{\sigma}{a}$.

\begin{align*}
  &\forall g \in X(a), \forall \sigma \in \PHsubl_{g \cup \{ a \}},
  \irB_a(\sigma) \DEF 
  \begin{cases}
    \irF_a(\sigma)
      & \text{ si } \irF_a(\sigma) \neq \emptyset\\
    \{ \PHget{\sigma}{a} \}
      & \text{ si } \irF_a(\sigma) = \emptyset
  \end{cases}\\
  &\text{où : } \irF_a(\sigma) \DEF \{ a_k \in \PHl_a \mid
    \exists b \in \PHs, \exists \PHfrappe{b_i}{a_j}{a_k} \in \PHa,
  \{ b_i, a_j \} \subset \allFocals{a}{g}{\sigma} \}\\
\end{align*}


La \propref{inference-edges} détaille l'inférence de toutes les influences locales existant
entre les composants, c'est-à-dire celles qui se produisent pour un seuil donné $t$.
L'idée principale derrière cette inférence est la suivante :
s'il existe une une influence
positive (\resp négative) d'un composant $b$ sur un autre composant $a$,
alors augmenter le niveau d'expression de $b$
va potentiellement faire faire augmenter (\resp diminuer) le niveau d'expression de $a$,
au moins dans certaines configurations (\eqref{edges-inference}).
Ainsi, ces influences locales se séparent en influences positives et négatives,
ce qui représente de potentiels arcs dans le graphe des interactions final.
De plus, l'étude des influences sur les groupes de régulateurs d'un composant $a$
permet aussi d'étudier les auto-influences de $a$
(\eqref{edges-inference-auto})
ce qui permettra potentiellement d'inférer des auto-arcs.
Finalement, il est nécessaire d'étudier le cas particulier où $a$ ne possède pas de régulateurs
(\eqref{edges-inference-noreg}).
Nous notons que cette méthode ignore naturellement tous les cas où il n'est pas possible
de distinguer une influence d'un composant sur un autre.

\begin{proposition}[Inférence des influences]
\proplabel{inference-edges}
  Nous définissons l'ensemble $\hat{E}_+$ (\resp $\hat{E}_-$)
  des \emph{influences locales positives} (\resp \emph{négatives})
  pour tout composant $a \in \components$ par :
  % Arcs a -> b, a ≠ b
  \begin{align}
    \begin{split}\eqlabel{edges-inference}
      \forall b \in \reg(a), \forall s \in \{ +, - \}, \\
      b \xrightarrow{t+1} a \in \hat{E}_s \Longleftrightarrow\ & \exists g \in X(a), b \in g,
      \exists \sigma \in \PHsubl_{g \cup \{ a \}}, \\
        &\qquad \{ b_t, b_{t+1} \} \subset \PHl_b \wedge b_t \in \sigma,\\
        &\qquad \exists a_j \in \irB_a(\sigma), \exists a_k \in \irB_a(\sigma\{b_{t+1}\}), \\
        &\qquad s = \f{signe}(k - j)
    \end{split}
    \end{align}
    % Auto-arcs depuis les groupes de régulateurs
    \begin{align}
    \begin{split}\eqlabel{edges-inference-auto}
      \forall s \in \{ +, - \}, \quad\qquad\qquad \\
      a \xrightarrow{t+1} a \in \hat{E}_s \Longleftrightarrow\ & \exists g \in X(a),
      \exists \sigma \in \PHsubl_{g \cup \{ a \}}, \\
        &\qquad \{ a_t, a_{t+1} \} \subset \PHl_a \wedge a_t \in \sigma,\\
        &\qquad \exists a_j \in \irB_a(\sigma), \exists a_k \in \irB_a(\sigma\{a_{t+1}\}), \\
        &\qquad s = \f{signe}(k - j)
    \end{split}
    \end{align}
    % Auto-arcs des composants sans prédécesseurs
    \begin{align}
    \begin{split}\eqlabel{edges-inference-noreg}
      \forall s \in \{ +, - \}, \quad\qquad\qquad \\
      a \xrightarrow{t+1} a \in \hat{E}_s \Longleftrightarrow\ & \reg(a) = \emptyset \wedge
        \{ a_t, a_{t+1} \} \subset \PHl_a, \\
        &\qquad \exists a_j \in \irB_a(\etat{a_t}), \exists a_k \in \irB_a(\etat{a_{t+1}}), \\
        &\qquad s = \f{signe}(k - j)
    \end{split}
  \end{align}
  où : $\f{signe}(x) = \begin{cases}
    + & \text{ si $x > 0$} \\
    - & \text{ si $x < 0$} \\
    0 & \text{ if $x = 0$}
  \end{cases}$
\end{proposition}

Nous sommes alors en mesure d'inférer les arcs du graphe des interactions final,
à partir de ces ensembles d'influences locales positives et négatives.
En effet, nous pouvons inférer une influence (globale) positive ou négative
d'un composant vers un autre
s'il n'existe que des influences locales correspondantes du même signe.
Une influence non-signée est inférée si, à l'inverse, il existe au moins deux influences
locales correspondantes de signes différents.
Enfin, le seuil de chaque influence (quel que soit son signe)
est égal au seuil minimum pour lequel une influence locale a été trouvée.
Nous formalisons cette inférence dans la \propref{inference-gi}.

\begin{proposition}[Inférence du graphe des interactions]
\proplabel{inference-gi}
  Nous inférons $\GI = (\components; E)$ à l'aide de la \propref{inference-edges} comme suit :
  \begin{align*}
    E_+ &= \{ \arc{a}{+}{t}{b} \mid \nexists a \xrightarrow{t'} b \in \hat{E}_-
      \wedge t = \min \{ r \mid a \xrightarrow{r} b \in \hat{E}_+ \}\} \\
    E_- &= \{ \arc{a}{-}{t}{b} \mid \nexists a \xrightarrow{t'} b \in \hat{E}_+
      \wedge t = \min \{ r \mid a \xrightarrow{r} b \in \hat{E}_- \}\} \\
    E_\uns &= \{ \arc{a}{\uns}{t}{b} \mid \exists a \xrightarrow{t'} b \in \hat{E}_+ \wedge
      \exists a \xrightarrow{t''} b \in \hat{E}_- \\
      & \qquad\qquad\qquad \wedge t = \min \{ r \mid
      a \xrightarrow{r} b \in \hat{E}_- \cup \hat{E}_+ \} \}
  \end{align*}
\end{proposition}




\begin{figure}[ht]
\centering
\scalebox{1.3}{
\begin{tikzpicture}
  \path[use as bounding box] (-4,-1.9) rectangle (4.5,3.9);

  \TSort{(0,0)}{a}{3}{l}
  \TSort{(3, 3)}{b}{2}{t}
  \TSort{(3,-1)}{c}{2}{b}

  \TSetTick{bc}{0}{00}
  \TSetTick{bc}{1}{01}
  \TSetTick{bc}{2}{10}
  \TSetTick{bc}{3}{11}
  % \TSetSortLbcel{bc}{$\neg a\wedge b$}
  \TSort{(-3,-0.5)}{bc}{4}{l}

  \THit{bc_3}{}{a_1}{.north west}{a_2}
  \THit{bc_0}{}{a_1}{.south west}{a_0}
  \path[bounce]
  \TBounce{a_1}{bend left}{a_2}{.south west}
  \TBounce{a_1}{bend right}{a_0}{.north west}
  ;

  \THit{b_0}{}{a_2}{.east}{a_1}
  \THit{b_1}{}{a_0}{.north east}{a_1}
  \path[bounce]
  \TBounce{a_2}{bend left}{a_1}{.north east}
  \TBounce{a_0}{bend right=20}{a_1}{.south}
  ;

  \THit{c_0}{bend right}{a_2}{.south east}{a_1}
  \THit{c_1}{bend right}{a_0}{.east}{a_1}
  \path[bounce]
  \TBounce{a_2}{bend left=20}{a_1}{.north}
  \TBounce{a_0}{bend right=30}{a_1}{.south east}
  ;

  \path[]
    (1.9,-1.3) edge[bend left=10,coopupdate] (-2.2,-0.7)
    (1.9, 3.3) edge[bend right=10,coopupdate] (-2.2,3)
  ;

  \THit{a_2}{bend left,out=40,in=80}{b_1}{.north west}{b_0}
  \path[bounce, bend right]
  \TBounce{b_1}{}{b_0}{.east}
  ;
\end{tikzpicture}
}
\caption{\figlabel{infer-ex-ph}%
  Exemple de Frappes de Processus canoniques avec trois composants ($a$, $b$ et $c$)
  et une sorte coopérative ($bc$).
  La dynamique de ce modèle est unitaire car il respecte bien le \vcrref{unitaire}.
  L'inférence du graphe des interactions peut donc être effectuée sur ce modèle.
}
\end{figure}



\begin{example}
  L'application de l'inférence du graphe des interactions aux Frappes de Processus canoniques
  de la \figref{infer-ex-ph} donne le graphe représenté à la \figref{infer-ig},
  contenant les arcs suivants :
  \begin{align*}
    E_+ &= \{\arcf{b}{+}{1}{a}, \arcf{c}{+}{1}{a}, \arcf{a}{+}{1}{a},
      \arcf{b}{+}{1}{b}, \arcf{c}{+}{1}{c}\} \\
    E_- &= \{\arcf{a}{-}{2}{b}\} \qquad\qquad\qquad\qquad\qquad
    E_\uns = \emptyset
  \end{align*}
  Ce graphe des interactions est proche de celui qui avait été proposé
  \vfigref{thomas}(gauche) bien qu'il ne soit pas équivalent,
  car chaque composant comporte une auto-action positive.
  Les auto-actions sur $b$ et $c$ sont la conséquence d'une stabilité globale
  sur plusieurs sous-états : en effet, $c$ n'évolue jamais,
  et $b$ n'évolue pas non plus lorsque $a_2$ n'est pas actif.
  L'auto-action sur $a$ est principalement causée par sa nature multi-valuée.
  
  \begin{figure}[ht]
  \centering
  \scalebox{1.2}{
  \begin{tikzpicture}[grn]
    \path[use as bounding box] (-1.3,-0.75) rectangle (3.5,1.5);
    \node[inner sep=0] (a) at (2,0) {a};
    \node[inner sep=0] (b) at (0,0) {b};
    \node[inner sep=0] (c) at (2,1.2) {c};
    \path[->]
      (b) edge[bend right] node[elabel, below=-2pt] {$+1$} (a)
      (c) edge node[elabel, right=-2pt] {$+1$} (a)
      (a) edge[bend right] node[elabel, above=-5pt] {$-2$} (b)
      (b) edge[in=-15+180, out=15+180, loop] node[elabel, left=-2pt] {+1} (b)
      (c) edge[in=15, out=-15, loop] node[elabel, right=-2pt] {+1} (c)
      (a) edge[in=15, out=-15, loop] node[elabel, right=-2pt] {+1} (a);
  \end{tikzpicture}
  }
  \caption{\figlabel{infer-ig}%
    Graphe des interactions inféré depuis les Frappes de Processus de la \figref{infer-ex-ph}.
  }
  \end{figure}
\end{example}

\begin{example}
  Si on remplace la sorte coopérative $bc$ des Frappes de Processus de la \figref{infer-ex-ph}
  par quatre actions
  \[ \PHfrappe{b_0}{a_1}{a_0} ; \PHfrappe{b_1}{a_1}{a_2} ;
     \PHfrappe{c_0}{a_1}{a_0} ; \PHfrappe{c_1}{a_1}{a_2} \]
  on obtient à nouveau le graphe des interactions donné à la \figref{infer-ig}.
\end{example}

\begin{example}
  L'ajout d'une action $\PHfrappe{a_2}{b_0}{b_1}$ aux Frappes de Processus canoniques de la
  \figref{infer-ex-ph} modifie le résultat de l'inférence.
  En effet, dans ce cas deux arcs non-signés vers $b$ sont inférés en lieu et place
  des arcs signés précédents :
  \begin{align*}
    E_+ &= \{\arcf{b}{+}{1}{a}, \arcf{c}{+}{1}{a}, \arcf{a}{+}{1}{a}, \arcf{c}{+}{1}{c}\}\\
    E_- &= \emptyset \qquad\qquad\qquad\qquad
    E_\uns = \{\arcf{a}{\uns}{2}{b}, \arcf{b}{\uns}{1}{b}\}
  \end{align*}
  Cela est dû au fait que les actions $\PHfrappe{a_2}{b_1}{b_0}$ et $\PHfrappe{a_2}{b_0}{b_1}$
  introduisent des oscillations causées uniquement par le processus $a_2$,
  ce qui implique une influence locale à la fois positive et négative,
  et est impossible à représenter au sein d'un modèle de Thomas.
\end{example}








\subsection{Inférence de la paramétrisation}
\seclabel{infer-param}

Une fois obtenu le graphe des interactions inféré selon la méthode proposée à la section
précédente, il est ensuite possible d'inférer une partie des paramètres discrets
propres à un modèle de Thomas, en fonction de la dynamique des Frappes de Processus canoniques
d'origine.
Cette inférence repose à nouveau sur une exploration exhaustive des comportements possibles
du modèle en fonction de l'état des prédécesseurs de chaque composant.
Cependant, cette inférence peut être partielle si le comportement modélisé
ne peut pas être représenté à l'aide d'un modèle de Thomas.
Dans ce cas, il est possible d'inférer une partie seulement des paramètres,
puis d'énumérer toutes les modèles compatibles avec cette paramétrisation partielle,
la dynamique du modèle et certaines contraintes de modélisation sur les paramètres.

\subsection{Inférence des paramètres}
\seclabel{infer-params}

Cette sous-section présente l'inférence des paramètres discrets indépendants
à partir d'un modèle de Frappes de Processus donné.
Ces résultats sont équivalents à ceux présentés par \citeasnoun{PMR10-TCSB},
auxquels nous ajoutons la notion de \emph{Frappes de Processus bien formées
pour l'inférence des paramètres}, définie au \crref{infer-params-ok},
et qui stipule que pour toute régulation de $a$ par $b$,
tous les processus de $\levels{b}{a}$ (\resp $\ulevels{b}{a}$)
possèdent la même influence sur $a$.

\begin{critere}[Frappes de Processus bien formées pour l'inférence des paramètres]
\crlabel{infer-params-ok}
  Des Frappes de Processus canoniques sont \emph{bien formées pour l'inférence des paramètres}
  si et seulement si leur dynamique est unitaire (\crref{unitaire})
  et si le graphe des interactions $(\components; E)$ inféré par \propref{inference-gi}
  vérifie :
  \begin{align*}
    \begin{split}
      \forall a \in \components &, \forall b \in \RRBreg{a},
        \forall N \in \{ \levels{b}{a}, \ulevels{b}{a} \}, \forall i,j \in N, \\
        %\forall (i, j \in \levels{b}{a} \vee i, j \in \ulevels{b}{a}), \\
      & \forall c \in \PHs, ( (b \neq a \wedge c = a) \vee
        (\exists \myupsilon \in \pred(a), c \in \conn(\myupsilon) \wedge b \in \compin(c)), \\
%        (c \in \PHpredec{a} \setminus \components \wedge b \in \PHdirectpredec{c})), \\
      & \qquad \PHfrappe{b_i}{c_k}{c_l}\in\PHa \Leftrightarrow \PHfrappe{b_j}{c_k}{c_l}\in\PHa
    \end{split}
  \end{align*}
\end{critere}

On souhaite dans la suite inférer le paramètre discret $K_{a,\omega}$,
pour un composant $a \in \components$ et un ensemble $\omega \subset \RRBreg{a}$
de ressources donnés.
Cette inférence se base, à l'instar de l'inférence du graphe des interactions, sur une analyse
exhaustive des sous-états des régulateurs de $a$.
Pour chaque sorte $b \in \RRBreg{a}$, on définit un \todo{\emph{contexte}} $C^b_{a,\omega}$
(\eqref{param-context}) qui recense tous les processus qui interagissent avec $a$
dans tous les sous-états représentés par l'ensemble de ressources $\omega$.
Le contexte d'une sorte coopérative $\myupsilon$ régulant $a$ est l'ensemble des processus
focaux correspondant à ces sous-états (\eqref{param-context-coop}).
Enfin, $C_{a,\omega}$ fait référence à l'union de tous ces contextes
(\eqref{param-context-total}).

\begin{align}
\eqlabel{param-context}
  \forall b \in \components,~
  C_{a,\omega}^b & \DEF \begin{cases}
    \levels{b}{a}  & \text{si $b \in \omega$,}\\
    \ulevels{b}{a} & \text{si $b \notin \omega$,}\\
    L_b            & \text{sinon}\\
  \end{cases}
  \\
\eqlabel{param-context-coop}
  \forall \myupsilon \in \pred(a) \cap \cs,
    C_{a,\omega}^\myupsilon & \DEF \{ \pfp_{\sigma}(\myupsilon) \mid
    \sigma \in \bigtimes{b \in \compin(\myupsilon)} C_{a,\omega}^b \} \\
\eqlabel{param-context-total}
  C_{a,\omega} & \DEF \bigcup_{b \in \pred(a)} C^b_{a,\omega}
\end{align}

Pour inférer le paramètre recherché, nous calculons les \emph{processus focaux} de $a$,
qui sont les processus vers lesquels tend le niveau d'expression de $a$ en présence
de certains autres processus (\defref{focals}).
Ainsi, $\focals(a, S, T)$ donne l'ensemble des processus de $a$ accessibles
en partant de n'importe quel processus dans $S$, et à condition de ne jouer que des
actions dont le frappeur est dans $T$.
Cette notion se base sur un graphe recensant tous les bonds que peuvent faire les processus
de $a$ ;
si ce graphe est acyclique, alors l'ensemble des processus focaux est l'ensemble des processus
de $a$ qui ne sont pas frappés ---~et vers lesquels $a$ va avoir tendance à évoluer.

\begin{definition}[$\focals(a,S,T)$]
\deflabel{focals}
  L'ensemble des \emph{processus focaux} de $a \in \components$ depuis $S \subset \PHl_a$
  pour le sous-état $T \subset \Proc$ est donné par :
  \[
    \focals(a, S, T) \DEF
    \begin{cases}
      \{ a_i \in V \mid \nexists (a_i,a_j)\in E\} & \text{si $(V,E)$ est acyclique},\\
      \emptyset & \text{sinon}\\
    \end{cases}
  \]
  où $(V,E)$ est le graphe orienté suivant :
  \begin{align*}
    E & \DEF \{ (a_j; a_k) \in (\PHl_a \times \PHl_a) \mid
      \exists \PHfrappe{b_i}{a_j}{a_k} \in \PHh^{(2)}, b_i \in T \wedge a_j \in S \} \\
    V & \DEF S \cup \{ a_k \in \PHl_a \mid \exists (a_j; a_k) \in E \}
    %\eqlabel{bounce-graph}
  \end{align*}
\end{definition}

Le paramètre $K_{a,\omega}$ détermine les niveaux d'expression vers lesquels tend $a$
en présence du contexte $C_{a,\omega}$.
Cette valeur peut être calculée à l'aide de $\focals$ qui permet justement
de retrouver les processus focaux en présence de certaines ressources.
Ainsi, on peut en conclure que $K_{a,\omega} = \focals(a,C^a_{a,\omega},C_{a,\omega})$
dans tous les cas où cette valeur est un intervalle non vide (\propref{inference-param}).

\begin{proposition}[Inférence des paramètres]
\proplabel{inference-param}
  Soient $\PH = (\PHs, \PHl, \PHh)$ des Frappes de Processus bien formées pour l'inférence des
  paramètres, $\GI = (\components, E)$ le graphe des interactions inféré pour $\PH$
  et $\omega \subset \RRBres{a}$ un ensemble de ressources de $a$.
  Si $\focals(a,C^a_{a,\omega},C_{a,\omega})$ est un intervalle non vide,
  avec $\focals(a,C^a_{a,\omega},C_{a,\omega}) = \segm{a_i}{a_j}$,
  alors $K_{a,\omega} = \segm{i}{j}$.
\end{proposition}



\begin{example}
%\label{ex:infer-param-runningPH-1}
  Si on l'applique aux Frappes de Processus de la \figref{infer-ex-ph},
  la méthode d'inférence des paramètres donnée dans cette section est conclusive sur tous
  les paramètres et donne :
  \begin{align*}
    K_{a, \emptyset} &= \segm{0}{0} &
    K_{b, \emptyset} &= \segm{0}{0} \\
    K_{a, \{a\}} &= \segm{0}{0} &
    K_{b, \{a\}} &= \segm{0}{0} \\
    K_{a, \{c\}} &= \segm{1}{1} &
    K_{b, \{b\}} &= \segm{1}{1} \\
    K_{a, \{b\}} &= \segm{1}{1} &
    K_{b, \{a,b\}} &= \segm{0}{0} \\
    K_{a, \{b,c\}} &= \segm{1}{1} &
    K_{c, \emptyset} &= \segm{0}{0} \\
    K_{a, \{a,b,c\}} &= \segm{2}{2} &
    K_{c, \{c\}} &= \segm{1}{1} \\
    K_{a,\{a,b\}} &= \segm{1}{1} &
    K_{a,\{a,c\}} &= \segm{1}{1}
  \end{align*}
\end{example}

\todo{Exemple partiellement conclusif}

En observant la \propref{inference-param}, on constate que l'inférence
de certains paramètres peut ne pas être possible.
Cela peut être notamment dû à des coopérations mal définies entre les régulateurs d'un composant :
lorsque deux régulateurs frappent un même composant de façon indépendante, leurs actions peuvent
avoir des effets opposés, créant des oscillations dans la dynamique.
Un tel indéterminisme ne peut pas être représenté à l'aide d'un modèle de Thomas étant donné
que dans une configuration de ressources données, un composant possède un unique attracteur,
représenté par le paramètre discret correspondant,
et ne peut donc évoluer que dans une seule direction.
Il est possible de résoudre ces cas non conclusifs
(autrement dit, de supprimer ces comportements oscillants)
en raffinant le modèle à l'aide de suppressions d'actions ou
en s'assurant que les coopérations sont correctement définies à l'aide de sortes coopératives
afin d'éviter des influences opposées depuis des régulateurs concurrents.



\subsection{Énumération des paramétrisations admissibles}
\seclabel{enum-param}

Lors de la construction d'un modèle de Thomas, trouver la paramétrisation compatible avec
le comportement désiré est nécessaire pour obtenir un modèle complet.
Cependant, cette étape possède une complexité inhérente à ce type de formalisme,
car le nombre de paramètres que contient le modèle croît exponentiellement dans
la taille du graphe des interactions
(plus précisément, dans le nombre de régulations vers chaque composant).
La méthode d'inférence des paramètres présentée précédemment permet cependant d'obtenir
certaines informations sur ces paramètres en fonction de la dynamique des Frappes de Processus
canoniques étudiées.
Ces informations permettent donc de restreindre l'espace des paramétrisations possibles,
et donc d'obtenir plus facilement le modèle recherché.

En d'autres termes,
lors de l'inférence d'un modèle de Thomas selon la méthode décrite précédemment,
il arrive que certains paramètres ne puissent pas être inférés.
Le modèle obtenu est alors partiel, et correspond à un ensemble
plus ou moins large de modèles complets.
En énumérant les valeurs possibles de chaque paramètre, il est envisageable de retrouver
l'ensemble des modèles \emph{compatibles} avec ces valeurs.

Nous délimitons tout d'abord la validité d'un paramètre (\crref{params-valide}) afin d'assurer
que toutes les transitions dans le modèle de Thomas résultant
sont permises par la dynamique des Frappes de Processus canoniques étudiées.
Cette propriété est vérifiée en s'assurant,
pour chaque configurations de ressources possibles,
de l'existence d'une frappe faisant bondir
le processus d'une sorte vers le paramètres correspondant.
Ainsi, conjointement avec le fait que les Frappes de Processus étudiées
sont bien formées pour l'inférence des paramètres,
nous assurons que pour toute transition dans le modèle de Thomas inféré,
il existe une transition équivalente dans les Frappes de Processus d'origine.
Nous remarquons par ailleurs que les paramètres inférés à l'aide de la \vpropref{inference-param}
vérifient déjà cette propriété.



\begin{critere}[Validité d'un paramètre]
\crlabel{params-valide}
  Un paramètre $K_{a,\omega}$ est \emph{valide} pour les Frappes de Processus $\PH$
  si et seulement si :
  \begin{align*}
    \forall a_i\in C^a_{a,\omega}, a_i \notin K_{a,\omega} \Longrightarrow
      (& \exists \PHfrappe{c_k}{a_i}{a_j}\in\PHa, c_k \in C^c_{a,\omega} \\
      & \wedge a_i < K_{a,\omega} \Rightarrow j > i \wedge  a_i > K_{a,\omega} \Rightarrow j <i )
  \end{align*}
\end{critere}

Nous utilisons de plus plusieurs contraintes de modélisation \citeaffixed{BernotSemBRN}{tirées de}
afin d'assurer une cohérence des paramètres avec les signes des régulations du graphe
des interactions préalablement inféré.
L'\emph{hypothèse des valeurs extrêmes} (\crref{param-enum-extreme})
stipule que les niveaux extrêmes d'un composant $a$ (c'est-à-dire $0$ et $l_a$)
doivent chacun apparaître dans au moins un paramètre.
L'\emph{hypothèse d'activité} (\crref{param-enum-activity})
stipule en outre que toutes les régulations doivent être fonctionnelles,
c'est-à-dire que pour chaque régulateur d'un composant,
il existe au moins une configuration dans laquelle la présence ou l'absence de ce régulateur
modifie le paramètre considéré.
Enfin, l'\emph{hypothèse de monotonicité} (\crref{param-enum-monotonicity})
stipule qu'ajouter un activateur (\resp inhibiteur) aux ressources d'un composant
ne peut qu'augmenter (\resp diminuer) la valeur du paramètre considéré.
La relation d'ordre $\leqsegm$ entre deux paramètres discrets s'applique à des segments
et est définie à la \vsecref{notations}.

\begin{critere}[Hypothèse des valeurs extrêmes]
\crlabel{param-enum-extreme}
  Soit $\GI = (\components, E)$ un graphe des interactions.
  Une paramétrisation $K$ sur $\GI$ satisfait l'\emph{hypothèse des valeurs extrêmes}
  si et seulement si :
  \begin{align*}
    \forall b \in \components, \RRBreg{b} \neq \emptyset \Longrightarrow
    \exists \omega \subset \RRBreg{b}, 0 \in K_{b,\omega} \wedge
    \exists \omega' \subset \RRBreg{b}, l_b \in K_{b,\omega'}
  \end{align*}
\end{critere}

\begin{critere}[Hypothèse d'activité]
\crlabel{param-enum-activity}
  Soit $\GI = (\components, E)$ un graphe des interactions.
  Une paramétrisation $K$ sur $\GI$ satisfait l'\emph{hypothèse d'activité}
  \begin{align*}
    \forall b \in \components, \forall a \in \RRBreg{b}, \exists \omega \subset \RRBreg{b},
    K_{b,\omega} \neq K_{b,\omega \cup \{ a \}}
  \end{align*}
\end{critere}

\begin{critere}[Hypothèse de monotonicité]
\crlabel{param-enum-monotonicity}
  Soit $\GI = (\components, E)$ un graphe des interactions.
  Une paramétrisation $K$ sur $\GI$ satisfait l'\emph{hypothèse de monotonicité}
  si et seulement si :
  \begin{align*}
    \forall b \in \components,
    \forall A^+ \subset \{ a \in \components \mid \arc{a}{+}{t}{b} \in E_+ \}&,
    \forall A^- \subset \{ a \in \components \mid \arc{a}{-}{t}{b} \in E_- \},\\
    K_{b,\omega \cup A^-} & \leqsegm K_{b,\omega \cup A^+}
  \end{align*}
\end{critere}

\todo{Ajouter exemple avec cas non conclusifs.}

% 
% \begin{example}\label{ex:enum-param-runningPH-1}
% The parametrization inferred in \pref{ex:infer-param-runningPH-1} was partial because $K_{a,\{a,b\}}$ and $K_{a,\{a,c\}}$ could not be inferred.
% It is however possible to enumerate all complete and admissible parametrizations
% compatible with both the inferred parameters, and the properties of this subsection.
% This enumeration gives 9 different parametrizations which correspond to the 3 possible values
% for each of the two parameters that could not be inferred:
% \begin{align*}
%   K_{a,\{a,b\}} &\in \{ \segm{1}{1}, \segm{1}{2}, \segm{2}{2} \} \\
%   K_{a,\{a,c\}} &\in \{ \segm{1}{1}, \segm{1}{2}, \segm{2}{2} \}
% \end{align*}
% We note that for all solutions, $0 \notin K_{a,\{a,b\}} \wedge 0 \notin K_{a,\{a,c\}}$.
% This is due to the monotonicity assumption (\pref{pro:param-enum-monotonicity}) which especially states that:
% \begin{align*}
%   K_{a,\{b\}} \leqsegm K_{a,\{a,b\}} \wedge
%   K_{a,\{c\}} \leqsegm K_{a,\{a,c\}}
% \end{align*}
% 
% Finally, we note that $\segm{1}{1}$ belongs to the possible values for both parameters.
% Therefore this enumeration allows, from the model in \pref{fig:runningPH-1},
% to find the behavior of the model refined with a cooperative sort described in \pref{fig:runningPH-2}.
% \end{example}

% Équivalence avec les réseaux d'automates synchronisés
% Équivalence avec les réseaux d'automates synchronisés

\section{Équivalence avec les réseaux d'automates synchronisés}
\seclabel{phm2an}

Nous nous intéressons ici au lien entre les Frappes de Processus avec actions plurielles
et les réseaux d'automates synchronisés.
Nous montrons notamment que ces deux formalismes sont équivalents
et nous exhibons pour cela deux traductions d'un formalisme vers l'autre.
%(\defref{phm2an,an2phm} \vdefpageref{phm2an}).
%(\defref[s]{phm2an} et \defref*[vref]{an2phm}).
Cette équivalence est intéressante car elle montre clairement le lien entre ce formalisme
de Frappes de Processus et celui plus répandu des réseaux d'automates synchronisés.
En effet, chaque action plurielle $\PHfrappemult{A}{B}$
possède la même dynamique qu'un ensemble de transitions synchronisées
partant chacune d'un processus de l'ensemble $A$ et
arrivant dans le processus de la même sorte de l'ensemble $A \recouvre B$%
\footnote{La notation $A \recouvre B$, formalisée à la \vdefref{recouvrementps}, représente
l'ensemble où chaque processus de $A$ a été remplacé par
le processus de $B$ de la même sorte, s'il existe.}.

\myskip

Nous rappelons tout d'abord la définition d'un réseau d'automates synchronisés (\defref{an})
ainsi que la relation de transition entre deux états d'un tel modèle (\defref{an-sem})
ce qui permet d'en définir la dynamique.

\begin{definition}[Réseau d'automates synchronisés]
\deflabel{an}
  Un \emph{réseau d'automates synchronisés} est un quadruplet $\AN = (\ANs; \ANl; \ANi; \ANt)$
  où :
  \begin{itemize}
    \item $\ANs \DEF \{a, b, \dots\}$ est l'ensemble fini et dénombrable des \emph{automates} ;
    \item $\ANl \DEF \bigtimes{a \in \ANs} \ANl_a$ est l'ensemble fini des \emph{états},
      où $\ANl_a = \{a_0, \ldots, a_{l_a}\}$ est l'ensemble fini et dénombrable
      des \emph{états locaux} de l'automate $a \in \ANs$ et $l_a \in \sN^*$,
      chaque état local appartenant à un unique automate :
      $\forall (a_i; b_j) \in \ANl_a \times \ANl_b, a \neq b \Rightarrow a_i \neq b_j$ ;
    \item $\ANi \DEF \{\ell_1, \dots, \ell_m\}$ est l'ensemble fini des
      \emph{libellés} de transitions ;
    \item $\ANt \DEF \{ \ANaction{a_i}{\ell}{a_j} \mid a \in \ANs \wedge a_i \in \ANl_a \wedge
      \ell \in \ANi \}$ est l'ensemble fini des \emph{transitions} entre états locaux.
  \end{itemize}
  Pour tout libellé $\ell \in \ANi$, on note
  $\precond{\ell} \DEF \{ a_i \mid \ANaction{a_i}{\ell}{a_j} \in \ANt \}$
  et $\postcond{\ell} \DEF \{ a_j \mid \ANaction{a_i}{\ell}{a_j} \in \ANt \}$.
%   et $\invcond{\ell} \DEF \{ a_i \mid \ANaction{a_i}{\ell}{a_i} \in \ANt \}$.
  L'ensemble des états locaux des automates est dénoté par
  $\ANProc \DEF \bigcup_{a \in \ANs} \ANl_a$.
  Enfin, étant donné un état global $s \in \ANl$, $s(a) = a_i \in \ANl_a$
  fait référence à l'état local de l'automate $a \in \ANs$.
\end{definition}

\begin{definition}[Sémantique des réseaux d'automates ($\ANtrans$)]
\deflabel{an-sem}
  Étant donné un réseau d'automates synchronisés $\AN = (\ANs; \ANl; \ANi; \ANt)$,
  un libellé $\ell$ est dit \emph{jouable} dans un état $s \in \ANl$ si et seulement si :
  $\forall a_i \in \precond{\ell}, s(a) = a_i$.
  Dans ce cas, on note $(s \play \ell)$ l'état résultant du jeu de toutes les transitions
  libellées par $\ell$, défini par :
  $s \play \ell = s \recouvre \postcond{\ell}$.
%   $\forall a_j \in \postcond{\ell}, (s \play \ell)(a) = a_j \wedge
%     \forall b \in \ANs, \ANl_b \cap \precond{\ell} = \emptyset \Rightarrow
%     (s \play \ell)(b) = s(b)$.
  De plus, on note alors : $s \ANtrans (s \play \ell)$.
%   Étant donné un réseau d'automates synchronisés $\AN = (\ANs; \ANl; \ANi; \ANt)$,
%   la relation de transition globale entre deux états du réseau
%   $\ANtrans \subset \ANl \times \ANl$ est définie par :
%   \begin{align*}
%     s \ANtrans s' \EQDEF \exists \ell \in \ANi, &\forall a_i \in \precond{\ell}, s(a) = a_i
%       \wedge \forall a_j \in \postcond{\ell}, s'(a) = a_j \\
%     \wedge &\forall b \in \ANs, \ANl_b \cap \precond{\ell} = \emptyset \Rightarrow s(b) = s'(b)
%   \end{align*}
\end{definition}

\begin{remark}
  Nous notons que les réseaux d'automates synchronisés ainsi définis sont non-déterministes,
  tant au niveau global du modèle qu'au niveau local des automates.
  Cette vision s'oppose à d'autres sémantiques des réseaux d'automates
  comme celles de \citeasnoun{Richard10} ou de \citeasnoun{RRT08},
  qui définissent la dynamique de leurs modèles à l'aide de fonctions de transition locales,
  qui sont par définition déterministes.
  Ces fonctions ont en effet la forme : $f_a : \ANl \rightarrow \ANl_a$
  et associent donc à chaque état global du modèle un état local (unique) pour chaque automate.
  La définition des réseaux d'automates synchronisés que nous proposons ici (\defref{an})
  n'empêche en revanche pas l'existence de deux libellés $\ell_1, \ell_2 \in \ANi$
  tels que $\precond{\ell_1} = \precond{\ell_2}$ mais $\postcond{\ell_1} \neq \postcond{\ell_2}$.
  Cela implique notamment l'existence de deux transitions entre état locaux
  $\ANaction{a_i}{\ell_1}{a_j}$ et $\ANaction{a_i}{\ell_2}{a_k}$
  avec $a_j \neq a_k$, d'où un non-déterminisme au niveau des automates.
\end{remark}

Pour tout modèle de Frappes de Processus avec actions plurielles $\PH$,
la \defref{phm2an} propose une traduction de $\PH$
en un réseau d'automates synchronisés $\phmtoan[\PH]$ équivalent,
et le \thmref{bisimulationphm2an} établit la bisimilarité entre les deux modèles.
La notation $\recouvre$ qui est utilisée dans la définition
qualifie le recouvrement d'un ensemble de processus de sortes distinctes
par un autre comprenant uniquement des processus issus des mêmes sortes
(\defref{recouvrementps}).
Cette notion de recouvrement est une extension
du recouvrement d'un état par un ensemble de processus
tel que précédemment formalisé à la \vdefref{recouvrement}.

\begin{definition}[Recouvrement ($\recouvre : \PHsublset \times \PHsublset \rightarrow \PHsublset$)]
\deflabel{recouvrementps}
  Étant donné un sous-état désordonné $ps \in \PHsublset$ et un processus $a_i \in \Proc$,
  tel que $a \in \sortes{ps}$, on définit :
  $(ps \recouvre a_i) = (ps \setminus \PHl_a) \cup \{ a_i \}$.
  On étend de plus cette définition
  au recouvrement par un ensemble de processus de sortes distinctes
  $ps' \in \PHsublset$ tel que $\sortes{ps'} \subset \sortes{ps}$
  comme étant le recouvrement de $ps$ par chaque processus de $ps'$ :
  $ps \recouvre ps' = ps \underset{a_i \in ps'}{\recouvre} a_i$.
\end{definition}

\begin{definition}[Réseau d'automates équivalent ($\phmtoansymbol$)]
\deflabel{phm2an}
  Le réseau d'automates synchronisés équivalent aux Frappes de Processus
  avec actions plurielles $\PH = (\PHs; \PHl; \PHh)$
  est défini par : $\phmtoan = (\PHs; \PHl; \ANi; \ANt)$, où :
  \begin{itemize}
    \item $\ANi = \{ \ell_h \mid h \in \PHh \}$ ; % est l'ensemble des libellés de transitions ;
    \item $\ANt = \{ \ANaction{a_i}{\ell_h}{a_j} \mid
      h \in \PHh \wedge h = \PHfrappemult{A}{B} \wedge a_i \in A \wedge a_j \in A \recouvre B \}$.
      % est l'ensemble des transitions locales.
  \end{itemize}
\end{definition}

\begin{theorem}[$\PH \approx \phmtoan$]
\thmlabel{bisimulationphm2an}
  Soient $\PH = (\PHs; \PHl; \PHh)$ des Frappes de Processus avec actions plurielles.
  On a :
  \[\forall s, s' \in \PHl, s \PHtrans s' \Longleftrightarrow s \trans{\phmtoan} s' \enspace.\]
\end{theorem}

\begin{proof}
  Soient $s, s' \in \PHl$.
  On pose : $\phmtoan = (\ANs; \ANl; \ANi; \ANt)$.
  
  ($\Rightarrow$) Supposons que $s \PHtrans s'$, c'est-à-dire qu'il existe une action $h \in \PHh$
    telle que $s' = s \play h$.
    Posons : $h = \PHfrappemult{A}{B}$.
    D'après la \defref{phm2an},
    l'existence de cette action dans $\PH$ implique celle d'un libellé $\ell_h$ dans $\phmtoan$
    ainsi que de l'ensemble de transitions
    $\ANt_h = \{ a_i \xrightarrow{\ell_h} a_j \mid a_i \in A \wedge a_j \in A \recouvre B \}$.
    Autrement dit, $\precond{\ell_h} = A$, donc $\ell_h$ est jouable dans $s$
    si et seulement si $A \subseteq s$.
    De plus, $\postcond{\ell_h} = \invariant{h} \cup B$, donc
    $(s \play \ell_h) = s \recouvre (\invariant{h} \cup B) = s \recouvre B = s'$
    car $\invariant{h} \subseteq A \subseteq s$.
  
  ($\Leftarrow$) Supposons que $s \trans{\phmtoan} s'$,
    c'est-à-dire qu'il existe un libellé $\ell \in \ANi$ et un ensemble de transitions
    ayant ce libellé : $\ANt_\ell = \{ a_i \xrightarrow{\ell} a_j \in \ANt \}$,
    tels que $s' = s \play \ell$.
    D'après la \defref{phm2an}, cela signifie notamment qu'il existe une action
    $h = \PHfrappemult{A}{B} \in \PHh$ telle que $\ell = \ell_h$, et que :
    $\ANt_\ell = \{ a_i \xrightarrow{\ell} a_j \mid a_i \in A \wedge a_j \in A \recouvre B \}$.
    Étant donné que $\invariant{h}$ et $\cible{h}$ forment une partition de $A$,
    $\ANt_\ell$ peut être découpé en deux ensembles, selon les invariants et les cibles de $h$ :
    $\ANt_\ell = \{ a_i \xrightarrow{\ell} a_i \mid a_i \in \invariant{h} \} \cup
      \{ a_i \xrightarrow{\ell} a_j \mid a_i \in \cible{h} \wedge a_j \in B \}$.
    Ainsi, $s' = s \recouvre (\invariant{h} \cup B) = s \recouvre B = s \play h$.
\end{proof}

Pour finir, nous proposons à la \defref{an2phm} la traduction inverse
d'un réseau d'automates synchronisés $\AN$
en des Frappes de Processus avec actions plurielles équivalentes $\antophm$.
Le \thmref{bisimulationan2phm} stipule que le modèle obtenu est bien bisimilaire
au modèle d'origine.
Enfin, le \thmref{equivphman} résume les résultats de cette section
en statuant l'équivalence d'expressivité entre les Frappes de Processus avec
actions plurielles et les réseaux d'automates synchronisés.

\begin{definition}[Frappes de Processus équivalentes ($\antophmsymbol$)]
\deflabel{an2phm}
  Les Frappes de Processus avec actions plurielles
  équivalentes au réseau d'automates synchronisés $\AN = (\PHs, \PHl, \ANi, \ANt)$
  sont définies par $\antophm = (\ANs, \ANl, \PHh)$, où :
%   $\PHh = \{ \PHfrappemult{\precond{\ell}}{(\postcond{\ell} \setminus \invcond{\ell})}
%     \mid \ell \in \ANi \}$.
  \[\PHh = \{ \PHfrappemult{\precond{\ell}}{B} \mid \ell \in \ANi \wedge
    B = \postcond{\ell} \setminus \{ a_i \in \ANProc \mid \ANaction{a_i}{\ell}{a_i} \in \ANt \}
    \}\]
\end{definition}

\begin{theorem}[$\AN \approx \antophm$]
\thmlabel{bisimulationan2phm}
  Soit $\AN = (\ANs; \ANl; \ANi; \ANt)$ un réseau d'automates synchronisés.
  On a :
  \[\forall s, s' \in \ANl, s \ANtrans s' \Longleftrightarrow s \trans{\antophm} s' \enspace.\]
\end{theorem}

\begin{proof}
  Soient $s, s' \in \PHl$.
  On pose : $\antophm = (\ANs; \ANl; \PHh)$.
  
  ($\Rightarrow$) Supposons que $s \ANtrans s'$,
    c'est-à-dire qu'il existe un libellé $\ell \in \ANi$ et un ensemble de transitions
    ayant ce libellé : $\ANt_\ell = \{ a_i \xrightarrow{\ell} a_j \in \ANt \}$,
    tels que $s' = s \play \ell$.
    D'après la traduction donnée à la \defref{an2phm}, il existe donc une action
    $h = \PHfrappemult{A}{B} \in \PHh$ telle que $A = \precond{\ell}$ et
    $B = \postcond{\ell} \setminus \{ a_i \in \ANProc \mid \ANaction{a_i}{\ell}{a_i} \in \ANt \}$.
    Or $s' = s \recouvre \postcond{\ell}
      = s \recouvre (B \cup \{ a_i \in \ANProc \mid \ANaction{a_i}{\ell}{a_i} \in \ANt \})
      = s \recouvre B$
    car $\{ a_i \in \ANProc \mid \ANaction{a_i}{\ell}{a_i} \in \ANt \} \subseteq s$.
    Ainsi, $h$ est jouable dans $s$ et $s' = s \play h$.
  
  ($\Leftarrow$) Supposons que $s \trans{\antophm} s'$,
    c'est-à-dire qu'il existe une action $h = \PHfrappemult{A}{B} \in \PHh$
    telle que $s' = s \play h$.
    D'après la traduction de la \defref{an2phm},
    cela signifie qu'il existe un libellé $\ell \in \ANi$ et un ensemble de transitions
    ayant ce libellé : $\ANt_\ell = \{ a_i \xrightarrow{\ell} a_j \in \ANt \}$,
    tels que : $A = \precond{\ell}$ et
    $B = \postcond{\ell} \setminus \{ a_i \in \ANProc \mid \ANaction{a_i}{\ell}{a_i} \in \ANt \}$.
    Comme $h$ est jouable dans $s$, alors $A \subseteq s$, donc $\ell$ est aussi jouable dans $s$.
    De plus, $s' = s \play h = s \recouvre B = s \recouvre (B \cup \invariant{h})
      = s \recouvre \postcond{\ell}$.
\end{proof}

\begin{theorem}[Équivalence entre réseaux d'automates synchronisés
  et Frappes de Processus avec actions plurielles]
\thmlabel{equivphman}
  Les Frappes de Processus avec actions plurielles sont aussi expressives
  que les réseaux d'automates synchronisés.
\end{theorem}

\begin{proof}
  D'après les \defref{phm2an,an2phm} et les \thmref{bisimulationphm2an,bisimulationan2phm}
  associés, tout modèle de Frappes de Processus avec actions plurielles peut être représenté
  à l'aide d'un réseau d'automates synchronisés, et inversement.
\end{proof}

%\todo{Exemple}


% Traduction en réseaux de Petri
% Traduction en réseaux de Petri

% \section{Réseaux de Petri}
%   \subsection{Réseaux de Petri saufs (une place par processus)}
%   \subsection{Réseaux de Petri bornés (une place par sorte)}

\section{Traduction en réseaux de Petri}
\seclabel{trad-petri}

\todo{Glu partout}

Dépliage (avec arcs inhibiteurs) \cite{baldan00}

\todo{Pour cette traduction, les indices des processus ont un sens particulier
et doivent être choisis dans $\sN$.}

\begin{definition}[Réseau de Petri ($\PT$)]
\deflabel{pt}
  Un \emph{réseau de Petri} est un 6-uplet
  $\PT = (\PTp; \PTt; \PTPre; \PTPost; \PTLect; \PTInh)$ où :
  \begin{itemize}
    \item $\PTp$ est l'ensemble fini des \emph{places},
    \item $\PTt$ est l'ensemble fini des \emph{transitions},
    \item $\PTPre \subset \PTp \times \sN \times \PTt$
%    \item $\PTPre : \PTp \times \PTt \rightarrow \sN$
      est l'ensemble des arcs entrant dans une transition,
    \item $\PTPost \subset \PTt \times \sN \times \PTp$
      est l'ensemble des arcs sortant d'une transition,
    \item $\PTLect \subset \PTp \times \sN \times \PTt$ est l'ensemble des arcs de lecture,
    \item $\PTInh \subset \PTp \times \sN \times \PTt$ est l'ensemble des arcs inhibiteurs,
  \end{itemize}
  avec : $\PTp \cap \PTt = \emptyset$ et $\PTp \cup \PTt \neq \emptyset$.
  De plus, chaque arc est unique ; autrement dit, pour chacun des ensembles
  $X \in \{ \PTPre, \PTPost, \PTLect, \PTInh \}$, on a :
  \[\forall (a; i; b) \in X, \forall (a'; i'; b') \in X,
    (a = a' \wedge b = b') \Rightarrow i = i'\]
  On note par ailleurs, pour tout couple d'éléments $a, b \in \PTp \cup \PTt$ :
%  $\forall X \in \{ \PTPre, \PTPost, \PTLect, \PTInh \},
  \[X(p, t) = \begin{cases}
              i & \text{ si } (a; i; b) \in X \\
              0 & \text{ sinon}
            \end{cases}\]
%   
%   place $p \in \PTp$ et toute transition $t \in \PTt$,
%   $\PTPost(t, p) = $
%   $\forall X \in \{ \PTPre, \PTPost, \PTLect, \PTInh \},
%     \forall p \in \PTp, \forall t \in \PTt$,
%   $X(p, t) \in \{ \PTPre, \PTPost, \PTLect, \PTInh \},$
  
  Un \emph{marquage} d'un réseau de Petri est une fonction $M : \PTp \rightarrow \sN$.
  Un \emph{réseau de Petri avec marquage initial} est un couple $(\PT; M_0)$
  où $\PT$ est un réseau de Petri et $M_0$ est un marquage de $\PT$.
%   
%   Pour toute transition $t \in \PTt$, on note de plus :
%   $\PTPre(t) = $
\end{definition}

\begin{definition}[Sémantique des réseaux de Petri ($\PTtrans$)]
\deflabel{pt-sem}
  Étant donnés un réseau de Petri $\PT = (\PTp; \PTt; \PTPre; \PTPost; \PTLect; \PTInh)$
  et un marquage $M$,
  une transition $t \in \PTt$ est dite \emph{jouable} dans $M$ si et seulement si :
  \[\forall p \in \PTp, M(p) \geq \PTPre(p, t) \wedge M(p) \geq \PTLect(p, t)
    \wedge M(p) < \PTInh(p, t)\]
  Dans ce cas, on note $(M \play t)$ le marquage résultant du jeu de cette transition depuis $M$,
  défini par :
  \[\forall p \in \PTp, (M \play t)(p) = M(p) - \PTPre(p, t) + \PTPost(t, p)\]
  De plus, on note alors : $M \PTtrans (M \play t)$.
\end{definition}

\begin{definition}[Réseau de Petri équivalent ($\phmtoptsymbol$)]
\deflabel{phm2pt}
  Le réseau de Petri équivalent aux Frappes de Processus
  avec actions plurielles $\PH = (\PHs; \PHl; \PHh)$
  est défini par : $\phmtoan = (\PTp; \PTt; \PTPre; \PTPost; \PTLect; \PTInh)$, où :
  \begin{itemize}
    \item $\PTp = \PHs$,
    \item $\PTt = \PHh$,
    \item $\PTPre = \{ (b, j-k, h) \mid h \in \PHh \wedge b_j \in \cible{h} \wedge
      b_k \in \bond{h} \wedge k-j < 0 \}$
    \item $\PTPost = \{ (h, k-j, b) \mid h \in \PHh \wedge b_j \in \cible{h} \wedge
      b_k \in \bond{h} \wedge k-j > 0 \}$
    \item $\PTLect = \{ (a, i, h) \mid h \in \PHh \wedge a_i \in \frappeur{h} \wedge i > 0 \}$
    \item $\PTInh = \{ (a, i+1, h) \mid h \in \PHh \wedge a_i \in \frappeur{h} \wedge i < l_a \}$
  \end{itemize}
  De plus, pour tout état $s \in \PHl$, on note
  $M^s$ le marquage correspondant, défini par :
  $\forall a \in \PTp, M^s(a) = i \text{ tel que } \PHget{s}{a} = a_i$.
\end{definition}

\begin{theorem}[$\PH \approx \phmtopt$]
\thmlabel{bisimulationphm2pt}
  Soient $\PH = (\PHs; \PHl; \PHh)$ des Frappes de Processus avec actions plurielles.
  On a :
  \[\forall s, s' \in \PHl, s \PHtrans s' \Longleftrightarrow
    M^s \trans{\phmtopt} M^{s'} \enspace.\]
\end{theorem}

\begin{proof}
  Soient $s, s' \in \PHl$.
  On pose : $\phmtopt = (\PTp; \PTt; \PTPre; \PTPost; \PTLect; \PTInh)$.
  Pour toute action $h \in \PHh$, on notera dans la suite :
  \begin{itemize}
    \item $\PTPre_h = \{ (b, j-k, h) \mid b_j \in \cible{h} \wedge
      b_k \in \bond{h} \wedge k-j < 0 \}$,
    \item $\PTPost_h = \{ (h, k-j, b) \mid b_j \in \cible{h} \wedge
      b_k \in \bond{h} \wedge k-j > 0 \}$,
    \item $\PTLect_h = \{ (a, i, h) \mid a_i \in \frappeur{h} \wedge i > 0 \}$,
    \item $\PTInh_h = \{ (a, i+1, h) \mid a_i \in \frappeur{h} \wedge i < l_a \}$.
  \end{itemize}
  
  ($\Rightarrow$) Supposons que $s \PHtrans s'$, c'est-à-dire qu'il existe une action $h \in \PHh$
    telle que $s' = s \play h$.
    Posons : $h = \PHfrappemult{A}{B}$.
    D'après la \defref{phm2pt},
    l'existence de cette action dans $\PH$ implique celle d'une transition $h \in \PTt$
    dans $\phmtopt$, ainsi que des arcs suivants :
    $\PTPre_h \subset \PTPre$, $\PTPost_h \subset \PTPost$,
    $\PTLect_h \subset \PTLect$ et $\PTInh_h \subset \PTInh$.
    Or, étant donné que $h$ est jouable dans $s$, on a :
    $\forall a_i \in \frappeur{h}, a_i \in s$, d'où :
    $\forall a \in \sortes{\frappeur{h}}, M^s(a) = i$.
    De plus, comme $\cible{h} \subset \frappeur{h}$, on en déduit :
    $\forall b \in \sortes{\cible{h}}, M^s(b) = j > j-k$.
    Ainsi, $h$ est jouable dans $\PT$ d'après la \defref{pt-sem}.
    De plus, d'après cette même définition,
    $\forall b \in \sortes{\bond{h}}, (M^s \play h)(b) = j + (k - j) = k$
    et $\forall a \in \PTp \setminus \sortes{\bond{h}}, (M^s \play h)(a) = M^s(a)$.
    Ainsi, $(M^s \play h) = M^{s \play h} = M^{s'}$.
  
  ($\Leftarrow$) Supposons que $M^s \trans{\phmtopt} M^{s'}$,
    c'est-à-dire qu'il existe une transition $h \in \PTt$ telle que $M^{s'} = M^s \play h$.
    D'après la \defref{phm2pt}, il existe alors
    une action $h = \PHfrappemult{A}{B} \in \PHh$ dans $\PH$,
    ainsi que les ensembles d'arcs suivants dans $\phmtopt$ :
    $\PTPre_h \subset \PTPre$, $\PTPost_h \subset \PTPost$,
    $\PTLect_h \subset \PTLect$ et $\PTInh_h \subset \PTInh$.
    Comme $h$ est jouable dans $M$, cela signifie alors, d'après la \defref{pt-sem}, que :
%     \[\forall p \in \PTp, M^s(p) \geq \PTPre(p, t) \wedge M^s(p) \geq \PTLect(p, t)
%       \wedge M^s(p) < \PTInh(p, t)\]
    $\forall p \in \frappeur{h}, M^s(p) = \PHget{s}{p}$
    (ainsi que : $\forall p \in \cible{h}, M^s(p) = \PHget{s}{p} \geq \PTPre(p, t)$).
    Ainsi, $h$ est jouable dans $s$ car $a \subseteq s$.
    Par ailleurs, $s \play h = s \recouvre B$ ;
    or, $\forall b \in \sortes{\bond{h}}, (M^s \play h)(b) = j + (k - j) = k$
    et $\forall a \in \PTp \setminus \sortes{\bond{h}}, (M^s \play h)(a) = M^s(a)$.
    Ainsi, $s \play h = s'$.
\end{proof}

\todo{Exemple}


% Traduction de modèles booléens Biocham
% Traduction de modèles booléens Biocham

\section{Traduction depuis la sémantique booléenne Biocham}
\seclabel{trad-biocham}

Nous proposons dans cette section une traduction des modèles Biocham dans la sémantique booléenne
en Frappes de Processus avec actions plurielles.
Biocham \citeaffixed{fages2004modelling}{pour \textit{Biochemical Abstract Machine}, voir}
est un environnement logiciel pour la modélisation et l'analyse de systèmes biochimiques.
Il propose notamment une syntaxe basée sur des règles de réactions pour représenter
un système de réactions biochimiques,
et un simulateur booléen permettant d'exécuter le système.
Ce simulateur possède la particularité d'interpréter le modèle comme étant un modèle booléen,
où les composants peuvent être présents ou absents,
sans prendre en compte les coefficients stœchiométriques ou les paramètres cinétiques
éventuels contenus dans les équations.

Notre traduction s'appuie sur les ressemblances entre le formalisme booléen de Biocham
et les Frappes de Processus avec actions plurielles, comme discutées au début de la
\secref{phm}.
Pour chaque équation de Biocham, cette traduction fait correspondre
un ensemble d'actions qui reproduit toutes les dynamiques (non-déterministes) possibles.
Nous montrons enfin que cette traduction produit effectivement un modèle équivalent,
ce qui montre que les Frappes de Processus sont au moins aussi expressives
que la sémantique booléenne de Biocham.

\myskip

Nous définissons dans la suite la notion de système d'équations biochimiques (\defref{bc})
et sa dynamique dans la sémantique booléenne de Biocham (\defref{bc-sem}).
Une équation biochimique est un triplet de la forme $X \xrightarrow{Y} Z$
où $X$, $Y$ et $Z$ sont des ensembles de composants ayant respectivement le rôle
de réactifs, catalyseurs et produits.
La signification intuitive équation est la suivante :
si, dans un état donné du système, tous les réactifs et tous les catalyseurs sont présents,
alors il est possible de créer tous les produits et de consommer une partie des catalyseurs.
Il est à noter que nous cherchons à garder l'expressivité la plus large de cette sémantique,
qui autorise de jouer une équation même si certains produits (dans $Z$) sont déjà présents
dans l'état considéré.
Par ailleurs, seule une partie des réactifs (dans $X$) est consommée, et ce de façon
non-déterministe, afin de conserver l'ensemble des dynamiques possibles d'un modèle non booléen
(et comportant donc des coefficients stœchiométriques).

% Nous proposons ensuite une traduction vers les Frappes de Processus avec actions plurielles
% (\defref{bc2phm})
% et nous montrons qu'elle possède la même dynamique (\thmref{bisimulationbc2phm}).
% Cette traduction fit correspondre, à chaque équation du système,
% un ensemble d'actions qui reproduit toutes les dynamiques non-déterministes possibles.

\begin{definition}[Système d'équations biochimiques]
\deflabel{bc}
  Un \emph{système d'équations biochimiques} tel que décrit par Biocham
  est un ensemble de réactions :
  \[\BCe = \{ X \xrightarrow{Y} Z \mid X \cap Y = Y \cap Z = X \cap Z = \emptyset \}\]
%   est un couple $(\BCc; \BCe)$ où :
%   \begin{itemize}
%     \item $\BCc$ est l'ensemble de composants booléens,
%     \item $\BCe = \{ X \xrightarrow{Y} Z \mid X, Y, Z \subset \BCc \wedge
%       X \cap Y = Y \cap Z = \emptyset \}$ est l'ensemble des équations biochimiques.
%   \end{itemize}
  Par ailleurs, on note $\BCc[\BCe]$
  l'ensemble de tous les composants mentionnés dans $\BCe$ :
  \[\BCc = \bigcup_{X \xrightarrow{Y} Z \in \BCe} X \cup Y \cup Z\]
  
  Pour toute équation biochimique $X \xrightarrow{Y} Z \in \BCe$,
  les éléments de $X$ sont appelés les \emph{réactifs}, ceux de $Y$ sont les \emph{catalyseurs}
  et ceux de $Z$ sont les \emph{produits}.
  
  Un \emph{état} d'un système d'équations biochimiques est un ensemble $S \subset \BCc$.
\end{definition}

\begin{remark}
\label{bc-nondeterminisme}
  La syntaxe véritable de Biocham permet d'intégrer des catalyseurs implicites,
  en relâchant la contrainte $X \cap Z = \emptyset$ ;
  autrement dit, il est possible d'avoir $X \cap Z \neq \emptyset$.
  Dans ce cas, les éléments de $X \cap Z$ sont aussi des catalyseurs,
  car ils conditionnent le jeu de la réaction mais ne sont pas modifiés par celle-ci.
  On peut réécrire de telles équations de la forme suivante,
  tout en assurant la même dynamique :
  $(X \setminus Z) \xrightarrow{Y \cup (X \cap Z)} (Z \setminus X)$.
  Par ailleurs, il est naturellement possible de représenter une réaction d'équilibre
  $X \overset{Y}{\longleftrightarrow} Z$ par deux réactions biochimiques
  $X \xrightarrow{Y} Z$ et $Z \xrightarrow{Y} X$.
\end{remark}

\begin{definition}[Sémantique booléenne d'un système d'équations biochimiques]
\deflabel{bc-sem}
  Soit $\BCe$ un système d'équations biochimiques
  et $S \subset \BCc$ un état de ce système.
  On note : $S \BCtrans S'$ si et seulement si :
  \[S \neq S' \wedge
    \exists X \xrightarrow{Y} Z \in \BCe,
    X \cup Y \subset S \wedge
    \exists X' \subset X, S' = (S \setminus X') \cup Z\]
%   Soient $\BCe$ un système d'équations biochimiques
%   et $S \subset \BCc$ un état de ce système.
%   Une équation $e = X \xrightarrow{Y} Z \in \BCe$ est dite \emph{jouable} dans $S$
%   si et seulement si $X \cup Y \subset S$.
%   On note alors $S \play e$ l'état résultant du jeu de $e$ dans $S$, défini par :
%   $S \play e = (S \setminus X) \cup Z$.
%   On notre par ailleurs : $S \BCtrans (S \play e)$.
\end{definition}

\begin{remark}
  Comme expliqué plus haut,
  cette définition prend compte de la sémantique non-déterministe de Biocham pour ce qui est
  de la consommation des réactifs.
  Il est possible de rendre cette définition déterministe (du point de vue de chaque réaction)
  en imposant $X' = X$ pour toutes les transitions.
  De même, cette définition ne prend pas en compte la présence éventuelle de produits
  dans le milieu avant de jouer une équation ;
  autrement dit, certains des produits d'une équation biochimique
  peuvent se trouver dans le milieu au moment où elle est jouée.
\end{remark}

La \defref{bc2phm} propose une traduction des systèmes d'équations biochimiques
en Frappes de Processus avec actions plurielles.
Le modèle obtenu comporte autant de sortes booléennes que de composants mentionnés ;
autrement dit, pour chaque composant $a$ mentionné, il existe une sorte avec deux processus
$a_0$ et $a_1$ dans le modèle obtenu.
De plus, pour chaque équation biochimique $X \xrightarrow{Y} Z$,
un ensemble d'actions est créé, chaque action étant de la forme :
$\PHfrappemult{(X_1 \cup Y_1 \cup Z'_0 \cup (Z \setminus Z')_1)}{(X'_0 \cup Z'_1)}$,
où $X$ et $Y$ sont les réactifs et catalyseurs nécessaires à l'initiation de la réaction,
$Z'$ représente les produits qui seront effectivement créés
(et $Z \setminus Z'$ représente ceux qui sont déjà présents) et
$X'$ représente le sous-ensemble des réactifs qui seront consommés
(défini de façon totalement non-déterministe).
Enfin, le \thmref{bisimulationbc2phm} stipule que le modèle obtenu
est fortement bisimilaire ---~et possède donc strictement la même dynamique.

\begin{definition}[Frappes de Processus équivalentes ($\bctophmsymbol$)]
\deflabel{bc2phm}
  Soit $\BCe$ un système d'équations biochimiques.
  On note $\bctophm = (\PHs, \PHl; \PHh)$ les Frappes de Processus avec actions plurielles
  équivalentes, définies par :
  \begin{itemize}
    \item $\PHs = \BCc$,
    \item $\PHl = \bigtimes{a \in \PHs} \PHl_a$, où $\forall a \in \PHs, \PHl_a = \{ a_0, a_1 \}$,
%    \item $\PHh = \{ \PHfrappemult{(X_1 \cup Y_1 \cup Z_0)}{(X_0 \cup Z_1)} \mid
    \item $\PHh = \{
      \PHfrappemult{(X_1 \cup Y_1 \cup Z'_0 \cup (Z \setminus Z')_1)}{(X'_0 \cup Z'_1)} \mid
      X \xrightarrow{Y} Z \in \BCe \wedge X' \subset X \wedge Z' \subset Z \wedge
      (X' \cup Z') \neq \emptyset \}$,
%     
%     \item $\PHh = \{ \PHfrappemult{A}{B} \mid \exists X \xrightarrow{Y} Z \in \BCe,
%       A = \{ a_1 \mid a \in X \cup Y \} \cup \{ a_0 \mid a \in Z \} \wedge
%       B = \{ a_0 \mid a \in X \} \cup \{ a_1 \mid a \in Z \} \}$.
%     
%     \item $\PHh = \{ \PHfrappemults{a_1 \mid a \in X \cup Y}{a_0}
%       \mid A = \{  \} X \cup Y \wedge B =  \}$
%     \item $\PHh = \{ \PHfrappemult{A}{B} \mid A = \{  \} X \cup Y \wedge B =  \}$
  \end{itemize}
  où, pour tout $i \in \{ 0, 1 \}$ et tout $N \subset \BCc$, on note :
  $N_i = \{ a_i \mid a \in N \}$.
%   où, pour toute équation $X \xrightarrow{Y} Z \in \BCe$, on note
%   $N_i = \{ a_i \mid a \in N \}$
%   avec $i \in \{ 0, 1 \}$ et $N \in \{ X, Y, Z \}$.
  
  Enfin, pour tout état $S \subset \BCc$, on note
  $\tophm{S} = S_1 \cup (\BCc \setminus S)_0$
  %\etat{a_1 \mid a \in S} \cup \etat{a_0 \mid a \notin S}$
  l'état correspondant dans $\bctophm$.
\end{definition}

\begin{remark}
  Pour supprimer le non-déterminisme de Biocham,
  tel qu'expliqué à la remarque \vpageref{bc-nondeterminisme},
  il est possible de modifier la \defref{bc2phm}
  afin d'imposer pour chaque action : $X' = X$,
  ou de faire les simplifications correspondantes.
\end{remark}

\begin{theorem}[$\PH \approx \bctophm$]
\thmlabel{bisimulationbc2phm}
  Soit $\BCe$ un système d'équations biochimiques.
  On a :
  \[\forall S, S' \subset \BCc, S \BCtrans S' \Longleftrightarrow
    \tophm{S} \trans{\bctophm} \tophm{S'} \enspace.\]
\end{theorem}

\begin{proof}
  Posons : $\bctophm = (\PHs, \PHl; \PHh)$.
  Soient $S, S' \subset \BCc$.
  
  ($\Rightarrow$) Supposons que $S \BCtrans S'$.
    D'après la \defref{bc-sem},
    cela signifie, outre $S \neq S'$,
    qu'il existe une équation $X \xrightarrow{Y} Z \in \BCe$
    telle que $X \cup Y \subset S$,
    et qu'il existe un ensemble $X' \subset X$ tel que $S' = (S \setminus X') \cup Z$.
    Posons $Z' = Z \cap S$
    et $h = \PHfrappemult{(X_1 \cup Y_1 \cup Z'_0 \cup (Z \setminus Z')_1)}{(X'_0 \cup Z'_1)}$.
    Comme $S \neq S'$, on note que $X' \neq \emptyset \vee Z' \neq \emptyset$.
    D'après la \defref{bc2phm}, on a donc : $h \in \PHh$.
    Par ailleurs, on a alors : $X_1 \cup Y_1 \subseteq \tophm{S}$,
    et par définition de $Z'$, $Z'_0 \subseteq \tophm{S}$
    et $(Z \setminus Z')_1 \subseteq \tophm{S}$,
    donc $h$ est jouable dans $\tophm{S}$.
%     Enfin, $\tophm{S} \play h = \tophm{S} \recouvre (X'_0 \cup Z'_1) =
%       \tophm{(S \setminus X') \cup Z'}$.
    Enfin, $\tophm{S'} = \tophm{(S \setminus X') \cup Z} = \tophm{(S \setminus X') \cup Z'} =
      \tophm{S \setminus X'} \recouvre Z'_1 = (\tophm{S} \recouvre Z'_1) \recouvre X'_0 =
      \tophm{S} \recouvre (Z'_1 \cup X'_0)$
    car $X$, $Y$ et $Z$ sont disjoints, et $X'$ et $Z'$ aussi.
  
  ($\Leftarrow$) Supposons que $\tophm{S} \trans{\bctophm} \tophm{S'}$.
    Cela signifie qu'il existe une action $h \in \PHh$
    telle que $\tophm{S'} = \tophm{S} \play h$.
    Par ailleurs, l'existence d'une telle action implique, d'après la \defref{bc2phm},
    l'existence d'une équation $X \xrightarrow{Y} Z \in \BCe$ telle que :
    $h = \PHfrappemult{(X_1 \cup Y_1 \cup Z'_0 \cup (Z \setminus Z')_1)}{(X'_0 \cup Z'_1)}$,
    avec $X' \subset X$, $Z' \subset Z$ et $(X' \cup Z') \neq \emptyset$.
    Comme $h$ est jouable dans $\tophm{S}$, cela signifie notamment que
    $X_1 \subseteq \tophm{S}$ et $Y_1 \subseteq \tophm{S}$
    car $X$, $Y$ et $Z$ sont disjoints ; % (et $Z'$ et $(Z \setminus Z')$ aussi).
    autrement dit : $X \subset S$ et $Y \subset S$.
    Enfin, avec le même raisonnement que ci-dessus, on obtient :
    $\tophm{S'} = \tophm{S} \recouvre (Z'_1 \cup X'_0) = \tophm{(S \setminus X') \cup Z}$,
    d'où : $S' = (S \setminus X') \cup Z$.
\end{proof}

%\todo{Exemple}




% \section{$\pi$-calcul ?}
% 
% \section{Automates finis communicants ?}


\section{Bilan}
Nous avons présenté dans ce chapitre plusieurs traductions :
\begin{itemize}
  \item depuis les Frappes de Processus,
    vers le modèle de Thomas (\secref{trad-thomas})
    et les réseaux de Petri (\secref{trad-petri}) ;
  \item vers les Frappes de Processus,
    depuis les réseaux discrets asynchrones (\secref{trad-rda})
    et les systèmes d'équations biochimiques de la sémantique booléenne de Biocham
    (\secref{trad-biocham}).
\end{itemize}
Par ailleurs, nous avons donné une preuve d'équivalence entre les Frappes de Processus
et les automates synchronisés (\secref{phm2an}) avec les traductions adéquates.
Ces différentes traductions sont résumées au sein de la \vfigref{contrib-equivalences}.

Nous insistons ici sur le fait que si chacune des traductions exposées ici
s'effectue depuis ou vers un formalisme particulier des Frappes de Processus,
il est néanmoins possible de naviguer entre les différentes sémantiques de Frappes
de Processus sans pertes d'expressivité, comme l'ont montré
les différents résultats du \chapref{sem}.
Ainsi, les résultats particuliers proposés dans le présent chapitre
se généralisent à toutes les sémantiques de Frappes de Processus,
ce qui permet non seulement d'élargir le cadre de ces traductions
à ces différentes sémantiques,
mais aussi de conclure quant aux rapports d'expressivité entre les Frappes de Processus
et les autres types de modélisation abordés.
Il est à noter qu'il n'est pas impropre de parler ici de «~Frappes de Processus~»,
sans préciser la sémantique, lorsqu'il est question d'expressivité,
car les résultats du \chapref{sem} ont bien montré leur équivalence au niveau
des capacités de représentation ---~à l'exception notable des Frappes de Processus standards
qui ont probablement moins d'expressivité que les autres sémantiques.
