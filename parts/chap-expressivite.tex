% Expressivité du PH et positionnement par rapport à d'autres formalismes

\chapter{Expressivité du PH et positionnement par rapport à d'autres formalismes}
\chaplabel{expressivite}

\section{Réseaux discrets asynchrones}
\seclabel{trad-rda}

\todo{Traduction depuis CS2Bio-TCS}

\section{Modèle de Thomas}
\seclabel{trad-thomas}
Nous notons pour finir que la complexité de cette méthode est exponentielle dans le nombre
de régulateurs de chaque composant, et linéaire dans le nombre total de composants.
\footnote{Ce travail a été réalisé dans le cadre d'une collaboration avec Katsumi Inoue.
Cette collaboration a débuté par un stage doctoral de trois mois dans l'Inoue Laboratory,
au National Institute of Informatics (Tokyo, Japon).}

\section{Réseaux de Petri}
  \subsection{Réseaux de Petri saufs (une place par processus)}
  \subsection{Réseaux de Petri bornés (une place par sorte)}

\section{$\pi$-calcul ?}

\section{Automates finis communicants ?}

\section{Biocham ?}
