% Applications

\chapter{Applications}
\chaplabel{applications}

\chapeau{
  Nous proposons dans ce chapitre d'appliquer les différents résultats
  vus au cours de cette thèse
  %présentées aux \chapref{as} et \chapref{expressivite}
  à des modèles de grande taille.
  Les deux principales applications sont le suivantes :
  \begin{itemize}
    \item la dynamique d'un modèle de récepteur de lymphocytes T de 94 composants
      est étudiée à l'aide des méthodes d'analyse statique développées
      au \chapref{as},
    \item la traduction en modèle de Thomas
      d'un modèle de récepteur des cellules de croissance épithéliale de 104 composants 
      représenté en Frappes de Processus canoniques
      à l'aide de la traduction adéquate définie au \chapref{expressivite}.
  \end{itemize}
}

\todo{Présenter Pint}

\section{Gros exemple analyse statique}
\seclabel{appli-as-tcell}

Nous souhaitons démontrer dans cette section la puissance et l'adaptabilité
de l'analyse statique par sous-approximation développée à la \secref{as}.
Nous l'appliquons pour cela à un réseau booléen asynchrone
comportant 94 composants, ce qui peut être considéré comme un grand modèle,
et modélisant le récepteur de lymphocyte~T (\textit{T-cell receptor}).
Ce modèle est traduit en Frappes de Processus standards
et nous montrons que notre méthode reste efficace malgré la taille du modèle
et nous observons de plus que toutes les analyses effectuées sont conclusives.

\myskip

Nous nous intéressons au modèle de récepteur de lymphocyte~T
proposé par \citeasnoun{tcrsig94},
qui se présente sous la forme d'un réseau booléen asynchrone comportant 94 composants.
On y distingue notamment des composants d'entrée, c'est-à-dire qui ne sont régulés
par aucun autre composant, et dont la valeur de départ est donc fixe,
et des composants de sortie qui à l'inverse ne régulent aucun autre composant.
Dans l'état initial, les composants d'entrée sont généralement
indépendamment choisis actifs ou inactifs selon ce qu'on souhaite observer,
et tous les autres composants sont inactifs.
Cela permet d'observer la propagation du signal d'entrée caractérisée par une (dés)activation
successive des autres composants du modèle,
et notamment d'observer si les composants de sortie peuvent être activés
en fonction de la configuration initiale des composants d'entrée.

\TODO

Ce modèle a été traduit automatiquement en Frappes de Processus canoniques
à l'aide d'un programme basé sur \TODO









\begin{comment}
In order to support the scalability and applicability of our under-approximation of reachability, we
apply our new approach for the analysis of large-scale model of the T-cell receptor (TCR)
signalling pathway \cite{tcrsig94}.
This model gathers 94 interacting components and is specified as a Boolean network.
The under-approximation presented in this paper has been implemented in the existing Pint
software\footnote{Pint is freely available at \url{http://loicpauleve.name/pint}.}.

The Boolean model has been automatically encoded into a Process Hitting with 2 classes of priority%
\footnote{Files are available at
\url{http://maxime.folschette.name/underapprox-tcrsig94.zip}.}.
Then, we verified the reachability for the independent activation of 4 outputs of the signalling
cascade (SRE, AP1, NFkB, NFAT) from all possible input combinations (CD45, CD28, TCRlib) using our
new reachability under-approximation (answering either \emph{yes} or \emph{inconclusive}) and a 
previously defined reachability over-approximation \cite{PMR12-MSCS} (answering either \emph{no} or
\emph{inconclusive}).
All result in conclusive decisions, and the under-approximation has been satisfied in 12 cases (over
32) proving the satisfiability of the concerned reachability property in the encoded Boolean network
(and non-satisfiability in the other cases).

Computations times are in the order of a few hundredths of a second on a 2.4GHz processor with 2GB
of RAM.
To give a comparison, we did the same experiments with a standard symbolic model-checker, libDDD
\cite{libddd}, known for its good performances, the input model being the Boolean network expressed
as a Petri net.
However, due to the large scale of the model, the program runs out of memory for all the experiments.

While ensuring a low complexity for the analysis of reachability in Boolean and discrete networks, our
under-approximation method reveals to be conclusive in numerous cases when applied to real
large-scale biological models, which were not tractable with exact model-checking.
\end{comment}



\section{Gros exemples ph2thomas}
\seclabel{appli-thomas-egfr}
