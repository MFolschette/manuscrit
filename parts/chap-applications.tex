% Applications

\chapter{Applications sur des exemples de grande taille}
\chaplabel{applications}

\chapeaupublis{
  Nous proposons dans ce chapitre d'appliquer les différents résultats
  vus au cours de cette thèse
  %présentées aux \chapref{as} et \chapref{expressivite}
  à des modèles de grande taille.
  Les deux principales applications sont les suivantes :
  \begin{itemize}
    \item différents raffinements successifs
      d'un modèle de récepteur des cellules de croissance épithéliale de 20 composants 
      représenté en Frappes de Processus canoniques
      sont comparés à l'aide de leur traduction en modèle de Thomas,
      à l'aide des résultats du \chapref{expressivite}.
%       d'un modèle de récepteur des cellules de croissance épithéliale de 104 composants 
%       représenté en Frappes de Processus canoniques
%       à l'aide de la traduction adéquate définie au \chapref{expressivite}.
    \item la dynamique d'un modèle de récepteur de lymphocytes T de 94 composants
      est étudiée à l'aide des méthodes d'analyse statique développées
      au \chapref{as},
  \end{itemize}
}{%
  Ces approches ont été développées dans \cite*{FPIMR12-CMSB} et \cite*{FPMR13-CS2Bio}.
}



Nous avons présenté au \chapref{phcanonique} de cette thèse les Frappes de Processus
canoniques qui permettent d'unifier les différents formalismes de Frappes de Processus
à l'aide d'une classe de modèles particulière.
Nous proposons ensuite dans ce même chapitre des méthodes permettant
d'analyser la dynamique de cette classe de modèles % sous l'aspect d'accessibilités locales,
et au \chapref{expressivite} nous donnons notamment une traduction vers le modèle de Thomas
de tout modèle décrit dans ce formalisme.

Afin de montrer l'efficacité des méthodes développées dans le cadre de cette thèse,
nous proposons dans cette section d'en illustrer l'utilisation sur des modèles
de grande taille.
Nous considérons comme des modèles de grande taille les modèles comportant plusieurs
dizaines ou centaines de composants
---~en somme, les modèles dont la taille ne permet actuellement pas d'en étudier
le comportement à l'aide de \textit{model checkers} classiques
devant calculer l'intégralité de la dynamique.
Les exemples que nous donnons ici montrent que nos méthodes peuvent être
appliquées à des modèles de grande taille.
De plus, nous détaillons la nature et l'utilisation de l'implémentation
des différentes méthodes mises en pratique dans la suite.
Ces implémentations sont intégrées à la bibliothèque \Pint%
\footnote{\Pint{} est une bibliothèque libre, disponible à \url{http://loicpauleve.name/pint/}
accompagnée des modèles mentionnés dans ce chapitre
ainsi que des implémentations réalisées au cours de cette thèse.}
précédemment existante,
principalement développée et maintenue par Loïc Paulevé
et qui regroupe plusieurs utilitaires et outils propres aux Frappes de Processus.
%précédemment développés
Les implémentations réalisées dans le cadre de cette thèse y ont été intégrées,
et sont donc disponibles en ligne et sous une licence libre.

\myskip

Nous appliquons à la \secref{appli-thomas-egfr} notre méthode d'inférence
du modèle de Thomas à partir de Frappes de Processus canoniques proposée \vsecref{trad-thomas} ;
nous nous intéressons dans le détail à plusieurs modèles
de récepteur de facteur de croissance épidermique contenant 20 composants,
puis plus généralement aux résultats fournis sur différents modèles.
Nous détaillons ensuite à la \secref{appli-as-tcrsig} l'application de la méthode
d'analyse statique présentée
au \chapref{as} à un modèle du récepteur de lymphocyte~T comprenant 94 composants.

\todoplustard{Section spécifique sur Pint et implémentation + schéma des formats entrée/sortie}



\section{Traductions vers le modèle de Thomas}
\seclabel{appli-thomas}


\subsection{Application au récepteur de facteur de croissance épidermique}
\seclabel{appli-thomas-egfr}

Nous nous intéressons ici à l'étude d'un modèle de récepteur de facteur de croissant épidermique
\cite{Sahin09}.
Ce modèle est représenté par un graphe des interactions contenant 20 composants et 52 arcs.
Une protéine appelée EGF peut être considérée comme un composant d'entrée
car elle ne comporte aucun régulateur,
et une chaîne de réactions permet l'activation de la protéine pRB qui est responsable
de la régulation de la division cellulaire.
Celle-ci est donc essentielle pour la prévention du développement de cancers.

Nous créons trois modèles de Frappes de Processus canoniques
à partir du graphe des interactions original, avec différents niveaux
de précision dans la modélisation des coopérations entre composants.
\begin{itemize}
  \item Le modèle (1) représente la dynamique généralisée du graphe des interactions,
    telle que théorisée par \citeasnoun{PMR10-TCSB},
    c'est-à-dire sans aucune information concernant les règles booléennes des coopérations,
    et ne contient donc aucune sorte coopérative (ni aucune action primaire) ;
  \item Le modèle (2) est partiellement raffiné : il intègre certaines des règles
    booléennes en question en tant que sortes coopératives,
    choisies d'après des expériences de \textit{knockdown} menées sur le système ;
  \item Le modèle (3) est le modèle totalement raffiné contenant toutes les coopérations.
\end{itemize}
Les modèles (2) et (3) consistent donc en des raffinements successifs du modèle (1)
qui est le plus général au niveau de la dynamique.
La constitution de chaque modèle est détaillée dans la suite,
et les résultats de l'inférence du graphe des interactions et de l'inférence des paramètres
sur ces trois modèles sont donnés dans la \tabref{appli-thomas-egfr}
et discutés plus bas.
Nous ne nous intéressons ici qu'à des paramètres entiers,
autrement dit à des segments réduits à une seule valeur.

\begin{table}[ht]
  \ZifferAn
  \begin{center}
  \begin{tabular}{r|l|l|l|l|l} %m{2cm}|m{2.5cm}|m{1.5cm}}
    \textbf{Modèle} & $\mathbf{|E|}$ & $\mathbf{|K|}$ & \textbf{Paramètres inférés} &
      \textbf{Modèles possibles} & \textbf{Points fixes}
  \\\hline\hline
    (1) & $52$ & $196$ & $20$ & $2^{176}\simeq 9,6\cdot10^{52}$ & $0$   % v1_0.ph
  \\\hline
    (2) & $51$ & $192$ & $98$ & $2^{94}\simeq 2,0\cdot10^{28}$ & $0$    % v2_1.ph
  \\\hline
    (3) & $51$ & $192$ & $192$ & $1$ & $3$                              % ori.ph
  \\\hline
  \end{tabular}
  \end{center}
  \caption{\tablabel{appli-thomas-egfr}%
    Résultats de l'inférence du graphe des interactions et des paramètres
    sur les trois modèles dérivés du récepteur EGF %\cite{Sahin09}
    avec différentes précisions dans la définition des coopérations.
    Le modèle (1) ne contient aucune coopération entre les composants.
    Certaines coopérations ont été incluses dans le modèle (2) sous la forme de 14
    sortes coopératives et le modèle (3) contient toutes les coopérations entre composants
    sous la forme de 22 sortes coopératives.
    La deuxième colonne donne le nombre d'arcs dans le graphe des interactions inféré
    % TODO : ref vers trad-thomas
    (le nombre de nœuds étant toujours celui du modèle, c'est-à-dire 20).
    La troisième colonne donne le nombre total de paramètres à définir
    (calculé à partir du graphe des interactions),
    la quatrième colonne donne le nombre de paramètres entiers qui ont pu être inférés,
    % TODO : idem
    et la cinquième colonne donne le nombre de modèles compatibles avec le modèle
    de Frappes de Processus canoniques étudié,
    qui dépend de façon exponentielle du nombre de paramètres n'ayant pu être inférés.
    Finalement, la dernière colonne donne le nombre de points fixes du modèle,
    calculé à l'aide du \vthmref{php-pf}.
  }
  \ZifferAus
\end{table}

\todoplustard{GI visuel ?}

Le modèle (1) comprend uniquement des interactions individuelles entre composants,
c'est-à-dire des activations ou inhibitions indépendantes d'un composant à l'autre,
obtenues d'après les régulations contenues dans le graphe des interactions.
Ainsi, le graphe des interactions inféré d'après le modèle (1) est exactement identique
au graphe des interactions utilisé pour créer ce modèle,
à l'exception d'une auto-activation sur l'entrée EGF,
qui est due à son absence de régulateurs,
comme expliqué à la \vsecref{trad-thomas},
et se retrouve aussi pour les modèles suivants.
Les seuls paramètres ayant pu être inférés sont ceux qui concernent les configurations extrêmes
des ressources de chaque régulation,
à savoir dans le cas où tous les activateurs sont présents et tous les inhibiteurs absents,
et dans le cas inverse.
Ce premier modèle abstrait donc un grand nombre de modèles de Thomas
(de l'ordre de $9\cdot10^{52}$)
étant donné que la quasi-totalité des paramètres restent indéterminés.

Afin de construire le modèle (2), 14 sortes coopératives ont été ajoutées dans le but de
modéliser les fonctions booléennes de plusieurs composants (consistant en des portes logiques
de type ET et OU).
Pour ce faire, les composants suivants ont été retenus du fait de leur importance dans la
chaîne de réactions : CDK4, CDK6, CycD1, ER\nbd \textalpha{} and c\nbd MYC.
En effet, d'après les expériences par \textit{knockdown} menées par \citeasnoun{Sahin09},
empêcher ces composants de s'exprimer menait à une réduction significative de la production
de pRB.
On peut en conclure que ces composants sont impliqués dans les fonctions booléennes
d'autres composants d'une façon à ce que le \textit{knockdown} des premiers
empêche la production des seconds (ce qui est typique des portes de type ET).
Afin de reproduire ce comportement, les fonctions booléennes des successeurs de ces composants
ont été modélisées sous la forme de sortes coopératives le cas échéant, c'est-à-dire celles de
CDK4, CDK6, prB, p21, p27, IGF1R, MYC, CycD1 et CycE1.
En théorie, 9 sortes coopératives auraient suffi, mais la factorisation des sortes coopératives
décrite \vpageref{factorisation-coop} a été utilisée afin de réduire la taille de celles-ci
---~en nombre de processus~--- lorsque c'était possible.
Les sortes coopératives ainsi ajoutées ont permis d'inférer environ la moitié des paramètres ;
cependant, le nombre de modèles de Thomas compatibles avec ce modèle reste très élevé
du fait des nombreux paramètres qui restent indéterminés.
De plus, ce graphe des interactions inféré comporte un arc de moins que
le graphe des interactions d'origine.
Cela est dû au fait que la fonction booléenne de pRB intégrée sous la forme
d'une sorte coopérative pouvait en fait être simplifiée
d'une façon telle que le composant CDK2 n'y apparaissait plus.
Aucun arc n'a alors été inféré entre CDK2 et pRB par notre méthode,
car il ne participait pas à modifier la dynamique.

Enfin, le modèle (3) a été construit en utilisant toutes les fonctions booléennes
fournies par \citeasnoun{Sahin09}.
Ces fonctions prennent la forme de 22 sortes coopératives qui permettent de retrouver
le comportement attendu du système.
Toutes ces coopérations étant correctement définies dans le modèle,
tous les paramètres peuvent être inférés et un seul modèle de Thomas est compatible
avec la dynamique des Frappes de Processus canoniques utilisées.
Nous notons de plus que ce modèle est le seul à \bemph{contenir des points fixes},
calculés à l'aide du \vthmref{php-pf},
et qui comprennent notamment les deux états stables correspondant à une propagation
totale du signal pour les deux cas initiaux,
c'est-à-dire dans le cas où EGF est initialement actif et dans le cas où il ne l'est pas.
Les deux autres modèles ne contiennent pas de point fixe car les coopérations manquantes
créent des oscillations qui sont la conséquence du comportement non-déterministe
du formalisme lorsqu'on en reste à un stade de précision incomplet.

\subsection{Temps d'exécution sur quelques modèles de grande taille}
\seclabel{appli-thomas-large}

L'implémentation de la méthode proposée à la \vsecref{trad-thomas} permet de traiter
des modèles de
Frappes de Processus canoniques de grande taille et obtenus à partir de données
trouvées dans la littérature.
Ont notamment été traités :
\label{refmodels}
\begin{itemize}
  \item les modèles du récepteur de croissance épidermique présentés
    à la \vsecref{appli-thomas-egfr} \cite{Sahin09},
    qui contiennent 20 composants et jusqu'à 22 sortes coopératives,
  \item un modèle de récepteur de lymphocyte~T \cite{Klamt06},
    contenant 40 composants et 14 sortes coopératives.
\end{itemize}
Pour chacun de ces modèles, les inférences du graphe des interactions et des paramètres
sont effectuées en moins d'une seconde sur un ordinateur de bureau
standard\footnote{Pour tous les exemples de cette section,
le logiciel a été lancé sur une machine comportant
un processeur de 3,4~Ghz avec 8~Go de mémoire vivre.}

D'autres modèles de plus grande taille portant sur les mêmes systèmes ont aussi été testés :
\begin{itemize}
  \item un modèle du récepteur de lymphocyte~T contenant 94 composants,
    qui sera par ailleurs présenté à la section suivante \cite{SaezRodriguez2007},
  \item un modèle du récepteur de croissance épidermique avec 104 composants \cite{Samaga2009}.
\end{itemize}
Ces deux modèles ont été obtenus par \citeasnoun{PMR12-MSCS} à l'aide d'une traduction
automatique depuis le format CellNetAnalyzer \cite{klamt2007structural}
vers le Frappes de Processus canoniques.

La composition de ces modèles et les résultats de l'inférence du modèle de Thomas
sont résumés dans la \tabref{appli-thomas-large}.
Nous notons notamment que, du fait de la très grande taille des paramétrisations
des modèles de 94 et 104 composants, l'inférence des paramètres n'a pas pu être réalisée.
Ce nombre anormal de paramètres est dû à la présence d'un très grand nombre de régulateurs
pour un même composant dans ces deux modèles, faisant croître la taille de la paramétrisation
de façon exponentielle.
Il est à noter que ces régulations sont principalement dues à des contraintes de modélisation
propres à la représentation d'origine au format CellNetAnalyzer ;
notre méthode devrait néanmoins être en mesure de traiter des modèles d'aussi grande taille
à condition que chaque composant ne possède qu'un nombre limité de régulateurs.
Cela indique de plus que de tels modèles seraient plus efficacement étudiés
en tant que Frappes de Processus qu'en tant que modèles de Thomas.

\begin{table}[ht]
  \ZifferAn
  \begin{center}
  \begin{tabular}{r|c|c|c|c|c}
    \textbf{Modèle} & $\mathbf{\card{\Gamma}}$ & $\mathbf{\card{\cs}}$ &
      \textbf{Inférence GI} & \textbf{Inférence K} & $\mathbf{\card{K}}$
  \\\hline\hline
    RFCE & %\cite{Sahin09} &
      $20$ & $22$ & < 1 s & < 1 s & 192
  \\\hline
    RLT & %\cite{Klamt06} &
      $40$ & $14$ & < 1 s & < 1 s & 143
  \\\hline
    RLT & %\cite{SaezRodriguez2007} &
      $94$ & $39$ & 10 s & --- & $2,1\cdot10^{9}$
  \\\hline
    RFCE & %\cite{Samaga2009} &
      $104$ & $89$ & 3 min 20 s & --- & $4,2\cdot10^{6}$
  \end{tabular}
  \end{center}
  \caption{\tablabel{appli-thomas-large}%
    Temps d'exécution et informations relatives aux modèles
    utilisés dans l'inférence du graphe des interactions et de la paramétrisation
    de quatre modèles de réseaux de régulation biologique.
    Dans la première colonne, RFCE signifie «~Récepteur de facteur de croissance épidermique~»
    et RLT signifie «~Récepteur de lymphocyte~T~».
    Les références dont sont tirés ces modèles sont données \vpageref{refmodels}.
    La deuxième colonne donne le nombre de composants de chaque modèle
    et la troisième donne le nombre de sortes coopératives utilisées pour modéliser
    des coopérations entre composants.
    La quatrième (\resp cinquième) colonne donne le temps de calcul de l'inférence
    du graphe des interactions (\resp de la paramétrisation) pour chaque modèle,
    et la dernière colonne renseigne sur le nombre de paramètres que contient chaque modèle.
    Du fait d'un trop grand nombre de paramètres, l'inférence de la paramétrisation
    n'a pas pu être effectuée sur les modèles de $94$ et $104$ composants par manque de mémoire.
  }
  \ZifferAus
\end{table}

\todoplustard{Gros exemples avec inférence des paramètres possible}

\subsubsection*{Implémentation}

La traduction de Frappes de Processus canoniques en modèle de Thomas développée
à la \vsecref{trad-thomas} a été implémentée%
\footnote{Cette implémentation a été réalisée dans le cadre d'un stage doctoral
en tant qu'invité dans l'équipe de
Katsumi Inoue (National Institute of Informatics, Tokyo, Japon).}
et intégrée à \Pint.
L'implémentation consiste principalement en trois programmes
écrits en programmation par ensembles de réponses
(ou \textit{Answer Set Programming}),
un paradigme de programmation logique permettant de se concentrer sur la modélisation
du problème plutôt que sur sa résolution \cite{Baral03}.
Ce paradigme est notamment pris en charge par le \textit{grounder-solver} \tool{Clingo}
\cite{gekakasc14b},
qui a déjà montré ses capacités dans la résolution de problèmes de complexité NP.
Les trois programmes de cette implémentation se chargent respectivement
de réaliser l'inférence
du graphe des interactions, des paramètres et de tous les modèles de Thomas compatibles.
Des compléments écrits en \tool{OCaml} assurent de plus la traduction du modèle vers
le langage de programmation par ensembles de réponses
puis la récupération des résultats du \textit{solver}.
Cette traduction peut être appelée par l'outil \texttt{ph2thomas}, par exemple avec la
ligne de commande suivante :
\cl{ph2thomas -i ERBB\_G1-S.ph}
pour effectuer l'inférence des paramètres sur le modèle de Frappes de Processus canoniques
du fichier \texttt{ERBB\_G1-S.ph}.
Il est possible de plus d'exporter le graphe des interactions inféré dans un fichier
\texttt{ERBB\_G1-S.dot} à l'aide de l'option \texttt{-{}-dot ERBB\_G1-S.dot}
ou de lancer l'énumération des modèles de Thomas compatibles
avec \texttt{-{}-full-enumerate} (ou \texttt{-{}-enumerate} pour se restreindre
à des paramètres entiers).

\todoplustard{Exemple visuel du fichier .DOT}



\section{Analyse statique du récepteur de lymphocyte~T}
\seclabel{appli-as-tcrsig}

Nous souhaitons démontrer dans cette section la puissance et l'adaptabilité
de l'analyse statique par sous-approximation développée à la \secref{as}.
Nous l'appliquons pour cela à un réseau booléen asynchrone
comportant 94 composants, ce qui peut être considéré comme un grand modèle,
et modélisant le récepteur de lymphocyte~T (\textit{T-cell receptor}).
Ce modèle avait auparavant été étudié par une méthode proche dans le cadre
des Frappes de Processus standards \cite{PMR12-MSCS},
qui ne permettaient cependant d'étudier qu'une approximation du modèle d'origine.
Nous utilisons ici une représentation en Frappes de Processus standards
en assignant une priorité haute à toutes les actions mettant à jour les sortes coopératives
ajoutées pour représenter les coopérations entre composants.
Nous constatons que
\bemph{notre méthode d'analyse statique reste efficace malgré la taille du modèle} ;
par ailleurs, nous toutes les analyses effectuées s'avèrent être conclusives.

\myskip

Nous nous intéressons au modèle de récepteur de lymphocyte~T
proposé par \citeasnoun{SaezRodriguez2007},
qui se présente sous la forme d'un réseau booléen asynchrone comportant 94 composants.
On y distingue notamment des composants d'entrée, c'est-à-dire qui ne sont régulés
par aucun autre composant, et dont la valeur de départ est donc fixe,
et des composants de sortie qui à l'inverse ne régulent aucun autre composant.
Dans l'état initial, les composants d'entrée sont généralement
indépendamment choisis actifs ou inactifs selon ce qu'on souhaite observer,
et tous les autres composants le sont en fonction des conditions initiales habituelles
observées expérimentalement ---~la majorité étant inactifs.
Cela permet d'observer la propagation du signal d'entrée caractérisée par une (dés)activation
successive des autres composants du modèle.

Le but de cette étude est de tester la possibilité d'activer chaque composant de sortie
en fonction de l'état initial (activé ou non) de chaque composant d'entrée.
Cela permet notamment de vérifier que le modèle est fonctionnel,
autrement dit de vérifier que
tous les composants de sortie peuvent être activés si toutes les entrées le sont
---~et qu'elles ne peuvent pas l'être si aucune entrée n'est active.
De plus, tous les autres comportements intermédiaires peuvent être analysés
afin de prévoir le comportement du système réel dans ces situations.

Le modèle comporte trois composants d'entrée (appelés CD4, CD28 et TCRlig)
et quatre composants de sortie (SRE, AP1, NFkB et NFAT).
Il a été traduit automatiquement en Frappes de Processus canoniques%
\footnote{tous les fichiers sont disponibles à
\url{http://maxime.folschette.name/underapprox-tcrsig94.zip}.}
depuis le format CellNetAnalyzer \cite{klamt2007structural}.
%à l'aide d'un programme basé sur \storef.
Les états initiaux sont construits en définissant tous les composants
comme étant inactifs sauf pour certains d'entre eux
(à savoir : CD45, CARD11, Bcl10, Malt1, Rac1r, Lckr, cCblr et IkB)
et pour les composants d'entrée choisis comme initialement actifs.
Pour chacun des 8 états initiaux $\ctx$ ainsi obtenus,
et pour chaque composant $x$ de sortie,
nous vérifions la propriété $P_\ctx^x$ qui s'exprime de la manière suivante :
«~Depuis l'état $\ctx$, existe-t-il une succession d'actions permettant d'atteindre
un état où $x_1$ est actif ?~»
En d'autre termes, nous testons l'atteignabilité de chaque processus $x_1$
indépendamment.
Nous utilisons pour cela les deux analyses statiques suivantes :
\begin{itemize}
  \item la sous-approximation développée à la \vsecref{as},
    et notamment le \thmref{approxinf},
    qui permet de répondre «~Oui~» ou «~Non-conclusif~»
    à une question d'atteignabilité locale,
  \item la sur-approximation proposée par \citeasnoun{PMR12-MSCS},
    appliquée au modèle fusionné d'après la \vdefref{fusion},
    qui permet de répondre «~Non~» ou «~Non-conclusif~».
\end{itemize}
En d'autres termes, la première méthode, issue du travail présenté dans cette thèse,
teste l'atteignabilité afin de répondre positivement ;
son utilisation se justifie par le fait qu'elle est propre aux Frappes de Processus canoniques,
qui sont utilisées pour représenter le modèle étudié.
Dans les cas où cette méthode ne permet pas de répondre (c'est-à-dire lorsqu'elle
répond «~Non-conclusif~»),
la seconde méthode est utilisée en complément afin de tester l'atteignabilité
et de répondre négativement.
Dans le cas où les deux méthodes répondraient «~Non-conclusif~», aucune conclusion ne pourrait
être formulée pour la question d'atteignabilité ;
cependant, ce cas ne s'est pas présenté pour le modèle étudié ici.

Il s'avère finalement que cette méthode est conclusive pour tous les cas testés ;
autrement dit, les deux méthodes ne répondent jamais «~Non-conclusif~»
pour une même question d'atteignabilité.
Sur les 32 cas de figure testés,
nous avons notamment pu répondre «~Oui~» dans 12 cas
grâce au \vthmref{approxinf}.
Lorsque tous les composants d'entrée sont inactifs, aucun composant de sortie ne peut
être activé ;
à l'inverse, lorsque tous les composants d'entrée sont actifs,
SRE, AP1 et NFAT peuvent être activés mais pas NFkB,
ce qui permet de corriger le modèle de façon à la rendre actif en ajoutant une action manquante.
Les temps de calcul sont de l'ordre de quelques centièmes de seconde sur un ordinateur
de bureau classique\footnote{Testé sur une machine comportant
un processeur de 2,4~Ghz avec 2~Go de mémoire vivre.},
ce qui permet d'effectuer un grand nombre de tests sur un même modèle.
À titre de comparaison, les mêmes analyses ont été effectuées 
sur une traduction en réseau de Petri du modèle étudié,
à l'aide du \textit{model checker} symbolique libDDD \cite{Kordon09libddd},
connu pour ses bonnes performances.
Cependant, du fait de la taille du modèle, le programme s'arrête sur un dépassement de mémoire
pour chacun des cas testés.

\myskip

Pour terminer, nous testons en complément l'atteignabilité simultanée
de l'activation des quatre composants de sortie (SRE, AP1, NFkB et NFAT).
Ce résultat est utile car les propriétés précédemment étudiées consistaient uniquement
en des atteignabilités indépendantes, et il demeure toujours la possibilité
que l'une des atteignabilité n'en entrave une autre.
Pour ce résultat supplémentaire, nous modifions le modèle comme expliqué à la
\vsecref{as-etat}, qui consiste à ajouter les éléments suivants :
\begin{itemize}
  \item une sorte coopérative $\uastotal$ entre les quatre composants considérés,
  \item toutes les actions primaires nécessaires à la mise à jour de $\uastotal$,
  \item une sorte $\uasreach$ comprenant deux processus ($\uasreach_0$ et $\uasreach_1$),
  \item une action secondaire $\PHfrappe{\uastotal_{1111}}{\uasreach_0}{\uasreach_1}$.
\end{itemize}
Nous pouvons alors calculer l'atteignabilité de $\uasreach_1$, qui permettra de déterminer
si les quatre composants peuvent être simultanément actifs depuis l'état initial.

L'implémentation répond alors «~Oui~», ce qui permet d'affirmer que les quatre composants
peuvent être activés simultanément,
ce qui permet de conclure définitivement que le modèle est fonctionnel.
La réponse a été obtenue quelques centièmes de seconde,
en ne considérant qu'un sous-ensemble des nœuds solution,
selon la méthode expliquée par la remarque \vpageref{subsetsolution}.

% Cet écart de temps part rapport aux résultats de la section précédente est causé
% par deux facteurs :
% \begin{itemize}
%   \item L'intégralité du modèle doit être représenté par le graphe de causalité locale,
%     contrairement aux atteignabilités précédentes qui n'en représentaient qu'une portion ;
%   \item L'implémentation cherche à appliquer le \thmref{approxinf} de sous-approximation sur
%     différents sous-ensembles de solutions du graphe, or l'énumération de ces sous-ensembles
%     est exponentielle dans le nombre de solutions du graphe de causalité locale.
% \end{itemize}
% Malgré cela, notre méthode reste conclusive en un temps inférieur à la minute,
% là où des implémentations jugées efficaces mais nécessitant le calcul
% exhaustif de la dynamique ne terminent pas, comme nous l'avons montré plus haut.

\myskip

Nous avons démontré avec cet exemple la puissance de l'analyse statique que nous proposons
à la \vsecref{as}, applicable aux réseaux booléens asynchrones.
Bien que la dynamique soit approximée, tous les cas étudiés dans cet exemple restent conclusifs,
ce qui permet d'étudier la dynamique d'un grand modèle de 94 composants,
chose impossible avec les outils de \textit{model checking} classiques.

\subsubsection*{Implémentation}

Nous avons implémenté l'analyse statique développée à la \vsecref{as}
et l'avons intégré à la bibliothèque \Pint
sur la base de celle, déjà existante, portant sur les Frappes de Processus standards.
L'outil en question s'appelle \texttt{ph-reach}, et il est nécessaire de spécifier l'option
\texttt{-{}-coop-priority}
pour que les actions de mise à jour des sortes coopératives soient priorisées,
et que le modèle soit interprété comme des Frappes de Processus canoniques.
Voici un exemple de ligne de commande :
\cl{ph-reach -{}-coop-priority -i tcrsig94.ph\\
  \clspace\clspace-{}-initial-state "cd4 1, cd45 1, cd28 1, tcrlig 1" ap1 1}
Cette ligne de commande utilise le modèle du fichier \texttt{tcrsig94.ph} ;
elle définit l'état initial du modèle comme étant
$s \recouvre \{ \text{CD4}_1, \text{CD45}_1, \text{CD28}_1, \text{TCRlig}_1 \}$
(où $s$ est défini préalablement dans le fichier \texttt{tcrsig94.ph}
et $\recouvre$ représente l'opérateur de recouvrement défini à la \vdefref{recouvrement})
et calcule l'atteignabilité de $\text{AP1}_1$ à partir de cet état.
Après exécution, l'outil retourne \texttt{True}, ce qui signifie que l'atteignabilité
testée à l'aide de la méthode de la \vsecref{as} est possible.
Les autres réponses sont \texttt{False},
calculé d'après les travaux précédents de \citeasnoun{PMR12-MSCS},
et \texttt{Inconc}, qui indique qu'une des deux approximations n'a permis de conclure.



\section{Bilan}
\seclabel{appli-bilan}

Les applications présentées dans ce chapitre illustrent l'utilisation des méthodes
développées dans cette thèse.
Les différents résultats montrent que ces méthodes peuvent être appliquées à des modèles
de grande taille et fournissent donc des résultats intéressants.

\myskip

La traduction vers le modèle de Thomas permet de rendre compte des abstractions
permises par les Frappes de Processus.
En effet, un seul modèle de Frappes de Processus permet de représenter les dynamiques
«~superposées~» d'un ensemble de modèles de Thomas.
Cet ensemble peut être très grand, comme c'est le cas pour le modèle (1) étudié à la
\secref{appli-thomas} qui est compatible avec un très grand nombre de modèles.
La dynamique d'une telle abstraction serait donc mieux étudiée à l'aide d'un modèle
de Frappes de Processus canoniques, qu'avec un grand nombre de modèles de Thomas.
Cependant, s'il est possible de raffiner un tel modèle suffisamment, la traduction
s'avère intéressante pour pouvoir l'étudier à l'aide de nouvelles approches.

L'application à la \secref{appli-as-tcrsig} de l'analyse statique développée au \chapref{as}
montre que cette méthode est très efficace pour l'analyse d'un modèle
comportant une centaine de composants.
Cependant, au vu des temps de calcul (n'excédant pas le dixième de seconde),
il semble tout à fait envisageable de l'appliquer à des modèles de taille supérieure,
allant jusqu'à plusieurs milliers de composants.
Nous notons par ailleurs que l'analyse a toujours été conclusive, malgré l'approximation de la
dynamique effectuée.
Cette conclusivité s'explique notamment par la forme du modèle considéré,
qui comporte des interactions «~simples~» entre composants :
des activations et des inhibitions, parfois sous la forme de coopérations.
Les approximations développées par \citeasnoun{PMR12-MSCS} puis par \citeasnoun{FPMR13-CS2Bio}
prennent parti de cette forme particulière, propre aux réseaux de régulation biologique,
et sont donc particulièrement adaptées à ce type de modèles.

\myskip

Les applications proposées dans ce chapitre montrent en quoi les résultats que nous proposons
ouvrent la voie à de nombreuses possibilités d'analyse.
Il devient en effet possible d'analyser de grands modèles de réseaux de régulation biologique,
ce qui permet de nombreuses applications.
Nous en proposons quelques-unes :
\begin{itemize}
  \item Le raffinement des coopérations d'un modèle peut être traité
    en ajoutant des sortes coopératives à des endroits clefs
    et en testant certaines propriétés dynamiques (présence de points fixes ou
    atteignabilité des composants de sortie),
    dans le prolongement de l'application proposée à la \secref{appli-thomas}.
  \item L'inférence des actions manquantes d'un modèle à partir de données expérimentales
    nouvelles : chaque comportement biologique peut être testé sur le modèle,
    et en cas d'échec, le graphe de causalité locale peut être analysé pour en déduire
    les comportements manquants.
  \item Grâce aux techniques puissantes d'analyse de la dynamique, il devient potentiellement
    possible d'énumérer un grand nombre de modèles et de les tester indépendamment.
\end{itemize}
