% Applications

\chapter{Applications sur des exemples de grande taille}
\chaplabel{applications}

\chapeau{
  Nous proposons dans ce chapitre d'appliquer les différents résultats
  vus au cours de cette thèse
  %présentées aux \chapref{as} et \chapref{expressivite}
  à des modèles de grande taille.
  Les deux principales applications sont le suivantes :
  \begin{itemize}
    \item la dynamique d'un modèle de récepteur de lymphocytes T de 94 composants
      est étudiée à l'aide des méthodes d'analyse statique développées
      au \chapref{as},
    \item la traduction en modèle de Thomas
      d'un modèle de récepteur des cellules de croissance épithéliale de 104 composants 
      représenté en Frappes de Processus canoniques
      à l'aide de la traduction adéquate définie au \chapref{expressivite}.
  \end{itemize}
}



\todo{Présenter Pint}



\section{Analyse statique du récepteur de lymphocyte~T}
\seclabel{appli-as-tcell}

Nous souhaitons démontrer dans cette section la puissance et l'adaptabilité
de l'analyse statique par sous-approximation développée à la \secref{as}.
Nous l'appliquons pour cela à un réseau booléen asynchrone
comportant 94 composants, ce qui peut être considéré comme un grand modèle,
et modélisant le récepteur de lymphocyte~T (\textit{T-cell receptor}).
Ce modèle est traduit en Frappes de Processus standards
et nous montrons que notre méthode reste efficace malgré la taille du modèle
et nous observons de plus que toutes les analyses effectuées sont conclusives.

\myskip

Nous nous intéressons au modèle de récepteur de lymphocyte~T
proposé par \citeasnoun{tcrsig94},
qui se présente sous la forme d'un réseau booléen asynchrone comportant 94 composants.
On y distingue notamment des composants d'entrée, c'est-à-dire qui ne sont régulés
par aucun autre composant, et dont la valeur de départ est donc fixe,
et des composants de sortie qui à l'inverse ne régulent aucun autre composant.
Dans l'état initial, les composants d'entrée sont généralement
indépendamment choisis actifs ou inactifs selon ce qu'on souhaite observer,
et tous les autres composants le sont en fonction des conditions initiales habituelles
observées expérimentalement ---~la majorité étant inactifs.
Cela permet d'observer la propagation du signal d'entrée caractérisée par une (dés)activation
successive des autres composants du modèle.

Le but de cette étude est de tester la possibilité d'activer chaque composant de sortie
en fonction de l'état initial (activé ou non) de chaque composant d'entrée.
Cela permet notamment de vérifier que le modèle est fonctionnel,
autrement dit de vérifier que
tous les composants de sortie peuvent être activés si toutes les entrées le sont
---~et qu'elles ne peuvent pas l'être si aucune entrée c'est active.
De plus, tous les autres comportements intermédiaires peuvent être analysés
afin de prévoir le comportement du système réel dans ces situations.

Le modèle comporte trois composants d'entrée (appelés CD4, CD28 et TCRlig)
et quatre composants de sortie (SRE, AP1, NFkB et NFAT).
Il a été traduit automatiquement en Frappes de Processus canoniques%
\footnote{tous les fichiers sont disponibles à
\url{http://maxime.folschette.name/underapprox-tcrsig94.zip}.}
à l'aide d'un programme basé sur \storef.
Les états initiaux sont construits en définissant tous les composants
comme étant inactifs sauf CD45, CARD11, Bcl10, Malt1, Rac1r, Lckr, cCblr
et les composants d'entrée sélectionnés.
Pour chacun des états 8 initiaux ainsi obtenus,
nous testons l'atteignabilité des processus $x_1$ de chaque composant $x$ de sortie
indépendamment à l'aide des deux analyses statiques suivantes :
\begin{itemize}
  \item La sous-approximation développée à la \vsecref{as},
    et notamment le \thmref{approxinf},
    qui permet de répondre «~Oui~» ou «~Non-conclusif~» ;
  \item La sur-approximation proposée par \citeasnoun{PMR12-MSCS},
    appliquée au modèle fusionné d'après la \vdefref{fusion},
    qui permet de répondre «~Non~» ou «~Non-conclusif~».
\end{itemize}
L'utilisation de ces deux approches conjointement permet d'obtenir un résultat dans un
plus grand nombre de cas,
l'analyse par sous-approximation présentée à la \vsecref{as} de cette thèse
ne permettant pas de répondre «~Non~».

Il s'avère finalement que cette méthode est conclusive pour tous les cas testés ;
autrement dit, les deux méthodes ne répondent jamais «~Non-conclusif~»
pour une même question d'atteignabilité.
Sur les 32 cas de figure testés,
nous avons notamment pu répondre «~Oui~» dans 12 cas
grâce au \vthmref{approxinf}.
% Lorsque tous les composants d'entrée sont inactifs, aucun composant de sortie ne peut
% être activé ;
% à l'inverse, lorsque tous les composants d'entrée sont actifs,
% SRE, AP1 et NFAT peuvent être activés mais pas NFkB.
Les temps de calcul sont de l'ordre de quelques centièmes de seconde sur un ordinateur
de bureau classique\footnote{Testé sur une machine comportant
un processeur de 2.4~Ghz avec 2~Gio de mémoire vivre.},
ce qui permet d'effectuer un grand nombre de tests sur un même modèle.
À titre de comparaison, les mêmes analyses ont été effectuées 
sur une traduction en réseau de Petri du modèle étudié,
à l'aide du \textit{model checker} symbolique libDDD \cite{Kordon09libddd},
connu pour ses bonnes performances.
Cependant, du fait de la taille du modèle, le programme effectue un dépassement de mémoire
pour chacun des cas testés.

\myskip

Nous avons démontré avec cet exemple la puissance de l'analyse statique que nous proposons
à la \vsecref{as}, applicable aux réseaux booléens asynchrones.
Bien que la dynamique soit approximée, tous les cas étudiés dans cet exemple restent conclusifs,
ce qui permet d'étudier la dynamique d'un grand modèle de 94 composants,
chose impossible avec les outils de \textit{model checking} classiques.



\section{Traduction vers le modèle de Thomas}
\seclabel{appli-thomas}


\section{Application au récepteur de facteur de croissance épidermique}
\seclabel{appli-thomas-egfr}

Nous nous intéressons ici à l'étude d'un modèle de récepteur de facteur de croissant épidermique
\cite{Sahin09}.
Ce modèle est représenté par un graphe des interactions contenant 20 composants et 52 arcs.
Une protéine appelée EGF peut être considérée comme un composant d'entrée
car elle ne comporte aucun régulateur,
et une chaîne e réactions permet l'activation de la protéine pRB qui est responsable
de la régulation de la division cellulaire.
Celle-ci est donc essentielle pour la prévention du développement de cancers.

Trois modèles de Frappes de Processus canoniques
sont créés à partir du graphe des interactions original, avec différents niveaux
de précision dans la modélisation des coopérations entre composants.
\todo{traduits comment ?}
\begin{itemize}
  \item Le modèle (1) représente la dynamique généralisée du graphe des interactions,
    telle que théorisée par \citeasnoun{PMR10-TCSB},
    c'est-à-dire sans aucune information concernant les règles booléennes des coopérations,
    et ne contient donc aucune sorte coopérative (ni aucune action primaire) ;
  \item Le modèle (2) est partiellement raffiné : il intègre quelques-unes des règles
    booléennes en question en tant que sortes coopératives,
    choisies d'après des expériences de \textit{knockdown} menées sur le système ;
  \item Le modèle (3) est le modèle totalement raffiné contenant toutes les coopérations.
\end{itemize}
Les modèles (2) et (3) consistent donc en des raffinements successifs du modèle (1)
qui est le plus général au niveau de la dynamique.
La constitution de chaque modèle est détaillée dans la suite,
et les résultats de l'inférence du graphe des interactions et de l'inférence des paramètres
sur ces trois modèles sont donnés dans la \tabref{appli-thomas-egfr}
et discutés plus bas.

\begin{table}[ht]
  \ZifferAn
  \begin{center}
  \begin{tabular}{r|l|l|l|l|l} %m{2cm}|m{2.5cm}|m{1.5cm}}
    \textbf{Modèle} & $\mathbf{|E|}$ & $\mathbf{|K|}$ & \textbf{Paramètres inférés} &
      \textbf{Modèles possibles} & \textbf{Points fixes}
  \\\hline\hline
    (1) & $52$ & $196$ & $20$ & $2^{176}\simeq 9,6\cdot10^{52}$ & $0$   % v1_0.ph
  \\\hline
    (2) & $51$ & $192$ & $98$ & $2^{94}\simeq 2,0\cdot10^{28}$ & $0$    % v2_1.ph
  \\\hline
    (3) & $51$ & $192$ & $192$ & $1$ & $3$                              % ori.ph
  \\\hline
  \end{tabular}
  \end{center}
  \caption{\tablabel{appli-thomas-egfr}%
    Résultats de l'inférence du graphe des interactions et des paramètres
    sur les trois modèles dérivés du récepteur EGF %\cite{Sahin09}
    avec différentes précisions dans la définition des coopérations.
    Le modèle (1) ne contient aucune coopération entre les composants.
    Certaines coopérations ont été incluses dans le modèle (2) sous la forme de 14
    sortes coopératives et le modèle (3) contient toutes les coopérations entre composants
    sous la forme de 22 sortes coopératives.
    La deuxième colonne donne le nombre d'arcs dans le graphe des interactions inféré
    à l'aide de \storef (le nombre de nœuds étant toujours celui du modèle, c'est-à-dire 20).
    La troisième colonne donne le nombre total de paramètres à définir
    (calculé à partir du graphe des interactions),
    la quatrième colonne donne le nombre de paramètres qui ont pu être inférés
    en utilisant \storef,
    et la cinquième colonne donne le nombre de modèles compatibles avec le modèle
    de Frappes de Processus canoniques étudié,
    qui dépend de façon exponentielle du nombre de paramètres n'ayant pu être inférés.
    Finalement, la dernière colonne donne le nombre de points fixes du modèle,
    calculé à l'aide du \vthmref{php-pf}.
  }
  \ZifferAus
\end{table}

\todoplustard{GI visuel ?}

Le modèle (1) comprend uniquement des interactions individuelles entre composants,
c'est-à-dire des activations ou inhibitions indépendantes d'un composant à l'autre,
obtenues d'après les régulations contenues dans le graphe des interactions.
Ainsi, le graphe des interactions inféré d'après le modèle (1) est exactement identique
au graphe des interactions utilisé pour créer ce modèle,
à l'exception d'une auto-activation sur l'entrée EGF,
qui est due à son absence de régulateurs.
\stodo{Préciser}
Les seuls paramètres ayant pu être inférés sont ceux qui concernent les configurations extrêmes
des ressources de chaque régulation,
à savoir dans le cas où tous les activateurs sont présents et tous les inhibiteurs absents,
et dans le cas inverse.
Ce premier modèle abstrait donc un grand nombre de modèle de Thomas
(de l'ordre de $9\cdot10^{52}$)
étant donné que la quasi-totalité des paramètres restent indéterminés.

Afin de construire le modèle (2), 14 sortes coopératives ont été ajoutées dans le but de
modéliser les fonctions booléennes de plusieurs composants (consistant en des portes ET et OU).
Pour ce faire, les composants suivants ont été retenus du fait de leur importance dans la
chaîne de réactions : CDK4, CDK6, CycD1, ER\nbd \textalpha{} and c\nbd MYC.
En effet, d'après les expériences par \textit{knockdown} menées par \citeasnoun{Sahin09}
empêcher ces composants de s'exprimer menait à une réduction significative de la production
de pRB.
On peut en conclure que ces composants sont impliqués dans les fonctions booléennes
d'autres composants d'une façon à ce que le \textit{knockdown} des premiers
empêche la production des seconds (ce qui est typique des portes de type ET).
Afin de reproduire ces requis, les fonctions booléennes des successeurs de ces composants
ont été modélisées sous la forme de sortes coopératives le cas échéant, c'est-à-dire celles de
CDK4, CDK6, prB, p21, p27, IGF1R, MYC, CycD1 et CycE1.
En théorie, 9 sortes coopératives auraient suffi, mais la factorisation des sortes coopératives
décrite à \storef a été utilisée afin de réduire la taille ---~en nombre de processus~---
de celles-ci lorsque c'était possible.

\TODO

\begin{comment}

In order to build model (2), 14 cooperative sorts were added in order to model the Boolean functions of several components
(consisting of AND and OR operands).
To do so, the following components were noticed due to their importance in the chain of reactions:
CDK4, CDK6, CycD1, ER$\alpha$ and c-MYC.
Indeed, knockdown experiments have been conducted in~\cite{Sahin09}
and the results showed that knocking down these components lead to an important decrease in the production of pRB.
We therefore concluded that these components were involved in other components' Boolean functions
in a way that the knockdown of the former was sufficient to prevent the production of the latter (which is typical of AND operands).
In order to reproduce such requirements, the Boolean functions of their successors,
that is CDK4, CDK6, prB, p21, p27, IGF1R, MYC, CycD1 and CycE1,
were modeled as cooperative sorts, if needed.
In theory, 9 cooperative sorts would have sufficed, but the chaining of cooperative sorts described
in \pref{ssec:PH} was used to reduce the number of processes in each cooperative sort.
As a result, the added cooperations allowed to infer about half the parameters;
however, the number of possible Thomas models that can be inferred from this PH is still significant
because of the numerous remaining unknown parameters.
Furthermore, we note that the inferred IG contains one edge less than the original IG. This is due to the fact that
one of the Boolean functions could in fact be simplified in a way that a component did not appear anymore in it.
No edge have therefore been inferred by our method in this case.

Finally, model (3) was build using all the Boolean functions provided in~\cite{Sahin09}.
These functions take the form of 22 cooperative sorts into the model in order to match the desired behavior of the system.
As all cooperations are fully defined in this model, all the parameters are inferred and only one Thomas model can be derived.
We note also that this PH model is the only one containing at least one fixed point.
In fact, the three found steady states include the two states that correspond to a complete propagation of the input signal,
that is, in the case where EGF is active and in the case where it is not.
The two other models contain no fixed point because some cooperations are not fully defined,
leading to oscillations that are a consequence of the nondeterministic behavior.



\subsection{Computation times on several large models}\label{ssec:cpu}

The current implementation can successfully handle large PH models of BRNs found in the literature such as:
\begin{itemize}
  \item the EGF receptor model from~\cite{Sahin09} with 20 components presented in the last
    subsection\footnote{All models mentioned in this section are available as examples distributed with \textsc{Pint}.},
  \item a T cell receptor model described as an IG in~\cite{Klamt06}, which contains 40 components and 14 cooperative sorts.
\end{itemize}
For each model, IG and parameters inferences are performed together in less than a second
on a standard desktop computer.

Bigger models related to the aforementioned systems were also tested with our implementation:
\begin{itemize}
  \item a model of the T cell receptor with 94 components, described in~\cite{SaezRodriguez2007},
  \item a model of the EGF receptor with 104 components, described in~\cite{Samaga2009}.
\end{itemize}
These two models were obtained in a previous work by an automatic translation from the CellNetAnalyzer~\cite{klamt2007structural} formalism.

The composition of all models and the results of the inferences are summarized in \pref{tb:computation}.
We note that due to a very high number of parameters, no parameters inference could be performed on the $94$ and $104$ component models.
These models would therefore be more efficiently studied as PH than as BRNs.
Finally, we note that the complexity of the method is exponential in the number of regulators of one
component and linear in the number of components.

\begin{table}[ht]
\begin{center}
  \begin{tabular}{r|c|c|c|c|c}
    \textbf{Model} & $\mathbf{|\Gamma|}$ & $\mathbf{|\PHs \setminus \Gamma|}$ & \textbf{$\Delta t$ IG} & \textbf{$\Delta t$ K} & $\mathbf{|K|}$
\\\hline\hline
    EGF receptor \cite{Sahin09} & $20$ & $22$ & <1s & <1s & 192
\\\hline
    T cell receptor \cite{Klamt06} & $40$ & $14$ & <1s & <1s & 143
\\\hline
    T cell receptor \cite{SaezRodriguez2007} & $94$ & $39$ & 10s & --- & $2.1\cdot10^{9}$
\\\hline
    EGF receptor \cite{Samaga2009} & $104$ & $89$ & 3min 20s & --- & $4.2\cdot10^{6}$
  \end{tabular}
\end{center}
\caption{%
  Computation times and several pieces of information related to the IG and parametrization inferences of four biological models.
  The second column gives the number of components of each model and
  the third column gives the number of cooperative sorts used to model joint actions.
  The fourth (resp.~fifth) column gives the computation times of the IG inference (resp.~the parametrization inference).
  The last column gives the number of parameters in each model.
  Due to the too high number of parameters, parametrization inference could not be performed on the $94$ and $104$ component models.
  }
\label{tb:computation}
\end{table}
\end{comment}
