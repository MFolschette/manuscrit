\section{Les Frappes de Processus standard}

\subsection{Définition}
\seclabel{ph}

Les Frappes de Processus permettent une modélisation atomique des interactions entre composants.
\Defref{ph}
\todo{Description}

\begin{definition}[Frappes de Processus]
\deflabel{ph}
  Les \emph{Frappes de Processus} sont définies
  par un triplet $\PH = (\PHs; \PHl; \PHh)$, où :
  \begin{itemize}
    \item $\PHs \DEF \{a, b, \dots\}$ est l'ensemble fini et dénombrable des \emph{sortes} ;
    \item $\PHl \DEF \underset{a \in \PHs}{\times} \PHl_a$ est l'ensemble fini des \emph{états},
      où $\PHl_a = \{a_0, \ldots, a_{l_a}\}$ est l'ensemble fini et dénombrable
      des \emph{processus} de la sorte $a \in \PHs$ et $l_a \in \sN^*$.
      Chaque processus appartient à une unique sorte :
      $\forall (a_i; b_j) \in \PHl_a \times \PHl_b, a \neq b \Rightarrow a_i \neq b_j$ ;
    \item $\PHh \DEF \{\PHfrappe{a_i}{b_j}{b_l} \mid (a; b) \in \PHs \times \PHs \wedge
      (a_i; b_j; b_l) \in \PHl_a \times \PHl_b \times \PHl_b \wedge
      b_j \neq b_l \wedge a = b \Rightarrow a_i = b_j \}$ est l'ensemble fini des \emph{actions}.
  \end{itemize}
\end{definition}
%
\noindent
On note $\PHproc \DEF \bigcup_{a \in \PHs} \PHl_a$ l'ensemble de tous les processus.
La sorte d'un processus $a_i$ est donnée par $\PHsort(a_i) = a$.
Étant donné un état $s \in \PHl$, le processus de la sorte $a \in \PHs$ présent dans $s$ est donné
par $\PHget{s}{a}$, \cad la coordonnée correspondant à $a$ dans l'état $s$.
Si $a_i \in \PHl_a$, nous définissons la notation : $a_i \in s \EQDEF \PHget{s}{a} = a_i$.
Pour toute action $h = \PHfrappe{a_i}{b_j}{b_k} \in \PHh$,
$a_i$ est appelé le \emph{frappeur}, $b_j$ la \emph{cible} et $b_k$ le \emph{bond} de $h$,
et on note : $\hitter{h} = a_i$, $\target{h} = b_j$ et $\bounce{h} = b_k$.

La \defref{substate} établit la notion de sous-état sur un ensemble de sortes,
\cad un ensemble de processus qui sont deux à deux de sortes différentes,
ce qui permet de ne considérer qu'une partie d'un état complet.
Nous notons qu'un état est \textit{a fortiori} un sous-état : $\PHl \subset \PHsubl$.
Le recouvrement d'un état $s$ par un processus $a_i$ est formalisé à la \defref{statecap}
par un état identique à $s$, sauf pour le processus de $a$ qui a été remplacé par $a_i$,
ce qui permet de définir la dynamique des Frappes de Processus avec $k$ classes de priorités
dans la \defref{play}.
La définition de recouvrement est aussi étendue à un sous-ensemble,
autrement dit, un ensemble de processus contenant au plus un processus par sorte.

\begin{definition}[Sous-états ($\PHsublize{\PHl}$)]
\deflabel{substate}
  Si $S \subset \PHs$ est un ensemble de sortes, un sous-ensemble sur $S$ est un élément de :
  $\PHsubl[\PHl]_S \DEF \bigtimes{a \in S} \PHl_a$.
  L'ensemble de tous les sous-ensembles est noté :
  $\PHsubl[\PHl] \DEF \bigcup_{S \in\powerset(\PHs)} \PHsubl[\PHl]_S$.
  
  \noindent
  De plus, si $\mysigma \in \PHsubl[\PHl]$ et $s \in \PHl$, on note alors :
  \[\mysigma \subseteq s \EQDEF \forall a_i \in \Proc, a_i \in \mysigma \Rightarrow a_i \in s
    \enspace.\]
  
  \noindent
  Enfin, si $S \subset \PHs$, on note :
  $\PHsublset_S = \{ \toset{ps} \subset \Proc \mid ps \in \PHsubl_S \}$
  et
  $\PHsublset = \{ \toset{ps} \subset \Proc \mid ps \in \PHsubl \}$.
\end{definition}
%
\begin{definition}[$\Cap : \PHl \times \PHproc \rightarrow \PHl$]
\deflabel{statecap}
  Étant donné un état $s \in \PHl$ et un processus $a_i \in \PHproc$,
  $(s \Cap a_i)$ est l'état défini par :
  $\PHget{(s \Cap a_i)}{a} = a_i \wedge \forall b \neq a, \PHget{(s \Cap a_i)}{b} = \PHget{s}{b}$.
  On étend de plus cette définition à un tout ensemble de processus,
  à condition que tous les processus soient tous de sortes différentes,
  par le recouvrement de l'état par chaque processus :
  $\forall ps \in \PHsubl[\PHl], s \Cap \toset{ps} = s \underset{a_i \in \toset{ps}}{\Cap} a_i$.
\end{definition}

Une propriété de jouabilité telle que décrite à la \defref{ppl}
est équivalente à une formule booléenne dont les atomes sont des processus dans $\Proc$.
Le langage des propriétés de jouabilité permet de décrire la présence d'une configuration
de processus actifs dans un état donné.
Il permet notamment de décrire en termes formels la jouabilité d'une action,
ce qui est immédiatement mis en pratique dans la \defref{fopph}.

\begin{definition}[Propriété de jouabilité ($\F$)]
  \label{def:ppl}
  Une \emph{propriété de jouabilité} est un élément du langage $\F$ défini \todo{inductivement ?} par :
  \begin{itemize}
    \item $\top$ et $\bot$ appartiennent à $\F$ ;
    \item si $a \in \PHs$ et $a_i \in \PHl_a$, alors $a_i$ appartient à $\F$
      et est appelé un \emph{atome} ;
    \item si $P \in \F$ et $Q \in \F$,
      alors $\neg P$, $P \wedge Q$ et $P \vee Q$ appartiennent à $\F$.
  \end{itemize}
%
  Si $P \in \F$ est une propriété de jouabilité et $\mysigma \in \PHsubl$ est un sous-ensemble,
  on note $\Feval{P}{\mysigma}$ l'\emph{évaluation} de $P$ dans $\mysigma$:
  \begin{itemize}
    \item si $P = a_i \in \PHl_a$ est un atome, avec $a \in \PHs$,
      alors $\Feval{a_i}{\mysigma}$ est vraie si et seulement si $a_i \in \mysigma$ ;
    \item si $P$ n'est pas un atome, alors $\Feval{P}{\mysigma}$ est vraie si et seulement si
      on peut l'évaluer récursivement comme vraie en utilisant la sémantique habituelle des
      opérateurs $\neg$, $\wedge$ et $\vee$ et des constantes $\top$ et $\bot$.
  \end{itemize}
%
  Une fonction $\Fopsymbol : \PHh \rightarrow \F$ associant à toute action une propriété de jouabilité
  est appelée un \emph{opérateur de jouabilité}.
\end{definition}

Étant donné que ce langage n'utilise que des opérateur logiques classiques,
les propriétés de la logique booléenne sont applicables aux propriétés de jouabilité,
à savoir celles concernant la distributivité, l'associativité et la commutativité,
ainsi que les lois de De Morgan concernant la négation.

Il en résulte notamment la propriété suivante, permettant d'évaluer la négation d'un atome,
et qui dérive naturellement du fait que si un processus n'est pas actif dans un état donné,
cela signifie alors qu'un autre processus de la même sorte l'est :
\[\forall a \in \PHs, \forall a_i \in \PHl_a, \forall \mysigma \in \PHsubl,
  \Feval{\neg a_i}{\mysigma} \Leftrightarrow
  \Feval{\bigvee_{\substack{a_j \in \PHl_a\\a_j \neq a_i}} a_j}{\mysigma}\]

Enfin, on note dans la suite :
\[\forall A \in \PHsublset, \Fconj{A} \equiv \bigwedge_{p \in A} p \enspace.\]

\todo{Gluer}

\begin{definition}[Opérateur de jouabilité ($\Fopsymbol : \PHh \rightarrow \F$)]
\deflabel{fopph}
  L'opérateur de jouabilité des Frappes de Processus est défini par :
  \[\forall h \in \PHh, \Fop{h} \equiv \hitter{h} \wedge \target{h} \enspace.\]
\end{definition}

\begin{definition}[Dynamique des Frappes de Processus ($\PHtrans$)]
\deflabel{play}
  Une action $h \in \PHh$ est dite \emph{jouable}
  dans l'état $s \in \PHl$ si et seulement si :
  $\Feval{\Fop{h}}{s}$.
%  $\target{h} \in s \wedge \Feval{\Fop{h}}{s}$.
  Dans ce cas, $(s \PHplay h)$ est l'état résultant du jeu de l'action $h$ dans $s$,
  et on le définit par : $(s \PHplay h) = s \Cap \bounce{h}$.
  De plus, on note alors : $s \PHtrans (s \PHplay h)$.

  Si $s \in \PHl$, un \emph{scénario} $\delta$ dans $s$
  est une séquence d'actions de $\PHh$ qui peuvent être jouées successivement dans $s$.
  L'ensemble de tous les scénarios dans $s$ est noté $\Sce(s)$.
\end{definition}



\subsection{Analyse statique}

\todo{Rappels sur les résultats d'analyse statique (cf. MSCS'10)}
