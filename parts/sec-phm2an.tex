% Équivalence avec les réseaux d'automates synchronisés

\section{Équivalence avec les réseaux d'automates synchronisés}
\seclabel{phm2an}

Nous nous intéressons ici au lien entre les Frappes de Processus avec actions plurielles
et les réseaux d'automates synchronisés.
Nous montrons notamment que ces deux formalismes sont équivalents
et nous exhibons pour cela deux traductions d'un formalisme vers l'autre.
%(\defref{phm2an,an2phm} \vdefpageref{phm2an}).
%(\defref[s]{phm2an} et \defref*[vref]{an2phm}).
Cette équivalence est intéressante car elle montre clairement le lien entre ce formalisme
de Frappes de Processus et celui plus répandu des réseaux d'automates synchronisés.
En effet, chaque action plurielle $\PHfrappemult{A}{B}$
possède la même dynamique qu'un ensemble de transitions synchronisées
partant chacune d'un processus de l'ensemble $A$ et
arrivant dans le processus de la même sorte de l'ensemble $A \recouvre B$%
\footnote{La notation $A \recouvre B$, formalisée à la \vdefref{recouvrementps}, représente
l'ensemble où chaque processus de $A$ a été remplacé par
le processus de $B$ de la même sorte, s'il existe.}.

\myskip

Nous rappelons tout d'abord la définition d'un réseau d'automates synchronisés (\defref{an})
ainsi que la relation de transition entre deux états d'un tel modèle (\defref{an-sem})
ce qui permet d'en définir la dynamique.

\begin{definition}[Réseau d'automates synchronisés]
\deflabel{an}
  Un \emph{réseau d'automates synchronisés} est un quadruplet $\AN = (\ANs; \ANl; \ANi; \ANt)$
  où :
  \begin{itemize}
    \item $\ANs \DEF \{a, b, \dots\}$ est l'ensemble fini et dénombrable des \emph{automates} ;
    \item $\ANl \DEF \bigtimes{a \in \ANs} \ANl_a$ est l'ensemble fini des \emph{états},
      où $\ANl_a = \{a_0, \ldots, a_{l_a}\}$ est l'ensemble fini et dénombrable
      des \emph{états locaux} de l'automate $a \in \ANs$ et $l_a \in \sN^*$,
      chaque état local appartenant à un unique automate :
      $\forall (a_i; b_j) \in \ANl_a \times \ANl_b, a \neq b \Rightarrow a_i \neq b_j$ ;
    \item $\ANi \DEF \{\ell_1, \dots, \ell_m\}$ est l'ensemble fini des
      \emph{libellés} de transitions ;
    \item $\ANt \DEF \{ \ANaction{a_i}{\ell}{a_j} \mid a \in \ANs \wedge a_i \in \ANl_a \wedge
      \ell \in \ANi \}$ est l'ensemble fini des \emph{transitions} entre états locaux.
  \end{itemize}
  Pour tout libellé $\ell \in \ANi$, on note
  $\precond{\ell} \DEF \{ a_i \mid \ANaction{a_i}{\ell}{a_j} \in \ANt \}$
  et $\postcond{\ell} \DEF \{ a_j \mid \ANaction{a_i}{\ell}{a_j} \in \ANt \}$.
%   et $\invcond{\ell} \DEF \{ a_i \mid \ANaction{a_i}{\ell}{a_i} \in \ANt \}$.
  L'ensemble des états locaux des automates est dénoté par
  $\ANProc \DEF \bigcup_{a \in \ANs} \ANl_a$.
  Enfin, étant donné un état global $s \in \ANl$, $s(a) = a_i \in \ANl_a$
  fait référence à l'état local de l'automate $a \in \ANs$.
\end{definition}

\begin{definition}[Sémantique des réseaux d'automates ($\ANtrans$)]
\deflabel{an-sem}
  Étant donné un réseau d'automates synchronisés $\AN = (\ANs; \ANl; \ANi; \ANt)$,
  un libellé $\ell$ est dit \emph{jouable} dans un état $s \in \ANl$ si et seulement si :
  $\forall a_i \in \precond{\ell}, s(a) = a_i$.
  Dans ce cas, on note $(s \play \ell)$ l'état résultant du jeu de toutes les transitions
  libellées par $\ell$, défini par :
  $s \play \ell = s \recouvre \postcond{\ell}$.
%   $\forall a_j \in \postcond{\ell}, (s \play \ell)(a) = a_j \wedge
%     \forall b \in \ANs, \ANl_b \cap \precond{\ell} = \emptyset \Rightarrow
%     (s \play \ell)(b) = s(b)$.
  De plus, on note alors : $s \ANtrans (s \play \ell)$.
%   Étant donné un réseau d'automates synchronisés $\AN = (\ANs; \ANl; \ANi; \ANt)$,
%   la relation de transition globale entre deux états du réseau
%   $\ANtrans \subset \ANl \times \ANl$ est définie par :
%   \begin{align*}
%     s \ANtrans s' \EQDEF \exists \ell \in \ANi, &\forall a_i \in \precond{\ell}, s(a) = a_i
%       \wedge \forall a_j \in \postcond{\ell}, s'(a) = a_j \\
%     \wedge &\forall b \in \ANs, \ANl_b \cap \precond{\ell} = \emptyset \Rightarrow s(b) = s'(b)
%   \end{align*}
\end{definition}

\begin{remark}
  Nous notons que les réseaux d'automates synchronisés ainsi définis sont non-déterministes,
  tant au niveau global du modèle qu'au niveau local des automates.
  Cette vision s'oppose à d'autres sémantiques des réseaux d'automates
  comme celles de \citeasnoun{Richard10} ou de \citeasnoun{RRT08},
  qui définissent la dynamique de leurs modèles à l'aide de fonctions de transition locales,
  qui sont par définition déterministes.
  Ces fonctions ont en effet la forme : $f_a : \ANl \rightarrow \ANl_a$
  et associent donc à chaque état global du modèle un état local (unique) pour chaque automate.
  La définition des réseaux d'automates synchronisés que nous proposons ici (\defref{an})
  n'empêche en revanche pas l'existence de deux libellés $\ell_1, \ell_2 \in \ANi$
  tels que $\precond{\ell_1} = \precond{\ell_2}$ mais $\postcond{\ell_1} \neq \postcond{\ell_2}$.
  Cela implique notamment l'existence de deux transitions entre état locaux
  $\ANaction{a_i}{\ell_1}{a_j}$ et $\ANaction{a_i}{\ell_2}{a_k}$
  avec $a_j \neq a_k$, d'où un non-déterminisme au niveau des automates.
\end{remark}

Pour tout modèle de Frappes de Processus avec actions plurielles $\PH$,
la \defref{phm2an} propose une traduction de $\PH$
en un réseau d'automates synchronisés $\phmtoan[\PH]$ équivalent,
et le \thmref{bisimulationphm2an} établit la bisimilarité entre les deux modèles.
La notation $\recouvre$ qui est utilisée dans la définition
qualifie le recouvrement d'un ensemble de processus de sortes distinctes
par un autre comprenant uniquement des processus issus des mêmes sortes
(\defref{recouvrementps}).
Cette notion de recouvrement est une extension
du recouvrement d'un état par un ensemble de processus
tel que précédemment formalisé à la \vdefref{recouvrement}.

\begin{definition}[Recouvrement ($\recouvre : \PHsublset \times \PHsublset \rightarrow \PHsublset$)]
\deflabel{recouvrementps}
  Étant donné un sous-état désordonné $ps \in \PHsublset$ et un processus $a_i \in \Proc$,
  tel que $a \in \sortes{ps}$, on définit :
  $(ps \recouvre a_i) = (ps \setminus \PHl_a) \cup \{ a_i \}$.
  On étend de plus cette définition
  au recouvrement par un ensemble de processus de sortes distinctes
  $ps' \in \PHsublset$ tel que $\sortes{ps'} \subset \sortes{ps}$
  comme étant le recouvrement de $ps$ par chaque processus de $ps'$ :
  $ps \recouvre ps' = ps \underset{a_i \in ps'}{\recouvre} a_i$.
\end{definition}

\begin{definition}[Réseau d'automates équivalent ($\phmtoansymbol$)]
\deflabel{phm2an}
  Le réseau d'automates synchronisés équivalent aux Frappes de Processus
  avec actions plurielles $\PH = (\PHs; \PHl; \PHh)$
  est défini par : $\phmtoan = (\PHs; \PHl; \ANi; \ANt)$, où :
  \begin{itemize}
    \item $\ANi = \{ \ell_h \mid h \in \PHh \}$ ; % est l'ensemble des libellés de transitions ;
    \item $\ANt = \{ \ANaction{a_i}{\ell_h}{a_j} \mid
      h \in \PHh \wedge h = \PHfrappemult{A}{B} \wedge a_i \in A \wedge a_j \in A \recouvre B \}$.
      % est l'ensemble des transitions locales.
  \end{itemize}
\end{definition}

\begin{theorem}[$\PH \approx \phmtoan$]
\thmlabel{bisimulationphm2an}
  Soient $\PH = (\PHs; \PHl; \PHh)$ des Frappes de Processus avec actions plurielles.
  On a :
  \[\forall s, s' \in \PHl, s \PHtrans s' \Longleftrightarrow s \trans{\phmtoan} s' \enspace.\]
\end{theorem}

\begin{proof}
  Soient $s, s' \in \PHl$.
  On pose : $\phmtoan = (\ANs; \ANl; \ANi; \ANt)$.
  
  ($\Rightarrow$) Supposons que $s \PHtrans s'$, c'est-à-dire qu'il existe une action $h \in \PHh$
    telle que $s' = s \play h$.
    Posons : $h = \PHfrappemult{A}{B}$.
    D'après la \defref{phm2an},
    l'existence de cette action dans $\PH$ implique celle d'un libellé $\ell_h$ dans $\phmtoan$
    ainsi que de l'ensemble de transitions
    $\ANt_h = \{ a_i \xrightarrow{\ell_h} a_j \mid a_i \in A \wedge a_j \in A \recouvre B \}$.
    Autrement dit, $\precond{\ell_h} = A$, donc $\ell_h$ est jouable dans $s$
    si et seulement si $A \subseteq s$.
    De plus, $\postcond{\ell_h} = \invariant{h} \cup B$, donc
    $(s \play \ell_h) = s \recouvre (\invariant{h} \cup B) = s \recouvre B = s'$
    car $\invariant{h} \subseteq A \subseteq s$.
  
  ($\Leftarrow$) Supposons que $s \trans{\phmtoan} s'$,
    c'est-à-dire qu'il existe un libellé $\ell \in \ANi$ et un ensemble de transitions
    ayant ce libellé : $\ANt_\ell = \{ a_i \xrightarrow{\ell} a_j \in \ANt \}$,
    tels que $s' = s \play \ell$.
    D'après la \defref{phm2an}, cela signifie notamment qu'il existe une action
    $h = \PHfrappemult{A}{B} \in \PHh$ telle que $\ell = \ell_h$, et que :
    $\ANt_\ell = \{ a_i \xrightarrow{\ell} a_j \mid a_i \in A \wedge a_j \in A \recouvre B \}$.
    Étant donné que $\invariant{h}$ et $\cible{h}$ forment une partition de $A$,
    $\ANt_\ell$ peut être découpé en deux ensembles, selon les invariants et les cibles de $h$ :
    $\ANt_\ell = \{ a_i \xrightarrow{\ell} a_i \mid a_i \in \invariant{h} \} \cup
      \{ a_i \xrightarrow{\ell} a_j \mid a_i \in \cible{h} \wedge a_j \in B \}$.
    Ainsi, $s' = s \recouvre (\invariant{h} \cup B) = s \recouvre B = s \play h$.
\end{proof}

Pour finir, nous proposons à la \defref{an2phm} la traduction inverse
d'un réseau d'automates synchronisés $\AN$
en des Frappes de Processus avec actions plurielles équivalentes $\antophm$.
Le \thmref{bisimulationan2phm} stipule que le modèle obtenu est bien bisimilaire
au modèle d'origine.
Enfin, le \thmref{equivphman} résume les résultats de cette section
en statuant l'équivalence d'expressivité entre les Frappes de Processus avec
actions plurielles et les réseaux d'automates synchronisés.

\begin{definition}[Frappes de Processus équivalentes ($\antophmsymbol$)]
\deflabel{an2phm}
  Les Frappes de Processus avec actions plurielles
  équivalentes au réseau d'automates synchronisés $\AN = (\PHs, \PHl, \ANi, \ANt)$
  sont définies par $\antophm = (\ANs, \ANl, \PHh)$, où :
%   $\PHh = \{ \PHfrappemult{\precond{\ell}}{(\postcond{\ell} \setminus \invcond{\ell})}
%     \mid \ell \in \ANi \}$.
  \[\PHh = \{ \PHfrappemult{\precond{\ell}}{B} \mid \ell \in \ANi \wedge
    B = \postcond{\ell} \setminus \{ a_i \in \ANProc \mid \ANaction{a_i}{\ell}{a_i} \in \ANt \}
    \}\]
\end{definition}

\begin{theorem}[$\AN \approx \antophm$]
\thmlabel{bisimulationan2phm}
  Soit $\AN = (\ANs; \ANl; \ANi; \ANt)$ un réseau d'automates synchronisés.
  On a :
  \[\forall s, s' \in \ANl, s \ANtrans s' \Longleftrightarrow s \trans{\antophm} s' \enspace.\]
\end{theorem}

\begin{proof}
  Soient $s, s' \in \PHl$.
  On pose : $\antophm = (\ANs; \ANl; \PHh)$.
  
  ($\Rightarrow$) Supposons que $s \ANtrans s'$,
    c'est-à-dire qu'il existe un libellé $\ell \in \ANi$ et un ensemble de transitions
    ayant ce libellé : $\ANt_\ell = \{ a_i \xrightarrow{\ell} a_j \in \ANt \}$,
    tels que $s' = s \play \ell$.
    D'après la traduction donnée à la \defref{an2phm}, il existe donc une action
    $h = \PHfrappemult{A}{B} \in \PHh$ telle que $A = \precond{\ell}$ et
    $B = \postcond{\ell} \setminus \{ a_i \in \ANProc \mid \ANaction{a_i}{\ell}{a_i} \in \ANt \}$.
    Or $s' = s \recouvre \postcond{\ell}
      = s \recouvre (B \cup \{ a_i \in \ANProc \mid \ANaction{a_i}{\ell}{a_i} \in \ANt \})
      = s \recouvre B$
    car $\{ a_i \in \ANProc \mid \ANaction{a_i}{\ell}{a_i} \in \ANt \} \subseteq s$.
    Ainsi, $h$ est jouable dans $s$ et $s' = s \play h$.
  
  ($\Leftarrow$) Supposons que $s \trans{\antophm} s'$,
    c'est-à-dire qu'il existe une action $h = \PHfrappemult{A}{B} \in \PHh$
    telle que $s' = s \play h$.
    D'après la traduction de la \defref{an2phm},
    cela signifie qu'il existe un libellé $\ell \in \ANi$ et un ensemble de transitions
    ayant ce libellé : $\ANt_\ell = \{ a_i \xrightarrow{\ell} a_j \in \ANt \}$,
    tels que : $A = \precond{\ell}$ et
    $B = \postcond{\ell} \setminus \{ a_i \in \ANProc \mid \ANaction{a_i}{\ell}{a_i} \in \ANt \}$.
    Comme $h$ est jouable dans $s$, alors $A \subseteq s$, donc $\ell$ est aussi jouable dans $s$.
    De plus, $s' = s \play h = s \recouvre B = s \recouvre (B \cup \invariant{h})
      = s \recouvre \postcond{\ell}$.
\end{proof}

\begin{theorem}[Équivalence entre réseaux d'automates synchronisés
  et Frappes de Processus avec actions plurielles]
\thmlabel{equivphman}
  Les Frappes de Processus avec actions plurielles sont aussi expressives
  que les réseaux d'automates synchronisés.
\end{theorem}

\begin{proof}
  D'après les \defref{phm2an,an2phm} et les \thmref{bisimulationphm2an,bisimulationan2phm}
  associés, tout modèle de Frappes de Processus avec actions plurielles peut être représenté
  à l'aide d'un réseau d'automates synchronisés, et inversement.
\end{proof}

\todo{Exemple}
