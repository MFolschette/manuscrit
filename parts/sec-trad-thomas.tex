% Inférence du modèle de Thomas

\section{Inférence du modèle de Thomas}
\seclabel{trad-thomas}

\TODO

\todo{Nous notons pour finir que la complexité de cette méthode est exponentielle dans le nombre
de régulateurs de chaque composant, et linéaire dans le nombre total de composants.
\footnote{Ce travail a été réalisé dans le cadre d'une collaboration avec Katsumi Inoue.
Cette collaboration a débuté par un stage doctoral de trois mois dans l'Inoue Laboratory,
au National Institute of Informatics (Tokyo, Japon).}}

\todo{Citer FPIMR-10}

Dans toute cette section, nous considérons un modèle de Frappes de Processus canoniques
$\PH = (\PHs; \PHl; (\PHh^{(1)}; \PHh^{(2)}))$.

\subsection{Inférence du graphe des interactions}
\seclabel{trad-thomas-gi}

Un graphe des interactions (\vdefref{thomas-gi}) est une représentation abstraite des
influences directes, positives ou négatives, entre les composants d'un système.
Comme discuté à la \vsecref{thomas-analyse}, le graphe des interactions permet
de caractériser efficacement les propriétés dynamiques globales du système,
à l'aide notamment de résultats comme les conjectures de Thomas,
qui apportent des résultats sur la présence d'oscillations ou d'états stables multiples.

Dans le cas d'un processus de modélisation d'un réseau de régulation biologique, le modèle
de Thomas est le point de départ e la spécification du modèle.
Cependant, il est courant que le graphe des interactions initialement conçu contienne des
influences qui n'ont pas d'impact sur la dynamique.
La méthode que nous proposons dans la suite s'appuie directement sur la dynamique d'un modèle
de Frappes de Processus canoniques, ce qui produit des graphes des interactions minimaux,
et permet d'affiner les conclusions de telles méthodes d'analyse statique.

L'intuition de cette inférence est que seuls les composants (les sortes dans $\components$)
figureront dans le graphe des interactions ;
les sortes coopératives (dans $\cs$) sont uniquement étudiées pour comprendre les actions
«~indirectes~» entre composants.



\subsubsection{Frappes de Processus bien-formées}

Nous notons que dans cette section, les indices des processus de composants
possèdent une importance particulière,
notamment pour contraindre le fait que la dynamique doit être unitaire (\crref{unitaire}).
Autrement dit, si on suppose que ces indices représentent des niveaux d'expression discrets
ordonnés,
par exemple si $b_0$, $b_1$ et $b_2$ représentent le fait que le composant $b$
est présent respectivement en faible, moyenne et forte concentration,
alors une action de la forme $\PHfrappe{a_1}{b_0}{b_2}$ n'est pas autorisée ;
en revanche, deux actions $\PHfrappe{a_1}{b_0}{b_1}$ et $\PHfrappe{a_1}{b_1}{b_2}$ le sont.
Naturellement, toute autre relations d'ordre entre les indices est admissible,
à condition qu'une contrainte d'unicité similaire puisse être définie.

\begin{equation}
\components = \{a \in \PHs \mid \nexists \PHfrappe{b_i}{a_j}{a_k} \in \PHa, |j - k| > 1\} \\
\eqlabel{PH-components}
\end{equation}

\begin{critere}[Dynamique unitaire]
\crlabel{unitaire}
  Toutes les actions secondaires de $\PH$ ne font pas de bond
  à plus d'un processus d'écart :
  $\forall \PHfrappe{a_i}{b_j}{b_k} \in \PHh^{(2)}, \card{j - k} = 1$
\end{critere}

\begin{example}
  Les Frappes de Processus canoniques représentées à la \figref{infer-ex-ph}
  possèdent une dynamique unitaire, et sont donc compatibles
  avec l'inférence du graphe des interactions proposée à la section suivante.
\end{example}

\begin{remark}
  Le \crref{unitaire} est naturellement vérifié pour tout modèle booléen,
  c'est-à-dire tel que : $\forall a \in \components, \card{\PHl_a} = 2$.
\end{remark}

\begin{remark}
  Il est possible de ne pas prendre en compte le \crref{unitaire},
  à condition de s'affranchir de l'aspect unitaire de la dynamique du modèle de Thomas.
  Les résultats de cette section restent alors théoriquement applicables.
\end{remark}

Nous considérons dans la suite que les Frappes de Processus canoniques $\PH$ respectent
le \crref{unitaire}.



\subsection{Inférence des interactions}
\seclabel{infer-gi}


L'inférence de cette section est directement inspirée des travaux de \citeasnoun{Richard10},
qui déduit les influences d'un graphe des interactions en fonction des évolutions des différents
composants, selon l'état de ses régulateurs.

\newcommand{\myupsilon}{\upsilon}

Pour tout composant $a$,
les \emph{prédécesseurs} de $a$, notés $\pred(a)$,
sont toutes les sortes ayant au moins une action frappant $a$.
Les \emph{régulateurs} de $a$, en revanche, notés $\reg(a)$, sont tous les composants
qui influent sur $a$, soit directement, soit à travers une sorte coopérative.
Il est à noter que les régulateurs définis de cette manière seront potentiellement
des régulateurs de $a$ dans le modèle de Thomas inféré,
tels que définis \vpageref{regulateurs},
ce qui explique pourquoi ces deux définitions sont proches.

\begin{align*}
  \forall a \in \components, \pred(a) &\DEF \{ b \in \PHs \mid
    \exists h \in \PHh, \sorte{\frappeur{h}} = b \wedge
    \sorte{\cible{h}} = a \} \\
  \forall a \in \components, \reg(a) &\DEF \{ \compin(b) \mid
    b \in \pred(a) \}
\end{align*}
Où $\compin(b)$ fait référence aux composants qui régulent la sorte coopérative $b$,
autrement dit aux sortes que $b$ représente
(cf.~\vdefref{comp}).

L'étude des influences d'un composant $b$ régulant un autre composant $a$ nécessite d'étudier
le groupe de régulateurs de $b$ qui vont influencer conjointement $a$.
Ces groupes de régulateurs sont aisément déterminés en observant les sortes coopératives.
Nous proposons ici de définir les groupes de régulateurs comme étant les composants connexes
d'un graphe reliant tous les régulateurs de $a$ qui sont représentés par une même sorte
coopérative :
\[
  \forall a \in \components,
  X(a) \DEF \mathcal{C} \big( (\reg(a),
  \{ \{b,c\} \subset \compin(\myupsilon) \mid
  \myupsilon \in \pred(a) \cap \cs \}) \big)
\]
Où $\mathcal{C}(G)$ représente l'ensemble des composantes connexes du graphe non orienté $G$.

Pour étudier l'influence d'un groupe de régulateurs $g$ sur un composant $a$,
nous effectuons une analyse exhaustive de toutes les configurations possibles de $g$.
Pour cela, il est nécessaire de définir un sous-état $\sigma$ sur les sortes de $g$,
et et de compléter ce sous-état par les processus de sorte coopérative
qui représentent l'état des composants dans $g$.
Nous définissons pour cela l'ensemble $\allFocals{a}{g}{\sigma}$
qui contient l'état des sortes de $g$, celui de $a$, et celui de toutes les sortes
coopératives frappant $a$.

%allFocals
\begin{align*}
  &\forall a \in \components, \forall g \in X(a), \forall \sigma \in \PHsubl_{g \cup \{ a \}}, \\
  & \quad
  \allFocals{a}{g}{\sigma} = \{ \PHget{\sigma}{b} \mid b \in \pred(a) \cap \components \}
  \cup \{ \pfp_\sigma(b) \mid b \in \pred(a) \cap \cs \}
\end{align*}
Avec :
\[
  \pfp_\sigma(b) = \pfp_{s \recouvre \sigma}(b)
\]
où le choix de $s \in \PHl$ est indifférent d'après le point (\ref{csss}) de la \vdefref{cs}.

Enfin, il est possible d'étudier localement la dynamique de $a$ en fonction du sous-état
$\sigma$ d'un groupe de régulateurs $g$ donné ;
cette dynamique locale se concentre donc uniquement sur les actions frappant $a$.
En effet, en faisant varier l'un des composants $b \in g$ et en observant le résultat
sur l'évolution de $a$ (tendance à l'augmentation ou à la diminution de son niveau d'expression),
il est possible d'en déduire l'influence locale de $b$ sur $a$ pour un niveau d'expression
de $b$ donné.
Pour cela, nous appelons $\irB_a(\sigma)$ l'ensemble des processus vers lesquels $a$ peut évoluer
depuis le sous-état $\sigma$ ;
naturellement, si aucune action ne frappe $a$ dans $\sigma$,
alors $\irB_a(\sigma) = \PHget{\sigma}{a}$.

\begin{align*}
  &\forall g \in X(a), \forall \sigma \in \PHsubl_{g \cup \{ a \}},
  \irB_a(\sigma) \DEF 
  \begin{cases}
    \irF_a(\sigma)
      & \text{ si } \irF_a(\sigma) \neq \emptyset\\
    \{ \PHget{\sigma}{a} \}
      & \text{ si } \irF_a(\sigma) = \emptyset
  \end{cases}\\
  &\text{où : } \irF_a(\sigma) \DEF \{ a_k \in \PHl_a \mid
    \exists b \in \PHs, \exists \PHfrappe{b_i}{a_j}{a_k} \in \PHa,
  \{ b_i, a_j \} \subset \allFocals{a}{g}{\sigma} \}\\
\end{align*}


La \propref{inference-edges} détaille l'inférence de toutes les influences locales existant
entre les composants, c'est-à-dire celles qui se produisent pour un seuil donné $t$.
L'idée principale derrière cette inférence est la suivante :
s'il existe une une influence
positive (\resp négative) d'un composant $b$ sur un autre composant $a$,
alors augmenter le niveau d'expression de $b$
va potentiellement faire faire augmenter (\resp diminuer) le niveau d'expression de $a$,
au moins dans certaines configurations (\eqref{edges-inference}).
Ainsi, ces influences locales se séparent en influences positives et négatives,
ce qui représente de potentiels arcs dans le graphe des interactions final.
De plus, l'étude des influences sur les groupes de régulateurs d'un composant $a$
permet aussi d'étudier les auto-influences de $a$
(\eqref{edges-inference-auto})
ce qui permettra potentiellement d'inférer des auto-arcs.
Finalement, il est nécessaire d'étudier le cas particulier où $a$ ne possède pas de régulateurs
(\eqref{edges-inference-noreg}).
Nous notons que cette méthode ignore naturellement tous les cas où il n'est pas possible
de distinguer une influence d'un composant sur un autre.

\begin{proposition}[Inférence des influences]
\proplabel{inference-edges}
  Nous définissons l'ensemble $\hat{E}_+$ (\resp $\hat{E}_-$)
  des \emph{influences locales positives} (\resp \emph{négatives})
  pour tout composant $a \in \components$ par :
  % Arcs a -> b, a ≠ b
  \begin{align}
    \begin{split}\eqlabel{edges-inference}
      \forall b \in \reg(a), \forall s \in \{ +, - \}, \\
      b \xrightarrow{t+1} a \in \hat{E}_s \Longleftrightarrow\ & \exists g \in X(a), b \in g,
      \exists \sigma \in \PHsubl_{g \cup \{ a \}}, \\
        &\qquad \{ b_t, b_{t+1} \} \subset \PHl_b \wedge b_t \in \sigma,\\
        &\qquad \exists a_j \in \irB_a(\sigma), \exists a_k \in \irB_a(\sigma\{b_{t+1}\}), \\
        &\qquad s = \f{signe}(k - j)
    \end{split}
    \end{align}
    % Auto-arcs depuis les groupes de régulateurs
    \begin{align}
    \begin{split}\eqlabel{edges-inference-auto}
      \forall s \in \{ +, - \}, \quad\qquad\qquad \\
      a \xrightarrow{t+1} a \in \hat{E}_s \Longleftrightarrow\ & \exists g \in X(a),
      \exists \sigma \in \PHsubl_{g \cup \{ a \}}, \\
        &\qquad \{ a_t, a_{t+1} \} \subset \PHl_a \wedge a_t \in \sigma,\\
        &\qquad \exists a_j \in \irB_a(\sigma), \exists a_k \in \irB_a(\sigma\{a_{t+1}\}), \\
        &\qquad s = \f{signe}(k - j)
    \end{split}
    \end{align}
    % Auto-arcs des composants sans prédécesseurs
    \begin{align}
    \begin{split}\eqlabel{edges-inference-noreg}
      \forall s \in \{ +, - \}, \quad\qquad\qquad \\
      a \xrightarrow{t+1} a \in \hat{E}_s \Longleftrightarrow\ & \reg(a) = \emptyset \wedge
        \{ a_t, a_{t+1} \} \subset \PHl_a, \\
        &\qquad \exists a_j \in \irB_a(\etat{a_t}), \exists a_k \in \irB_a(\etat{a_{t+1}}), \\
        &\qquad s = \f{signe}(k - j)
    \end{split}
  \end{align}
  où : $\f{signe}(x) = \begin{cases}
    + & \text{ si $x > 0$} \\
    - & \text{ si $x < 0$} \\
    0 & \text{ if $x = 0$}
  \end{cases}$
\end{proposition}

Nous sommes alors en mesure d'inférer les arcs du graphe des interactions final,
à partir de ces ensembles d'influences locales positives et négatives.
En effet, nous pouvons inférer une influence (globale) positive ou négative
d'un composant vers un autre
s'il n'existe que des influences locales correspondantes du même signe.
Une influence non-signée est inférée si, à l'inverse, il existe au moins deux influences
locales correspondantes de signes différents.
Enfin, le seuil de chaque influence (quel que soit son signe)
est égal au seuil minimum pour lequel une influence locale a été trouvée.
Nous formalisons cette inférence dans la \propref{inference-gi}.

\begin{proposition}[Inférence du graphe des interactions]
\proplabel{inference-gi}
  Nous inférons $\GI = (\components; E)$ à l'aide de la \propref{inference-edges} comme suit :
  \begin{align*}
    E_+ &= \{ \arc{a}{+}{t}{b} \mid \nexists a \xrightarrow{t'} b \in \hat{E}_-
      \wedge t = \min \{ r \mid a \xrightarrow{r} b \in \hat{E}_+ \}\} \\
    E_- &= \{ \arc{a}{-}{t}{b} \mid \nexists a \xrightarrow{t'} b \in \hat{E}_+
      \wedge t = \min \{ r \mid a \xrightarrow{r} b \in \hat{E}_- \}\} \\
    E_\uns &= \{ \arc{a}{\uns}{t}{b} \mid \exists a \xrightarrow{t'} b \in \hat{E}_+ \wedge
      \exists a \xrightarrow{t''} b \in \hat{E}_- \\
      & \qquad\qquad\qquad \wedge t = \min \{ r \mid
      a \xrightarrow{r} b \in \hat{E}_- \cup \hat{E}_+ \} \}
  \end{align*}
\end{proposition}




\begin{figure}[ht]
\centering
\scalebox{1.3}{
\begin{tikzpicture}
  \path[use as bounding box] (-4,-1.9) rectangle (4.5,3.9);

  \TSort{(0,0)}{a}{3}{l}
  \TSort{(3, 3)}{b}{2}{t}
  \TSort{(3,-1)}{c}{2}{b}

  \TSetTick{bc}{0}{00}
  \TSetTick{bc}{1}{01}
  \TSetTick{bc}{2}{10}
  \TSetTick{bc}{3}{11}
  % \TSetSortLbcel{bc}{$\neg a\wedge b$}
  \TSort{(-3,-0.5)}{bc}{4}{l}

  \THit{bc_3}{}{a_1}{.north west}{a_2}
  \THit{bc_0}{}{a_1}{.south west}{a_0}
  \path[bounce]
  \TBounce{a_1}{bend left}{a_2}{.south west}
  \TBounce{a_1}{bend right}{a_0}{.north west}
  ;

  \THit{b_0}{}{a_2}{.east}{a_1}
  \THit{b_1}{}{a_0}{.north east}{a_1}
  \path[bounce]
  \TBounce{a_2}{bend left}{a_1}{.north east}
  \TBounce{a_0}{bend right=20}{a_1}{.south}
  ;

  \THit{c_0}{bend right}{a_2}{.south east}{a_1}
  \THit{c_1}{bend right}{a_0}{.east}{a_1}
  \path[bounce]
  \TBounce{a_2}{bend left=20}{a_1}{.north}
  \TBounce{a_0}{bend right=30}{a_1}{.south east}
  ;

  \path[]
    (1.9,-1.3) edge[bend left=10,coopupdate] (-2.2,-0.7)
    (1.9, 3.3) edge[bend right=10,coopupdate] (-2.2,3)
  ;

  \THit{a_2}{bend left,out=40,in=80}{b_1}{.north west}{b_0}
  \path[bounce, bend right]
  \TBounce{b_1}{}{b_0}{.east}
  ;
\end{tikzpicture}
}
\caption{\figlabel{infer-ex-ph}%
  Exemple de Frappes de Processus canoniques avec trois composants ($a$, $b$ et $c$)
  et une sorte coopérative ($bc$).
  La dynamique de ce modèle est unitaire car il respecte bien le \vcrref{unitaire}.
  L'inférence du graphe des interactions peut donc être effectuée sur ce modèle.
}
\end{figure}



\begin{example}
  L'application de l'inférence du graphe des interactions aux Frappes de Processus canoniques
  de la \figref{infer-ex-ph} donne le graphe représenté à la \figref{infer-ig},
  contenant les arcs suivants :
  \begin{align*}
    E_+ &= \{\arcf{b}{+}{1}{a}, \arcf{c}{+}{1}{a}, \arcf{a}{+}{1}{a},
      \arcf{b}{+}{1}{b}, \arcf{c}{+}{1}{c}\} \\
    E_- &= \{\arcf{a}{-}{2}{b}\} \qquad\qquad\qquad\qquad\qquad
    E_\uns = \emptyset
  \end{align*}
  Ce graphe des interactions est proche de celui qui avait été proposé
  \vfigref{thomas}(gauche) bien qu'il ne soit pas équivalent,
  car chaque composant comporte une auto-action positive.
  Les auto-actions sur $b$ et $c$ sont la conséquence d'une stabilité globale
  sur plusieurs sous-états : en effet, $c$ n'évolue jamais,
  et $b$ n'évolue pas non plus lorsque $a_2$ n'est pas actif.
  L'auto-action sur $a$ est principalement causée par sa nature multi-valuée.
  
  \begin{figure}[ht]
  \centering
  \scalebox{1.2}{
  \begin{tikzpicture}[grn]
    \path[use as bounding box] (-1.3,-0.75) rectangle (3.5,1.5);
    \node[inner sep=0] (a) at (2,0) {a};
    \node[inner sep=0] (b) at (0,0) {b};
    \node[inner sep=0] (c) at (2,1.2) {c};
    \path[->]
      (b) edge[bend right] node[elabel, below=-2pt] {$+1$} (a)
      (c) edge node[elabel, right=-2pt] {$+1$} (a)
      (a) edge[bend right] node[elabel, above=-5pt] {$-2$} (b)
      (b) edge[in=-15+180, out=15+180, loop] node[elabel, left=-2pt] {+1} (b)
      (c) edge[in=15, out=-15, loop] node[elabel, right=-2pt] {+1} (c)
      (a) edge[in=15, out=-15, loop] node[elabel, right=-2pt] {+1} (a);
  \end{tikzpicture}
  }
  \caption{\figlabel{infer-ig}%
    Graphe des interactions inféré depuis les Frappes de Processus de la \figref{infer-ex-ph}.
  }
  \end{figure}
\end{example}

\begin{example}
  Si on remplace la sorte coopérative $bc$ des Frappes de Processus de la \figref{infer-ex-ph}
  par quatre actions
  \[ \PHfrappe{b_0}{a_1}{a_0} ; \PHfrappe{b_1}{a_1}{a_2} ;
     \PHfrappe{c_0}{a_1}{a_0} ; \PHfrappe{c_1}{a_1}{a_2} \]
  on obtient à nouveau le graphe des interactions donné à la \figref{infer-ig}.
\end{example}

\begin{example}
  L'ajout d'une action $\PHfrappe{a_2}{b_0}{b_1}$ aux Frappes de Processus canoniques de la
  \figref{infer-ex-ph} modifie le résultat de l'inférence.
  En effet, dans ce cas deux arcs non-signés vers $b$ sont inférés en lieu et place
  des arcs signés précédents :
  \begin{align*}
    E_+ &= \{\arcf{b}{+}{1}{a}, \arcf{c}{+}{1}{a}, \arcf{a}{+}{1}{a}, \arcf{c}{+}{1}{c}\}\\
    E_- &= \emptyset \qquad\qquad\qquad\qquad
    E_\uns = \{\arcf{a}{\uns}{2}{b}, \arcf{b}{\uns}{1}{b}\}
  \end{align*}
  Cela est dû au fait que les actions $\PHfrappe{a_2}{b_1}{b_0}$ et $\PHfrappe{a_2}{b_0}{b_1}$
  introduisent des oscillations causées uniquement par le processus $a_2$,
  ce qui implique une influence locale à la fois positive et négative,
  et est impossible à représenter au sein d'un modèle de Thomas.
\end{example}








\subsection{Inférence de la paramétrisation}
\seclabel{infer-param}

Une fois obtenu le graphe des interactions inféré selon la méthode proposée à la section
précédente, il est ensuite possible d'inférer une partie des paramètres discrets
propres à un modèle de Thomas, en fonction de la dynamique des Frappes de Processus canoniques
d'origine.
Cette inférence repose à nouveau sur une exploration exhaustive des comportements possibles
du modèle en fonction de l'état des prédécesseurs de chaque composant.
Cependant, cette inférence peut être partielle si le comportement modélisé
ne peut pas être représenté à l'aide d'un modèle de Thomas.
Dans ce cas, il est possible d'inférer une partie seulement des paramètres,
puis d'énumérer toutes les modèles compatibles avec cette paramétrisation partielle,
la dynamique du modèle et certaines contraintes de modélisation sur les paramètres.

\subsection{Inférence des paramètres}
\seclabel{infer-params}

Cette sous-section présente l'inférence des paramètres discrets indépendants
à partir d'un modèle de Frappes de Processus donné.
Ces résultats sont équivalents à ceux présentés par \citeasnoun{PMR10-TCSB},
auxquels nous ajoutons la notion de \emph{Frappes de Processus bien formées
pour l'inférence des paramètres}, définie au \crref{infer-params-ok},
et qui stipule que pour toute régulation de $a$ par $b$,
tous les processus de $\levels{b}{a}$ (\resp $\ulevels{b}{a}$)
possèdent la même influence sur $a$.

\begin{critere}[Frappes de Processus bien formées pour l'inférence des paramètres]
\crlabel{infer-params-ok}
  Des Frappes de Processus canoniques sont \emph{bien formées pour l'inférence des paramètres}
  si et seulement si leur dynamique est unitaire (\crref{unitaire})
  et si le graphe des interactions $(\components; E)$ inféré par \propref{inference-gi}
  vérifie :
  \begin{align*}
    \begin{split}
      \forall a \in \components &, \forall b \in \RRBreg{a},
        \forall N \in \{ \levels{b}{a}, \ulevels{b}{a} \}, \forall i,j \in N, \\
        %\forall (i, j \in \levels{b}{a} \vee i, j \in \ulevels{b}{a}), \\
      & \forall c \in \PHs, ( (b \neq a \wedge c = a) \vee
        (\exists \myupsilon \in \pred(a), c \in \conn(\myupsilon) \wedge b \in \compin(c)), \\
%        (c \in \PHpredec{a} \setminus \components \wedge b \in \PHdirectpredec{c})), \\
      & \qquad \PHfrappe{b_i}{c_k}{c_l}\in\PHa \Leftrightarrow \PHfrappe{b_j}{c_k}{c_l}\in\PHa
    \end{split}
  \end{align*}
\end{critere}

On souhaite dans la suite inférer le paramètre discret $K_{a,\omega}$,
pour un composant $a \in \components$ et un ensemble $\omega \subset \RRBreg{a}$
de ressources donnés.
Cette inférence se base, à l'instar de l'inférence du graphe des interactions, sur une analyse
exhaustive des sous-états des régulateurs de $a$.
Pour chaque sorte $b \in \RRBreg{a}$, on définit un \todo{\emph{contexte}} $C^b_{a,\omega}$
(\eqref{param-context}) qui recense tous les processus qui interagissent avec $a$
dans tous les sous-états représentés par l'ensemble de ressources $\omega$.
Le contexte d'une sorte coopérative $\myupsilon$ régulant $a$ est l'ensemble des processus
focaux correspondant à ces sous-états (\eqref{param-context-coop}).
Enfin, $C_{a,\omega}$ fait référence à l'union de tous ces contextes
(\eqref{param-context-total}).

\begin{align}
\eqlabel{param-context}
  \forall b \in \components,~
  C_{a,\omega}^b & \DEF \begin{cases}
    \levels{b}{a}  & \text{si $b \in \omega$,}\\
    \ulevels{b}{a} & \text{si $b \notin \omega$,}\\
    L_b            & \text{sinon}\\
  \end{cases}
  \\
\eqlabel{param-context-coop}
  \forall \myupsilon \in \pred(a) \cap \cs,
    C_{a,\omega}^\myupsilon & \DEF \{ \pfp_{\sigma}(\myupsilon) \mid
    \sigma \in \bigtimes{b \in \compin(\myupsilon)} C_{a,\omega}^b \} \\
\eqlabel{param-context-total}
  C_{a,\omega} & \DEF \bigcup_{b \in \pred(a)} C^b_{a,\omega}
\end{align}

Pour inférer le paramètre recherché, nous calculons les \emph{processus focaux} de $a$,
qui sont les processus vers lesquels tend le niveau d'expression de $a$ en présence
de certains autres processus (\defref{focals}).
Ainsi, $\focals(a, S, T)$ donne l'ensemble des processus de $a$ accessibles
en partant de n'importe quel processus dans $S$, et à condition de ne jouer que des
actions dont le frappeur est dans $T$.
Cette notion se base sur un graphe recensant tous les bonds que peuvent faire les processus
de $a$ ;
si ce graphe est acyclique, alors l'ensemble des processus focaux est l'ensemble des processus
de $a$ qui ne sont pas frappés ---~et vers lesquels $a$ va avoir tendance à évoluer.

\begin{definition}[$\focals(a,S,T)$]
\deflabel{focals}
  L'ensemble des \emph{processus focaux} de $a \in \components$ depuis $S \subset \PHl_a$
  pour le sous-état $T \subset \Proc$ est donné par :
  \[
    \focals(a, S, T) \DEF
    \begin{cases}
      \{ a_i \in V \mid \nexists (a_i,a_j)\in E\} & \text{si $(V,E)$ est acyclique},\\
      \emptyset & \text{sinon}\\
    \end{cases}
  \]
  où $(V,E)$ est le graphe orienté suivant :
  \begin{align*}
    E & \DEF \{ (a_j; a_k) \in (\PHl_a \times \PHl_a) \mid
      \exists \PHfrappe{b_i}{a_j}{a_k} \in \PHh^{(2)}, b_i \in T \wedge a_j \in S \} \\
    V & \DEF S \cup \{ a_k \in \PHl_a \mid \exists (a_j; a_k) \in E \}
    %\eqlabel{bounce-graph}
  \end{align*}
\end{definition}

Le paramètre $K_{a,\omega}$ détermine les niveaux d'expression vers lesquels tend $a$
en présence du contexte $C_{a,\omega}$.
Cette valeur peut être calculée à l'aide de $\focals$ qui permet justement
de retrouver les processus focaux en présence de certaines ressources.
Ainsi, on peut en conclure que $K_{a,\omega} = \focals(a,C^a_{a,\omega},C_{a,\omega})$
dans tous les cas où cette valeur est un intervalle non vide (\propref{inference-param}).

\begin{proposition}[Inférence des paramètres]
\proplabel{inference-param}
  Soient $\PH = (\PHs, \PHl, \PHh)$ des Frappes de Processus bien formées pour l'inférence des
  paramètres, $\GI = (\components, E)$ le graphe des interactions inféré pour $\PH$
  et $\omega \subset \RRBres{a}$ un ensemble de ressources de $a$.
  Si $\focals(a,C^a_{a,\omega},C_{a,\omega})$ est un intervalle non vide,
  avec $\focals(a,C^a_{a,\omega},C_{a,\omega}) = \segm{a_i}{a_j}$,
  alors $K_{a,\omega} = \segm{i}{j}$.
\end{proposition}



\begin{example}
%\label{ex:infer-param-runningPH-1}
  Si on l'applique aux Frappes de Processus de la \figref{infer-ex-ph},
  la méthode d'inférence des paramètres donnée dans cette section est conclusive sur tous
  les paramètres et donne :
  \begin{align*}
    K_{a, \emptyset} &= \segm{0}{0} &
    K_{b, \emptyset} &= \segm{0}{0} \\
    K_{a, \{a\}} &= \segm{0}{0} &
    K_{b, \{a\}} &= \segm{0}{0} \\
    K_{a, \{c\}} &= \segm{1}{1} &
    K_{b, \{b\}} &= \segm{1}{1} \\
    K_{a, \{b\}} &= \segm{1}{1} &
    K_{b, \{a,b\}} &= \segm{0}{0} \\
    K_{a, \{b,c\}} &= \segm{1}{1} &
    K_{c, \emptyset} &= \segm{0}{0} \\
    K_{a, \{a,b,c\}} &= \segm{2}{2} &
    K_{c, \{c\}} &= \segm{1}{1} \\
    K_{a,\{a,b\}} &= \segm{1}{1} &
    K_{a,\{a,c\}} &= \segm{1}{1}
  \end{align*}
\end{example}

\todo{Exemple partiellement conclusif}

En observant la \propref{inference-param}, on constate que l'inférence
de certains paramètres peut ne pas être possible.
Cela peut être notamment dû à des coopérations mal définies entre les régulateurs d'un composant :
lorsque deux régulateurs frappent un même composant de façon indépendante, leurs actions peuvent
avoir des effets opposés, créant des oscillations dans la dynamique.
Un tel indéterminisme ne peut pas être représenté à l'aide d'un modèle de Thomas étant donné
que dans une configuration de ressources données, un composant possède un unique attracteur,
représenté par le paramètre discret correspondant,
et ne peut donc évoluer que dans une seule direction.
Il est possible de résoudre ces cas non conclusifs
(autrement dit, de supprimer ces comportements oscillants)
en raffinant le modèle à l'aide de suppressions d'actions ou
en s'assurant que les coopérations sont correctement définies à l'aide de sortes coopératives
afin d'éviter des influences opposées depuis des régulateurs concurrents.



\subsection{Énumération des paramétrisations admissibles}
\seclabel{enum-param}

Lors de la construction d'un modèle de Thomas, trouver la paramétrisation compatible avec
le comportement désiré est nécessaire pour obtenir un modèle complet.
Cependant, cette étape possède une complexité inhérente à ce type de formalisme,
car le nombre de paramètres que contient le modèle croît exponentiellement dans
la taille du graphe des interactions
(plus précisément, dans le nombre de régulations vers chaque composant).
La méthode d'inférence des paramètres présentée précédemment permet cependant d'obtenir
certaines informations sur ces paramètres en fonction de la dynamique des Frappes de Processus
canoniques étudiées.
Ces informations permettent donc de restreindre l'espace des paramétrisations possibles,
et donc d'obtenir plus facilement le modèle recherché.

En d'autres termes,
lors de l'inférence d'un modèle de Thomas selon la méthode décrite précédemment,
il arrive que certains paramètres ne puissent pas être inférés.
Le modèle obtenu est alors partiel, et correspond à un ensemble
plus ou moins large de modèles complets.
En énumérant les valeurs possibles de chaque paramètre, il est envisageable de retrouver
l'ensemble des modèles \emph{compatibles} avec ces valeurs.

Nous délimitons tout d'abord la validité d'un paramètre (\crref{params-valide}) afin d'assurer
que toutes les transitions dans le modèle de Thomas résultant
sont permises par la dynamique des Frappes de Processus canoniques étudiées.
Cette propriété est vérifiée en s'assurant,
pour chaque configurations de ressources possibles,
de l'existence d'une frappe faisant bondir
le processus d'une sorte vers le paramètres correspondant.
Ainsi, conjointement avec le fait que les Frappes de Processus étudiées
sont bien formées pour l'inférence des paramètres,
nous assurons que pour toute transition dans le modèle de Thomas inféré,
il existe une transition équivalente dans les Frappes de Processus d'origine.
Nous remarquons par ailleurs que les paramètres inférés à l'aide de la \vpropref{inference-param}
vérifient déjà cette propriété.



\begin{critere}[Validité d'un paramètre]
\crlabel{params-valide}
  Un paramètre $K_{a,\omega}$ est \emph{valide} pour les Frappes de Processus $\PH$
  si et seulement si :
  \begin{align*}
    \forall a_i\in C^a_{a,\omega}, a_i \notin K_{a,\omega} \Longrightarrow
      (& \exists \PHfrappe{c_k}{a_i}{a_j}\in\PHa, c_k \in C^c_{a,\omega} \\
      & \wedge a_i < K_{a,\omega} \Rightarrow j > i \wedge  a_i > K_{a,\omega} \Rightarrow j <i )
  \end{align*}
\end{critere}

Nous utilisons de plus plusieurs contraintes de modélisation \citeaffixed{BernotSemBRN}{tirées de}
afin d'assurer une cohérence des paramètres avec les signes des régulations du graphe
des interactions préalablement inféré.
L'\emph{hypothèse des valeurs extrêmes} (\crref{param-enum-extreme})
stipule que les niveaux extrêmes d'un composant $a$ (c'est-à-dire $0$ et $l_a$)
doivent chacun apparaître dans au moins un paramètre.
L'\emph{hypothèse d'activité} (\crref{param-enum-activity})
stipule en outre que toutes les régulations doivent être fonctionnelles,
c'est-à-dire que pour chaque régulateur d'un composant,
il existe au moins une configuration dans laquelle la présence ou l'absence de ce régulateur
modifie le paramètre considéré.
Enfin, l'\emph{hypothèse de monotonicité} (\crref{param-enum-monotonicity})
stipule qu'ajouter un activateur (\resp inhibiteur) aux ressources d'un composant
ne peut qu'augmenter (\resp diminuer) la valeur du paramètre considéré.
La relation d'ordre $\leqsegm$ entre deux paramètres discrets s'applique à des segments
et est définie à la \vsecref{notations}.

\begin{critere}[Hypothèse des valeurs extrêmes]
\crlabel{param-enum-extreme}
  Soit $\GI = (\components, E)$ un graphe des interactions.
  Une paramétrisation $K$ sur $\GI$ satisfait l'\emph{hypothèse des valeurs extrêmes}
  si et seulement si :
  \begin{align*}
    \forall b \in \components, \RRBreg{b} \neq \emptyset \Longrightarrow
    \exists \omega \subset \RRBreg{b}, 0 \in K_{b,\omega} \wedge
    \exists \omega' \subset \RRBreg{b}, l_b \in K_{b,\omega'}
  \end{align*}
\end{critere}

\begin{critere}[Hypothèse d'activité]
\crlabel{param-enum-activity}
  Soit $\GI = (\components, E)$ un graphe des interactions.
  Une paramétrisation $K$ sur $\GI$ satisfait l'\emph{hypothèse d'activité}
  \begin{align*}
    \forall b \in \components, \forall a \in \RRBreg{b}, \exists \omega \subset \RRBreg{b},
    K_{b,\omega} \neq K_{b,\omega \cup \{ a \}}
  \end{align*}
\end{critere}

\begin{critere}[Hypothèse de monotonicité]
\crlabel{param-enum-monotonicity}
  Soit $\GI = (\components, E)$ un graphe des interactions.
  Une paramétrisation $K$ sur $\GI$ satisfait l'\emph{hypothèse de monotonicité}
  si et seulement si :
  \begin{align*}
    \forall b \in \components,
    \forall A^+ \subset \{ a \in \components \mid \arc{a}{+}{t}{b} \in E_+ \}&,
    \forall A^- \subset \{ a \in \components \mid \arc{a}{-}{t}{b} \in E_- \},\\
    K_{b,\omega \cup A^-} & \leqsegm K_{b,\omega \cup A^+}
  \end{align*}
\end{critere}

\todo{Ajouter exemple avec cas non conclusifs.}

% 
% \begin{example}\label{ex:enum-param-runningPH-1}
% The parametrization inferred in \pref{ex:infer-param-runningPH-1} was partial because $K_{a,\{a,b\}}$ and $K_{a,\{a,c\}}$ could not be inferred.
% It is however possible to enumerate all complete and admissible parametrizations
% compatible with both the inferred parameters, and the properties of this subsection.
% This enumeration gives 9 different parametrizations which correspond to the 3 possible values
% for each of the two parameters that could not be inferred:
% \begin{align*}
%   K_{a,\{a,b\}} &\in \{ \segm{1}{1}, \segm{1}{2}, \segm{2}{2} \} \\
%   K_{a,\{a,c\}} &\in \{ \segm{1}{1}, \segm{1}{2}, \segm{2}{2} \}
% \end{align*}
% We note that for all solutions, $0 \notin K_{a,\{a,b\}} \wedge 0 \notin K_{a,\{a,c\}}$.
% This is due to the monotonicity assumption (\pref{pro:param-enum-monotonicity}) which especially states that:
% \begin{align*}
%   K_{a,\{b\}} \leqsegm K_{a,\{a,b\}} \wedge
%   K_{a,\{c\}} \leqsegm K_{a,\{a,c\}}
% \end{align*}
% 
% Finally, we note that $\segm{1}{1}$ belongs to the possible values for both parameters.
% Therefore this enumeration allows, from the model in \pref{fig:runningPH-1},
% to find the behavior of the model refined with a cooperative sort described in \pref{fig:runningPH-2}.
% \end{example}