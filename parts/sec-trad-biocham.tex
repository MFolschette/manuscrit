% Traduction de modèles booléens Biocham

\section{Traduction depuis la sémantique booléenne Biocham}
\seclabel{trad-biocham}

Nous proposons dans cette section une traduction des modèles Biocham dans la sémantique booléenne
en Frappes de Processus avec actions plurielles.
Biocham \citeaffixed{fages2004modelling}{pour \textit{Biochemical Abstract Machine}, voir}
est un environnement logiciel pour la modélisation et l'analyse de systèmes biochimiques.
Il propose notamment une syntaxe basée sur des règles de réactions pour représenter
un système de réactions biochimiques,
et un simulateur booléen permettant d'exécuter le système.
Ce simulateur possède la particularité d'interpréter le modèle comme étant un modèle booléen,
où les composants peuvent être présents ou absents,
sans prendre en compte les coefficients stœchiométriques ou les paramètres cinétiques
éventuels contenus dans les équations.

Notre traduction s'appuie sur les ressemblances entre le formalisme booléen de Biocham
et les Frappes de Processus avec actions plurielles, comme discutées au début de la
\secref{phm}.
Pour chaque équation de Biocham, cette traduction fit correspondre
un ensemble d'actions qui reproduit toutes les dynamiques (non-déterministes) possibles.
Nous montrons enfin que cette traduction produit effectivement un modèle équivalent,
ce qui montre que les Frappes de Processus sont au moins aussi expressives
que la sémantique booléenne de Biocham.

\myskip

Nous définissons dans la suite la notion de système d'équations biochimiques (\defref{bc})
et sa dynamique dans la sémantique booléenne de Biocham (\defref{bc-sem}).
Une équation biochimique est un triplet de la forme $X \xrightarrow{Y} Z$
où $X$, $Y$ et $Z$ sont des ensembles de composants ayant respectivement le rôle
de réactifs, catalyseurs et produits.
La signification intuitive équation est la suivante :
si, dans un état donné du système, tous les réactifs et tous les catalyseurs sont présents,
alors il est possible de créer tous les produits et de consommer une partie des catalyseurs.
Il est à noter que nous cherchons à garder l'expressivité la plus large de cette sémantique,
qui autorise de jouer une équation même si certains produits (dans $Z$) sont déjà présents
dans l'état considéré.
Par ailleurs, seule une partie des réactifs (dans $X$) est consommée, et ce de façon
non-déterministe, afin de conserver l'ensemble des dynamiques possibles d'un modèle non booléen
(et comportant donc des coefficients stœchiométriques).

% Nous proposons ensuite une traduction vers les Frappes de Processus avec actions plurielles
% (\defref{bc2phm})
% et nous montrons qu'elle possède la même dynamique (\thmref{bisimulationbc2phm}).
% Cette traduction fit correspondre, à chaque équation du système,
% un ensemble d'actions qui reproduit toutes les dynamiques non-déterministes possibles.

\begin{definition}[Système d'équations biochimiques]
\deflabel{bc}
  Un \emph{système d'équations biochimiques} tel que décrit par Biocham
  est un ensemble de réactions :
  \[\BCe = \{ X \xrightarrow{Y} Z \mid X \cap Y = Y \cap Z = X \cap Z = \emptyset \}\]
%   est un couple $(\BCc; \BCe)$ où :
%   \begin{itemize}
%     \item $\BCc$ est l'ensemble de composants booléens,
%     \item $\BCe = \{ X \xrightarrow{Y} Z \mid X, Y, Z \subset \BCc \wedge
%       X \cap Y = Y \cap Z = \emptyset \}$ est l'ensemble des équations biochimiques.
%   \end{itemize}
  Par ailleurs, on note $\BCc[\BCe]$
  l'ensemble de tous les composants mentionnés dans $\BCe$ :
  \[\BCc = \bigcup_{X \xrightarrow{Y} Z \in \BCe} X \cup Y \cup Z\]
  
  Pour toute équation biochimique $X \xrightarrow{Y} Z \in \BCe$,
  les éléments de $X$ sont appelés les \emph{réactifs}, ceux de $Y$ sont les \emph{catalyseurs}
  et ceux de $Z$ sont les \emph{produits}.
  
  Un \emph{état} d'un système d'équations biochimiques est un ensemble $S \subset \BCc$.
\end{definition}

\begin{remark}
\label{bc-nondeterminisme}
  La syntaxe véritable de Biocham permet d'intégrer des catalyseurs implicites,
  en relâchant la contrainte $X \cap Z = \emptyset$ ;
  autrement dit, il est possible d'avoir $X \cap Z \neq \emptyset$.
  Dans ce cas, les éléments de $X \cap Z$ sont aussi des catalyseurs,
  car ils conditionnent le jeu de la réaction mais ne sont pas modifiés par celle-ci.
  On peut réécrire de telles équations de la forme suivante,
  tout en assurant la même dynamique :
  $(X \setminus Z) \xrightarrow{Y \cup (X \cap Z)} (Z \setminus X)$.
  Par ailleurs, il est naturellement possible de représenter une réaction d'équilibre
  $X \overset{Y}{\longleftrightarrow} Z$ par deux réactions biochimiques
  $X \xrightarrow{Y} Z$ et $Z \xrightarrow{Y} X$.
\end{remark}

\begin{definition}[Sémantique booléenne d'un système d'équations biochimiques]
\deflabel{bc-sem}
  Soit $\BCe$ un système d'équations biochimiques
  et $S \subset \BCc$ un état de ce système.
  On note : $S \BCtrans S'$ si et seulement si :
  \[S \neq S' \wedge
    \exists X \xrightarrow{Y} Z \in \BCe,
    X \cup Y \subset S \wedge
    \exists X' \subset X, S' = (S \setminus X') \cup Z\]
%   Soient $\BCe$ un système d'équations biochimiques
%   et $S \subset \BCc$ un état de ce système.
%   Une équation $e = X \xrightarrow{Y} Z \in \BCe$ est dite \emph{jouable} dans $S$
%   si et seulement si $X \cup Y \subset S$.
%   On note alors $S \play e$ l'état résultant du jeu de $e$ dans $S$, défini par :
%   $S \play e = (S \setminus X) \cup Z$.
%   On notre par ailleurs : $S \BCtrans (S \play e)$.
\end{definition}

\begin{remark}
  Comme expliqué plus haut,
  cette définition prend compte de la sémantique non-déterministe de Biocham pour ce qui est
  de la consommation des réactifs.
  Il est possible de rendre cette définition déterministe (du point de vue de chaque réaction)
  en imposant $X' = X$ pour toutes les transitions.
  De même, cette définition ne prend pas en compte la présence éventuelle de produits
  dans le milieu avant de jouer une équation ;
  autrement dit, certains des produits d'une équation biochimique
  peuvent se trouver dans le milieu au moment où elle est jouée.
\end{remark}

La \defref{bc2phm} propose une traduction des systèmes d'équations biochimiques
en Frappes de Processus avec actions plurielles.
Le modèle obtenu comporte autant de sortes booléennes que de composants mentionnés ;
autrement dit, pour chaque composant $a$ mentionné, il existe une sorte avec deux processus
$a_0$ et $a_1$ dans le modèle obtenu.
De plus, pour chaque équation biochimique $X \xrightarrow{Y} Z$,
un ensemble d'actions est créé, chaque action étant de la forme :
$\PHfrappemult{(X_1 \cup Y_1 \cup Z'_0 \cup (Z \setminus Z')_1)}{(X'_0 \cup Z'_1)}$,
où $X$ et $Y$ sont les réactifs et catalyseurs nécessaires à l'initiation de la réaction,
$Z'$ représente les produits qui seront effectivement créés
(et $Z \setminus Z'$ représente ceux qui sont déjà présents) et
$X'$ représente le sous-ensemble des réactifs qui seront consommés
(défini de façon totalement non-déterministe).
Enfin, le \thmref{bisimulationbc2phm} stipule que le modèle obtenu
est fortement bisimilaire ---~et possède donc strictement la même dynamique.

\begin{definition}[Frappes de Processus équivalentes ($\bctophmsymbol$)]
\deflabel{bc2phm}
  Soit $\BCe$ un système d'équations biochimiques.
  On note $\bctophm = (\PHs, \PHl; \PHh)$ les Frappes de Processus avec actions plurielles
  équivalentes, définies par :
  \begin{itemize}
    \item $\PHs = \BCc$,
    \item $\PHl = \bigtimes{a \in \PHs} \PHl_a$, où $\forall a \in \PHs, \PHl_a = \{ a_0, a_1 \}$,
%    \item $\PHh = \{ \PHfrappemult{(X_1 \cup Y_1 \cup Z_0)}{(X_0 \cup Z_1)} \mid
    \item $\PHh = \{
      \PHfrappemult{(X_1 \cup Y_1 \cup Z'_0 \cup (Z \setminus Z')_1)}{(X'_0 \cup Z'_1)} \mid
      X \xrightarrow{Y} Z \in \BCe \wedge X' \subset X \wedge Z' \subset Z \wedge
      (X' \cup Z') \neq \emptyset \}$,
%     
%     \item $\PHh = \{ \PHfrappemult{A}{B} \mid \exists X \xrightarrow{Y} Z \in \BCe,
%       A = \{ a_1 \mid a \in X \cup Y \} \cup \{ a_0 \mid a \in Z \} \wedge
%       B = \{ a_0 \mid a \in X \} \cup \{ a_1 \mid a \in Z \} \}$.
%     
%     \item $\PHh = \{ \PHfrappemults{a_1 \mid a \in X \cup Y}{a_0}
%       \mid A = \{  \} X \cup Y \wedge B =  \}$
%     \item $\PHh = \{ \PHfrappemult{A}{B} \mid A = \{  \} X \cup Y \wedge B =  \}$
  \end{itemize}
  où, pour tout $i \in \{ 0, 1 \}$ et tout $N \subset \BCc$, on note :
  $N_i = \{ a_i \mid a \in N \}$.
%   où, pour toute équation $X \xrightarrow{Y} Z \in \BCe$, on note
%   $N_i = \{ a_i \mid a \in N \}$
%   avec $i \in \{ 0, 1 \}$ et $N \in \{ X, Y, Z \}$.
  
  Enfin, pour tout état $S \subset \BCc$, on note
  $\tophm{S} = S_1 \cup (\BCc \setminus S)_0$
  %\etat{a_1 \mid a \in S} \cup \etat{a_0 \mid a \notin S}$
  l'état correspondant dans $\bctophm$.
\end{definition}

\begin{remark}
  Pour supprimer le non-déterminisme de Biocham,
  tel qu'expliqué à la remarque \vpageref{bc-nondeterminisme},
  il est possible de modifier la \defref{bc2phm}
  afin d'imposer pour chaque action : $X' = X$,
  ou de faire les simplifications correspondantes.
\end{remark}

\begin{theorem}[$\PH \approx \bctophm$]
\thmlabel{bisimulationbc2phm}
  Soit $\BCe$ un système d'équations biochimiques.
  On a :
  \[\forall S, S' \subset \BCc, S \BCtrans S' \Longleftrightarrow
    \tophm{S} \trans{\bctophm} \tophm{S'} \enspace.\]
\end{theorem}

\begin{proof}
  Posons : $\bctophm = (\PHs, \PHl; \PHh)$.
  Soient $S, S' \subset \BCc$.
  
  ($\Rightarrow$) Supposons que $S \BCtrans S'$.
    D'après la \defref{bc-sem},
    cela signifie, outre $S \neq S'$,
    qu'il existe une équation $X \xrightarrow{Y} Z \in \BCe$
    telle que $X \cup Y \subset S$,
    et qu'il existe un ensemble $X' \subset X$ tel que $S' = (S \setminus X') \cup Z$.
    Posons $Z' = Z \cap S$
    et $h = \PHfrappemult{(X_1 \cup Y_1 \cup Z'_0 \cup (Z \setminus Z')_1)}{(X'_0 \cup Z'_1)}$.
    Comme $S \neq S'$, on note que $X' \neq \emptyset \vee Z' \neq \emptyset$.
    D'après la \defref{bc2phm}, on a donc : $h \in \PHh$.
    Par ailleurs, on a alors : $X_1 \cup Y_1 \subseteq \tophm{S}$,
    et par définition de $Z'$, $Z'_0 \subseteq \tophm{S}$
    et $(Z \setminus Z')_1 \subseteq \tophm{S}$,
    donc $h$ est jouable dans $\tophm{S}$.
%     Enfin, $\tophm{S} \play h = \tophm{S} \recouvre (X'_0 \cup Z'_1) =
%       \tophm{(S \setminus X') \cup Z'}$.
    Enfin, $\tophm{S'} = \tophm{(S \setminus X') \cup Z} = \tophm{(S \setminus X') \cup Z'} =
      \tophm{S \setminus X'} \recouvre Z'_1 = (\tophm{S} \recouvre Z'_1) \recouvre X'_0 =
      \tophm{S} \recouvre (Z'_1 \cup X'_0)$
    car $X$, $Y$ et $Z$ sont disjoints, et $X'$ et $Z'$ aussi.
  
  ($\Leftarrow$) Supposons que $\tophm{S} \trans{\bctophm} \tophm{S'}$.
    Cela signifie qu'il existe une action $h \in \PHh$
    telle que $\tophm{S'} = \tophm{S} \play h$.
    Par ailleurs, l'existence d'une telle action implique, d'après la \defref{bc2phm},
    l'existence d'une équation $X \xrightarrow{Y} Z \in \BCe$ telle que :
    $h = \PHfrappemult{(X_1 \cup Y_1 \cup Z'_0 \cup (Z \setminus Z')_1)}{(X'_0 \cup Z'_1)}$,
    avec $X' \subset X$, $Z' \subset Z$ et $(X' \cup Z') \neq \emptyset$.
    Comme $h$ est jouable dans $\tophm{S}$, cela signifie notamment que
    $X_1 \subseteq \tophm{S}$ et $Y_1 \subseteq \tophm{S}$
    car $X$, $Y$ et $Z$ sont disjoints ; % (et $Z'$ et $(Z \setminus Z')$ aussi).
    autrement dit : $X \subset S$ et $Y \subset S$.
    Enfin, avec le même raisonnement que ci-dessus, on obtient :
    $\tophm{S'} = \tophm{S} \recouvre (Z'_1 \cup X'_0) = \tophm{(S \setminus X') \cup Z}$,
    d'où : $S' = (S \setminus X') \cup Z$.
\end{proof}
