% Traduction de modèles booléens Biocham

\section{Traduction de modèles booléens Biocham}

Nous proposons dans cette section une traduction des modèles Biocham dans la sémantique booléenne
en Frappes de Processus avec actions plurielles.

\myskip

Biocham (\textit{Biochemical Abstract Machine})
est un environnement logiciel pour la modélisation et l'analyse de systèmes biochimiques.
Il propose notamment une syntaxe basée sur des règles de réactions pour représenter
un système de réactions biochimiques,
et un simulateur booléen permettant d'exécuter le système.
Ce simulateur possède la particularité d'interpréter le modèle comme étant un modèle booléen,
où les composants peuvent être présents ou absents,
sans prendre en compte les coefficients stœchiométriques ou les paramètres cinétiques
éventuels contenus dans les équations.

Nous définissons dans la suite un système d'équations biochimiques
tel que représenté dans la sémantique booléenne de Biocham,
et nous proposons une traduction vers les Frappes de Processus avec actions plurielles,
dont nous montrons qu'elle possède la même dynamique.

\todo{Ajouter refs + citation Biocham}

\begin{definition}[Système d'équations biochimiques]
  Un \emph{système d'équations biochimiques} tel que décrit par Biocham
  est un ensemble de réactions :
  \[\BCe = \{ X \xrightarrow{Y} Z \mid X \cap Y = Y \cap Z = X \cap Z = \emptyset \}\]
%   est un couple $(\BCc; \BCe)$ où :
%   \begin{itemize}
%     \item $\BCc$ est l'ensemble de composants booléens,
%     \item $\BCe = \{ X \xrightarrow{Y} Z \mid X, Y, Z \subset \BCc \wedge
%       X \cap Y = Y \cap Z = \emptyset \}$ est l'ensemble des équations biochimiques.
%   \end{itemize}
  Par ailleurs, on note $\BCc[\BCe]$
  l'ensemble de tous les composants mentionnés dans $\BCe$ :
  \[\BCc = \bigcup_{X \xrightarrow{Y} Z \in \BCe} X \cup Y \cup Z\]
  
  Pour toute équation biochimique $X \xrightarrow{Y} Z \in \BCe$,
  les éléments de $X$ sont appelés les \emph{réactifs}, ceux de $Y$ sont les \emph{catalyseurs}
  et ceux de $Z$ sont les \emph{produits}.
  
  Un \emph{état} d'un système d'équations biochimiques est un ensemble $S \subset \BCc$.
\end{definition}

\begin{remark}
  La syntaxe véritable de Biocham permet d'intégrer des catalyseurs implicites,
  autrement dit d'avoir des équations $X \xrightarrow{Y} Z \in \BCe$
  telles quel $X \cap Z \neq \emptyset$.
  Cependant, on peut réécrire de telles équations de la forme suivante,
  tout en assurant la même dynamique :
  $(X \setminus Z) \xrightarrow{Y \cup (X \cap Z)} (Z \setminus X)$.
\end{remark}

\begin{remark}
  Il est naturellement possible de représenter une réaction d'équilibre
  $X \overset{Y}{\longleftrightarrow} Z$ par deux réactions biochimiques
  $X \xrightarrow{Y} Z$ et $Z \xrightarrow{Y} X$.
\end{remark}

\begin{definition}[Sémantique booléenne d'un système d'équations biochimiques]
\deflabel{bc-sem}
  Soit $\BCe$ un système d'équations biochimiques
  et $S \subset \BCc$ un état de ce système.
  On note : $S \BCtrans S'$ si et seulement si :
  \[s \neq S' \wedge
    \exists X \xrightarrow{Y} Z \in \BCe,
    X \cup Y \subset S \wedge
    \exists X' \subset X, S' = (S \setminus X') \cup Z\]
%   Soient $\BCe$ un système d'équations biochimiques
%   et $S \subset \BCc$ un état de ce système.
%   Une équation $e = X \xrightarrow{Y} Z \in \BCe$ est dite \emph{jouable} dans $S$
%   si et seulement si $X \cup Y \subset S$.
%   On note alors $S \play e$ l'état résultant du jeu de $e$ dans $S$, défini par :
%   $S \play e = (S \setminus X) \cup Z$.
%   On notre par ailleurs : $S \BCtrans (S \play e)$.
\end{definition}

\begin{remark}
  Cette définition prend compte de la sémantique indéterministe de Biocham pour ce qui est
  de la consommation des réactifs.
  Il est possible de rendre cette définition déterministe (du point de vue d'une réaction)
  en imposant $X' = X$ pour toutes les transitions.
  De même, cette définition ne prend pas en compte la présence éventuelle de produits
  dans le milieu avant de jouer une équation ;
  autrement dit, certains ou tous les produits d'une équation biochimique
  peuvent se trouver dans le milieu au moment où elle est jouée.
\end{remark}

\begin{definition}[Frappes de Processus équivalentes ($\bctophmsymbol$)]
\deflabel{bc2phm}
  Soit $\BCe$ un système d'équations biochimiques.
  On note $\bctophm = (\PHs, \PHl; \PHh)$ les Frappes de Processus avec actions plurielles
  équivalentes, définies par :
  \begin{itemize}
    \item $\PHs = \BCc$,
    \item $\PHl = \bigtimes{a \in \PHs} \PHl_a$, où $\forall a \in \PHs, \PHl_a = \{ a_0, a_1 \}$,
%    \item $\PHh = \{ \PHfrappemult{(X_1 \cup Y_1 \cup Z_0)}{(X_0 \cup Z_1)} \mid
    \item $\PHh = \{
      \PHfrappemult{(X_1 \cup Y_1 \cup Z'_0 \cup (Z \setminus Z')_1)}{(X'_0 \cup Z'_1)} \mid
      X \xrightarrow{Y} Z \in \BCe \wedge X' \subset X \wedge Z' \subset Z \wedge
      (X' \cup Z') \neq \emptyset \}$,
%     
%     \item $\PHh = \{ \PHfrappemult{A}{B} \mid \exists X \xrightarrow{Y} Z \in \BCe,
%       A = \{ a_1 \mid a \in X \cup Y \} \cup \{ a_0 \mid a \in Z \} \wedge
%       B = \{ a_0 \mid a \in X \} \cup \{ a_1 \mid a \in Z \} \}$.
%     
%     \item $\PHh = \{ \PHfrappemults{a_1 \mid a \in X \cup Y}{a_0}
%       \mid A = \{  \} X \cup Y \wedge B =  \}$
%     \item $\PHh = \{ \PHfrappemult{A}{B} \mid A = \{  \} X \cup Y \wedge B =  \}$
  \end{itemize}
  où, pour tout $i \in \{ 0, 1 \}$ et tout $N \subset \BCc$, on note :
  $N_i = \{ a_i \mid a \in N \}$.
%   où, pour toute équation $X \xrightarrow{Y} Z \in \BCe$, on note
%   $N_i = \{ a_i \mid a \in N \}$
%   avec $i \in \{ 0, 1 \}$ et $N \in \{ X, Y, Z \}$.
  
  Enfin, pour tout état $S \subset \BCc$, on note
  $\tophm{S} = S_1 \cup (\BCc \setminus S)_0$
  %\etat{a_1 \mid a \in S} \cup \etat{a_0 \mid a \notin S}$
  l'état correspondant dans $\bctophm$.
\end{definition}

\begin{theorem}[$\PH \approx \bctophm$]
\thmlabel{bisimulationbc2phm}
  Soit $\BCe$ un système d'équations biochimiques.
  On a :
  \[\forall S, S' \subset \BCc, S \BCtrans S' \Longleftrightarrow
    \tophm{S} \trans{\bctophm} \tophm{S'} \enspace.\]
\end{theorem}

\begin{proof}
  Posons : $\bctophm = (\PHs, \PHl; \PHh)$.
  Soient $S, S' \subset \BCc$.
  
  ($\Rightarrow$) Supposons que $S \BCtrans S'$.
    D'après la \defref{bc-sem},
    cela signifie, outre $S \neq S'$,
    qu'il existe une équation $X \xrightarrow{Y} Z \in \BCe$
    telle que $X \cup Y \subset S$,
    et qu'il existe un ensemble $X' \subset X$ tel que $S' = (S \setminus X') \cup Z$.
    Posons $Z' = Z \cap S$
    et $h = \PHfrappemult{(X_1 \cup Y_1 \cup Z'_0 \cup (Z \setminus Z')_1)}{(X'_0 \cup Z'_1)}$.
    Comme $S \neq S'$, on note que $X' \neq \emptyset \vee Z' \neq \emptyset$.
    D'après la \defref{bc2phm}, on a donc : $h \in \PHh$.
    Par ailleurs, on a alors : $X_1 \cup Y_1 \subseteq \tophm{S}$,
    et par définition de $Z'$, $Z'_0 \subseteq \tophm{S}$
    et $(Z \setminus Z')_1 \subseteq \tophm{S}$,
    donc $h$ est jouable dans $\tophm{S}$.
%     Enfin, $\tophm{S} \play h = \tophm{S} \recouvre (X'_0 \cup Z'_1) =
%       \tophm{(S \setminus X') \cup Z'}$.
    Enfin, $\tophm{S'} = \tophm{(S \setminus X') \cup Z} = \tophm{(S \setminus X') \cup Z'} =
      \tophm{S \setminus X'} \recouvre Z'_1 = (\tophm{S} \recouvre Z'_1) \recouvre X'_0 =
      \tophm{S} \recouvre (Z'_1 \cup X'_0)$
    car $X' \cup Z' = \emptyset$.
  
  ($\Leftarrow$) Supposons que $\tophm{S} \trans{\bctophm} \tophm{S'}$.
    Cela signifie qu'il existe une action $h = \PHfrappemult{A}{B} \in \PHh$
    telle que $\tophm{S'} = \tophm{S} \play h$.
    Par ailleurs, l'existence d'une telle action implique, d'après la \defref{bc2phm},
    l'existence d'une équation $X \xrightarrow{Y} Z \in \BCe$ telle que :
    $h = \PHfrappemult{(X_1 \cup Y_1 \cup Z'_0 \cup (Z \setminus Z')_1)}{(X'_0 \cup Z'_1)}$,
    avec $X' \subset X$, $Z' \subset Z$ et $(X' \cup Z') \neq \emptyset$.
    
    \TODO
  
\end{proof}
