\section{Notations}
\seclabel{notations}

\begin{description}
  \item[Nombres réels]
    On note $\sR$ l'ensemble des nombres réels.
    %et $\sR^* = \sR \setminus \{ 0 \}$ l'ensemble des nombres réels strictement positifs.
    Si $i, j \in \sR$, on note $\interv{i}{j} = \{ x \in \sR \mid i \leq x \leq j \}$
    l'ensemble des nombres réels entre $i$ et $j$ compris.
  
  \item[Entiers naturels]
    On note $\sN$ l'ensemble des entiers naturels,
    $\sN^* = \sN \setminus \{ 0 \}$ l'ensemble des entiers naturels strictement positifs,
    et $\sNN = \sN \setminus \{ 0, 1 \}$ l'ensemble des entiers naturels
    supérieurs ou égaux à $2$.
    Si $i, j \in \sN$, $i < j$, on note $\segm{i}{j} = \{ i, i+1, \ldots, j-1, j \}$
    l'ensemble des entiers naturels entre $i$ et $j$ compris.
    
    Pour tous entiers $i, j, k \in \sN$, on note :
    $k < \segm{i}{j} \EQDEF k < i$ et $k > \segm{i}{j} \EQDEF k > j$.
    De plus, si $i_1, i_2, j_1, j_2 \in \sN$, on note :
    \begin{align*}
      \segm{i_1}{j_1} \leqsegm \segm{i_2}{j_2} &\EQDEF (i_1 \leq i_2 \wedge j_1 \leq j_2) \\
      \text{et :} \qquad \segm{i_1}{j_1} \ltsegm \segm{i_2}{j_2}
        &\EQDEF (i_1<i_2 \wedge j_1\leq j_2) \vee (i_1\leq i_2\wedge j_1 < j_2)
    \end{align*}
  
  \item[Entiers relatifs]
    On note $\mathbb{Z}$ l'ensemble des entiers relatifs.
    
    Par ailleurs, la fonction $\f{signe}$ est définie sur les entiers relatifs comme suit :
    \begin{align*}
      \f{signe} : \mathbb{Z} &\rightarrow \{ +, -, \varnothing \} \\
      n & \mapsto \begin{cases}
        + & \text{ si $x > 0$} \\
        - & \text{ si $x < 0$} \\
        \varnothing & \text{ if $x = 0$}
      \end{cases}
    \end{align*}
  
  \item[Séquences]
    Si $n \in \sN$, on note $e_1 \cons \ldots \cons e_n$ la séquence finie formée
    des éléments $e_1, \ldots, e_n$ si $n \geq 1$,
    et on note $\emptyseq$ la séquence vide (si $n = 0$).
    
    Pour toute séquence finie $E = e_1 \cons \ldots \cons e_n$,
    on note $\card{E} = n$ la longueur de cette séquence,
    et $\indexes{E} = \segm{1}{\card{E}}$ l'ensemble des indices de cette séquence.
    Pour tout $i \in \indexes{E}$, on note $A_i = e_i$ le $i$\textsuperscript{e} élément de $E$,
    et pour tout $i,j \in \indexes{E}$, on note $E_{i..j} = e_i \cons \ldots \cons e_j$
    la sous-séquence formée des éléments $i$ à $j$ de $E$ ;
    naturellement, $E_{i..j} = \emptyseq$ si $i > j$.
    
    On note de plus : $e \in E \Leftrightarrow \exists i \in \indexes{E}, e = E_i$
  
  \item[Ensembles]
    Le cardinal d'un ensemble $A$ est noté $\card{A}$
    et son ensemble des parties est noté $\powerset(A)$.
    
    Si $A$ et $B$ sont deux ensembles, on note
    $A \cup B$ leur union, $A \cap B$ leur intersection et $A \times B$ leur produit cartésien.
    
    Si $n \in \sN$, et $\{ A_i \}_{i \in \segm{1}{n}}$ est un ensemble d'ensembles, on note
    $\bigcup_{i \in \segm{1}{n}} A_i = A_1 \cup A_2 \cup \ldots \cup A_n$ leur union,
    $\bigcap_{i \in \segm{1}{n}} A_i = A_1 \cap A_2 \cap \ldots \cap A_n$ leur intersection et
    $\bigtimes{i \in \segm{1}{n}} A_i = A_1 \times A_2 \times \ldots \times A_n$
    leur produit cartésien.
    Cette définition peut naturellement être étendue à une séquence d'ensembles.
    De plus, par convention :
    $\bigcup_{\emptyset} = \bigcap_{\emptyset} = \bigtimes{\emptyset} =
      \bigcup_{\emptyseq} = \bigcap_{\emptyseq} = \bigtimes{\emptyseq} = \emptyset$.
    
    Si $E = e_1 \cons \ldots \cons e_n$ est une séquence,
    on note $\toset{E} = \{ e_1, \ldots, e_n \}$ l'ensemble correspondant.
    De même, si $T = (t_1, \ldots, t_n)$ est un $n$\nbd uplet,
    on définit : $\toset{T} = \{ t_1, \ldots, t_n \}$.
  
  \item[Fonctions et plus petit point fixe]
    Si $A$ et $B$ sont deux ensembles,
    on note $f : A \rightarrow B$
    si $f$ est une fonction qui associe chaque élément de $A$ à un élément de $B$.
    
    On note de plus $\lfp{x_0}{x}{x'}$ le plus petit point fixe plus grand que $x_0$
    de la fonction $x \mapsto x'$, s'il existe.
  
  \item[Recouvrement]
    L'opérateur $\recouvre$, qui désigne le recouvrement d'un état par un autre,
    est donné à la \vdefref{recouvrement}.
    Il est ensuite étendu
    aux contextes à la \vdefref{ctxrecouvrement}
    ainsi qu'aux ensembles de processus à la \vdefref{recouvrementps}.

\end{description}

\todoplustard{Ajouter des références aux autres notations définies dans le corps de la thèse}
