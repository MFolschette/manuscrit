\section{Notations}
\seclabel{notations}

On note $\sN$ l'ensemble des entiers naturels.
Si $i, j \in \sN$, $i < j$, on note $\segm{i}{j} = \{ i, i+1, \ldots, j-1, j \}$
l'ensemble des entiers naturels entre $i$ et $j$ compris.
On note $\emptyseq$ la séquence vide.

L'ensemble des parties d'un ensemble $A$ est noté : $\powerset(A)$.
Si $A$ et $B$ sont deux ensembles, on note
$A \cup B$ leur union, $A \cap B$ leur intersection et $A \times B$ leur produit cartésien.
Si $n \in \sN$, et $(A_i)_{i \in \segm{1}{n}}$ est une séquence d'ensembles, on note
$\bigcup_{i \in \segm{1}{n}} A_i = A_1 \cup A_2 \cup \ldots \cup A_n$ leur union,
$\bigcap_{i \in \segm{1}{n}} A_i = A_1 \cap A_2 \cap \ldots \cap A_n$ leur intersection,
$\bigtimes{i \in \segm{1}{n}} A_i = A_1 \times A_2 \times \ldots \times A_n$ leur produit cartésien.
De plus, par convention :
$\bigcup_{\emptyset} = \bigcap_{\emptyset} = \bigtimes{\emptyset} = \emptyset$.

\todo{indices}

\todo{$\toset{ps}$}
