
\subsection{Définition}
\seclabel{phcanonique-def}

Nous donnons dans cette section la définition des Frappes de Processus canoniques,
qui sont un sous-ensemble des Frappes de Processus avec classes de priorités.
Un modèle de Frappes de Processus est dit \emph{canonique} s'il comporte 2 classes de priorités
et qu'il respecte un certain nombre de critères concernant l'utilisation des actions prioritaires.
Il faut notamment que celles-ci soient uniquement utilisées pour la mise à jour des sortes
coopératives, et que cette mise à jour soit correctement formée (pas d'action manquantes ou en trop).
Cette forme particulière des Frappes de Processus assure un certain nombre de bonnes propriétés
qui seront explicitées par la suite et qui permettent l'utilisation des méthodes d'analyse statique
développées à la section \todo{suivante}.
Dans la suite de cette section, nous travaillons sur des modèles de Frappes de Processus avec $k$
classes de priorités, pour $k \in \sN^*$.



Nous définissons pour commencer la \emph{réduction} d'un modèle donné de Frappes de Processus
comme le modèle équivalent dont on a retiré toutes les actions qui ne sont pas de priorité 1
(\defref{reduction}).
Cette définition permet dans la suite de considérer le comportement des actions les plus prioritaires
d'un modèle afin d'en contraindre la forme.

\begin{definition}[Réduction ($\reductionsymbol{\PH}$)]
\deflabel{reduction}
  Si $\PH = (\PHs; \PHl; \PHh^{\langle k \rangle})$ sont des Frappes de Processus avec $k$
  classes de priorités, on note $\reduction[\PH] = (\PHs; \PHl; \PHh^{(1)})$
  la \emph{réduction} de $\PH$.
  $\reduction$ est un modèle de Frappes de Processus standard
  (ou encore : avec 1 classe de priorité).
  Si $s \in \PHl$, on note de plus : $\reduction[\Sce](s)$ l'ensemble des scénarios
  depuis l'état $s$ dans les Frappes de Processus $\reduction$.
\end{definition}

\begin{example}
  La \figref{ph-livelock} donne un exemple de Frappes de Processus avec 2 classes de priorités avec :
  \begin{align*}
    \PHs &= \{ a, b, c, ab \} \enspace, \\
    \PHl_a &= \{ a_0, a_1 \} \enspace, & \PHl_b &= \{ b_0, b_1 \} \enspace, \\
    \PHl_c &= \{ c_0, c_1 \} \enspace, & \PHl_{ab} &= \{ ab_{00}, ab_{01}, ab_{10}, ab_{11} \}
    \enspace.
  \end{align*}
  On note que :
  $\{ \PHhit{ab_{11}}{c_0}{c_1}, \PHhit{a_1}{a_1}{a_0}, \PHhit{a_0}{b_0}{b_1} \} \subseteq \PHh^{(2)}$.

\begin{figure}[tb]
  \centering
  \scalebox{1.2}{
  \begin{tikzpicture}
    \TSort{(0,0)}{a}{2}{l}
    \TSort{(0,4)}{b}{2}{l}
    \TSort{(7,2.5)}{c}{2}{r}

    \TSetTick{ab}{0}{00}
    \TSetTick{ab}{1}{01}
    \TSetTick{ab}{2}{10}
    \TSetTick{ab}{3}{11}
    \TSort{(2,2.5)}{ab}{4}{t}

    \THit{a_0}{prio}{ab_3}{.south}{ab_1}
    \THit{a_0}{prio}{ab_2}{.south}{ab_0}
    \THit{a_1}{prio}{ab_1}{.south}{ab_3}
    \THit{a_1}{prio}{ab_0}{.south}{ab_2}

    \THit{b_0}{prio}{ab_3}{.north}{ab_2}
    \THit{b_0}{prio}{ab_1}{.north}{ab_0}
    \THit{b_1}{prio}{ab_2}{.north}{ab_3}
    \THit{b_1}{prio}{ab_0}{.north}{ab_1}
    
    \THit{a_1}{selfhit}{a_1}{.west}{a_0}
    \THit{b_1}{selfhit}{b_1}{.west}{b_0}
    \THit{a_0.north}{bend left}{b_0}{.west}{b_1}
    \THit{b_0.south}{bend right=60}{a_0}{.west}{a_1}

    \THit{ab_3}{}{c_0}{.west}{c_1}

  \path[bounce, bend right=55]
      \TBounce{ab_0}{}{ab_2}{.west}
      \TBounce{ab_1}{}{ab_3}{.west}
  ;
  \path[bounce, bend left=20]
      \TBounce{ab_3}{}{ab_1}{.south east}
      \TBounce{ab_2}{}{ab_0}{.south east}
  ;
    \path[bounce, bend right=20]
      \TBounce{ab_3}{}{ab_2}{.north east}
      \TBounce{ab_1}{}{ab_0}{.north east}
    ;
    \path[bounce, bend left=30]
      \TBounce{ab_2}{}{ab_3}{.west}
      \TBounce{ab_0}{}{ab_1}{.west}
    ;
    \path[bounce, bend right]
      \TBounce{a_1}{}{a_0}{.north west}
      \TBounce{b_1}{}{b_0}{.north west}
    ;
    \path[bounce, bend left]
      \TBounce{a_0}{}{a_1}{.south west}
      \TBounce{b_0}{}{b_1}{.south west}
    ;
    \path[bounce, bend left]
      \TBounce{c_0}{}{c_1}{.south west}
    ;
    \TState{a_1, b_0, ab_2, c_0}
  \end{tikzpicture}
  }
  \caption{
  \figlabel{ph-livelock}
    Un exemple de Frappes de Processus canonique.
    Les actions de priorité 1 sont dessinées en traits doubles tandis que les actions
    de priorité 2 sont représentées avec des traits simples.
    Les processus grisés présentent un exemple d'état de départ :
    $\PHstate{a_1, b_0, c_0, ab_{10}}$.
  }
\end{figure}

\end{example}



\subsection{Restrictions}
\seclabel{restrictions}

Les Frappes de Processus canoniques consistent en une classe particulière de Frappes de Processus
avec 2 classes de priorités, et cette section a pour objectif d'en donner la définition.
Les restrictions permettant de définit des Frappes de Processus canoniques portent sur
les actions les plus prioritaires mettant à jour les sortes coopératives.
Cependant, afin de poser un cadre plus général et permettre une réutilisation de ces définitions
\stodo{ref},
nous considérerons le cas de Frappes de Processus avec $k$ classes de priorités dans les définitions
et critères qui suivent.

Dans la suite, nous appelons \emph{actions primaires} les actions de l'ensemble $\PHh^{(1)}$,
c'est-à-dire les actions les plus prioritaires.
Ces actions auront pour unique but de mettre à jour des sortes coopératives, permettant
à celles-ci de modéliser des portes logiques sans décalage temporel \stodo{ref}.
De fait, on peut dans certains cas considérer ces actions comme «~non biologiques~»,
ou «~instantanées~»,
car elles sont présentes principalement pour des raisons de modélisation.
À l'inverse, les actions de l'ensemble $\PHh \setminus \PHh^{(1)}$, qui ne sont pas de priorité 1,
seront appelées \emph{actions secondaires}.
Comme elles permettent de représenter les différentes réactions et régulations intervenant au sein du
système, elles peuvent par conséquent être qualifiées d'actions «~biologiques~» ou «~lentes~».

La définition de la forme canonique des Frappes de Processus début avec la
\crref{tf} qui stipule que la dynamique ne doit pas contenir de séquence infinie
d'actions primaires.
En effet, pour que ces actions effectuent une mise à jour des sortes coopératives sans perturber
le reste du système, il est nécessaire qu'elles ne puissent pas préempter les actions secondaires
indéfiniment.
Sans cela, le modèle serait victime du paradoxe de Zénon,
où une suite infinie d'actions peut être jouée en un temps nul et ainsi
bloquer l'évolution du système.
Il est donc nécessaire de postuler que tous les scénarios d'actions primaires sont finis,
ce qui a par ailleurs pour conséquence que $\reductionsce$ est un ensemble fini.

\begin{condition}[Terminaison finie]
\crlabel{tf}
  La dynamique de $\reduction$ ne contient aucun cycle :
  $\exists N \in \sN, \forall s \in \PHl, \forall \delta \in \reductionsce(s),
    |\delta| \leq N$.
\end{condition}

Étant donnée une sorte $a \in \PHs$ et un état $s \in \PHl$,
on note $\pfp_s(a)$ (\defref{pfp}) l'ensemble des processus de la sorte $a$ qui peuvent être présents
après avoir joué tous les scénarios d'actions de priorité 1 depuis l'état $s$.
Cet ensemble est toujours défini du fait de la \crref{tf}.

\todo{Redéfinir $\pfp$ avec $\compin$ de façon à simplifier et rendre locale la définition ?}

\begin{definition}[$\pfp : \PHl \times \PHs \rightarrow \powerset(\PHproc)$]
\deflabel{pfp}
  Pour tout état $s \in \PHl$ et pour toute sorte $a \in \PHs$,
  \begin{align*}
    \pfp_s(a) = \{ \get{(s\play\delta)}{a} \in \PHl_a &\mid \delta \in \reductionsce(s)
          \wedge\nexists h\in\PHh^{(1)}, (\delta; h) \in \reductionsce(s) \}
  \end{align*}
\end{definition}

Nous définissons dans la suite la notion de \emph{composant bien formé} (\defref{component})
et de \emph{sorte coopérative bien formée} (\defref{cs}).
Un composant bien formé n'est frappé que par des actions secondaires,
car il ne subit que des influences «~biologiques~».
Une sorte coopérative bien formée n'est frappée que par des actions primaires «~non biologiques~»
qui convergent toujours vers le même processus en fonction de l'état des sortes qui l'influencent
(\defref{comp}), afin qu'elle représente chaque configuration de ces sortes par un unique processus.
La \crref{part} stipule alors que l'ensemble des sortes des Frappes de Processus canoniques
($\PHs$) est une partition entre un ensemble de composants bien formés (noté $\components$)
et un ensemble de sortes coopératives bien formées (noté $\cs$).

\begin{definition}[Composant bien formé ($\components$)]
\deflabel{component}
  Une sorte $a \in \PHs$ est un \emph{composant bien formé} si et seulement si :
    \[\forall h \in \PHh, \PHsort(\target{h}) = a \Rightarrow \prio(h) \geq 2 \enspace.\]
\end{definition}

\begin{definition}[Influence ($\compin : \PHs \to \powerset(\components)$)]
\deflabel{comp}
  Pour toute sorte $a \in \PHs$, on définit : $\compin(a) \DEF \conn(a) \cap \components$ où
  $\conn(a) \subset \PHs$ est le plus petit ensemble de sortes satisfaisant :
  \begin{align*}
    a \in \conn(a) & \\
    \forall h \in \PHh^{(1)},
      \sort{\target{h}} \in \conn(a) & \Rightarrow \sort{\hitter{h}} \in \conn(a)
  \end{align*}
\end{definition}

\begin{definition}[Sorte coopérative bien formée ($\cs$)]
\deflabel{cs}
  Une sorte $a \in \PHs$ est une \emph{sorte coopérative bien formée} si et seulement si :
  \begin{enumerate}[(i)]
    \item $\forall h \in \PHh, \sort{\target{h}} = a \Rightarrow \prio(h) = 1$,
    \item \label{csai} $\forall s \in \PHl, \card{\pfp_s(a)} = 1$,
    \item \label{css} $\forall a_i \in \PHl_a, \exists s \in \PHl, a_i \in \pfp_s(a)$,
    \item $\forall \mysigma \in \PHsubl[\PHl]_{\compin(a)}, \forall s, s' \in \PHl,
        (\mysigma \subseteq s \wedge \mysigma \subseteq s') \Rightarrow \pfp_s(a) = \pfp_{s'}(a)$.
  \end{enumerate}
\end{definition}

\begin{condition}[Partition des composants et des sortes coopératives]
\crlabel{part}
  \[\PHs = \components \cup \cs \wedge \components \cap \cs = \emptyset\]
\end{condition}

Il est à noter qu'une sorte $a$ qui n'est frappée par aucune action,
c'est-à-dire telle que $\forall h \in \PHh, \sort{\target{h}} \neq a$,
est en accord à la fois avec la définition de composant bien formé
et avec celle de sorte coopérative bien formée.
Cette sorte peut être arbitrairement et indifféremment affectée à $\components$ ou à $\cs$.
Un tel cas peut se produire pour des composants ayant un rôle d'«~entrée~»,
c'est-à-dire dont le niveau d'expression est uniquement déterminé par l'état initial,
ou par certaines constructions particulières de sortes coopératives créées par la traduction
\stodo{ref}.

Pour toute sorte $a \in \PHs$ et tout état $s \in \PHl$,
étant donné le point (\ref{csai}) de la \defref{cs}, on a toujours :
$\exists a_i \in \PHl_a, \pfp_s(a) = \{ a_i \}$.
On notera donc dans la suite : $\pfp_s(a) = a_i$.
De plus, du fait du point (\ref{css}) de la \defref{cs}, on introduit la notation
$\csState(a_i)$ permettant de caractériser l'ensemble des sous-états représentés par le
processus $a_i$ de toute sorte coopérative bien formée $a$ (\defref{csState}).

\begin{definition}[$\csState : \PHproc \rightarrow \powerset(\PHproc)$]
\label{def:csState}
  Pour tout $a \in \cs$ et $a_i \in \PHl_a$, 
    \[\csState(a_i) \DEF \{ \toset{ps} \mid ps \in \PHsubl[\PHl]_{\compin(a)} \wedge
      \exists s \in \PHl, (ps \subseteq s \wedge \pfp_s(a) = a_i) \}\]
\end{definition}

Dans la suite, on écrira simplement «~composant~» (\resp «~sorte coopérative~»)
en lieu et place de «~composant bien formé~» (\resp «~sorte coopérative bien formée~»).

\begin{example}
  Les Frappes de Processus de la \figref{ph-livelock} comprennent trois composants ($a$, $b$ et $c$)
  et une sorte coopérative ($ab$).
  La sorte coopérative $ab$ modélise la coopération entre $a$ et $b$ car $\compin(ab) = \{ a, b \}$.
\end{example}

Enfin, la \defref{phcanonique} définit la notion de Frappes de Processus canoniques.
La forme particulière de ces Frappes de Processus permettra de développer une méthode efficace
d'analyse statique pour le problème d'atteignabilité défini à la \secref{sa},
grâce notamment aux résultats développés dans la section suivante.

\begin{definition}[Frappes de Processus canoniques]
\deflabel{phcanonique}
  Les \emph{Frappes de Processus canoniques}
  sont des Frappes de Processus avec 2 classes de priorités
  respectant les \allcr.
\end{definition}



\subsection{Conséquences des restrictions}

Dans cette section, nous donnons plusieurs résultats généraux qui peuvent être dérivés des
restrictions de la \secref{restrictions}.
Nous considérons donc dans la suite des Frappes de Processus avec $k$ classes de priorités
$\PH = (\PHs; \PHl; \PHa^{\langle k \rangle})$, avec $k \in \sN^*$,
respectant les \allcr.
\todo{Avec 2 classes de priorités ?}
Ces résultats permettront de construire la méthode d'analyse statique de la \stodo{section suivante}.

Pour tout état $s \in \PHl$, on appelle $\update(s)$ l'état dans lequel tous les composants
ont le même processus actif que dans $s$
mais où sortes coopératives ont été mises à jour en fonction (\defref{update}).
Cet état est unique du fait des propriétés de $\pfp$ données dans la section précédente.
Le \lemref{update} stipule ensuite que depuis tout état $s$, il existe un scénario d'actions primaires
mettant à jour toutes les sortes coopératives de façon à arriver dans $\update(s)$.
%ghtrjy <= Raphaël was here!

\begin{definition}[$\update : \PHl \rightarrow \PHl$]
\deflabel{update}
  Pour tout $s \in \PHl$, on définit :
  \begin{align*}
    \update(s) = s \Cap \{ \pfp_{s}(a) \mid a \in \cs \} \enspace.
  \end{align*}
\end{definition}

\begin{lemma}
\lemlabel{update}
  $\forall s \in \PHl, \exists \delta \in \reductionsce(s), s \PHplay \delta = \update(s)$
\end{lemma}

\begin{demo}[\Lemref{update}]
  Soit $s \in \PHl$ un état.
  Soit $a \in \cs$ une sorte coopérative telle que $\PHget{s}{a} \neq \pfp_s(a)$.
  Étant donnée la définition de $\pfp_s(a)$, il existe un scénario $\delta$ mettant à jour $a$
  dans $s$ tel que :
  $\forall \delta' \in \reductionsce(s \PHplay \delta),
    \PHget{(s \PHplay \delta \PHplay \delta')}{a} = \pfp_s(a)$.
  \todo{À revoir}
\end{demo}

Le \lemref{hcompcomp} stipule que pour un état donné $s \in \PHl$, et pour toute action secondaire
$h = \PHhit{a_i}{b_j}{b_k} \in \PHh$ où $a$ et $b$ sont des composants,
si $\PHget{s}{a} = a_i$ et $\PHget{s}{b} = b_j$, alors
$h$ peut toujours être jouée après une série d'actions primaires,
et ces actions n'empêchent pas son jeu.

De façon complémentaire, le \lemref{hcscomp} énonce le même résultat si $a$ est une sorte coopérative,
sous la condition que $a$ soit déjà à jour dans $s$.

\begin{lemma}
\lemlabel{hcompcomp}
  $\forall s \in \PHl, \forall a,b \in \components, \forall h = \PHhit{a_i}{b_j}{b_k} \in \PHh,$\\
  $(\PHget{s}{a} = a_i \wedge \PHget{s}{b} = b_j) \Rightarrow
    (\exists \delta \in \reductionsce(s),
%    (s \PHplay \delta) \PHtrans (s \PHplay \delta \PHplay h))$
    \Feval{\Fopphp{h}}{s \PHplay \delta})$
\end{lemma}

\begin{demo}[\Lemref{hcompcomp}]
  Soient $s \in \PHl$ un état, $a,b \in \components$ deux composants et
  $h = \PHhit{a_i}{b_j}{b_k} \in \PHh$ une action.
  D'après le \lemref{update}, il existe un scénario $\delta$ tel que :
  $(s \PHplay \delta) = \update(s)$.
  Étant donné que $a$ et $b$ sont des composants,
  $a_i \in (s \PHplay \delta)$ et $b_j \in (s \PHplay \delta)$.
  De plus, par définition de $\update(s)$, aucune action primaire n'est jouable
  dans $(s \PHplay \delta)$.
  $h$ est donc jouable dans $(s \PHplay \delta)$.
\end{demo}

\begin{lemma}
\lemlabel{hcscomp}
  $\forall s \in \PHl, \forall a \in \cs, \forall b \in \components,
    \forall h = \PHhit{a_i}{b_j}{b_k} \in \PHh,$\\
  $(\PHget{s}{a} = a_i \wedge \PHget{s}{b} = b_j \wedge \pfp_s(a) = a_i) \Rightarrow
    (\exists \delta \in \reductionsce(s),
%    (s \PHplay \delta) \PHtrans (s \PHplay \delta \PHplay h))$
    \Feval{\Fopphp{h}}{s \PHplay \delta})$
\end{lemma}

\begin{demo}[\Lemref{hcscomp}]
  Similaire à la preuve du \lemref{hcompcomp} ;
  étant donné que $a_i \in \pfp_s(a)$, on a : $a_i \in (s \PHplay \delta)$.
\end{demo}
