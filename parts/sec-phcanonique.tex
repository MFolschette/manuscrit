
\section{Les Frappes de Processus canoniques}
\seclabel{phcanonique}

Nous donnons dans cette section la définition des Frappes de Processus canoniques,
qui sont un sous-ensemble des Frappes de Processus avec classes de priorités
(\secref{phcanonique-def}).
Un modèle de Frappes de Processus est dit \emph{canonique} s'il comporte 2 classes de priorités
et qu'il respecte un certain nombre de critères concernant l'utilisation des actions prioritaires.
Il faut notamment que celles-ci soient uniquement utilisées pour la mise à jour des sortes
coopératives, et que cette mise à jour soit correctement formée,
c'est-à-dire qu'elle ne comporte pas d'actions en trop ou manquantes.
Un certain nombre de résultats peuvent être déduits de la forme particulière de ces
Frappes de Processus (\secref{phcanonique-resultats}).
La \vfigref{livelock} propose un exemple simple de Frappes de Processus canoniques.

% Cette forme particulière des Frappes de Processus assure un certain nombre de bonnes propriétés
% qui seront explicitées par la suite et qui permettent l'utilisation des méthodes d'analyse statique
% développées à la \secref{as}.
% Il est à noter de plus que les autres sémantiques de Frappes de Processus
% présentées au \chapref{sem}
% peuvent être traduites en Frappes de Priorités canoniques (\secref{phcanonique-equiv}).



\subsection{Critères et définitions}
\seclabel{phcanonique-def}

Nous donnons dans la suite les définitions et critères permettant de caractériser entièrement
les Frappes de Processus canoniques (\defref{phcanonique}).
Cependant, afin de proposer un cadre assez général et permettre une réutilisation des éléments
de cette section,
nous considérerons dans un premier temps le cas de Frappes de Processus avec $k$
classes de priorités dans les définitions et critères qui suivent, pour $k \in \sNN$.
Cela permettra aussi de définir les Frappes de Processus pseudo-canoniques
(\defref{phpseudocanonique}),
qui forment un ensemble plus général ayant des propriétés proches.

\myskip

Nous définissons pour commencer la \emph{réduction} d'un modèle donné de Frappes de Processus
comme le modèle équivalent dont on a retiré toutes les actions qui ne sont pas de priorité 1
(\defref{reduction}).
Cette définition permet dans la suite de considérer le comportement des actions les plus prioritaires
d'un modèle afin d'en contraindre la forme.
Il est à noter cependant que cette réduction provoque une perte d'informations,
et n'est donc pas comparable aux réductions de modèles comme \cite{Naldi09}.

\begin{definition}[Réduction ($\reductionsymbol{\PH}$)]
\deflabel{reduction}
  Si $\PH = (\PHs; \PHl; \PHh^{\langle k \rangle})$ sont des Frappes de Processus avec $k$
  classes de priorités, on note $\reduction[\PH] = (\PHs; \PHl; \PHh^{(1)})$
  la \emph{réduction} de $\PH$.
  $\reduction$ est un modèle de Frappes de Processus standard
  (ou encore : avec 1 classe de priorité).
  Si $s \in \PHl$, on note de plus : $\reduction[\Sce](s)$ l'ensemble des scénarios
  depuis l'état $s$ dans les Frappes de Processus $\reduction$.
\end{definition}

%\subsection{Restrictions}
%\seclabel{restrictions}

Dans la suite, nous appelons \emph{actions primaires} les actions de l'ensemble $\PHh^{(1)}$,
c'est-à-dire les actions les plus prioritaires.
Ces actions auront pour unique but de mettre à jour des sortes coopératives, permettant
à celles-ci de modéliser des portes logiques sans décalage temporel,
comme expliqué \vexpageref{livelock-merge}.
De fait, on peut dans certains cas considérer ces actions comme «~non biologiques~»,
ou «~instantanées~»,
car elles sont présentes principalement pour des raisons de modélisation.
À l'inverse, les actions de l'ensemble $\PHh \setminus \PHh^{(1)}$, qui ne sont pas de priorité 1,
seront appelées \emph{actions secondaires}.
Comme elles permettent de représenter les différentes réactions et régulations intervenan
au sein du
système, elles peuvent par conséquent être qualifiées d'actions «~biologiques~» ou «~lentes~».

\begin{example}
  Nous prendrons comme exemple, tout au long de cette section,
  les Frappes de Processus avec 2 classes de priorités représentées à la \figref{livelock},
  qui sera étudié plus en détail \vexpageref{livelock-merge}
  mais permettra entre-temps d'illustrer certaines définitions.
  Nous constatons de prime abord que les actions primaires de ce modèle
  permettent uniquement de mettre à jour la sorte coopérative $ab$,
  tandis que les actions secondaires frappent les autres composants.
\end{example}


La définition de la forme canonique des Frappes de Processus débute avec le
\crref{tf} qui stipule que la dynamique ne doit pas contenir de séquence infinie
d'actions primaires.
En effet, pour que ces actions effectuent une mise à jour des sortes coopératives sans perturber
le reste du système, il est nécessaire qu'elles ne puissent pas préempter les actions secondaires
indéfiniment.
Sans cela, le modèle pourrait exhiber un comportement dit de Zénon \cite{zhang92zeno},
où une suite infinie d'actions peut être jouée en un temps nul et ainsi
bloquer l'évolution du système.
Il est donc nécessaire de postuler que tous les scénarios d'actions primaires sont finis,
ce qui a par ailleurs pour conséquence que $\reductionsce$ est un ensemble fini.

\begin{critere}[Terminaison finie]
\crlabel{tf}
  La dynamique de $\reduction$ ne contient aucun cycle :
  $\exists N \in \sN, \forall s \in \PHl, \forall \delta \in \reductionsce(s),
    |\delta| \leq N$.
\end{critere}

Étant donnée une sorte $a \in \PHs$ et un état $s \in \PHl$,
on note $\pfp_s(a)$ (\defref{pfp}) l'ensemble des processus de la sorte $a$ qui peuvent être présents
après avoir joué tous les scénarios d'actions de priorité 1 depuis l'état $s$.
Cet ensemble est toujours défini du fait du \crref{tf}.

\begin{definition}[$\pfp : \PHl \times \PHs \rightarrow \powerset(\PHproc)$]
\deflabel{pfp}
  Pour tout état $s \in \PHl$ et pour toute sorte $a \in \PHs$,
  \begin{align*}
    \pfp_s(a) = \{ \get{(s\play\delta)}{a} \in \PHl_a &\mid \delta \in \reductionsce(s)
          \wedge\nexists h\in\PHh^{(1)}, (\delta \cons h) \in \reductionsce(s) \}
  \end{align*}
\end{definition}

Nous définissons dans la suite la notion de \emph{composant bien formé} (\defref{component})
et de \emph{sorte coopérative bien formée} (\defref{cs}).
Un composant bien formé n'est frappé que par des actions secondaires,
car il ne subit que des influences «~biologiques~».
Une sorte coopérative bien formée n'est frappée que par des actions primaires «~non biologiques~»
qui convergent toujours vers le même processus en fonction de l'état des sortes qui l'influencent
(\defref{comp}), afin qu'elle représente chaque configuration de ces sortes par un unique processus.
Le \crref{part} stipule alors que l'ensemble des sortes des Frappes de Processus canoniques
($\PHs$) est une partition entre un ensemble de composants bien formés (noté $\components$)
et un ensemble de sortes coopératives bien formées (noté $\cs$).

\begin{definition}[Composant bien formé ($\components$)]
\deflabel{component}
  Une sorte $a \in \PHs$ est un \emph{composant bien formé} si et seulement si :
    \[\forall h \in \PHh, \PHsort(\target{h}) = a \Rightarrow \prio(h) \geq 2 \enspace.\]
  On note $\components$ l'ensemble des composants bien formés du modèle.
\end{definition}

\begin{definition}[Influence ($\compin : \PHs \to \powerset(\components)$)]
\deflabel{comp}
  Pour toute sorte $a \in \PHs$, on définit :
  $\compin(a) \DEF (\conn(a) \cap \components) \setminus \{ a \}$ où
  $\conn(a) \subset \PHs$ est le plus petit ensemble de sortes satisfaisant :
  \begin{align*}
    a \in \conn(a) & \\
    \forall h \in \PHh^{(1)},
      \sort{\target{h}} \in \conn(a) & \Rightarrow \sort{\hitter{h}} \in \conn(a)
  \end{align*}
\end{definition}

\begin{definition}[Sorte coopérative bien formée ($\cs$)]
\deflabel{cs}
  Une sorte $a \in \PHs$ est une \emph{sorte coopérative bien formée} si et seulement si :
  \begin{enumerate}[(i)]
    \item $\forall h \in \PHh, \sort{\target{h}} = a \Rightarrow \prio(h) = 1$,
    \item \label{csai} $\forall s \in \PHl, \card{\pfp_s(a)} = 1$,
    \item \label{css} $\forall a_i \in \PHl_a, \exists s \in \PHl, a_i \in \pfp_s(a)$,
    \item \label{csss} $\forall \mysigma \in \PHsubl[\PHl]_{\compin(a)}, \forall s, s' \in \PHl,
        (\mysigma \subseteq s \wedge \mysigma \subseteq s') \Rightarrow \pfp_s(a) = \pfp_{s'}(a)$.
  \end{enumerate}
  On note $\cs$ l'ensemble des sortes coopératives bien formées du modèle.
\end{definition}

\begin{example}
  Le modèle de Frappes de Processus avec 2 classes de priorités de la \figref{livelock}
  possède trois composants bien formés : $\components = \{ a, b, c \}$
  et une sorte coopérative bien formée : $\cs = \{ ab \}$.
  On en déduit notamment :
  \begin{align*}
                   && \conn(c) &= \{ c \} &\text{ et } &&\conn(ab) &= \{ a, b, ab \} \enspace, \\
    \text{d'où : } && \compin(c) &= \{ c \} &\text{ et } &&\compin(ab) &= \{ a, b \} \enspace. \\
  \end{align*}
  D'après $\compin(ab) = \{ a, b \}$,
  on peut donc considérer que la sorte coopérative $ab$ modélise une coopération
  entre $a$ et $b$.
\end{example}

\begin{critere}[Partition des composants et des sortes coopératives]
\crlabel{part}
  $\components$ et $\cs$ forment une partition de $\PHs$ :
  \[\PHs = \components \cup \cs \wedge \components \cap \cs = \emptyset\]
\end{critere}

\begin{example}
  Le modèle de la \figref{livelock} vérifie le \crref{part}.
\end{example}

Il est à noter qu'une sorte $a$ qui n'est frappée par aucune action,
c'est-à-dire telle que $\forall h \in \PHh, \sort{\target{h}} \neq a$,
est en accord à la fois avec la définition de composant bien formé
et avec celle de sorte coopérative bien formée.
Cette sorte peut être arbitrairement et indifféremment affectée à $\components$ ou à $\cs$.
Un tel cas peut se produire pour des composants ayant un rôle d'«~entrée~»,
c'est-à-dire dont le niveau d'expression est uniquement déterminé par l'état initial,
ou par certaines constructions particulières de sortes coopératives créées par la traduction
depuis les réseaux discrets asynchrones proposée à la \vsecref{trad-rda}.

Pour toute sorte $a \in \PHs$ et tout état $s \in \PHl$,
étant donné le point (\ref{csai}) de la \defref{cs}, on a toujours :
$\exists a_i \in \PHl_a, \pfp_s(a) = \{ a_i \}$.
On notera donc dans la suite : $\pfp_s(a) = a_i$.
De plus, grâce au point (\ref{css}) de la \defref{cs}, on peut introduire la notation
$\csState(a_i)$ qui permet de caractériser l'ensemble des sous-états représentés par le
processus $a_i$ de toute sorte coopérative bien formée $a$ (\defref{csState}).

\begin{definition}[$\csState : \PHproc \rightarrow \powerset(\PHproc)$]
\label{def:csState}
  Pour tout $a \in \cs$ et $a_i \in \PHl_a$, 
    \[\csState(a_i) \DEF \{ \toset{ps} \mid ps \in \PHsubl[\PHl]_{\compin(a)} \wedge
      \exists s \in \PHl, (ps \subseteq s \wedge \pfp_s(a) = a_i) \}\]
\end{definition}

La notation $\toset{ps}$,
qui permet de considérer un tuple comme un ensemble,
a été définie à la \vsecref{notations}.
Dans la suite, on écrira simplement «~composant~» (\resp «~sorte coopérative~»)
en lieu et place de «~composant bien formé~» (\resp «~sorte coopérative bien formée~»).

Enfin, la \defref{phcanonique} définit la notion de \emph{Frappes de Processus canoniques}.
La forme particulière de ces Frappes de Processus permettra de développer une méthode efficace
d'analyse statique pour le problème d'atteignabilité défini à la \secref{as},
grâce notamment aux résultats développés dans la section suivante.
La \defref{phpseudocanonique} offre de plus une définition plus étendue du concept
de \emph{Frappes de Processus pseudo-canoniques avec $k$ classes de priorités},
dont le nombre de classes de priorités n'est pas limité à 2,
mais dont la classe de priorité la plus forte possède les mêmes restrictions.
Cette définition étendue sera notamment utile pour des questions de traduction,
à la \secref{aplatissement}.

\begin{definition}[Frappes de Processus pseudo-canoniques]
\deflabel{phpseudocanonique}
  Si $k \in \sNN$, les \emph{Frappes de Processus pseudo-canoniques avec $k$ classes de priorités}
  sont les Frappes de Processus avec $k$ classes de priorités
  respectant les \allcr.
\end{definition}

\begin{definition}[Frappes de Processus canoniques]
\deflabel{phcanonique}
  Les \emph{Frappes de Processus canoniques}
  sont les Frappes de Processus pseudo-canoniques avec 2 classes de priorités
  (c'est-à-dire respectant les \allcr).
\end{definition}



\begin{example}
\exlabel{livelock-merge}
  Les Frappes de Processus de la \figref{livelock}
  sont des Frappes de Processus canoniques,
  car elles respectent les \allcr.
  Ce modèle décrit un comportement de coopération entre les deux processus $a_1$ et $b_1$.
  En effet, l'action $\PHfrappe{ab_{11}}{c_0}{c_1}$
  permet de représenter fidèlement la coopération entre $a_1$ et $b_1$ pour
  frapper $c_0$ et le faire bondir en $c_1$.
  De plus, il est possible de bondir en $a_1$ ou en $b_1$ si ces deux processus sont absents
  grâce aux actions $\PHfrappe{b_0}{a_0}{a_1}$ et $\PHfrappe{a_0}{b_0}{b_1}$ ;
  autrement dit, $a_1$ (\resp $b_1$) est accessible uniquement lorsque $b_0$
  (\resp $a_0$) est actif.
  Enfin, $a$ et $b$ s'auto-inhibent grâce aux actions
  $\PHfrappe{a_1}{a_1}{a_0}$ et $\PHfrappe{b_1}{b_1}{b_0}$.
  
  Comparons les Frappes de Processus canoniques $\PH = (\PHs; \PHl; (\PHa^{(1)}; \PHa^{(2)}))$
  de la \figref{livelock} avec leur version fusionnée $\PH' = (\PHs; \PHl; \PHh')$,
  c'est-à-dire $\PH' = \PHmerge(\PH)$, et donc $\PHh' = \PHh^{(1)} \cup \PHh^{(2)}$,
  selon la \vdefref{fusion}.
  En d'autres termes, $\PH'$ est la version sans priorités de $\PH$,
  c'est-à-dire où toutes les actions sont présentes mais où la notion de classes de priorités
  a été supprimée.
  
  Intéressons-nous à la dynamique de $\PH'$.
  Comme expliqué à la \vsecref{sc},
  ce modèle comporte une sorte coopérative $ab$ qui est définie de la façon
  dont le sont toutes les sortes coopératives
  en Frappes de Processus standards, faute de priorités.
  Cette version sans priorités introduit des comportements supplémentaires qui ne sont pas
  désirables pour représenter une porte logique exacte.
  En effet, on pourrait s'attendre à ce que $ab_{11}$ représente la présence simultanée
  de $a$ et $b$ dans un état passé,
  ou, en d'autres termes, que seule la présence simultanée de $a_1$ et $b_1$ puisse
  permettre l'activation de $ab_{11}$.
  Or on constate que ce n'est pas le cas, car le processus $c_1$ peut être
  activé depuis l'état initial $\etat{a_0, b_0, c_0, ab_{00}}$.
  Le scénario suivant permet par exemple d'activer
  ce processus, sans pour autant que $a_1$ et $b_1$ ne soient simultanément présents,
  et bien que le processus initial de $ab$ soit cohérent avec ceux de $a$ et $b$ :
  $\PHstate{a_1, b_0, c_0, ab_{10}} \PHtrans[\PH']
  \PHstate{a_0, b_0, c_0, ab_{10}} \PHtrans[\PH']
  \PHstate{a_0, b_1, c_0, ab_{10}} \PHtrans[\PH']
  \PHstate{a_0, b_1, c_0, ab_{11}}$.
  En effet, il se trouve que l'activation de $ab_{11}$ ne représente pas la présence simultanée
  de $a_1$ et $b_1$ dans un même état, mais leur présence consécutive dans
  différents états.
  En d'autres termes, la présence de $ab_{11}$ dans un état n'implique pas
  $a_1 \wedge b_1$ dans le même état, ni même
  $\tempop(a_1 \wedge b_1)$, mais plutôt : $\tempop(a_1) \wedge \tempop(b_1)$
  où $\tempop$ signifie : «~précédemment~».
  On constate d'ailleurs que l'analyse statique développée dans~\cite{PMR12-MSCS}
  répond positivement quand à l'accessibilité de $c_1$ depuis l'état
  $\PHstate{a_1, b_0, c_0, ab_{10}}$.
  
  Un tel comportement est indésirable si on souhaite utiliser les sortes coopératives en tant
  que portes logiques exactes, c'est-à-dire sans ce décalage temporel menant à une sur-approximation
  de la dynamique.
  Les Frappes de Processus canoniques permettent alors de modéliser un comportement (faiblement)
  bisimilaire au comportement d'un réseau discret asynchrone,
  comme montré à la \vsecref{trad-rda},
  les actions primaires permettant la mise à jour
  sans délai des sortes coopératives, à l'instar du modèle $\PH$ de la \figref{livelock}.
  En effet, le processus $ab_{11}$ est impossible à activer si $a_1$ et $b_1$ ne sont
  pas simultanément actifs, et ce cas ne peut se présenter que dans l'état initial.
  Il s'agit d'un jardin d'Eden dans le sens développé par
  Loïc \cite[p.~123]{Pauleve11}.
  
  \begin{figure}[ht]
    \centering
    \scalebox{1.2}{
    \begin{tikzpicture}
      \TSort{(0,0)}{a}{2}{l}
      \TSort{(0,4)}{b}{2}{l}
      \TSort{(7,2)}{c}{2}{r}

      \TSetTick{ab}{0}{00}
      \TSetTick{ab}{1}{01}
      \TSetTick{ab}{2}{10}
      \TSetTick{ab}{3}{11}
      \TSort{(4,1)}{ab}{4}{r}

      \THit{a_0}{prio}{ab_3}{.west}{ab_1}
      \THit{a_0}{prio}{ab_2}{.south west}{ab_0}
      \THit{a_1}{prio}{ab_1}{.west}{ab_3}
      \THit{a_1}{prio}{ab_0}{.south west}{ab_2}

      \THit{b_0}{prio}{ab_3}{.north west}{ab_2}
      \THit{b_0}{prio}{ab_1}{.north west}{ab_0}
      \THit{b_1}{prio}{ab_2}{.west}{ab_3}
      \THit{b_1}{prio}{ab_0}{.west}{ab_1}
      
      \THit{ab_3}{}{c_0}{.west}{c_1}
      
      \path[bounce, bend right = 70]
        \TBounce{ab_3}{}{ab_1}{.north}
        \TBounce{ab_2}{}{ab_0}{.north west}
        \TBounce{ab_3}{}{ab_2}{.north west}
        \TBounce{ab_1}{}{ab_0}{.north}
      ;
      \path[bounce, bend left = 70]
        \TBounce{ab_1}{}{ab_3}{.south}
        \TBounce{ab_0}{}{ab_2}{.south}
        \TBounce{ab_2}{}{ab_3}{.south west}
        \TBounce{ab_0}{}{ab_1}{.south west}
      ;
      \path[bounce, bend left]
        \TBounce{c_0}{}{c_1}{.south west}
      ;
      
      \TAction{a_1}{a_1.west}{a_0.north west}{selfhit}{right}
      \TAction{b_1}{b_1.west}{b_0.north west}{selfhit}{right}
      \TAction{a_0.south west}{b_0.west}{b_1.south west}{bend left=90}{left}
      \TAction{b_0}{a_0.west}{a_1.south west}{bend right=50}{left}
      
      \TState{a_0, b_0, ab_0, c_0}
    \end{tikzpicture}
    }
    \caption{\figlabel{livelock}%
      Un exemple de Frappes de Processus canoniques.
      Les actions de priorité 1 sont dessinées en traits doubles tandis que les actions
      de priorité 2 sont représentées avec des traits simples.
      Les processus grisés présentent un exemple d'état de départ :
      $\PHstate{a_0, b_0, c_0, ab_{00}}$.
    }
  \end{figure}
\end{example}

%\todoplustard{Dynamique explicite ?}



\subsection{Résultats préliminaires sur les Frappes de Processus canoniques}
\seclabel{phcanonique-resultats}

Dans cette section, nous donnons plusieurs résultats généraux qui peuvent être dérivés des
restrictions de la \secref{phcanonique-def}.
Nous considérons donc dans la suite des Frappes de Processus
canoniques $\PH = (\PHs; \PHl; (\PHa^{(1)}; \PHa^{(2)}))$.
Ces résultats permettront de construire la méthode d'analyse statique de la \secref{as}.

Pour tout état $s \in \PHl$, on appelle $\update(s)$ l'état dans lequel tous les composants
ont le même processus actif que dans $s$
mais où les sortes coopératives ont été mises à jour en fonction
de l'état des sortes qui les influencent (\defref{update}).
Cet état est unique du fait du point (\ref{csai}) de la \vdefref{cs}.
La notation «~$\recouvre$~» représente le recouvrement d'un état par ensemble de processus,
qui avait été donné à la \vdefref{recouvrement}.
%des propriétés de $\pfp$ données dans la section précédente.
Le \lemref{update} stipule ensuite que depuis tout état $s$, il existe un scénario d'actions primaires
mettant à jour toutes les sortes coopératives de façon à arriver dans $\update(s)$.
%ghtrjy <= Raphaël was here!

\begin{definition}[$\update : \PHl \rightarrow \PHl$]
\deflabel{update}
  Pour tout $s \in \PHl$, on définit :
  \begin{align*}
    \update(s) = s \Cap \{ \pfp_{s}(a) \mid a \in \cs \} \enspace.
  \end{align*}
\end{definition}

\begin{lemma}
\lemlabel{update}
  $\forall s \in \PHl, \exists \delta \in \reductionsce(s), s \PHplay \delta = \update(s)$
\end{lemma}

\begin{proof} %[\Lemref{update}]
  Soit $s \in \PHl$ un état.
  Soit $a \in \cs$ une sorte coopérative telle que $\PHget{s}{a} \neq \pfp_s(a)$.
  Étant donnée la \defref{cs}, il existe un scénario $\delta$ mettant à jour $a$
  dans $s$ tel que :
  $\forall \delta' \in \reductionsce(s \PHplay \delta),
    \PHget{(s \PHplay \delta \PHplay \delta')}{a} = \pfp_s(a)$.
  Ce raisonnement est applicable à chacune des sortes coopératives du modèle
  indifféremment de leur ordre.
\end{proof}

Le \lemref{hcompcomp} stipule que pour un état donné $s \in \PHl$, et pour toute action secondaire
$h = \PHhit{a_i}{b_j}{b_k} \in \PHh$ où $a$ et $b$ sont des composants,
si $\PHget{s}{a} = a_i$ et $\PHget{s}{b} = b_j$, alors
$h$ peut toujours être jouée après une série d'actions primaires,
et ces actions ne perturbent pas cette jouabilité
(autrement dit, elles ne modifient pas les processus $a_i$ et $b_j$).

De façon complémentaire, le \lemref{hcscomp} énonce le même résultat si $a$ est une sorte coopérative,
sous la condition que $a$ soit déjà à jour dans $s$.

\begin{lemma}
\lemlabel{hcompcomp}
  $\forall s \in \PHl, \forall a,b \in \components, \forall h = \PHhit{a_i}{b_j}{b_k} \in \PHh,$\\
  $(\PHget{s}{a} = a_i \wedge \PHget{s}{b} = b_j) \Rightarrow
    (\exists \delta \in \reductionsce(s),
%    (s \PHplay \delta) \PHtrans (s \PHplay \delta \PHplay h))$
    \Feval{\Fopphp{h}}{s \PHplay \delta})$
\end{lemma}

\begin{proof} %[\Lemref{hcompcomp}]
  Soient $s \in \PHl$ un état, $a,b \in \components$ deux composants et
  $h = \PHhit{a_i}{b_j}{b_k} \in \PHh$ une action.
  D'après le \lemref{update}, il existe un scénario $\delta$ tel que :
  $(s \PHplay \delta) = \update(s)$.
  Étant donné que $a$ et $b$ sont des composants,
  $a_i \in (s \PHplay \delta)$ et $b_j \in (s \PHplay \delta)$.
  De plus, par définition de $\update(s)$,
  toutes les sortes coopératives ont été mises à jour dans $(s \PHplay \delta)$
  et aucune action primaire n'y est donc jouable.
  Ainsi, $h$ est jouable dans $(s \PHplay \delta)$.
\end{proof}

\begin{lemma}
\lemlabel{hcscomp}
  $\forall s \in \PHl, \forall a \in \cs, \forall b \in \components,
    \forall h = \PHhit{a_i}{b_j}{b_k} \in \PHh,$\\
  $(\PHget{s}{a} = a_i \wedge \PHget{s}{b} = b_j \wedge \pfp_s(a) = a_i) \Rightarrow
    (\exists \delta \in \reductionsce(s),
%    (s \PHplay \delta) \PHtrans (s \PHplay \delta \PHplay h))$
    \Feval{\Fopphp{h}}{s \PHplay \delta})$
\end{lemma}

\begin{proof} %[\Lemref{hcscomp}]
  Similaire à la preuve du \lemref{hcompcomp} ;
  étant donné que $a_i \in \pfp_s(a)$, on a : $a_i \in (s \PHplay \delta)$.
\end{proof}
