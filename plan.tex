\documentclass[a4paper]{report}
%\documentclass[a4paper]{report}

% Langue et encodage d'entrée
\usepackage[utf8]{inputenc}
\usepackage[T1]{fontenc}
\usepackage[french]{babel}

% Maths
\usepackage{amsmath}
\usepackage{amsfonts}
\usepackage{amssymb}
\usepackage{stmaryrd}

% Références internes
\usepackage{varioref}
\usepackage[french]{refstyle}
\usepackage{hyperref}

% Types de théorèmes
\usepackage{ntheorem}
\theorembodyfont{\upshape}
\newtheorem{definition}{Définition}{}{}
\newtheorem*{example}{Exemple}{}{}
\newtheorem{property}{Propriété}{}{}
\newtheorem{proposition}{Proposition}{}{}
\newtheorem{theorem}{Théorème}{}{}
\newtheorem{demo}{Démonstration}{}{}

% Divers
\usepackage{color}   % Couleurs du texte
\usepackage{url}     % Mise en forme + liens pour URLs
\usepackage{array}   % Tableaux évolués
\usepackage{comment} % Blocs de commentaires



% Paquets temporaires
%\usepackage[notref,notcite]{showkeys}

% Commandes À FAIRE
%\definecolor{darkgreen}{rgb}{0,0.5,0}
%\definecolor{darkblue}{rgb}{0,0,0.5}
\newcommand{\todo}[1]{\textcolor{red}{\textbf{[[#1]]}}}
\newcommand{\TODO}{\todo{TODO}}
%\newcommand{\idea}[1]{\textcolor{darkgreen}{[#1]}}
%\newcommand{\warn}{\textcolor{purple}{\textbf{/!\textbackslash}}}
%\newcommand{\mq}[1]{\textcolor{darkblue}{[MQ : #1]}}
%\newcommand{\pref}[1]{\todo{Ref: #1}}



% Macros relatives à la traduction de PH avec arcs neutralisants vers PH à k-priorités fixes

% Macros générales
%\newcommand{\ie}{\textit{i.e.} }

% Notations générales pour PH
\newcommand{\PH}{\mathcal{PH}}
%\newcommand{\PHs}{\mathcal{S}}
\newcommand{\PHs}{\Sigma}
%\newcommand{\PHp}{\mathcal{P}}
%\newcommand{\PHp}{\textcolor{red}{\mathcal{P}}}
%\newcommand{\PHproc}{\mathcal{P}}
\newcommand{\Sa}[1]{\mathbf{S}^{\setminus #1}}
\newcommand{\PHproc}{\mathbf{Proc}}
\newcommand{\Proc}{\PHproc}
\newcommand{\PHh}{\mathcal{H}}
\newcommand{\PHa}{\PHh}
%\newcommand{\PHa}{\mathcal{A}}
\newcommand{\PHl}{\mathcal{L}}
\newcommand{\PHn}{\mathcal{N}}

\newcommand{\PHhitter}{\mathsf{\textcolor{red}{hitters}}}

\newcommand{\PHhitters}{\mathsf{hitters}}
\newcommand{\PHtarget}{\mathsf{target}}
\newcommand{\PHbounce}{\mathsf{bounce}}
%\newcommand{\PHsort}{\Sigma}
\newcommand{\PHsort}[1]{\PHs(#1)}
\newcommand{\PHprio}[1]{\mathsf{prio}(#1)}
\newcommand{\hitters}[1]{\PHhitters(#1)}
\newcommand{\target}[1]{\PHtarget(#1)}
\newcommand{\bounce}[1]{\PHbounce(#1)}
%\newcommand{\PHsort}{\Sigma}
\newcommand{\sort}[1]{\PHsort{#1}}
\newcommand{\prio}{\PHprio}

%\newcommand{\PHfrappeur}{\mathsf{frappeur}}
%\newcommand{\PHcible}{\mathsf{cible}}
%\newcommand{\PHbond}{\mathsf{bond}}
%\newcommand{\PHsorte}{\mathsf{sorte}}
%\newcommand{\PHbloquant}{\mathsf{bloquante}}
%\newcommand{\PHbloque}{\mathsf{bloquee}}

%\newcommand{\PHfrappeR}{\textcolor{red}{\rightarrow}}
%\newcommand{\PHmonte}{\textcolor{red}{\Rsh}}

\newcommand{\PHhitA}{\rightarrow}
\newcommand{\PHhitB}{\Rsh}
%\newcommand{\PHfrappe}[3]{\mbox{$#1\PHhitA#2\PHhitB#3$}}
%\newcommand{\PHfrappebond}[2]{\mbox{$#1\PHhitB#2$}}
\newcommand{\PHhit}[3]{#1\PHhitA#2\PHhitB#3}
\newcommand{\PHfrappe}{\PHhit}
\newcommand{\PHhbounce}[2]{#1\PHhitB#2}
\newcommand{\PHobj}[2]{\mbox{$#1\PHhitB^*\!#2$}}
\newcommand{\PHconcat}{::}
%\newcommand{\PHneutralise}{\rtimes}

\def\PHget#1#2{{#1[#2]}}
%\newcommand{\PHchange}[2]{#1\langle #2 \rangle}
%\newcommand{\PHchange}[2]{(#1 \Lleftarrow #2)}
%\newcommand{\PHarcn}[2]{\mbox{$#1\PHneutralise#2$}}
\newcommand{\PHplay}{\cdot}

\newcommand{\PHstate}[1]{\mbox{$\langle #1 \rangle$}}

\def\supp{\mathsf{support}}
\def\first{\mathsf{first}}
\def\last{\mathsf{last}}

\def\DNtrans{\rightarrow_{ADN}}
\def\DNdef{(\mathbb F, \langle f^1, \dots, f^n\rangle)}
\def\DNdep{\f{dep}}
\newcommand{\trans}[1]{\rightarrow_{#1}}
%\def\PHPtrans{\trans{PH}}
\def\get#1#2{#1[{#2}]}
\def\encodeF#1{\mathbf{#1}}
\def\toPH{\encodeF{PH}}
\def\card#1{|#1|}
\def\decode#1{\llbracket#1\rrbracket}
\def\encode#1{\llparenthesis#1\rrparenthesis}
\def\Hits{\PHa}
\def\hit{\PHhit}
\def\play{\cdot}

\input{macros/macros-ph}
\input{macros/macros-abstr}
\input{macros/tikzstyles}



\newcommand{\nbd}{\nobreakdash-}
\newcommand{\cad}{c.\nbd à\nbd d.\ }
%\newcommand{\ie}{\cad}


\hyphenation{jou-a-bi-li-té}



\author{Maxime Folschette}
\title{Plan de manuscrit de thèse}
\date{Version du \today}

\begin{document}

%\maketitle
\setcounter{tocdepth}{3}
\tableofcontents



\chapter{Introduction}

\chapter{État de l'art de la modélisation}
  \section{Modèle de Thomas}
  \section{PH standard et analyses}
    \subsection{Définition}
    \subsection{Analyse statique}

\chapter{Représentations temporelles du PH}
  \section{Actions plurielles (équivalent automates \& réactions biochimiques)}
    \subsection{Équivalence : cf. modélisation avec arcs neutralisants ?}
  \section{PH avec préemption des actions}
    \subsection{PH avec priorités quelconques (restriction cible/bond)}
    \subsection{PH avec arcs neutralisants}

\chapter{Représentation canonique et analyse du PH}
  \section{PH canonique (avec sortes coopératives priorisées)}
    \subsection{Définition}
    \subsection{Équivalence avec priorités quelconques (?)}
  \section{Analyse statique}
    \subsection{Cf. CS2Bio}

\chapter{Expressivité du PH par rapport à d'autres formalismes}
  \section{Modèle de Thomas}
  \section{Réseaux de Petri}
    \subsection{Réseaux de Petri saufs (une place par processus)}
    \subsection{Réseaux de Petri bornés (une place par sorte)}
  \section{$\pi$-calcul ?}
  \section{Automates finis communicants ?}
  \section{Biocham ?}

\chapter{Applications}

\chapter{Conclusion}



\begin{comment}
\chapter{Introduction}

\chapter{État de l'art de la modélisation et de l'analyse des réseaux de régulation biologique}

\chapter{Le Process Hitting standard}
  \section{Définition}
  \section{Analyse statique (?)}
  \section{Expressivité par rapport à d'autres formalismes}
    \subsection{Réseaux de Petri saufs}
      \idee{Une place par processus}
    \subsection{Réseaux de Petri bornés}
      \idee{Une place par sorte}
    \subsection{$\pi$-calcul}
      \idee{Cf. thèse de Loïc}
    \subsection{Automates finis communicants}
      \idee{Cf. thèse de Loïc}
    \subsection{Biocham ?}

\chapter{Extension du Process Hitting avec sortes coopératives priorisées (\PHsc)}
  \section{Définition}
    \idee{Cf. CS2Bio}
  \section{Analyse statique : raffinement de la sous-approximation}
    \idee{Cf. CS2Bio}
  \section{Expressivité par rapport au PH standard}
  \section{Expressivité par rapport à d'autres formalismes}
    \subsection{Réseaux de Régulation Biologique}
      \mq{traduction possible dans les deux sens avec hypothèses}

\chapter{Autres extension du Process Hitting}
  \section{PH avec arcs actions coopératives (?)}
    \idee{De la forme : $\PHhit{A}{b_i}{b_j}$.}\\
    \mq{PH avec arcs actions coopératives = \PHsc}
  \section{PH avec priorités multiples et actions coopératives (\PHp)}
    \mq{\PHp{} = \PHsc}
  \section{PH avec arcs neutralisants}
    \mq{PH avec arcs neutralisants = \PHsc}
  \section{PH avec actions complexes/plurielles}
    \idee{De la forme : $A \xrightarrow{C} B$ ou encore : $A \rightarrow B$.}\\
    \mq{PH avec actions complexes = \PHsc}

\chapter{Conclusion}
\end{comment}



\end{document}
